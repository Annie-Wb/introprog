%!TEX encoding = UTF-8 Unicode
\documentclass{simpleslides}

%\usepackage{beamerthemesplit}
\usepackage[orientation=landscape,size=custom,width=16,height=9,scale=0.6,debug]{beamerposter} 


\renewcommand{\vecka}{1}
\newcommand{\veckotema}{Introduktion}

%\title[Föreläsning, EDAA45 pgk, Björn Regnell, senast uppdaterad: \today]{Vecka \vecka. \veckotema}
%\subtitle{Programmering, grundkurs}
\title{Programmering, grundkurs}
\author{Björn Regnell}
%\institute{Datavetenskap, LTH, Lunds universitet}
%\date{EDAA45, Lp1-2, HT \CurrentYear}
\date{}

\begin{document}

\frame{\titlepage}
\setnextsection{\vecka}


\frame{\tableofcontents}

\section{Introduktion}

\Subsection{Vad är programmering?}

{
  \setbeamercolor{background canvas}{bg=black}
  \frame[plain]{\centering\includegraphics[height=1.0\textheight]{../../img/eniac}}
}

\begin{Slide}{Vad är programmering?}
  \begin{itemize}
  \item Programmering innebär att ge instruktioner till en datamaskin (dator).
  \end{itemize}
  
  \begin{minipage}{.8\textwidth}
  \begin{itemize}
  \item Ada Lovelace publicerade det första programmet redan på 1800-talet ämnat för en kugghjulsdator.
  \end{itemize}
  \end{minipage}%
  \begin{minipage}{.2\textwidth}
  \centering\includegraphics[width=0.6\columnwidth]{../../img/ada}
  \end{minipage}%
  % \begin{itemize}
  % \item \href{https://sv.wikipedia.org/wiki/Programmering}{sv.wikipedia.org/wiki/Programmering}
  % \item \href{https://en.wikipedia.org/wiki/Computer\_programming}{en.wikipedia.org/wiki/Computer\_programming}
  % \item Ha picknick i \href{http://kartor.lund.se/wiki/lundanamn/index.php/Ada_Lovelace-parken}{Ada Lovelace-parken} på Brunnshög!
  % \end{itemize}
  \end{Slide}


\begin{Slide}{\large Enkel datormodell}
  \begin{tikzpicture}[node distance=2.0cm]
  \node (input)  [startstop]               {Indata-enhet};
  \node (cpu)    [process, below of=input] {CPU};
  \node (output) [startstop,below of=cpu]  {Utdata-enhet};
  
  \node (mem) [right of=cpu, xshift=0.4\textwidth, draw = black, thick] {
  \begin{minipage}{0.5\textwidth}\centering
  \textbf{Minne} med minnesceller
  \vspace{1em}
  
  \begin{tabular}{|l | l|}
  address & innehåll \\ \hline
  0   & 42 \\ \hline
  1   & 13 \\ \hline
  2   & 18 \\ \hline
  3   & 21 \\ \hline
  4   & 55 \\ \hline
  5   & 64 \\ \hline
  6   & 48 \\ \hline
  ... & ...
  \end{tabular}
  \end{minipage}
  };
  
  \draw [arrow] (input) -- (cpu);
  \draw [arrow] (cpu) -- (output);
  \draw [arrow] (cpu) -- (mem);
  \draw [arrow] (mem) -- (cpu);
  
  \end{tikzpicture}
  
  {\hfill
  \begin{minipage}{0.65\textwidth}\vspace{1em}
  Minnet innehåller endast \Alert{heltal} som \newline representerar \Emph{data} \Alert{och} \Emph{instruktioner}.
  \end{minipage}
  }
  \end{Slide}



 

\begin{Slide}{Terminologi}
  \begin{itemize}
    \item \Emph{programmera}         = \pause att skapa program (koda, utveckla) \pause 
    \item \Emph{program}             = instruktioner till en dator (kod, mjukvara) \pause
    \item \Emph{programmerare}       = en som programmerar (utvecklare) \pause
    \item \Emph{programmeringsspråk} = ett språk skapat speciellt för att underlätta för människor att programmera
  \end{itemize}
\end{Slide}

\begin{Slide}{En milstolpe i datorhistorien}
  \begin{minipage}{.8\textwidth}
    \begin{itemize}
    \item Grace Hopper uppfann \Emph{kompilatorn} 1952. \\ \href{https://en.wikipedia.org/wiki/Grace\_Hopper}{en.wikipedia.org/wiki/Grace\_Hopper}
    \end{itemize}
    \end{minipage}%
    \begin{minipage}{.2\textwidth}
    \centering\includegraphics[width=0.6\columnwidth]{../../img/grace}
    \end{minipage}%
  

\end{Slide}



\begin{Slide}{Vad är en kompilator?}
  \begin{figure} 
\centering\begin{tikzpicture}[node distance=1.4cm,scale=0.8]
  \node (input) [startstop] {Källkod};
  \node(inptext) [right of=input, text width=2cm, scale=0.8,xshift=1.5cm]{För\\människor};
  \node (compile) [process, below of=input] {Kompilator};
  \node (output) [startstop, below of=compile] {Maskinkod};
  \node(outtext) [right of=output, text width=2cm, scale=0.8,xshift=1.5cm]{För\\maskiner};
  \draw [arrow] (input) -- (compile);
  \draw [arrow] (compile) -- (output);
  \end{tikzpicture}%
\end{figure} 
\begin{itemize}\small
  \item Ett \Emph{programmeringsspråk} används av människor för att skriva \Emph{källkod} som kan översättas av en \Emph{kompilator} till \Emph{maskinspråk} som i sin tur \Emph{exekveras} av en dator.
\end{itemize}

\end{Slide}
  
  
\begin{Slide}{Exempel: en liten del ur ett stort program}
\begin{Code}
  while (w.lastEventType != Event.WindowClosed) {
    w.awaitEvent(10)  // wait for next event for max 10 milliseconds

    if (w.lastEventType == Event.Undefined) 
      println(s"lastEventType is undefined")
    else 
      println(Event.show(w.lastEventType))

    w.lastEventType match {
      case Event.KeyPressed    => println("lastKey == " + w.lastKey)
      case Event.KeyReleased   => println("lastKey == " + w.lastKey)
      case Event.MousePressed  => println("lastMousePos == " + w.lastMousePos)
      case Event.MouseReleased => println("lastMousePos == " + w.lastMousePos)
      case Event.WindowClosed  => println("Goodbye!"); System.exit(0)
      case _ =>
    }

    Thread.sleep(100) // wait for 0.1 seconds
  }
\end{Code}
\end{Slide}


\begin{Slide}{Virtuell maskin (VM) == abstrakt hårdvara}
\begin{multicols}{2}
\begin{itemize}\small
\item En VM är en ''dator'' implementerad i mjukvara som kan tolka en generell ''maskinkod'' som \Emph{översätts under körning} till den \Alert{verkliga} maskinens specifika maskinkod.

\item Med en VM blir källkoden \Emph{plattformsoberoende} och fungerar på många olika maskiner.

\item Exempel JVM: \\ \Emph{Java Virtual Machine}


\end{itemize}

\columnbreak %---------

%https://www.sharelatex.com/blog/2013/08/29/tikz-series-pt3.html
\begin{tikzpicture}[node distance=1.4cm]
\node (input) [startstop] {Källkod};
\node (compile) [process, below of=input] {Kompilator};
\node (output) [startstop, below of=compile] {Generell ''maskinkod''};
\node (interp) [process, below of=output] {Virtuell Maskin};
\node (output2) [startstop, below of=interp] {Specifik maskinkod};
\draw [arrow] (input) -- (compile);
\draw [arrow] (compile) -- (output);
\draw [arrow] (output) -- (interp);
\draw [arrow] (interp) -- (output2);
\end{tikzpicture}
\end{multicols}
\end{Slide}


\begin{Slide}{Olika språk}
\begin{table}
  \centering\Large
  \begin{tabular}{r | l}
    \multicolumn{2}{c}{\textit{språk}}\\\hline 
    \Emph{naturligt} & \Alert{artificiellt} \\ \pause
    svenska & klingon \\
    engelska & esperanto \\
    italienska & Scala \\
    latin & ALGOL  \\
  \end{tabular}
\end{table}
%https://en.wikipedia.org/wiki/List_of_languages_by_total_number_of_speakers
\end{Slide}

\begin{Slide}{Exempel: populära programmeringsspråk}
  \begin{itemize}
    \item Java
    \item C
    \item C++
    \item C\#
    \item Python
    \item JavaScript
    \item \Emph{Scala}
    \end{itemize}  
    %\url{https://redmonk.com/sogrady/files/2020/02/lang.rank_.120.wm_.png}
\end{Slide}

\begin{Slide}{Olika sätt att programmera på}
  \begin{itemize}
  \item[] Exempel på olika %\href{https://en.wikipedia.org/wiki/Programming_paradigm}{programmeringsparadigm} 
  programmeringsparadigm:
  \begin{itemize}
  \item \Emph{imperativ programmering:} \Alert{satser} som förändrar data, tillstånd
  \item \Emph{objektorienterad programmering:} imperativ programmering med \Alert{objekt} som sammanför relaterad data och operationer
  \item \Emph{funktionsprogrammering:} \Alert{funktioner} och \Alert{uttryck} samverkar; undviker föränderlig data och tillståndsändringar
  \item \Emph{logikprogrammering:} programmet är uppbyggt av logiska uttryck som beskriver olika fakta eller villkor och exekveringen utgörs av en \Alert{bevisprocedur} som söker efter värden som uppfyller alla fakta och villkor
  \end{itemize}
  \end{itemize}
\end{Slide}


\begin{Slide}{Olika delar av utvecklingsprocessen}
  Iterativt arbete i samverkande delprocesser:
  \begin{itemize}
    \item Krav
    \item Design
    \item Test
    \item Drift
  \end{itemize}
\end{Slide}

\begin{Slide}{Olika språkregler}
  \begin{table}
    \centering\Large
    \begin{tabular}{r | l}
      \multicolumn{2}{c}{\textit{grammatik}}\\\hline 
      \Emph{syntax} & \Alert{semantik} \\ \pause
      utformning & innebörd \\
      hur man skriver & vad det betyder \\
      kombinera delar & tolka helhet \\
    \end{tabular}
  \end{table}
\end{Slide}

\Subsection{Vad består ett program av?}
  
\begin{Slide}{}
  \begin{table}
    \centering\Large
    \begin{tabular}{r | l}
      \Alert{instruktioner} & \Emph{data} \\ 
    \end{tabular}
  \end{table}
\end{Slide}



\begin{Slide}{Olika delar av ett program}
  \begin{itemize}
    \item \Emph{Nyckelord} är ord med speciell betydelse, t.ex. \code{if}, \code{else}
    \item \Emph{Deklarationer} definierar nya ord: \code{def gurka = 42}
    \item \Emph{Satser} gör något, har en \Alert{effekt}: \code{print("hej")}
    \item \Emph{Uttryck} ger ett \Alert{resultat}: \code{1 + 1}
    \item \Emph{Data} är information som behandlas: t.ex. heltalet \code{42}
  \end{itemize}
\end{Slide}


\begin{Slide}{Olika typer av data}
  \begin{table}
    \centering\Large
    \begin{tabular}{r | l}
      \multicolumn{2}{c}{\textit{data}}\\\hline 
      \Emph{indata} & \Alert{utdata} \pause \\ 
      \multicolumn{2}{c}{delresultat} \\
      % \multicolumn{2}{c}{litteraler} \\
      % \multicolumn{2}{c}{variabler} \\
      % \multicolumn{2}{c}{datastrukturer} \\
    \end{tabular}
  \end{table}
\end{Slide}


\begin{Slide}{Vad händer med data över tid?}
  \begin{table}
    \centering\Large
    \begin{tabular}{r | l}
      \multicolumn{2}{c}{\textit{data}}\\\hline 
      \Emph{oföränderlig} & \Alert{förändringsbar} \\ 
      persistent & flyktig \\
      konstant & uppdateras \\
      \emph{immutable} & \emph{mutable} \\
    \end{tabular}
  \end{table}
\end{Slide}



\begin{Slide}{Olika kodstrukturer}
  Instruktionerna i ett program ordnas i strukturer: (SARA)
  \begin{itemize}
    \item \Emph{Sekvens}: instruktioners \Alert{ordning} avgör resultatet
    \item \Emph{Alternativ}: olika resultat \Alert{beroende} på data
    \item \Emph{Repetition}: instruktioner \Alert{upprepas} många gånger
    \item \Emph{Abstraktion}: nya byggblock skapas för \Alert{återanvändning}
  \end{itemize}
\end{Slide}

\begin{Slide}{De enklaste byggstenarna}
  \begin{itemize}
    \item \Emph{litteraler}: skapa enkla värden, \code{42}
    \item \Emph{uttryck}: beräkna nya värden ur enkla värden, \code{42 + 1}
    \item \Emph{variabler}: sätter namn på värden, \code{val ålder = 42} 
  \end{itemize}
\end{Slide}

%%%%%%%%%%%%%
%%%%%%%%%%%%%
%%% FÖRENKLAT TILL HIT



\begin{Slide}{Litteraler}
  \begin{itemize}
  \item En litteral representerar ett fixt \Emph{värde} i koden och används för att skapa \Alert{data} som programmet ska bearbeta.
  \item Exempel: \\
  \begin{tabular}{l l}
  \code|42| & heltalslitteral\\
  \code|42.0| & decimaltalslitteral\\
  \code|'!'| & teckenlitteral, omgärdas med 'enkelfnuttar' \\
  \code|"hej"| & stränglitteral, omgärdas med ''dubbelfnuttar'' \\
  \code|true| & litteral för sanningsvärdet ''sant''\\
  \end{tabular}
  \item Literaler har en \Emph{typ} som avgör vad man kan göra med dem.
  \end{itemize}
  \end{Slide}
  
  \begin{Slide}{Exempel på inbyggda datatyper i Scala}\SlideFontSmall
  \begin{itemize}
  \item Alla värden, uttryck och variabler har en \href{https://sv.wikipedia.org/wiki/Datatyp}{\Emph{datatyp}}, t.ex.:
  \begin{itemize}\footnotesize
  \item \code{Int} för heltal
  \item \code{Long} för \textit{extra} stora heltal (tar mer minne)
  \item \code{Double} för decimaltal, så kallade flyttal med flytande decimalpunkt
  \item \code{String} för strängar
  \end{itemize}
  
  \item Kompilatorn håller reda på att uttryck kombineras på ett \Emph{typsäkert} sätt. Annars blir det \Alert{kompileringsfel}.
  
  \item Scala och Java är s.k. \href{https://sv.wikipedia.org/wiki/Typsystem}{\Emph{statiskt typade}} språk, vilket innebär att kontroll av typinformation sker vid kompilering \Eng{compile time}\footnote{Andra språk, t.ex. Python och Javascript är \Emph{dynamiskt typade} och där skjuts typkontrollen upp till körningsdags \Eng{run time} \\ Vilka är för- och nackdelarna med statisk vs. dynamisk typning?}.
  
  \item Scala-kompilatorn gör \href{https://en.wikipedia.org/wiki/Type_inference}{\Emph{typhärledning}}: man \Alert{slipper skriva typerna} om kompilatorn kan lista ut dem med hjälp av typerna hos deluttrycken.
  
  \end{itemize}
  \end{Slide}
  
  
  \begin{Slide}{Grundtyper i Scala}\SlideFontSmall
  Dessa \Emph{grundtyper} \Eng{basic types} finns inbyggda i Scala:
  
  \begin{table}[H]
  \renewcommand{\arraystretch}{1.4}
  \begin{tabular}{p{0.24\textwidth}|p{0.21\textwidth}|l}
  \textit{Svenskt namn} & \textit{Engelskt namn} & \Emph{Grundtyper} \\ \hline
  heltalstyp & integral type & \texttt{Byte}, \texttt{Short}, \texttt{Int}, \texttt{Long}, \texttt{Char} \\
  flyttalstyp  &  floating point \newline number types & \texttt{Float}, \texttt{Double} \\
  numeriska typer & numeric types & heltalstyper och flyttalstyper \\
  strängtyp \newline (teckensekvens) & string type & \texttt{String}  \\
  sanningsvärdestyp  \newline (boolesk typ)& truth value type & \texttt{Boolean} \\
  \end{tabular}
  \end{table}
  
  \end{Slide}
  
  \begin{Slide}{Grundtypernas implementation i JVM}\SlideFontSmall
  \begin{table}[H]
  \renewcommand{\arraystretch}{1.4}
  \begin{tabular}{l|l|l|l}
  \Alert{Grundtyp} i &  Antal                &      Omfång&\Alert{primitiv typ} i\\
   \Emph{Scala} & bitar & minsta/största värde &\Emph{Java} \& \Emph{JVM}\\ \hline
  \texttt{Byte}   &  8  & $-2^7$ ... $2^7-1$   & \texttt{byte} \\
  \texttt{Short}  &  16 & $-2^{15}$ ... $2^{15}-1$ & \texttt{short} \\
  \texttt{Char}   &  16 & $0$ ... $2^{16}-1$ & \texttt{char} \\
  \texttt{Int}    &  32 & $-2^{31}$ ... $2^{31}-1$ & \texttt{int} \\
  \texttt{Long}   &  64 & $-2^{63}$ ... $2^{63}-1$ & \texttt{long} \\
  \texttt{Float}  &  32 & ± $3.4028235 \cdot 10^{38}$  & \texttt{float} \\
  \texttt{Double} &  64 & ± $1.7976931348623157 \cdot 10^{308}$ & \texttt{double} \\
  \end{tabular}
  \end{table}
  
  Grundtypen \texttt{String} lagras som en \emph{sekvens} av 16-bitars tecken av typen \texttt{Char} och kan vara av godtycklig längd (tills minnet tar slut).
  
  \end{Slide}
  
  
  \begin{Slide}{Uttryck}
  \begin{itemize}
  \item Ett \Emph{uttryck} består av en eller flera delar som efter \Emph{evaluering} ger ett \Alert{resultat}.
  \item Delar i ett uttryck kan t.ex. vara: \\ litteraler (42), operatorer (+), funktioner (sin), ...
  \item Exempel:
  \begin{itemize}
  \item Ett enkelt uttryck: \\ \code{42.0}
  \item Sammansatta uttryck: \\
  \code{40 + 2} \\
  \code{(20 + 1) * 2} \\
  \code{sin(0.5 * Pi)} \\
  \code{"hej" + " på " + "dej"}
  \end{itemize}
  
  \item När programmet tolkas sker \Emph{evaluering} av uttrycket, vilket ger ett resultat i form av ett \Emph{värde} som har en \Emph{typ}.
  \end{itemize}
  \end{Slide}
  
  
  \begin{Slide}{Variabler}\SlideFontSmall
  \begin{itemize}
  \item En \Emph{variabel} kan tilldelas värdet av ett enkelt eller sammansatt uttryck.
  \item En variabel har ett \Emph{variabelnamn}, vars utformning följer språkets regler för s.k. \Emph{identifierare}.
  \item En ny variabel införs i en \Emph{variabeldeklaration} och då den kan ges ett värde, \Emph{initialiseras}. Namnet användas som \Emph{referens} till värdet.
  \item Exempel på variabeldeklarationer i Scala, notera \Emph{nyckelordet} \code{val}:
  \begin{Code}
  val a = 0.5 * Pi
  val length = 42 * sin(a)
  val exclamationMarks = "!!!"
  val greetingSwedish = "Hej på dej" + exclamationMarks
  \end{Code}
  
  \item Vid exekveringen av programmet lagras variablernas värden i minnet och deras respektive värde hämtas ur minnet när de \Emph{refereras}.
  
  \item Variabler som deklareras med \code{val} kan endast tilldelas ett värde \Alert{en enda gång}, vid den initialisering som sker vid deklarationen.
  \end{itemize}
  
  \end{Slide}
  
  
  \begin{Slide}{Regler för identifierare}
  \begin{itemize}
  \item \Emph{Enkel} identifierare: t.ex. \code{gurka2tomat}
  \begin{itemize}
  \item Börja med bokstav
  \item ...följt av bokstäver eller siffror
  \item Kan även innehålla understreck
  \end{itemize}
  
  \item \Emph{Operator}-identifierare, t.ex. \code{+:}
  \begin{itemize}
  \item Börjar med ett \Emph{operatortecken}, t.ex. \code{+ - * / : ? ~ #}
  \item Kan följas av fler operatortecken
  \end{itemize}
  
  
  \item En identifierare får \Alert{inte} vara ett \Emph{reserverat ord}, se \href{https://cs.lth.se/pgk/quickref}{snabbreferensen} för alla reserverade ord i Scala \& Java.
  
  \item \Emph{Bokstavlig} identifierare: \code{`kan innehålla allt`}
  \begin{itemize}
  \item Börjar och slutar med \Emph{backticks}  \code{` `}
  \item Kan innehålla vad som helst (utom backticks)
  \item Kan användas för att undvika krockar med reserverade ord: \texttt{\code{`}val\code{`}}
  \end{itemize}
  
  \end{itemize}
  \end{Slide}


  \begin{Slide}{Att bygga strängar: konkatenering och interpolering}
    \begin{itemize}
    \item Man kan \Emph{konkatenera} strängar med operatorn + \\ \code{"hej" + " på " + "dej"}
    \item Efter en sträng kan man konkatenera vilka uttryck som helst; uttryck inom parentes evalueras först och värdet görs sen om till en sträng före konkateneringen:
    \begin{Code}
    val x = 42
    val msg = "Dubbla värdet av " + x + " är " + (x * 2) + "."
    \end{Code}
    \item Man kan i Scala (men inte Java) få hjälp av kompilatorn att övervaka bygget av strängar med \Emph{stränginterpolatorn} \Alert{s}:
    \begin{Code}
    val msg = s"Dubbla värdet av $x är ${x * 2}."
    \end{Code}
    
    \end{itemize}
    \end{Slide}
    
    \begin{Slide}{Heltalsaritmetik}\SlideFontSmall
    \begin{itemize}
    \item De fyra räknesätten skrivs som i matematiken (vanlig \href{https://en.wikipedia.org/wiki/Order_of_operations}{precedens}):
    \begin{REPL}
    scala> 3 + 5 * 2 - 1
    res0: Int = 12
    \end{REPL}
    \item \Emph{Parenteser} styr \Alert{evalueringsordningen}:
    \begin{REPL}
    scala> (3 + 5) * (2 - 1)
    res1: Int = 8
    \end{REPL}
    \item \Emph{Heltalsdivision} sker med \Alert{decimaler avkortade}:
    \begin{REPL}
    scala> 41 / 2
    res2: Int = 20
    \end{REPL}
    \item \href{https://en.wikipedia.org/wiki/Modulo_operation}{\Emph{Moduloräkning}} med restoperatorn \code{%}
    \begin{REPL}
    scala> 41 % 2
    res3: Int = 1
    \end{REPL}
    \end{itemize}
    \end{Slide}
    
    
    \begin{Slide}{Flyttalsaritmetik}\SlideFontSmall
    \begin{itemize}
    \item Decimaltal representeras med s.k. \href{https://sv.wikipedia.org/wiki/Flyttal}{\Emph{flyttal}} av typen \code{Double}:
    \begin{REPL}
    scala> math.Pi
    res4: Double = 3.141592653589793
    \end{REPL}
    
    \item Stora tal så som $\pi*10^{12}$ skrivs:
    \begin{REPL}
    scala> math.Pi * 1E12
    res5: Double = 3.141592653589793E12
    \end{REPL}
    \item Det finns \Alert{inte} oändligt antal decimaler vilket ger problem med \Alert{avvrundingsfel}:
    \begin{REPL}
    scala> 0.0000000000001
    res6: Double = 1.0E-13
    
    scala> 1E10 + 0.0000000000001
    res7: Double = 1.0E10
    \end{REPL}
    \end{itemize}
    \end{Slide}
    
    % \ifkompendium\else
    % \begin{SlideExtra}{På rasten}
    % En per grupp kommer fram hit och tar en grupp-skylt
    % \begin{itemize}
    %   \item Samlas i era samarbetsgrupper i foajen
    %   \item D1.01a längst västerut (mot havet), D1.12b längst österut
    %   \item Lär allas namn
    %   \item Bestäm tid för första möte
    %   \item Vid behov: \\ Bestäm vem som mejlar till de i gruppen som inte är här idag 
    % \end{itemize}  
    % \end{SlideExtra}
    % \fi
    
    \Subsection{Funktioner}
    
    \begin{Slide}{Definiera namn på uttryck}
    \begin{itemize}
    \item Med nyckelordet \code{def} kan man låta ett \Emph{namn} betyda samma sak som ett \Emph{uttryck}.
    \item Exempel:
    \begin{Code}
    def gurklängd = 42 + x
    \end{Code}
    \item Uttrycket till höger evalueras \Alert{varje} gång \Emph{anrop} sker,\\
    d.v.s. varje gång namnet används på annat ställe i koden.
    \begin{Code}
    gurklängd
    \end{Code}
    
    \end{itemize}
    \end{Slide}
    
    \begin{Slide}{Funktion, argument, parameter}\SlideFontSmall
    \begin{itemize}
    \item En \Emph{funktion} räknar ut \Alert{resultat} baserat på indata som kallas \Emph{argument}.
    
    \item Argument ges namn genom deklaration av \Emph{parametrar}.
    
    \item Exempel på deklaration av en funktion med en parameter:
    \begin{Code}
    def dubblera(x: Int) = 2 * x
    \end{Code}
    
    \item Parametrarnas typ \Alert{måste} beskrivas efter \Emph{kolon}.
    \item Kompilatorn kan härleda \Emph{returtypen}, men den kan också med fördel, för tydlighetens skull, anges \Alert{explicit}:
    \begin{Code}
    def dubblera(x: Int): Int = 2 * x
    \end{Code}
    
    \item Observera att namnet \code{x} blir ett ''nytt fräscht'' \Emph{lokalt namn} som \Alert{bara finns och syns  ''inuti'' funktionen} och har inget med ev. andra \code{x} utanför funktionen att göra.
    
    \item Beräkningen sker först vid \Alert{anrop} av funktionen:
    \begin{REPL}
    scala> dubblera(42)
    res1: Int = 84
    \end{REPL}
    
    \end{itemize}
    \end{Slide}
    
    
    
    
    
    
    \begin{Slide}{Färdiga matte-funktioner i paketet \texttt{scala.math}}\SlideFontSmall
    \begin{itemize}
    \item I paketet \texttt{\Emph{scala.math}} finns många användbara funktioner: t.ex.\\
    \code{math.random()} ger slumptal mellan \code{0.0} och \code{0.99999999999999999}
    \begin{REPLnonum}
    scala> val x = math.random()
    x: Double = 0.27749191749889635
    
    scala> val length = 42.0 * math.sin(math.Pi / 3.0)
    length: Double = 36.373066958946424
    \end{REPLnonum}
    
    \item Studera dokumentationen här: \\{\SlideFontTiny
    \url{http://www.scala-lang.org/api/current/scala/math/}}
    
    \item Paketet \code{scala.math} delegerar ofta till Java-klassen \texttt{\Emph{java.lang.Math}} som är dokumenterad här: \\{\SlideFontTiny
    \url{https://docs.oracle.com/javase/8/docs/api/java/lang/Math.html}}
    
    \end{itemize}
    \end{Slide}
    
    
    
\Subsection{Logik}
     
\begin{Slide}{Logiska uttryck}
\begin{itemize}
\item ''Räkna'' med sanningvärden,
s.k. booelsk algebra % \href{https://en.wikipedia.org/wiki/George_Boole}{George Boole}

\item Endast två enkla värden: \code{true false}
\item Logiska operatorer:\\
\code{&&} och, \texttt{||} eller, \texttt{!} icke, \code{==} likhet, \code{!=} olikhet, \\
relationer: \code{> < >= <=}

\item Exempel på logiska uttryck:
\begin{itemize}
\item[] \code{true && true}
\item[] \code{false || true}
\item[] \code{!false}
\item[] \code{42 == 43}
\item[] \code{42 != 43}
\item[] \code{(42 >= 43) || (1 + 1 == 2)}
\end{itemize}

\end{itemize}
\end{Slide}
    
\begin{Slide}{De Morgans lagar}

%\href{https://en.wikipedia.org/wiki/Augustus_De_Morgan}{\Emph{De Morgans lagar}} 
Beskriver vad som händer om man \Alert{negerar} ett logiskt uttryck. Kan användas för att göra \Emph{förenklingar}.

%$p$ och $q$ är logiska uttryck, $\neg$ står för ''icke'', $\wedge$ för ''och'', $\vee$ för ''eller'':
%\begin{eqnarray*}
%\neg (p \wedge q) & \Longleftrightarrow & (\neg p) \vee (\neg q)\\
%\neg (p \vee q) & \Longleftrightarrow & (\neg p) \wedge (\neg q)\\
%\end{eqnarray*}

\begin{itemize}
\item I all deluttryck sammanbundna med \code{&&} eller \code{||}, \\ ändra alla \code{&&} till \code{||} och omvänt.
\item Negera alla ingående deluttryck. En relation negeras genom att man byter \texttt{==} mot \texttt{!=}, \texttt{<} mot \texttt{>=}, etc.
\end{itemize}

Exempel på förenkling med de Morgans lagar:

\begin{Code}[escapechar=X,backgroundcolor=,frame=none,basicstyle=\ttfamily\fontsize{10}{12}\selectfont]
! (a < b || (a == 1 && b == 1))             X$\iff$X
! (a < b) && ! (a == 1 && b == 1)           X$\iff$X
! (a < b) && (! (a == 1) || ! (b == 1))     X$\iff$X
a >= b && (a != 1 || b != 1)
\end{Code}
\end{Slide}
    
\begin{Slide}{Alternativ med if-uttryck}
\begin{itemize}
\item Ett if-uttryck börjar med nyckelordet \code{if}, följt av ett logiskt uttryck inom parentes och två grenar.
\begin{Code}
def slumpgrönsak = if (math.random() < 0.8) "gurka" else "tomat"
\end{Code}
\item Den ena grenen evalueras om uttrycket är \code{true}
\item Den andra \code{else}-grenen evalueras om uttrycket är \code{false}
\begin{REPLnonum}
scala> slumpgrönsak
res13: String = gurka

scala> slumpgrönsak
res14: String = gurka

scala> slumpgrönsak
res15: String = tomat

\end{REPLnonum}
\end{itemize}
\end{Slide}
    
\Subsection{Satser}


%%%%%%%%%%%%%%%%%%%%%%%%
\begin{Slide}{Variabeldeklaration och tilldelningssats}\SlideFontTiny

\begin{itemize}
\item En \Emph{variabeldeklaration} medför att \Alert{plats i datorns minne} reserveras så att värden av den typ som variabeln kan referera till får plats där.

\item Vid deklaration ska variabeln \Emph{initialiseras} med ett startvärde.

\item En \code{val}-deklaration ger en variabel som efter initialisering inte kan ändras.


\begin{multicols}{2}
Dessa deklarationer...
\begin{lstlisting}
var x = 42
val y = x + 1
\end{lstlisting}
... ger detta innehåll någonstans i minnet:

%http://tex.stackexchange.com/questions/18521/tikz-matrix-as-a-replacement-for-tabular
\begin{tikzpicture}[]
\matrix [matrix of nodes, row sep=0, column 2/.style={nodes={rectangle,draw,minimum width=3em}}]
{
x   & 42 \\
y   & 43 \\
};
\end{tikzpicture}
%\end{column}

%\end{columns}

\end{multicols}


\item Med en \Emph{tilldelningssats} ges en tidigare \code{var}-deklarerad variabel ett nytt värde:
\begin{lstlisting}
x = 13
\end{lstlisting}

\item Det gamla värdet försvinner för alltid och det nya värdet lagras istället: \\
\begin{tikzpicture}[]
\matrix [matrix of nodes, row sep=0, column 2/.style={nodes={rectangle,draw,minimum width=3em}}]
{
x   & 13 \\
y   & 43 \\
};
\end{tikzpicture}

Observera att \code{y} här inte påverkas av att x ändrade värde.
\end{itemize}
\end{Slide}
    
\begin{Slide}{Tilldelningssatser är \emph{inte} matematisk likhet}\SlideFontSmall

\begin{itemize}

\item Likhetstecknet används alltså för att \Emph{tilldela} variabler nya värden och det är \Alert{inte} samma sak som matematisk likhet. Vad händer här?
\begin{lstlisting}
x = x + 1
\end{lstlisting}

\item Denna syntax är ett arv från de gamla språken C, Fortran mfl.

\item I \href{https://en.wikipedia.org/wiki/Assignment_(computer_science)}{andra språk} används  t.ex.  \\\vspace{1em}
\texttt{x := x + 1}  \hspace{2em} eller  \hspace{2em} \texttt{x <- x + 1} \\\vspace{0.5em}

\item Denna syntax visar kanske bättre att tilldelning är en \Emph{stegvis process}:

\begin{enumerate}\SlideFontTiny
\item Först beräknas \Emph{uttrycket till höger} om tilldelningstecknet.
\item Sedan \Emph{ersätts värdet} som variabelnamnet refererar till av det beräknade uttrycket. Det gamla värdet \Alert{försvinner för alltid}.
\end{enumerate}

\end{itemize}

\end{Slide}
    
    
\begin{Slide}{Förkortade tilldelningssatser}
\begin{itemize}
\item Det är vanligt att man vill tilldela en variabel ett nytt värde som beror av det gamla, så som i \\\code{x = x + 1}

\item Därför finns \Emph{förkortade tilldelningssatser} som gör så att man sparar några tecken och det blir tydligare (?) vad som sker (när man vant sig vid detta skrivsätt):
\begin{Code}
x += 1
\end{Code}

\item Uttrycket ovan expanderas av kompilatorn till \code{x = x + 1}
\end{itemize}


\end{Slide}
    
    
\begin{Slide}{Exempel på förkortade tilldelningssatser}
\begin{REPLnonum}
scala> var x = 42
x: Int = 42

scala> x *= 2

scala> x
res0: Int = 84


scala> x /= 3

scala> x
res2: Int = 28
\end{REPLnonum}
\end{Slide}
    
    
\begin{Slide}{Övning: Tilldelningar i sekvens}\footnotesize

%\begin{columns}
%\begin{column}{0.32\textwidth}
\begin{minipage}{0.32\textwidth}

Rita hur minnet ser ut efter varje rad nedan:
\vskip1em
\begin{lstlisting}[ numbers=left,]
var u = 42
var x = 10
var y = 2 * x + 1
x = 20
var z = y + x + y - x
x += 1; y *= 2
\end{lstlisting}
%\end{column}
\end{minipage}\hspace{2em}%
%\begin{column}{0.6\textwidth}
\begin{minipage}{0.6\textwidth}


\scriptsize En variabel som ännu inte \Emph{initierats} har ett \Alert{odefinierat} värde, anges nedan med frågetecken.
\begin{table}[]
\centering\scriptsize
%http://tex.stackexchange.com/questions/83930/what-are-the-different-kinds-of-boxes-in-latex
\newcommand{\mybox}[1]{\raisebox{-0.5mm}{\framebox(21,14){#1}}\vspace{0.5mm}}
\begin{tabular}{@{}ccccccc}
  & rad 1 & rad 2 & rad 3 & rad 4  & rad 5 & rad 6\\
u& \mybox{42 } &  \mybox{}   &   \mybox{}   & \mybox{} & \mybox{} & \mybox{} \\
x& \mybox{? } &  \mybox{}   &   \mybox{}   & \mybox{} & \mybox{}  & \mybox{} \\
y& \mybox{? } &  \mybox{}   &   \mybox{}   & \mybox{} & \mybox{}  & \mybox{} \\
z& \mybox{? } &  \mybox{}   &   \mybox{}   & \mybox{} & \mybox{}  & \mybox{} \\
\end{tabular}
\end{table}

%\end{column}
%\end{columns}
\end{minipage}%
\end{Slide}
    
    
    
\begin{Slide}{Variabler som ändrar värden är knepiga}
\begin{itemize}
\item Kod som innehåller variabler som \Alert{förändras} över tid är ofta svårare att läsa och begripa.

\item Många buggar beror på att variabler av misstag förändras på felaktiga och oanade sätt.

\item Föränderliga värden blir speciellt svåra i kod som körs jämlöpande (parallellt).

\item I kod som körs i skarpt läge med många användare (s.k. produktionskod) är därför \code{val} att föredra, medan \code{var} endast används om det \Alert{verkligen} behövs.
\item Alltså: räkna hellre ut nya värden än förändra befintliga.
\end{itemize}
\end{Slide}
    
    
    
\begin{Slide}{Kontrollstrukturer: alternativ och repetition}\SlideFontSmall
Används för att kontrollera (förändra) sekvensen och skapa \Emph{alternativa} vägar genom koden. Vägen  bestäms vid körtid.
\begin{itemize}
\item if-sats:
\begin{Code}
if (math.random() < 0.8) println("gurka") else println("tomat")
\end{Code}
\end{itemize}

Olika sorters \Emph{loopar} för att repetera satser. Antalet repetitioner ges vid körtid.
\begin{itemize}
\item \code{while}-sats: bra när man \Alert{inte vet hur många gånger} det kan bli.
\begin{Code}
while (math.random() < 0.8) println("gurka")
\end{Code}

\item \code{for}-sats: bra när man \Alert{vill ange antalet repetitioner}:
\begin{Code}
for (i <- 1 to 10) println(s"gurka nr $i")
\end{Code}

\end{itemize}
\end{Slide}
    
    % %%%% GE mer detaljer eller överlåta till övning???
    % %\begin{Slide}{if-sats}
    % %\begin{itemize}
    % %\item
    % %\end{itemize}
    % %\end{Slide}
    % %
    % %\begin{Slide}{while-sats}
    % %\begin{itemize}
    % %\item
    % %\end{itemize}
    % %\end{Slide}
    % %
    % %\begin{Slide}{for-sats}
    % %\begin{itemize}
    % %\item
    % %\end{itemize}
    % %\end{Slide}

\begin{Slide}{Procedurer}\SlideFontSmall
\begin{itemize}
\item En \Emph{procedur} är en funktion som \Alert{gör} något intressant, men som \Alert{inte} lämnar något intressant returvärde.
\item Exempel på procedur i standardbiblioteket: \code{println("hej")}
\item Du \Emph{deklarerar egna procedurer} genom att ange \texttt{\Alert{Unit}} som returvärdestyp. Då returneras värdet \texttt{\Alert{()}} som betyder ''inget''.
\end{itemize}
\begin{REPLnonum}
scala> def hej(x: String): Unit = println(s"Hej på dej $x!")
hej: (x: String)Unit

scala> hej("Herr Gurka")
Hej på dej Herr Gurka!

scala> val x = hej("Fru Tomat")
Hej på dej Fru Tomat!
x: Unit = ()
\end{REPLnonum}
\begin{itemize}
\item Det som \Alert{görs} kallas (sido)\Emph{effekt}. Ovan är utskriften själva effekten.
\item Även funktioner kan ha sidoeffekter. De kallas då \Alert{oäkta} funktioner.
\end{itemize}
\end{Slide}

\begin{Slide}{Problemlösning: nedbrytning i abstraktioner som sen kombineras}\SlideFontSmall
\begin{itemize}
\item En av de allra viktigaste principerna inom programmering är \Emph{funktionell nedbrytning} där  \Emph{underprogram} i form av funktioner och procedurer skapas för att bli byggstenar som kombineras till mer avancerade funktioner och procedurer.

\item Genom de namn som definieras skapas \Emph{återanvändbara abstraktioner} som kapslar in det funktionen gör till ett ''byggblock''.

\item Bra ''byggblock'' gör det lättare att lösa svåra programmeringsproblem.

\item Abstraktioner som beräknar eller gör \Emph{en enda, väldefinierad sak} är enklare att använda, jämfört med de som gör många, helt olika saker.

\item Abstraktioner med \Emph{välgenomtänkta namn} är enklare att använda, jämfört med kryptiska eller missvisande namn.
\end{itemize}

\end{Slide}


\end{document}
