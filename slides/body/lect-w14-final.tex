%!TEX encoding = UTF-8 Unicode
%!TEX root = ../lect-w14.tex

%%%


%http://tex.stackexchange.com/questions/135393/how-to-draw-bar-pie-chart
% \definecolor{c1}{RGB}{220,57,18}
% \definecolor{c2}{RGB}{255,153,0}
% \definecolor{c3}{RGB}{102,140,217}
% \definecolor{c4}{RGB}{16,150,24}
% \definecolor{c5}{RGB}{153,0,153}




% \makeatletter

% \tikzstyle{chart}=[
%     legend label/.style={font={\scriptsize},anchor=west,align=left},
%     legend box/.style={rectangle, draw, minimum size=5pt},
%     axis/.style={black,semithick,->},
%     axis label/.style={anchor=east,font={\tiny}},
% ]

% \tikzstyle{bar chart}=[
%     chart,
%     bar width/.code={
%         \pgfmathparse{##1/2}
%         \global\let\bar@w\pgfmathresult
%     },
%     bar/.style={very thick, draw=white},
%     bar label/.style={font={\bf\small},anchor=north},
%     bar value/.style={font={\footnotesize}},
%     bar width=.75,
% ]

% \tikzstyle{pie chart}=[
%     chart,
%     slice/.style={line cap=round, line join=round, very thick,draw=white},
%     pie title/.style={font={\bf}},
%     slice type/.style 2 args={
%         ##1/.style={fill=##2},
%         values of ##1/.style={}
%     }
% ]

% \pgfdeclarelayer{background}
% \pgfdeclarelayer{foreground}
% \pgfsetlayers{background,main,foreground}


% \newcommand{\pie}[3][]{
%     \begin{scope}[#1]
%     \pgfmathsetmacro{\curA}{90}
%     \pgfmathsetmacro{\r}{1}
%     \def\c{(0,0)}
%     \node[pie title] at (90:1.3) {#2};
%     \foreach \v/\s in{#3}{
%         \pgfmathsetmacro{\deltaA}{\v/100*360}
%         \pgfmathsetmacro{\nextA}{\curA + \deltaA}
%         \pgfmathsetmacro{\midA}{(\curA+\nextA)/2}

%         \path[slice,\s] \c
%             -- +(\curA:\r)
%             arc (\curA:\nextA:\r)
%             -- cycle;
%         \pgfmathsetmacro{\d}{max((\deltaA * -(.5/50) + 1) , .5)}

%         \begin{pgfonlayer}{foreground}
%         \path \c -- node[pos=\d,pie values,values of \s]{$\v\%$} +(\midA:\r);
%         \end{pgfonlayer}

%         \global\let\curA\nextA
%     }
%     \end{scope}
% }

% \newcommand{\legend}[2][]{
%     \begin{scope}[#1]
%     \path
%         \foreach \n/\s in {#2}
%             {
%                   ++(0,-10pt) node[\s,legend box] {} +(5pt,0) node[legend label] {\n}
%             }
%     ;
%     \end{scope}
% }


\ifkompendium\else

\Subsection{Sista läsveckan}

\begin{Slide}{Sista läsveckan}

  \begin{itemize}
    \item Det är inga föreläsningar denna vecka.
    \item Endast resurstider schemalagda sista veckan, se \url{https://cs.lth.se/pgk/schema/timeedit/}
    \item Se instruktioner för \Emph{muntligt prov} här: \url{https://fileadmin.cs.lth.se/pgk/lect-w12.pdf}
    \item Gör klart och redovisa \Alert{alla} labbar och projektet
    \item Senaste datum för anmälning till valfria tentamen är i mitten av december, se: \\\href{https://www.student.lth.se/mina-studier/tentamen/}{www.student.lth.se/mina-studier/tentamen/}
  \end{itemize}

\end{Slide}



% \Subsection{Repetition forts.}

% \begin{Slide}{På begäran!}
% \url{http://cs.lth.se/pgk/wish}
% % 2018 \url{https://goo.gl/forms/n4mxwjcAyAuF9Y8k2}
% % 2019 \url{https://forms.gle/z3FvAXnviiqMjbieA}
% \end{Slide}

% \begin{Slide}{Föränderlig punkt  i Scala och Java}\SlideFontSmall
% I Scala:
% \begin{Code}
% class Point(var x: Int, var y: Int)
% \end{Code}
% \pause
% I Java:
% \begin{Code}[language=Java,basicstyle=\ttfamily\SlideFontSize{5.2}{6}]
% public class JPoint {
%     private int x;
%     private int y;

%     public JPoint(int x, int y){
%         this.x = x;
%         this.y = y;
%     }

%     public int getX(){
%         return x;
%     }

%     public int getY(){
%         return y;
%     }

%     public void setX(int x){
%         this.x = x;
%     }

%     public void setY(int y){
%         this.y = y;
%     }
% }
% \end{Code}
% \end{Slide}

% \begin{Slide}{Punkt med räknare och setter i Scala}
% Övning: lägg till getter och setter för y-koordinaten.
% \begin{Code}
% class Point(private var myX: Int, private var myY: Int){
%   import Point._
%   def x = myX
%   def x_=(value: Int): Unit = {
%     myX = value
%   }
%   myCount += 1   // kod i klasskroppen körs vid konstruktion
% }

% object Point {
%   private var myCount = 0
%   def count = myCount
% }
% \end{Code}
% \SlideFontSmall
% Man brukar kalla privata attribut som har getter (och ev. setter) för något i stil med \code{myX} eller vanligare \code{_x} för att namnet inte ska krocka med getter/setter.
% \end{Slide}



% \begin{Slide}{Punkt med räknare i Java}

% \begin{Code}[language=Java,basicstyle=\ttfamily\SlideFontSize{5.2}{6}]
% public class JPoint {
%     private int x;
%     private int y;

%     static private int count = 0;  // static: finns bara en upplaga av detta attribut

%     public JPoint(int x, int y){
%         this.x = x;
%         this.y = y;
%         count++;
%     }

%     public int getX(){
%         return x;
%     }

%     public int getY(){
%         return y;
%     }

%     public void setX(int x){
%         this.x = x;
%     }

%     public void setY(int y){
%         this.y = y;
%     }

%     static public int getCount(){
%        return count;
%     }
% }
% \end{Code}


% \end{Slide}

% \begin{Slide}{Gamla Java-kursen}
% Du hittar genomgång av olika Java-begrepp i gamla Java-kursen som finns här: \\\vspace{1em}
% \url{http://cs.lth.se/eda016/}
% \\\vspace{2em}

% Övning: träna på att översätta Java-exempel till scala

% \end{Slide}


%\Subsection{Utblick}

% \begin{Slide}{Framtidens (data)ingenjör}
% Vårt utbildninguppdrag vid institutionen för Datavetenskap:
% \begin{itemize}
%   \item Utbilda \Emph{dataingenjörer} för \Alert{framtidens} arbetsmarknad
%   \item Säkerställa att alla typer av LTH-ingenjörer har \Emph{tillräckliga} kunskaper i datavetenskap och modern systemutveckling
%   \item Förbereda för \Emph{livslångt lärande} inom \Alert{datavetenskap}: rätt grundkunskaper för att kunna lära sig ny teknik/forskning
%   \item Tillgodo se behovet av den \Emph{djupa datavetenskapliga kompetens} som dagens och morgondagens näringsliv är i så \Alert{skriande behov} av: \\
%   \url{https://computersweden.idg.se/2.2683/1.693037/bransch-it-jobb} \\
% \end{itemize}
% \Alert{''Nytt branschlarm: Behövs 70000 fler som jobbar med it''}
% \end{Slide}
%
% \begin{Slide}{Vad är de två allra största utmaningarna för framtiden inom software engineering?}
%   \begin{enumerate}
%     \item Kompetensbristen
%     \item
%     \pause Att hantera den ständigt ökande \Alert{komplexiteten}
%   \end{enumerate}
%   \pause(Del)lösning: \pause \Emph{kraftfullare abstraktionsmekanismer}
% \end{Slide}
%
%
% \begin{Slide}{Kraftfulla abstraktionsmekanismer i Scala}\SlideFontTiny
% \setlength{\leftmargini}{-1em}
% \begin{itemize}
% \item Funktioner som parametrar:
% \begin{Code}
% def sort[T](xs: Vector[T], lessThan: T => Boolean): Vector[T] = ???
% \end{Code}
%
% \item Extension methods (metodutvidgning):
% \begin{Code}
% implicit class DecoratePerson(p: Person) {
%   def lessThan(other: Person): Boolean = ???
% }
% \end{Code}
%
% \item Implicita parametrar:
% \begin{Code}
% def sort[T](xs: Vector[T])(implicit ord: Ordering[T]): Vector[T] = ???
% \end{Code}
%
% \item Implicita konverteringar mellan typer
%
% \item Typklasser (jmf Haskell) i Scala med traits + implicita värden
%
% \item Scala 3.0: Implicita funktioner för att abstrahera över kontext
%
% \end{itemize}
% {\noindent\tiny Föredrag av Odersky: \\
% ''Plain functional programming'' \url{https://youtu.be/YXDm3WHZT5g}\\
% ''What to leave implicit?'' \url{https://youtu.be/Oij5V7LQJsA?list=PLLMLOC3WM2r5Ei2mnSHCD-ZD04AXovttL}}
% \end{Slide}
%
%
% \begin{Slide}{Framtidens kurser för D-are}
% \begin{itemize}
%   \item \href{http://cs.lth.se/utbildning/}{Undervisningen i Datavetenskap} har fördubblats på 5 år
%   \item D-programmet utvecklas: flera nya kurser införda/på gång
%   \item Exempel på kurser som direkt bygger på grundkursen i programmering med Scala, \code{pgk}:
% \begin{itemize}
% \item Fördjupningskursen (Java)
% \item \Alert{Ny sedan 2016:} Utvärdering av programvarusystem (R)
% \item \Alert{Ny sedan 2016:} Diskreta strukturer (Clojure)
% \item Programvaruutveckling i grupp (Java)
% \item Objekt-orienterad modellering och design (Java)
% \item \Alert{Numera obligatorisk:} Funktionsprogrammering (Haskell)
% \end{itemize}
% \item Exempel på pågående/föreslagen kursutveckling:
% \begin{itemize}
%   \item Machine Learning
%   \item Concurrency, distribution, real-time embedded systems
%   \item Open Source Software Engineering
%   \item Development for web apps, cloud, back-end, (front-end)
%   \item Tool chain for continuous software engineering
% \end{itemize}
% \end{itemize}
% \end{Slide}
%
%
% \begin{Slide}{Mål med nya grundkursen i Scala}
% \begin{itemize}
%   \item Hantera stora spridningen i förkunskaper. Andel nybörjare:
%   \begin{itemize}
%     \item 2015: ca 20\% har aldrig kodat före kursstart
%     \item 2016: ca 30\% har aldrig kodat före kursstart
%     \item 2017: ca 40\% har aldrig kodat före kursstart
%   \end{itemize}
%   \item Hantera stora spridningen i förmåga att ta sig över trösklar:
%   \begin{itemize}
%     \item Många studenter har hög ambition och hög motivation
%     \item Några studenter har stora svårigheter i början
%   \end{itemize}
%   \item Modernisera innehåll och pedagogik:
%   \begin{itemize}
%     \item grundläggande objektfunktionell programmering
%     \item använda ett modernt, kraftfullt samlingsbibliotek
%     \item oföränderliga datastrukturer
%     \item samarbete mellan studenter
%     \item allt kursmaterial är fri öppenkällkod med studentdeltagande i den kontinuerliga utvecklingen:
%     \url{https://github.com/lunduniversity/introprog}
%   \end{itemize}
% \end{itemize}
% \end{Slide}
%
%
% \begin{Slide}{Innehåll i nya grundkursen i Scala}\SlideFontTiny
% Kursens hemsida: \textbf{\url{http://cs.lth.se/pgk}} \\ \vspace{1em}
%
% \noindent\resizebox{0.8\columnwidth}{!}{\fontsize{8}{10}\selectfont
% %!TEX encoding = UTF-8 Unicode
\begin{tabular}{l|l|l|l}
\textit{W} & \textit{Modul} & \textit{Övn} & \textit{Lab} \\ \hline \hline
W01 & Introduktion & expressions & kojo \\
W02 & Program och kontrollstrukturer & programs & -- \\
W03 & Funktioner och abstraktion & functions & irritext \\
W04 & Objekt och inkapsling & objects & blockmole \\
W05 & Klasser och datamodellering & classes & blockbattle0 \\
W06 & Mönster och felhantering & patterns & blockbattle1 \\
W07 & Sekvenser och enumerationer & sequences & shuffle \\
TP & -- & -- & -- \\
W08 & Nästlade och generiska strukturer & matrices & life \\
W09 & Mängder och tabeller & lookup & words \\
W10 & Arv och komposition & inheritance & snake0 \\
W11 & Varians och kontextparametrar & context & snake1 \\
W12 & Fördjupning, Projekt & extra & Projekt0 \\
W13 & Repetition & examprep & Projekt1 \\
W14 & MUNTLIGT PROV & Munta & Munta \\
TP & VALFRI TENTAMEN & -- & -- \\
\end{tabular}

% }
%
% \vspace{1em}
% Kursmaterial är öppen källkod: \textbf{\url{https://github.com/lunduniversity/introprog}}
% \end{Slide}
%
%
%
% \begin{Slide}{Förstaspråk på LTH}%\small
% Historiska förstaspråk på LTH för D-are:
% \begin{table}
% \begin{tabular}{l l}
% (Algol) & (förhistoria\footnote{Scalas uppfinnare Prof. Martin Odersky vid EPFL i Schweiz har skapat stora delar av Java-kompilatorn. Han var doktorand hos Prof. Niklaus Wirth, som ligger bakom Algol, Pascal, Modula etc.}, programmering med hålkort) \\
%  Pascal & 1982, CSE program started\\
%  Simula &  1990\\ % , first OO language; Invented in Scandianiva
%   Java &  1997 \\
% \Emph{Scala} &  2016 \\
% \end{tabular}
% \end{table}
%
% \end{Slide}

%
% \begin{Slide}{Varför Scala? -- ur pedagogisk synvinkel}
% \Emph{Lätt} för nybörjare och \Alert{intressant} för icke-nybörjare:
% \begin{itemize}
% \item Enhetlig semantik; alla värden är objekt
% \item Koncis syntax: drunknar inte i ''bokstavssoppa''
% \item Uttrycksfull semantik: kan demonstrera många koncept
% \item Statisk typning hjälper till med förståelse av abstraktioner
% \item Statisk typning gör att kompilatorn kan hitta buggar tidigt
% \item Interaktivt lärande med Scala REPL
% \item Multi-paradigm, pragmatiskt: \\ imperativt, objekt-orienterat, funktionellt
% \item Modernt språk som utvecklas med forskning och praxis
% \item Språket och verktygen är fri öppen källkod
% \end{itemize}
% \pause {\SlideFontSmall Några \Alert{risker} men som vi identifierade men som \Emph{inte} visat sig vara problem: verktygsmognad, tillgång till kursmaterial för nybörjare, kritisk massa i communityn, industriell acceptans och relevans.}
% %\begin{itemize}\fontsize{9}{10}\selectfont
% %\item Tools not becoming mature fast enough?
% %\item Lack of beginner-oriented teaching material?
% %\item Future industrial relevance?
% %\item Critical mass of community?
% %\end{itemize}
% \end{Slide}
%
% \begin{Slide}{Varför inte Python? -- ur pedagogisk synvinkel}\SlideFontSmall
% Det finns många fördelar med Python som även gäller Scala, t.ex. enhetligt och kraftfullt standardbibliotek och koncis syntax, men också många nackdelar som inte Scala har:
% \begin{itemize}
% \item \Alert{Indenteringssyntaxen} kan göra det svårare att förstå koncepten block, lokala variabler och namnrymd, speciellt precis i början av lärandet, och speciellt vid nästlade strukturer.
% \item \Alert{Avsaknad av typinformation} vid abstraktion och buggrättning - utan t.ex. parametertyper kan nybörjaren råka blanda ihop t.ex. om argument är samlingar eller enskilt element. Dynamisk typning ger onödiga svårigheter där kompilatorn annars kunde ha hjälpt till.
% \item Att (felstavade) variabler kan införas \Alert{utan deklaration} medför mycket svårhittade namnöverskuggningsbuggar även för icke-nybörjare.
% \item Att objektmodellen är så pass långt från "riktig" objektorientering gör att de möjliga \Alert{lärandemålen kring objektorientering begränsas} starkt.
%   \end{itemize}
% \end{Slide}


% \begin{Slide}{Betyg, D-are: 2015 Java; 2016 Scala}
%
%   2015 Java, januaritenta D, 91st
%
%   \pgfplotstableread{
%   betyg antal
%   0 12
%   3 04
%   4 26
%   5 49
%   }{\mydataFifteen}
%
%   \begin{minipage}{0.4\textwidth}
%   \hspace*{-0.0cm}
%   \begin{tikzpicture}[scale=0.8, every node/.style={scale=0.8}]
%       \begin{axis}[
%               ybar,
%               bar width=.5cm,
%               width=\textwidth,
%               height=0.9\textwidth,
%               symbolic x coords={0,3,4,5},
%               xtick=data,
%               nodes near coords,
%               nodes near coords align={vertical},
%               ymin=0,ymax=50,
%               ylabel={Antal},
%               xlabel={Betyg},
%           ]
%           \addplot table[x=betyg,y=antal]{\mydataFifteen};
%       \end{axis}
%   \end{tikzpicture}
%   \end{minipage}%
%   \begin{minipage}{0.3\textwidth}
%   \vspace*{-1cm}\hspace*{1.5cm}
%   \begin{tikzpicture}
%   [
%       pie chart,
%       slice type={NOLL}{c1},
%       slice type={TRE}{c4},
%       slice type={FYRA}{c2},
%       slice type={FEM}{c3},
%       pie values/.style={font={\small}},
%       scale=1.3, every node/.style={scale=0.8}
%   ]
%       \pie{}{13/NOLL,4/TRE,29/FYRA,54/FEM}
%       \legend[shift={(1.5cm,1cm)}]{{0}/NOLL,{3}/TRE, {4}/FYRA, {5}/FEM}
%   \end{tikzpicture}
%   \end{minipage}%
%
%
% \vspace*{1em}
%
% 2016 Scala, januaritenta D, 86st
% \pgfplotstableread{
% betyg antal
% 0 17
% 3 12
% 4 29
% 5 28
% }{\mydataSixteen}
%
% \begin{minipage}{0.4\textwidth}
% \hspace*{-0.0cm}
% \begin{tikzpicture}[scale=0.8, every node/.style={scale=0.8}]
%     \begin{axis}[
%             ybar,
%             bar width=.5cm,
%             width=\textwidth,
%             height=0.9\textwidth,
%             symbolic x coords={0,3,4,5},
%             xtick=data,
%             nodes near coords,
%             nodes near coords align={vertical},
%             ymin=0,ymax=50,
%             ylabel={Antal},
%             xlabel={Betyg},
%         ]
%         \addplot table[x=betyg,y=antal]{\mydataSixteen};
%     \end{axis}
% \end{tikzpicture}
% \end{minipage}%
% \begin{minipage}{0.3\textwidth}
% \vspace*{-1cm}\hspace*{1.5cm}
% \begin{tikzpicture}
% [
%     pie chart,
%     slice type={NOLL}{c1},
%     slice type={TRE}{c4},
%     slice type={FYRA}{c2},
%     slice type={FEM}{c3},
%     pie values/.style={font={\small}},
%     scale=1.3, every node/.style={scale=0.8}
% ]
%     \pie{}{20/NOLL,14/TRE,34/FYRA,33/FEM}
%     \legend[shift={(1.5cm,1cm)}]{{0}/NOLL,{3}/TRE, {4}/FYRA, {5}/FEM}
% \end{tikzpicture}
% \end{minipage}%
%
%
% \end{Slide}
%
%
%
%
% \begin{Slide}{Betygens värde i fördjupningskursen 2015:\\EDA016 Java => EDAA01 Java}
% \href{https://jsfiddle.net/dyb0kpvh/}{\includegraphics[width=1.05\textwidth]{../about/course-experience-first-year/img/2015}}
% \end{Slide}
%
% \begin{Slide}{Betygens värde i fördjupningskursen 2016:\\EDAA45 Scala => EDAA01 Java}
% \href{https://jsfiddle.net/wa7fr8po/}{\includegraphics[width=1.05\textwidth]{../about/course-experience-first-year/img/2016}}
% \end{Slide}
%


% \begin{Slide}{Hur populärt blir Scala bland utvecklare i framtiden?}
% Mest älskade språk på \code{stackoverflow} (2016)
% \includegraphics[width=0.9\textwidth]{../img/w14/most-loved-2016.png}
% \end{Slide}
%
% \begin{Slide}{Hur ser framtidens jobbmarknad för Scala ut?}\SlideFontTiny
%
% \hspace{-2.5em}\begin{minipage}{1.0\textwidth}
% Jobbmarknaden för Scala växer globalt och i Sverige:
%
% \begin{minipage}{0.48\textwidth}
% \href{www.indeed.com/jobtrends/q-scala.html}{www.indeed.com/jobtrends/q-scala.html}\\
% \vspace{1em}
% \includegraphics[width=1.0\textwidth]{../img/w14/scala-jobs-indeed-2017.png}~~
% \end{minipage}
% \hfill\begin{minipage}{0.48\textwidth}
% \vspace{1.25em}
% \href{https://www.linkedin.com/jobs/search/?keywords=scala&location=Sweden&locationId=se%3A0}{https://www.linkedin.com/jobs/search/?keywords=scala}\\
% \includegraphics[width=1.2\textwidth]{../img/w14/scala-jobs-sweden-linkedin-2017-dec07.png}
% \end{minipage}
% \end{minipage}
% \end{Slide}



% \begin{Slide}{Scala versus Python}
%   \begin{itemize}
%     \item 
%     Två språk som blir allt mer populära för ''big data'' och AI: \\ Scala \& Python
%     \item Intressant blogginlägg om vilket språk som funkar bäst som förstaspråk för de som ska bli systemutvecklare:
%     {\tiny\url{https://medium.com/@drmarkclewis/picking-a-languages-for-introductory-cs-the-argument-againstpython-4331cca26cfa}}
%     \item Hur ser framtiden ut för Scala?
%   \end{itemize}
% \end{Slide}




\fi
