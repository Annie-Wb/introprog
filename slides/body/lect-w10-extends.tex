%!TEX encoding = UTF-8 Unicode
%!TEX root = ../lect-w10.tex

%%%


%\begin{Slide}{TODO: Begrepp att förklara}
%  Tänk igenom ordningen:
%  \begin{itemize}
%    \item OO, arv, supertyp, subtyp, bastyp, polymorfism, ...
%  \end{itemize}
%\end{Slide}


\Subsection{Vad är arv?}

\begin{Slide}{Vad är arv?}

\begin{minipage}{0.4\textwidth}
\raggedright Arv \Eng{inheritance} beskriver relationen \\
$X$ \Emph{är en} $Y$

\end{minipage}
\begin{minipage}{0.4\textwidth}
\includegraphics[width=1.5\textwidth]{../img/gurka-tomat-715x800}
\end{minipage}
\end{Slide}


\begin{Slide}{Varför behövs arv?}
\begin{itemize}
\item Man kan använda arv för att dela upp kod i:
\begin{itemize}
\item \Emph{generella} (gemensamma) delar och
\item \Emph{specifika} (specialanpassade) delar.
\end{itemize}

\item Man kan åstadkomma \Emph{kontrollerad flexibilitet}:
\begin{itemize}
\item Klientkod kan \Emph{utvidga} \Eng{extend} ett givet API med egna specifika tillägg.
\end{itemize}

\item Man kan använda arv för att deklarera en gemensam \Emph{bastyp} så att generiska samlingar kan ges en mer specifik elementtyp.
\begin{itemize}
\item Det räcker att man vet bastypen för att kunna anropa gemensamma metoder på alla element i samlingen.
\end{itemize}
\end{itemize}
\end{Slide}

\begin{Slide}{Klassdiagram med UML (Unified Modeling Language)}
\begin{center}
\begin{tikzpicture}
\node (Class) [umlclass, rectangle split parts = 3, xshift = -2cm, yshift=-1.5cm, text width = 4.7cm, scale=0.8]  {
            \textbf{\centerline{Name}}
            \nodepart[]{second}attr1: Type \newline attr2: Type
            \nodepart[]{third}method1(a: Type): Type \newline  method2(b: Type): Type
       };
\node (explain1)[above of = Class, yshift=1.5cm, text width=5.5cm]{UML Klassdiagram:};
%\node (explain2)[below of = Class, yshift=-1cm]{Ibland utelämnar man attribut/metoder};

\pause

\node [umlclass, rectangle split parts = 3, text width = 3.5cm]  at (4,0) (Robot) {
            \texttt{\textbf{\centerline{Robot}}}
            \nodepart[]{second} \code{name: String}
            \nodepart[]{third} \code{work(): Unit}
       };
\node [umlclass, rectangle split parts = 3, text width = 3.5cm]  at (4, -3) (TalkingRobot)  {
            \texttt{\textbf{\centerline{TalkingRobot}}}
            \nodepart[]{second} \code{phrase: String}
            \nodepart[]{third} \code{talk(): Unit}
        };
\draw[umlarrow] (TalkingRobot.north) -- ++(0,0.8) -| (Robot.south);

\end{tikzpicture}
\end{center}
Mer om UML-diagram i senare kurser: pfk, omd.\\
\href{https://en.wikipedia.org/wiki/Class_diagram}{\SlideFontTiny en.wikipedia.org/wiki/Class\_diagram}

\end{Slide}

\begin{Slide}{Exempel: Robot som bastyp för två subtyper}
\begin{center}
\newcommand{\TextBox}[1]{\raisebox{0pt}[1em][0.5em]{#1}}
\begin{tikzpicture}[inner sep=0.5em]
\node [umlclass, rectangle split parts = 2, xshift=0cm, text width=4cm] (AbstractRobot)  {
           \texttt{\textit{\textbf{\centerline{\TextBox{Robot}}}}}
           \nodepart[]{second}\TextBox{\code{def work(): Unit}}
       };

\node [umlclass, rectangle split parts = 2, text width=4cm]  at (3cm,-3cm) (MuteRobot) {
           \texttt{\textbf{\centerline{\TextBox{MuteRobot}}}}
           \nodepart[]{second} \TextBox{~}
       };

\node [umlclass, rectangle split parts = 2, text width=4cm] at (-3cm,-3cm) (TalkingRobot)  {
           \texttt{\textbf{\centerline{\TextBox{TalkingRobot}}}}
           \nodepart[]{second} \TextBox{\code{def talk(): Unit}}
       };
\draw[umlarrow] (TalkingRobot.north) -- ++(0,0.5) -| (AbstractRobot.south);
\draw[umlarrow] (MuteRobot.north) -- ++(0,0.5) -| (AbstractRobot.south);
\end{tikzpicture}
\end{center}
{\SlideFontSmall (Ibland utelämnar man attribut och/eller metoder i ett UML-diagram för att minska antalet detaljer i modellen och ge en överblick.)}
\end{Slide}

\begin{Slide}{Alternativ till arv: komposition}\SlideFontSmall
\begin{itemize}
\item Som alternativ till att klassen X ärver klassen Y kan man i stället använda \Emph{komposition} \Eng{composition}, som innebär att klassen X \Alert{har ett attribut} som \Emph{refererar} till klassen Y.  
\end{itemize}

\begin{multicols}{2}

\newcommand{\TextBox}[1]{\raisebox{0pt}[1em][0.5em]{#1}}

\begin{center}

Exempel på arv i \code{snake}-labben:\\SnakeGame \Alert{är} ett BlockGame\\~\\

\tikzstyle{umlclass}=[rectangle, draw=black,  thick, anchor=north, text width=3cm, rectangle split, rectangle split parts = 3]
\begin{tikzpicture}[inner sep=0.5em,scale=0.8, every node/.style={transform shape}]

  \node [umlclass, rectangle split parts = 1, xshift=0cm, yshift=5cm] (BaseType)  {
              \textbf{\centerline{\TextBox{\code{BlockGame}}}}
%              \nodepart[align=left]{second}\code{def x: T} \newline \code{def y: T}
          };


  \node [umlclass, rectangle split parts = 1, xshift=0cm, yshift=3.0cm] (SubType)  {
              \textbf{\centerline{\TextBox{\code{SnakeGame}}}}
%              \nodepart[align=left]{second}\code{val x: Int} \newline \code{val y: Int}
          };

\draw[umlarrow] (SubType.north) -- ++(0,0.5) -| (BaseType.south);
\end{tikzpicture}

\end{center}

\columnbreak

\noindent Ett alternativ vore \Emph{komposition}:\\SnakeGame \Alert{har} ett BlockGame\\~\\
%\begin{center}
%\newcommand{\TextBox}[1]{\raisebox{0pt}[1em][0.5em]{#1}}
\begin{tikzpicture}[inner sep=0.5em,scale=0.8, every node/.style={transform shape}]
\node [umlclass, rectangle split parts = 2, xshift=0cm, text width=5.3cm] (game)  {
           \texttt{\textbf{\centerline{\TextBox{SnakeGame}}}}
           \nodepart[]{second}\TextBox{\code{val blockGame: BlockGame}}
       };
\end{tikzpicture}
%\end{center}

\end{multicols}
\begin{itemize}
\item Komposition där \Emph{X har en Y} är ofta ett bättre alternativ än arv om: 
\begin{enumerate} \SlideFontTiny
\item det \emph{inte} finns en tydlig \Emph{X-är-en-Y}-relation 
\item det \emph{inte} behövs en \Emph{gemensam bastyp}
\item det \emph{inte} är önskvärt ärva och exponera \emph{alla} Y:s medlemmar via X
\end{enumerate}
\end{itemize}
\end{Slide}

\begin{Slide}{Exempel på komposition i snake-labben}

\begin{multicols}{2}


En \code{Player} \Emph{har en} \footnote{Läs mer om komposition och de relaterade begreppen aggregering, association, ägarskap etc. här:\\\url{https://en.wikipedia.org/wiki/Object_composition}}
 \code{snake}.\\~\\

\newcommand{\TextBox}[1]{\raisebox{0pt}[1em][0.5em]{#1}}
\begin{tikzpicture}[inner sep=0.5em]
\node [umlclass, rectangle split parts = 2, xshift=0cm, text width=4cm] (game)  {
           \texttt{\textbf{\centerline{\TextBox{Player}}}}
           \nodepart[]{second}\TextBox{\code{val snake: Snake}}
       };
\end{tikzpicture}

\columnbreak

~\\~\\

\begin{Code}
class Player(val snake: Snake)
\end{Code} 

\end{multicols}

\end{Slide}

\begin{Slide}{Behovet av gemensam bastyp}\SlideFontSmall
\begin{REPLsmall}
scala> case class Gurka(vikt: Int)

scala> case class Tomat(vikt: Int)

scala> val gurkor = Vector(Gurka(200), Gurka(300))
val gurkor: Vector[Gurka] = Vector(Gurka(200), Gurka(300))

scala> gurkor.map(_.vikt)
res0: Vector[Int] = Vector(200, 300)

scala> val grönsaker = Vector(Gurka(200), Tomat(42))
val grönsaker: Vector[Gurka | Tomat] = Vector(Gurka(200), Tomat(42))

scala> grönsaker.map(_.vikt)
-- [E008] Not Found Error: --------------------------
1 |grönsaker.map(_.vikt)
  |              ^^^^^^
  |              value vikt is not a member of Gurka | Tomat
\end{REPLsmall}
Hur får vi detta att fungera som vi vill? \pause\\$\rightarrow$ Skapa en bastyp \Emph{bastyp} med gemensamt attribut \code{vikt}!
\end{Slide}

\begin{Slide}{Varför syns inte gemensam medlem i en typunion?}
\begin{REPLsmall}
scala> val grönsaker = Vector(Gurka(200), Tomat(42))
val grönsaker: Vector[Gurka | Tomat] = Vector(Gurka@15f11bfb,Tomat@16a499d1)

scala> grönsaker.map(_.vikt)
-- [E008] Not Found Error: --------------------------
1 |grönsaker.map(_.vikt)
  |              ^^^^^^
  |              value vikt is not a member of Gurka | Tomat
\end{REPLsmall}
\begin{itemize}\SlideFontSmall
  \item Typerna \code{Grönsak} och \code{Tomat} är \Alert{orelaterade}. (\code{AnyRef} saknar \code{vikt})
  \item En medlem som råkar ha samma namn har inte alltid samma innebörd.
  \item I Scala måste du explicit relatera medlemmarna genom \Emph{arv} av bastyp med gemensam medlem, eller så får du \Alert{matcha} på unionens delar. 
  \pause\item Du kan erbjuda en sådan matchning som en \Emph{extensionsmetod}:
\end{itemize}
\begin{REPLsmall}
scala> extension (gEllerT: Gurka | Tomat) def vikt = gEllerT match
         case g: Gurka => g.vikt
         case t: Tomat => t.vikt

scala> val vikter = grönsaker.map(_.vikt)
val vikter: Vector[Int] = Vector(200, 42)
\end{REPLsmall}
\end{Slide}


\begin{Slide}{Skapa en gemensam bastyp med arv}
Typen \textit{\textbf{\texttt{Grönsak}}} är en \Emph{bastyp} i nedan arvshierarki:

\vspace{1em}
\begin{center}
\newcommand{\TextBox}[1]{\raisebox{0pt}[1em][0.5em]{#1}}
\tikzstyle{umlclass}=[rectangle, draw=black,  thick, anchor=north, text width=2cm, rectangle split, rectangle split parts = 3]
\begin{tikzpicture}[inner sep=0.5em]
\node [umlclass, rectangle split parts = 1, xshift=0cm] (BaseType)  {
            \textit{\textbf{\centerline{\TextBox{\code{Grönsak}}}}}
            %\nodepart[]{second}\TextBox{\code{val vikt: Int}}
        };

\node [umlclass, rectangle split parts = 1]  at (2cm,-2cm) (SubType1) {
            \textbf{\centerline{\TextBox{\code{Gurka}}}}
            %\nodepart[]{second} \TextBox{~}
        };

\node [umlclass, rectangle split parts = 1] at (-2cm,-2cm) (SubType2)  {
            \textbf{\centerline{\TextBox{\code{Tomat}}}}
            %\nodepart[]{second} \TextBox{talk(): void}
        };
\draw[umlarrow] (SubType1.north) -- ++(0,0.5) -| (BaseType.south);
\draw[umlarrow] (SubType2.north) -- ++(0,0.5) -| (BaseType.south);
\end{tikzpicture}

\pause
\vspace{2em} Pilen ~ \tikz\draw[umlarrow] (0,0) -- (0,0.5); ~ betecknar \Emph{arv} och utläses ''\Alert{är en}''

\pause
{\vspace{1em}\SlideFontSmall 
Typerna \code{Tomat} och \code{Gurka} är \Emph{subtyper} typen \code{Grönsak}.\\
Bastyp som ej ska instansieras direkt är \Alert{abstrakt} (visas med \emph{kursiv} stil).
}
\end{center}
\end{Slide}



\Subsection{Bastyp, \texttt{trait}, \texttt{extends}}



\begin{Slide}{Skapa en gemensam bastyp med \texttt{trait} och \texttt{extends}}\SlideFontSmall
Med en \code{trait Grönsak} kan klasserna \code{Gurka} och \code{Tomat} få en gemensam abstrakt \Emph{bastyp} genom att båda \Emph{subtyperna} gör \code{extends Grönsak}:
\begin{REPL}
scala> trait Grönsak

scala> case class Gurka(vikt: Int) extends Grönsak

scala> case class Tomat(vikt: Int) extends Grönsak

scala> val grönsaker = Vector(Gurka(200), Tomat(42))
val grönsaker: Vector[Grönsak] = Vector(Gurka(200), Tomat(42))


\end{REPL}
\pause
Men det är ännu \Alert{inte} som vi vill ha det:
\begin{REPLnonum}
scala> grönsaker.map(_.vikt)
-- [E008] Not Found Error: --------------------------
1 |grönsaker.map(_.vikt)
  |              ^^^^^^
  |              value vikt is not a member of Grönsak
\end{REPLnonum}
\end{Slide}



\begin{Slide}{En gemensam bastyp med gemensamma delar}\SlideFontSmall
Placera gemensamma medlemmar i bastypen:

\vspace{1em}
\begin{center}
\newcommand{\TextBox}[1]{\raisebox{0pt}[1em][0.5em]{#1}}
\tikzstyle{umlclass}=[rectangle, draw=black,  thick, anchor=north, text width=3cm, rectangle split, rectangle split parts = 3]
\begin{tikzpicture}[inner sep=0.5em]
\node [umlclass, rectangle split parts = 2, xshift=0cm] (BaseType)  {
            \textit{\textbf{\centerline{\TextBox{\code{Grönsak}}}}}
            \nodepart[]{second}\TextBox{\code{val vikt: Int}}
        };

\node [umlclass, rectangle split parts = 1]  at (2cm,-3cm) (SubType1) {
            \textbf{\centerline{\TextBox{\code{Gurka}}}}
            %\nodepart[]{second} \TextBox{~}
        };

\node [umlclass, rectangle split parts = 1] at (-2cm,-3cm) (SubType2)  {
            \textbf{\centerline{\TextBox{\code{Tomat}}}}
            %\nodepart[]{second} \TextBox{talk(): void}
        };
\draw[umlarrow] (SubType1.north) -- ++(0,0.5) -| (BaseType.south);
\draw[umlarrow] (SubType2.north) -- ++(0,0.5) -| (BaseType.south);
\end{tikzpicture}
\end{center}
\vspace{1em}
\begin{itemize}
\item Alla grönsaker har nu attributet \code{val vikt}.
\item Det specifika värdet på vikten definieras \Alert{inte} i bastypen.
\item Medlemen \code{vikt} kallas  \Emph{abstrakt} eftersom den \Alert{saknar implementation} och kan därför inte instansieras direkt.
\end{itemize}
\end{Slide}





\begin{Slide}{Placera gemensamma delar i bastypen}
\SlideFontSmall
Vi inkluderar det gemensamma attributet \code{val vikt} som en \Emph{abstrakt medlem} i bastypen:

\begin{Code}
trait Grönsak:
  val vikt: Int    // implementation saknas, inget =

case class Gurka(vikt: Int) extends Grönsak

case class Tomat(vikt: Int) extends Grönsak
\end{Code}
Nu har du explicit sagt till kompilatorn att du vill att alla grönsaker har en vikt:
\begin{REPLsmall}
scala> val grönsaker = Vector(Gurka(200), Tomat(42))
val grönsaker: Vector[Grönsak] = Vector(Gurka(200), Tomat(42))

scala> grönsaker.map(_.vikt)
val res0: Vector[Int] = Vector(200, 42)
\end{REPLsmall}
Den abstrakta medlemmen \code{vikt} i den abstrakta typen \code{Grönsak} \Emph{implementeras} i en konkret subklass, här som en klassparameter.
\end{Slide}





\begin{Slide}{Scalas typhierarki}\SlideFontSmall
En förenklad bild av den översta delen av typhierarkin i Scala:
\vspace{0.5em}
\begin{center}
\newcommand{\TextBox}[1]{\raisebox{0pt}[1em][0.5em]{#1}}
\tikzstyle{umlclass}=[rectangle, draw=black,  thick, anchor=north, text width=2.5cm, rectangle split, rectangle split parts = 3]
\begin{tikzpicture}[inner sep=0.5em]
\node [umlclass, rectangle split parts = 1, xshift=0cm] (BaseType)  {
            \textit{\textbf{\centerline{\TextBox{\code{Any}}}}}
            %\nodepart[]{second}\TextBox{\code{def toString: String}}
        };

\node [umlclass, rectangle split parts = 1]  at (2cm,-2cm) (SubType1) {
            \textit{\textbf{\centerline{\TextBox{\code{AnyRef}}}}}
            %\nodepart[]{second} \TextBox{~}
        };

\node [umlclass, rectangle split parts = 1] at (-2cm,-2cm) (SubType2)  {
            \textit{\textbf{\centerline{\TextBox{\code{AnyVal}}}}}
            %\nodepart[]{second} \TextBox{talk(): void}
        };
\draw[umlarrow] (SubType1.north) -- ++(0,0.5) -| (BaseType.south);
\draw[umlarrow] (SubType2.north) -- ++(0,0.5) -| (BaseType.south);
\end{tikzpicture}
\end{center}
\begin{itemize}\SlideFontSmall
\item De numeriska typerna \code{Int}, \code{Double}, etc är subtyper till \Emph{\code{AnyVal}} och kallas \Emph{värdetyper} och lagras på ett speciellt, effektivt sätt i minnet.
\item Alla dina egna klasser är subtyper till \Emph{\texttt{AnyRef}} och kallas \Emph{referenstyper} och kan (direkt eller indirekt) konstrueras med \code{new}.
\item \texttt{AnyRef} motsvaras av \Alert{\code{java.lang.Object}} i JVM.
\pause\item {\SlideFontTiny (Det finns även \code{Matchable} som är subtyp till \code{Any} och supertyp till \code{AnyRef} och \code{AnyVal}. Typen \code{Matchable} behövs för att skilja mellan typer som kan undersökas med mönstermatchning och andra s.k. \Emph{opaka typer} (överkurs).) }
\end{itemize}
\end{Slide}



\begin{Slide}{Implicita supertyper till dina egna klasser}
Alla dina egna typer ingår underförstått i Scalas typhierarki:

\vspace{1em}
\begin{center}
\newcommand{\TextBox}[1]{\raisebox{0pt}[1em][0.5em]{#1}}
\tikzstyle{umlclass}=[rectangle, draw=black,  thick, anchor=north, text width=2cm, rectangle split, rectangle split parts = 3]
\begin{tikzpicture}[inner sep=0.5em, scale=0.8, every node/.style={scale=0.8}]
\node [umlclass, rectangle split parts = 1, xshift=0cm] at (0,-0.3cm)(BaseType)  {
            \textit{\textbf{\centerline{\TextBox{\code{Any}}}}}
            %\nodepart[]{second}\TextBox{\code{def toString: String}}
        };

\node [umlclass, rectangle split parts = 1]  at (2cm,-2cm) (SubType1) {
            \textit{\textbf{\centerline{\TextBox{\code{AnyRef}}}}}
            %\nodepart[]{second} \TextBox{~}
        };

\node [umlclass, rectangle split parts = 1] at (-2cm,-2cm) (SubType2)  {
            \textit{\textbf{\centerline{\TextBox{\code{AnyVal}}}}}
            %\nodepart[]{second} \TextBox{talk(): void}
        };


\node [umlclass, rectangle split parts = 1] at (2cm,-3.5cm) (SubSubType)  {
            \textit{\textbf{\centerline{\TextBox{\code{Grönsak}}}}}
            %\nodepart[]{second} \TextBox{talk(): void}
        };

\node [umlclass, rectangle split parts = 1] at (3.5cm,-5.25cm) (SubSubSubType1)  {
            \textbf{\centerline{\TextBox{\code{Gurka}}}}
            %\nodepart[]{second} \TextBox{talk(): void}
        };

\node [umlclass, rectangle split parts = 1] at (0.5cm,-5.25cm) (SubSubSubType2)  {
            \textbf{\centerline{\TextBox{\code{Tomat}}}}
            %\nodepart[]{second} \TextBox{talk(): void}
        };


\draw[umlarrow] (SubType1.north) -- ++(0,0.3) -| (BaseType.south);
\draw[umlarrow] (SubType2.north) -- ++(0,0.3) -| (BaseType.south);
\draw[umlarrow] (SubSubType.north) -- (SubType1.south);
\draw[umlarrow] (SubSubSubType1.north) -- ++(0,0.3) -| (SubSubType.south);
\draw[umlarrow] (SubSubSubType2.north) -- ++(0,0.3) -| (SubSubType.south);
\end{tikzpicture}
\end{center}
\end{Slide}






\begin{Slide}{Vad är en trait?}
\begin{itemize}
\item \Alert{Trait} betyder \Emph{egenskap}.

\item En trait liknar en klass, \Alert{men} speciella regler gäller:

\begin{itemize}

\item den \Emph{kan} innehålla delar som \Emph{saknar implementation}

\item den \Emph{kan mixas} med flera andra traits så att olika koddelar kan återanvändas på flexibla sätt.

\item den \Alert{kan inte} instansieras direkt eftersom den är abstrakt.

\item Från Scala 3 \Emph{kan} den ha \Emph{parametrar} (precis som klasser).
\end{itemize}
\end{itemize}

\end{Slide}

\begin{Slide}{Vad används en trait till?}
En \code{trait} kan användas för att skapa en bastyp som kan vara hemvist för gemensamma delar hos subtyper:
\begin{Code}
trait Bastyp { val x = 42 }                 // Bastyp har medlemmen x
class Subtyp1 extends Bastyp { val y = 43 } // Subtyp1 ärver x, har även y
class Subtyp2 extends Bastyp { val z = 44 } // Subtyp2 ärver x, har även z
\end{Code}
\pause\vspace{-0.5em}
\begin{REPLsmall}
scala> val a = new Subtyp1
val a: Subtyp1 = Subtyp1@51016012

scala> a.x
val res0: Int = 42

scala> a.y
val res1: Int = 43

scala> a.z
-- Error:
  value z is not a member of Subtyp1

scala> new Bastyp
--Error:
  Bastyp is a trait; it cannot be instantiated
\end{REPLsmall}

\end{Slide}


\begin{Slide}{En trait kan ha abstrakta medlemmar}
\begin{Code}
trait X { val x: Int }   // x är abstrakt, d.v.s. saknar implementation
class A extends X { val x = 42 }   // x ges en implementation
class B extends X { val x = 43 }   // x ges en annan implementation
\end{Code}
\pause\vspace{-0.5em}
\begin{REPL}
scala> val a = new A
val a: A = A@5faeada1

scala> val b = B()    // fungerar utan new men då behövs ()
val b: B = B@cb51256

scala> val xs = Vector(a,b)
val xs: Vector[X] = Vector(A@5faeada1, B@cb51256)

scala> xs.map(_.x)
val res0: Vector[Int] = Vector(42, 43)

scala> class Y { val y: Int }
-- Error: 
  class Y needs to be abstract, since val y: Int in class Y is not defined
\end{REPL}
\end{Slide}

\begin{Slide}{En trait kan ha parametrar}
\begin{Code}
trait X(val x: Int)
class A extends X(42)  
class B(y: Int) extends X(y) // värdet av y blir argument till x i X  
\end{Code}
\pause\vspace{-0.5em}
\begin{REPLsmall}
scala> val a = A()
val a: A = A@5faeada1

scala> val b = B(43)
val b: B = B@cb51256

scala> val xs = Vector(a,b)
val xs: Vector[X] = Vector(A@5faeada1, B@cb51256)

scala> xs.map(_.x)
val res0: Vector[Int] = Vector(42, 43)

scala> b.y 
--  Error: 
  value y cannot be accessed as a member of (b : B)
\end{REPLsmall}
\SlideFontTiny
Hur kan vi göra medlemmen \code{y} synlig? \pause \\Lägg till \code{val} framför parameternamnet, eller använd en \code{case}-klass.
\end{Slide}
  


\ifkompendium
\begin{Slide}{Abstrakta och konkreta medlemmar}
\scalainputlisting[numbers=left,numberstyle=]{../compendium/examples/workspace/w07-inherit/src/vego1.scala}
\end{Slide}
\else
\begin{Slide}{Abstrakta och konkreta medlemmar}
\vspace{-0.5em}\scalainputlisting[numbers=left,numberstyle=,basicstyle=\fontsize{6}{7}\ttfamily\selectfont]{../compendium/examples/workspace/w07-inherit/src/vego1.scala}
\end{Slide}
\fi

\begin{Slide}{Undvika kodduplicering med hjälp av arv}
\ifkompendium
\scalainputlisting[numbers=left,numberstyle=,basicstyle=\fontsize{10}{12}\ttfamily\selectfont]{../compendium/examples/workspace/w07-inherit/src/vego2.scala}
\else
  \scalainputlisting[numbers=left,numberstyle=,basicstyle=\fontsize{6}{7.3}\ttfamily\selectfont]{../compendium/examples/workspace/w07-inherit/src/vego2.scala}
\fi
\end{Slide}




\begin{Slide}{Varför kan kodduplicering orsaka problem?}
\begin{itemize}
\item Mer att skriva (inte jättestort problem)
\pause
\item Fler kodrader att läsa och förstå
\pause
\item Fler kodrader som påverkas vid tillägg
\pause

\item Fler kodrader att underhålla:
\begin{itemize}
\item Om man rättar en bug på ett ställe måste man komma ihåg att göra \Alert{exakt samma ändring} på alla de ställen där kodduplicering förekommer $\rightarrow$ \Alert{risk för nya buggar}
\end{itemize}

\pause

\item Principen på engelska: \code{ def DRY = "Don't Repeat Yourself!"}

\pause

\end{itemize}
Det \emph{finns} tillfällen när \Alert{kodduplicering} \Emph{faktiskt är att föredra}: \pause t.ex. om man vill att olika delar av koden ska vara \Alert{helt oberoende} av varandra.
\end{Slide}


\Subsection{Subtypspolymorfism och dynamisk bindning}


\begin{Slide}{Subtypspolymorfism och dynamisk bindning}\SlideFontTiny
\begin{Code}[basicstyle=\SlideFontSize{6.2}{7.5}\ttfamily\selectfont]
trait Robot { def work(): Unit }

case class CleaningBot(name: String) extends Robot:
  def work(): Unit = println(" Städa Städa")

case class TalkingBot(name: String) extends Robot:
  def work(): Unit = println(" Prata Prata")
\end{Code}
\Emph{Polymorfism} betyder ''många former''. Referenserna r och bot nedan kan ha olika ''former'', d.v.s de kan referera till olika sorters robotar. \\ \Emph{Dynamisk bindning} innebär att körtidstypen avgör vilken metod som körs.
\begin{REPL}[numbers=left, basicstyle=\color{white}\SlideFontSize{6.2}{7.5}\ttfamily\selectfont]
scala> def robotDoWork(bot: Robot) = { print(bot); bot.work() }

scala> var r: Robot = CleaningBot("Wall-E")

scala> robotDoWork(r)
CleaningBot(Wall-E) Städa Städa

scala> r = TalkingBot("C3PO")

scala> robotDoWork(r)
TalkingBot(C3PO) Prata Prata
\end{REPL}
\end{Slide}
  
  

\ifkompendium
\begin{Slide}{Exempel: Överskuggning och \texttt{override}}
  \vspace{-0.5em}\scalainputlisting[numbers=left,numberstyle=,basicstyle=\fontsize{11}{13}\ttfamily\selectfont]{../compendium/examples/workspace/w07-inherit/src/vego3.scala}
\end{Slide}
\else
\begin{Slide}{Exempel: Överskuggning och \texttt{override}} \SlideFontSmall
  \vspace{-0.5em}\scalainputlisting[numbers=left,numberstyle=,basicstyle=\fontsize{6}{7.3}\ttfamily\selectfont]{../compendium/examples/workspace/w07-inherit/src/vego3.scala}
\end{Slide}
\fi


\begin{Slide}{En final medlem kan ej överskuggas}
\ifkompendium
\vspace{-0.5em}\scalainputlisting[numbers=left,numberstyle=,basicstyle=\fontsize{11}{13}\ttfamily\selectfont]{../compendium/examples/workspace/w07-inherit/src/vego4.scala}
\else
\vspace{-0.5em}\scalainputlisting[numbers=left,numberstyle=,basicstyle=\fontsize{7}{9}\ttfamily\selectfont]{../compendium/examples/workspace/w07-inherit/src/vego4.scala}
\fi
\end{Slide}


\begin{Slide}{Protected ger synlighet begränsad till subtyper}
\begin{REPLsmall}
scala> trait Super:
         private val minHemlis = 42
         protected val vårHemlis = 42

scala> class Sub extends Super { def avslöjad = minHemlis }
- Error: Not found: minHemlis

scala> class Sub extends Super { def avslöjad = vårHemlis }

scala> val s = Sub()
val s: Sub = Sub@2eee9593

scala> s.avslöjad
val res0: Int = 42

scala> s.minHemlis
-- Error: 
  value minHemlis is not a member of Sub - did you mean s.vårHemlis?

scala> s.vårHemlis
-- Error: 
  Access to protected value vårHemlis not permitted because enclosing object
  is not a subclass of trait Super where target is defined
\end{REPLsmall}
\end{Slide}


\begin{Slide}{Filnamnsregler och -konventioner}\SlideFontSmall
\begin{itemize}
\item I flera språk, t.ex. Java, gäller dessa regler (men \Alert{inte} i Scala):
\begin{itemize}\SlideFontTiny
\item Det går bara ha \Alert{en enda publik klass per kodfil}.
\item Kodfilen måste ha \Alert{samma namn} som den publika klassen, t.ex. \code{KlassensNamn.java}
\end{itemize}
\item Scala är \Emph{friare}:
\begin{itemize}\SlideFontTiny
\item I Scala får man ha \Emph{så många} icke-privata klasser/traits/singelobjekt i samma kodfil \Emph{som man vill}.
\item I Scala får man döpa kodfilerna \Emph{oberoende} av deras innehåll. \pause 
\end{itemize}

\item Dessa \Emph{konventioner} brukar användas i Scala:
\begin{itemize}\SlideFontTiny
\item Om en kodfil bara innehåller \Emph{en enda} klass/trait/singelobjekt ge filen samma namn som innehållet, t.ex. \code{KlassensNamn.scala}
\item Om en kodfil innehåller \Emph{flera} saker, döp filen till något som återspeglar hela innehållet och använd \Emph{liten begynnelsebokstav}, t.ex. \code{drawing-utils.scala} eller \code{bastypensNamn.scala}
\end{itemize}
\end{itemize}
\end{Slide}


\begin{Slide}{Klasser, arv och klassparametrar}\SlideFontTiny
Klasser kan ärva andra typer (klasser och traits). Om supertypen har parametrar så \Alert{måste} subtypen ge argument efter \code{extends}.

\ifkompendium
\scalainputlisting[numbers=left,numberstyle=,basicstyle=\fontsize{10}{12}\ttfamily\selectfont]{../compendium/examples/workspace/w07-inherit/src/personExample1.scala}
\else
\scalainputlisting[numbers=left,numberstyle=,basicstyle=\fontsize{6.4}{7.7}\ttfamily\selectfont]{../compendium/examples/workspace/w07-inherit/src/personExample1.scala}
\fi 
\end{Slide}


\begin{Slide}{Statisk och dynamisk typ}\SlideFontSmall
\begin{Code}
    var p: Person = Forskare("Robin Smith", "Lund", "Professor Dr")
\end{Code}
\begin{itemize}\SlideFontTiny
\item Den \Emph{statiska typen} för \code{p} är \code{Person} vilket gör att vi sedan kan låta \code{p} referera till \emph{andra} instanser som är av typen Person.
\begin{Code}
p = Student("Kim Robinson", "Lund", "Data")
\end{Code}

\pause

\item Med ''statisk typ'' menas den typinformation som finns vid \Alert{kompileringstid}.

\pause
\item Den \Emph{dynamiska typen}, även kallad \Emph{körtidstypen} som gäller vid exekvering kan vara mer specifik: den dynamiska typen för \code{p} är nu efter ovan tilldelning Student, men också Akademiker och Person.

\pause

\item  Man kan undersöka om den dynamiska typen för \code{p} är \code{EnVissTyp} med \\ \code{p.isInstanceOf[EnVissTyp]} (men använd hellre \code{match})

\pause

\item  Man kan säga åt kompilatorn: \Alert{''jag garanterar att p är av typen \code{EnVissTyp} så du kan omforma den statiska typen till \code{EnVissTyp} och så får jag stå ut med körtidsfel om jag ljuger''} genom att göra \Emph{typomvandling} \Eng{type casting} med    
\code{p.asInstanceOf[EnVissTyp]}   (detta är mycket ovanligt i normal Scala-kod, eftersom typtest + typomvandling görs säkrare med \code{match})
\end{itemize}
\end{Slide}



\ifkompendium
\begin{Slide}{Inmixning}\SlideFontTiny
Man kan ärva flera traits. Detta kallas \Emph{inmixning} \Eng{mix-in} och görs med en komma-separerad typ-lista efter \code{extends}.
\scalainputlisting[numbers=left,numberstyle=,basicstyle=\fontsize{10}{12.5}\ttfamily\selectfont]{../compendium/examples/workspace/w07-inherit/src/personExample2.scala}
\end{Slide}
\else
\begin{Slide}{Inmixning}\SlideFontTiny
Man kan ärva flera traits. Detta kallas \Emph{inmixning} \Eng{mix-in} och görs med en komma-separerad typ-lista efter \code{extends}.
\scalainputlisting[numbers=left,numberstyle=,basicstyle=\fontsize{7}{9}\ttfamily\selectfont]{../compendium/examples/workspace/w07-inherit/src/personExample2.scala}
\end{Slide}
\fi



\begin{Slide}{\texttt{isInstanceOf} och \texttt{asInstanceOf}}\SlideFontTiny
Testa körtidstyp med \code{isInstanceOf[Typ]}. Lova kompilatorn (och ta själv ansvar för) att det är en viss körtidstyp med \code{asInstanceOf[Typ]}. OBS! Använd hellre \code{match}.
\scalainputlisting[numbers=left,numberstyle=,basicstyle=\small\SlideFontSize{6.2}{7.6}\ttfamily\selectfont]{../compendium/examples/workspace/w07-inherit/src/personExample3.scala}
\end{Slide}



  


\begin{Slide}{Anonym klass}\SlideFontSmall
Om man har en abstrakt typ med saknade implementationer kan man fylla i det som fattas i dessa i ett extra block som ''hängs på'' vid instansiering:
\begin{REPL}
scala> trait Grönsak { val vikt: Int }
// defined trait Grönsak

scala> new Grönsak    // eller Grönsak()
-- Error:
1 |new Grönsak
  |    ^^^^^^^
  |    Grönsak is a trait; it cannot be instantiated

scala> new Grönsak { val vikt = 42 }
val res0: Grönsak = anon1@4e3f2908
\end{REPL}
Man får då vad som kallas en \Emph{anonym klass}. (I detta fall en ganska konstig grönsak som inte är någon speciell sorts grönsak, men som ändå har en vikt.)

\vspace{0.5em}

Den allra enklaste (och mest meningslösa) anonyma klassen är:
\begin{REPLsmall}
scala> new {}
val res0: Object = anon1@5bb37371
\end{REPLsmall}
\end{Slide}


\begin{Slide}{Hur förhindra subtypning?}
Du kan förhindra subtypning på dessa sätt:  
\begin{itemize}
\item Med \code{final} skapar du en final klass som ej går att ärva:
\begin{Code}
final class Person(name: String)
\end{Code}   
\begin{REPLsmall}
scala> object Björn extends Person("Björn")
-- Error:
1 |object Björn extends Person("Björn")
  |       ^
  |       object Björn cannot extend final class Person

\end{REPLsmall}
\item Med \code{sealed} kan du försegla en hel typhierarki:
\begin{REPLsmall}
scala> sealed trait T  // tryck Alt+TAB för att vänta med evaluering
     | final class A extends T
     | final class B extends T
     |
scala> new T{}
-- Error:
1 |new T{}
|^
|Cannot extend sealed trait T in a different source file
\end{REPLsmall}
Med en förseglad typhierarki kan du bara ärva bastypen inom samma kodfil. 
\end{itemize}
\end{Slide}


\begin{Slide}{Förseglade typer med \texttt{sealed}}\SlideFontTiny
Med en \code{sealed} kan du skapa en \Emph{förseglad} uppräkning:
\begin{Code}
sealed trait Färg(val toInt: Int)
object Färg:
  val values = Vector(Spader, Hjärter, Ruter, Klöver)
  case object Spader  extends Färg(0)
  case object Hjärter extends Färg(1)
  case object Ruter   extends Färg(2)
  case object Klöver  extends Färg(3)
\end{Code}
Nyckelordet \code{sealed} \Alert{förhindrar} vidare subtypning av \code{Färg} i \Emph{annan kodfil} och \Alert{ger varning} om matchning är ofullständig -- tips för att undvika körtidsfel.

\begin{REPLsmall}
scala> Färg.values(0) match { case Färg.Spader => "hej" }
-- Warning:
1 |Färg.values(0) match { case Färg.Spader => "hej" }
  |^^^^^^^^^^^^^^
  |match may not be exhaustive.
  |
  |It would fail on pattern case: Hjärter, Ruter, Klöver
val res0: String = hej
\end{REPLsmall}
Använd hellre \code{enum} så får du både \code{sealed} och mer godis på köpet!
\end{Slide}
  
        
  

\begin{Slide}{Öppen klass signalerar uppmuntrad subtypning}\SlideFontSmall
\begin{itemize}
\item Om man ärver en klass får man tillgång till alla medlemmar som inte är privata och kan byta till godtycklig implementation om typerna stämmer.
\item Detta kan vara riskabelt om den som skrivit klassen inte planerat för detta och noga dokumenterat hur klassen är tänkt att användas vid arv. 
\item Genom att skapa \Emph{öppna klasser} med nyckelordet \code{open} signalerar du att klassen är tänkt att vara en supertyp vid arv.
\begin{Code}
open class Gurka(val vikt: Int, val pris: Double):
  /** Gör override på denna om du vill ha annat alternativ. */
  def alternativ: String = s"Det går precis lika bra med selleri!"  
\end{Code}
\begin{REPLsmall}
scala> object StorGurkan extends Gurka(1000, 1_000_000): // ärv på bäst du vill!
         override def alternativ = "kan kanske också funka med en betongpelare"
\end{REPLsmall}
\item \code{open} krävs om du vill tysta varning vid \Alert{arv från en annan kodfil}.
\item Du kan även komma undan varningen med \code{import scala.language.adhocExtensions} 
\end{itemize}  

\url{https://youtu.be/aFmIS5qeetA?t=206}
\end{Slide}


\begin{Slide}{Trait eller abstrakt klass?}\SlideFontSmall
Nyckelordet \code{abstract} behövs framför \code{class} om abstrakta medlemmar:
\begin{REPLsmall}
scala> class X { val x: Int }
1 |class X { val x: Int }
  |      ^
  |      class X needs to be abstract, since val x: Int in class X is not defined 

scala> abstract class X { val x: Int }  // fungerar!
\end{REPLsmall}  
Men går det inte lika bra med en trait? \pause Det går ofta \href{https://youtu.be/aFmIS5qeetA?t=221}{precis lika bra med en trait}.%
\label{slideW07:traitorclass}%
\begin{multicols}{2}
\noindent Använd en \Emph{trait} om...
\begin{itemize}\SlideFontTiny
\item ...du är osäker på vilket som är bäst. (Du kan alltid ändra till en abstrakt klass senare.)
\item ...du vill kunna mixa in din trait tillsammans med andra traits.
\item ...du vill göra din trait, om inmixad, transparent vid typhärledning.
%\item ...du vill skapa ett flexibelt gränssnitt som del i ett api.

\end{itemize}

\columnbreak

\noindent Använd en (abstrakt) \Alert{klass} om...
\begin{itemize}\SlideFontTiny
\item ...du vill begränsa inmixning -- man kan bara ärva från en enda klass. 
\item ...du vill vidarebefordra parameter till supertyp (se nästa bild).
\item ...du vill ärva din klass i Java-kod.
\item ...du vill minimera tid för omkompilering vid ändringar (spar tid vid stora projekt).
\end{itemize}
\end{multicols}
\end{Slide}

\begin{Slide}{En trait får ej vidarebefordra parametrar}\SlideFontSmall
En trait får inte skicka vidare parametrar till en supertyp (det skulle bli knepiga problem annars vid inmixning):
\begin{REPLnonum}
scala> trait X(x: Int)
// defined trait X

scala> trait Y(y: Int) extends X(y)
-- Error:
1 |trait Y(y: Int) extends X(y)
  |                        ^^^^
  |  trait Y may not call constructor of trait X
\end{REPLnonum}
Men det funkar fint med en \code{abstract class} eller en \code{class} (som ju \Alert{inte} får vara inmixningar):
\begin{REPLnonum}
scala> abstract class Y(y: Int) extends X(y)                                                           
// defined class Y
\end{REPLnonum}
{ \SlideFontTiny Mer detaljer här: \href{https://dotty.epfl.ch/docs/reference/other-new-features/trait-parameters.html}{dotty.epfl.ch/docs/reference/other-new-features/trait-parameters.html}}
\end{Slide}



