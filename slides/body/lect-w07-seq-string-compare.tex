%!TEX encoding = UTF-8 Unicode
%!TEX root = ../lect-w07.tex

%%%

\Subsection{Jämföra strängar}

\begin{Slide}{Att sortera och jämföra strängar lexikografiskt}\SlideFontSmall
Teckenstandard \href{https://sv.wikipedia.org/wiki/UTF-8}{UTF-8}: Alla stora bokstäver är \href{https://www.youtube.com/watch?v=MijmeoH9LT4}{''mindre''} än alla små:
\begin{REPLnonum}
scala> Array("hej","Hej","gurka").sorted
\end{REPLnonum}
\pause\vspace{-1.2em}
\begin{REPLnonum}
res0: Array[String] = Array(Hej, gurka, hej)\end{REPLnonum}
\pause
\begin{itemize}
\item Antag att vi vill lösa detta problem ''från scratch'': \\ \Emph{att sortera en sekvens med strängar}
\item Följdfrågor:
\begin{itemize}\SlideFontTiny
 \item Vad betyder det att två strängar är ''lika''?
\item Vad betyder det att en sträng är ''mindre'' än en annan?
\end{itemize}
\item För att sortera en strängsekvens behöver vi lösa dessa delproblemen:
\begin{itemize}\SlideFontTiny
\item \Emph{att} \Alert{jämföra strängar}
\item \Emph{sökning i sekvenser}
\item \Emph{SWAP} (om på-plats-sortering i förändringsbar sekvens)
\end{itemize}
\end{itemize}
\pause {\SlideFontTiny Vi använder här strängjämförelse, sökning och sortering för att illustrera typiska \Emph{imperativa algoritmer}. \Alert{Normalt} använder man \Emph{färdiga lösningar} på dessa problem!}

\end{Slide}

\begin{Slide}{Jämföra strängar: likhet}\SlideFontSmall
Antag att vi inte kan göra \code{s1 == s2} utan bara kan jämföra strängar tecken för tecken,
t.ex. så här: \code{s1(i) == s2(i)}. Antag också att vi inte har tillgång till annat än metoderna \code{length} och \code{apply} på strängar, samt  \code{while} och variabler av grundtyp. \Emph{Lös problemet att \emph{avgöra om två strängar är lika}.}

\pause
\begin{itemize}
\item Indata: två strängar
\item Utdata: \code{true} om lika annars \code{false}
\end{itemize}
\begin{enumerate}
\item Klura ut din lösningsidé
\item Formulera algoritmen i pseudokod
\item Implementera algoritmen i Scala: \\\code{def isEqual(s1: String, s2: String): Boolean} = ???
\end{enumerate}
\end{Slide}

\begin{Slide}{Algoritmexempel: stränglikhet, pseudokod}
\begin{Code}
def isEqual(s1: String, s2: String): Boolean = 
  if (/* lika längder */) then
    var foundDiff = false
    var i = /* första index */
    while !foundDiff && /* i inom indexgräns */ do
      if /* tecken på plats i är olika */ then foundDiff = true
      else i = /* nästa index */
    end while
    !foundDiff
  else false
end isEqual
\end{Code}

\pause\noindent Detta är en variant av s.k. \Emph{linjärsökning} där vi söker från början i en sekvens till vi hittar det vi söker efter (här söker vi efter tecken som skiljer sig åt).
\\\pause\vspace{1em}

\noindent Hur ser implementationen i exekverbar Scala ut?
\end{Slide}

\begin{Slide}{Algoritmexempel: stränglikhet, implementation}\SlideFontSmall
\begin{Code}
def isEqual(s1: String, s2: String): Boolean = 
  if s1.length == s2.length then
    var foundDiff = false
    var i = 0
    while !foundDiff && i < s1.length do
      if s1(i) != s2(i) then foundDiff = true
      else i += 1
    end while
    !foundDiff
  else false
end isEqual
\end{Code}

% \pause
% {\SlideFontTiny \emph{Fördjupning:} Jämför ovan med implementationen av \code{String.equals} här:\\
% \href{http://hg.openjdk.java.net/jdk8u/jdk8u60/jdk/file/935758609767/src/share/classes/java/lang/String.java#l976}{hg.openjdk.java.net/jdk8u/jdk8u60/jdk/file/935758609767/src/share/classes/java} \\ och använd \code{timed} nedan och jämför prestanda med \code{isEqual} ovan.\\
% Obs! Mät efter flera körningar då JVM optimerar bytekoden efter ett tag (s.k. ''uppvärmning'').}

% \vspace{-0.25em}\begin{Code}
% def timed[T](block: => T): (Double, T) = {
%   val (t, res) = (System.nanoTime, block)
%   ((System.nanoTime - t) / 1e9, res)
% }
% \end{Code}

\end{Slide}
 
% \begin{Slide}{Algoritmexempel: stränglikhet, prestanda}
% \begin{REPL}
% scala> val enMiljon = 1000000

% scala> val s = Array.fill(enMiljon)('x').mkString

% scala> val t = s.updated(enMiljon - 1, 'y')

% scala> timed { s == t }
% res42: (Double, Boolean) = (3.76459E-4,false)

% scala> timed { isEqual(s,t) }
% res43: (Double, Boolean) = (3.31597E-4,false)
% \end{REPL}
% Ovan är kört efter ''uppvärmning'' på i7-4790K CPU @ 4.00GHz \\
% Skillnaden inom mätfelmarginalen!
% \end{Slide}



\begin{Slide}{Jämföra strängar: ''mindre än''}\SlideFontSmall
Med \code{s1 < s2} menar vi att strängen s1 ska sorteras före strängen s2 enligt hur de enskilda tecknen är ordnade med uttrycket \code{s1(i) < s2(i)}. \\
Antag också att vi inte har tillgång till annat än metoderna \code{length} och \code{apply} på strängar, samt  \code{while} och variabler av grundtyp, samt \code{math.min}
\\\Emph{Lös problemet att \emph{avgöra om en sträng är ''mindre'' än en annan}.}\\
\begin{itemize}
\item Indata: två strängar, s1, s2
\item Utdata: \code{true} om s1 ska sorteras före s2 annars \code{false}
\end{itemize}
\begin{enumerate}
\item Klura ut din lösningsidé
\item Formulera algoritmen i pseudokod
\item Implementera algoritmen i Scala: \\\code{def isLessThan(s1: String, s2: String): Boolean} = ???
\end{enumerate}
\end{Slide}

\begin{Slide}{Jämföra strängar: ''mindre än''}\SlideFontSmall
Pseudokod:
\begin{Code}
def isLessThan(s1: String, s2: String): Boolean = 
  val minLength = /* minimum av längderna på s1 och s2 */

  def firstDiff(s1: String, s2: String): Int =
    /* index för första skillnaden (om de börjar lika: minLength) */

  val diffIndex = firstDiff(s1, s2)
  if diffIndex == minLength then /* s1 är kortare än s2 */
  else /* tecknet s1(diffIndex) är mindre än tecknet s2(diffIndex) */
\end{Code}
\end{Slide}

\begin{Slide}{Jämföra strängar: ''mindre än''}\SlideFontSmall
\begin{Code}
def isLessThan(s1: String, s2: String): Boolean = 
  val minLength = math.min(s1.length, s2.length)

  def firstDiff(s1: String, s2: String): Int = 
    var foundDiff = false
    var i = 0
    while !foundDiff && i < minLength do
      if (s1(i) != s2(i)) foundDiff = true
      else i += 1
    end while
    i
  end firstDiff  

  val diffIndex = firstDiff(s1, s2)
  if diffIndex == minLength then s1.length < s2.length
  else s1(diffIndex) < s2(diffIndex)
end isLessThan
\end{Code}
\end{Slide}

