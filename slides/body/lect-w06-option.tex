%!TEX encoding = UTF-8 Unicode
%!TEX root = ../lect-w06.tex

%%%


\ifkompendium\else
%\Subsection{Kapsla in speciella värden: Option[kanske saknas] och Try[kanske misslyckas]}
\Subsection{Hantera speciella värden med inkapsling}

{
\setbeamertemplate{navigation symbols}{}
\setbeamercolor{background canvas}{bg=black}
\begin{frame}[plain]
    \color{white}{Inkapsling av speciella värden så att krasch kan undvikas}
    \makebox[\linewidth]{\includegraphics[width=\paperwidth]{../img/crystal.jpg}}
\end{frame}
}
\fi

\Subsection{Hantera saknade värden med \texttt{Option}}

\begin{Slide}{Hur hantera saknade värden?}\SlideFontSmall
Olika sätt att hantera saknade värden:
\begin{itemize}
\item Hitta på ett specialvärde: exempel -1 för saknat värde
\item \code{null} om värde saknas (vanligt i Java m.fl. språk, mkt ovanligt i Scala)
\item Använd en samling och låt tom samling representera saknat värde: \\
\code{val sums = Vector(Vector(42),Vector(32),Vector(),Vector(21))}
\pause
\item \code{Option[A]} gemensam bastyp för: \\
  \code{None} som representerar \Alert{saknat värde}, och \\ \code{Some[A]} som representerar att \Emph{värde finns}
\end{itemize}
\end{Slide}



\begin{Slide}{En gemensam bastyp för ett värde som kanske saknas}\SlideFontSmall\ifkompendium\footnotesize\fi
\vspace{-0.0em}\begin{center}
\newcommand{\TextBox}[1]{\raisebox{0pt}[1em][0.5em]{#1}}
\tikzstyle{umlclass}=[rectangle, draw=black,  thick, anchor=north, text width=3cm, rectangle split, rectangle split parts = 3]
\begin{tikzpicture}[inner sep=0.5em]
\node [umlclass, rectangle split parts = 2, xshift=0cm, text width=3.5cm] (BaseType)  {
            \textit{\textbf{\centerline{\TextBox{\code{Option[A]}}}}}
            \nodepart[]{second}
            \TextBox{\code{def get: A}}\newline
            \TextBox{\code{def isEmpty: Boolean}}

        };

\node [umlclass, rectangle split parts = 1]  at (-2.5cm,-3.0cm) (SubType1) {
            \textbf{\centerline{\TextBox{\code{Some[A]}}}}
            % \nodepart[]{second} \TextBox{\code{val x: A}}
        };

\node [umlclass, rectangle split parts = 1] at (2.5cm,-3.0cm) (SubType2)  {
            \textbf{\centerline{\TextBox{\code{None}}}}
        };
\draw[umlarrow] (SubType1.north) -- ++(0,0.5) -| (BaseType.south);
\draw[umlarrow] (SubType2.north) -- ++(0,0.5) -| (BaseType.south);
\end{tikzpicture}
\end{center}
\pause
\vspace{-0.5em}\begin{REPL}
scala> var x: Option[Int] = Some(42)

scala> x.isEmpty
val res0: Boolean = false

scala> x = None

scala> x.isEmpty
val res1: Boolean = true
\end{REPL}
\end{Slide}


\begin{Slide}{Option för hantering av ev. saknade värden}\SlideFontSmall
Alla vill inte berätta för Facebook vad de har för kön. \\ Förbättra Facebooks kod med ett litet Scala-program:
\begin{Code}
enum Gender:
  case Male, Female

case class Person(name: String, gender: Option[Gender])
\end{Code}
\pause
\begin{REPL}
scala> val p1 = Person("Björn",  Some(Gender.Male))
scala> val p2 = Person("Sandra", Some(Gender.Female))
scala> val p3 = Person("Kim",  None)
scala> val g2 = p2.gender
scala> def show(g: Option[Gender]): String = g match {
         case Some(x) => x.toString
         case None    => "unknown"
       }
scala> show(g2)
scala> show(p3.gender)
scala> val ps = Vector(p1,p2,p3)
scala> ps.map(_.gender).map(show)   // None ignoreras av map
\end{REPL}
\end{Slide}

\begin{Slide}{Några smidiga metoder på \code{Option}}\SlideFontSmall
Metoden \code{getOrElse} gör att man ofta kan undvika matchning.
\begin{Code}
var opt: Option[Int] = None

val x = opt.getOrElse(42)   // get the value, give default if missing
\end{Code}

Flera av de vanliga samlingsmetoderna funkar, t.ex. \code{foreach} och \code{map}.
\begin{Code}
opt.foreach(x => println(x)) // only done if value exists

opt.map(x => x + 1)          // only done if value exists

opt = Some(42)               // change opt to now have some value

opt.foreach(x => println(x)) // done as value now exists

opt.map(x => x + 1)          // done as value now exists

\end{Code}
\end{Slide}


\begin{Slide}{Några samlingsmetoder som ger en \code{Option}, övning}
\begin{REPL}
scala> val (xs, ys) = (Vector(1,2,3), Vector())

scala> xs.headOption
res0: ???

scala> ys.headOption
res1: ???

scala> xs.find(_ > 1)
res2: ???

scala> xs.find(_ > 5)
res3: ???

scala> val huvudstad = Map("Sverige" -> "Sthlm", "Skåne" -> "Malmö")

scala> huvudstad.get("Skåne")
res4: ???

scala> huvudstad.get("Danmark")
res5: ???
\end{REPL}
\end{Slide}

\begin{Slide}{Några samlingsmetoder som ger en \code{Option}, svar}
\begin{REPL}
scala> val (xs, ys) = (Vector(1,2,3), Vector())

scala> xs.headOption
res0: Option[Int] = Some(1)

scala> ys.headOption
res1: Option[Nothing] = None

scala> xs.find(_ > 1)
res2: Option[Int] = Some(2)

scala> xs.find(_ > 5)
res3: Option[Int] = None

scala> val huvudstad = Map("Sverige" -> "Sthlm", "Skåne" -> "Malmö")

scala> huvudstad.get("Skåne")
res4: Option[String] = Some(Malmö)

scala> huvudstad.get("Danmark")
res5: Option[String] = None
\end{REPL}
\end{Slide}
