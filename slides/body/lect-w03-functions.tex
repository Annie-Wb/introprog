%!TEX encoding = UTF-8 Unicode
%!TEX root = ../lect-w03.tex


% \ifkompendium\else
% \Subsection{Scala 2.12.10}

% \begin{SlideExtra}{Scala 2.12.10}
%   Den 10:e September släpptes Scala 2.12.10 med viktigaste förbättringen att filtrering i Scaladoc nu fungerar igen. 
%   \\~\\
%   \url{https://www.scala-lang.org/news/2.12.10}
%   \\~\\ Du kan gärna installera 2.12.10 om du vill, men inte nödvändigt.
%   \\~\\ När du kollar api-dokumentationen för Scalas standardbibliotek så använd denna länk:\\ 
%   \url{https://www.scala-lang.org/api/2.12.10}

% \end{SlideExtra}
% \fi

\Subsection{Abstraktion}

\begin{Slide}{Vad är abstraktion?}
\begin{itemize}
  \item \Emph{Abstraktion} innebär att skapa en förenklad \Emph{modell} ur konkreta detaljer 
  \item Vi ''hittar på'' nya \Emph{begrepp} som ger oss återanvändbara ''byggblock'' för våra tankar och vår kommunikation
  \item Vi får ett abstrakt \Emph{namn} som kan användas i stället för en massa \Alert{konkreta detaljer}
  \item Skilj på abstraktionens \Emph{namn} (begrepp, koncept), dess \Emph{användning} (anrop) och dess detaljerade \Emph{beskrivning} (definition, implementation)
  \item \Emph{Funktioner} (som du redan känner från matematiken) är en av våra \Alert{viktigaste} abstraktionsmekanismer
\end{itemize}
\url{https://sv.wikipedia.org/wiki/Abstraktion}
\url{https://en.wikipedia.org/wiki/Abstraction}
\end{Slide}

\begin{Slide}{Exempel på abstraktionsmekanismer inom datavetenskapen}
Vi kommer att behandla flera olika, alltmer \Emph{kraftfulla} abstraktionsmekanismer i denna kurs:
\begin{itemize}
  \item Funktioner
  \item Objekt
  \item Klasser
  \item Arv
  \item Generiska strukturer
  \item Kontextuella abstraktioner
\end{itemize}
Dessa abstraktionsmekanismer blir \Emph{extra kraftfulla} om de \Alert{kombineras}! 
\end{Slide}

\Subsection{Vad är en funktion?}


\ifkompendium\else
\begin{SlideExtra}{Om veckans tema: Funktioner}
\begin{itemize}
  \item Funktioner är en av de viktigaste abstraktionsmekanismerna inom datavetenskapen
  \item Du kan redan massor om funktioner, bl.a. från matematiken.
  \item Denna vecka ska vi fördjupa förståelsen:
  \begin{itemize}\SlideFontSize{6}{8}
    \item överlagring
    \item enhetlig access
    \item defaultargument
    \item namngivna argument
    \item lokala funktioner
    \item funktioner som äkta värden
    \item anonyma funktioner
    \item klammerparentes vid ensam paramenter
    \item multipla parameterlistor
    \item egendefinierade kontrollstrukturer
    \item fördröjd evaluering (''call-by-name'')
    \item stegade funktioner (''Curry-funktioner'')
    \item fångad variablelrymd (''closure'')
    \end{itemize}
\end{itemize}  
\end{SlideExtra}

\begin{SlideExtra}{Om veckans tema: Funktioner}
  \includegraphics[width=1.0\textwidth]{../img/coffee-grinder}
\end{SlideExtra}
\fi


\begin{Slide}{Funktion: deklaration och anrop}
\SlideOnly{\setlength{\leftmargini}{0pt}}

\code{def} funktionsnamn(parameterdeklarationer): returtyp = uttryck
\vspace{0.5em}

% \begin{Code}
% def namn(param1: Typ1, param2: Typ2): Returtyp = uttryck
% \end{Code}

\begin{itemize}\SlideFontSmall
  \item En funktion har ett \Emph{huvud} och efter \code{=} kommer dess \Emph{kropp}.
  \item En \Alert{namngiven} funktion \Emph{deklareras} med nyckelordet \code{def}
  \item En funktion kan ha \Emph{parametrar} som deklareras i huvudet. 
  \item \Alert{Kroppen} ska vara ett \Emph{uttryck} (ev. ett block med flera uttryck).
  \item \Emph{Parametrar} binds till \Emph{argument} vid \Alert{anrop}.
  \item Uttrycket i funktionens kropp \Emph{evalueras} vid \Alert{varje anrop}. 
  \item Värdet av uttrycket blir funktionens \Emph{returvärde}. 
\end{itemize}

\pause
Exempel:
\begin{Code}
def öka(a: Int, b: Int): Int = a + b
\end{Code}
\pause
\begin{REPLnonum}
scala> öka(42, 1)
val res0: Int = 43
\end{REPLnonum}

\end{Slide}


\begin{Slide}{Deklarera funktioner, överlagring}
\begin{itemize}
\item Överlagrade funktioner i samma namnrymd:
\begin{REPL}
scala> object matte:
         def öka(a: Int): Int = a + 1
         def öka(a: Int, b: Int): Int = a + b

scala> matte.öka(1)
val res0: Int = 2

scala> matte.öka(1, 2)
val res1: Int = 3

\end{REPL}
\item Båda funktionerna ovan kan finnas samtidigt! Trots att de har \Emph{samma namn} är de \Alert{olika funktioner}; kompilatorn kan skilja dem åt med hjälp av de \Alert{olika parameterlistorna}.

\item Detta kallas \Emph{överlagring} \Eng{overloading} av funktioner.
\item Överlagring ger \Alert{flexibilitet i användningen}; vi slipper hitta på nytt namn så som \code{öka2} vid 2 parametrar.
\end{itemize}
\end{Slide}



\begin{Slide}{Funktioner med defaultargument}\SlideFontSmall

\begin{itemize}
\item Vi kan ofta åstadkomma samma flexibilitet som vid överlagring, men med \Alert{en enda} funktion, om vi i stället använder \Emph{defaultargument}:
\begin{REPLnonum}
scala> def inc(a: Int, b: Int = 1) = a + b

scala> inc(42, 2)
val res0: Int = 44

scala> inc(42, 1)
val res1: Int = 43

scala> inc(42)
val res2: Int = 43

\end{REPLnonum}
\item Om ett argument utelämnas och parametern deklarerats med defaultargument så appliceras detta. Kompilatorn fyller alltså i argumentet åt oss, om det är \Alert{entydigt} vilken parameter som avses.
\end{itemize}
\end{Slide}


\begin{Slide}{Funktioner med namngivna argument}
\begin{itemize}\SlideFontTiny
\item Genom att använda \Emph{namngivna argument} behöver man inte hålla reda på ordningen på parametrarna, bara man känner till parameternamnen.
\item Namngivna argument går fint att \Alert{kombinera} med defaultargument.
\begin{REPLnonum}[basicstyle=\SlideFontSize{7}{9}\ttfamily\color{white}]
scala> def namn(
         förnamn: String,
         efternamn: String,
         förnamnFörst: Boolean = true,
         ledtext: String = "Namn:"
       ): String = 
         if förnamnFörst 
         then s"$ledtext $förnamn $efternamn"
         else s"$ledtext $efternamn, $förnamn"

scala> namn(ledtext = "Name:", efternamn = "Coder", förnamn = "Kim")
val res0: String = Name: Kim Coder
\end{REPLnonum}
\end{itemize}
\end{Slide}


\begin{Slide}{Enhetlig access}\SlideFontSmall
\begin{itemize}
\item Om en funktion \Emph{deklareras} \Alert{med} tom parameterlista \code{()} så \emph{ska} den \Emph{anropas} \Alert{med} tom parameterlista. \hfill (Undantag: Java-metoder)
\begin{REPLsmall}
scala> def tomParameterlista() = 42

scala> tomParameterlista()
val res1: Int = 42

scala> tomParameterlista                                                                                                                    
1 |tomParameterlista
  |^^^^^^^^^^^^^^^^^
  |method tomParameterlista must be called with () argument
\end{REPLsmall}

\item En parameterlös funktion deklarerad \Alert{utan} \code{()} ska anropas \Alert{utan} \code{()}. 
\begin{REPLsmall}
scala> def ingenParameterlista = 42
scala> ingenParameterlista()
1 |ingenParameterlista()
  |^^^^^^^^^^^^^^^^^^^
  |method ingenParameterlista does not take parameters
\end{REPLsmall}
\pause
\item Deklaration utan \code{()} möjliggör \Emph{enhetlig access}: implementationen kan ändras från \code{val} till \code{def} eller tvärtom, \Emph{utan} att \Alert{användandet} påverkas.
%\item Om parameterlista saknas får man alltså \Alert{inte} använda \code{()} vid anrop:

\end{itemize}
\end{Slide}

\Subsection{Hur ser det ut i minnet när funktioner anropas?}

\begin{Slide}{Anropsstacken och objektheapen}\SlideFontSmall
Minnet som innehåller ett programs data är uppdelat i två delar:
\begin{itemize}
\item \Emph{Anropsstacken}: 
\begin{itemize}\SlideFontSmall
\item På anropsstacken läggs en \Emph{aktiveringspost} \Eng{stack frame\footnote{\href{https://en.wikipedia.org/wiki/Call_stack}{en.wikipedia.org/wiki/Call\_stack}}, activation record} för varje funktionsanrop med plats för \Alert{parametrar} och \Alert{lokala variabler}.
\item Aktiveringsposten \Alert{raderas} när \Emph{returvärdet} har levererats.
\item Stacken \Alert{växer} vid \Emph{nästlade funktionsanrop}, då en funktion i sin tur anropar en annan funktion.
\end{itemize}
\item \Emph{Objektheapen}: I objektheapen\footnote{\href{https://en.wikipedia.org/wiki/Memory_management}{en.wikipedia.org/wiki/Memory\_management}}$^{,}$\footnote{Ej att förväxlas med datastrukturen heap  \href{https://sv.wikipedia.org/wiki/Heap}{sv.wikipedia.org/wiki/Heap}} sparas alla objekt (data) som allokeras under körning. Heapen städas då och då av \Emph{skräpsamlaren} \Eng{garbage collector}, och minne som inte används längre frigörs. \\\vspace{0.5em}
%\href{http://stackoverflow.com/questions/1565388/increase-heap-size-in-java}{stackoverflow.com/questions/1565388/increase-heap-size-in-java}
% \href{https://stackoverflow.com/questions/1441373/increase-jvm-heap-size-for-scala}{stackoverflow.com/questions/1441373/increase-jvm-heap-size-for-scala}
% \begin{REPLnonum}
% scala -J-Xmx16g -J-XX:-UseGCOverheadLimit Main
% \end{REPLnonum}
\end{itemize}
\end{Slide}

% \begin{Slide}{Aktiveringspost}\SlideFontSmall
% Nästlade anrop ger växande anropsstack.
% \begin{REPL}
% scala> def f(x: Int, y: Int): Int = { val z = x + y; println(z); z}
% scala> def g(a: Int, b: Int): Int = { val c = a + b; println(c); f(c, 2 * c) }
% scala> def h(i: Int): Int = { val n = 5; g(i, i * n) }
% scala> h(2)
% \end{REPL}
%
% \pause
% \Alert{Stacken}
%
% \begin{tabular}{|r | l | l |} \hline
%
% variabel & värde & Aktiveringspost för anrop av... \\ \hline \hline
% \pause
%  i & 2 & h \\
%  n & 5 & \\ \hline
%  \pause
%  a & 2 & g \\
%  b & 10 &  \\
%  c & 12  &  \\  \hline
%  \pause
%  x & 12  & f \\
%  y & 24 &  \\
%  z & 36 & \\ \hline
% \end{tabular}
% \end{Slide}

\begin{Slide}{Anropsstacken och aktiveringsposter}\SlideFontSmall
Nästlade anrop ger växande anropsstack. Vid varje anrop allokeras en s.k. \Emph{aktiveringspost} \Eng{activation record} med plats i minnet för parametrar, lokala variabler och ev. returvärde. När funktionen är klar så raderas aktiveringsposten och stacken krymper.
\begin{REPLsmall}
scala> def h(x: Int, y: Int): Unit = { val z = x + y; println(z) }
       def g(a: Int, b: Int): Unit = { val x = 1; h(x + 1, a + b) }
       def f(): Unit = { val n = 5; g(n, 2 * n) }

scala> f()

\end{REPLsmall}

\pause
\Alert{Stacken} med 3 aktiveringsposter då f anropar g som anropar h:

\begin{tabular}{|r | l | l |} \hline

variabel & värde & Anrop av... \\ \hline \hline
\pause
 n & 5 & f \\ \hline
 \pause
 a & 5 & g \\
 b & 10 &  \\
 x & 1  &  \\  \hline
 \pause
 x & 2  & h \\
 y & 15 &  \\
 z & 17 & \\ \hline
\end{tabular}
\end{Slide}

% \begin{Slide}{Anropsstacken i Kojo Desktop}
% Tryck på orange playknapp i Kojo och se anropsstacken.\vspace{0.5em}

% \includegraphics[width=1.0\textwidth]{../img/kojo-trace.png}  
% \end{Slide}

% \begin{Slide}{Anropsstacken i VS Code}
% Lägg till en brytpunkt på rad 4 nedan och klicka på debug över \code{@main} i VS Code och se anropsstacken.\vspace{0.5em}

% \includegraphics[width=0.85\textwidth]{../img/vscode-trace.png}  
% \end{Slide}


\begin{Slide}{Vad är en stack trace?}\SlideFontSmall
När du letar buggar vid körtidsfel har du nytta av att \Alert{noga studera} \Emph{utskriften av anropsstacken} \Eng{stack trace}:
% \begin{CodeSmall}
% // Program i filen BMI.scala
% object BMI: 
%   def main(args: Array[String]): Unit = 
%     println(bmi(args(0).toInt, args(1).toInt))

%   def bmi(heightCm: Int, weightKg: Int) = 
%     safeDiv(weightKg, heightCm * heightCm) 

%   def safeDiv(numerator: Int, denominator: Int): (Int, String) = 
%     if (denominator == 0) (numerator / denominator, "")  // ser du buggen?
%     else (0, "division by zero") 
% \end{CodeSmall}
  
\begin{Code}[numbers=left]
// Program i filen bmi.scala

@main 
def bmi(heightCm: Int, weightKg: Int) = 
  safeDiv(weightKg, heightCm * heightCm) 

def safeDiv(numerator: Int, denominator: Int): (Int, String) = 
  if denominator == 0 then (numerator / denominator, "")  // ser du buggen?
  else (0, "division by zero")

\end{Code}
\begin{REPL}
> scala run bmi.scala -- 0 42
Exception in thread "main" java.lang.ArithmeticException: / by zero
        // HÄR KOMMER STACK TRACE pga körtidsfel - se nästa bild
\end{REPL}
\end{Slide}

\begin{Slide}{Hur läsa en stack trace?}
\begin{REPL}
Exception in thread "main" java.lang.ArithmeticException: / by zero
        at bmi$package$.safeDiv(bmi.scala:8)
        at bmi$package$.bmi(bmi.scala:5)
        at bmi.main(bmi.scala:3)

\end{REPL}
\begin{itemize}\SlideFontSmall
  \item En \Emph{stack trace} skrivs ut efter en krasch p.g.a. körtidsfel.
  \item Körtidsfel känns igen med ordet \Alert{Exception}.
  \item Först kommer en beskrivning av felet som orsakat kraschen, här: \\\code{java.lang.ArithmeticException: / by zero} 
  \item Därefter visas anropsstacken.
  \item För varje funktionsanrop anges: \Emph{\texttt{klass.metod(kodfil:radnummer)}}
  \item Main-funktioner läggs i ett singelobjekt i ett speciellt paket
  \item Singelobjekt i Scala kodas som en Java-klass med dollar-tecken efter namnet, eftersom det inte finns singelobjekt i JVM.
  %\item Efter anropet av din \code{main}-procedur ligger JVM-interna anrop (\code{java.base} etc.) som du inte behöver bry dig om. %% syns inte längre i Scala 3 stack trace
\end{itemize}
\end{Slide}
  

\begin{Slide}{Lokala funktioner}\SlideFontSmall
Med lokala funktioner kan delproblem lösas med nästlade abstraktioner.

\begin{CodeSmall}
def gissaTalet(max: Int, min: Int = 1): Unit = 
  def gissat = io.StdIn.readLine(s"Gissa talet mellan $min och $max: ").toInt

  val hemlis = (math.random() * (max - min) + min).toInt

  def skrivLedtrådOmEjRätt(gissning: Int): Unit =
    if gissning > hemlis then println(s"$gissning är för stort :(")
    else if gissning < hemlis then println(s"$gissning är för litet :(")

  def inteRätt(gissning: Int): Boolean = 
    skrivLedtrådOmEjRätt(gissning)
    gissning != hemlis
  

  def loop: Int = { var i = 1; while inteRätt(gissat) do i += 1; i }

  println(s"Du hittade talet $hemlis på $loop gissningar :)")
\end{CodeSmall}

Lokala, nästlade funktionsdeklarationer är tyvärr inte tillåtna i många andra språk, t.ex. Java.\footnote{\href{http://stackoverflow.com/questions/5388584/does-java-support-inner-local-sub-methods}{\SlideFontSize{8}{9}stackoverflow.com/questions/5388584/does-java-support-inner-local-sub-methods}}

\end{Slide}

\Subsection{En funktion är ett värde}

\begin{Slide}{Funktioner är äkta värden i Scala}\SlideFontSmall
\begin{itemize}
\item En funktion är ett \Alert{äkta värde}.
\item Vi kan till exempel tilldela en variabel ett \Emph{funktionsvärde}.
\pause
\item Med hjälp \Alert{enbart} \Emph{funktionsnamnet} får vi funktionen som har ett \Alert{värde} (inga argument har applicerats än):
\begin{REPLnonum}
scala> def add(a: Int, b: Int) = a + b

scala> val f = add  
val f: (Int, Int) => Int = Lambda7210/0x0000000841e4e040@1ce2db23

scala> f(21, 21)
val res0: Int = 42
\end{REPLnonum}
\item Ett funktionsvärde har en \Alert{typ} precis som alla värden: \\
\code{f: (Int, Int) => Int}
\pause
\item Ett funktionsvärde har till skillnad från en funktionsdeklaration inget namn (variabeln \code{f} har ett namn, men inte själva funktionen). Den kallas därför en \Emph{anonym} funktion eller \Alert{lambda} (mer om detta snart).
\end{itemize}
\end{Slide}

\begin{Slide}{Funktionsvärden kan vara argument}\SlideFontSmall
Funktioner kan ha funktioner som parametrar:
\begin{REPL}
scala> def tvåGånger(x: Int, f: Int => Int) = f(f(x))

scala> def öka(x: Int) = x + 1

scala> def minska(x: Int) = x - 1

scala> tvåGånger(42, öka)
val res1: Int = 44

scala> tvåGånger(42, minska)
val res1: Int = 40
\end{REPL}
En funktion som har funktionsvärden som indata (eller utdata) kallas en\\ \Emph{högre ordningens funktion}  \Eng{higher-order function}.

\end{Slide}



\begin{Slide}{Applicera funktioner på element i samlingar med \texttt{map}}\SlideFontSmall
\begin{Code}
def öka(x: Int) = x + 1

def minska(x: Int) = x - 1

val xs = Vector(1, 2, 3)
\end{Code}
\pause
Metoden \Emph{\texttt{map}} fungerar på alla Scala-samlingar och tar \Emph{en funktion som argument} och applicerar denna funktion på alla element och \Alert{skapar en ny samling} med resultaten:
\begin{REPL}
scala> xs.map(öka)
val res0: ???   // vad blir resultatet?

scala> xs.map(minska)
val res1: ???   // vad blir resultatet?
\end{REPL}
\end{Slide}


\begin{Slide}{Applicera funktioner på element i samlingar med \texttt{map}}\SlideFontSmall
\begin{Code}
def öka(x: Int) = x + 1

def minska(x: Int) = x - 1

val xs = Vector(1, 2, 3)
\end{Code}
Metoden \Emph{\texttt{map}} fungerar på alla Scala-samlingar och tar \Emph{en funktion som argument} och applicerar denna funktion på alla element och \Alert{skapar en ny samling} med resultaten:
\begin{REPL}
scala> xs.map(öka)
val res0: scala.collection.immutable.Vector[Int] = Vector(2, 3, 4)

scala> xs.map(minska)
val res1: scala.collection.immutable.Vector[Int] = Vector(0, 1, 2)
\end{REPL}
Metoden \Emph{\texttt{map}} är en smidig och ofta använd \Alert{högre ordningens funktion}.
\end{Slide}

\Subsection{Äkta funktioner}

\begin{Slide}{Äkta funktioner}
\begin{itemize}\SlideFontSmall
\item En \Emph{äkta} \Eng{pure} funktion är en funktion som ger ett resultat som \Alert{enbart} beror av dess argument. Alltså som funktioner i matematiken.
\item En äkta (matematisk) funktion är \Emph{referentiellt transparent} \Eng{referentially transparent}. Det innebär att \Alert{varje anrop} \Emph{kan bytas ut} mot \Alert{värdet av} \Emph{funktionskroppen} där parametrarna ersatts med motsvarande argument före evaluering.
\item En äkta funktion har \Alert{inga sidoeffekter}, t.ex. utskrift, skriva/läsa filer,  eller uppdateringar av variabler \Alert{synliga utanför} funktionen.
\item Exempel:
\begin{Code}
def add(x: Int, y: Int): Int = x + y              // äkta funktion
def rnd(n: Int): Int = (math.random() * n).toInt  // oäkta funktion
\end{Code} 
\begin{itemize}\SlideFontTiny

\item Uttrycket \code{add(41, 1)} kan ersättas med 41 +1 som i sin tur kan ersättas med 42 utan att det påverkar resultatet. Resultatet av \code{add(41, 1)} blir \Emph{samma varje gång} funktionen appliceras med dessa argument
\item Uttrycket \code{rnd(42)} kan \Alert{inte} bytas ut mot ett specifikt värde. \\Alltså: \emph{ej referentiellt transparent}.
\end{itemize}  
\end{itemize}  
\end{Slide}

\begin{Slide}{Exempel på oäkta funktioner: slumptal}

  \begin{itemize}
    \item Funktioner vars värden på något sätt beror av slumpen är \Alert{inte} äkta funktioner.
    \item Även om samma argument ges vid upprepad applicering, så kan ju resultatet bli olika.
    \item Studera dokumentationen för \code{scala.util.Random} här:\\ \href{https://www.scala-lang.org/api/current/scala/util/Random.html}{\SlideFontSmall https://www.scala-lang.org/api/current/scala/util/Random.html}
    \item Du har nytta av funktionen \code{Random.nextInt} och slumptalsfrö \Eng{random seed} i veckans uppgifter.
  \end{itemize}

\end{Slide}

\begin{Slide}{Slumptalsfrö: få samma slumptal varje gång}\SlideFontTiny
\begin{itemize}
\item Om man använder slumptal kan det vara svårt att leta buggar, eftersom det blir \Alert{olika varje gång} man kör programmet och buggen kanske bara uppstår ibland.

\item Med klassen \code{scala.util.Random} kan man skapa \Emph{pseudo}-slumptalssekvenser.
\pause
\item Om man ger ett s.k. \Emph{frö} \Eng{seed}, av heltalstyp, som argument till konstruktorn när man skapar en instans av klassen \code{scala.util.Random}, får man samma ''slumpmässiga'' sekvens \Alert{varje gång} man kör programmet.

\begin{Code}
  val seed = 42
  val rnd = util.Random(seed) // skapa ny slumpgenerator med frö 42
  val r = rnd.nextInt(6)      // något av heltalen 0, 1, 2, 3, 4, 5
\end{Code}
\pause
\item Om man \Alert{inte} ger ett \Emph{frö} så sätts fröet till ''\emph{a value very likely to be distinct from any other invocation of this constructor}''. Då vet vi inte vilket fröet blir och det blir olika varje gång man kör programmet.
\begin{Code}
  val rnd = util.Random() // OLIKA frö vid varje programkörning
  val r = rnd.nextInt(6) 
\end{Code}
\end{itemize}
\end{Slide}

%\begin{Slide}{Syresättning av hjärnan vid sövande föreläsning}
%Prova nedan kod som finns här:\\
%%\href{https://github.com/lunduniversity/introprog/blob/master/compendium/examples/workspace/w05-seqalg/src/NanananananananaNanananananananaBatman.scala}{\SlideFontTiny github.com/lunduniversity/introprog/.../NanananananananaNanananananananaBatman.scala} \\
%
%
%
%\vspace{0.65em}\scalainputlisting[numbers=left,numberstyle=,basicstyle=\fontsize{6.5}{8}\ttfamily\selectfont]{../compendium/examples/workspace/w05-seqalg/src/FixSleepyBrain.scala}
%
%\pause
%Medan du lyssnar till: \href{https://www.youtube.com/watch?v=zUwEIt9ez7M}{\SlideFontSmall www.youtube.com/watch?v=zUwEIt9ez7M}\\
%Eller: \href{https://www.youtube.com/watch?v=rvXxlXg_V-k}{\SlideFontSmall www.youtube.com/watch?v=rvXxlXg\_V-k}
%\end{Slide}

\Subsection{Anonyma funktioner}


\begin{Slide}{Anonyma funktioner}
\begin{itemize}\SlideFontSmall
\item  Man behöver inte ge funktioner namn. De kan i stället skapas med hjälp av \Emph{funktionslitteraler}.\footnote{Även kallat ''lambda-värde'' eller bara ''lambda'' efter den s.k. lambdakalkylen. \href{https://en.wikipedia.org/wiki/Anonymous_function}{en.wikipedia.org/wiki/Anonymous\_function}}

\item En funktionslitteral har ...
\begin{enumerate}
\item en parameterlista (utan funktionsnamn, utan returtyp),
\item sedan den reserverade teckenkombinationen \code{=>}
\item och sedan ett uttryck (eller ett block).
\end{enumerate}
\pause
\item Exempel:
\begin{Code}[basicstyle=\ttfamily\SlideFontSize{9}{11}]
(x: Int, y: Int) => x + y
\end{Code}
Vilken typ har denna funktionslitteral? \pause \hfill\code{(Int, Int) => Int}
\pause
\item Om kompilatorn kan gissa typerna från sammanhanget så behöver typerna inte anges i själva  funktionslitteralen:
\begin{Code}[basicstyle=\ttfamily\SlideFontSize{9}{11}]
val f: (Int, Int) => Int = (x, y) => x + y
\end{Code}
\end{itemize}
\end{Slide}


\begin{Slide}{Applicera anonyma funktioner på element i samlingar}\SlideFontSmall
Anonym funktion skapad med funktionslitteral direkt i anropet:
\begin{REPL}
scala> val xs = Vector(1, 2, 3)

scala> xs.map((x: Int) => x + 1)
res0: scala.collection.immutable.Vector[Int] = Vector(2, 3, 4)
\end{REPL}
\pause
Eftersom kompilatorn här kan härleda typen \code{Int} så behövs den inte:
\begin{REPL}
scala> xs.map(x => x + 1)
res1: scala.collection.immutable.Vector[Int] = Vector(2, 3, 4)
\end{REPL}
\pause
Om man bara använder parametern en enda gång i funktionen så kan man byta ut parameternamnet mot ett understreck.
\begin{REPL}
scala> xs.map(_ + 1)
res2: scala.collection.immutable.Vector[Int] = Vector(2, 3, 4)
\end{REPL}
\end{Slide}



\begin{Slide}{Platshållarsyntax för anonyma funktioner}\SlideFontSmall
Understreck i funktionslitteraler kallas \Emph{platshållare} \Eng{placeholder} och medger ett förkortat skrivsätt \Alert{om} den parameter som understrecket representerar används \Alert{endast en gång}.
\begin{Code}[basicstyle=\ttfamily\fontsize{10}{12}\selectfont]
_ + 1
\end{Code}
Ovan expanderas av kompilatorn till följande funktionslitteral \\(där namnet på parametern är godtyckligt):
\begin{Code}[basicstyle=\ttfamily\fontsize{10}{12}\selectfont]
x => x + 1
\end{Code}
\pause
Det kan förekomma flera understreck; det första avser första parametern, det andra avser andra parametern etc.
\begin{Code}[basicstyle=\ttfamily\fontsize{10}{12}\selectfont]
_ + _
\end{Code}
\pause
... expanderas till:
\begin{Code}[basicstyle=\ttfamily\fontsize{10}{12}\selectfont]
(x, y) => x + y
\end{Code}
\end{Slide}


\begin{Slide}{Exempel på platshållarsyntax med \texttt{reduceLeft}}\SlideFontSmall
Metoden \code{reduceLeft} applicerar en funktion på de två första elementen i en sekvens och tar sedan resultatet som första argument och nästa element som andra argument och upprepar detta genom hela samlingen.
\begin{REPL}
scala> def summa(x: Int, y: Int) = x + y

scala> val xs = Vector(1, 2, 3, 4, 5)

scala> xs.reduceLeft(summa)
res20: Int = 15

scala> xs.reduceLeft((x, y) => x + y)
res21: Int = 15

scala> xs.reduceLeft(_ + _)
res22: Int = 15

scala> xs.reduceLeft(_ * _)
res23: Int = 120
\end{REPL}
\end{Slide}


\begin{Slide}{Predikat, med och utan namn}
\begin{itemize}\SlideFontTiny
\item En funktion som har \code{Boolean} som returtyp kallas för ett \Emph{predikat}. 
\item Exempel:
\begin{Code}
def isTooLong(name: String): Boolean = name.length > 10

def isTall(heightInMeters: Double, limit: Double = 1.78): Boolean = 
  heightInMeters > limit
\end{Code}
\item Predikat ges ofta ett namn som börjar på \code{is} eller \code{has} så att man lätt kan se att det är ett predikat när man läser kod som anropar funktionen.
\item Många av samlingsmetoderna i Scalas standardbibliotek tar predikat som funktionsargument. Exempel med predikat som anonym funktion: 
\begin{REPLnonum}
scala> val parts = Vector(3, 1, 0, 5).partition(_ > 1)
val parts: (Vector[Int], Vector[Int]) = 
  (Vector(3, 5),Vector(1, 0))
\end{REPLnonum} 
\item Studera snabbreferensen och försök hitta samlingsmetoder som tar predikat som funktionsargument. \url{https://fileadmin.cs.lth.se/pgk/quickref.pdf} \\I anropsexempel med predikat-argument används bokstaven \code{p}.
\end{itemize}  
\end{Slide}

\begin{Slide}{Funktionsvärde vid tom parameterlista: använd ''thunk''}\SlideFontSmall
\begin{itemize}\SlideFontSmall

\item Om du vill ha funktionen som ett värde så skriv bara namnet och inte parameterlistan (samma exempel som tidigare):
\begin{REPLsmall}
scala> def add(a: Int, b: Int) = a + b

scala> val f = add     // inget anrop sker 
val f: (Int, Int) => Int = Lambda7210/0x0000000841e4e040@1ce2db23
\end{REPLsmall}

\item Vid \Alert{tom parameterlista} behövs anonym funktion som \Emph{fördröjer anrop}: 
\begin{REPLsmall}
scala> def a() = 42
def a(): Int

scala> val b = a
1 |val b = a
  |        ^
  |        method a must be called with () argument

scala> val b = () => a()   // anonym funktion, fördröjd evaluering
val b: () => Int = Lambda7214/0x0000000841e50440@565d794
\end{REPLsmall}
\item Notera typen: \code{() => Int} ~~Ett sådant funktionsvärde kallas \Alert{thunk}\\ \url{https://en.wikipedia.org/wiki/Thunk} 
\end{itemize}
\end{Slide}



\Subsection{Skapa din egen kontrollstruktur}  

\begin{Slide}{Hur fungerar egentligen \code{upprepa} i Kojo?}
\begin{Code}[basicstyle=\ttfamily\SlideFontSize{14}{16}]
upprepa(10) {
  println("hej")
}
\end{Code}

\pause
Vi ska nu se hur vi, genom att kombinera ett antal koncept, kan skapa egna kontrollstrukturer likt upprepa ovan:
\begin{itemize}
\item klammerparentes vid ensam paramenter
\item multipla parameterlistor
\item namnanrop (fördröjd evaluering)
\end{itemize}
\end{Slide}



\begin{Slide}{Multipla parameterlistor}
Vi har tidigare sett att man kan ha mer än en parameter:
\begin{REPLnonum}
scala> def add(a: Int, b: Int) = a + b

scala> add(21, 21)
res0: Int = 42
\end{REPLnonum}
Man kan även ha \Alert{mer än en} \Emph{parameterlista}:
\begin{REPLnonum}
scala> def add(a: Int)(b: Int) = a + b

scala> add(21)(21)
res1: Int = 42
\end{REPLnonum}
\Eng{multiple parameter lists}

\href{http://docs.scala-lang.org/style/declarations.html#multiple-parameter-lists}{\SlideFontTiny docs.scala-lang.org/style/declarations.html\#multiple-parameter-lists}
\end{Slide}



\begin{Slide}{Värdeanrop och namnanrop}\SlideFontSmall
Det vi sett hittills är \Emph{värdeanrop}: argumentet evalueras \Alert{först} innan dess \Alert{värde} \emph{sedan} appliceras:
\begin{REPL}
scala> def byValue(n: Int): Unit = for i <- 1 to n do print(" " + n)

scala> byValue(21 + 21)
 42 42 42 42 42 42 42 42 42 42 42 42 42 42 42 42 42 42 42 42 42 42 42 42 42 42 42 42 42 42 42 42 42 42 42 42 42 42 42 42 42 42

scala> byValue({print(" hej"); 21 + 21})
 hej 42 42 42 42 42 42 42 42 42 42 42 42 42 42 42 42 42 42 42 42 42 42 42 42 42 42 42 42 42 42 42 42 42 42 42 42 42 42 42 42 42 42
\end{REPL}
\pause
Men man kan med \code{=>} före parametertypen åstadkomma \Emph{namnanrop}: argumentet \Alert{''klistras in''} i stället för \Alert{namnet} och evalueras \Alert{varje gång} (kallas även \Emph{fördröjd evaluering}):
\begin{REPL}
scala> def byName(n: => Int): Unit = for i <- 1 to n do print(" " + n)

scala> byName({print(" hej"); 21 + 21})
 hej hej 42 hej 42 hej 42 hej 42 hej 42 hej 42 hej 42 hej 42 hej 42 hej 42 hej 42 hej 42 hej 42 hej 42 hej 42 hej 42 hej 42 hej 42 hej 42 hej 42 hej 42 hej 42 hej 42 hej 42 hej 42 hej 42 hej 42 hej 42 hej 42 hej 42 hej 42 hej 42 hej 42 hej 42 hej 42 hej 42 hej 42 hej 42 hej 42 hej 42 hej 42 hej 42
\end{REPL}
\Alert{Kluring}: Varför skrivs ''hej'' ut en extra gång i början? \pause ledtråd: \texttt{1 to \Alert{n}}
%evalueringen av n i 1 to n ger ett extra hej
\end{Slide}

\begin{Slide}{Klammerparenteser vid ensam parameter}
Så här har vi sett nyss att man man göra:
\begin{REPL}
scala> def twice(action: => Unit): Unit = { action; action }

scala> twice( { print("hej"); print("san ") } )
hejsan hejsan
\end{REPL}

Det ser rätt klyddigt ut med \code+({+  och \code+})+ eller vad tycker du? \pause Men...
För alla funktioner \code{f} gäller att: \\ det är helt ok att byta ut vanliga parenteser: \hfill\code{f(uttryck)} \\ mot krullparenteser: \hfill\code|f{uttryck}| \\ \Alert{om} parameterlistan har \Alert{exakt en} parameter.

\vspace{0.5em}Man kan alltså skippa yttre parentesparet för \Alert{bättre läsbarhet}:
\begin{REPLnonum}
scala> twice { print("hej"); print("san ") }
\end{REPLnonum}
\end{Slide}



\begin{Slide}{Skapa din egen kontrollstruktur}\SlideFontSmall
\begin{itemize}
\item Genom att \Alert{kombinera} \Emph{multipla parameterlistor} med \Emph{namnanrop} med \Emph{klammerparentes vid ensam parameter} kan vi skapa vår egen kontrollstruktur: \code{upprepa} \pause
\begin{Code}
upprepa(42){
  if math.random() < 0.5 then print(" gurka")
  else print(" tomat")
}
\end{Code}
Hur då?
\pause
 Till exempel så här:
\begin{Code}
def upprepa(n: Int)(block: => Unit) = for i <- 0 until n do block
\end{Code}

\pause

\begin{REPLnonum}
gurka gurka gurka tomat tomat gurka gurka gurka gurka tomat tomat tomat tomat tomat
\end{REPLnonum}
\end{itemize}
\end{Slide}


\begin{Slide}{Kolon vid ensam parameter}\SlideFontSmall
Du kan i Scala 3 i stället för klammerparentes vid ensam parameter använda kolon för att få färre ''krullisar'' \Eng{fewer braces}.
\begin{Code}
  upprepa(42):
    if math.random() < 0.5 
    then print(" gurka")
    else print(" tomat")
\end{Code}
Denna förenklade syntax föregicks av långa diskussioner innan den till slut accepterades.\footnote{
Den nyfikne kan läsa förslaget före omröstning här: \\ \url{https://docs.scala-lang.org/sips/fewer-braces.html}}
\end{Slide}

\begin{Slide}{Stegade funktioner, ''Curry-funktioner''}
Om en funktion har multipla parameterlistor kan man skapa \Emph{stegade funktioner}, även kallat \Emph{partiellt applicerade} funktioner \Eng{partially applied functions} eller \Emph{''Curry''-funktioner}.
\begin{REPLnonum}
scala> def add(x: Int)(y: Int) = x + y

scala> val öka = add(1)
val öka: Int => Int = Lambda7339/0x0000000841eb7040@19c8add7

scala> Vector(1,2,3).map(öka)
val res0: Vector[Int] = Vector(2, 3, 4)

scala> Vector(1,2,3).map(add(2))
val res1: Vector[Int] = Vector(3, 4, 5)
\end{REPLnonum}
\end{Slide}


\begin{Slide}{Funktion med fångad variabelrymd: \textit{closure}}
\begin{Code}
def f(x: Int): Int => Int = 
  val a = 42 + x
  def g(y: Int): Int = y + a
  g
\end{Code}
Funktionen \code{g} \Alert{fångar} den lokala variabeln \code{a} i ett funktionsobjekt.
\pause
\begin{REPLnonum}
scala> val funkis = f(1)
val funkis: Int => Int = Lambda7356/0x0000000841ed2840@1bda26bc

scala> funkis(2)
val res0: Int = 45
\end{REPLnonum}
\pause
Ett funktionsobjekt med ''fångade'' variabler kallas \Alert{closure}. \\
(Mer om funktioner som objekt senare.)
\end{Slide}

\ifkompendium\else
\begin{SlideExtra}{Översikt av begrepp vi gått igenom hittills}
\begin{enumerate}
\item överlagring
\item utelämna tom parameterlista (enhetlig access)
\item defaultargument
\item namngivna argument
\item lokala funktioner
\item funktioner som äkta värden
\item anonyma funktioner
\item klammerparentes vid ensam paramenter
\item multipla parameterlistor
\item namnanrop (fördröjd evaluering)
\item egendefinierade kontrollstrukturer
\item stegade funktioner (''Curry-funktioner'')
\item fångad variablelrymd i funktionsobjekt (''closure'')
\end{enumerate}
\end{SlideExtra}
\fi



\Subsection{Kort om rekursion}

\begin{Slide}{Rekursiva funktioner}
\begin{itemize}
\item Funktioner som \Alert{anropar sig själv} kallas \Emph{rekursiva}.


\begin{REPLnonum}
scala> def fakultet(n: Int): Int =
         if n < 2 then 1 else n * fakultet(n - 1)

scala> fakultet(5)
val res0: Int = 120
\end{REPLnonum}

\item För varje nytt anrop läggs en ny aktiveringspost på stacken.

\item I aktiveringsposten sparas varje returvärde som gör att \code{5 * (4 * (3 * (2 * 1)))} kan beräknas.

\item Rekrusionen avbryts när man når \Emph{basfallet}, här \code{n < 2}

\item En rekursiv funktion \Alert{måste} ha en returtyp.

\end{itemize}

\end{Slide}

\begin{Slide}{Loopa med rekursion}
\begin{Code}
def gissaTalet(max: Int, min: Int = 1): Unit =
  def gissat = 
    io.StdIn.readLine(s"Gissa talet mellan [$min, $max]: ").toInt

  val hemlis = (math.random() * (max - min) + min).toInt

  def skrivLedtrådOmEjRätt(gissning: Int): Unit =
    if gissning > hemlis then println(s"$gissning är för stort :(")
    else if (gissning < hemlis) println(s"$gissning är för litet :(")

  def ärRätt(gissning: Int): Boolean = 
    skrivLedtrådOmEjRätt(gissning)
    gissning == hemlis

  def loop(n: Int = 1): Int = if ärRätt(gissat) then n else loop(n + 1)

  println(s"Du hittade talet $hemlis på ${loop()} gissningar :)")
\end{Code}
\end{Slide}


\begin{Slide}{Rekursiva datastrukturer}
\begin{itemize}
\item Datastrukturena Lista och Träd är exempel på datastrukturer som passar bra ihop med rekursion.
\item Båda dessa datastrukturer kan beskrivas rekursivt:
\begin{itemize}
\item En lista består av ett huvud och en lista, som i sin tur består av ett huvud och en lista, som i sin tur...
\item Ett träd består av grenar till träd som i sin tur består av grenar till träd som i sin tur, ...
\end{itemize}
\item Dessa datastrukturer bearbetas med fördel med rekursiva algoritmer.
\item I denna kursen ingår rekursion endast ''för kännedom'': \\ du ska veta vad det är och kunna skapa en enkel rekursiv funktion, t.ex. fakultets-beräkning. Du kommer jobba mer med rekursion och rekursiva datastrukturer i fortsättningskursen.
\end{itemize}
\end{Slide}

\Subsection{Automatisk omkompilering}

\begin{Slide}{Kompilera om det som ändrats vid varje sparning}
\begin{itemize}
  \item Den kreativa programmeringsprocessen innehåller många korta cykler av koda, ändra, testa.
  \item Det blir \Alert{många omkompileringar} och då vill man gärna slippa skriva samma kommando om och om igen. %En lösning är att skapa ett skript, t.ex. i språket \Emph{bash}, som kör kompileringen.
  \item Vid \Emph{varje liten ändring} vill man \Alert{kompilera om} det som ändrats och se om det fortfarande kompilerar utan fel. 
  \item Då kan du använda:\\\code{scala compile . --watch}\\Ändringar bevakas och kompileras om direkt.
  %\item  Du kan också använda ett s.k. \Emph{byggverktyg},\\t.ex. Scala Build Tool (\code{sbt}), se Appendix F.

\end{itemize}
\end{Slide}

% \begin{Slide}{Bash-skript för kompilering}\SlideFontSmall
% \begin{itemize}
%   \item Det gamla skriptspråket \Emph{bash} funkar i Linux och MacOS.
%   \item Bash är smidigt för enkla program som använder terminalkommando, men syntaxen är knepig och det finns många fallgropar.
%   \item I ett bash-skript kan du t.ex. kompilera och köra ett program. Exempel i filen \code{build.sh} nedan:
% \begin{Code}
% scalac mitt-program.scala && scala MinMain
% \end{Code}
% Med pil-upp kan du enkelt kompilera om efter varje ändring:
% \begin{REPLnonum}
% > sh build.sh
% \end{REPLnonum}
%   \item Det går att få \Emph{bash} och ubuntu-terminalen att funka i Windows 10 med WSL (Windows Linux Subsystem) där du kan välja Ubuntu 18.04 LTS: \\
%   {\SlideFontTiny\url{https://docs.microsoft.com/en-us/windows/wsl/install-win10}}
% \end{itemize}
% {\noindent   Det finns dock stora begränsningar med WSL och om du vill ha Linux ''på riktigt'' rekommenderas att du installera Ubuntu med dual-boot: \SlideFontTiny\url{https://linoxide.com/distros/install-ubuntu-18-04-dual-boot-windows-10/}}
% \end{Slide}

% \begin{Slide}{Scala Build Tool: \texttt{sbt}}
% \begin{itemize}
%   \item Ett \Emph{byggverktyg}, exempelvis \code{sbt}, kan användas för att kompilera, testköra, ladda ner, paketera och distribuera kodbibliotek och applikationer.
%   \item Du behöver en fil som heter \code{build.sbt} som innehåller:\\\code{scalaVersion := "3.2.2"}
%   \item Det är enkelt att använda \code{sbt} för att kompilera om ditt program varje gång du gör \code{Ctrl+S}:
% \begin{REPLnonum}
% > sbt
% sbt> ~compile   // med ~ görs omkompilering vid ändring
% \end{REPLnonum}
% Tecknet \code{~} kallas \emph{tilde} och skrivs med högra Alt-tangenten nere och två tryck på tangenten bredvid Enter.

%   \item Läs mer om \code{sbt} i Appendix F och här: \\\url{https://www.scala-sbt.org}

% \end{itemize}

% \end{Slide}


\ifkompendium\else
\Subsection{Veckans uppgifter}

\begin{SlideExtra}{Mål med övning \ExeWeekTHREE}
\begin{itemize}\SlideFontSmall
  \input{../compendium/modules/w03-functions-exercise-goals.tex}
\end{itemize}
\end{SlideExtra}

\begin{SlideExtra}{Mål med laboration \LabWeekTHREE}
\begin{itemize}
  \input{../compendium/modules/w03-functions-lab-goals.tex}
\end{itemize}
Ni ska spela \Emph{varandras} textspel i din \Alert{samarbetsgrupp}.\\
Läs labbinstruktioner:\url{http://fileadmin.cs.lth.se/pgk/compendium.pdf/}
\end{SlideExtra}


\begin{SlideExtra}{Tips till ditt textspel.}
\begin{CodeSmall}
"Yes".toLowerCase.startsWith("y")    // true
"hejsan".contains("ejsa")            // true
"42".toInt                           // 42 
"?".toInt                            // ger krasch (undantaget NumberFormatException)
"?".toIntOption.getOrElse(42)        // 42 (toIntOption kan inte krascha)
// ett annat sätt att förhindra krasch med try ... catch:
val x = try { "?".toInt } catch { case e: Exception => 42 }

val i = 42
s"Livets mening är $i!" // dollar $ före namn vid stränginterpolering med s""
s"Livets mening är inte ${i-1}!"  // klamrar ${} vid evaluering av uttryck

"""|en sträng som spänner över
   |flera rader där marginalen fram till vertikalstreck
   |är bortplockad med stripMargin (kan kombineras med s-interpolatorn)
""".stripMargin

math.random() < 0.8                  // true i 80% av fallen
scala.util.Random.nextInt(42)      // ger slumptal mellan 0 och 41
scala.io.StdIn.readLine("prompt>") // ger sträng som användaren skriver

Thread.sleep(1000)    // sova i 1 sekunder, en lagom irriterande fördröjning
\end{CodeSmall}
Kolla \Emph{snabbreferensen} vad mer du kan göra med strängar!
\end{SlideExtra}

\begin{SlideExtra}{Exempel på en början till ett textspel}
  Här finns en exempel på en enkel \emph{början} på ett textspel som du stegvis kan ändra och bygga ut till något du själv vill göra:
  \url{https://github.com/lunduniversity/introprog/tree/master/workspace/w03_irritext}

\begin{itemize}
  \item Vilka begrepp och principer ger koden träning i?
\end{itemize}

\end{SlideExtra}

\begin{SlideExtra}{Jobba så här}
\begin{itemize}\SlideFontTiny
  \item Skriv koden i editorn vs code som du startar med \code{code .}
  \item Dra igång 3 olika terminalfönster:
  \begin{enumerate}\SlideFontTiny
  \item \code{scala compile . --watch}~\\så att din kod kompileras om automatiskt vid varje sparning med Ctrl+S.
  \item \code{scala repl .}~\\så att du kan göra mindre undersökningar rad för rad, medan du tänker. Ctrl+D avslutar REPL, så du kan börja om efter kodändring.
  \item \code{scala run .}~\\för att se om programmet funkar som det ska. Om programmet väntar på input kan du behöva avbryta för att köra om efter ändring. 
  \end{enumerate}
  \item Börja enkelt och bygg vidare steg för steg.
  \item Bygg om koden allteftersom den växer genom att införa nya abstraktioner med väl valda namn (s.k. ''refaktorisering'') .
  \item Fixa \Alert{alla} kompileringsfel och \Alert{alla} körtidsfel \Emph{innan} du går vidare.
  \item Fokusera på kodens \Alert{läsbarhet}: Om en funktion blir stor så försök dela upp den i flera funktioner med bra namn. 
  \item En åtgärd som förbättrar läsbarheten utan att ändra hur koden fungerar ur användarens synvinkel kallas \Emph{refaktorisering} \Eng{refactoring}. \\\url{https://sv.wikipedia.org/wiki/Omstrukturering_av_kod}\\\url{https://en.wikipedia.org/wiki/Code_refactoring} 
\end{itemize}

\end{SlideExtra}

\fi
