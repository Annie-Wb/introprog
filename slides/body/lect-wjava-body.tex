%!TEX encoding = UTF-8 Unicode
%!TEX root = ../lect-w12.tex

%%%



\Subsection{Jämförelse Scala och Java}
\begin{Slide}{Övning \texttt{scalajava} och labb \texttt{javatext}}\SlideFontSmall
%\Emph{Labbförberedelse:}
Valfria uppgifter som ger dig en flygande start i efterföljande kurs:
\begin{itemize}
\item Övning \texttt{scalajava}:
\begin{itemize}\SlideFontTiny
\item Översätta från Java till Scala och från Scala till Java
\item Undersöka autoboxning \Eng{autoboxing}
\item Använda \code{import scala.jdk.CollectionConverters.*}
%\item[] Utfasad \Eng{deprecated} sedan Scala 2.13.0: \code{import scala.collection.JavaConverters._}
\end{itemize}
\item Laboration \code{javatext}:
\begin{itemize}\SlideFontTiny
  \item Gör ett textspel för terminalen huvudsakligen i Java men vissa delar i Scala, enligt krav, tips och inspiration i labb-instruktionerna.

\end{itemize}
\item Tips: Scala CLI och sbt kan blanda \code{.scala} och \code{.java} i samma projekt.
\end{itemize}
\end{Slide}


\begin{Slide}{''Hello world!'' i Java.}

\noindent Ett minimalt huvudprogram i Java:
\begin{Code}[language=Java]
public class Hello {
    public static void main(String[] args) {
        System.out.println("Hello world!");
    }
}
\end{Code}

% Motsvarande i Scala:
% \begin{Code}
% object Hello:
%   def main(args: Array[String]): Unit = println("Hello world!")
% \end{Code}

% I Scala (men inte Java) går det även att skriva typsäkra main-funktioner på toppnivå med annoteringen \code{@main}.
% \begin{Code}
% @main def run(n: Int): Unit = println("Hello world!" * n)
% \end{Code}
\end{Slide}


\begin{Slide}{Testa Java i \texttt{jshell}}
\begin{itemize}\SlideFontSmall
  \item Java har en motsvarighet till Scalas REPL: kommandot \code{jshell}
\end{itemize}
\begin{REPLsmall}
> jshell
|  Welcome to JShell -- Version 11.0.11
|  For an introduction type: /help intro

jshell> /help intro
|  
|                                   intro
|                                   =====
|  
|  The jshell tool allows you to execute Java code, getting immediate results.
|  You can enter a Java definition (variable, method, class, etc), like:  int x = 8
|  or a Java expression, like:  x + x
|  or a Java statement or import.
|  These little chunks of Java code are called 'snippets'.
|  
|  There are also the jshell tool commands that allow you to understand and
|  control what you are doing, like:  /list
|  
|  For a list of commands: /help

jshell> 
\end{REPLsmall}
\end{Slide}

\begin{Slide}{Grundläggande likheter och skillnader Java--Scala}\SlideFontSmall
\Emph{Några likheter:}
\begin{itemize}\SlideFontTiny
\item Kompilerar till bytekod som kör på JVM på många olika plattformar.
\item Statisk typning: ger snabb maskinkod, kompilatorn kan ge stöd vid förändring av kod (s.k. refactoring) och hittar många buggar redan vid kompilering.
\end{itemize}

\noindent \Emph{Liknande men} \Alert{viss skillnad:}
\ifkompendium
\\~\\
\else
\vspace{-1em}\begin{multicols}{2}
\fi
\Emph{Java}
\begin{itemize}\SlideFontTiny
\item \Emph{Objektorientering}, men inte ''äkta'' \Eng{pure} eftersom inte alla värden är objekt

\item Primitiva typer är inte objekt; representeras effektivt, normalt \Emph{utan boxning}

\item Visst stöd för \Emph{funktionsprogrammering}

%\item Typer måste anges, ibland två gånger (variabeldeklaration + instansiering)
\end{itemize}

\ifkompendium\else
\columnbreak
\fi

\noindent\Emph{Scala}
\begin{itemize}\SlideFontTiny
\item \Emph{Äkta objektorienterat} eftersom alla värden är objekt, även funktioner

\item \code{AnyVal}-instanser är äkta objekt men representeras ändå effektivt, normalt \Emph{utan boxning}

\item Omfattande stöd för \Emph{funktionsprogrammering}

%\item Typer kan för det mesta härledas av kompilatorn.
\end{itemize}
\ifkompendium\else
\end{multicols}
\fi
\end{Slide}



\begin{Slide}{Huvudprogram i Scala och Java}
\begin{multicols}{2}
  \noindent\Emph{Scala 2}
\begin{CodeSmall}[basicstyle=\footnotesize\ttfamily\SlideFontSize{6}{8},backgroundcolor=\color{white},
  frame=none]
object Main {
  def main(args: Array[String]): Unit = {
    println("Hello!")
  }
}
\end{CodeSmall}

\columnbreak

\noindent\Emph{Java}
\begin{CodeSmall}[language=Java,basicstyle=\footnotesize\ttfamily\SlideFontSize{6}{8},backgroundcolor=\color{white},
  frame=none]
public class JMain {
  public static void main(String[] args){
    System.out.println("Hello!");
  }
}
\end{CodeSmall}
\end{multicols}

\pause\vspace{1em}\noindent\Emph{Scala 3}
\begin{CodeSmall}[basicstyle=\footnotesize\ttfamily\SlideFontSize{6}{8},backgroundcolor=\color{white},
  frame=none]
@main 
def sumFirst(n: Int, xs: Int*): Unit = println(xs.take(n).sum)
  
\end{CodeSmall}

\end{Slide}


\begin{Slide}{Loopa genom argumenten i ett Java-huvudprogram}
\begin{REPLnonum}
> code HelloJavaArgs.java
\end{REPLnonum}
\begin{Code}[language=Java]
public class HelloJavaArgs {
    public static void main(String[] args) {
    int i = 0;
    while (i < args.length) {
      System.out.println(args[i]);
      i = i + 1;
    }
  }
}
\end{Code}
Kompilera och kör:
\begin{REPL}
> javac HelloJavaArgs.java
> java HelloJavaArgs hej gurka tomat
hej
gurka
tomat
\end{REPL}
\end{Slide}


\begin{Slide}{HIGHSCORE implementerad i Java}
\begin{Code}[language=Java]
import java.util.Scanner;

public class HighScore {
    public static void main(String[] args){
        Scanner scan = new Scanner(System.in);
        System.out.println("Hur många poäng fick du?");
        int points =  scan.nextInt();
        System.out.println("Vad var highscore före senaste spelet?");
        int highscore = scan.nextInt();
        if (points > highscore) {
            System.out.println("GRATTIS!");
        } else {
            System.out.println("Försök igen!");
        }
    }
}
\end{Code}
\end{Slide}



\begin{Slide}{Några saker som finns i Scala men inte i Java}\SlideFontTiny
\ifkompendium\else
\vspace{-1em}\begin{multicols}{2}
\fi
\begin{itemize}\SlideFontSize{6.8}{7.2}
\item \code{case}-klasser

\item Lokala funktioner

\item Metoder som operatorer

\item Infix operatornotation

\item Defaultargument

\item Namngivna argument

\item Engångsinitialisering: \code{val}

\item Fördröjd initialisering: \code{lazy val}

\item Enhetlig access för \code{def}, \code{val}, \code{var}

\item Egna setters med \code{def namn_=}

\item Namnanrop, fördröjd evaluering

\item Matchning, mönster och garder

\item Klassparametrar, primärkonstruktor

\item Singelobjekt: \code{object}

\item Kompanjonsobjekt

\item Inmixning: \code{trait}

\item \code{for}-\code{yield}-uttryck

\item Block är uttryck; slipper \code{return}

\item Tomma värdet () av typen \code{Unit}

\item \code{Option}, \code{Some}, \code{None}  (Java har \code{Optional} som ger en del, men inte allt...)

\item \code{Try}, \code{Success}, \code{Failure}

\item Samlingarna i Scalas standardbibliotek, speciellt de \Emph{oföränderliga} samlingarna \code{Vector}, \code{Map}, \code{Set}, \code{List}, etc.

\item Innehållslikhet med \code{==} för oföränderliga strukturer, inkl. \code{< <= > >= } på strängar

\item \Alert{Enhetlig} användning av samlingar \Emph{inkl. Array} (förutom innehållslikhet för Array)

\item Kontextuella abstraktioner \code{given} \code{using}

\item Mer precis synlighet \code{private[mypack]}

\item Namnändring vid \code{import}

\item Flexibel filstruktur och filnamngivning

\item Flexibel nästling av klasser, objekt, traits

\item Typ-alias och abstrakta typer med \code{type}

\item Extensionsmetoder \code{extension}

\item ...
\end{itemize}
\ifkompendium\else
\end{multicols}
\fi
\end{Slide}


\begin{Slide}{Några saker som finns i Java men inte i Scala}
\ifkompendium\else
\vspace{-0.7em}\begin{multicols}{2}\SlideFontTiny
\fi
\begin{itemize}
\item[\textbf{\texttt{+}}] Snabbare kompilering

\item[\textbf{\texttt{+}}] Mognare verktygsstöd

\item Variabeldeklaration utan initialisering

\item Förändringsbara parametrar

\item C-liknande prefix- och postfix-inkrementering och -dekrementering: \code{i++ ++i i-- --i}

\item C-liknande \code{for}-sats

\item Semikolon krävs efter alla satser

\item \jcode{return} krävs i alla metoder som har returvärde

\item Nyckelordet \jcode{void}

\item Parenteser efter alla metoder

\item Specialsyntax för indexering av array \code{[]} ej som i andra samlingar

\item Hoppa ut ur loop med \jcode{break} \\ \href{https://docs.oracle.com/javase/tutorial/java/nutsandbolts/branch.html}{\SlideFontSize{6}{8.5}docs.oracle.com/javase/tutorial/}

\item \jcode{switch} ''faller igenom'' om du glömmer skriva \jcode{break}

\item Kontrollerade undantag \Eng{checked exceptions} och \jcode{throws}

\item ...
\end{itemize}
\ifkompendium\else
\end{multicols}
\fi
\end{Slide}


\begin{Slide}{Begränsningar med funktionsprogrammering i Java}
\begin{itemize}\SlideFontTiny
\item[] Av alla dessa funktionsprogrammeringskoncept i Scala...
\begin{itemize}\SlideFontSize{6}{8}
\item \Emph{överlagring}
\item \Emph{anonyma funktioner}
\item mönstermatchning
\item funktioner etc. på toppnivå
\item utelämna tom parameterlista (enhetlig access)
\item defaultargument
\item namngivna argument
\item lokala funktioner
\item funktioner som äkta värden
\item klammerparentes vid ensam paramenter
\item multipla parameterlistor
\item egendefinierade kontrollstrukturer
\item namnanrop (fördröjd evaluering)
\item stegade funktioner (''Curry-funktioner'')
\item fångad variablelrymd i funktionsobjekt (''closure'')
\item ad hoc polymorfism (''typklasser'')
\item kontextuella abstraktioner
\end{itemize}
\item[] ...kan man i Java %\footnote{\SlideFontTiny\href{https://en.wikipedia.org/wiki/Java_version_history}{en.wikipedia.org/wiki/Java\_version\_history}}  
endast göra: \Emph{överlagring} (''overloading'') och \Emph{anonyma funktioner} (''lambda'') där det senare har starka begränsningar. %\footnote{\SlideFontTiny\href{https://en.wikipedia.org/wiki/Anonymous_function\#Java_Limitations}{en.wikipedia.org/wiki/Anonymous\_function\#Java\_Limitations}}

%\item \vspace{0.5em} En av de saker jag saknar mest i Java: \Alert{lokala funktioner}!
%\item[] Det är \Alert{kombinationen} av alla koncept som \Alert{skapar uttryckskraften} i Scala.
\end{itemize}
\ifkompendium\else
\vfill\SlideFontSize{6}{8} 
\fi Läs mer om Java här:\\
\url{{https://en.wikipedia.org/wiki/Java_version_history}}
\url{https://en.wikipedia.org/wiki/Anonymous_function\#Java_Limitations}

\end{Slide}

\begin{Slide}{Grundtyper i Scala och primitiva typer Java}\SlideFontSmall
\begin{table}[H]
\renewcommand{\arraystretch}{1.4}
\begin{tabular}{l|l|l|l}
\Alert{Grundtyp} i &  Antal                &      Omfång&\Alert{primitiv typ} i\\
  \Emph{Scala} & bitar & minsta/största värde &\Emph{Java} \\ \hline
\texttt{Byte}   &  8  & $-2^7$ ... $2^7-1$   & \texttt{byte} \\
\texttt{Short}  &  16 & $-2^{15}$ ... $2^{15}-1$ & \texttt{short} \\
\texttt{Char}   &  16 & $0$ ... $2^{16}-1$ & \texttt{char} \\
\texttt{Int}    &  32 & $-2^{31}$ ... $2^{31}-1$ & \texttt{int} \\
\texttt{Long}   &  64 & $-2^{63}$ ... $2^{63}-1$ & \texttt{long} \\
\texttt{Float}  &  32 & ± $3.4028235 \cdot 10^{38}$  & \texttt{float} \\
\texttt{Double} &  64 & ± $1.7976931348623157 \cdot 10^{308}$ & \texttt{double} \\
\end{tabular}
\end{table}
\end{Slide}
  


\begin{Slide}{Javas switch-sats}\SlideFontSmall
De flesta C-liknande språk (men inte Scala) har en \jcode{switch}-sats som man kan använda istället för (vissa) nästlade if-else-satser:
\javainputlisting[basicstyle=\ttfamily\SlideFontSize{5}{6}\selectfont]{../compendium/examples/match/Switch.java}
{\SlideFontTiny
\code{switch} i Java har stora begränsningar (fungerar t.ex. bara på primitiva typer och några till, tex String); i framtiden planerar man att anamma en del av det \code{match} i Scala kan.
}
\end{Slide}


\begin{Slide}{Javas switch-sats utan break}\SlideFontSmall
Saknad \jcode{break}-sats ''faller igenom'' till efterföljande gren:

\javainputlisting[basicstyle=\ttfamily\SlideFontSize{6}{7}\selectfont]{../compendium/examples/match/SwitchNoBreak.java}
En glömd \jcode{break} kan ge svårhittad bugg...
\end{Slide}

\begin{Slide}{Javas switch-sats med glömd break}\SlideFontSmall

\vspace{-0.5em}\javainputlisting[basicstyle=\ttfamily\SlideFontSize{5.5}{6.8}\selectfont]{../compendium/examples/match/SwitchForgotBreak.java}

\vspace{-0.7em}\pause
\begin{REPLsmall}
> java SwitchForgotBreak
Skriv grönsak:
gurka
gott!
gott!
\end{REPLsmall}

\end{Slide}


% \begin{Slide}{Regler för identifierare i Java}\footnotesize
% När kompilatorn ''läser''  \footnote{man säger ofta ''parsa'' i stället för ''läsa'' när kompilatorn tolkar koden} koden och och försöker hitta variabelnamn, antar den att du följer de entydiga syntaktiska reglerna för språket.  \\ \vskip1em För namn i Java gäller följande regler: %https://docs.oracle.com/javase/tutorial/java/nutsandbolts/variables.html
% \begin{itemize}
% \item Namn får inte vara \href{https://docs.oracle.com/javase/tutorial/java/nutsandbolts/_keywords.html}{reserverade ord}
% \item Stora och små bokstäver spelar roll \Eng{case sensistive} \\ \lstinline{int highScore;} och \lstinline{int highscore;} ger alltså två \textit{olika} variabler
% \item Namnet måste börja med en bokstav, ett understreck \_ eller ett dollartecken \$
% \item Namn får \textit{inte} innehålla blanktecken
% \item Namn får innehålla bokstäver, siffror, understreck \_ och dollartecken \$, men \textit{inte} andra specialtecken (alltså inte \lstinline~%&@!{(})/+-*~ etc.)
% \end{itemize}
% \end{Slide}

\ifkompendium  % bad hardcoded hack to fix column alignment problems
\clearpage
\fi

\begin{Slide}{Syntax för variabeldeklaration i Scala och Java}\SlideFontSmall
Exempel på variabeldeklarationer i
\begin{multicols}{2}
\noindent\Emph{Scala}
\begin{CodeSmall}[basicstyle=\ttfamily\SlideFontSize{7}{10},backgroundcolor=\color{white},
  frame=none]
  var i1: Int = 0
  var i2 = 0
  var i3 = 0: Int
  var p1: Point = new Point(0, 0)
  var p2 = new Point(0, 0)
  var (x, y) = (0, 0)
  val a = 0
  final val Constant = 42
\end{CodeSmall}
\begin{itemize}\SlideFontTiny
\item i2 härledd typ; går ej i Java 8,9 men finns med \code{var} i Java 10

\item i3 typ varhelst i uttryck; går ej i Java

\item (x, y) mönster i init; går ej i Java

\item \code{val} ger ''engångsinit''; ingen exakt motsvarighet i Java men \code{final} kan ofta användas i stället
\end{itemize}

\columnbreak

\noindent\Emph{Java}
\begin{CodeSmall}[language=Java,morekeywords={var},basicstyle=\ttfamily\SlideFontSize{7}{10},backgroundcolor=\color{white},
  frame=none]
  int i1 = 0;
  var i2 = 0; // från Java 10
  int i4;
  Point p1 = new Point(0, 0);
  var p2 = new Point(0, 0); 
  final int CONSTANT = 42;
\end{CodeSmall}
\begin{itemize}\SlideFontTiny
\item i4 ej explicit init; går ej i Scala
\end{itemize}
\end{multicols}

\end{Slide}


\begin{Slide}{For-sats i Scala och Java}
\begin{multicols}{2}
\noindent\Emph{Scala}
\begin{CodeSmall}[basicstyle=\ttfamily\SlideFontSize{6}{8},backgroundcolor=\color{white},
  frame=none]
val s = "Abbasillen"

// Loopa över index framlänges:

for i <- 0 until s.length do
  println(s(i))


// Loopa över index baklänges:

for i <- s.length-1 to 0 by -1 do
  println(s(i))

\end{CodeSmall}

\columnbreak

\noindent\Emph{Java}
\begin{CodeSmall}[language=Java,basicstyle=\ttfamily\SlideFontSize{6}{8},backgroundcolor=\color{white},
  frame=none]
String s = "Abbasillen";

// Loopa över index framlänges:

for (int i = 0; i < s.length(); i++) {
    System.out.println(s.charAt(i));
}

// Loopa över index baklänges:

for (int i = s.length()-1; i >= 0; i--) {
    System.out.println(s.charAt(i));
}
\end{CodeSmall}
\end{multicols}
I Scala är \code{s.indices} att föredra!
\end{Slide}

\ifkompendium  % bad hardcoded hack to fix column alignment problems
\clearpage
\fi


\begin{Slide}{For-sats i Scala med indices}
\begin{multicols}{2}
\noindent\Emph{Scala}
\begin{CodeSmall}[basicstyle=\ttfamily\SlideFontSize{6}{8},backgroundcolor=\color{white},
  frame=none]
val s = "Abbasillen"

// Loopa över index framlänges:

for i <- s.indices do
  println(s(i))


// Loopa över index baklänges:

for i <- s.indices.reverse do
  println(s(i))

\end{CodeSmall}

\columnbreak

\noindent\Emph{Java}
\begin{CodeSmall}[language=Java,basicstyle=\footnotesize\ttfamily\SlideFontSize{6}{8},backgroundcolor=\color{white},
  frame=none]
String s = "Abbasillen";

// Loopa över index framlänges:

for (int i = 0; i < s.length(); i++) {
    System.out.println(s.charAt(i));
}

// Loopa över index baklänges:

for (int i = s.length()-1; i >= 0; i--) {
    System.out.println(s.charAt(i));
}
\end{CodeSmall}
\end{multicols}
\end{Slide}


\begin{Slide}{For-satser och arrayer i Java}\SlideFontSmall
En for-sats i Java har följande struktur:
\begin{Code}[language=Java, basicstyle=\fontsize{10}{12}\ttfamily\selectfont]
for (initialisering; slutvillkor; inkrementering) {
    sats1;
    sats2;
    ...
}
\end{Code}
En primitiv heltals-array deklareras så här i Java:
\begin{Code}[language=Java, basicstyle=\fontsize{9}{11}\ttfamily\selectfont]
int[] xs = new int[42];  // rymmer 42 st heltal, init 0:or
int[] ys = {10, 42, -1}; // initera med 3 st heltal
\end{Code}
Exempel på for-sats: fyll en array med 1:or
\begin{Code}[language=Java, basicstyle=\fontsize{9}{11}\ttfamily\selectfont]
for (int i = 0; i < xs.length; i++){ 
  xs[i] = 1;    // indexera med [i]
}
\end{Code}

\end{Slide}
  
\begin{Slide}{Implementation av SEQ-COPY i Java med \texttt{for}-sats}
\begin{minipage}{0.55\textwidth}
\vspace{-0.5em}
\javainputlisting[numbers=left,numberstyle=,basicstyle=\fontsize{6.5}{8}\ttfamily\selectfont]{../compendium/examples/workspace/w05-seqalg/src/SeqCopyForJava.java}
\end{minipage}
\begin{minipage}{0.44\textwidth}\SlideFontTiny\vspace{-1.5em}\ifkompendium\small\fi
~~~Lite syntax och semantik för Java:
\begin{itemize}
\item En Java-klass med enbart statiska medlemmar motsvarar ett singelobjekt i Scala.

\item Typen kommer \Alert{före} namnet.

\item Man \Alert{måste} skriva \code{return}.

\item Man \Alert{måste} ha semikolon efter varje sats.

\item Metodnamn \Alert{måste} följas av parenteser; om inga parametrar finns används \code{()}

\item En array i Java är inget vanligt objekt, men har ett ''attribut'' \code{length} som ger antal element.

\item \Emph{Övning}: skriv om med \code{while}-sats i stället; har samma syntax i Scala \& Java.

\end{itemize}
\end{minipage}

\end{Slide}
  


\begin{Slide}{Element för element med speciell for-each-sats i Java}
\begin{multicols}{2}
\noindent\Emph{Scala}
\begin{CodeSmall}[basicstyle=\ttfamily\SlideFontSize{6}{8},backgroundcolor=\color{white},
  frame=none]
val s = "Abbasillen"

// Loopa över alla tecken:

for ch <- s do println(ch)


\end{CodeSmall}

\columnbreak

\noindent\Emph{Java}
\begin{CodeSmall}[language=Java,basicstyle=\ttfamily\SlideFontSize{6}{8},backgroundcolor=\color{white},
  frame=none]
String s = "Abbasillen";

// Loopa över alla tecken:

for (char ch: s.toCharArray()) {
  System.out.println(ch);
}
\end{CodeSmall}
\end{multicols}

\pause
{\noindent\SlideFontSmall
\code{s.foreach(println)  } går ej i Java men
från Java 8 finns metoden \code{chars} som ger en \code{IntStream} och då kan man: \\
 \jcode{str.chars().forEachOrdered(i -> System.out.println((char) i));}
 }
\end{Slide}





%\Subsection{Klasser i Java}

\begin{Slide}{Typisk utformning av Java-klass}
Typisk ''anatomi'' hos en Java-klass:
\begin{Code}[language=Java]
public class Klassnamn {
    attribut, normalt privata
    konstruktorer, normalt publika
    metoder: publika getters, och vid förändringsbara objekt även setters
    metoder: privata abstraktioner för internt bruk
    metoder: publika abstraktioner tänkta att användas av klientkoden
}
\end{Code}
\href{http://www.oracle.com/technetwork/java/codeconventions-141855.html#1852}{\SlideFontSize{9}{8}www.oracle.com/technetwork/java/codeconventions-141855.html\#1852}
\end{Slide}


\begin{Slide}{Statiska medlemmar i Java}
\begin{itemize}
\item Man kan \Alert{inte} deklarera explicita singelobjekt i Java och det finns inget nyckelord \code{object}.

\item I stället kan man deklarera \Emph{statiska medlemmar} i en klass med Java-nyckelordet \jcode{static}.

\item Exempel på hur vi ska göra detta inuti en klassen \code{JPerson}:

\begin{Code}[language=Java,basicstyle=\SlideFontSize{10}{12}\ttfamily\selectfont]
public static final int ADULT_AGE = 18;
\end{Code}

\item Effekten blir den samma som ett singelobjekt i Scala:
\begin{itemize}
\item Alla statiska medlemmar i en Java-klass allokeras automatiskt och hamnar i en egen singulär ''klassinstans'' som existerar oberoende av de dynamiska instanserna.
\item De statiska medlemmarna accessas med punktnotation genom klassnamnet, \Alert{utan} \code{new}:
\begin{Code}[language=Java,basicstyle=\SlideFontSize{11}{13}\ttfamily\selectfont]
System.out.println(JPerson.ADULT_AGE);
\end{Code}

\end{itemize}


\end{itemize}
\end{Slide}
  




\begin{Slide}{Exempel: oföränderlig klass i Scala och Java}
\ifkompendium
\noindent\Emph{Scala:}
\else
\SlideFontTiny\vspace{-1em}\begin{multicols}{2}
\fi
\begin{CodeSmall}[basicstyle=\ttfamily\SlideFontSize{5.7}{6.7}]
class Person(val name: String, val age: Int):
  def isAdult = age >= Person.AdultAge

object Person:
  val AdultAge = 18
\end{CodeSmall}

\ifkompendium
\noindent\Emph{Java:}
\else
\columnbreak
\fi

\pause
\begin{CodeSmall}[language=Java,basicstyle=\ttfamily\SlideFontSize{5.7}{6.7}]
public class JPerson {
    private String name;
    private int age;
    public static final int ADULT_AGE = 18;

    public JPerson(String name, int age) {
      this.name = name;
      this.age = age;
    }

    public String getName() {
        return name;
    }

    public int getAge() {
        return age;
    }

    public boolean isAdult() {
        return age >= ADULT_AGE;
    }
}
\end{CodeSmall}
Lär dig detta mönster för en typisk Java-klass utantill så du snabbt får grejerna på plats!
\ifkompendium\\\else
\end{multicols}
\pause\vspace{-13em}%
\fi%
~\\\Alert{Övning:}\\
\ifkompendium
Gör \code{Person} + \code{JPerson} \Alert{förändringsbara} så att namnet och åldern går att uppdatera och följande krav uppfylls:
\else
Gör \code{Person} + \code{JPerson} \Alert{förändringsbara}\\så att namnet och åldern går att uppdatera\\och följande krav uppfylls:
\fi
\begin{itemize}
\item namnet ska ges vid konstruktion,
\item åldern ska initieras till 0 vid konstr.,
\item åldern ska aldrig kunna bli negativ.
\end{itemize}
\end{Slide}


\begin{Slide}{Exempel: Scala-klassen Complex}
\scalainputlisting[basicstyle=\ttfamily\SlideFontSize{7}{9},numbers=left,numberstyle=\SlideFontSize{7}{7}\ttfamily\selectfont]{../compendium/examples/complex7.scala}
{\SlideFontTiny Scala:~~\url{https://github.com/lunduniversity/introprog/blob/master/compendium/examples/complex7.scala}}

{\SlideFontTiny Java:~~~\url{https://github.com/lunduniversity/introprog/blob/master/compendium/examples/JComplex.scala}}

\end{Slide}
  

\begin{Slide}{Exempel: Motsvarande Java-klassen JComplex}\SlideFontTiny\SlideOnly{\vspace{-1em}}
\javainputlisting[basicstyle=\SlideFontSize{5}{5.7}\ttfamily\selectfont,numbers=left,numberstyle=\SlideFontSize{5}{5}\ttfamily\selectfont]{../compendium/examples/JComplex.java}
\end{Slide}

  
  


\begin{Slide}{Exempel: Använda JComplex i Scala-kod}
\begin{REPL}
$ javac JComplex.java
$ scala
Welcome to Scala 2.12.9 (OpenJDK 64-Bit Server VM, Java 1.8.0_222).
Type in expressions for evaluation. Or try :help.

scala> val jc1 = JComplex(3, 4)
jc1: JComplex = 3.0 + 4.0i

scala> val polarForm = (jc1.getR, jc1.getFi)
polarForm: (Double, Double) = (5.0,0.6435011087932844)

scala> val jc2 = JComplex(1, 2)
jc2: JComplex = 1.0 + 2.0i

scala> jc1 add jc2
res0: JComplex = 4.0 + 6.0i
\end{REPL}
\end{Slide}




\begin{Slide}{Exempel: Använda JComplex i Java-kod}\SlideFontSmall
\javainputlisting[basicstyle=\SlideFontSize{8}{10}\ttfamily\selectfont]{../compendium/examples/JComplexTest.java}
\begin{itemize}
\item I Java måste man skriva \code{new}.  
\item I Java måste man skriva \Alert{tomma parentes-par} efter metodnamnet vid \Alert{anrop av parameterlösa metoder}.

\item \Alert{Tupler finns inte} i Java, så det går inte på ett enkelt sätt att skapa par av värden som i Scala; ovan görs polär form till en sträng för utskrift.

\item \Alert{Operatornotation för metoder finns inte} i Java, så man måste i Java använda punktnotation och skriva: \code{jc1.add(jc2)}
\end{itemize}
\end{Slide}
  
  
  







\begin{Slide}{Exempel: Förändringsbar klass i Scala och Java}
\ifkompendium
\noindent\Emph{Scala:}
\else  
\SlideFontTiny\vspace{-1.75em}\begin{multicols}{2}
\fi 
\begin{CodeSmall}[basicstyle=\ttfamily\SlideFontSize{5}{6}]
class MutablePerson(var name: String):
  private var _age = 0

  def age: Int = _age

  def age_=(a: Int): Unit =
    if (a >= 0) _age = a else _age = 0  
      // eller hellre kasta undantag?

  def isAdult: Boolean =
    age >= MutablePerson.AdultAge

object MutablePerson:
  val AdultAge = 18
\end{CodeSmall}

\ifkompendium
~\\\noindent\Emph{Java:}
\else
\columnbreak
\fi 

\pause

\begin{CodeSmall}[language=Java,basicstyle=\ttfamily\SlideFontSize{5}{6}]
public class JMutablePerson {
    private String name;
    private int age = 0;
    public static final int ADULT_AGE = 18;

    public JMutablePerson(String name) {
      this.name = name;
    }

    public String getName() {
        return name;
    }

    public void setName(String name) {
        this.name = name;
    }

    public int getAge() {
        return age;
    }

    public void setAge(int age) {
        if (age >= 0) {
          this.age = age;
        } else {
          this.age = 0;
        }
    }

    public boolean isAdult() {
        return age >= ADULT_AGE;
    }
}
\end{CodeSmall}
\ifkompendium\else
\end{multicols}
\fi
\end{Slide}


\ifkompendium\pagebreak\fi

% \begin{Slide}{Övning: Implementera dessa specifikationer}
% \ifkompendium\else  
% \begin{multicols}{2}
% \fi
% {\hskip-0.31em\colorbox{black!70}{\parbox{\dimexpr0.44\textwidth-20\fboxsep-1.9\fboxrule\relax}{\fontsize{7}{8}\selectfont\color{white}{\textit{Specification} \textbf{Vegetable}}}}}

% \ifkompendium\else
% \vspace{-2em}
% \fi

% \begin{CodeSmall}
% /** Representerar en grönsak. */
% class Vegetable(val name: String):
%   /** Returnerar nuvarande vikt i gram. */
%   def weight: Int = ???

%   /** Ändrar vikten till w gram.
%    *  w ska vara positiv, blir annars 0 */
%   def weight_=(w: Int): Unit = ???
% \end{CodeSmall}

% \ifkompendium\else
% \columnbreak
% \fi

% \begin{JavaSpec}{class JVegetable}
% /** Skapar en grönsak. */
% JVegetable(String name);

% /** Returnerar namnet. */
% String getName();

% /** Returnerar nuvarande vikt i gram. */
% int getWeight();

% /** Ändrar vikten till weight gram.
%  *  w ska vara positiv, blir annars 0 */
% void setWeight(int weight);
% \end{JavaSpec}
% \ifkompendium\else
% \end{multicols}
% \pause\SlideFontTiny
% \fi
% \noindent Fördjupning:\\ Kasta undantaget \code{IllegalArgumentException} vid försök till negativ vikt.\\
% Läs om undantag i Java här: \href{https://docs.oracle.com/javase/tutorial/essential/exceptions/index.html}{docs.oracle.com/javase/tutorial/essential/exceptions/}

% \end{Slide}




\begin{Slide}{Scalas ''case-klass-godis'' finns inte i Java}
\ifkompendium\else
\SlideFontTiny\vspace{-0.5em}\begin{multicols}{2}
\fi 

En oföränderlig datatyp implementeras i \Emph{Scala} helst som en \pause\code{case}-klass:

\begin{CodeSmall}[basicstyle=\ttfamily\SlideFontSize{5.7}{6.7}]
case class Person(name: String, age: Int):
  def isAdult = age >= Person.AdultAge

object Person:
  val AdultAge = 18
\end{CodeSmall}

\pause
\ifkompendium\else\columnbreak\fi

\noindent En oföränderlig datatyp i \Emph{Java} med \Alert{motsvarande} funktionalitet kräver egen implementation av dessa metoder:
\ifkompendium\else\vspace{-0.25em}\fi
\begin{itemize}
\item en getter för varje attribut
\item \code{equals}
\item \code{hashCode} (förklaras i forts.kurs)
\item \code{apply} \\ (men man kallar nog den \code{create} el. likn.; namnet måste ju skrivas)
\item \code{toString}
\item \code{copy} \\ (men det finns ju inte namngivna parametrar och default-argument så denna blir osmidig)
\item \code{unapply} \\ (men det finns ju inte mönstermatchning så denna struntar man nog i)
\end{itemize}

\ifkompendium\else
\end{multicols}
\fi
\end{Slide}




\Subsection{Array i Java}

\begin{Slide}{Repetition: Den primitiva typen Array i JVM}
\begin{itemize}
\item Primitiva arrayer (\code{Array} i Scala, \code{[]} i Java) har \Emph{fördelar}:%
\footnote{\href{http://stackoverflow.com/questions/2843928/benefits-of-arrays}{stackoverflow.com/questions/2843928/benefits-of-arrays}}
\begin{itemize}\SlideFontSmall
\item Det är den snabbaste indexerbara datastrukturen i JVM: att läsa och uppdatera ett element på en viss plats är mycket effektivt om man vet platsens index.
\item Fungerar lika bra med både primitiva värden och objektreferenser
\end{itemize}
\item ... men också \Alert{nackdelar}:
\begin{itemize}\SlideFontSmall
\item Man måste bestämma sig för antalet element som ska allokeras när man gör \code{new}.
\item Man kan ta i lite extra när man allokerar om man behöver plats för fler senare, men då måste man hålla reda på hur många platser man använder och veta var nästa lediga plats finns.
\item Det är krångligt att stoppa in \Eng{insert} och ta bort \Eng{delete} element.
\item Vill man ha fler platser måste man allokera en helt ny, större array och kopiera över alla befintliga element.
\end{itemize}

\end{itemize}
\end{Slide}




\begin{Slide}{Syntax för Array i Scala och Java}
\begin{multicols}{2}
\Emph{Scala}

\begin{CodeSmall}[basicstyle=\ttfamily\SlideFontSize{6}{8}\ifkompendium 
,backgroundcolor=\color{white},frame=none 
\fi
]
var xs = Array(42, 43, 44)




val n = xs.length

var strings = new Array[String](42)
// eller      Array.ofDim[String](42)
// eller      Array.fill(42)(null: String)

strings(0) = "first"

strings(1) = "second"
\end{CodeSmall}

\columnbreak

\Emph{Java}

\begin{CodeSmall}[language=Java,basicstyle=\ttfamily\SlideFontSize{6}{8}\ifkompendium 
,backgroundcolor=\color{white},frame=none 
\fi
]
int[] xs = new int[]{42, 43, 44};

// samma som ovan, men kortare:
int[] xs2 = {42, 43, 44};

int n = xs.length;  // EJ length()

String[] strings = new String[42];



strings[0] = "first";

strings[1] = "second";
\end{CodeSmall}

\end{multicols}
\end{Slide}






\begin{Slide}{Exempel: Polygon med primitiv array i Java}
\begin{Code}[numberstyle=,numbers=left,language=Java]
public class Polygon {
    private Point[] vertices; // array med hörnpunkter
    private int n;            // antalet hörnpunkter

    /** Skapar en polygon */
    public Polygon() {
        vertices = new Point[1];
        n = 0;
    }

    ...
\end{Code}
\end{Slide}

\begin{Slide}{Polygon med primitiv array i Java: stoppa in sist och vid behov skapa mer plats}\SlideFontSmall
Implementera:\\
\jcode{private void extend()                // dubbla storleken}\\
\jcode{public void addVertex(int x, int y)  // lägg till hörnpunkt}
\pause
\begin{Code}[numberstyle=,numbers=left,language=Java]
    private void extend(){
        Point[] oldVertices = vertices;
        vertices = new Point[2 * vertices.length]; // skapa dubbel plats
        for (int i = 0; i < oldVertices.length; i++) {  // kopiera
            vertices[i] = oldVertices[i];
        }
    }

    public void addVertex(int x, int y) {
        if (n == vertices.length) extend();
        vertices[n] = new Point(x, y);
        n++;
    }
\end{Code}
\end{Slide}


\begin{Slide}{Polygon med primitiv array i Java: stoppa in mitt i på angiven plats }\SlideFontSmall
Implementera:\\
\jcode{/** Sätt in hörnpunkt på plats pos */}\\
\jcode{public void insertVertex(int pos, int x, int y)}
\pause
\begin{Code}[numberstyle=,numbers=left,language=Java]
    public void insertVertex(int pos, int x, int y) {
        if (n == vertices.length) extend();   // utöka vid behov
        for (int i = n; i > pos; i--) {       // flytta element bakifrån
            vertices[i] = vertices[i - 1];
        }
        vertices[pos] = new Point(x, y);
        n++;
    }
\end{Code}
\end{Slide}


%\Subsection{Scanner}

\begin{Slide}{Scanna filer och strängar med \texttt{java.util.Scanner}}\SlideFontTiny
\setlength{\leftmargini}{0pt}
\begin{itemize}
\item I Scala kan man läsa från fil så här (se quickref sid 3 längst ner):

\begin{Code}
val names = scala.io.Source.fromFile("src/names.txt").getLines.toVector
\end{Code}

\item Klassen \code{java.util.Scanner} kan också läsa från fil (se Java Snabbref sid 4):


\begin{Code}
def readFromFile(fileName: String): Vector[String] = {
  val file = new java.io.File(fileName)
  val scan = new java.util.Scanner(file)
  val buffer = scala.collection.mutable.ArrayBuffer.empty[String]
  while (scan.hasNext) {
    buffer += scan.next
  }
  scan.close
  buffer.toVector
}
\end{Code}

\item Med \code{new java.util.Scanner(System.in)} kan man även scanna tangentbordet.

\item Med \code{new java.util.Scanner("hej 42")} kan man även scanna en sträng.

\item Scanna \code{Int} och \code{Double} med metoderna \code{nextInt} och \code{nextDouble}.\\Se doc: \href{https://docs.oracle.com/javase/8/docs/api/java/util/Scanner.html}{\SlideFontTiny docs.oracle.com/javase/8/docs/api/java/util/Scanner.html}
\end{itemize}
\end{Slide}


\begin{Slide}{Exempel: Scanner}
\begin{REPL}
scala> val scan = new java.util.Scanner("hej 42 42.0   42 slut")

scala> scan.hasNext
res0: Boolean = true

scala> scan.hasNextInt
res1: Boolean = false

scala> scan.next
res2: String = hej

scala> scan.hasNextInt
res3: Boolean = true

scala> scan.nextInt
res4: Int = 42

scala> while (scan.hasNext) println(scan.next)
42.0
42
slut
\end{REPL}
\end{Slide}



%\Subsection{scala.jdk.CollectionConverters}

\begin{Slide}{Använda Java-samlingar i Scala med \texttt{CollectionConverters}}\SlideFontSmall
Med hjälp av \code{import scala.jdk.CollectionConverters.*} \\
får du smidig \Emph{interoperabilitet} med Java och dess standardbibliotek, \\
speciellt metoderna \Alert{\code{asJava}} och \Alert{\code{asScala}}:
\begin{REPL}
scala> import scala.jdk.CollectionConverters.*

scala> Vector(1,2,3).asJava
res0: java.util.List[Int] = [1, 2, 3]

scala> val xs = new java.util.ArrayList[String]()
xs: java.util.ArrayList[String] = []

scala> xs.add("hej")
res1: Boolean = true

scala> xs.asScala
res2: scala.collection.mutable.Buffer[String] = Buffer(hej)
\end{REPL}

\noindent Läs mer här: %
\ifkompendium\\\fi%
\scriptsize%
\url{https://docs.scala-lang.org/overviews/collections-2.13/conversions-between-java-and-scala-collections.html}

\end{Slide}




%\Subsection{ArrayList}

\begin{Slide}{Generiska samlingar i Java}
\begin{itemize}
\item Från och med version 5 av Java (2004) så introducerades \Emph{generics} vilket möjliggör skapandet av klasser som kan erbjuda generell behandling av olika typer av objekt.

\item Generiska klasser i Java känns igen med syntaxen \code{Klassnamn<Typ>}, till exempel  \code{ArrayList<Point>}

\item Fördjupning: \href{https://docs.oracle.com/javase/tutorial/extra/generics/intro.html}{docs.oracle.com/javase/tutorial/extra/generics/intro.html}, mer om detta i fördjupningskursen.

\end{itemize}
\end{Slide}

\begin{Slide}{Om ArrayList i Java}\SlideFontSmall
\code{java.util.ArrayList} liknar \code{scala.collection.mutable.ArrayBuffer} som båda har dessa fördelar:
\begin{itemize}
\item Lagrar sina element internt i snabbindexerade primitiva arrayer.
\item Fungerar för alla typer av objekt.
\item Utökar samlingens storlek av sig själv vid behov.
\end{itemize}
Det finns dock vissa nackdelar med \code{ArrayList} i Java\\(som inte gäller för \code{ArrayBuffer} i Scala):
\begin{itemize}
\item Fungerar \Alert{inte} rakt av med primitiva typer \code{int}, \code{double}, \code{char}, ... \\ (men det finns sätt komma runt detta, tack vare s.k. wrapper-klasser och autoboxning; mer om detta snart)

\item Namnet \code{ArrayList} är inte helt lyckat, eftersom ordet ''lista'' normalt används för länkade snarare än array-liknande strukturer.
\end{itemize}
\end{Slide}

\begin{Slide}{Polygon med ArrayList i Java}\SlideFontSmall
Klassen \code{Polygon}, nu med ett attribut av typen \code{ArrayList<Point>}:
\begin{Code}[numberstyle=,language=Java]
public class Polygon {
    private ArrayList<Point> vertices; // lista med hörnpunkter

    /** Skapar en polygon */
    public Polygon() {
        vertices = new ArrayList<Point>();
    }

    ...
\end{Code}
Det behövs inget attribut \code{n} eftersom vi inte själva behöver hålla reda på antalet allokerade platser: allokering, insättning, och utökning av antalet platser sköts helt automatiskt av \code{ArrayList}-klassen vid behov.
\end{Slide}

\begin{Slide}{Några viktiga operationer på ArrayList<E>}
\SlideFontTiny\url{https://docs.oracle.com/en/java/javase/17/docs/api/java.base/java/util/ArrayList.html}
\begin{Code}[numberstyle=,language=Java]
/** Tar reda på elementet på plats pos */
E get(int pos);

/** Lägger in objektet obj sist */
void add(E obj);

/** Lägger in obj på plats pos; efterföljande flyttas */
void add(int pos, E obj);

/** Tar bort elementet på plats pos och returnerar det */
E remove(int pos);

/** Tar reda på antalet element i listan */
int size();
\end{Code}
Lär dig vad som finns om ArrayList i snabbreferensen för Java\\
Överkurs för den nyfikne: kolla implementation av ArrayList här: \\ \url{https://hg.openjdk.org/jdk8/jdk8/jdk/file/tip/src/share/classes/java/util/ArrayList.java#l106}
\end{Slide}


\begin{Slide}{Övning ArrayList: new och add}
Skriv Java-kod som skapar en lista med element av typen \code{Point} och lägger in tre punkter i listan med koordinaterna:\\ (50, 50), (50,10) och (30, 40).
\pause
~\\~\\ Lösning: \ifkompendium\else\\~\\\fi
\begin{Code}[numberstyle=,language=Java]
ArrayList<Point> vertices = new ArrayList<Point>();
vertices.add(new Point(50, 50));
vertices.add(new Point(50, 10));
vertices.add(new Point(30, 40));
\end{Code}
\end{Slide}


\begin{Slide}{For-each-sats i Java:}\SlideFontSmall
\begin{itemize}
\item  Antag att vi vill gå igenom alla element i en lista.
\begin{Code}[numberstyle=,language=Java]
        ArrayList<String> words = new ArrayList<String>();
\end{Code}
\item Det finns två olika typer av \jcode{for}-satser i Java som kan göra detta:
\begin{itemize}\SlideFontSmall
\item  Vanlig \jcode{for}-sats:
\begin{Code}[numberstyle=,language=Java]
for (int i = 0; i < words.size(); i++) {
    System.out.println(i + ": " + words.get(i));
}
\end{Code}

\item  Så kallad \Emph{for-each-sats} med denna syntax:\\
\jcode+for (Elementtyp element: samling) { ... }+ \\
\vspace{1em}Exempel:
\begin{Code}[numberstyle=,language=Java]
for (String s: words) {
    System.out.println(s);
}
\end{Code}
Men vi får ingen indexvariabel då...
\end{itemize}
\end{itemize}
\end{Slide}


\begin{Slide}{Polygon med ArrayList: metoderna blir enklare}
\begin{Code}[numberstyle=,language=Java]
    public void addVertex(int x, int y) {
        vertices.add(new Point(x, y));
    }

    public void move(int dx, int dy) {
        for (Point p: vertices){
            p.move(dx, dy);
        }
    }

    public void insertVertex(int pos, int x, int y) {
        vertices.add(pos, new Point(x, y));
    }

    public void removeVertex(int pos) {
        vertices.remove(pos);
    }
\end{Code}

Se hela lösningen här:
\href{https://github.com/lunduniversity/introprog/tree/master/compendium/examples/scalajava/list/Polygon.java}{compendium/examples/scalajava/list/Polygon.java}
\end{Slide}

\begin{Slide}{Polygon med ArrayList: iterera över alla hörnpunkter i draw med indexering}
\begin{Code}[numberstyle=,language=Java]
    public void draw(SimpleWindow w) {
        if (vertices.size() == 0) {
            return;
        }
        Point start = vertices.get(0);
        w.moveTo(start.getX(), start.getY());
        for (int i = 1; i < vertices.size(); i++) {
            w.lineTo(vertices.get(i).getX(),
                     vertices.get(i).getY());
        }
        w.lineTo(start.getX(), start.getY());
    }
\end{Code}

Övning: Skriv om med for-each-sats.
\end{Slide}

\begin{Slide}{Polygon med ArrayList: iterera över alla hörnpunkter i draw med foreach-sats}
\begin{Code}[numberstyle=,language=Java]
    public void draw(SimpleWindow w) {
        if (vertices.size() == 0) {
            return;
        }
        Point start = vertices.get(0);
        w.moveTo(start.getX(), start.getY());
        for (Point p: vertices){
            w.lineTo(p.getX(), p.getY());
        }
        w.lineTo(start.getX(), start.getY());
    }
\end{Code}

Se hela lösningen här:
\href{https://github.com/lunduniversity/introprog/tree/master/compendium/examples/scalajava/list/Polygon.java}{compendium/examples/scalajava/list/Polygon.java}
\end{Slide}




\begin{Slide}{Övning ArrayList: implementera metoden hasVertex}
Skriv kod som implementerar denna metod i klassen \code{Polygon}:
\begin{Code}[numberstyle=,language=Java]
/** Undersöker om polygonen har någon hörnpunkt med koordinaterna x, y. */
public boolean hasVertex(int x, int y) {
    ???
}
\end{Code}
\end{Slide}

\begin{Slide}{Lösning ArrayList: implementera metoden hasVertex}
\begin{Code}[numberstyle=,language=Java]
    public boolean hasVertex(int x, int y) {
        for (Point p: vertices) {
            if (p.getX() == x && p.getY() == y) {
                return true;
            }
        }
        return false;
    }
\end{Code}
\end{Slide}


\begin{Slide}{For-each-sats med array}
For-each-sats fungerar även med primitiv array:
\begin{Code}[numberstyle=,language=Java]
        String[] stringArray = {"hej", "på", "dej"};
        for (String s: stringArray) {
            System.out.println(s);
        }
\end{Code}
\end{Slide}





\Subsection{Autoboxning i Java}



\begin{Slide}{Generiska klasser (t.ex. ArrayList) med primitiva typer}
Detta går tyvärr \Alert{INTE} i Java: \\
  \sout{\texttt{ArrayList<int> list = new ArrayList<int>();}}
  
\pause
\begin{itemize}\SlideFontSmall
\item Hur gör man om man vill ha heltalselement (eller andra primitiva värden) i en generisk samling?

\item Javas lösning på problemet består av två delar:
\begin{itemize}\SlideFontSmall
\item Klasser som packar in primitiva typer, \Eng{wrapper classes}
\item Speciella regler för implicita konverteringar, s.k. ''auto-boxing'' \Eng{Boxing / Unboxing conversions}
\end{itemize}
\end{itemize}
\SlideFontTiny\vspace{1em}
Ofta fungerar det fint, men det finns fallgropar.\\
(Om du är nyfiken på alla intrikata detaljer, se
\href{https://docs.oracle.com/javase/tutorial/java/data/autoboxing.html}{Java tutorial} och   \href{https://docs.oracle.com/javase/specs/jls/se8/html/jls-5.html#jls-5.1.7}{Javaspecifikationen}.)
\end{Slide}

\begin{Slide}{Wrapper-klassen \code{Integer}}\SlideFontSmall
En skiss av klassen \code{Integer} \\ (ligger i paketet \href{http://docs.oracle.com/javase/8/docs/api/java/lang/package-summary.html}{\code{java.lang}} och importeras därmed implicit):

\ifkompendium\vspace{1em}\fi%
\begin{minipage}{0.65\textwidth}
\begin{Code}[numberstyle=,language=Java,backgroundcolor=\color{white},
  frame=none]
public class Integer {
    private int value;

    public static final MIN_VALUE = -2147483648;
    public static final MAX_VALUE = 2147483647;

    public Integer(int value) {
        this.value = value;
    }

    public int intValue() {
        return value;
    }
    ...
}
\end{Code}
\end{minipage}
\begin{minipage}{0.33\textwidth}
\centering\includegraphics[width=0.95\textwidth]{../img/box}
\end{minipage}
Javadoc för klasen \code{Integer} finns här: \\
\SlideFontTiny\url{http://docs.oracle.com/javase/8/docs/api/java/lang/Integer.html}
\end{Slide}





\begin{Slide}{Wrapper-klasser i \code{java.lang}}\SlideFontSmall
\begin{tabular}{l | l}
\Emph{Primitiv typ}                  & \Emph{Inpackad typ}                 \\ \hline

 boolean & Boolean\\
 byte & Byte\\
 short& Short\\
 char & Character\\
 int & Integer\\
 long & Long\\
 float & Float\\
 double & Double\\
\end{tabular}
\end{Slide}


\begin{Slide}{Övning: primitiva versus inpackade typer}
Med papper och penna:
\begin{itemize}
\item Deklarera en variabel med namnet \code{gurka} av den primitiva heltalstypen och initiera den till värdet 42.
\item Deklarera en referensvariabel med namnet  \code{tomat} av den inpackade (''wrappade'') heltalstypen och initiera den till värdet 43.
\item Rita hur det ser ut i minnet.
\end{itemize}
\end{Slide}

\begin{Slide}{Exempel: Lista med heltal utan autoboxning}
\lstinputlisting[language=Java, basicstyle=\small\ttfamily\SlideFontSize{6.7}{8.5},backgroundcolor=\color{white},frame=none
]{../compendium/examples/scalajava/generics/TestIntegerList.java}
\SlideFontTiny Koden finns här: \href{https://github.com/lunduniversity/introprog/tree/master/compendium/examples/scalajava/generics/TestIntegerList.java}{compendium/examples/scalajava/TestIntegerList.java}
\end{Slide}




\begin{Slide}{Specialregler för wrapper-klasser}\SlideFontSmall

\begin{itemize}
\item Om ett \code{int}-värde förekommer där det behövs ett \code{Integer}-objekt, så lägger kompilatorn \Alert{automatiskt} ut kod som skapar ett \code{Integer}-objekt som packar in värdet.
\item Om ett \code{Integer}-objekt förekommer där det behövs ett \code{int}-värde, lägger kompilatorn \Alert{automatiskt} ut kod som anropar metoden \code{intValue()}.
\end{itemize}
Samma gäller mellan alla primitiva typer och dess wrapper-klasser:

\begin{tabular}{r c l}
 {\lstinline!boolean!} &$\Leftrightarrow$& {\lstinline!Boolean!} \\
 {\lstinline!byte!} &$\Leftrightarrow$& {\lstinline!Byte!}\\
 {\lstinline!short!}&$\Leftrightarrow$& {\lstinline!Short!}\\
 {\lstinline!char!} &$\Leftrightarrow$& {\lstinline!Character!}\\
 {\lstinline!int!} &$\Leftrightarrow$& {\lstinline!Integer!}\\
 {\lstinline!long!} &$\Leftrightarrow$& {\lstinline!Long!}\\
 {\lstinline!float!} &$\Leftrightarrow$& {\lstinline!Float!}\\
 {\lstinline!double!} &$\Leftrightarrow$&{\lstinline!Double!}\\
\end{tabular}

\end{Slide}






\begin{Slide}{Exempel: Lista med heltal och autoboxning}
\lstinputlisting[language=Java, basicstyle=\small\ttfamily\SlideFontSize{6}{8}
,backgroundcolor=\color{white},
  frame=none]{../compendium/examples/scalajava/generics/TestIntegerListAutoboxing.java}
\SlideFontTiny Koden finns här: \href{https://github.com/lunduniversity/introprog/tree/master/compendium/examples/scalajava/generics/TestIntegerList.java}{scalajava/generics/TestIntegerListAutoboxing.java}
\end{Slide}

\begin{Slide}{Fallgropar vid autoboxning}
\begin{itemize}
\item Jämförelser med \code{==} och \code{!=} \\
\href{https://github.com/lunduniversity/introprog/blob/master/compendium/examples/scalajava/generics/TestPitfall1.java}
{\SlideFontSmall  compendium/examples/scalajava/generics/TestPitfall1.java}
\item[]
\item Kompilatorn hittar inte förväxlad parameterordning, t.ex. \code{add(pos, item)} i fel ordning: \sout{\code{add(item, pos)}}\\
\href{https://github.com/lunduniversity/introprog/blob/master/compendium/examples/scalajava/generics/TestPitfall2.java}
{\SlideFontSmall compendium/examples/scalajava/generics/TestPitfall2.java}
\end{itemize}
\end{Slide}

\Subsection{Equals i Java}

\begin{Slide}{Referenslikhet eller innehållslikhet i Scala och Java}\SlideFontSmall
Det finns två \Alert{principiellt olika} sorters \Emph{likhet}:
\begin{itemize}
\item \Emph{Referenslikhet} \Eng{reference equality}: två referenser anses lika om de refererar till \Emph{samma instans} i minnet.
\item \Emph{Innehållslikhet}, ä.k. strukturlikhet \Eng{structural equality}: två referenser anses lika om de refererar till objekt med \Emph{samma innehåll}.

\pause

\item I Scala finns flera metoder som testar likhet:
\begin{itemize}\SlideFontSmall
\item metoden \code{eq} testar \Alert{referenslikhet} och \code{r1.eq(r2)} ger \code{true} om \code{r1} och \code{r2} refererar till \Emph{samma} instans.

\item metoden \code{ne} testar referens\textbf{o}likhet och \code{r1.ne(r2)} ger \code{true} om \code{r1} och \code{r2} refererar till \Alert{olika} instanser.

\item metoden \code{==} som anropar metoden \code{equals} som default testar referenslikhet men som \Alert{kan överskuggas} om man \Emph{själv vill bestämma} om det ska vara referenslikhet eller strukturlikhet.
\end{itemize}

\pause

\item Scalas \Emph{standardbibliotek} och \Emph{grundtyperna} \code{Int}, \code{String} etc. testar \Emph{innehållslikhet} genom metoden \code{==}
\pause
\item I \Alert{Java} är det \Alert{annorlunda}: symbolen \code{==} är ingen metod i Java utan \Alert{specialsyntax} som vid instansjämförelse alltid testar \Alert{referenslikhet}, medan metoden \code{equals} kan överskuggas med valfri likhetstest.
\end{itemize}
\end{Slide}

\begin{Slide}{Fallgrop med samlingar: metoden contains kräver implementation av equals}\SlideFontSmall
Antag att vi vill implementera \code{hasVertex()} i klassen \code{Polygon} genom att använda metoden \code{contains} på en lista. Hur gör vi då?
\pause
\begin{Code}[numberstyle=,language=Java]
public boolean hasVertex(int x, int y) {
    return vertices.contains(new Point(x, y)); // FUNKAR INTE om ...
    // ... inte Point har en equals som kollar innehållslikhet
}
\end{Code}
Vi behöver implementera metoden \code{equals(Object obj)} i klassen \code{Point} som kollar innehållslikhet och ersätter den \code{equals} som finns i \code{Object} som kollar referenslikhet, eftersom metoden \code{contains} i klassen \code{ArrayList} anropar \code{equals} när den letar igenom listan efter lika objekt. \\
Se exempel här: \href{https://github.com/lunduniversity/introprog/tree/master/compendium/examples/scalajava/generics/TestPitfall3.java}{compendium/examples/scalajava/generics/TestPitfall3.java} \\


\vspace{1em}{\SlideFontTiny\noindent Det krävs ofta även att man även ersätter  \href{http://stackoverflow.com/questions/27581/what-issues-should-be-considered-when-overriding-equals-and-hashcode-in-java}{\code{hashCode}}, mer om det i fortsättningskursen.}
\end{Slide}


\begin{Slide}{Fullständigt recept för \texttt{equals}}
För den nyfikne inför fortsättningskursen efter jul: 

\vspace{1em}\noindent
Läs om fallgropar för att implementera equals i \Emph{Java} här: \\
\href{http://www.artima.com/lejava/articles/equality.html}{www.artima.com/lejava/articles/equality.html}


\vspace{1em}\noindent
Läs receptet för att implementera equals i \Emph{Scala} här: \\
\href{http://www.artima.com/pins1ed/object-equality.html#28.4}{www.artima.com/pins1ed/object-equality.html\#28.4}
\end{Slide}




\begin{Slide}{Villkorsuttryck i Java}\SlideFontSmall
Det går att använda villkorsuttryck i Java, men med syntax från språket C:
\begin{multicols}{2}
  \noindent\Emph{Scala}
\begin{CodeSmall}[basicstyle=\ttfamily\SlideFontSize{6}{8},backgroundcolor=\color{white},
  frame=none]
var r = math.random()
var answer = if r > 0.5 then 42 else 0
\end{CodeSmall}

\columnbreak

\noindent\Emph{Java}
\begin{CodeSmall}[language=Java,basicstyle=\ttfamily\SlideFontSize{6}{8},backgroundcolor=\color{white},
  frame=none]
double r = Math.random();
int answer = (r > 0.5) ? 42 : 0;
\end{CodeSmall}
\end{multicols}

\end{Slide}




\begin{Slide}{Typtest och typkonvertering}

\begin{multicols}{2}
  \noindent\Emph{Scala}
\begin{CodeSmall}[basicstyle=\small\ttfamily\SlideFontSize{6}{8},backgroundcolor=\color{white},
  frame=none]
var x = "hej"

var isString = x.isInstanceOf[String]

var y = 42

var z = y.asInstanceOf[Double]

\end{CodeSmall}

\columnbreak

\noindent\Emph{Java}
\begin{CodeSmall}[language=Java,basicstyle=\small\ttfamily\SlideFontSize{6}{8},backgroundcolor=\color{white},
  frame=none]
String x = "hej";

boolean isString = x instanceof String;

int y = 42;

double z = (double) y;
\end{CodeSmall}
\end{multicols}

\pause {\SlideFontTiny Detta görs ju i Scala bäst med \code{match} och typmönster!}

\end{Slide}

\begin{Slide}{Regler för överskuggning i Java}
\url{http://docs.oracle.com/javase/tutorial/java/IandI/override.html}
\end{Slide}
  


\begin{Slide}{Fånga undantag i Scala och Java}
Typisk skillnad mellan Scala och Java:\\konstruktioner som är \Emph{uttryck} i Scala är ofta \Alert{satser} i Java.
\begin{multicols}{2}
  \noindent\Emph{Scala}
\begin{CodeSmall}[basicstyle=\ttfamily\SlideFontSize{6}{8},backgroundcolor=\color{white},
  frame=none]
val a = try 2 / 0 catch
  case e: ArithmeticException => 0


val b = try 4 / 2 catch 
  case e: ArithmeticException => 0
\end{CodeSmall}

\columnbreak

\noindent\Emph{Java}
\begin{CodeSmall}[language=Java,basicstyle=\ttfamily\SlideFontSize{6}{8},backgroundcolor=\color{white},
  frame=none]
int a;
try {
    a = 2 / 0;
} catch (ArithmeticException e) {
    a = 0;
}

int b;
try {
    b = 4 / 2;
} catch (ArithmeticException e) {
    b = 0;
}

\end{CodeSmall}
\end{multicols}
\end{Slide}




\begin{Slide}{Gränssnittet \texttt{List} i Java}\SlideFontSmall
\begin{itemize}
\item I Java finns inte \code{trait} och inmixning.

\item I stället finns \jcode{interface} som liknar \code{trait} men är mer begränsad vad gäller vilka medlemmar som får finnas.

\item Man kan bara göra \code{extends} på exakt en annan klass, men man kan i Java göra \jcode{implements} på flera \jcode{interface}.%(Jämför Scalas \code{with} på \code{trait}s)

\item Exempel:
\begin{Code}[language=Java,backgroundcolor=\color{white},
  frame=none]
public class ArrayList<E> extends AbstractList<E>
    implements List<E>, RandomAccess, Cloneable, java.io.Serializable
\end{Code}

\item Att implementera ett gränssnitt innebär att uppfylla ett kontrakt som utlovar att vissa speciella metoder finns tillgängliga.

\item Gränssninttet \code{List} uppfylls av en av dess implementationer \code{ArrayList} \\

på liknande sätt i Scala där gränssnittet \code{Seq} uppfylls av \code{Vector} etc.

\item[] \jcode{List<String> xs = new ArrayList<String>();}

\item Liknande exempel från övninegn Hangman: \\\jcode{Set<Character> found = new HashSet<Character>();}

\item En Scala-trait med enbart abstrakta medlemmar kompileras till ett Java-interface i JVM bytekode.

\item Mer om gränssnitt i Java i fördjupningskursen.

\end{itemize}
\end{Slide}

\begin{Slide}{Det går inte att skapa generisk Array i Java}\SlideFontTiny
\begin{itemize}
\item I Java kan man \Alert{inte} skapa en primitiv array av godtycklig typ enligt generisk typparameter: \sout{\code{T[] xs = new T[42]}}

\item Man måste istället skapa en array av den mest generella referenstypen: \\
\code{Object[] xs = new Object[42]} \\
och sedan typtesta och typkonvertera under körtid; se t.ex. implementationen av \code{ArrayList} på rad 119: \href{http://developer.classpath.org/doc/java/util/ArrayList-source.html}{http://developer.classpath.org/doc/java/util/ArrayList-source.html}

\item[]
\pause
\item Detta går faktiskt att göra i Scala med hjälp av \code{reflect.ClassTag} \pause så här: \\
\begin{REPLsmall}
scala> def fyll[T](n: Int, x: T): Array[T] = Array.fill(n)(x)
-- Error:
1 |def fyll[T](n: Int, x: T): Array[T] = Array.fill(n)(x)
  |                                                      ^
  |  No ClassTag available for T

scala> def fyll[T: reflect.ClassTag](n: Int, x: T): Array[T] = Array.fill(n)(x)

scala> fyll(42, "hej")
res2: Array[String] = Array(hej, hej, hej, hej, hej, hej, hej, hej, hej, hej, hej, hej, hej, hej, hej, hej, hej, hej, hej, hej, hej, hej, hej, hej, hej, hej, hej, hej, hej, hej, hej, hej, hej, hej, hej, hej, hej, hej, hej, hej, hej, hej)

\end{REPLsmall}


\end{itemize}


\end{Slide}

\begin{Slide}{Jämföra strängar i Java}\SlideFontTiny
\begin{itemize}
\item I Java kan man \Alert{inte} jämföra strängar med operatorerna \code{<}, \code{<=}, \code{>}, och \code{>=}

\item Dessutom ger operatorerna \code{==} och \code{!=} \emph{inte} innehålls(o)likhet utan \Alert{referens(o)likhet} \code{:(}

\item Istället \Alert{måste} man i Java använda metoderna \code{equals} och \code{compareTo}
\\Dessa fungerar även i Scala eftersom strängar i Scala och Java är av samma typ, nämligen \code{java.lang.String}.
\pause
\item \code{s1.compareTo(s2)} ger ett heltal som är:
\begin{itemize}\SlideFontTiny
\item \code{0} om s1 och s2 har samma innehåll
\item \Alert{negativt} om s1 < s2 i lexikografisk mening, alltså s1 ska sorteras \Alert{före}
\item \Emph{positivt} om s1 > s2 i lexikografisk mening, alltså s1 ska sorteras \Emph{efter}
\end{itemize}

\pause
\item Undersök följande:
\begin{REPL}
scala> new String("hej") eq new String("hej") // motsvarar == i Java
scala> "hej".equals("hej")                    // samma som == i Scala
scala> "hej".compareTo("hej")
scala> "hej".compareTo("HEJ")         // alla stora är 'före' alla små
scala> "HEJ".compareTo("hej")
\end{REPL}

\item Observera att \code{"hej" == "hej"} i Java ibland evalueras till \code{true}. Detta beror inte på att \code{==} testar innehållslikhet utan på något som kallas stränginternalisering (eng. \textit{string interning}). \\
Mer om stränginternalisering för den nyfikna: \\
\href{https://stackoverflow.com/questions/10578984/what-is-java-string-interning}{https://stackoverflow.com/questions/10578984/what-is-java-string-interning}

\end{itemize}

\href{http://docs.oracle.com/javase/8/docs/api/java/lang/String.html#compareTo-java.lang.String-}{docs.oracle.com/javase/8/docs/api/java/lang/String.html\#compareTo-java.lang.String-}
\end{Slide}


\begin{Slide}{Jämföra strängar i Java: exempel}\SlideFontSmall
Vad skriver detta Java-program ut?
\javainputlisting{../compendium/examples/StringEqTest.java}
\pause
\begin{REPL}
$ javac StringEqTest.java
$ java StringEqTest
false
true
0
\end{REPL}
\end{Slide}


% \Subsection{StringBuilder}

% \begin{Slide}{Förändringsbar eller oföränderlig?}
% \begin{itemize}
% \item Om den underliggande \Emph{oföränderliga} datastrukturen är \Alert{smart} implementerad så att den \Emph{återanvänder redan allokerade objekt} -- vilket ju är ofarligt eftersom de aldrig kommer att ändras -- så är oföränderlighet \Emph{minst lika snabbt} som förändring på plats.

% \item Det är först när man gör \Alert{väldigt många} upprepade ändringar på, för datastrukturen ogynnsam plats, som det blir långsamt.

% \item Hur många är ''väldigt många''?  \\ \pause $\rightarrow$ Det ska vi undersöka nu.

% \end{itemize}
% \end{Slide}

% \begin{Slide}{String eller StringBuilder?}
% \begin{itemize}
% \item Strängar i JVM är \Emph{oföränderliga}.

% \item Implementationen av sekvensdatastrukturen \code{java.lang.String} är \Alert{mycket effektivt} implementerad, där \Emph{redan allokerade objekt ofta kan återanvänds}.

% \item \Alert{MEN} väldigt många tillägg på slutet blir långsamt. Därför finns den föränderliga \code{StringBuilder} metoden \code{append} som är snabbare än \code{+} på \code{String} vid mycket stora strängar. (Hur stora?)

% \pause
% \item Undersök dokumentationen för \code{StringBuilder} här:
% {\SlideFontTiny\url{https://docs.oracle.com/javase/8/docs/api/java/lang/StringBuilder.html}}

% \pause
% \item För vilka teckensekvensalgoritmer är det lönt att använda \code{StringBuilder}? \\
% \pause $\rightarrow$ Det ska vi undersöka nu.

% \end{itemize}
% \end{Slide}

% \ifkompendium\else
% \begin{SlideExtra}{Timer}\SlideFontSmall
% \setlength{\leftmargini}{0pt}
% \begin{itemize}
% \item \href{https://docs.oracle.com/javase/8/docs/api/java/lang/System.html#currentTimeMillis--}{\code{System.currentTimeMillis}} ger tiden i millisekunder sedan januari 1970.

% \item Med \code|Timer.measure{ xxx }| nedan kan man mäta tiden det tar för xxx.

% \item Ett par \code{(elapsedMillis, result)} returneras som innehåller tiden det tar att köra blocket, samt resultatet av blocket.
% \end{itemize}
% \vspace{0em}\scalainputlisting[numbers=left,numberstyle=,basicstyle=\fontsize{6.5}{8}\ttfamily\selectfont]{../compendium/examples/workspace/w05-seqalg/src/Timer.scala}
% \end{SlideExtra}


% \begin{SlideExtra}{NanananananananaNanananananananaBatman}\SlideFontTiny
% Prova denna kod:
% \href{https://github.com/lunduniversity/introprog/blob/master/compendium/examples/workspace/w05-seqalg/src/NanananananananaNanananananananaBatman.scala}{compendium/examples/workspace/w05-seqalg/src/\\NanananananananaNanananananananaBatman.scala} \\

% medan du lyssnar till: \href{https://www.youtube.com/watch?v=oDc-1zfffMw}{www.youtube.com/watch?v=oDc-1zfffMw}

% \vspace{-0.7em}\scalainputlisting[numbers=left,numberstyle=,basicstyle=\ttfamily\SlideFontSize{4}{5}]{../compendium/examples/workspace/w05-seqalg/src/NanananananananaNanananananananaBatman.scala}
% \end{SlideExtra}
% \fi



