%!TEX encoding = UTF-8 Unicode
%!TEX root = ../lect-w02.tex


\ifkompendium

\noindent Ett program innehåller satser och uttryck. En \Emph{kontrollstruktur}, t.ex. \code{while}, styr i vilken \Alert{ordning} satser och uttryck exekveras. Data kan placeras i en \Emph{datastruktur}, t.ex. en \code{Vector}, så att man senare kan komma åt data igen.    
\fi


\ifkompendium\else
\begin{SlideExtra}{Från förra veckan: \texttt{val} \texttt{var} \texttt{def}}
  Vad är det för skillnad på (och likhet mellan) \code{val}, \code{var} och \code{def}? \pause
  \begin{itemize}
    \item Med \code{val} deklareras en variabel som tilldelas ett värde vid initialisering och som sedan \Emph{aldrig ändras}.
    \item Med \code{var} deklareras en variabel som tilldelas ett värde vid initialisering och som sedan \Alert{kan uppdateras} hur många gånger som helst med hjälp av tilldelningssatser.
    \item Med \code{def} deklareras en funktion som körs vid \Emph{varje anrop}
  \end{itemize}
  \pause\vspace{1em} En \Alert{konstighet i REPL}: 
  \begin{itemize}
  \item Man kan i REPL \Emph{deklarera} variabler och funktioner med \Emph{samma namn} \Alert{flera gånger} på samma nivå. 
  \item Detta ger i vanliga, fristående program \Emph{kompileringsfel}. 
  \end{itemize}
\end{SlideExtra}

\begin{SlideExtra}{Från förra veckan: \texttt{if} \texttt{then} \texttt{else}}\SlideFontSmall
\Emph{Spelets regler:} Singla slant. Om du får krona har du vunnit. \\[0.5em]
Förenkla dessa if-uttryck:
\begin{Code}
def singlaSlant: Boolean = if math.random() < 0.5 then true else false

def harVunnit(slantÄrKrona: Boolean): Boolean = 
  if slantÄrKrona == true then true else false
\end{Code}
\pause förenkling av ''kaka-på-kaka'': \textbf{if uttryck then true else false}
\begin{Code}
def singlaSlant: Boolean = math.random() < 0.5
\end{Code}
\pause förenkling av ''kaka-på-kaka'':  \textbf{uttryck == true}
\begin{Code}
def harVunnit(slantÄrKrona: Boolean): Boolean = slantÄrKrona
\end{Code}
\pause Hur ska vi ändra \code{harVunnit} om vi ändrar reglerna till att Klave ger vinst? \\
\pause Repetera även \Alert{de Morgans lagar} från vecka 1 så du har koll på dem.
\end{SlideExtra}

\fi 

\Subsection{Datastrukturer}

\begin{Slide}{Vad är en datastruktur?}\SlideFontSmall
\begin{itemize}
\item En \href{https://sv.wikipedia.org/wiki/Datastruktur}{datastruktur} är en struktur för organisering av data som...
\begin{itemize}\SlideFontTiny
\item kan innehålla \Alert{många} element,
\item kan \Emph{refereras} till som en \Alert{helhet}, och
\item ger möjlighet att \Emph{komma åt} \Alert{enskilda element}.
\end{itemize}

\item En \Emph{samling} \Eng{collection} är en datastruktur som kan innehålla många element av \Alert{samma typ}.

\item Exempel på olika samlingar där elementen är organiserade på olika vis: \\
\vspace{0.5em}
\begin{tabular}{l c}
\Emph{Sekvens} & \includegraphics[width=5cm]{../img/list.pdf} \\
\Emph{Träd}  & \includegraphics[width=2.2cm]{../img/tree.pdf} \\
\Emph{Graf}  & \includegraphics[width=2.2cm]{../img/graph.pdf} \\
\end{tabular}
\end{itemize}
{
\SlideFontTiny \vspace{1em }\hskip2em
Mer om sekvenser \& träd i \href{http://cs.lth.se/edaa01vt}{EDAA01 pfk}.
Mer om träd, grafer i \href{http://cs.lth.se/edaa40}{Diskreta strukturer.}
}

\end{Slide}

\begin{Slide}{Några samlingar i \texttt{scala.collection}}\SlideFontSmall
\SlideOnly{\setlength{\leftmargini}{0pt}}
\begin{itemize}
\item En \Emph{samling} \Eng{collection} är en datastruktur som kan innehålla många element av \Alert{samma typ}.
\item En \Emph{sekvens} \Eng{sequence} är en samling där alla element är ordnade.

\item Exempel på \Emph{färdiga samlingar} i Scalas standardbibliotek där elementen är organiserade  internt på \Alert{olika} vis så att samlingen får olika egenskaper som passar \Alert{olika användningsområden}:
\begin{itemize}\SlideFontTiny
\item \texttt{scala.collection.immutable.\Emph{Vector}}, sekvens med snabb access \Alert{överallt}.
\item \texttt{scala.collection.immutable.\Emph{List}}, sekvens med snabb access \Alert{i början}.
\item \texttt{scala.collection.immutable.\Emph{Set}}, \texttt{scala.collection.\Alert{mutable}.\Emph{Set}}, mängd med unika element; ej i sekvens men snabb innehållstest.
\item \texttt{scala.collection.immutable.\Emph{Map}}, \texttt{scala.collection.\Alert{mutable}.\Emph{Map}}, mängd med par av nyckel \& tillhörande värde, snabb access via nyckel.
\item \texttt{scala.collection.\Alert{mutable}.\Emph{ArrayBuffer}}, förändringsbar sekvens kan ändra storlek.
\item \texttt{scala.\Emph{Array}}, förändringsbar sekvens som \Alert{inte} kan ändra storlek. Alla element är lagrade efter varandra i minnet: snabbast access av alla samlingar, men har speciella begränsningar.
\end{itemize}
\end{itemize}
\end{Slide}


\begin{Slide}{Olika strukturer för att hantera data}
\begin{itemize}\SlideFontSmall
\item \Emph{Tupel} \Eng{tuple}
\begin{itemize}\SlideFontTiny
\item samla flera datavärden t.ex. \code{(1, "hej", true)} i element \Emph{\code{_1}}, \Emph{\code{_2}}, \Emph{\code{_3}} 
\item elementen kan vara av \Alert{olika} typ
\end{itemize}
\item \Emph{Enumeration} (även kallad \emph{uppräkning}) \Eng{enumeration}
\begin{itemize}\SlideFontTiny
\item Namnge uppräknade värden t.ex. \code+enum Color { case Red, Black }+
\item Värdena har ordningsnummer och är alla av \Alert{samma} typ (här \code{Color})
\end{itemize}
\item \Emph{Klass} \Eng{class}
\begin{itemize}\SlideFontTiny
\item samlar data i \Emph{attribut} med (väl valda!) namn
\item attributen kan vara av \Alert{olika} typ
\item definierar även \Emph{metoder} som använder attributen \\ (kallas även \Emph{operationer} på data)
\end{itemize}

\item \Emph{Färdig samling}
  \begin{itemize}
  \item speciella klasser som samlar data i element av \Alert{samma} typ
  \item exempel: \code{scala.collection.immutable.}\Emph{\code{Vector}}
  \item har ofta \emph{många} färdiga \Emph{bra-att-ha-metoder}, \\ se snabbreferensen \url{http://cs.lth.se/pgk/quickref}
  \end{itemize}

\item \Emph{Egenimplementerade samlingar}
  \begin{itemize}
  \item $\rightarrow$ fördjupningskurs
  \end{itemize}

\end{itemize}
\end{Slide}





\begin{Slide}{Vad är en vektor?}\SlideFontSmall
En \Emph{vektor}\footnote{Vektor kallas ibland på svenska även \href{https://sv.wikipedia.org/wiki/F\%C3\%A4lt_\%28datastruktur\%29}{fält}, men det skapar stor förvirring eftersom det engelska ordet \emph{field} ofta används för \emph{attribut} (förklaras senare).}
\Eng{vector} är en \Emph{sekvens} som är \Alert{snabb} att \Emph{indexera} i.
Åtkomst av element i en sekvens som t.ex. heter \code{xs} sker i Scala med \code{xs.apply(platsnummer)}:

\begin{REPL}
scala> val heltal = Vector(42, 13, -1, 0, 1)
val heltal: scala.collection.immutable.Vector[Int] = Vector(42, 13, -1, 0, 1)

scala> heltal.apply(0)   // platsnummer räknas från noll
val res0: Int = 42

scala> heltal(1)         // man kan i Scala skippa .apply före (
val res1: Int = 13

scala> heltal(5)         // ger körtidsfel då sjätte platsen inte finns
java.lang.IndexOutOfBoundsException: 5
  at scala.collection.immutable.Vector.checkRangeConvert(Vector.scala:132)
\end{REPL}
Utelämnar du \code{.apply} så skapar kompilatorn automatiskt ett anrop av \code{apply}.
\end{Slide}

\begin{Slide}{En konceptuell bild av en vektor}

\begin{REPLnonum}
scala> val heltal = Vector(42, 13, -1, 0, 1)

scala> heltal(0)
val res0: Int = 42
\end{REPLnonum}

\begin{tikzpicture}[font=\ttfamily]
\matrix [matrix of nodes, row sep=0, column 2/.style={nodes={rectangle,draw,minimum width=3em}}] (var) at (0cm, 2.8cm)
{
heltal   &  \makebox(16,12){ }\\
};
\matrix [matrix of nodes, draw=black,row sep=0, column 2/.style={nodes={rectangle,draw,minimum width=4em}}] (vec) at (4cm, 1cm)
{
\textit{plats} &  \\
0   &  \makebox(16,12){42}\\
1   &  \makebox(16,12){13}\\
2   &  \makebox(16,12){-1}\\
3   &  \makebox(16,12){0}\\
4   &  \makebox(16,12){1}\\
};
\filldraw[black] (0.7cm,2.8cm) circle (3pt) node[] (ref) {};
 \draw [arrow] (ref) -- (vec);
\end{tikzpicture}

%\vspace{1em} Elementen ligger på rad någonstans i minnet.
\end{Slide}



\begin{Slide}{En samling strängar}

\begin{itemize}
\item En vektor kan lagra \Emph{många} värden av samma typ.
\item Elementen kan vara till exempel heltal eller strängar.
\item Eller faktiskt vad som helst. (En s.k. \emph{generisk} samling.)
\end{itemize}

\begin{REPL}
scala> val grönsaker = Vector("gurka","tomat","paprika","selleri")
grönsaker: scala.collection.immutable.Vector[String] =
  Vector(gurka, tomat, paprika, selleri)

scala> val g = grönsaker(1)
val g: String = tomat

scala> val xs = Vector(42, "gurka", true, 42.0)
val xs: Vector[Matchable] = Vector(42, gurka, true, 42.0)
\end{REPL}
\SlideFontSmall Notera typen \code{Matchable} som betyder ''\Emph{nästan vilken typ som helst}''\\
%\footnote{
%(\code{Matchable} liknar \code{Object} i Java/C\# men är \Alert{mer generell}; 
(Mer om \code{Matchable} senare.)
%}
\end{Slide}



\Subsection{Kontrollstrukturer}


\begin{Slide}{Vad är en kontrollstruktur?}
\begin{itemize}
\item En \Emph{kontrollstruktur} påverkar i vilken ordning (sekvens) satser exekveras och uttryck evalueras.
\begin{itemize}
\item[] Exempel på \Emph{inbyggda} kontrollstrukturer:
\\\vspace{0.5em}\code{for}-\code{do}-sats \\ \code{while}-\code{do}-sats \\ \code{for}-\code{yield}-uttryck
\end{itemize}

\item[]

\item I Scala kan man definiera \Alert{egna} kontrollstrukturer.
\begin{itemize}
\item[] Exempel: \code{upprepa} som du använt i Kojo
\\\vspace{0.5em}\code|upprepa(4){fram; höger}|
\end{itemize}
\end{itemize}
\end{Slide}

\ifkompendium\else
\begin{SlideExtra}{Mitt första program: en oändlig loop på ABC80}
\begin{minipage}{0.8\textwidth}
\begin{verbatim}
10 print "hej"
20 goto 10
\end{verbatim}
\includegraphics[width=0.8\textwidth]{../img/abc80.jpg}
\end{minipage}%
\begin{minipage}{0.2\textwidth}
\pause
\begin{verbatim}
hej
hej
hej
hej
hej
hej
hej
hej
hej
hej
hej
hej
<Ctrl+C>
\end{verbatim}
\end{minipage}
\end{SlideExtra}
\fi

\begin{Slide}{Loopa genom elementen i en vektor}
En \code{for}-\code{do}-\Emph{sats} som skriver ut alla element i en vektor:
\begin{REPL}
scala> val grönsaker = Vector("gurka","tomat","paprika","selleri")

scala> for g <- grönsaker do println(g)
gurka
tomat
paprika
selleri

\end{REPL}
\code{for ... do ...} gör så att följande händer:
\begin{itemize}
  \item Plocka ut \Emph{varje element} ur samlingen. 
  \item \Emph{Namnet} före pilen (här \code{g}) \Alert{refererar} till ett \Emph{nytt} värde för varje runda i loopen.
  \item Detta namn motsvarar en \Emph{lokal} \code{val}-variabel.
\end{itemize}
\end{Slide}


\begin{Slide}{Bygg ny samling från befintlig med for-yield-uttryck}
Ett \code{for}-\code{yield}-\Emph{uttryck} som \Emph{skapar en \Alert{ny} samling}.

\begin{Code}[basicstyle=\ttfamily\fontsize{12}{14}\selectfont]
for g <- grönsaker yield s"god $g"
\end{Code}

\begin{REPL}
scala> val grönsaker = Vector("gurka","tomat","paprika","selleri")

scala> val åsikter = for (g <- grönsaker) yield s"god $g"
val åsikter: Vector[String] =
  Vector(god gurka, god tomat, god paprika, god selleri)
\end{REPL}

\end{Slide}


\begin{Slide}{Samlingen \code{Range} håller reda på intervall}
\begin{itemize}
\item Med en \code{Range(start, slut)} kan du skapa ett \Emph{intervall}: \\ från och med \code{start} till (men inte med) \code{slut}
\end{itemize}

\begin{REPLnonum}
scala> Range(0, 42)
val res0: Range =
  Range(0, 1, 2, 3, 4, 5, 6, 7, 8, 9, 10, 11, 12, 13, 14,
    15, 16, 17, 18, 19, 20, 21, 22, 23, 24, 25, 26, 27, 28,
    29, 30, 31, 32, 33, 34, 35, 36, 37, 38, 39, 40, 41)
\end{REPLnonum}

\begin{itemize}
\item Men alla värden däremellan skapas inte förrän de behövs:
\end{itemize}

\begin{REPL}
scala> val jättestortIntervall = Range(0, Int.MaxValue)
val jättestortIntervall: Range = Range(0, 1, 2, 3, 4, 5, ...

scala> jättestortIntervall.end
val res1: Int = 2147483647

scala> jättestortIntervall.toVector
java.lang.OutOfMemoryError: GC overhead limit exceeded
\end{REPL}

\end{Slide}

\begin{Slide}{Loopa med Range}
\code{Range} används i for-loopar för att hålla reda på antalet rundor.
\begin{REPLnonum}
scala> for i <- Range(0, 6) do print(s" gurka $i")
 gurka 0 gurka 1 gurka 2 gurka 3 gurka 4 gurka 5
\end{REPLnonum}
Du kan skapa en \code{Range} med \code{until} efter ett heltal:
\begin{REPLnonum}
scala> 1 until 7
val res1: Range =
  Range(1, 2, 3, 4, 5, 6)

scala> for i <- 1 until 7 do print(s" tomat $i")
 tomat 1 tomat 2 tomat 3 tomat 4 tomat 5 tomat 6

\end{REPLnonum}
\end{Slide}

\begin{Slide}{Loopa med Range skapad med \texttt{to}}

Med \code{to} efter ett heltal får du en \code{Range} till och \Emph{med} sista:
\begin{REPLnonum}
scala> 1 to 6
res2: Range.Inclusive =
  Range(1, 2, 3, 4, 5, 6)

scala> for i <- 1 to 6 do print(" gurka " + i)
 gurka 1 gurka 2 gurka 3 gurka 4 gurka 5 gurka 6

\end{REPLnonum}


\end{Slide}



\begin{Slide}{Vad är en \code{Array}?}


\begin{itemize}
\item En \href{https://en.wikipedia.org/wiki/Array_data_structure}{\code{Array}} liknar en \code{Vector} men har en särställning i JVM:
\begin{itemize}
\item Lagras som en sekvens i minnet på efterföljande adresser.
\item \Emph{Fördel}: snabbaste samlingen för element-access i JVM.
\item Men det finns en hel del \Alert{nackdelar} som vi ska se senare.
\end{itemize}

\end{itemize}

\begin{REPLnonum}
scala> val heltal = Array(42, 13, -1, 0 , 1)
\end{REPLnonum}

\begin{tikzpicture}[font=\ttfamily,scale=0.75, every node/.style={scale=0.75}]
\matrix [matrix of nodes, row sep=0, column 2/.style={nodes={rectangle,draw,minimum width=3em}}] (var) at (0cm, 2.8cm)
{
heltal   &  \makebox(16,12){ }\\
};
\matrix [matrix of nodes, draw=black,row sep=0, column 2/.style={nodes={rectangle,draw,minimum width=4em}}] (vec) at (4cm, 1cm)
{
\textit{plats} &  \\
0   &  \makebox(16,12){42}\\
1   &  \makebox(16,12){13}\\
2   &  \makebox(16,12){-1}\\
3   &  \makebox(16,12){0}\\
4   &  \makebox(16,12){1}\\
};
\filldraw[black] (0.7cm,2.8cm) circle (3pt) node[] (ref) {};
 \draw [arrow] (ref) -- (vec);
\end{tikzpicture}
\end{Slide}

\begin{Slide}{Några likheter \& skillnader mellan \texttt{Vector} och \texttt{Array}}\SlideFontSmall
\begin{multicols}{2}
\begin{REPLnonum}
scala> val xs = Vector(1,2,3)
\end{REPLnonum}

\columnbreak

\begin{REPLnonum}
scala> val xs = Array(1,2,3)
\end{REPLnonum}
\end{multicols}


Några likheter mellan \texttt{Vector} och \texttt{Array}
\begin{itemize}
\item Båda är samlingar som kan innehålla många element.

\item Med båda kan man snabbt accessa vilket element som helst: \code{xs(2)}
\end{itemize}
Några viktiga skillnader:

\vspace{-0.5em}\begin{multicols}{2}
\Emph{Vector}
\begin{itemize}
\item Är \Emph{oföränderlig}: du kan lita på att elementreferenserna aldrig någonsin kommer att ändras.

\item Är \Emph{snabb på att skapa en delvis förändrad kopia}, t.ex. tillägg/borttagning/uppdatering mitt i sekvensen.

\end{itemize}


\columnbreak

\Alert{Array}
\begin{itemize}
\item Är \Alert{föränderlig}: \code{xs(2) = 42}

\item Är \Alert{snabb} om man bara vill läsa eller skriva på befintliga platser.

\item Är \Alert{långsam} om man vill lägga till eller ta bort element mitt i sekvensen.
\item Kan \Alert{ej} ändra storlek.

\end{itemize}
\end{multicols}
\end{Slide}



\Subsection{Fristående applikation}

\begin{Slide}{Kompilering i terminalen}
  När du ska skriva kod i en editor, kompilera i terminalen och köra ditt program som en \Emph{fristående applikation}, så behövs: 
  \begin{itemize}
    \item En editor: \Emph{VS Code} med tillägget \Alert{Scala (Metals)}
    \item Körmiljön \Emph{OpenJDK} 
    \item Kommandoverktyg i terminalen: \Alert{\texttt{scala}} eller \texttt{scala-cli}
    \item Installera så här: \url{http://cs.lth.se/pgk/verktyg}
    \item Läs mer i Appendix C.
    \item Tips om du kör Windows: installera %\href{https://docs.microsoft.com/en-us/windows/terminal/get-started}
    {nya Windows Terminal}

  \end{itemize}
    Få hjälp i kanalerna \texttt{\#installationskrångel} och \texttt{\#frågor-och-svar} på vår Discord-server eller fråga handledare på resurstid.
  
\end{Slide}

\begin{Slide}{Scala Command Line Interface (CLI)}
\begin{itemize}
\item Utvecklingen av ett nytt kommandogränssnitt \Eng{Command Line Interface (CLI)} för Scala startades 2022 i ett öppenkällkodsprojekt som leds av Virtuslab. 
\item I augusti 2024 blev \Alert{\texttt{scala-cli}} det nya \Emph{\texttt{scala}}%\footnote{I skrivande stund så har skiftet ännu inte skett. När det sker kan du ersätta alla förekomster av \code{scala-cli} med det kortare \code{scala}.}
\item Du kan nu ersätta \code{scala-cli} med \code{scala}
\item Läs mer i Appendix C och F, samt här: \url{https://scala-cli.virtuslab.org/}
\item Se vad Scala CLI kan göra med underkommandot \texttt{help}
\begin{REPLnonum}
  scala help
\end{REPLnonum}
\end{itemize}
\end{Slide}

\begin{Slide}{Ett minimalt fristående program i Scala}\SlideFontSmall
Spara nedan Scala-kod i filen \code{hej.scala}:
\begin{Code}
@main def run = println("Hej Scala!")
\end{Code}

Kompilera och kör i terminalen:
\begin{REPL}
> scala run hej.scala 
Compiling project (Scala 3.5.0, JVM (21))
Compiled project (Scala 3.5.0, JVM (21))
Hej Scala! 
\end{REPL}

Innan körning kompileras dina kodfiler automatiskt vid behov. Du kan se maskinkoden i en underkatalog i till katalogen \texttt{.scala-build}: 
\begin{REPL}
> ls .scala-build/*/classes/main
'hej$package.class'  'hej$package$.class'  'hej$package.tasty'   run.class   run.tasty
\end{REPL}
\end{Slide}


\begin{Slide}{Loopa genom en samling med en \texttt{while}-sats}
\begin{REPLnonum}
scala> val xs = Vector("Hej","på","dej","!!!")
val xs: Vector[String] =
  Vector(Hej, på, dej, !!!)

scala> xs.size
val res0: Int = 4

scala> var i = 0
val i: Int = 0

scala> while i < xs.size do { println(xs(i)); i = i + 1 }
Hej
på
dej
!!!
\end{REPLnonum}
\end{Slide}


\begin{Slide}{Strängargument till i ett program med primitiv main}
Skriv och spara nedan kod i filen \texttt{helloargs1scala}
\begin{REPLnonum}
> code helloargs1.scala
\end{REPLnonum}
\begin{Code}
object HelloScalaArgs:
  def main(args: Array[String]): Unit = // en primitiv main-metod utan @main
    var i = 0
    while i < args.size do
      println(args(i))
      i = i + 1
\end{Code}
En primitiv \code{main}-metod har ej \code{@main} och måste vara i ett objekt. \\
Kompilera och kör med programargument efter \code{--}
\begin{REPL}
> scala run helloargs1.scala -- morot gurka tomat
morot
gurka
tomat
\end{REPL}
\end{Slide}

\begin{Slide}{Typsäkra argument till i ett program med @main}
  \SlideFontSmall
Skriv och spara nedan kod i filen \texttt{helloargs2.scala}
\begin{REPLnonum}
> code helloargs2.scala
\end{REPLnonum}
\begin{Code}
@main def hej(heltal: Int, resten: String*): Unit =  // notera * efter String
  for i <- 0 until heltal do println(resten(i))
\end{Code}
Med \code{@main} behövs inget objekt.\\
Kompilera och kör med programargument efter \code{--}
\begin{REPL}
> scala run helloargs2.scala -- 2 morot gurka tomat
morot
gurka
> scala run helloargs2.scala -- aj morot gurka tomat
Illegal command line: java.lang.NumberFormatException: For input string: "aj"
\end{REPL}
Med \code{@main} genereras automatiskt en primitiv main som kollar att argumenten har rätt typ.
\end{Slide}


\begin{Slide}{För kännedom: Scala-\textbf{skript}}
\begin{itemize}
  \SlideFontSmall
  \item 
Scala-kod kan köras som ett \Emph{skript}.
\item Ett skript finns i en enda fristående fil med ändelsen \code{.sc}
\item Skript behöver inget huvudprogram. 
\item Skript har automatiskt alla programargument i strängsekvensen \code{args}

\begin{Code}
// spara detta i filen 'myscript.sc'
println("Hej alla mina argument:")
for a <- args do println(s"Hej: $a") 
\end{Code}
\begin{REPLnonum}
> scala run myscript.sc -- ett två tre
Hej alla mina argument:
Hej: ett
Hej: två
Hej: tre
\end{REPLnonum}
\end{itemize}
\pause {\SlideFontTiny Ett Scala-skript kan ej anropa andra skript utan speciella åtgärder, se detaljer här: \url{https://scala-cli.virtuslab.org/docs/guides/scripting/scripts/}
}
\end{Slide}



\Subsection{Algoritmer: stegvisa lösningar}

\begin{Slide}{Vad är en algoritm?}
En \href{https://sv.wikipedia.org/wiki/Algoritm}{algoritm} är en sekvens av instruktioner som beskriver hur man löser ett problem.

\vspace{1em}\Emph{Exempel}:
\begin{itemize}
\item	 baka en kaka
\pause\item räkna ut din pensionsprognos
\pause\item köra bil
\pause\item kolla om highscore i ett spel
\end{itemize}
\ifkompendium\else
\begin{tikzpicture}[overlay]
\node[xshift=0.85\textwidth, scale=2.0] { \includegraphics[width=0.25\textwidth]{../img/highscore}};
\end{tikzpicture}
\fi
\end{Slide}


\ifkompendium\else
\begin{SlideExtra}{Algoritm-exempel: HIGHSCORE}
\Emph{Problem}: Kolla om high-score i ett spel \\ \vspace{1em}

\Emph{Varför?} \pause Så att de som spelar uppmuntras att spela mer :) \\ \vspace{1em}

\Emph{Algoritm:}\pause
\begin{enumerate}
\item $points$ $\leftarrow$ poängen efter senaste spelet
\item $highscore$ $\leftarrow$ bästa resultatet innan senaste spelet
\item \Key{om} $points$ är större än $highscore$ \Key{så}
\begin{enumerate}[ ~~]
\item  Skriv ''Försök igen!''
\end{enumerate}
\Key{annars}
\begin{enumerate}[ ~~]
\item  Skriv ''Grattis!''
\end{enumerate}
\end{enumerate}
\pause
\scriptsize \Alert{Hittar du buggen?}

\pause Utskriften blir fel; vänd villkor eller byt plats på grenarna i if-satsen
\end{SlideExtra}
\fi

\ifkompendium\else
\begin{SlideExtra}{HIGHSCORE implementerad i Scala}
\begin{Code}
import scala.io.StdIn.readLine

@main 
def run = 
  val points = readLine("Hur många poäng fick du?").toInt
  val highscore = readLine("Vad var highscore före senaste spelet?").toInt
  val msg = if points > highscore then "GRATTIS!" else "Försök igen!"
  println(msg)
\end{Code}
\SlideFontSmall %(Mer om \code{import} senare.)\\
\pause
Är det en bugg eller en feature att det står\\ \texttt{points > highscore} \\ och inte \\ \texttt{points >= highscore} \\ ?
\pause Man får ej GRATTIS om poäng == highscore vilket är tråkigt :)
\end{SlideExtra}



\fi

\begin{Slide}{Algoritmexempel: N-FAKULTET}
\begin{algorithm}[H]
 \SetKwInOut{Input}{Indata}\SetKwInOut{Output}{Utdata}

 \Input{heltalet $n$}
 \Output{produkten av de första $n$ positiva heltalen}
 ~\\
 $prod \leftarrow 1$ \\
 $i \leftarrow 2$  \\
 \While{$i \leq n$}{
  $prod \leftarrow prod * i$\\
  $i \leftarrow i + 1$
 }
 $prod$
\end{algorithm}
\pause\vspace{1em}
\begin{itemize}\SlideFontSmall
\item Vad händer om $n$ är noll?
\item Vad händer om $n$ är ett?
\item Vad händer om $n$ är två?
\item Vad händer om $n$ är tre?
\end{itemize}
\end{Slide}

\begin{Slide}{Algoritmexempel: MIN}
\begin{algorithm}[H]
 \SetKwInOut{Input}{Indata}\SetKwInOut{Output}{Utdata}

 \Input{Array $args$ med strängar som alla innehåller heltal}
 \Output{minsta heltalet }
 ~\\
 $min \leftarrow$ det största heltalet som kan uppkomma  \\
 $n \leftarrow $ antalet heltal \\
 $i \leftarrow 0$ \\
 \While{$i < n$}{
   $x \leftarrow args(i).toInt$ \\
   \If{( x < $min$)}{$min \leftarrow x$}
   $i \leftarrow i + 1$
 }
 $min$
\end{algorithm}
\pause{\hfill \SlideFontTiny \Emph{Testa med indata}: \code{args = Array("2", "42", "1", "2")}}
\end{Slide}


\Subsection{Funktioner skapar struktur}

\ifkompendium
\noindent En program delas ofta upp i många olika \Emph{funktioner}. En funktion kan ha parametrar och ge ett returvärde. Om du delar upp ditt program i många enkla funktioner med bra namn, så blir ditt program lättare att läsa och begripa. Om en vältestad och buggfri funktion användas på flera ställen, så kan risken för buggar minskas.
\fi 

\begin{Slide}{Mall för funktionsdefinitioner}
\code{def} funktionsnamn(parameterdeklarationer): returtyp = uttryck

\pause\vspace{0.3em}\SlideFontSmall
\Emph{Exempel}:

\begin{Code}[basicstyle=\ttfamily\fontsize{9}{11}\selectfont]
def öka(i: Int): Int = i + 1
\end{Code}
\pause Returtypen kan härledas av kompilatorn:
\begin{Code}[basicstyle=\ttfamily\fontsize{9}{11}\selectfont]
def öka(i: Int) = i + 1
\end{Code}
Men för att få hjälp av kompilatorn är det bra att ange returtyp!

\pause 

Om flera parametrar använd kommatecken. Om flera satser använd indentering (och eventuell valfria klammerparenteser).
\begin{Code}[basicstyle=\ttfamily\fontsize{8}{10}\selectfont]
def isHighscore(points: Int, high: Int): Boolean = {
  val highscore: Boolean = points > high
  if highscore then println(":)") else println(":(")
  highscore
}
\end{Code}
\pause Ovan funktion har \Alert{sidoeffekten} att skriva ut en smiley.
\end{Slide}

\begin{Slide}{Bättre många små abstraktioner som gör en sak var}

\begin{Code}[basicstyle=\ttfamily\fontsize{8}{11}\selectfont]
def isHighscore(points: Int, high: Int): Boolean = points > high

def printSmiley(isHappy: Boolean): Unit =
  if isHappy then println(":)") else print(":(")
\end{Code}

\pause\vspace{1em}
\begin{REPLnonum}
  printSmiley(isHighscore(113,99))
\end{REPLnonum}

\pause
\begin{itemize}
  \item Denna bättre \code{isHighscore} är nu en \Emph{äkta funktion} som alltid ger samma svar för samma inparametrar och \Alert{saknar sidoeffekter}; dessa funktioner är ofta lättare att förstå.
  \item Funktioner som ger ett booleskt värde kallas för \Emph{predikat}.
\end{itemize}

\end{Slide}



\begin{Slide}{Vad är ett block?}

\begin{itemize}
\item Ett block \Emph{kapslar in} flera satser/uttryck och ser ''utifrån'' ut som en enda sats/uttryck.

\item Ett block skapas med hjälp av klammerparenteser (''krullparenteser'')

\item [] {\fontsize{14}{18}\selectfont \code|{ uttryck1; uttryck2; ... uttryckN }|}\\~

\pause

\item I Scala (till skillnad från många andra språk) har ett block ett \Emph{värde} och är alltså ett \Emph{uttryck}.

\item Värdet ges av \Emph{sista uttrycket} i blocket.

\begin{REPLnonum}
scala> val x = { println(1 + 1); println(2 + 2); 3 + 3 }
2
4
x: Int = 6
\end{REPLnonum}


\end{itemize}

\end{Slide}

\begin{Slide}{Namn i block blir \textbf{lokala}}
Synlighetsregler:
\begin{enumerate}
\item Identifierare deklarerade inuti ett block blir \Emph{lokala}.

\item Lokala namn \Alert{överskuggar} namn i yttre block om samma.


\item Namn syns i nästlade underblock.

\end{enumerate}

\begin{REPL}
scala> def a = { val lokaltNamn = 42; println(lokaltNamn) }
scala> a 
42

scala> println(lokaltNamn)                                                                                                                  
1 |println(lokaltNamn)
  |        ^^^^^^^^^^
  |        Not found: lokaltNamn

scala> def b = { val x = 42; { val x = 76; println(x) }; println(x) }
scala> def c = { val x = 42; { val b = x + 1; println(b) } }
scala> b  // vad händer?
scala> c  // vad händer?
\end{REPL}

\end{Slide}


\begin{Slide}{Parameter och argument}

Skilj på parameter och argument!
\begin{itemize}
\item En \Alert{parameter} är det deklarerade namnet som används \Alert{lokalt} i en funktion för att referera till...

\item \Emph{argumentet} som är värdet som skickas med \Emph{vid anrop} och binds till det lokala parameternamnet.

\end{itemize}


\begin{REPLnonum}
scala> val ettArgument = 42

scala> def öka(minParameter: Int) = minParameter + 1

scala> öka(ettArgument)
\end{REPLnonum}


Speciell syntax: anrop med s.k. \Emph{namngivet argument}
\begin{REPLnonum}
scala> öka(minParameter = ettArgument)
\end{REPLnonum}
{\SlideFontSmall Namngivna argument kan ges i valfri ordning; då riskerar man inte fel ordning.}

\end{Slide}

\begin{Slide}{Procedurer}\SlideFontSmall
\begin{itemize}
\item En \Emph{procedur} är en funktion som \Alert{gör} något intressant, men som \Alert{inte} lämnar något intressant returvärde.
\item Exempel på befintlig procedur: \code{println("hej")}
\item Du \Emph{deklarerar egna procedurer} genom att ange \texttt{\Alert{Unit}} som returvärdestyp. Då ges värdet \texttt{\Alert{()}} som betyder ''inget''.
\end{itemize}
\begin{REPLsmall}$%dummydollar 
scala> def hej(x: String): Unit = println(s"Hej på dej $x!")

scala> hej("Herr Gurka")
Hej på dej Herr Gurka!

scala> val x = hej("Fru Tomat")
Hej på dej Fru Tomat!

scala> :type x 
Unit

scala> println(x)    // vad händer?
\end{REPLsmall}
\begin{itemize}
\item Det som \Alert{görs} kallas (sido)\Emph{effekt}. Ovan är utskriften själva effekten.
\item Funktioner kan också ha sidoeffekter. De kallas då \Alert{oäkta} funktioner.
\end{itemize}
\end{Slide}

\begin{Slide}{''Ingenting'' \emph{är} faktiskt någonting i Scala}
\begin{itemize}
\item I många språk (Java, C, C++, ...) är funktioner som saknar värden speciella.
 Java m.fl. har speciell syntax för procedurer med nyckelordet \jcode{void}, men \Alert{inte} Scala.

\item I Scala är procedurer inte specialfall; de är vanliga funktioner som returnerar ett värde som \Emph{representerar} ingenting, nämligen () som är av typen Unit.

\item På så sätt blir procedurer inget undantag utan följer vanlig syntax och semantik precis som för alla andra funktioner.

\item Detta är typiskt för Scala: generalisera koncepten och vi slipper besvärliga undantag! \\(Men vi måste förstå generaliseringen...)


\item [] {\SlideFontSmall
\url{https://en.wikipedia.org/wiki/Void_type}
\url{https://en.wikipedia.org/wiki/Unit_type}
}

\end{itemize}

\end{Slide}

\begin{Slide}{Problemlösning: nedbrytning i abstraktioner som sen kombineras}\SlideFontSmall
\begin{itemize}
\item En av de allra viktigaste principerna inom programmering är \Emph{funktionell nedbrytning} där  \Emph{underprogram} i form av funktioner och procedurer skapas för att bli byggstenar som kombineras till mer avancerade funktioner och procedurer.

\item Genom de namn som definieras skapas \Emph{återanvändbara abstraktioner} som kapslar in det funktionen gör.

\item Problemet blir med bra byggblock lättare att lösa.

\item Abstraktioner som beräknar eller gör \Emph{en enda, väldefinierad sak} är enklare att använda, jämfört med de som gör många, helt olika saker.

\item Abstraktioner med \Emph{välgenomtänkta namn} är enklare att använda, jämfört med kryptiska eller missvisande namn.
\end{itemize}

\end{Slide}



\begin{Slide}{Exempel på \textbf{funktionell nedbrytning}}

Kojo-labben gav exempel på \Emph{funktionell nedbrytning} där ett antal abstraktioner skapas och återanvänds.

\begin{Code}
// skapa abstraktioner som bygger på varandra

def kvadrat = upprepa(4){fram; höger}

def stapel = {
  upprepa(10){kvadrat; hoppa}
  hoppa(-10*25)
} 

def rutnät = upprepa(10){stapel; höger; fram; vänster}

// huvudprogram

sudda; sakta(200)
rutnät
\end{Code}
\end{Slide}


\begin{Slide}{Varför abstraktion?}
\begin{itemize}
\item Stora program behöver delas upp annars blir det mycket svårt att förstå och bygga vidare på programmet.
\item Vi behöver kunna välja namn på saker i koden \textit{lokalt}, utan att det krockar med samma namn i andra delar av koden.
\item Abstraktioner hjälper till att hantera och kapsla in komplexa delar så att de blir enklare att använda om och om igen.

\item Exempel på \Emph{abstraktionsmekanismer} i Scala:
\begin{itemize}

\item \href{https://sv.wikipedia.org/wiki/Klass_\%28programmering\%29}{Klasser} är ''byggblock'' med kod som används för att skapa \href{https://sv.wikipedia.org/wiki/Objektorienterad_programmering\#Objekt}{objekt}, innehållande delar som hör ihop. \\ Nyckelord: \code{class} och \code{object}

\item \href{https://en.wikipedia.org/wiki/Method_\%28computer_programming\%29}{Metoder} är funktioner som finns i klasser/objekt och används för att lösa specifika uppgifter.  Nyckelord: \code{def}

\item \href{https://en.wikipedia.org/wiki/Java_package}{Paket} används för att organisera kodfiler i en hierarkisk katalogstruktur och skapa namnrymder. \\Nyckelord: \Key{package}

\end{itemize}

\end{itemize}
\end{Slide}


\Subsection{Katalogstruktur för kodfiler med paket}



\begin{Slide}{Från källkod till maskinkod med JVM}
\begin{tikzpicture}[node distance=1.5cm]
\node (input) [startstop] {\texttt{hello.scala}};
\node(inptext) [right of=input, text width=5.5cm, scale=1.2,xshift=3.5cm]{Källkodsfil};
\node (compile) [process, below of=input] {\texttt{scalac}};
\node(comptext) [right of=compile, text width=7.2cm, scale=1.0,xshift=4.5cm]{Kompilatorn skapar \textit{abstrakt} maskinkod (s.k. bytekod)};
\node (output) [startstop, below of=compile] {\texttt{hello.class}};
\node(outtext) [right of=output, text width=5.5cm, scale=1.2,xshift=3.5cm]{\texttt{.class}-fil med bytekod};
\node (jvm) [process, below of=output] {JVM};
\node(jvmtext) [right of=jvm, text width=7.2cm, scale=1.0,xshift=4.5cm]{\textit{Java Virtual Machine}\\Översätter bytekod till \textit{konkret}\\ maskinkod som passar din specifika CPU \textbf{under körning} (s.k. interpretering)};
\draw [arrow] (input) -- (compile);
\draw [arrow] (compile) -- (output);
\draw [arrow] (output) -- (jvm);
\end{tikzpicture}
\end{Slide}




\begin{Slide}{Paket}\SlideFontSmall

\begin{Code}
package greeting

@main def run = println("Hello world!")
\end{Code}

\begin{itemize}
\item Paket \Eng{package} ger struktur åt koden och skapar namnrymder. 

\item Paket kan vara \Emph{nästlade}: ofta finns paket i paket i paket.

\item Paket är speciellt bra om man har mycket kod i många kodfiler. 

\item Kompilatorn placerar maskinkoden i kataloger enligt paketstrukturen.%
\footnote{\SlideFontTiny Katalogstrukturen för källkoden \emph{måste} i många andra språk, t.ex. Java, \emph{exakt motsvara paketstrukturen}, men detta är inte nödvändigt i Scala -- alla Scala-kodfiler kan ligga i samma katalog på toppnivå eller i underkatalog med valfritt namn, oavsett hur din kod använder \code{package}.} 
\item[] Är du nyfiken, kolla underkataloger i \code{.scala-build}:
\begin{REPLsmall}
ls -R .scala-build
\end{REPLsmall}

\end{itemize}

% \vspace{1em}
% \begin{tikzpicture}[node distance=1.5cm,scale=0.8, every node/.style={transform shape}]
% \node (input) [startstop] {\texttt{greeting/Hello.scala}};
% \node(inptext) [right of=input, text width=4cm, scale=1.2,xshift=4.5cm]{\lstinline{package greeting}\\\lstinline{object Hello  ... }};
% \node (compile) [process, below of=input] {\texttt{scalac  greeting/Hello.scala}};
% \node (output) [startstop, below of=compile] {\texttt{greeting/Hello.class}};
% \node(outtext) [right of=output, text width=4cm, scale=1.2,xshift=4.5cm]{Paketens maskinkod hamnar i katalog med samma namn som paketnamnet};
% \node (jvm) [process, below of=output] {\texttt{scala greeting.Hello}};
% \draw [arrow] (input) -- (compile);
% \draw [arrow] (compile) -- (output);
% \draw [arrow] (output) -- (jvm);
% \end{tikzpicture}

\end{Slide}

\begin{Slide}{Import}
Med hjälp av punktnotation kommer man åt innehåll i ett paket.\\
\begin{Code}
val age = scala.io.StdIn.readLine("Ange din ålder:")
\end{Code}

En \code{import}-sats...

\begin{Code}
import scala.io.StdIn.readLine
\end{Code}

...gör så att kompilatorn ''ser'' namnet, och man slipper skriva hela sökvägen till namnet:
\begin{Code}
val age = readLine("Ange din ålder:")
\end{Code}

Man säger att det importerade namnet hamnar \Emph{\textit{in scope}}.
\end{Slide}


\begin{Slide}{Jar-filer}
\begin{itemize}\SlideFontTiny
\item 
\texttt{jar}-filer liknar \texttt{zip}-filer och används för att sammanföra många kompilerade kodfiler i \Emph{en komprimerad fil} för enkel distribution och körning.
\item Du använder jar-filer med optionen \code{--jar}
\begin{REPLsmall}
scala run . --jar introprog.jar
\end{REPLsmall}
\item Du kan skapa egna jar-filer med \texttt{scala package} där optionen \code{--library} gör så att endast den komilerade koden inkluderas. Utan optionen \code{--library} så görs jar-filen exekverbar. Med optionen \code{--assembly} tas allt med i jar-filen som behövs för att köra jar-filen helt fristående med ett dubbelklick eller \code{java -jar myapp.jar}
\begin{REPLsmall}
scala package . --library --output myapp.jar
scala run --jar myapp.jar
scala package . --assembly --output my-fat-jar-app.jar
java -jar my-fat-jar-app.jar
\end{REPLsmall}
Läs mer om jar-filer i Appendix F.
\end{itemize}

% \vspace{2em}
% \begin{tikzpicture}[node distance=1.5cm,scale=0.8, every node/.style={transform shape}]
% \node (input) [startstop] {\texttt{greeting/}};
% \node(inptext) [right of=input, text width=4cm, scale=1.2,xshift=4.5cm]{en katalog med filer};
% \node (jar) [process, below of=input]
% {\texttt{jar cvf minjarfil.jar greeting}};

% \node (output) [startstop, below of=compile] {\texttt{minjarfil.jar}};

% \node(outtext) [right of=output, text width=4cm, scale=1.2,xshift=4.5cm]{En jar-fil med alla filer inpackade};

% \node (jvm) [process, below of=output] {\texttt{scala -cp minjarfil.jar}};

% \node(outtextjvm) [right of=jvm, text width=4cm, scale=1.2,xshift=4.5cm]{Lägg jar-filen till \\ ''classpath''};
% \draw [arrow] (input) -- (jar);
% \draw [arrow] (jar) -- (output);
% \draw [arrow] (output) -- (jvm);
% \end{tikzpicture}
\end{Slide}

\ifkompendium\else

\subsection{Att göra denna vecka}
%%%
\begin{SlideExtra}{Att göra denna vecka}

\begin{enumerate}
%\item Laborationer är \Alert{obligatoriska}.\\ Ev. sjukdom måste anmälas \Alert{före} via mejl till kursansvarig!
\item Gör övning \texttt{programs}
\item OBS! Ingen lab denna vecka w02. Använd tiden att komma ikapp om du ligger efter!
\item Träffas i samarbetsgrupper och hjälp varandra att förstå.
\item Vi har nosat på flera koncept som vi kommer tillbaka till senare: du kommer förstå mer detaljer på djupet då.
\item Om ni inte redan gjort det: \\Visa \href{https://github.com/bjornregnell/lth-eda016-2015/tree/master/assignments}{samarbetskontrakt} för handledare på resurstid.
\item \Alert{Koda på resurstiderna} och få hjälp och tips!
\end{enumerate}
\end{SlideExtra}

\begin{SlideExtra}{Veckans övning: \texttt{\ExeWeekTWO}}\SlideFontTiny
\vspace{-0.5em}
\setlength{\leftmargini}{0pt}
\begin{itemize}
\input{../compendium/modules/w02-programs-exercise-goals.tex}
\end{itemize}
\end{SlideExtra}

\begin{SlideExtra}{Labb läsvecka 3:  \texttt{\LabWeekTHREE}}\SlideFontSmall
\begin{itemize}
\item Skapa ett \Emph{lagom} irriterande textspel som körs i terminalen:
\begin{itemize}\SlideFontTiny
  \item ska vara \Emph{lagom} irriterande om man \Emph{först läser koden}
  \item får gärna vara \Alert{orimligt} irriterande om man \Alert{inte läser koden}
  \item koden ska vara \Emph{lättläst} och uppdelad i \Emph{många små funktioner} med bra, förklarande funktionsnamn, parameternamn och variablenamn
\end{itemize}
\item Använd en editor, kompilera och kör i terminalen
\item Mål: skapa \Emph{eget} program med \Emph{många små funktioner} och träna på \Emph{alla begrepp} vi använt hittills. Ju fler begrepp du kan använda på olika sätt desto bättre. Fokusera på det \Alert{du} behöver träna mest på.
\item Spela varandras textspel inom din samarbetsgrupp
\item Utveckla ditt spel \Emph{stegvis} och spela varandras halvfärdiga spel i flera omgångar. Ge varandra tips om förbättringar för att spelet ska bli mer lagom irriterande på ett kul sätt och för att koden ska bli mer lättläst.
\item Skriv ner den återkoppling du fått av din grupp inför labbredovisningen.
\item Läs igenom labbuppgiften redan nu och börja fundera. Ta dig inte ''vatten över huvudet''. Ta små steg i början och ha hela tiden körbar kod.


\end{itemize}
\end{SlideExtra}
\fi

\ifkompendium\else
\begin{SlideExtra}{}
  \Huge \Emph{Alla kodar loss!} \\~\\ Vi ses nästa vecka!
\end{SlideExtra}
\fi 
