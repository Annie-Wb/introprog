%!TEX encoding = UTF-8 Unicode
%!TEX root = ../lect-w03.tex

\ifkompendium\else
\begin{SlideExtra}{Hämta beställda bokpaket, snabbref!}
  \begin{itemize}
    \item Du som \Alert{ännu inte} hämtat beställd trycksak: 
    \item[] Kom överens med Birger om tid via mejl till \url{birger.swahn@cs.lth.se}: \\
    \item[] Hämtas hos Birger vid Datavetenskaps \Emph{expedition} på andra våningen i E-huset trapphus A
  \end{itemize}
\end{SlideExtra}


\begin{SlideExtra}{Kursombud}
\begin{itemize}
%\item Glädjande nog är det många intresserade!
\item Om du är intresserad: fyll i \Alert{enkät} om \Emph{kursombud} i Canvas.
\item Kursombud träffar vid behov kursansvarig på rast mellan föreläsningar eller chattar med varandra och kursansvarig på Discord \Emph{under kursens gång}.
\item Kursombud träffar kursansvarig och programledning \Emph{efter kursen} och diskuterar \Alert{kursutvärderingen} CEQ.
\item Minst 2st D:are och 2st C:are.
%\item \url{https://www.dsek.se/sektionen/srd/kursombud/}
\item Läs mer om studierådet här:\\{\SlideFontSmall\url{https://www.dsek.se/committees/srd}}
%\item Vi lottar med lite lajvkodning inspirerat av:
\end{itemize}
% \begin{REPL}
% scala> val kursombud = Vector("Kim Finkodare", "Robin Schnellhacker")
% scala> scala.util.Random.shuffle(kursombud).take(1)
% \end{REPL}
\end{SlideExtra}
\fi
