%!TEX encoding = UTF-8 Unicode
%!TEX root = ../lect-w04.tex

\ifkompendium\else
\begin{SlideExtra}{Denna vecka: Objekt}
\begin{itemize}\SlideFontSmall
\item Övning \texttt{objects} innehåller bland annat:
\begin{itemize}\SlideFontTiny
\item hur man kan kapsla in funktioner och variabler i singelobjekt
\item hur man kan skapa namnrymder, hantera namnöverskuggning, och använda punktnotation
%\item skapa objekt med hjälp av tupler
%\item lat initialisering
%\item jar-fil, paket, namnbyte vid import
\item använda färdiga klasser, t.ex. \code{java.awt.Color}
\item händelsehantering i ett grafiskt fönster
\end{itemize}

\item På laboration \texttt{blockmole} lär du dig bland annat:
\begin{itemize}\SlideFontTiny
  \item att dela upp din kod i flera singelobjekt
  \item att använda färdig klass: \code{introprog.PixelWindow}
  \item att skapa ett större program i form av ett grafiskt spel
  \item använda tidigare begrepp: uttryck, program och funktioner
\end{itemize}

\item Senaste versionen av kursbiblioteket \code{introprog} är \Emph{\LibVersion} 
\begin{itemize}\SlideFontTiny
\item Jar-fil kan laddas ned här: \url{https://fileadmin.cs.lth.se/introprog.jar}
\item Eller låt \texttt{scala-cli} ladda ner den med
\begin{Code}
//> using scala 3.7.2
//> using dep se.lth.cs::introprog:1.4.0  
\end{Code} 
\item \url{https://github.com/lunduniversity/introprog-scalalib}
\end{itemize}
\end{itemize}
\end{SlideExtra}
\fi


\Subsection{Vad är ett objekt?}

\ifkompendium\else
\begin{SlideExtra}{Objekt är ungefär som äggkartonger}
  \begin{tabular}{l r}
    \includegraphics[width=0.5\textwidth]{../img/egg-box}
    &
    \includegraphics[width=0.5\textwidth]{../img/egg-box-closed}
  \end{tabular}
  Men de kan innehålla mer än bara ägg...
\end{SlideExtra}
\fi

\begin{Slide}{Vad rymmer sköldpaddan i Kojo i sitt tillstånd?}
\centering
\includegraphics[width=0.7\textwidth]{../img/kojo}

\pause position, riktning, färg, bredd, penna uppe/nere, fyll-färg
\end{Slide}



\begin{Slide}{Vad är ett objekt?}
\begin{itemize}
\item Ett objekt är en abstraktion som...
\begin{itemize}
  \item kan innehålla \Emph{data} som objektet ''håller reda på'' och
  \item kan erbjuda \Emph{operationer} som \emph{gör} något eller ger ett \emph{värde}
\end{itemize}

\pause

\item Exempel: Sköldpaddan i Kojo \includegraphics[width=0.08\textwidth]{../img/turtle.png}
\begin{itemize}
  \item Vilken \Emph{data} sparas av sköldpaddan?
  \pause
  \item[] position, rikting, pennfärg, ...

  \item Vilka \Emph{operationer} kan man be sköldpaddan att utföra?

  \item[] fram, höger, vänster, ...
  \pause


\end{itemize}

\item Terminologi:
\begin{itemize}
  \item objektets \Emph{data} sparas i variabler som kallas \Alert{attribut}
  \item alla variablers \Emph{värden} utgör tillsammans objektets \Alert{tillstånd}
  \item \Emph{operationerna} är funktioner i objektet och kallas \Alert{metoder}
  \item objektets \Emph{delar} (attribut, metoder, etc.) kallas \Alert{medlemmar}
\end{itemize}
\end{itemize}
\end{Slide}



\begin{Slide}{Deklarera, allokera, referera}
Olika saker man kan göra med objekt:
\begin{itemize}
  \item \Emph{deklarera}: att skriva kod som beskriver objekt; \\
  finns flera sätt: singelobjekt, klass, tupel, ...
  \item \Emph{allokera}: att skapa plats i minnet för objektet vid körtid
  \item \Emph{referera}: att använda objektet via ett namn;\\
  man kommer åt innehållet i ett objekt med \Alert{punktnotation}: \\
  \code{ref.medlem}
  \pause
  \item (\Emph{avallokera}): att frigöra minne för objekt som inte längre används;
  detta \Alert{sker automatiskt} i Scala, Java, C\#, m.fl tack vare \Emph{skräpsamlaren}, men i många andra språk,
  t.ex. C++, får man själv hålla reda på avallokering,
  vilket är knepigt och det blir lätt svåra buggar.
\end{itemize}
\end{Slide}


\begin{Slide}{Olika sätt att allokera objekt}\SlideFontSmall
\begin{enumerate}

\item Använda en \Emph{färdig funktion} som skapar ett objekt åt oss, t.ex. \code{apply}:
\begin{Code}
Vector(1,2,3)  // skapa Vector-objekt med apply-metod
Vector.apply(1,2,3)  // explicit apply
\end{Code}
{\SlideFontTiny En funktion som skapar objekt kallas \Alert{fabriksmetod} \Eng{factory method}.\vspace{0.5em}}

\item Göra \code{new} på en klass (mer om klasser senare):
\begin{Code}
new introprog.PixelWindow() // skapa ett fönsterobjekt
\end{Code}
{
\SlideFontSmall Med \code{new} kan man skapa \Alert{många upplagor} av samma typ av objekt.\\
I Scala 3 kan \code{new} ofta utelämnas: \code{introprog.PixelWindow()}
}

\item Deklarera ett \Emph{singelobjekt} med nyckelordet \code{object}
\begin{itemize}\SlideFontSmall
  \item Ett singelobjekt finns i exakt \Alert{en} upplaga.
  \item Allokeras \Alert{automatiskt} första gången man refererar objektet; \\
  man behöver inte, och kan inte, skriva \code{new}.
  \pause
  \item Medlemmar i ett Scala-singelobjekt liknar \jcode{static}-medlemmar i en Java/C++/C\#-klass.
\end{itemize}
\item Använda en \Emph{tupel}, exempel: \code{val p = (200, 300)}
\end{enumerate}
\end{Slide}

\Subsection{Singelobjekt}

\begin{Slide}{Vad är ett singelobjekt?}
\begin{itemize}\SlideFontSmall
\item Ett singelobjekt \Eng{singelton} deklareras med nyckelordet \code{object} och används för att samla \Emph{medlemmar} \Eng{members} som \Alert{hör ihop}.
\item Ett singelobjekt kallas också \Emph{modul} \Eng{module}.
\item Medlemmarna kan t.ex. vara \Emph{variabler} (\code{val}, \code{var}) och \Emph{metoder} (\code{def}). 
\item En \Alert{metod} är en \Emph{funktion} som finns i ett objekt. Metoder kallas även \Emph{operationer}.
\item Exempel: singelobjekt/modul som hanterar highscore:
\begin{Code}
object Highscore {
  var highscore = 0
  def isHighscore(points: Int): Boolean = points > highscore
}
\end{Code}
\item Krullparenteser är valfria i Scala 3:\\~~~du kan använda kolon och indentering i stället.
\pause
\item Tanken är ofta att abstraktioner ska vara användbar i annan kod, för att underlätta när man bygger applikationer, och kallas då ett \Emph{API} (Application Programming Interface). Exempel: ett highscore-API.
\end{itemize}
\end{Slide}


\begin{Slide}{Allokering: minne reserveras med plats för data}
\begin{Code}
object Highscore:
  var highscore = 0
  def isHighscore(points: Int): Boolean = points > highscore

\end{Code}
\pause
\begin{tikzpicture}[font=\large\sffamily]
\matrix [matrix of nodes, row sep=0, column 2/.style={nodes={rectangle,draw,minimum width=0.8cm}}] (mat)
{
\texttt{Highscore}   &  \makebox(10,10){ }\\
%\texttt{g2}   &  \makebox(16,12){ }\\
};
\node[cloud, cloud puffs=13.0, cloud ignores aspect, minimum width=2cm, minimum height=3.8cm,
 align=center, draw] (x) at (5.8cm, -1.5cm) {
 \begin{tabular}{r l}
 \texttt{highscore} & \fbox{~~~~~0~~} \\

 \end{tabular}
 };
\filldraw[black] (1.2cm,0.0cm) circle (3pt) node[] (ref) {};
 \draw [arrow, line width=0.7mm] (ref) -- (x);
% \node[cloud, cloud puffs=15.7, cloud ignores aspect, %minimum width=5cm, minimum height=2cm,
% align=center, draw] (g2) at (5cm, -2cm) {Gurka-\\objekt};
% \filldraw[black] (0.4cm,-0.4cm) circle (3pt) node[] (g2ref) {};
% \draw [arrow] (g2ref) -- (g2);
\end{tikzpicture}
\end{Slide}


\begin{Slide}{Punktnotation, tillståndsförändring med tilldelning}
\begin{REPLnonum}
scala> Highscore.isHighscore(5)
res0: Boolean = true

scala> Highscore.highscore = 42
\end{REPLnonum}
\pause
\begin{tikzpicture}[font=\large\sffamily]
\matrix [matrix of nodes, row sep=0, column 2/.style={nodes={rectangle,draw,minimum width=0.8cm}}] (mat)
{
\texttt{Highscore}   &  \makebox(10,10){ }\\
%\texttt{g2}   &  \makebox(16,12){ }\\
};
\node[cloud, cloud puffs=13.0, cloud ignores aspect, minimum width=2cm, minimum height=3.8cm,
 align=center, draw] (x) at (5.8cm, -1.5cm) {
 \begin{tabular}{r l}
 \texttt{highscore} & \fbox{~~~~42~~} \\

 \end{tabular}
 };
\filldraw[black] (1.2cm,0.0cm) circle (3pt) node[] (ref) {};
 \draw [arrow, line width=0.7mm] (ref) -- (x);
% \node[cloud, cloud puffs=15.7, cloud ignores aspect, %minimum width=5cm, minimum height=2cm,
% align=center, draw] (g2) at (5cm, -2cm) {Gurka-\\objekt};
% \filldraw[black] (0.4cm,-0.4cm) circle (3pt) node[] (g2ref) {};
% \draw [arrow] (g2ref) -- (g2);
\end{tikzpicture}
\end{Slide}


\begin{Slide}{Punktnotation och operatornotation}
Punktnotation där metodanropet har \Alert{ett} enda argument:\\~\\
\code{objekt.metod(argument)}
\\~\\kan även skrivas med infix \Emph{operatornotation}:\\~\\
\code{objekt metod argument}
\\~\\Exempel:\\
\code{1 + 2}\\\pause\vspace{0.5em}
\code{Highscore isHighscore 1000}
\pause 
{
\SlideFontSmall
\\\vspace{0.5em}Operatornotation med metoder vars namn börjar med bokstäver kommer i framtiden kräva deklaration med \code{infix} före \code{def}, detta för att uppmuntra konsekvent användning.
}
\end{Slide}


\begin{Slide}{Namnrymd och skuggning}
\begin{itemize}\SlideFontSmall
  \item En \Emph{namnrymd}\footnote{\url{https://sv.wikipedia.org/wiki/Namnrymd}}  \Eng{namespace} är en omgivning (kontext) i vilken alla namn är unika. Genom att skapa flera olika namnrymder
  kan man undvika ''\Alert{krockar}'' mellan lika namn med olika betydelser (homonymer). \\
  Exempel: mejladresser \code{kim@företag1.se}  ~$\neq$~  \code{kim@företag2.se}
  \item Medlemmarna i ett singelobjekt finns i en egen namnrymd,
  där alla namn måste vara unika på samma nivå. De ''krockar'' inte med namn ''utanför'' objektet. Dock kan det förekomma \Alert{skuggning} \Eng{shadowing}:
\end{itemize}
\begin{Code}
object Game {

  val highscore = 42   // ett annat värde än Game.Highscore.highscore

  object Highscore:
    var highscore = 0  // ett annat värde än Game.highscore
    def isHighscore(points: Int): Boolean = points > highscore
}
\end{Code}

\end{Slide}



\begin{Slide}{Inkapsling: att dölja interna delar}\SlideFontSmall
Med nyckelordet \code{private} döljs interna delar för omvärlden.
Privata medlemmar kan bara refereras \emph{inifrån} objektet.
Denna princip kallas \Emph{inkapsling} \Eng{encapsulation}.
\begin{CodeSmall}
object Highscore:
  private var myHighscore = 0        // namnet myHighscore syns ej utåt
  def highscore: Int = myHighscore   // en s.k. getter ger ett attributvärde
  def isHighscore(points: Int): Boolean = points > myHighscore
  def update(points: Int): Unit = if isHighscore(points) then myHighscore = points
\end{CodeSmall}
Varför har man nytta av detta?
\pause

\begin{itemize}
  \item Förhindra att man av misstag ändrar objekts tillstånd på fel sätt.
  \item Förhindra användning av kod som i framtiden kan komma att ändras.
  \item Erbjuder en enklare ''utsida'' genom dölja komplexitet ''på insidan''.
  \item Inte ''skräpa ner'' namnrymden med ''onödiga'' namn.
\end{itemize}
Nackdelar?
\pause
\begin{itemize}
  \item Begränsar användningen, har ej tillgång till alla delar.
  \item Svårare att experimentera med ett API medan man försöker förstå det.
\end{itemize}
\end{Slide}



\begin{Slide}{Idiom: Privata variabler med understreck vid ''krock''}\SlideFontSmall
\Emph{Idiom}: (d.v.s. ett typiskt, allmänt accepterat sätt att skriva kod)
\begin{itemize}
  \item Om namnet på en privat variabel krockar med namnet på en getter
  brukar man börja det privata namnet med ett understreck:
\end{itemize}

\begin{CodeSmall}
object Highscore:
  private var _highscore = 0
  def highscore: Int = _highscore
  def isHighscore(points: Int): Boolean = points > _highscore
  def update(points: Int): Unit = if isHighscore(points) then _highscore = points
\end{CodeSmall}

\pause

{\SlideFontTiny Namnkrock mellan metoder och variabler uppkommer inte i Java m.fl. språk, där dessa finns i \emph{olika} namnrymder.
Men i Scala har man valt att principen om \Emph{enhetlig access} ska gälla och alla medlemmar (både metoder och variabler) finns därmed i en gemensam namnrymd.}

\end{Slide}

\begin{Slide}{Principen om enhetlig access}\SlideFontSmall
  \begin{itemize}
    \item I Scala så ser access av attribut och anrop av metoder, som är deklarerade utan parameterlista, likadana ut. 
\begin{Code}
object A1 { val a = 42 }  
object A2 { def a = (41 + math.random()).round.toInt }
\end{Code}
\begin{REPLnonum}
scala> A1.a
scala> A2.a  
\end{REPLnonum}
    \item Många andra språk har olika syntax för access av attribut och anrop av metoder (t.ex. Java m.fl., där alla metodanrop måste ha parenteser).
    \item Fördel: Det går lätt att ändra i implementationen och växla mellan att använda attribut och använda metoder utan att den kod som använder din implementation behöver ändras.
    \item Nackdel: Det kan bli namnkrockar mellan metoder och attribut eftersom de finns i samma namnrymd.
  \end{itemize}
  
\end{Slide}



\begin{Slide}{Exempel: singelobjektet med förändringsbart tillstånd} \SlideFontSmall
\begin{Code}[basicstyle=\ttfamily\fontsize{9}{11}\selectfont]
object mittBankkonto:
  val kontonr: Long        = 123456789L
  var saldo: Int           = 1000
  def ärSkuldsatt: Boolean = saldo < 0
\end{Code}
\begin{REPLnonum}
scala> mittBankkonto.saldo -= 25000

scala> mittBankkonto.ärSkuldsatt
res0: Boolean = true
\end{REPLnonum}

(Vi ska i nästa vecka se hur man med s.k. klasser kan skapa många upplagor av samma  typ av objekt, så att vi kan ha flera olika bankkonto.)
\end{Slide}



\begin{Slide}{Exempel: tillstånd, attribut}
Ett objekts \Emph{tillstånd} är den samlade uppsättningen av värden av alla de attribut som finns i objektet.
\begin{Code}[basicstyle=\ttfamily\fontsize{9}{11}\selectfont]
object mittBankkonto:
  val kontonr: Long        = 123456789L
  var saldo: Int           = 1000
  def ärSkuldsatt: Boolean = saldo < 0
\end{Code}
\begin{tikzpicture}[font=\large\sffamily]
\matrix [matrix of nodes, row sep=0, column 2/.style={nodes={rectangle,draw,minimum width=0.8cm}}] (mat)
{
\texttt{mittBankkonto}   &  \makebox(10,10){ }\\
%\texttt{g2}   &  \makebox(16,12){ }\\
};
\node[cloud, cloud puffs=13.0, cloud ignores aspect, minimum width=2cm, minimum height=3.8cm,
 align=center, draw] (x) at (5.6cm, -1.4cm) {
 \begin{tabular}{r l}
 \texttt{kontonr} & \fbox{123456789L} \\
 \texttt{saldo} & \fbox{1000}\\
 \end{tabular}
 };
\filldraw[black] (1.7cm,0.0cm) circle (3pt) node[] (ref) {};
 \draw [arrow, line width=0.7mm] (ref) -- (x);
% \node[cloud, cloud puffs=15.7, cloud ignores aspect, %minimum width=5cm, minimum height=2cm,
% align=center, draw] (g2) at (5cm, -2cm) {Gurka-\\objekt};
% \filldraw[black] (0.4cm,-0.4cm) circle (3pt) node[] (g2ref) {};
% \draw [arrow] (g2ref) -- (g2);
\end{tikzpicture}
\end{Slide}


\begin{Slide}{Tillståndsändring}

När en variabel tilldelas ett nytt värde sker en \Emph{tillståndsändring}. Ett \Emph{förändringsbart objekt} \Eng{mutable object} har ett \Emph{förändringsbart tillstånd} \Eng{mutable state}.

\begin{REPLnonum}
scala> mittBankkonto.saldo -= 25000

scala> mittBankkonto.saldo
res1: Int = -24000
\end{REPLnonum}
\begin{tikzpicture}[font=\large\sffamily]
\matrix [matrix of nodes, row sep=0, column 2/.style={nodes={rectangle,draw,minimum width=0.8cm}}] (mat)
{
\texttt{mittBankkonto}   &  \makebox(10,10){ }\\
%\texttt{g2}   &  \makebox(16,12){ }\\
};
\node[cloud, cloud puffs=13.0, cloud ignores aspect, minimum width=2cm, minimum height=3.8cm,
align=center, draw] (x) at (5.6cm, -1.4cm) {
  \begin{tabular}{r l}
 \texttt{kontonr} & \fbox{123456789L} \\
 \texttt{saldo} & \fbox{-24000}\\
 \end{tabular}
 };
\filldraw[black] (1.7cm,0.0cm) circle (3pt) node[] (ref) {};
 \draw [arrow, line width=0.7mm] (ref) -- (x);
% \node[cloud, cloud puffs=15.7, cloud ignores aspect, %minimum width=5cm, minimum height=2cm,
% align=center, draw] (g2) at (5cm, -2cm) {Gurka-\\objekt};
% \filldraw[black] (0.4cm,-0.4cm) circle (3pt) node[] (g2ref) {};
% \draw [arrow] (g2ref) -- (g2);
\end{tikzpicture}
\end{Slide}

\Subsection{Paket}

\begin{Slide}{Modul}

  \begin{itemize}
    \item En modul samlar kod som utgör en sammanhållen, avgränsad \Emph{uppsättning abstraktioner} som kan användas av annan kod  för att lösa ett specifikt (del)problem.
    \item I Scala finns två sätt att skapa moduler:\footnote{\href{https://en.wikipedia.org/wiki/Modular_programming}{en.wikipedia.org/wiki/Modular\_programming}}
    \begin{itemize}
      \item \Emph{singelobjekt} med nyckelordet \code{object} och
      \item \Emph{paket} med nyckelordet \code{package}
      \pause
      \item Liknar varandra; t.ex. kan man använda punktnotation och göra \code{import} på medlemmar i både singelobjekt och paket.
      \pause
      \item Skillnader:
      \begin{itemize} 
        \item paket medför att \Alert{underkataloger} för maskinkoden skapas vid kompilering
        \item objekt kan ärva medlemmar från klasser och traits (mer om det senare)
        %\item paket får inte i Scala 2 innehålla variabel- och funktions- deklarationer på topp-nivå; dessa  måste i Scala 2 ligga inuti singelobjekt eller klasser
      \end{itemize}
      %\item I Scala 2 kan toppnivådekl. placeras i ett \code{package object}
    \end{itemize}
  
  \end{itemize}
\end{Slide}

\begin{Slide}{Deklarera paket}
Med nyckelordet \code{package} först i en kodfil ges alla deklarationer en gemensam namnrymd.\\
\vspace{1em}
Denna kod ligger i filen \texttt{f1.scala}:
\begin{Code}
package mittpaket

object A: 
  def hälsa: Unit = println(B.hälsning)
\end{Code}

Denna kod ligger i filen \texttt{f2.scala}:
\begin{Code}
package mittpaket

object B:
  def hälsning: String = "hejsan"
\end{Code}
Singelobjekten \code{A} och \code{B} finns båda i namnrymden \code{mittpaket}.
\end{Slide}

\begin{Slide}{Kompilera paket}\SlideFontSmall
Paketdeklarationer medför att kompilatorn placerar bytekodfiler i en katalog med samma namn som paketet:
\begin{REPL}
> scalac f1.scala f2.scala     // samkompilering av två filer
> ls
f1.scala  f2.scala  mittpaket
> ls mittpaket
A.class  'A$.class'   A.tasty   
B.class  'B$.class'   B.tasty 
\end{REPL}
\pause
Idiom, syntax och semantik:
\begin{itemize}
  \item Paketnamn brukar bestå av enbart små bokstäver.
  \item Om paketnamn innehåller punkt(er), skapas nästlade underpaket, exempel:  \code{p1.p2.p3} kompilerar kod till katalogen \code{p1/p2/p3}
  \item Du kan ha flera paket och även nästlade paket i \Alert{samma} kodfil, genom att använda klammerparentes (eller kolon+indentering):\\
  \code|package p1 { object A; package p2 { object B }}|
\end{itemize}
\end{Slide}

\begin{Slide}{Paket i REPL}
Paket funkar inte i REPL:
\begin{REPLnonum}
scala> package mittpaket { def hej = println("Hej") }
-- [E103] Syntax Error: -------------------------------
1 |package mittpaket { def hej = println("Hej") }
  |^^^^^^^
  |this kind of statement is not allowed here

\end{REPLnonum}
\end{Slide}


\Subsection{Tupler}


\begin{Slide}{Vad är en tupel?}\SlideFontSmall

\begin{itemize}
\item En $n$-tupel är ett objekt som samlar $n$ st objekt i en enkel datastruktur med koncis syntax;
du behöver bara parenteser och kommatecken för att skapa tupel-objekt: ~~\code{(1,'a',"hej")}
\item Elementen kan alltså vara av \Alert{olika} typ.

\item
\code{(1,'a',"hej")} är en \Emph{3-tupel} av typen: \code{(Int, Char, String)}

\pause

\item Du kan komma åt de enskilda elementen med \Emph{\code{_1}}, \Emph{\code{_2}}, ...  \Emph{\code{_}$n$}
\item Du kan även använda \Emph{\code{apply(0)}}, \Emph{\code{apply(1)}}, ...  \Emph{\code{apply(n-1)}}

\begin{REPL}
scala> val t = ("hej", 42, math.Pi)
t: (String, Int, Double) = (hej,42,3.141592653589793)

scala> t._1    // direkt access
res0: String = hej

scala> t(1)    // notera användningen av apply
res1: Int = 42
\end{REPL}

\pause

\item Tupler är praktiska när man inte vill ta det lite större arbetet att skapa en egen klass.
(Men med klasser kan man göra mycket mer än med tupler.)

% \item I Scala 2 kan du skapa tupler upp till en storlek av 22 element.
% \\ Behöver du fler element, använd i stället en samling, t.ex. \code{Vector}.
\end{itemize}

\end{Slide}


\begin{Slide}{Tupler som parametrar och returvärde.}\SlideFontSmall

\begin{itemize}

\item Tupler är smidiga som \Emph{parametrar} om man vill kombinera värden som hör ihop, till exempel
 x- och y-värdena i en punkt: \code{(3, 4)}
\pause
\item Tupler är smidiga när man på ett enkelt och typsäkert sätt
vill låta en funktion \Emph{returnera mer än ett värde}.

\begin{REPLsmall}
scala> def längd(p: (Double, Double)): Double = math.hypot(p._1, p._2)

scala> def vinkel(p: (Double, Double)): Double = math.atan2(p._1, p._2)

scala> def polär(p: (Double, Double)): (Double, Double) = (längd(p), vinkel(p))

scala> polär((3,4))
res2: (Double, Double) = (5.0,0.6435011087932844)

\end{REPLsmall}
\vspace{0.5em}
\item Om typerna passar kan man skippa dubbla parenteser vid \Emph{ensamt tupel-argument}:
\begin{REPL}
scala> polär(3,4)
res3: (Double, Double) = (5.0,0.6435011087932844)
\end{REPL}
\item[] {\SlideFontTiny\href{https://sv.wikipedia.org/wiki/Pol\%C3\%A4ra_koordinater}{https://sv.wikipedia.org/wiki/Polära\_koordinater}}


\end{itemize}
\end{Slide}



\begin{Slide}{Ett smidigt sätt att skapa 2-tupler med metoden \texttt{->}}
Det finns en metod vid namn \code{->} som kan användas på objekt av \Alert{godtycklig} typ för att \Emph{skapa par}:

\vspace{0.8em}
\begin{REPL}
scala> ("Ålder", 42)
res0: (String, Int) = (Ålder,42)

scala> "Ålder".->(42)
res1: (String, Int) = (Ålder,42)

scala> "Ålder" -> 42
res2: (String, Int) = (Ålder,42)

scala> Vector("Ålder" -> 42, "Längd" -> 178, "Vikt" -> 65)
res3: scala.collection.immutable.Vector[(String, Int)] =
        Vector((Ålder,42), (Längd,178), (Vikt,65))


\end{REPL}
\end{Slide}


\begin{Slide}{Typalias för att abstrahera typnamn}\SlideFontSmall
Med hjälp av nyckelordet \code{type} kan man deklarera ett \Emph{typalias} för att ge ett \Alert{alternativt} namn till en viss typ. Exempel:
\begin{REPL}
scala> type Pt = (Int, Int)            // typalias
scala> type Pts = Vector[Pt]           // nästlat typalias

scala> def distToOrigo(pt: Pt): Double = math.hypot(pt._1, pt._2)

scala> val xs: Pts = Vector((1,1), (2,2), (3,4))
val xs: Pts = Vector((1,1), (2,2), (3,4))

scala> xs.head
val res0: Pt = (1,1)

scala> xs.map(distToOrigo)                                                                  
val res1: Vector[Double] = Vector(1.4142135623730951, 2.8284271247461903, 5.0)
\end{REPL}

Typalias kan vara bra när:
\begin{itemize}
\item man har en lång och krånglig typ och vill använda ett kortare namn,

\item man vill kunna lätt byta implementation senare\\(t.ex. om man vill använda en egen klass i stället för en tupel).
\end{itemize}
\end{Slide}


\Subsection{Fördröjd initialisering}

\begin{Slide}{Lata variabler och fördröjd initialisering}
Med nyckelordet \code{lazy} före \code{val} sker ''\Emph{lat}'' evaluering av initialiseringsuttrycket. Motsatsen (det normala i Scala) kallas \Alert{strikt} evaluering.
\begin{REPL}
scala> val strikt = Vector.fill(1000000)(math.random())
strikt: scala.collection.immutable.Vector[Double] =
 Vector(0.7583305221813246, 0.9016192590993339, 0.770022134260162, 0.15667718184929746, ...

scala> lazy val lat = Vector.fill(1000000)(math.random())
lat: scala.collection.immutable.Vector[Double] = <lazy>

scala> lat
res0: scala.collection.immutable.Vector[Double] =
  Vector(0.5391685014341797, 0.14759775960530275, 0.722606095900537, 0.9025572787055386, ...
\end{REPL}

En \code {lazy val} initialiseras \Alert{inte} vid deklarationen utan när den \Alert{refereras första gången}. Uttrycket som anges i deklarationen evalueras med s.k. \Emph{fördröjd evaluering} (även ''lat'' evaluering).
\end{Slide}




\begin{Slide}{Singelobjekt är lata}

\begin{itemize}
  \item Singelobjekt allokeras \Alert{inte} direkt vid deklaration; allokeringen sker först då objektet refereras första gången.

\pause

  \item Exempel:

\end{itemize}

\begin{Code}[basicstyle=\ttfamily\fontsize{7.4}{11}\selectfont]
object mittLataObjekt:
  println("jag är lat")
  val storArray = { println("skapar stor Array"); Array.fill(10000)(42) }
  lazy val ännuStörreArray = Array.fill(Int.MaxValue)(42)
\end{Code}

När sker utskrifterna?

När allokeras variablerna?

\end{Slide}




\begin{Slide}{Vad är skillnaden mellan \texttt{val}, \texttt{var}, \texttt{def}, \texttt{lazy val}?}
\begin{Code}[basicstyle=\ttfamily\fontsize{8}{11}\selectfont]
object exempel:
  println("hej exempel")
  val förAlltidSammaReferens  = {println("hej val"); math.random()}
  var kanÄndrasMedTilldelning = {println("hej var"); math.random()}
  def evaluerasVidVarjeAnrop  = {println("hej def"); math.random()}
  lazy val fördröjdInit = {println("hej lazy val"); math.random()}
\end{Code}
I vilken ordning sker utskrifterna?


\pause
{\vspace{2em}\SlideFontTiny Lat evaluering är en viktig princip inom funktionsprogrammering som möjliggör effektiva, oföränderliga datastrukturer där element allokeras först när de behövs. \\
\href{https://en.wikipedia.org/wiki/Lazy_evaluation}{en.wikipedia.org/wiki/Lazy\_evaluation}
}
\end{Slide}


\begin{Slide}{Be kompilatorn att varna vid initialiseringsproblem}
\SlideFontSmall
Initialisering i fel ordning kan ge oväntade överraskningar:
\begin{REPLsmall}
scala> { val b = a; val a = 42 }
val b: Int = 0   // default-värdet för Int är noll och a har ännu inte fått värdet 42
val a: Int = 42
\end{REPLsmall}
Med kompilator-optionen \code{-Wsafe-init} får du en välbehövlig varning. \\
Skriv såhär i din kod om du vill ha denna option påslagen:
\begin{CodeSmall}
//> using options -Wsafe-init

@main def run =
  val b = a
  val a = 42
  println(b)

\end{CodeSmall}
\begin{REPLsmall}
>  scala run .
[error] a is a forward reference extending over the definition of b
[error]   val b = a
[error]           ^
\end{REPLsmall}
\end{Slide}

\begin{Slide}{Be kompilatorn ge fler bra varningar}\SlideFontSmall
Slå på mer utförliga meddelanden och varningar:
\begin{CodeSmall}
//> using options -unchecked -deprecation -Wunused:all -Wvalue-discard -Wsafe-init
\end{CodeSmall}
\begin{tabular}{l p{8.5cm}}
\texttt{-unchecked} & Extra varningar vid flera fall av osäker kod. \\
\texttt{-deprecation} & Förklaring vid användning av utgående funktioner. \\
\texttt{-Wunused:all} & Varning om deklarationer ej används. \\
\texttt{-Wvalue-discard} & Varning vid förlorat värde. \\
\texttt{-Wsafe-init} & Varna vid användning av ännu ej initialiserade variabler. \\
\end{tabular}

\pause\vspace*{1em}
Om du tycker vissa specifika varningar är irriterande kan du slå av dem med \code{@annotation.nowarn} 
\begin{CodeSmall}
  @annotation.nowarn 
  val b = a
  val a = 42 
\end{CodeSmall}
  
\end{Slide}

\Subsection{Funktioner är objekt}

\begin{Slide}{Programmeringsparadigm}
\href{https://en.wikipedia.org/wiki/Programming_paradigm}{en.wikipedia.org/wiki/Programming\_paradigm}:
\begin{itemize}
\item \Emph{Imperativ programmering}: programmet är uppbyggt av sekvenser av olika satser som läser och \Alert{ändrar} tillstånd
\item \Emph{Objektorienterad programmering}: en sorts imperativ programmering där programmet består av objekt som kapslar in tillstånd och erbjuder operationer som läser och \Alert{ändrar} tillstånd.
\item \Emph{Funktionsprogrammering}: programmet är uppbyggt av samverkande (äkta) funktioner som \Alert{undviker} föränderlig data och tillståndsändringar. Oföränderliga datastrukturer skapar effektiva program i kombination med lat evaluering och rekursion.
\end{itemize}
\end{Slide}


\begin{Slide}{Funktioner är äkta objekt i Scala}
Scala visar hur man kan \Alert{förena} \Eng{unify} \\ \Emph{objektorientering} och \Emph{funktionsprogrammering}: \\\vspace{0.5em}

\textbf{En funktion är ett objekt som har en \code{apply}-metod.}
\pause
\begin{REPLnonum}
scala> object öka:
         def apply(x: Int) = x + 1

scala> öka.apply(1)
res0: Int = 2

scala> öka(1)   // metoden apply behöver ej skrivas explicit
res1: Int = 2
\end{REPLnonum}
\end{Slide}



\begin{Slide}{Fördjupning: Äkta funktionsobjekt är av funktionstyp}
Egentligen, mer precist:\\
\textbf{En funktion är ett objekt \Alert{av funktionstyp} som har en \code{apply}-metod.}
\pause
\begin{REPLnonum}
scala> object öka extends (Int => Int):
         def apply(x: Int) = x + 1
 
scala> öka(1)
res2: Int = 2

scala> Vector(1,2,3).map(öka)
res3: scala.collection.immutable.Vector[Int] = Vector(2, 3, 4)

scala> öka.   // tryck TAB
... andThen   apply   compose ... toString ...
\end{REPLnonum}
Mer om \code{extends} senare i kursen... %extends (Int => Int skrivs om till Function1[Int, Int]
\end{Slide}



\Subsection{Använda färdiga klasser}

\begin{Slide}{Vad är en klass?}
Singelobjekt finns bara i exakt EN upplaga:
\begin{Code}
object mittBankkonto:
  val kontonr: Long        = 123456789L
  var saldo: Int           = 1000
  def ärSkuldsatt: Boolean = saldo < 0
\end{Code}
Om vi vill ha flera bankkonton behöver vi en \Alert{klass} \Eng{class}.
\end{Slide}

\begin{Slide}{Vad är en klass?}
En klass kan användas för att skapa många objekt av samma typ. Varje upplaga har sitt eget tillstånd och kallas en \Emph{instans} av klassen (mer om detta nästa vecka).
\begin{Code}
class Bankkonto(val kontonr: Long, var saldo: Int): // klassbeskrivning
  def ärSkuldsatt: Boolean = saldo < 0
\end{Code}
\pause
\begin{REPL}
scala> val bk1 = new Bankkonto(123456789L, 1000 )   // instansiera en klass
bk1: Bankkonto = Bankkonto@5d7399f9

scala> val bk2 = new Bankkonto(6789012L, -200 )
bk2: Bankkonto = Bankkonto@286855ea

scala> bk1.saldo
res0: Int = 1000

scala> bk2.ärSkuldsatt
res1: Boolean = true
\end{REPL}
\end{Slide}

\begin{Slide}{Använda klassen \code{Color}}\SlideFontSmall
\begin{itemize}
\item I JDK (Java Development Kit) finns hundratals paket (moduler) och tusentals färdiga klasser.
\footnote{\SlideFontTiny\url{https://stackoverflow.com/questions/3112882/}}

\item En av dessa klasser heter \code{Color} och ligger i paketet \code{java.awt} och används för att representera RGB-färger med ett tal som beskriver andelen Rött, Grönt och Blått.
\end{itemize}
\begin{REPL}
scala> val röd = java.awt.Color(255, 0, 0)    //  en maximalt röd färg

scala> import java.awt.Color  // namnet Color tillgängligt i aktuell namnrymd

scala> Color.    // tryck TAB och se alla publika medlemmar
\end{REPL}
\pause
\begin{itemize}
\item Använd klassen \code{java.awt.Color} på veckans övning.
\item Hur ska jag veta hur jag kan använda en färdig klass?
\pause
\begin{enumerate}\SlideFontTiny
  \item Läs koden, visar ''insidan'' med all sin komplexitet; kan vara knepigt...
  \item Läs \Emph{dokumentationen}, visar ''utsidan'' som är enklare (?) än ''insidan''
  \item \Alert{Experimentera} med hjälp av REPL och/eller en IDE
\end{enumerate}
\end{itemize}

\end{Slide}


\Subsection{Extensionsmetoder}

\begin{Slide}{Lägg till metoder i efterhand med \texttt{extension}}\SlideFontSmall
\begin{itemize}\SlideFontSmall
\item Ofta vill man kunna lägga till metoder på godtyckliga typer i efterhand, speciellt när det gäller typer som finns i kod som någon annan skrivit.
\item Detta går att göra i Scala med nyckelordet \code{extension}:\\{\SlideFontTiny\code{extension (s: String) def skrikBaklänges = s.reverse.toUpperCase}}
\item En \Emph{extensionsmetod} kan anropas med \Alert{punktnotation} som om den vore en medlem av typen.
\item Det går också att anropa en extensionsmetod som en fristående funktion utan punktnotation.
\end{itemize}  
\begin{REPL}
scala> extension (s: String) def skrikBaklänges = s.reverse.toUpperCase
def skrikBaklänges(s: String): String

scala> "hejsan".skrikBaklänges
val res1: String = NASJEH

scala> skrikBaklänges("goddag")
val res2: String = GADDOG
\end{REPL}
\end{Slide}

\begin{Slide}{Kollektiva extensionsmetoder}
\begin{itemize}
  \item 
Det går bra att sammanföra flera funktioner under en och samma \code{extension} så här:
\begin{Code}
extension (s: String)
  def baklänges = s.reverse
  def skrik = s.toUpperCase
\end{Code}
\item Detta kallas \Emph{kollektiva extensionsmetoder} \Eng{collective extension methods}. 
\item Notera att det \emph{inte} ska vara något kolon efter \code{extension}-deklarationens första rad.
  
\end{itemize}
\end{Slide}

\Subsection{Mer om import}

\begin{Slide}{Import av alla namn i en viss modul}\SlideFontSmall
\begin{itemize}
\item Man kan importera \Alert{alla} namn i en viss modul (singelobjekt eller paket). Detta kallas på engelska för \emph{wildcard import}.

\begin{itemize}\SlideFontTiny
  \item Syntax:  \code|import p1.p2.*|
\end{itemize}

\item Exempel:
\begin{REPL}
scala> import java.awt.*  // importera ALLA namn i paketet awt
\end{REPL}
\item \Emph{Fördelar}:
\begin{enumerate}\SlideFontTiny
  \item Slipper skriva import på varje enskilt namn.
  \item De abstraktioner som är tänkta att användas tillsammans blir alla synliga i aktuell namnrymd \Eng{in scope}.
\end{enumerate}
\item \Alert{Nackdelar}:
\begin{enumerate}\SlideFontTiny
  \item Kan ge namnkrockar och svåra buggar vid namnskuggning.
  \item Man ''skräpar ner'' sin namnrymd med namn som kanske inte är tänkta att användas, men som vid misstag, t.ex. felstavning, ändå ger effekt.
  \item Man kan inte genom att studera import-deklarationerna se exakt vilka namn som används, vilket kan göra det svårare att förstå vad koden gör.
\end{enumerate}
\end{itemize}
\end{Slide}


\begin{Slide}{Namnbyte vid import}
\begin{itemize}
\item Man kan undvika namnkrockar med \Emph{namnbyte vid import}.
\item Syntax:  \code|import p1.p2.befintligtNamn as nyttNamn|
\item Exempel:
\begin{REPL}
scala> import java.awt.Color as JColor //importera och byt namn

scala> val grön = JColor(0, 255, 0)  //skapa instans med nya namnet
grön: java.awt.Color = java.awt.Color[r=0,g=255,b=0]
\end{REPL}
%\item Detta går inte att göra i Java, men t.ex. i C\# och Kotlin.
\end{itemize}
\end{Slide}

\begin{Slide}{Exkludera (gömma) namn vid import}
\begin{itemize}
\item Man kan undvika namnkrockar vid import genom att exkludera vissa namn \Eng{import hiding}.
\item Syntax:  \code|import p1.p2.exkluderaMig as _|
\item Exempel:
\begin{REPL}
scala> import java.awt.{Event as _, *}  // importera allt UTOM Event
\end{REPL}
\item Kan kombineras med namnbyte och allimport:
\begin{REPL}
scala> import java.awt.{Event as _, Color as JColor, *}
\end{REPL}
%\item Detta går inte att göra i Java.
\end{itemize}
\end{Slide}

\begin{Slide}{Lokal import-deklaration}
\begin{itemize}
\item Man kan begränsa ''nedskräpningen'' av namnrymden genom att göra import-deklarationer så lokalt som möjligt, till exempel i ett objekt eller i en funktionskropp.
\item Exempel:
\begin{Code}
object A:
  def x = 
    import java.awt.Color.RED
    /* ... namnet RED syns bara lokalt i denna funktion */
\end{Code}
%\item Detta går inte i Java, där import ska stå i början av filen.
\end{itemize}
\end{Slide}

\begin{Slide}{Export}
\begin{itemize}\SlideFontSmall
  \item \code{import} ger direkt synlighet \Emph{lokalt} inuti en namnrymd
  \item Med \code{export} kan du göra \emph{motsatsen} till import: \\
göra medlemmar direkt synliga \Alert{utanför} en namnrymd.
\begin{CodeSmall}
object A:
  import java.awt.Color.* // gör färger synliga direkt inuti detta objekt
  def test = RED          // färgen RED synlig direkt i lokala namnrymden

object B:
  export java.awt.Color.* // RED blir medlem som syns utåt via B.RED
  export math.{sin, cos}  // sin och cos blir metoder i B
\end{CodeSmall}

\begin{REPLsmall}
scala> A.RED
-- [E008] Not Found Error: ---------------------------------------------
1 |A.RED
  |^^^^^
  |value RED is not a member of object A

scala> B.RED
val res0: java.awt.Color = java.awt.Color[r=255,g=0,b=0]

scala> (B.cos(0), B.sin(0))
val res1: (Double, Double) = (1.0,0.0)
\end{REPLsmall}
\end{itemize}

\end{Slide}

% \Subsection{Dokumentation}

% \begin{Slide}{Skapa dokumentation}
% \code{scaladoc} är ett program som tar Scala-kod som indata och skapar en webbsajt med dokumentation.

% \vspace{2em}
% \begin{tikzpicture}[node distance=1.5cm,scale=0.8, every node/.style={transform shape}]
% \node (input) [startstop] {\texttt{src/*.scala}};

% \node(inptext) [right of=input, text width=4cm, scale=1.2,xshift=4.5cm]{en katalog \texttt{src} som innehåller \texttt{.scala}-filer};

% \node (scaladoc) [process, below of=input]
% {\texttt{scaladoc}};

% \node (output) [startstop, below of=scaladoc] {\texttt{index.html} ~~~med mera...};

% \node(outtext) [right of=output, text width=4cm, scale=1.2,xshift=4.5cm]{En webbsajt};


% \draw [arrow] (input) -- (scaladoc);
% \draw [arrow] (scaladoc) -- (output);
% \end{tikzpicture}

% \vspace*{1em}Läs mer i Appendix E: Dokumentation.
% %\vspace{2em} För Java-kod finns motsvarande program som heter \code{javadoc}.
% \end{Slide}

\Subsection{Använda färdiga kodbibliotek}

\begin{Slide}{Använda dokumentation för färdiga klasser.}
\begin{itemize}
  \item Dokumentation för standardbiblioteket i Scala finns här:  \\ \url{https://www.scala-lang.org/api/}
  \item Övning: Leta upp dokumentationen för metoden \code{reduceLeft} i klassen \code{Vector}.
  \item[]
  \item Dokumentation för standardbiblioteket i Java finns här:  \\ 
  \url{https://docs.oracle.com/en/java/javase/11/docs/api/index.html}
  %\url{https://docs.oracle.com/javase/8/docs/api/}
  \item Övning: Leta upp dokumentationen för \code{java.awt.Color}
  \item Läs mer i Appendix E om dokumentation.

\end{itemize}
\end{Slide}
  

\begin{Slide}{Vad är en jar-fil?}
\begin{itemize}
  \item Jar-filer används för att distribuera färdigkompilerad kod så att andra kan använda den enkelt
  \item Förkortningen \Emph{jar} kommer från ''Java Archive''
  \item En \Emph{jar}-fil följer ett standardiserat filformat och används för att \Alert{paketera flera filer} i en och samma fil, exempelvis:
  \begin{itemize}
    \item \texttt{.class}-filer med bytekod
    \item resursfiler för en applikation t.ex. bilder \code{.png}, \code{.jpg}, etc
    \item information om vilken klass som innehåller \code{main}-funktionen
    \item etc.
  \end{itemize}
  \item En \code{.jar}-fil komprimeras på samma sätt som en \code{.zip}-fil.
  \item Fördjupning för den intresserade:\\
  {\SlideFontTiny\url{https://en.wikipedia.org/wiki/JAR_(file_format)}}
\end{itemize}
\end{Slide}

\begin{Slide}{Öppen källkod på Maven Central}
\begin{itemize}
  \item På \Emph{Maven Central} som hanteras av företaget Sonatype finns tusentals öppet tillgängliga kodbibliotek publicerade som jarfiler.
  \item Du kan söka bland alla Scala-bibliotek här: \\\url{https://index.scala-lang.org/}
  \item Du kan söka bland alla bibliotek här: \\\url{https://search.maven.org/}
\end{itemize}
\end{Slide}


\begin{Slide}{Vad är \emph{classpath}?}\SlideFontSmall
\begin{itemize}
  \item Hur hittar kompilatorn färdiga moduler?
\pause
\item Kompilatorerna \code{scalac} och \code{javac} och programmen \code{scala-cli} och \code{java} som kör igång JVM använder \Alert{en lista med filsökvägar} kallad \Emph{classpath} när de söker efter kompilerad kod.
\pause
\item Scalas standardbibliotek läggs automatiskt på classpath.
\item Med hjälp av optionen \code{--jar} kan du lägga till en jar-fil till classpath. % genom att ge en lista med sökvägar separerade med kolon\footnote{I Windows skiljs sökvägar med semikolon istället för kolon.}.
\item Exempel: (punkt används för att ange aktuell katalog)
\begin{REPLnonum}
scala-cli run . --jar introprog.jar
\end{REPLnonum}
\end{itemize}
\end{Slide}


\begin{Slide}{Färdiga grafikmetoder i klassen \texttt{PixelWindow}}\SlideFontSmall

\begin{itemize}
\item På labben ska du använda en \texttt{.jar}-fil med kodbiblioteket \code{introprog}.
%\item En \Emph{klass} är en ''mall'' för att göra \Emph{objekt}.
\item Där finns klassen \code{PixelWindow} som kan skapa ritfönster.
%\item När man skapar ett objekt från en klass använder man nyckelordet \code{new}.
\item Du kan starta REPL så här om du har laddat ner jar-filen manuellt från\\\url{https://fileadmin.cs.lth.se/introprog.jar}  %\pause
\begin{REPLnonum}
> scala-cli repl --jar introprog.jar 
\end{REPLnonum}
\item Testa \code{PixelWindow} i REPL med:
\begin{REPLnonum}
scala> val w = introprog.PixelWindow(300, 200, "hejsan")
\end{REPLnonum}
\item Studera dokumentationen för \code{introprog.PixelWindow} här: \\\url{https://fileadmin.cs.lth.se/pgk/api}
\end{itemize}
\end{Slide}


\begin{Slide}{Automatiska beroenden med Scala CLI i REPL:}
\begin{itemize}\SlideFontSmall
\item Du kan istället låta \code{scala-cli} \Emph{automatiskt} ladda ner ett färdigt kodbibliotek som är publicerat på Maven Central och lägga det på classpath med optionen \code{--dep} som är en förkortning av \emph{dependency}. 
\item Notera antalet kolon i adressen till kodbiblioteket:
\begin{REPLsmall}
> scala-cli repl . --dep se.lth.cs::introprog:1.4.0
Welcome to Scala 3.1.3 (17.0.3, Java OpenJDK 64-Bit Server VM).
Type in expressions for evaluation. Or try :help.

scala> introprog.Dialog.show("hello introprog")
\end{REPLsmall}
\end{itemize}
\end{Slide}


\begin{Slide}{Köra program + kodbiblitek med Scala CLI}\SlideFontTiny
\begin{itemize}
\item \code{scala-cli} kan inkludera kodbibliotek från Maven Central om du skriver en ''magisk'' kommentar i början av din \code{.scala-}filen:
\begin{Code}
//> using scala 3.7.2
//> using dep se.lth.cs::introprog:1.4.0

@main def run = introprog.Dialog.show("hello introprog")
\end{Code}
Notera \texttt{>} efter \texttt{//}

\item När du kör ditt program såhär så kommer Scala CLI att ladda ner kodbiblioteket om det inte redan är gjort:
\begin{REPLsmall}
> scala-cli run .
\end{REPLsmall}
\item Läs mer här:\\\url{https://index.scala-lang.org/lunduniversity/introprog-scalalib} och i Appendix C, stycket om Scala CLI. Mer om \code{//> using} här:
\item[] \url{https://scala-cli.virtuslab.org/docs/reference/directives}
\end{itemize}

\end{Slide}

\begin{Slide}{Kompilera om vid varje ändring}\SlideFontSmall
Ange optionen \code{--watch} så körs kommandot om varje gång du sparar en scala-fil med Ctrl+S.
\begin{REPLsmall}
> scala-cli compile . --watch
\end{REPLsmall}
Kan skrivas kortare:
\begin{REPLsmall}
> scala-cli compile . -w
\end{REPLsmall}
Fungerar också för run-kommandot, men det är inte lika användbart om appen är interaktiv och väntar på input från användaren innan den avslutas. 
\begin{REPLsmall}
> scala-cli run . -w
\end{REPLsmall}
\Emph{Gör så små ändringar som möjligt} och kompilera och testa vid \Alert{varje} ändring! Många ändringar kan ge svårhittade följdfel...
    
  
\end{Slide}

% \begin{Slide}{Scala Build Tool: \texttt{sbt}}\SlideFontSmall
% \begin{itemize}
% \item Du kan låta byggverktyget \code{sbt} automatiskt sköta omkompilering och körning vid varje Ctrl+S med kommandot \code{~run}
% och enbart omkompilering med \code{~compile}
% \begin{REPL}
% > sbt
% [info] Set current project to hellosbt (in build file:/home/bjornr/tmp/hellosbt/)
% [info] sbt server started at 127.0.0.1:5320
% sbt:hellosbt> ~run
% \end{REPL}
% Avsluta med Enter. Om flera \code{main} gör: \code{ ~runMain DettaMainObj}
% \item Kod antas finnas direkt i \Emph{aktuell katalog} eller i \code{src/main/scala}
% \item Lägg \code{.jar}-filer i en katalog \code{lib} så hamnar de automatiskt på classpath
% \item Skapa en fil med namnet \code{build.sbt} för att säkerställa rätt version:
% \begin{Code}
% scalaVersion := "3.2.2"
% \end{Code}
% I filen \code{project/build.properties} kan du ange versionen på \code{sbt}:
% \begin{Code}
% sbt.version=1.8.2
% \end{Code}
% Se Appendix F i kompendiet och \url{http://www.scala-sbt.org/}
% \end{itemize}
% \end{Slide}



% \begin{Slide}{Använda \texttt{introprog} tillsammans med \texttt{sbt}}
% Lägg raden med \code{libraryDependencies} i filen \code{build.sbt} efter \code{scalaVersion}:
% \begin{Code}
% scalaVersion := "3.2.2"
% libraryDependencies += "se.lth.cs" %% "introprog" % "1.3.1"
% \end{Code}
% Första gången du kör \code{compile}, \code{run} eller \code{console} i \code{sbt} kommer en jar-fil med paketet \code{introprog} automatiskt att laddas ner från Maven Central.
% \end{Slide}


\ifkompendium\else


\Subsection{Veckans övning och laboration}

\begin{SlideExtra}{Övning \texttt{objects}}\SlideFontTiny
\setlength{\leftmargini}{0pt}
\begin{itemize}
%!TEX encoding = UTF-8 Unicode
%!TEX root = ../exercises.tex

\item Kunna skapa och använda objekt som moduler.
%\item Kunna förklara vad ett block och en lokal variabel är.
%\item Kunna skapa och använda lokala funktioner och förklara nyttan med dessa.
\item Kunna förklara hur nästlade block påverkar namnsynlighet och namnöverskuggning.
\item Kunna förklara begreppen synlighet, privat medlem, namnrymd och namnskuggning.
%https://it-ord.idg.se/ord/overskuggning/

\item Kunna skapa och använda tupler.
\item Kunna skapa funktioner som har multipla returvärden.
\item Kunna förklara den semantiska relationen mellan funktioner och objekt i Scala.

\item Kunna förklara kopplingen mellan paketstruktur och kodfilstruktur.
\item Kunna använda färdiga kodbibliotek i jar-filer.

\item Kunna använda import av medlemmar i objekt och paket.
\item Kunna byta namn vid import.
\item Kunna förklara skillnaden mellan import och export.

\item Kunna skapa och använda variabler med fördröjd initialisering.
%\item Kunna förklara när parenteserna kan skippas runt tupel-argument.

\end{itemize}
\end{SlideExtra}

\begin{SlideExtra}{Lab \texttt{blockmole}}\SlideFontTiny
%\setlength{\leftmargini}{0pt}
\begin{itemize}
%!TEX encoding = UTF-8 Unicode
%!TEX root = ../labs.tex

\item Kunna förklara hur singelobjekt kan användas som moduler.
\item Kunna förklara hur åtkomst av medlemmar i singelobjekt sker.
\item Kunna skapa kod som reagerar på och förändrar objekts tillstånd.
\item Kunna förklara nyttan med att samla namngivna konstanter i egen modul.
\item Kunna förklara hur import påverkar synlighet av namn.
\item Kunna ge exempel på en situation där man har nytta av namnbyte vid import.
\item Kunna redogöra för skillnaden mellan paket och singelobjekt.
\item Kunna skapa och använda tupler.

\end{itemize}

\end{SlideExtra}
\fi
