%!TEX encoding = UTF-8 Unicode
%!TEX root = ../lect-w09.tex

\Subsection{Kodgranskning}


\begin{Slide}{Att läsa kod} \SlideFontSmall
  En förutsättning för att kunna \emph{skriva} bra kod är att kunna \emph{läsa} kod aktivt. Du behöver kontinuerlig träning i att \Emph{läsa} kod! (inte bara skriva)\\ \vspace{0.5em}\textbf{Hur läsa kod?}
  \pause
  \begin{itemize}\SlideFontSmall
    \item Skapa överblick av vilka kodfiler som finns och vad som finns i vilken fil.
    \item Studera dokumentation om vad som är \Emph{syftet} med olika abstraktioner.
    \item Studera klassparametrar och metodhuvuden. \Emph{Typerna} är dina \Alert{tankeverktyg}: ''Vad kommer in?'' och ''Vad kommer ut?''
    \item Ställ dig under läsningen frågorna: \\ ''Vad finns?'' och ''Vad fattas?'' för att du ska kunna lösa din uppgift.
    \item Läs iterativt. Vänta med implementationsdetaljer. Hoppa mellan deklaration och användning.
    \item Studera \Emph{beroenden}: \code{Matrix -> Life -> LifeWindow -> Main}
    \item Två strategier i din stegvisa läsning som kan mixas: \\Bottom-Up: Börja med delar med \Alert{minst} beroenden till andra delar. \\ Top-Down: Börja med de delar som används av \Emph{huvudprogrammet}.
    \item Var aktiv när du läser! Anteckna; skriv ner frågor; experimentera i REPL.
  \end{itemize}
\end{Slide}

\begin{Slide}{Kodgranskning i industriell systemutveckling}
\begin{itemize}
\item Ett effektivt sätt att upptäcka fel är att människor \Emph{noga läser igenom} sin egen och andras kod, och försöker hitta relevanta \Alert{problem och förbättringsmöjligheter}. 
\item Man blir ofta ''hemmablind'' när det gäller ens egen kod. Därför kan någon annans, oberoende granskning med ''nya, friska'' ögon vara mycket fruktbar. 
\item I samband med kodgranskning kan man med fördel försöka bedöma  huruvida koden är:
\begin{itemize}
\item lätt att läsa, 
\item lätt att ändra i,  
\item annat som är viktigt för den framtida utvecklingen.
\end{itemize}
\item Ofta hittar man vid granskning även enkla programmeringsmisstag, så som felaktiga villkor och loop-räknare som inte räknas upp på rätt sätt etc.
\end{itemize}
\end{Slide}

\begin{Slide}{Nyttan med kodgranskningar}
Väl genomförda kodgranskningar är effektiva och nyttiga:
\begin{itemize}
\item Bra på att upptäcka problem, även sådana som varken kompilator eller testning hittar.
\item Sprider kunskap inom en arbetsgrupp, speciellt mellan erfarna och juniora utvecklare.
\item En god kodgranskningsprocess främjar en kultur av gemensamt ägarskap och respektfull samverkan.
\end{itemize}
\end{Slide}


\input{body/lect-w09-inspections-snake.tex}

\ifkompendium\else
\begin{Slide}{Gästföreläsning om kodgranskningar i praktiken}
\begin{itemize}
\item \Alert{Sprid info till alla}: På andra föreläsningen i nästa vecka kommer \Emph{gäst från industrin}: den erfarne utvecklaren \Alert{Gustaf Lundh} från Axis och gästföreläser om kodgranskningar i praktiken. 
\item Gästföreläsningen ger dig en flygande start inför de kodgranskningar du ska genomföra i grupplabben \code{snake}.
\item \Alert{Missa inte det!}
\end{itemize}
\end{Slide}
\fi
