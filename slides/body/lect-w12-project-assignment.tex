%!TEX encoding = UTF-8 Unicode
%!TEX root = ../lect-w12.tex

%%%

\ifkompendium\else
\begin{SlideExtra}{Kom på föreläsning om systemutveckling!}
\begin{itemize}
  \item Tors. \Alert{12/12 kl 15-17 i E:1406} ger Björn Regnell en föreläsning i C:arnas kurs om digitalisering EITA65: 
\begin{itemize}
  \item \Emph{Kravhantering} för mjukvara: hur bestämma vad koda?
  \item Hur bäst dra nytta av \Emph{öppen källkod}?
\end{itemize}
  \item Alla \Alert{D-are, W-are} som går pgk är också \\\Emph{hjärtligt välkomna!!!}
\end{itemize}
  
\end{SlideExtra}
\fi 

\Subsection{Om projektuppgiften}

\begin{Slide}{Om din avslutande projektuppgift}\SlideFontSmall
\Emph{Läs noga kompendium Del 1, kapitel -1 avsnitt ''Projektuppgift''!} \\
Några viktiga punkter:
\begin{itemize}
\item Mål: skapa ett stort program med många samverkande klasser/moduler.
\item Du väljer själv projektuppgift. Börja redan idag att planera ditt arbete!
\item Projektet går över \Alert{två veckor}. Schemalagda labbar i december är \Alert{obligatoriska} för de som har kvar obligatoriska moment.
%\item Undvik \Alert{för simpel} uppgift och att ta dig \Alert{vatten över huvudet}!

\item I kompendium Del 2, kapitel 13, finns flera förslag att välja bland, men du kan också definiera ett eget projekt som passar just dig.

%\item Övning \code{extra} i kapitel 14, där du gör en enkel web-server och experimenterar med trådar, kan vara lämplig som grund för projektuppgift för de som vill fördjupa sig \Alert{bortom} kursens innehåll.

\item Inför redovisningen ska du skapa automatiskt genererad dokumentation utifrån relevanta dokumentationskommentarer för minst hälften av dina publika metoder, enligt instruktioner i Appendix E.

\item Redovisning sker i datorsal på schemalagd tid:
\begin{itemize}\SlideFontTiny
  \item Förklara hur din kod fungerar.
  \item Beskriv framväxten av ditt program.
  \item Gå igenom den genererade dokumentationen av din kod.
\end{itemize}
\end{itemize}

\end{Slide}

\begin{Slide}{Projektuppgifter}\SlideFontTiny

\begin{itemize}\SlideFontTiny
\item \code{bank}
\begin{itemize}\SlideFontTiny
\item känd domän: skapa bank med transaktionshistorik 
\item oföränderlig data tillsammans med tillståndsförändring
\end{itemize}

% \item \code{tabular}
% \begin{itemize}\SlideFontTiny
% \item behandling av data i tabellform 
% \item matris, oföränderlig data
% \end{itemize}

\item \code{music}
\begin{itemize}\SlideFontTiny
\item skapa ett enkelt kompositionsverktyg som spelar musik
\item sätta sig in i en domänmodell
\end{itemize}

\item \code{photo} 
\begin{itemize}\SlideFontTiny
\item en enkel variant av photoshop 
\item inblick i enkel matrismatematik
\end{itemize}


\item egendefinierat projekt 
\begin{itemize}\SlideFontTiny
\item Lagom svårt: ej för enkel uppgift, men ta dig inte vatten över huvudet!
\item Diskutera med en handledare och dokumentera egna uppgiften och dess omfattning och relation till lärandemålen i kursen så att andra handledare och kursansvarig kan förstå varför projektet passar i kursen.
\item Du måste få OK från en handledare innan du startar egendefinierat projekt.
\end{itemize}


\end{itemize}

\end{Slide}

\begin{Slide}{Skapa dokumentation}
Scala CLI kan ta dokumentationskommentarerna i källkoden och skapa en webbsajt med dokumentation.

\vspace{2em}
\begin{tikzpicture}[node distance=1.5cm,scale=0.8, every node/.style={transform shape}]

\node (input) [startstop] {\texttt{.}};

\node(inptext) [right of=input, text width=4cm, scale=1.2,xshift=4.5cm]{Aktuell katalog med \texttt{.scala}-filer};

\node (scaladoc) [process, below of=input]
{\texttt{scala doc .}};

\node (output) [startstop, below of=scaladoc] {\texttt{./scala-doc}};

\node(outtext) [right of=output, text width=4cm, scale=1.2,xshift=4.5cm]{\texttt{index.html} och en massa andra filer som ger en hel webbsajt!};


\draw [arrow] (input) -- (scaladoc);
\draw [arrow] (scaladoc) -- (output);
\end{tikzpicture}

\vspace{1em} Öppna \texttt{scala-doc/index.html} i en webbläsare.
\end{Slide}
  
  

\begin{Slide}{Dokumentationskommentarer}\footnotesize
För att kod ska bli begriplig för människor är det bra att dokumentera vad den gör. Det finns \Emph{tre olika sorters kommentarer}:
\begin{lstlisting}
// Enradskommentarer börjar med dubbla snedstreck
//       men de gäller bara till radslut

/* Flerradskommentarer börjar med
   snedstreck-asterisk
   och slutar med asterisk-snedstreck.  */

/** Dokumentationskommentarer placeras före
 *   t.ex. en funktion och berättar vad den gör
 *   och vad eventuella parametrar används till.
 *   Börjar med snedstreck-asterisk-asterisk.
 *   Varje ny kommentarsrad börjar med asterisk.
 *   Avslutas med asterisk-stjärna.
 */
\end{lstlisting}
Kommentarer påverkar inte hur maskinen exekverar koden, men hjälper människor att använda koden och verktyg att visa hjälp. Se Appendix E.
\end{Slide}

\ifkompendium\else


\Subsection{Avslutning: munta och tenta}


\begin{SlideExtra}{Avslutning och uppsamling}

\begin{itemize}\SlideFontSmall

\item Gör en \Alert{detaljerad} plan dag-för-dag för din kursavslutning.

\item Föreläsningar denna vecka: om projekt, om munta, jämföra Scala--Java.

\item Föreläsningarna nästa vecka består av repetition och tentaträning.

\item Föreläsningarna anpassas efter era önskemål (se \#önska i Discord). % (omröstning: grumligt, nyfiken).

\item Inga föreläsningar sista läsveckan. \Alert{Sista  pgk-föreläsningen} är \LastLectureDate.

\item Schematider i \Alert{sista läsveckan} är till för \Emph{muntligt prov, projektredovisning, redovisning av restlabbar}.
\item 
\item Det är ok att redovisa projekt och göra munta i förtid i mån av plats. \\Men PLUGGA NOGA inför muntan!!! (Se nästa slajd.)

\item Alla labbar klara innan du muntar (OK munta parallellt med projektet)

\item För att få göra den valfria tentan krävs att \Alert{alla} obligatoriska moment (kontrollskrivning, labbar, projekt, muntligt prov) är \Emph{godkända}. 

\item \Emph{Godkänd} grundkurs är \Alert{krav} för att få påbörja efterföljande \code{pfk}.

\item \Emph{Kolla din status i Canvas} på sida med inklippt ögonblicksbild ur vårt system SAM där du ser vad du har \Emph{klarat} och vad du ev. har kvar.

\end{itemize}

\end{SlideExtra}

\begin{SlideExtra}{Obligatoriskt muntligt prov}
\begin{itemize}\SlideFontSmall
  \item Syftet med det muntliga provet är att säkerställa att alla som påbörjar efterföljande kurs har tillräckliga \Emph{grundkunskaper}, med fokus på \Emph{konceptuell förståelse}.
  \item Det muntliga provet tar mellan 10 och 30 min beroende på hur många frågor handledare behöver ställa för att kunna kan avgöra om du har tillräcklig förståelse.
  \item Enda tillåtna hjälpmedel: snabbref, och på anmodan av handledare även REPL. Medtag papper och penna!
  \item \Alert{Alla labbar måste vara godkända} innan du får göra  provet.
  \item Det är ok att göra det muntliga provet även innan projektet är klart, i mån av tid om  det inte är redovisningskö.
  \item Om du är godkänd på alla obligatoriska moment får du minst 3:a i kursen och du har då klarat förkunskapskraven för efterföljande kurser.
  \item Du ber handledare att få göra muntligt prov på samma sätt som du tidigare bett om att få redovisa labbar. Munta sker på plats i datorsal.
  \item Träna på provet här: \url{https://cs.lth.se/pgk/muntabot}
\end{itemize}  
\end{SlideExtra}

\begin{SlideExtra}{Gör den valfria tentan!}
\begin{itemize}\SlideFontSmall
  \item \Emph{Alla} som kan \& vill uppmuntras göra den valfria tentan!
  \item \Alert{Alla} obligatoriska moment måste vara klara innan du får tentera.
  \item Den valfria tentan ger högre betyg om du klarar gränserna för 4:a el. 5:a.
  \item Tentan liknar de extentor som finns på kurshemsidan.
  \item Tentan är på plats och skrivs med papper och penna. 
  \item Enda hjälpmedel: snabbreferensen.
  \item Snabbreferens beställs i Canvas och hämtas hos \url{birger.swahn@cs.lth.se}
  \item Tentan ges en gång om året i januari. Det är tillåtet att ''plussa'' ett senare år om du inte är nöjd med ditt betyg.
  \item \Alert{OBS! Obligatorisk anmälan} i LTH:s ordinarie tentasystem. \\ %Anmälningssystemet är öppet sedan i går och stänger den 14/12.
  \url{https://www.student.lth.se/mina-studier/tentamen/} 
\end{itemize}
\end{SlideExtra}


  

% \Subsection{Grumligt-lådan \& Nyfiken-på-lådan}
% \begin{Slide}{Grumligt-lådan och Nyfiken-på-lådan}
% \begin{itemize}
% \item Skriv lapp i \Alert{GRUMLIGT}-lådan om du har något \Alert{grundläggande begrepp} i kursen som du fortfarande tycker är \Alert{svårt att begripa}.
%
% \item[]
%
% \item Skriv lapp i \Emph{NYFIKEN-PÅ}-lådan om du vill veta mer om något ämne inom programmering och som går \Emph{bortom grunderna}.
% \end{itemize}
% \end{Slide}

\fi