%!TEX encoding = UTF-8 Unicode
%!TEX root = ../lect-w10.tex

%%%

\ifkompendium\else

\Subsection{Gäst: Gustaf Lundh, Axis}

\input{body/lect-w09-inspections-snake.tex}

\begin{Slide}{Gästföreläsning: "Kodgranskningar i praktiken"}
Hjärtligt välkommen: \\ \Emph{Gustaf Lundh}, Senior Developer, Axis
\end{Slide}


\Subsection{Om grupplaborationen \texttt{snake}}

\begin{Slide}{Grupplaboration: \texttt{snake} över 2 veckor}
\begin{minipage}{0.6\textwidth}
\includegraphics[width=1.0\textwidth]{../img/snake-twoplayer}
\end{minipage}%
\begin{minipage}{0.4\textwidth}
\begin{itemize}
\input{../compendium/modules/w10-inheritance-lab-goals.tex}
\end{itemize}
\end{minipage}%
\end{Slide}



\begin{Slide}{Arvshierarki i \texttt{snake}: Olika typer av varelser}
  \begin{center}
  \newcommand{\TextBox}[1]{\raisebox{0pt}[1em][0.5em]{#1}}
  \tikzstyle{umlclass}=[rectangle, draw=black,  thick, anchor=north, text width=2.5cm, rectangle split, rectangle split parts = 3]
  \begin{tikzpicture}[inner sep=0.5em,scale=0.8, every node/.style={transform shape}]

    \node [umlclass, rectangle split parts = 1, xshift=-1.0cm, yshift=4.5cm] (BaseType)  {
                \textit{\textbf{\centerline{\TextBox{\code{Entity}}}}}
  %              \nodepart[align=left]{second}\code{def x: T} \newline \code{def y: T}
            };


    \node [umlclass, rectangle split parts = 1, xshift=-3cm, yshift=2.5cm] (SubType1)  {
                \textit{\textbf{\centerline{\TextBox{\code{CanMove}}}}}
  %              \nodepart[align=left]{second}\code{val x: Int} \newline \code{val y: Int}
            };

  \node [umlclass, rectangle split parts = 1, xshift=-4.5cm, yshift=0.5cm] (SubSubType0)  {
              {\textbf{\centerline{\TextBox{\code{Snake}}}}}
  %            \nodepart[]{second}\TextBox{\code{val dim: Int}}
  };


  \node [umlclass, rectangle split parts = 1, xshift=0.75cm, yshift=2.5cm] (SubType2)  {
              \textit{\textbf{\centerline{\TextBox{\code{CanTeleport}}}}}
  %            \nodepart[]{second}\TextBox{\code{val dim: Int}}
          };

  \node [umlclass, rectangle split parts = 1, xshift=-1.0cm, yshift=0.5cm] (SubSubType1)  {
              {\textbf{\centerline{\TextBox{\code{Apple}}}}}
  %            \nodepart[]{second}\TextBox{\code{val dim: Int}}
          };

  \node [umlclass, rectangle split parts = 1, xshift=2.5cm, yshift=0.5cm] (SubSubType2)  {
              {\textbf{\centerline{\TextBox{\code{Banana}}}}}
  %            \nodepart[]{second}\TextBox{\code{val dim: Int}}
          };


  \draw[umlarrow] (SubType1.north) -- ++(0,0.5) -| (BaseType.south);
  \draw[umlarrow] (SubType2.north) -- ++(0,0.5) -| (BaseType.south);
  \draw[umlarrow] (SubSubType1.north) -- ++(0,0.5) -| (SubType2.south);
  \draw[umlarrow] (SubSubType2.north) -- ++(0,0.5) -| (SubType2.south);
  \draw[umlarrow] (SubSubType0.north) -- ++(0,0.5) -| (SubType1.south);

  \end{tikzpicture}
  \end{center}
\end{Slide}


\begin{Slide}{Arvshierarki i \texttt{snake}: Olika typer av spel}
  \begin{center}
  \newcommand{\TextBox}[1]{\raisebox{0pt}[1em][0.5em]{#1}}
  \tikzstyle{umlclass}=[rectangle, draw=black,  thick, anchor=north, text width=3cm, rectangle split, rectangle split parts = 3]
  \begin{tikzpicture}[inner sep=0.5em,scale=0.8, every node/.style={transform shape}]

    \node [umlclass, rectangle split parts = 1, xshift=0cm, yshift=5cm] (BaseType)  {
                \textit{\textbf{\centerline{\TextBox{\code{BlockGame}}}}}
  %              \nodepart[align=left]{second}\code{def x: T} \newline \code{def y: T}
            };


    \node [umlclass, rectangle split parts = 1, xshift=0cm, yshift=3.0cm] (SubType)  {
                \textit{\textbf{\centerline{\TextBox{\code{SnakeGame}}}}}
  %              \nodepart[align=left]{second}\code{val x: Int} \newline \code{val y: Int}
            };

  \node [umlclass, rectangle split parts = 1, xshift=-3cm, yshift=0.5cm] (SubSubType1)  {
              {\textbf{\centerline{\TextBox{\code{OnePlayerGame}}}}}
  %            \nodepart[]{second}\TextBox{\code{val dim: Int}}
          };

  \node [umlclass, rectangle split parts = 1, xshift=3cm, yshift=0.5cm] (SubSubType2)  {
              {\textbf{\centerline{\TextBox{\code{TwoPlayerGame}}}}}
  %            \nodepart[]{second}\TextBox{\code{val dim: Int}}
          };

  \draw[umlarrow] (SubType.north) -- ++(0,0.5) -| (BaseType.south);
  \draw[umlarrow] (SubSubType1.north) -- ++(0,0.5) -| (SubType.south);
  \draw[umlarrow] (SubSubType2.north) -- ++(0,0.5) -| (SubType.south);

  \end{tikzpicture}
  \end{center}
\end{Slide}

\begin{Slide}{Krav vid respektive gruppstorlek}
Krav som minst ska implementeras vid olika gruppstorlek:

\vspace{0.5em}
  \begin{tabular}{r | c c c c c c}
    \Alert{Krav} / \Emph{Antal personer} & 1       & 2       & 3       & 4       & 5       & 6 \\ \hline
    \texttt{Player}       & $\surd$ & $\surd$ & $\surd$ & $\surd$ & $\surd$ & $\surd$ \\
    \texttt{OnePlayerGame}& $\surd$ &         &         &         & $\surd$ & $\surd$ \\
    \texttt{TwoPlayerGame}&         & $\surd$ & $\surd$ & $\surd$ & $\surd$ & $\surd$ \\
    \texttt{Snake}        & $\surd$ & $\surd$ & $\surd$ & $\surd$ & $\surd$ & $\surd$ \\
    \texttt{Apple}        & $\surd$ &         & $\surd$ & $\surd$ & $\surd$ & $\surd$ \\
    \texttt{Banana}       &         &         &         & $\surd$ &         & $\surd$ \\
    \texttt{Monster}       &         &         & $\surd$  &  &         & $\surd$ \\
    \texttt{Settings}       & $\surd$ & $\surd$ & $\surd$ & $\surd$ & $\surd$ & $\surd$ \\
  \end{tabular}

\vspace{0.5em}
\begin{itemize}\SlideFontSmall
\item Uppgiften lämpar sig bäst för minst 3 gruppmedlemmar. 
\item Prata med handledare: slå ihop två små grupper till en med max 6 st.
\item Om du har särskilda skäl, t.ex. många restlabbar, kan du efter godkännande från kursansvarig göra labben enskilt.

\end{itemize}
\end{Slide}


\begin{Slide}{Instruktioner Grupplaboration (text ur kompendiet)}
\begin{itemize}\SlideFontTiny
\input{../compendium/team-lab-prep-items.tex}
\end{itemize}
\end{Slide}

\begin{Slide}{Redovisning på obligatorisk schemalagd labbtid}
Första \code{snake}-veckan (w10):
\begin{itemize}\SlideFontTiny
\item Redovisa uppdelning av uppgiften mellan gruppmedlemmar och hur långt du har kommit med din del. Visa en skriftlig plan för implementationsarbetet för andra veckan, med valda deluppgifter och sluttidpunkter.
\item Redovisa arbetet med granskningar, minst checklista och planering för arbetet med granskningar i andra veckan. Bra om ni du gjort någon granskning.
\item Jobba vidare med \code{snake} och få hjälp av handledare om du kör fast. 
\end{itemize}
Andra \code{snake}-veckan (w11):
\begin{itemize}\SlideFontTiny
\item Redogör för uppdelningen av uppgiften och avgränsningen av ditt huvudansvar.
\item Förklara övergripande hela kodbasens struktur och syftet med de olika kod-delarna.
\item Förklara mer ingående de delar där du bidragit till implementationen.
\item Gör en kort demonstration av det körande systemet och visa vilka valfria uppgifter ni valt att göra.
\item Beskriv utmaningar med systemutveckling i grupp utifrån dina egna erfarenheter.
\item Redogör för nyttan och utmaningarna med kodgranskningar utifrån dina egna erfarenheter.  
\end{itemize}
\end{Slide}

\begin{Slide}{Övning w10: \texttt{inheritance}}
\begin{itemize}\SlideFontTiny
\input{../compendium/modules/w10-inheritance-exercise-goals.tex}
\end{itemize}
\end{Slide}

\fi
