%!TEX encoding = UTF-8 Unicode
%!TEX root = ../lect-w01.tex

%%%%%%%%%%%%%%%%%%%%%%%%%%%%%%%%%%%%%%

\Subsection{Att lära denna läsvecka \texttt{w01}}

\ifkompendium\else  %%%%%%%%%%%%%%%%%%%%%%%%%%%%%%%%%%%%%%%%%%%%%%%%%
\begin{SlideExtra}{Att lära denna läsvecka \texttt{w01}}
%!TEX encoding = UTF-8 Unicode

    Modul \Emph{Introduktion}: Övn \Alert{\texttt{expressions}} $\rightarrow$ Labb \Alert{\texttt{kojo}}
    \begin{multicols}{3}\SlideFontTiny
    $\square$ sekvens \\
$\square$ alternativ \\
$\square$ repetition \\
$\square$ abstraktion \\
$\square$ editera \\
$\square$ kompilera \\
$\square$ exekvera \\
$\square$ datorns delar \\
$\square$ virtuell maskin \\
$\square$ litteral \\
$\square$ värde \\
$\square$ uttryck \\
$\square$ identifierare \\
$\square$ variabel \\
$\square$ typ \\
$\square$ tilldelning \\
$\square$ namn \\
$\square$ val \\
$\square$ var \\
$\square$ def \\
$\square$ definiera och anropa funktion \\
$\square$ funktionshuvud \\
$\square$ funktionskropp \\
$\square$ procedur \\
$\square$ inbyggda grundtyper \\
$\square$ println \\
$\square$ typen Unit \\
$\square$ enhetsvärdet () \\
$\square$ stränginterpolatorn s \\
$\square$ aritmetik \\
$\square$ slumptal \\
$\square$ logiska uttryck \\
$\square$ de Morgans lagar \\
$\square$ if \\
$\square$ true \\
$\square$ false \\
$\square$ while \\
$\square$ for \\
$\square$ dod: operativsystem \\
    \end{multicols}
    
\end{SlideExtra}
\fi

\Subsection{Om programmering}

\ifkompendium\else  %%%%%%%%%%%%%%%%%%%%%%%%%%%%%%%%%%%%%%%%%%%%%%%%%

\begin{SlideExtra}{Att skapa koden som styr världen}
\begin{multicols}{2}\footnotesize
I stort sett \Alert{alla} delar av samhället är beroende av programkod:
\begin{itemize}\scriptsize
\item kommunikation
\item transport
\item byggsektorn
\item statsförvaltning
\item finanssektorn
\item media \& underhållning
\item sjukvård
\item övervakning
\item integritet
\item upphovsrätt
\item miljö \& energi
\item sociala relationer
\item utbildning
\item ...
\end{itemize}
\columnbreak %---------
Hur blir ditt framtida yrkesliv som systemutvecklare?
\begin{itemize}
\item  Det är sedan lång tid en \Alert{skriande brist} på utvecklare och det blir bara värre... \\
\href{https://cio.idg.se/2.1782/1.710012/kompetenslarm-jobb-om-fem-ar?queryText=kompetensbrist}{CIO 2018-11-09} %\\
%\href{http://computersweden.idg.se/2.2683/1.663879/oppen-kallkod-brist-kompetens}{CS 2016-08-23} 

\item Störst brist är det på \Emph{kvinnliga} utvecklare: \\
\href{https://www.svt.se/nyheter/inrikes/stor-brist-pa-programmerare-kvinnor-lockas-till-yrket}{SVT 2016-12-03}

\item Global kompetensmarknad \\
  \href{https://cio.idg.se/2.1782/1.648294/hitta-it-kompetens/sida/2/global-rekrytering-aktivt-hr-arbete}{CIO 2016-02-01}\\
  \href{http://computersweden.idg.se/2.2683/1.630901/det-finns-programmerare-och-sa-finns-det-programmerare}{CS 2015-06-14} \\
  \href{http://computersweden.idg.se/2.2683/1.662186/25-miljoner-utvecklare?queryText=miljoner\%20utvecklare}{CS 2016-07-14 }
\end{itemize}
\end{multicols}

\end{SlideExtra}

\begin{SlideExtra}{Utveckling av mjukvara i praktiken}
\begin{itemize}
\item \Emph{Inte bara kodning:} kravbeslut, releaseplanering, design, test, versionshantering, kontinuerlig integration, driftsättning, återkoppling från dagens användare, ekonomi \& investering, gissa om morgondagens användare, ...
\item \Emph{Teamwork:} Inte ensamma hjältar utan autonoma team i decentraliserade organisationer med innovationsuppdrag
\item \Emph{Snabbhet:} Att koda innebär att hela tiden uppfinna nya ''byggstenar'' som ökar organisationens förmåga att snabbt skapa värde med hjälp av mjukvara. \Alert{Öppen källkod}. Skapa kraftfulla API:er.
\item \Emph{Livslångt lärande:} Lär nytt och dela med dig hela tiden. Exempel på pedagogisk utmaning: hjälp andra förstå och använda ditt API $\implies$ \Alert{Samarbetskultur}
\end{itemize}
\end{SlideExtra}


\fi %%%%%%%%%%%%%%%%%%%%%%%%%%%%%%%%%%%%%%%%%%%%%%%%%%%%


% \ifkompendium\else
% \SlideImg{Programming unplugged: Två frivilliga?}{../img/unplugged}
% \SlideImg{Editera och exekvera ett program}{../img/kojo}
% \fi

%%%

\ifkompendium\else
\SlideImg{Vad är en dator?}{../img/eniac}
\fi

\begin{Slide}{Hur fungerar en dator?}
\begin{tikzpicture}[node distance=2.0cm]
\node (input)  [startstop]               {Indata-enhet};
\node (cpu)    [process, below of=input] {CPU};
\node (output) [startstop,below of=cpu]  {Utdata-enhet};

\node (mem) [right of=cpu, xshift=0.4\textwidth, draw = black, thick] {
\begin{minipage}{0.5\textwidth}\centering
\textbf{Minne} med minnesceller
\vspace{1em}

\begin{tabular}{|l | l|}
address & innehåll \\ \hline
0   & 42 \\ \hline
1   & 13 \\ \hline
2   & 18 \\ \hline
3   & 21 \\ \hline
4   & 55 \\ \hline
5   & 64 \\ \hline
6   & 48 \\ \hline
... & ...
\end{tabular}
\end{minipage}
};

\draw [arrow] (input) -- (cpu);
\draw [arrow] (cpu) -- (output);
\draw [arrow] (cpu) -- (mem);
\draw [arrow] (mem) -- (cpu);

\end{tikzpicture}

{\hfill
\begin{minipage}{0.65\textwidth}\vspace{1em}
Minnet innehåller endast \Alert{heltal} som \newline representerar \Emph{data} \Alert{och} \Emph{instruktioner}.
\end{minipage}
}
\end{Slide}

\begin{Slide}{Vad är programmering?}
\begin{itemize}
\item Programmering innebär att ge instruktioner till en maskin.
\item Ett \Emph{programmeringsspråk} används av människor för att skriva \Emph{källkod} som kan översättas av en \Emph{kompilator} till \Emph{maskinspråk} som i sin tur \Emph{exekveras} av en dator.
\end{itemize}


\begin{minipage}{.8\textwidth}
\begin{itemize}
\item Ada Lovelace publicerade det första programmet redan på 1800-talet ämnat för en kugghjulsdator.
\end{itemize}
\end{minipage}%
\begin{minipage}{.2\textwidth}
\centering\includegraphics[width=0.6\columnwidth]{../img/ada}
\end{minipage}%
\begin{itemize}
\item \href{https://sv.wikipedia.org/wiki/Programmering}{sv.wikipedia.org/wiki/Programmering}
\item \href{https://en.wikipedia.org/wiki/Computer\_programming}{en.wikipedia.org/wiki/Computer\_programming}
\item Ha picknick i \href{http://kartor.lund.se/wiki/lundanamn/index.php/Ada_Lovelace-parken}{Ada Lovelace-parken} på Brunnshög!
\end{itemize}
\end{Slide}


\begin{Slide}{Vad är en kompilator?}
\begin{minipage}{.35\textwidth}
\includegraphics[width=0.6\textwidth]{../img/grace}
\end{minipage}%
\begin{minipage}{.67\textwidth}
%https://www.sharelatex.com/blog/2013/08/29/tikz-series-pt3.html
\begin{tikzpicture}[node distance=1.4cm,scale=0.8]
  \node (input) [startstop] {Källkod};
  \node(inptext) [right of=input, text width=2cm, scale=0.8,xshift=1.5cm]{För\\människor};
  \node (compile) [process, below of=input] {Kompilator};
  \node (output) [startstop, below of=compile] {Maskinkod};
  \node(outtext) [right of=output, text width=2cm, scale=0.8,xshift=1.5cm]{För\\maskiner};
  \draw [arrow] (input) -- (compile);
  \draw [arrow] (compile) -- (output);
  \end{tikzpicture}%

\vspace{1em}\noindent Grace Hopper uppfann kompilatorn 1952. \\ \href{https://en.wikipedia.org/wiki/Grace\_Hopper}{\SlideFontTiny en.wikipedia.org/wiki/Grace\_Hopper}
\end{minipage}%
\end{Slide}


\begin{Slide}{Virtuell maskin (VM) == abstrakt hårdvara}
\begin{multicols}{2}
\begin{itemize}
\item En VM är en ''dator'' implementerad i mjukvara som kan tolka en abstrakt ''maskinkod'' som \Emph{översätts under körning} till den \Alert{verkliga} maskinens konkreta maskinkod.

\item Med en VM blir källkoden \Emph{plattformsoberoende} och fungerar på många olika maskiner.

\item Exempel JVM: \\ \Emph{Java Virtual Machine}


\end{itemize}

\columnbreak %---------

%https://www.sharelatex.com/blog/2013/08/29/tikz-series-pt3.html
\begin{tikzpicture}[node distance=1.4cm]
\node (input) [startstop] {Källkod};
\node (compile) [process, below of=input] {Kompilator};
\node (output) [startstop, below of=compile] {Abstrakt ''maskinkod''};
\node (interp) [process, below of=output] {VM interpreterar};
\node (output2) [startstop, below of=interp] {Konkret maskinkod};
\draw [arrow] (input) -- (compile);
\draw [arrow] (compile) -- (output);
\draw [arrow] (output) -- (interp);
\draw [arrow] (interp) -- (output2);
\end{tikzpicture}
\end{multicols}
\end{Slide}

\begin{Slide}{Vad består ett program av?}
\begin{itemize}
\item Text som följer entydiga språkregler (grammatik):
\begin{itemize}
\item \Emph{Syntax}: textens konkreta utseende
\item \Emph{Semantik}: textens betydelse (vad maskinen gör/beräknar)
\end{itemize}
\item \Emph{Nyckelord}: ord med speciell betydelse, t.ex. \code{if}, \code{while}
\item \Emph{Deklarationer}: definitioner av nya ord: \code{def gurka = 42}
\item \Emph{Satser} är instruktioner som \Alert{gör} något: \code{print("hej")}
\item \Emph{Uttryck} är instruktioner som beräknar ett \Alert{resultat}: \code{1 + 1}
\item \Emph{Data} är information som behandlas: t.ex. heltalet \code{42}
\item Instruktioner ordnas i kodstrukturer: \Alert{SARA}
\begin{itemize}
  \item \Emph{Sekvens}: ordningen spelar roll för vad som händer
  \item \Emph{Alternativ}: olika resultat beroende på uttrycks värde
  \item \Emph{Repetition}: instruktioner upprepas många gånger
  \item \Emph{Abstraktion}: nya byggblock skapas för att återanvändas
\end{itemize}
\end{itemize}
\end{Slide}

\begin{Slide}{Exempel på programmeringsspråk}
Det finns massor med olika språk och det kommer ständigt nya.
\vspace{1em}
\begin{multicols}{2}
\begin{itemize}
\item Java
\item C
\item C++
\item C\#
\item Python
\item JavaScript
\item Scala
\end{itemize}

\columnbreak %---------

Några topplistor:
\begin{itemize}
\item \href{https://redmonk.com/sogrady/2023/05/16/language-rankings-1-23/}{Redmonk}   
%\item \href{https://redmonk.com/sogrady/2022/03/28/language-rankings-1-22/}{Redmonk}   
%\item \href{https://redmonk.com/sogrady/2020/07/27/language-rankings-6-20/}{Redmonk language rankings}   
%\item \href{https://redmonk.com/sogrady/2019/07/18/language-rankings-6-19/}{Redmonk language rankings}   
%\item \href{https://redmonk.com/sogrady/2018/08/10/language-rankings-6-18/}{Redmonk language rankings}   
\item \href{http://pypl.github.io/PYPL.html}{PYPL}
\item \href{http://www.tiobe.com/index.php/content/paperinfo/tpci/index.html}{TIOBE}
\end{itemize}
% \vspace{1em}
% \includegraphics[width=0.8\columnwidth]{../img/pypl}\\\SlideFontSmall{[PYPL (2016)]}
\end{multicols}

\end{Slide}

\ifkompendium\else
\begin{SlideExtra}{Redmonk Language Rankings: Github, Stackoverflow}
\includegraphics[width=0.80\columnwidth]{../img/redmonk-Q123}
\end{SlideExtra}
% \begin{SlideExtra}{Några språks utveckling över tid enl. PYPL}
% \TODO\includegraphics[width=0.95\columnwidth]{../img/pypl-06-20}
% \end{SlideExtra}

\fi

\begin{Slide}{Olika programmeringsparadigm}
\begin{itemize}
\item Det finns många olika \href{https://en.wikipedia.org/wiki/Programming_paradigm}{programmeringsparadigm} (sätt att programmera på), till exempel:
\begin{itemize}\SlideFontSmall
\item \Emph{imperativ programmering:} programmet är uppbyggt av satser som påverkar systemets tillstånd
\item \Emph{objektorienterad programmering:} en sorts imperativ programmering där programmet består av objekt som kapslar in data och erbjuder operationer som bearbetar dessa data
\item \Emph{funktionsprogrammering:} programmet är uppbyggt av samverkande funktioner som undviker förändringar av data
\item \Emph{deklarativ programmering, logikprogrammering:} programmet är uppbyggt av logiska uttryck som beskriver olika fakta eller villkor och exekveringen utgörs av en bevisprocedur som söker efter värden som uppfyller fakta och villkor
\end{itemize}
\end{itemize}
Denna kurs behandlar de tre första.
\end{Slide}


\begin{Slide}{Hello world}\SlideFontSmall

Kör rad för rad i Scala REPL (Read-Evaluate-Print-Loop):
\begin{REPLsmall}
> scala-cli repl
Welcome to Scala 3.3.0 (17.0.8, Java OpenJDK 64-Bit Server VM).
Type in expressions for evaluation. Or try :help.

scala> println("Hello World!")
Hello World!
\end{REPLsmall}

\pause
\noindent \code{@main} framför valfri funktion anger var ett fristående program ska starta:
\begin{Code}
@main def hi = println("Hello world!")
\end{Code}
\noindent Spara texten ovan i filen \code{hello.scala} och kompilera ditt program:
\begin{REPLsmall}
> scala-cli compile hello.scala
\end{REPLsmall}
Kör ditt program:
\begin{REPLsmall}
> scala-cli run hello.scala
Hello World!
\end{REPLsmall}
Det räcker med \code{scala-cli run} då koden kompileras automatiskt vid behov.
\end{Slide}

\begin{Slide}{Utvecklingscykeln}
editera; kompilera; hitta fel och förbättringar; editera; kompilera; hitta fel och förbättringar; editera; kompilera; hitta fel och förbättringar; editera; kompilera; hitta fel och förbättringar; editera; kompilera; hitta fel och förbättringar; editera; kompilera; hitta fel och förbättringar; ...

\begin{Code}
upprepa(1000){
  editera
  kompilera
  testa
}
\end{Code}
\end{Slide}

\begin{Slide}{Utvecklingsverktyg}
\begin{itemize}\SlideFontSmall
\item Din verktygskunskap är mycket viktig för din produktivitet.
\item Lär dig kortkommandon för vanliga handgrepp.
\item Alla verktyg som behövs finns förinstallerade på \Emph{LTH:s linuxdatorer}. Om din egen burk krånglar: kör på skolans burkar så du ej fördröjs!

\item Verktyg vi använder i kursen:
\begin{itemize}\SlideFontTiny
\item Scala \Emph{REPL}: från övn 1
\item Barnvänlig Scala-programmering med Kojo: Lab 1
\item \Emph{Texteditor} för kod, t.ex \Emph{VS code}: från övn 2
\item Kompilera och kör fristående program med \Emph{\texttt{scala-cli}}: från övn 2
\end{itemize}
\item Andra verktyg som är bra att lära sig: \\(ingår i EDAA60 Datorer och datoranvändning)
\begin{itemize}\SlideFontTiny
\item Git för versionshantering
\item GitHub för kodlagring -- men \Alert{inte} av lösningar till labbar!
\item Linux/Ubuntu och nyttiga terminalkommando
\end{itemize}
\end{itemize}
\end{Slide}

\begin{Slide}{Installera verktyg på din egen dator}
  När du ska skriva kod i en editor, kompilera i terminalen och köra ditt program som en \Emph{fristående applikation}, så behövs: 
  \begin{itemize}
    \item En editor: \Emph{VS Code} med tillägget \Alert{Scala (Metals)}
    \item Körmiljön \Emph{OpenJDK} 
    \item Kommandoverktyg för terminalen: \Alert{\texttt{scala-cli}}
    \item Se instruktioner här: \url{http://cs.lth.se/pgk/verktyg}
    \item Läs mer i Appendix C.
    \item Tips om du kör Windows: installera %\href{https://docs.microsoft.com/en-us/windows/terminal/get-started}
    {nya Windows Terminal}

    \item Installationshjälp:
    \begin{enumerate}
      \item \Emph{Install}\Alert{Lunch}: E:2116, 12-13, w01:tis-fre, w02:mån-fre 
      \item Pluggkvällar som SRD ordnar.
      \item \texttt{\#frågor-och-svar} på vår Discord-server
      \item Fråga handledare på resurstid (i mån av tid).
    \end{enumerate}
    
  \end{itemize}
\end{Slide}

\begin{Slide}{Scala Command Line Interface (CLI)}
\begin{itemize}
\item Utvecklingen av ett nytt kommandogränssnitt \Eng{Command Line Interface (CLI)} för Scala startades 2022 i ett öppenkällkodsprojekt som leds av Virtuslab. 
\item Under 2023 ersätter \Alert{\texttt{scala-cli}} det gamla \Emph{\texttt{scala}}-kommandot.\footnote{I skrivande stund så har skiftet ännu inte skett. När det sker kan du ersätta alla förekomster av \code{scala-cli} med det kortare \code{scala}.}
\item Läs mer i Appendix C och F, samt här: \url{https://scala-cli.virtuslab.org/}
\item Du kan se vad Scala CLI kan göra via hjälp-optionen:
\begin{REPLnonum}
> scala-cli help
\end{REPLnonum}
\end{itemize}
\end{Slide}


\begin{Slide}{Tips och trix med \texttt{scala} i terminalen}\SlideFontSmall
\begin{itemize}
\item Skriv \code{:help} i REPL så får du se vilka \Emph{kommando} som finns.
\item Du kan \Emph{avsluta} REPL med \code{:q} eller trycka Ctrl+D.
\item Ett \Emph{vertikalstreck visas} om du trycker ENTER mitt i en ofullständigt rad. Detta indikerar att du kan fortsätta skriva på ny rad innan tolkning sker.
\item Om du vill att REPL ska vänta att tolka raden du skrivit och istället ge dig \Emph{ännu en rad}, så tryck först ner ESC-tangenten och sedan ENTER. 
\item Om du vill förhindra att REPL ger ny rad efter ENTER vid ofullständig rad, så skriv ett \Emph{semikolon} och tryck ENTER.
\item Starta \code{repl} med punkt efter blanktecken om du vill ha tillgång till koden i alla scala-filer i \Emph{aktuell katalog} i din REPL-session:\\ 
\code{scala-cli repl .}
\item Kör med punkt efter blanktecken så kompileras och exekveras alla scala-filer i \Emph{aktuell katalog} \Alert{och} eventuella \Emph{underkataloger}:\\
\code{scala-cli run .}
\end{itemize}
\end{Slide}


\Subsection{De enklaste beståndsdelarna: litteraler, uttryck, variabler}


\begin{Slide}{Litteraler}
\begin{itemize}
\item En litteral representerar ett fixt \Emph{värde} i koden och används för att skapa \Alert{data} som programmet ska bearbeta.
\item Exempel: \\
\begin{tabular}{l l}
\code|42| & heltalslitteral\\
\code|42.0| & decimaltalslitteral\\
\code|'!'| & teckenlitteral, omgärdas med 'enkelfnuttar' \\
\code|"hej"| & stränglitteral, omgärdas med ''dubbelfnuttar'' \\
\code|true| & litteral för sanningsvärdet ''sant''\\
\end{tabular}
\item Literaler har en \Emph{typ} som avgör vad man kan göra med dem.
\end{itemize}
\end{Slide}

\begin{Slide}{Exempel på inbyggda datatyper i Scala}\SlideFontSmall
\begin{itemize}
\item Alla värden, uttryck och variabler har en \href{https://sv.wikipedia.org/wiki/Datatyp}{\Emph{datatyp}}, t.ex.:
\begin{itemize}\footnotesize
\item \code{Int} för heltal
\item \code{Long} för \textit{extra} stora heltal (tar mer minne)
\item \code{Double} för decimaltal, så kallade flyttal med flytande decimalpunkt
\item \code{String} för strängar
\end{itemize}

\item Kompilatorn håller reda på att uttryck kombineras på ett \Emph{typsäkert} sätt. Annars blir det \Alert{kompileringsfel}.

\item Scala och Java är s.k. \href{https://sv.wikipedia.org/wiki/Typsystem}{\Emph{statiskt typade}} språk, vilket innebär att kontroll av typinformation sker vid kompilering \Eng{compile time}\footnote{Andra språk, t.ex. Python och Javascript är \Emph{dynamiskt typade} och där skjuts typkontrollen upp till körningsdags \Eng{run time} \\ Vilka är för- och nackdelarna med statisk vs. dynamisk typning?}.

\item Scala-kompilatorn gör \href{https://en.wikipedia.org/wiki/Type_inference}{\Emph{typhärledning}}: man \Alert{slipper skriva typerna} om kompilatorn kan lista ut dem med hjälp av typerna hos deluttrycken.

\end{itemize}
\end{Slide}


\begin{Slide}{Grundtyper i Scala}\SlideFontSmall
Dessa \Emph{grundtyper} \Eng{basic types} finns inbyggda i Scala:

\begin{table}[H]
\renewcommand{\arraystretch}{1.4}
\begin{tabular}{p{0.24\textwidth}|p{0.21\textwidth}|l}
\textit{Svenskt namn} & \textit{Engelskt namn} & \Emph{Grundtyper} \\ \hline
heltalstyp & integral type & \texttt{Byte}, \texttt{Short}, \texttt{Int}, \texttt{Long}, \texttt{Char} \\
flyttalstyp  &  floating point \newline number types & \texttt{Float}, \texttt{Double} \\
numeriska typer & numeric types & heltalstyper och flyttalstyper \\
strängtyp \newline (teckensekvens) & string type & \texttt{String}  \\
sanningsvärdestyp  \newline (boolesk typ)& truth value type & \texttt{Boolean} \\
\end{tabular}
\end{table}

\end{Slide}

\begin{Slide}{Grundtypernas omfång}\SlideFontSmall
\begin{table}[H]
\renewcommand{\arraystretch}{1.4}
\begin{tabular}{l|l|l}
\Alert{Grundtyp}& Antal bitar &  Omfång: minsta \& största värde\\\hline
\texttt{Byte}   &  8  & $-2^7$ ... $2^7-1$    \\
\texttt{Short}  &  16 & $-2^{15}$ ... $2^{15}-1$ \\
\texttt{Char}   &  16 & $0$ ... $2^{16}-1$ \\
\texttt{Int}    &  32 & $-2^{31}$ ... $2^{31}-1$ \\
\texttt{Long}   &  64 & $-2^{63}$ ... $2^{63}-1$ \\
\texttt{Float}  &  32 & ± $3.4028235 \cdot 10^{38}$ \\
\texttt{Double} &  64 & ± $1.7976931348623157 \cdot 10^{308}$ \\
\end{tabular}
\end{table}

Grundtypen \texttt{String} lagras som en \emph{sekvens} av 16-bitars tecken av typen \texttt{Char} och kan vara av godtycklig längd (tills minnet tar slut).

\end{Slide}


\begin{Slide}{Uttryck}
\begin{itemize}
\item Ett \Emph{uttryck} består av en eller flera delar som efter \Emph{evaluering} ger ett \Alert{resultat}.
\item Delar i ett uttryck kan t.ex. vara: \\ litteraler (42), operatorer (+), funktioner (sin), ...
\item Exempel:
\begin{itemize}
\item Ett enkelt uttryck: \\ \code{42.0}
\item Sammansatta uttryck: \\
\code{40 + 2} \\
\code{(20 + 1) * 2} \\
\code{sin(0.5 * Pi)} \\
\code{"hej" + " på " + "dej"}
\end{itemize}

\item När programmet tolkas sker \Emph{evaluering} av uttrycket, vilket ger ett resultat i form av ett \Emph{värde} som har en \Emph{typ}.
\end{itemize}
\end{Slide}


\begin{Slide}{Variabler}\SlideFontSmall
\begin{itemize}
\item En \Emph{variabel} kan tilldelas värdet av ett enkelt eller sammansatt uttryck.
\item En variabel har ett \Emph{variabelnamn}, vars utformning följer språkets regler för s.k. \Emph{identifierare}.
\item En ny variabel införs i en \Emph{variabeldeklaration} och då den kan ges ett värde, \Emph{initialiseras}. Namnet användas som \Emph{referens} till värdet.
\item Exempel på variabeldeklarationer i Scala, notera \Emph{nyckelordet} \code{val}:
\begin{Code}
val a = 0.5 * Pi
val length = 42 * sin(a)
val exclamationMarks = "!!!"
val greetingSwedish = "Hej på dej" + exclamationMarks
\end{Code}

\item Vid exekveringen av programmet lagras variablernas värden i minnet och deras respektive värde hämtas ur minnet när de \Emph{refereras}.

\item Variabler som deklareras med \code{val} kan endast tilldelas ett värde \Alert{en enda gång}, vid den initialisering som sker vid deklarationen.
\end{itemize}

\end{Slide}


\begin{Slide}{Regler för identifierare}
\begin{itemize}
\item \Emph{Enkel} identifierare: t.ex. \code{gurka2Tomat}
\begin{itemize}
\item Börja med bokstav
\item ...följt av bokstäver eller siffror
\item Kan även innehålla understreck
\end{itemize}

\item \Emph{Operator}-identifierare, t.ex. \code{+:}
\begin{itemize}
\item Börjar med ett \Emph{operatortecken}, t.ex. \code{+ - * / : ? ~ #}
\item Kan följas av fler operatortecken
\end{itemize}


\item En identifierare får \Alert{inte} vara ett \Emph{reserverat ord}, se \href{http://cs.lth.se/pgk/quickref}{snabbreferensen} för alla reserverade ord i Scala.

\item \Emph{Bokstavlig} identifierare: \code{`kan innehålla allt`}
\begin{itemize}
\item Börjar och slutar med \Emph{backticks}  \code{` `}
\item Kan innehålla vad som helst (utom backticks)
\item Kan användas för att undvika krockar med reserverade ord: \texttt{\code{`}val\code{`}}
\end{itemize}

\end{itemize}
\end{Slide}


\begin{Slide}{Att bygga strängar: konkatenering och interpolering}
\begin{itemize}
\item Man kan \Emph{konkatenera} strängar med operatorn + \\ \code{"hej" + " på " + "dej"}
\item Efter en sträng kan man konkatenera vilka uttryck som helst; uttryck inom parentes evalueras först och värdet görs sen om till en sträng före konkateneringen:
\begin{Code}
val x = 42
val msg = "Dubbla värdet av " + x + " är " + (x * 2) + "."
\end{Code}
\item Man kan i Scala få hjälp av kompilatorn att övervaka bygget av strängar med \Emph{stränginterpolatorn} \Alert{s}:
\begin{Code}
val msg = s"Dubbla värdet av $x är ${x * 2}."
\end{Code}

\end{itemize}
\end{Slide}

\begin{Slide}{Heltalsaritmetik}\SlideFontSmall
\begin{itemize}
\item De fyra räknesätten skrivs som i matematiken (vanlig \href{https://en.wikipedia.org/wiki/Order_of_operations}{precedens}):
\begin{REPL}
scala> 3 + 5 * 2 - 1
res0: Int = 12
\end{REPL}
\item \Emph{Parenteser} styr \Alert{evalueringsordningen}:
\begin{REPL}
scala> (3 + 5) * (2 - 1)
val res1: Int = 8
\end{REPL}
\item \Emph{Heltalsdivision} sker med \Alert{decimaler avkortade}:
\begin{REPL}
scala> 41 / 2
val res2: Int = 20
\end{REPL}
\item \href{https://en.wikipedia.org/wiki/Modulo_operation}{\Emph{Moduloräkning}} med restoperatorn \code{%}
\begin{REPL}
scala> 41 % 2
val res3: Int = 1
\end{REPL}
\end{itemize}
\end{Slide}


\begin{Slide}{Flyttalsaritmetik}\SlideFontTiny
\begin{itemize}
\item Decimaltal representeras med s.k. \href{https://sv.wikipedia.org/wiki/Flyttal}{\Emph{flyttal}} av typen \code{Double}:
\begin{REPL}
scala> math.Pi
val res4: Double = 3.141592653589793
\end{REPL}

\item Stora tal så som $\pi*10^{12}$ skrivs:
\begin{REPL}
scala> math.Pi * 1E12
val res5: Double = 3.141592653589793E12
\end{REPL}
\item Det finns \Alert{inte} oändligt antal decimaler vilket ger problem med \Alert{avvrundingsfel}:
\begin{REPL}
scala> 0.1 + 0.2
val res6: Double = 0.30000000000000004

scala> 1E10 + 0.0000000000001
val res7: Double = 1.0E10

scala> BigDecimal("0.1") + BigDecimal("0.2")  // BigDecimal funkar
val res8: BigDecimal = 0.3
\end{REPL}
\end{itemize}
\SlideFontTiny\vspace{-0.2em}\hfill Läs mer här: \url{https://0.30000000000000004.com}
\end{Slide}

% \ifkompendium\else
% \begin{SlideExtra}{På rasten}
% En per grupp kommer fram hit och tar en grupp-skylt
% \begin{itemize}
%   \item Samlas i era samarbetsgrupper i foajen
%   \item D1.01a längst västerut (mot havet), D1.12b längst österut
%   \item Lär allas namn
%   \item Bestäm tid för första möte
%   \item Vid behov: \\ Bestäm vem som mejlar till de i gruppen som inte är här idag 
% \end{itemize}  
% \end{SlideExtra}
% \fi

\Subsection{Funktioner}

\begin{Slide}{Definiera namn på uttryck}
\begin{itemize}
\item Med nyckelordet \code{def} kan man låta ett \Emph{namn} betyda samma sak som ett \Emph{uttryck}.
\item Exempel:
\begin{Code}
def gurklängd = 42 + x
\end{Code}
\item Uttrycket till höger evalueras \Alert{varje} gång \Emph{anrop} sker,\\
d.v.s. varje gång namnet används på annat ställe i koden.
\begin{Code}
gurklängd
\end{Code}

\end{itemize}
\end{Slide}

\begin{Slide}{Funktion, argument, parameter}\SlideFontSmall
\begin{itemize}
\item En \Emph{funktion} räknar ut \Alert{resultat} baserat på indata som kallas \Emph{argument}.

\item Argument ges namn genom deklaration av \Emph{parametrar}.

\item Exempel på deklaration av en funktion med en parameter:
\begin{Code}
def dubblera(x: Int) = 2 * x
\end{Code}

\item Parametrarnas typ \Alert{måste} beskrivas efter \Emph{kolon}.
\item Kompilatorn kan härleda \Emph{returtypen}, men den kan också med fördel, för tydlighetens skull, anges \Alert{explicit}:
\begin{Code}
def dubblera(x: Int): Int = 2 * x
\end{Code}

\item Observera att namnet \code{x} blir ett ''nytt fräscht'' \Emph{lokalt namn} som \Alert{bara finns och syns  ''inuti'' funktionen} och har inget med ev. andra \code{x} utanför funktionen att göra.

\item Beräkningen sker först vid \Alert{anrop} av funktionen:
\begin{REPL}
scala> dubblera(42)
res1: Int = 84
\end{REPL}

\end{itemize}
\end{Slide}






\begin{Slide}{Färdiga matte-funktioner i paketet \texttt{scala.math}}\SlideFontSmall
\begin{itemize}
\item I paketet \texttt{\Emph{scala.math}} finns många användbara funktioner: t.ex.\\
\code{math.random()} ger slumptal mellan \code{0.0} och \code{0.99999999999999999}
\begin{REPLnonum}
scala> val x = math.random()
x: Double = 0.27749191749889635

scala> val length = 42.0 * math.sin(math.Pi / 3.0)
length: Double = 36.373066958946424
\end{REPLnonum}

\item Studera dokumentationen här: \\{\SlideFontTiny
\url{http://www.scala-lang.org/api/current/scala/math/}}

\item Paketet \code{scala.math} delegerar ofta till Java-klassen \texttt{\Emph{java.lang.Math}} som är dokumenterad här: \\{\SlideFontTiny
\url{https://docs.oracle.com/javase/8/docs/api/java/lang/Math.html}}

\end{itemize}
\end{Slide}



\Subsection{Logik}

\begin{Slide}{Logiska uttryck}\SlideFontSmall
\begin{minipage}{.8\textwidth}
\begin{itemize}
\item Datorn kan ''räkna'' med sanning och falskhet: \\
s.k. \href{https://en.wikipedia.org/wiki/Boolean_algebra}{booelsk algebra} efter \href{https://en.wikipedia.org/wiki/George_Boole}{George Boole}

\item Enkla logiska uttryck: (finns bara två stycken)
\begin{itemize}
\item[] \code{true}
\item[] \code{false}
\end{itemize}
\end{itemize}
\end{minipage}%
\begin{minipage}{.2\textwidth}
\centering\includegraphics[width=0.9\columnwidth]{../img/boole}
\end{minipage}%
\begin{itemize}


\item Sammansatta logiska uttryck med logiska operatorer:\\
\code{&&} och, \texttt{||} eller, \texttt{!} icke, \code{==} likhet, \code{!=} olikhet,
relationer: \code{> < >= <=}

\item Exempel:
\begin{itemize}
\item[] \code{true && true}
\item[] \code{false || true}
\item[] \code{!false}
\item[] \code{42 == 43}
\item[] \code{42 != 43}
\item[] \code{(42 >= 43) || (1 + 1 == 2)}
\end{itemize}

\end{itemize}
\end{Slide}

\begin{Slide}{De Morgans lagar}

\href{https://en.wikipedia.org/wiki/Augustus_De_Morgan}{\Emph{De Morgans lagar}} beskriver vad som händer om man \Alert{negerar} ett logiskt uttryck. Kan användas för att göra \Emph{förenklingar}.

%$p$ och $q$ är logiska uttryck, $\neg$ står för ''icke'', $\wedge$ för ''och'', $\vee$ för ''eller'':
%\begin{eqnarray*}
%\neg (p \wedge q) & \Longleftrightarrow & (\neg p) \vee (\neg q)\\
%\neg (p \vee q) & \Longleftrightarrow & (\neg p) \wedge (\neg q)\\
%\end{eqnarray*}

\begin{itemize}
\item I alla deluttryck sammanbundna med \code{&&} eller \code{||}, \\ ändra alla \code{&&} till \code{||} och omvänt.
\item Negera alla ingående deluttryck. En relation negeras genom att man byter \texttt{==} mot \texttt{!=}, \texttt{<} mot \texttt{>=}, etc.
\end{itemize}

Exempel på förenkling där de Morgans lagar används upprepat:

\begin{Code}[escapechar=X,backgroundcolor=,frame=none,basicstyle=\ttfamily\fontsize{10}{12}\selectfont]
! (a < b || (a == 1 && b == 1))             X$\iff$X
! (a < b) && ! (a == 1 && b == 1)           X$\iff$X
! (a < b) && (! (a == 1) || ! (b == 1))     X$\iff$X
a >= b && (a != 1 || b != 1)
\end{Code}
\end{Slide}

\begin{Slide}{Alternativ med if-uttryck}
\begin{itemize}
\item Ett if-uttryck börjar med nyckelordet \code{if}, följt av ett logiskt uttryck (villkor) inom parentes och två grenar.
\begin{Code}
def slumpgrönsak = if math.random() < 0.8 then "gurka" else "tomat"
\end{Code}
\item Uttrycket efter \code{then} blir resultatet om villkoret är \code{true}
\item Uttrycket efter \code{else} blir resultatet om villkoret är \code{false}
\begin{REPLnonum}
scala> slumpgrönsak
res13: String = gurka

scala> slumpgrönsak
res14: String = gurka

scala> slumpgrönsak
res15: String = tomat

\end{REPLnonum}
\end{itemize}
\end{Slide}

\Subsection{Satser}

\begin{Slide}{Uttryck eller sats?}
Skillnad mellan uttryck och sats:
\begin{itemize}
\item Ett uttryck ger ett \Emph{resultat}. Exempel: \code{1+1}
\item En sats har en \Alert{effekt}. \\Exempel: utskrift, spara på fil, tilldela variabel nytt värde.
\end{itemize}
Skriv ett \Emph{uttryck} när du är intresserad av \Emph{värdet} som beräknas.\\
Skriv en \Alert{sats} när du vill att något ska \Alert{göras}.\\~\\
{\SlideFontSmall Både satser och uttryck kan i sin tur innehålla satser och uttryck i godtyckligt komplexa \textbf{nästlade strukturer} (mer om det senare).}
\end{Slide}

%%%%%%%%%%%%%%%%%%%%%%%%
\begin{Slide}{Variabeldeklaration och tilldelningssats}\SlideFontTiny

\begin{itemize}
\item En \Emph{variabeldeklaration} medför att \Alert{plats i datorns minne} reserveras så att värden av den typ som variabeln kan referera till får plats där.

\item Vid deklaration ska variabeln \Emph{initialiseras} med ett startvärde.

\item En \code{val}-deklaration ger en variabel som efter initialisering inte kan ändras.


\begin{multicols}{2}
Dessa deklarationer...
\begin{lstlisting}
var x = 42
val y = x + 1
\end{lstlisting}
... ger detta innehåll någonstans i minnet:

%http://tex.stackexchange.com/questions/18521/tikz-matrix-as-a-replacement-for-tabular
\begin{tikzpicture}[]
\matrix [matrix of nodes, row sep=0, column 2/.style={nodes={rectangle,draw,minimum width=3em}}]
{
x   & 42 \\
y   & 43 \\
};
\end{tikzpicture}
%\end{column}

%\end{columns}

\end{multicols}


\item Med en \Emph{tilldelningssats} ges en tidigare \code{var}-deklarerad variabel ett nytt värde:
\begin{lstlisting}
x = 13
\end{lstlisting}

\item Det gamla värdet försvinner för alltid och det nya värdet lagras istället: \\
\begin{tikzpicture}[]
\matrix [matrix of nodes, row sep=0, column 2/.style={nodes={rectangle,draw,minimum width=3em}}]
{
x   & 13 \\
y   & 43 \\
};
\end{tikzpicture}

Observera att \code{y} här inte påverkas av att x ändrade värde.
\end{itemize}
\end{Slide}

\begin{Slide}{Tilldelningssatser är \emph{inte} matematisk likhet}\SlideFontSmall

\begin{itemize}

\item Likhetstecknet används alltså för att \Emph{tilldela} variabler nya värden och det är \Alert{inte} samma sak som matematisk likhet. Vad händer här?
\begin{lstlisting}
x = x + 1
\end{lstlisting}

\item Denna syntax är ett arv från de gamla språken C, Fortran mfl.

\item I \href{https://en.wikipedia.org/wiki/Assignment_(computer_science)}{andra språk} används  t.ex.  \\\vspace{1em}
\texttt{x := x + 1}  \hspace{2em} eller  \hspace{2em} \texttt{x <- x + 1} \\\vspace{0.5em}

\item Denna syntax visar kanske bättre att tilldelning är en \Emph{stegvis process}:

\begin{enumerate}\SlideFontTiny
\item Först beräknas \Emph{uttrycket till höger} om tilldelningstecknet.
\item Sedan \Emph{ersätts värdet} som variabelnamnet refererar till av det beräknade uttrycket. Det gamla värdet \Alert{försvinner för alltid}.
\end{enumerate}

\end{itemize}

\end{Slide}


\begin{Slide}{Förkortade tilldelningssatser}
\begin{itemize}
\item Det är vanligt att man vill tilldela en variabel ett nytt värde som beror av det gamla, så som i \\\code{x = x + 1}

\item Därför finns \Emph{förkortade tilldelningssatser} som gör så att man sparar några tecken och det blir tydligare (?) vad som sker (när man vant sig vid detta skrivsätt):
\begin{Code}
x += 1
\end{Code}

\item Uttrycket ovan expanderas av kompilatorn till \code{x = x + 1}
\end{itemize}


\end{Slide}


\begin{Slide}{Exempel på förkortade tilldelningssatser}
\begin{REPLnonum}
scala> var x = 42
val x: Int = 42

scala> x *= 2

scala> x
val res0: Int = 84


scala> x /= 3

scala> x
val res1: Int = 28
\end{REPLnonum}
\end{Slide}


\ifkompendium\else
\begin{Slide}{Övning: Tilldelningar i sekvens}\footnotesize

%\begin{columns}
%\begin{column}{0.32\textwidth}
\begin{minipage}{0.32\textwidth}

Rita hur minnet ser ut efter varje rad nedan:
\vskip1em
\begin{lstlisting}[ numbers=left,]
var u = 42
var x = 10
var y = 2 * x + 1
x = 20
var z = y + x + y - x
x += 1; y *= 2
\end{lstlisting}
%\end{column}
\end{minipage}\hspace{2em}%
%\begin{column}{0.6\textwidth}
\begin{minipage}{0.6\textwidth}


\scriptsize En variabel som ännu inte \Emph{initierats} har ett \Alert{odefinierat} värde, anges nedan med frågetecken.
\begin{table}[]
\centering\scriptsize
%http://tex.stackexchange.com/questions/83930/what-are-the-different-kinds-of-boxes-in-latex
\newcommand{\mybox}[1]{\raisebox{-0.5mm}{\framebox(21,14){#1}}\vspace{0.5mm}}
\begin{tabular}{@{}ccccccc}
 & rad 1 & rad 2 & rad 3 & rad 4  & rad 5 & rad 6\\
u& \mybox{42 } &  \mybox{}   &   \mybox{}   & \mybox{} & \mybox{} & \mybox{} \\
x& \mybox{? } &  \mybox{}   &   \mybox{}   & \mybox{} & \mybox{}  & \mybox{} \\
y& \mybox{? } &  \mybox{}   &   \mybox{}   & \mybox{} & \mybox{}  & \mybox{} \\
z& \mybox{? } &  \mybox{}   &   \mybox{}   & \mybox{} & \mybox{}  & \mybox{} \\
\end{tabular}
\end{table}

%\end{column}
%\end{columns}
\end{minipage}%
\end{Slide}
\fi



\begin{Slide}{Variabler som ändrar värden kan vara knepiga}
\begin{itemize}
\item Kod som innehåller variabler som \Alert{förändras} över tid är ofta svårare att läsa och begripa.

\item Många buggar beror på att variabler av misstag förändras på felaktiga och oanade sätt.

\item Föränderliga värden blir speciellt svåra i kod som körs jämlöpande (parallellt).

\item I kod som körs i skarpt läge med många användare (s.k. produktionskod) är därför \code{val} att föredra, medan \code{var} endast används om det \Alert{verkligen} behövs.
\item Alltså: räkna hellre ut nya värden än förändra befintliga.
\end{itemize}
\end{Slide}
%%%%%%%%%%


\Subsection{Kontrollstrukturer}

\begin{Slide}{Kontrollstrukturer: alternativ och repetition}\SlideFontSmall
Används för att kontrollera (förändra) sekvensen och skapa \Emph{alternativa} vägar genom koden. Vägen  bestäms vid körtid.
\begin{itemize}
\item if-sats:
\begin{Code}
if math.random() < 0.8 then println("gurka") else println("tomat")
\end{Code}
\end{itemize}

Olika sorters \Emph{loopar} för att repetera satser. Antalet repetitioner ges vid körtid.
\begin{itemize}
\item \code{while}-sats: bra när man \Alert{inte vet hur många gånger} det kan bli.
\begin{Code}
while math.random() < 0.8 do println("gurka")
\end{Code}

\item \code{for}-sats: bra när man \Alert{vill ange antalet repetitioner}:
\begin{Code}
for i <- 1 to 10 do println(s"gurka nr $i")
\end{Code}

\end{itemize}
\end{Slide}

\begin{Slide}{Scala-2-syntax för kontrollstrukturer fungerar i Scala 3}\SlideFontSmall
I Scala 2 användes en gammal syntax för kontrollstrukturer som liknar mer C/C++/Java. Den är tillåten i Scala 3, men nya mer lättlästa syntaxen är att föredra.
\begin{itemize}
\item Scala-2-syntax för alternativ: parenteser men inget \code{then}
\begin{Code}
if (math.random() < 0.8) println("gurka") else println("tomat")
\end{Code}
\end{itemize}

Scala-2-syntax för repetition:
\begin{itemize}
\item \code{while}-sats: parenteser men inget \code{do}
\begin{Code}
while (math.random() < 0.8) println("gurka")
\end{Code}

\item \code{for}-sats: parenteser men inget \code{do}
\begin{Code}
for (i <- 1 to 10) println(s"gurka nr $i")
\end{Code}

\item Kojo Desktop funkar ännu bara med Scala 2 och gamla syntaxen, men Kojo kan även köras med Scala 3 (se hur i kompendiet).

\end{itemize}

\end{Slide}

\begin{Slide}{Repetera många satser}
Om du vill göra flera saker i sekvens inne i en repetition så kan du skriva flera satser inom \Emph{klammer-parenteser}:
\begin{Code}
while math.random() < 0.8 do {
  println("gurka")
  println("tomat")
}
println("Repetitionen är klar!")
\end{Code}
\pause
Du kan efter vissa nyckelord (t.ex. \code{do}, \code{then}, \code{else}) välja bort klammer-parenteser \Eng{optional braces}. 
\begin{Code}
while math.random() < 0.8 do
  println("gurka")
  println("tomat")

println("Repetitionen är klar!")
\end{Code}
Då är det \Emph{indenteringen} som avgör vilka satser som ingår. \\Detta fungerar i Scala 3 (men inte i Scala 2).
\end{Slide}


\begin{Slide}{Procedurer}\SlideFontSmall
\begin{itemize}
\item En \Emph{procedur} är en funktion som \Alert{gör} något intressant, men som \Alert{inte} lämnar något intressant returvärde.
\item Exempel på procedur i standardbiblioteket: \code{println("hej")}
\item Du \Emph{deklarerar egna procedurer} genom att ange \texttt{\Alert{Unit}} som returvärdestyp. Då returneras värdet \texttt{\Alert{()}} som betyder ''inget''.
\end{itemize}
\begin{REPLnonum}
scala> def hej(x: String): Unit = println(s"Hej på dej $x!")
hej: (x: String)Unit

scala> hej("Herr Gurka")
Hej på dej Herr Gurka!

scala> val x = hej("Fru Tomat")
Hej på dej Fru Tomat!
x: Unit = ()
\end{REPLnonum}
\begin{itemize}
\item Det som \Alert{görs} kallas (sido)\Emph{effekt}. Ovan är utskriften själva effekten.
\item Även funktioner kan ha sidoeffekter. De kallas då \Alert{oäkta} funktioner.
\end{itemize}
\end{Slide}

\begin{Slide}{Problemlösning: nedbrytning i abstraktioner som sen kombineras}\SlideFontSmall
\begin{itemize}
\item En av de allra viktigaste principerna inom programmering är \Emph{funktionell nedbrytning} där  \Emph{underprogram} i form av funktioner och procedurer skapas för att bli byggstenar som kombineras till mer avancerade funktioner och procedurer.

\item Genom de namn som definieras skapas \Emph{återanvändbara abstraktioner} som kapslar in det funktionen gör till ett ''byggblock''.

\item Bra ''byggblock'' gör det lättare att lösa svåra programmeringsproblem.

\item Abstraktioner som beräknar eller gör \Emph{en enda, väldefinierad sak} är enklare att använda, jämfört med de som gör många, helt olika saker.

\item Abstraktioner med \Emph{välgenomtänkta namn} är enklare att använda, jämfört med kryptiska eller missvisande namn.
\end{itemize}

\end{Slide}

\Subsection{Veckans uppgifter}

\begin{Slide}{Övning \texttt{expressions} och labb \texttt{kojo}}
\label{w01:kojo-slide}
\begin{itemize}
  \item På övningen kör du Scala REPL för att träna på SARA.
  \item[] \Alert{Läs i Appendix} och på kursens hemsida under ''Verktyg'' om hur du installerar och får igång Scala REPL.
  \item På laborationen använder du barnvänliga \Emph{Kojo} för träna på SARA, med fokus på abstraktion.
  \item Det finns två olika sätt att använda Kojo:
  \begin{enumerate}
  \item Grafikbiblioteket \textbf{\texttt{kojolib}} i ett fristående Scala program med hjälp av en professionell kodeditor och kompilering och exekvering i terminalen. \Emph{Fungerar fint med nya Scala 3}.
  \item Skrivbordsappen \textbf{Kojo Desktop} med inbyggd barnvänlig editor (endast Scala 2).
  %\item Webbappen \textbf{\url{http://kojo.lu.se/}} som körs direkt i din webbläsare (endast Scala 2, begränsade funktioner).
  \end{enumerate}
\end{itemize}  
Alternativ 1 rekommenderas, men om du försenas av tekniskt strul, så kom igång med 2 el. 3 så länge tills du fått hjälp.
\end{Slide}

\begin{Slide}{Köa med Sigrid}
För att köa till handledare på plats i sal i pgk: \url{cs.lth.se/sigrid}
\begin{itemize}
\item Direkt när undervisningspasset \Emph{börjar}: starta en session med ditt förnamn, kurskod EDAA45 och rummets namn. Gör detta även om du inte behöver hjälp från start! Då kan \Alert{ambulanser} se antal studenter i varje rum.
\item Inget lösenord behövs.
\item Två olika köer i varje rum: \Emph{hjälpkö} och \Alert{redovisningskö}
\begin{itemize}
  \item Ställ dig i hjälpkö om du vill få vägledning och ställa frågor 
  \item Ställ dig i redovisningskö om du är klar att redovisa en labb
\end{itemize}
\item Du måste klicka på \Emph{Uppdatera} -- annars händer inget!
\item OBS! \Alert{Köar inte}+\Emph{Uppdatera} så fort handledare anländer!
\item Om du går på extra pass i mån av plats så kan du se vilket rum som har kortast kö här: \url{cs.lth.se/sigrid/monitor}
\end{itemize}
\end{Slide}

\begin{Slide}{Sigrid in action}
Så här ser det ut när student står i hjälpkö efter att först ha klickat på \Alert{Hjäälp!!!} \emph{och} sedan på \Emph{Uppdatera}-knappen:\\

\includegraphics[width=0.55\textwidth]{../img/sigrid.png}

\noindent GLÖM INTE \Alert{Köar inte} + \Emph{Uppdatera} när handledare \emph{anländer}!
\end{Slide}

\ifkompendium\else

\begin{SlideExtra}{Om veckans övning: \code{expressions}}\SlideFontSmall

\begin{itemize}
%!TEX encoding = UTF-8 Unicode

\item Förstå vad som händer när satser exekveras och uttryck evalueras.
\item Känna till betydelsen av begreppen sekvens, alternativ, repetition och abstraktion.
\item Känna till litteralerna för enkla värden, deras typer och omfång.
\item Kunna deklarera och använda variabler och tilldelning, samt kunna rita bilder av minnessituationen då variablers värden förändras.
\item Förstå skillnaden mellan olika numeriska typer, kunna omvandla mellan dessa och vara medveten om noggrannhetsproblem som kan uppstå.
\item Förstå booleska uttryck och värdena \code{true} och \code{false}, samt kunna förenkla booleska uttryck.
\item Förstå skillnaden mellan heltalsdivision och flyttalsdivision, samt användning av rest vid heltalsdivision.
\item Förstå precedensregler och användning av parenteser i uttryck.
\item Kunna använda \code{if}-satser och \code{if}-uttryck.
\item Kunna använda \code{for}-satser och \code{while}-satser.
\item Kunna använda \code{math.random()} för att generera slumptal i olika intervaller.
\item Kunna beskriva skillnader och likheter mellan en procedur och en funktion.

\end{itemize}

\end{SlideExtra}

\begin{SlideExtra}{Om veckans labb: \code{kojo}}
\begin{itemize}
\input{../compendium/modules/w01-intro-lab-goals.tex}
\end{itemize}
\end{SlideExtra}

\fi

%%%%%%%%%%%%%%%%%%%%%%%%%%%%%%%%%%%%%% nedan redan meddelat i introveckan
%\ifkompendium\else
%\Subsection{Meddelande från \href{http://lth.se/code}{Code@LTH}}
%\fi
