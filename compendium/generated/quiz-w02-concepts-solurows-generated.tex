  kompilera & 1 & ~~\Large$\leadsto$~~ &  C & maskinkod skapas ur en eller flera källkodsfiler \\ 
  skript & 2 & ~~\Large$\leadsto$~~ &  G & ensam kodfil, huvudprogram behövs ej \\ 
  objekt & 3 & ~~\Large$\leadsto$~~ &  M & samlar variabler och funktioner \\ 
  @main & 4 & ~~\Large$\leadsto$~~ &  E & där exekveringen av kompilerat program startar \\ 
  programargument & 5 & ~~\Large$\leadsto$~~ &  L & kan överföras via parametern args till main \\ 
  datastruktur & 6 & ~~\Large$\leadsto$~~ &  J & många olika element i en helhet; elementvis åtkomst \\ 
  samling & 7 & ~~\Large$\leadsto$~~ &  K & datastruktur med element av samma typ \\ 
  sekvenssamling & 8 & ~~\Large$\leadsto$~~ &  F & datastruktur med element i en viss ordning \\ 
  Array & 9 & ~~\Large$\leadsto$~~ &  D & en förändringsbar, indexerbar sekvenssamling \\ 
  Vector & 10 & ~~\Large$\leadsto$~~ &  B & en oföränderlig, indexerbar sekvenssamling \\ 
  Range & 11 & ~~\Large$\leadsto$~~ &  N & en samling som representerar ett intervall av heltal \\ 
  yield & 12 & ~~\Large$\leadsto$~~ &  A & används i for-uttryck för att skapa ny samling \\ 
  algoritm & 13 & ~~\Large$\leadsto$~~ &  H & stegvis beskrivning av en lösning på ett problem \\ 
  implementation & 14 & ~~\Large$\leadsto$~~ &  I & en specifik realisering av en algoritm \\ 