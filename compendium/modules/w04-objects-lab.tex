%!TEX encoding = UTF-8 Unicode
%!TEX root = ../compendium1.tex

\Lab{\LabWeekFOUR}
\begin{Goals}
%!TEX encoding = UTF-8 Unicode
%!TEX root = ../labs.tex

\item Kunna förklara hur singelobjekt kan användas som moduler.
\item Kunna förklara hur åtkomst av medlemmar i singelobjekt sker.
\item Kunna skapa kod som reagerar på och förändrar objekts tillstånd.
\item Kunna förklara nyttan med att samla namngivna konstanter i egen modul.
\item Kunna förklara hur import påverkar synlighet av namn.
\item Kunna ge exempel på en situation där man har nytta av namnbyte vid import.
\item Kunna redogöra för skillnaden mellan paket och singelobjekt.
\item Kunna skapa och använda tupler.

\end{Goals}

\begin{Preparations}
\item Gör övning \texttt{\ExeWeekFOUR} och repetera övning \texttt{\ExeWeekTHREE}.
\item Repetera appendix~\ref{appendix:terminal}, ~\ref{appendix:compile}, och ~\ref{appendix:debug}. 
\item Hämta given kod via \href{https://github.com/lunduniversity/introprog/tree/master/workspace/}{kursen github-plats} eller via hemsidan under \href{https://cs.lth.se/pgk/download/}{Download}.
\end{Preparations}



\subsection{Bakgrund}


\begin{minipage}{0.48\textwidth}
\begin{figure}[H]
  \centering
  \includegraphics[width=\textwidth]{../img/blockmole-sky-grass.png}
%  \caption{En lundensisk blockmullvad fångad på bild under aktivt grävanade.}
  \label{lab:blockmole:fig:mole}
\end{figure}
\end{minipage}%
%
\hfill\begin{minipage}{0.45\textwidth}
\noindent\textbf{Blockmullvad} (\textit{Talpa laterculus}) är ett fantasidjur i familjen mullvadsdjur.
Den är känd för sitt karaktäristiska kvadratiska utseende.
Den lever mest ensam i sina underjordiska gångar som, till skillnad från den verkliga mullvadens (\emph{Talpa europaea}) gångar, har helt raka väggar.
\end{minipage}



\subsection{Obligatoriska uppgifter}


\Task \emph{Skapa katalog och kodfil.}
Du ska, steg för steg, skapa ett program som låter användaren interagera med en levande blockmullvad. Använd en editor, t.ex. VS \texttt{code}, kompilera ditt program i terminalen med \texttt{scala-cli compile . --watch} och kör i annat terminalfönster med \texttt{scala-cli run .}

\Subtask
Skapa en ny fil med namnet \texttt{blockmole.scala} i en ny katalog i din hemkatalog, till exempel \texttt{\textasciitilde/pgk/w04/lab/blockmole.scala}, där \texttt{\textasciitilde} är din hemkatalog.
\begin{REPLnonum}
> mkdir -p ~/pgk/w04/lab
> code ~/pgk/w04/lab/blockmole.scala
\end{REPLnonum}


\Subtask
Navigera till din nya katalog och kontrollera att din nya fil finns där.
\begin{REPLnonum}
> cd ~/pgk/w04/lab/
> ls
blockmole.scala
\end{REPLnonum}

\Subtask
Gör en paketdeklaration i början av filen \code{blockmole.scala} så att koden du ska skriva nedan ingår i paketet \code{blockmole}.

\Subtask
Deklarera sedan ett singelobjekt med namnet \code{Main} med en \code{@main def run}-procedur som skriver ut texten: \texttt{"Keep on digging!"}

\Subtask
Kompilera ditt program med \texttt{scala-cli compile .} och kontrollera med \\\texttt{ls .scala-build/*/classes/*} att några filer som slutar på \texttt{class} har skapats i en underkatalog. \Pen Vilket namn har underkatalogen med ditt programs maskinkodsfiler? Varför fick underkatalogen detta namn?

\Subtask
Kör kommandot \texttt{scala-cli run . --main-class blockmole.run} för att exekvera ditt program och kontrollera utskriften i terminalfönstret.

\vspace{1em}\noindent Nu har du skrivit ett program som uppmanar en blockmullvad att fortsätta gräva. Det programmet är inte så användbart, eftersom mullvadar inte kan läsa. Nästa steg är därför att skriva ett grafiskt program.%, snarare än ett textbaserat.



\Task \emph{Skapa en grundstruktur för programmet.}
I mindre program fungerar det bra att samla alla funktioner i ett singelobjekt, men i stora program blir det lättare att hitta i koden och förstå vad den gör om man har flera moduler med olika ansvar. Ditt program ska ha följande övergripande struktur:

\begin{Code}
package blockmole

object Color:
  // Skapar olika färger som behövs i övriga moduler
  ???

object BlockWindow:
  // Har ett introprog.PixelWindow och ritar blockgrafik
  ???

object Mole: // Representerar en blockmullvad som kan gräva
  def dig(): Unit = println("Här ska det grävas!")

object Main:
  def drawWorld(): Unit = println("Ska rita ut underjorden!")

  @main def run = 
    drawWorld()
    Mole.dig()
\end{Code}

\noindent Skapa programskelettet ovan i filen \code{blockmole.scala} och se till att koden kompilerar utan fel och går att köra med utskrifter som förväntat. Funktionen \code{???} i skelettet används som platshållare för att koden ska kunna kompileras trots att singelobjektens kroppar just nu är tomma (mer om detta i kapitel 5).  Byt ut \code{???} mot den faktiska koden för Color och BlockWindow i kommande deluppgifter.

Vi lägger i denna laboration alla moduler i samma fil, men i andra situationer när  modulerna blir stora och/eller ska återanvändas av flera olika program är det bra att ha dem i olika filer så att de kan kompileras och testas separat.


\Task \emph{Lägg till färger i färgmodulen.} I singelobjektet \code{Color} ska vi skapa färger med hjälp av Java-klassen \code{java.awt.Color}. Eftersom vårt singelobjektnamn ''krockar'' med namnet på Java-färgklassen så byter vi namn på Java-klassen till \code{JColor} i importdeklarationen.

\Subtask
Lägg in en importdeklaration med namnbytet direkt efter paketdeklarationen. Vi lägger importen så att den syns i hela paketet eftersom flera objekt behöver tillgång till \code{JColor}. Säkerställ att koden fortfarande kompilerar utan fel.

\Subtask
Skapa sedan nedan färger i objektet \code{Color}:
\begin{Code}
object Color:
  val black  = new JColor(  0,   0,   0)
  val mole   = new JColor( 51,  51,   0)
  val soil   = new JColor(153, 102,  51)
  val tunnel = new JColor(204, 153, 102)
  val grass  = new JColor( 25, 130,  35)
\end{Code}


\Task \emph{Skapa ett ritfönster i modulen för blockgrafik.} Lägg till nedan tre variabler i singelobjektet \code{BlockWindow}:

\begin{Code}
  val windowSize = (30, 50)  // (width, height) in number of blocks
  val blockSize  = 10        // number of pixels per block

  val window = new PixelWindow(???, ???, ???)
\end{Code}

\begin{itemize}%[noitemsep]
  \item Importera \code{introprog.PixelWindow} lokalt i \code{BlockWindow}. (En lokal import-deklaration är bra här eftersom det bara är detta objekt som behöver tillgång till \code{PixelWindow}.)
  \item Gör så att storleken på \code{window} motsvarar blockstorleken gånger bredd resp. höjd i \code{windowSize}.
  \item Ge fönstret en lämplig titel, t.ex. \code{"Digging Blockmole"}.
  \item När du kompilerar behöver du se till att \code{introprog} finns tillgänglig på \code{classpath} (se övning \texttt{\ExeWeekFOUR}).
  \item Om du glömt ordningen på parametrarna till klassen \code{PixelWindow} så kolla i dokumentationen för \code{PixelWindow} \footnote{\url{http://cs.lth.se/pgk/api/}}. Använd namngivna argument vid skapandet av fönstret. Tycker du att koden blir mer läsbar med namngivna argument? \footnote{Det går tyvärr inte att använda namngivna argument när man instansierar Java-klasser i Scala, men PixelWindow är implementerad i Scala så här fungerar det fint.}
\end{itemize}

För att testa fönstret, lägg till en enkel testritning genom att i proceduren \code{drawWorld} använda \code{BlockWindow.window}, till exempel:
\begin{Code}
  def drawWorld(): Unit = 
    BlockWindow.window.line(100, 10, 200, 20)
\end{Code}
Kompilera och kör och säkerställ att allt fungerar som förväntat.


\Task \emph{Skapa procedur för blockgrafik.} Nu har du gjort ett grafiskt program, men ännu syns ingen mullvad.
Det är dags att skapa koordinatsystemet i blockmullvadens blockvärld.

\Subtask\Pen
Säkerställ att du kan förklara hur koordinaterna i ett \code{PixelWindow} tolkas, genom att med papper och penna rita en enkel skiss av ungefär var positionerna $(0,0)$, $(300, 0)$, $(0, 300)$ och $(300, 300)$ ligger i ett fönster som är 300 bildpunkter brett och 500 bildpunkter högt. Använd figur \ref{lab:blockmole:coords} för att förklara relationen mellan underliggande fönsterkoordinater och blockkoordinater. Notera att y-axeln pekar nedåt.

\begin{figure}
\begin{center}
\includegraphics[width=0.42\textwidth]{../img/block-xy}
\end{center}
\caption{Varje block består av många pixlar. Det markerade blocket har koordinat (1,1) i blockkoordinater medan blockets översta vänstra pixel har koordinat (7,7) i \code{PixelWindow}-koordinater, om det t.ex. går sju-gånger-sju pixlar per block. Vad är block-koordinaten för blocket till höger om det markerade blocket i bilden? Vad är dess \code{PixelWindow}-koordinater för översta vänstra och nedersta högra pixlarna?}\label{lab:blockmole:coords}
\end{figure}

\Subtask
Koordinatsystem i \code{BlockWindow} ska ha kvadratiska, \emph{stora} bildpunkter som består av många fönsterpixlar. 
Vi kallar dessa stora bildpunkter för \emph{block} för att lättare skilja dem från de enpixelstora bildpunkterna i \code{PixelWindow}. 

I block-koordinatsystemet för \code{BlockWindow} gäller följande:

\begin{framed}
\noindent \emph{Blockstorleken} anger sidan i kvadraten för ett block räknat i antalet pixlar. Om blockstorleken är $b$, så ligger koordinaten $(x, y)$ i \code{BlockWindow} på koordinaten $(bx, by)$ i \code{PixelWindow}.

\end{framed}

\noindent Implementera funktionen \code{block} i modulen \code{BlockWindow} enligt nedan, så att en kvadrat ritas ut när proceduren anropas. Parametern \code{pos} anger block-koordinaten och parametern \code{color} anger färgen. Typ-alias-deklarationen av \code{Pos} ger ett beskrivande typnamn för en 2-tupel av heltal, som vi kan använda i parameterlistor för att betecknande positioner i ett \code{BlockWindow}. Se dokumentationen av \code{fill}-metoden i \code{PixelWindow}. Observera att du behöver räkna om block-koordinaterna i \code{pos} till fönsterkoordinater i \code{windows.fill}. Fyll i det som saknas nedan.
\begin{Code}
  type Pos = (Int, Int)

  def block(pos: Pos)(color: JColor = JColor.gray): Unit = 
    val x = ??? //räkna ut blockets x-koordinat i pixelfönstret
    val y = ??? //räkna ut blockets y-koordinat i pixelfönstret
    window.fill(???)
\end{Code}
Säkerställ att koden kompilerar utan fel.

% %%% Borttagen fråga, då vi använder fill istf att rita en massa linjer som man redan gjort på övningen
% \Subtask\Pen
% Metoden \code{block} ska rita ett antal linjer.
% Hur många linjer ritas ut?
% I vilken ordning ritas linjerna?
% Skriv ner dina svar inför redovisningen.

\Subtask
För att testa din procedur, anropa funktionen \code{BlockWindow.block} några gånger i  \\\code{Main.drawWorld}, dels med utelämnat defaultargument, dels med olika färger ur färgmodulen. Kompilera och kör ditt program och kontrollera att allt fungerar som det ska.



\Task \emph{Skapa rektangelprocedur och underjorden.} Du ska nu skriva en procedur med namnet \code{rectangle} som ritar en rektangel med hjälp av proceduren \code{block}. Sen ska du använda \code{rectangle} i \code{Main.drawWorld} för att rita upp mullvadens underjordiska värld.

\Subtask
Lägg till proceduren \code{rectangle} i grafikmodulen. Procedurhuvudet ska ha följande parametrar uppdelade i tre olika paramterlistor, samt returtyp \code{Unit}:
\begin{Code}
(leftTop: Pos)(size: (Int, Int))(color: JColor = JColor.gray)
\end{Code}

Parametern \code{leftTop} anger blockkoordinaten för rektangelns övre vänstra hörn, och \code{size} anger \code{(bredd, höjd)} uttryckt i antal block.

Använd denna nästlade repetition för att rita ut rektangeln:

\begin{Code}
for y <- ??? do
	for x <- ??? do
		block(x, y)(color)
\end{Code}

\Subtask\Pen
I vilken ordning ritas blocken i rektangeln ut (lodrätt eller vågrätt)? Om du är osäker kan du lägga in en utskrift av \code{(x, y)} i den innersta loopen för att se ordningen.

\Subtask\Pen En annan lösning är att i stället anropa \code{fill}-metoden i \code{PixelWindow} direkt för att rita en motsvarande stor rektangel \emph{utan} nästlad loop. Vilka argument ska \code{fill}-anropet då ha?

\Subtask Lägg följande kod i \code{Main.drawWorld} så att programmet ritar ut underjorden (det vill säga en massa jord där blockmullvaden kan gräva sina tunnlar) och även lite gräs.

\begin{Code}
def drawWorld(): Unit = 
  BlockWindow.rectangle(0, 0)(size = (30, 4))(Color.grass)
  BlockWindow.rectangle(0, 4)(size = (30, 46))(Color.soil)
\end{Code}

\Subtask Anropa \code{Main.drawWorld} i \code{Main.main} och testa att det fungerar. Om någon del av fönstret förblir svart istället för att få gräsfärg eller jordfärg, kontrollera att \code{block} och \code{rectangle} är korrekt implementerade.

\Task
I \code{PixelWindow} finns funktioner för att känna av tangenttryckningar och mus\-klick.
Du ska använda de funktionerna för att styra en blockmullvad. Studera dokumentationen för \code{awaitEvent} och \code{Event} i \code{PixelWindow}, samt koden i exempelprogrammet \code{TestPixelWindow} i paketet \code{introprog.examples}.

\Subtask
Lägg till denna funktion i \code{BlockWindow}:
\begin{Code}
  val maxWaitMillis = 10

  def waitForKey(): String = 
    window.awaitEvent(maxWaitMillis)
    while window.lastEventType != PixelWindow.Event.KeyPressed do
      window.awaitEvent(maxWaitMillis) // skip other events
    println(s"KeyPressed: ${window.lastKey}")
    window.lastKey
\end{Code}
\noindent Det finns olika sorters händelser som ett \code{PixelWindow} kan reagera på, till exempel tangenttryckningar och musklick.
Funktionen som du precis lagt in väntar på en händelse i ditt \code{PixelWindow} med hjälp av (\code{window.awaitEvent}) ända tills det kommer en tangenttryckning (\code{KEY_EVENT}).
När det kommit en tangenttryckning anropas \code{window.lastKey} för att ta reda på vilken bokstav eller vilket tecken det blev, och det resultatet blir också resultatet av \code{waitForKey}, eftersom det ligger sist i blocket.

\Subtask
Utöka proceduren \code{Mole.dig} enligt nedan:
\begin{Code}
  def dig(): Unit = 
    var x = BlockWindow.windowSize._1 / 2
    var y = BlockWindow.windowSize._2 / 2
    var quit = false
    while !quit do
      BlockWindow.block(x, y)(Color.mole)
      val key = BlockWindow.waitForKey()
      if      key == "w" then ???
      else if key == "a" then ???
      else if key == "s" then ???
      else if key == "d" then ???
      else if key == "q" then quit = true
    end while
\end{Code}

\Subtask Fyll i alla \code{???} så att \code{'w'} styr mullvaden ett steg uppåt, \code{'a'} ett steg åt vänster, \code{'s'} ett steg nedåt och \code{'d'} ett steg åt höger.

\Subtask Kontrollera så att \code{main} bara innehåller två anrop: ett till \code{drawWorld} och ett till \code{dig}. Kompilera och kör ditt program för att se om programmet reagerar på tangenterna w, a, s och d.

\Subtask Om programmet fungerar kommer det bli många mullvadar som tillsammans bildar en lång mask, och det är ju lite underligt. Lägg till ett anrop i \code{Mole.dig} som ritar ut en bit tunnel på position $(x, y)$ efter anropet till \code{BlockWindow.waitForKey} men innan \code{if}-satserna. Kompilera och kör ditt program för att gräva tunnlar med din blockmullvad.

\subsection{Kontrollfrågor}\Checkpoint

\noindent Repetera teorin för denna vecka och var beredd på att kunna svara på dessa frågor när det blir din tur att redovisa vad du gjort under laborationen:

\begin{enumerate}[noitemsep]
\item Hur ändras mullvadens koordinater när den rör sig uppåt på skärmen? 
\item Hur representeras färger med RGB?
\item Vad är en tupel och hur används tupler i denna labb?
\item Vad innebär punktnotation?
\item Ge exempel på användning av \code{import} och förklara vad som händer.
\item Vad är fördelen med skuggning och lokala namn?
\item Vi använde flera singelobjekt som olika s.k. \code{moduler} i denna laboration. Vad är fördelen med att att dela upp koden i moduler?
\item Gå igenom målen med laborationen och kontrollera så du har uppfyllt dem.
\end{enumerate}




%\clearpage

\subsection{Frivilliga extrauppgifter}

\Task
Mullvaden kan för tillfället gräva sig utanför fönstret.
Lägg till några \code{if}-satser i början av \code{while}-satsen som upptäcker om \code{x} eller \code{y} ligger utanför fönstrets kant och flyttar i så fall tillbaka mullvaden precis innanför kanten.

\Task
Mullvadar är inte så intresserade av livet ovanför jord, men det kan vara trevligt att se hur långt ner mullvaden grävt sig.
Lägg till en himmelsfärg i objektet \code{Color} och rita ut himmel ovanför gräset i \code{Mole.drawWorld}.
Justera också det du gjorde i föregående uppgift, så att mullvaden håller sig på marken. \emph{Tips:} Du har nytta av en interaktiv färgväljare som du kan få genom att anropa \code{introprog.Dialog.selectColor()} i Scala REPL.

\Task
Ändra så att mullvaden inte lämnar någon tunnel efter sig när den springer på gräset.

\Task
Låt mullvaden fortsätta gräva även om man inte trycker ned någon tangent. Tangenttryckning ska ändra riktningen.

\Subtask
Skapa en ny metod \code{BlockWindow.waitForKeyNonBlocking} som möjliggör tangentbordsavläsning som ej blockerar exekveringen enligt nedan:

\begin{Code}
  def waitForKeyNonBlocking(): String  = 
    import PixelWindow.Event.{KeyPressed, Undefined}

    window.awaitEvent(maxWaitMillis)
    while 
      window.lastEventType != KeyPressed  &&
      window.lastEventType != Undefined) 
    do window.awaitEvent(maxWaitMillis)
    if window.lastEventType == KeyPressed then window.lastKey else ""
\end{Code}

\Subtask
Lägg till en ny metod \code{BlockWindow.delay} som ska göra det möjligt att hindra blockmullvaden från att springa alltför fort:
\begin{Code}
  def delay(millis: Int): Unit = Thread.sleep(millis)
\end{Code}


\Subtask
Skapa en ny metod \code{Mole.keepOnDigging} som från början är en kopia av metoden \code{dig}. Gör följande tillägg/ändringar:
\begin{enumerate}[nolistsep,noitemsep]

\item Lägg till två variabler \code{var dx} och \code{var dy} i början, som ska hålla reda på riktningen som blockmullvaden gräver. Initialisera dem till \code{0} respektive {1}.

\item Lägg in en fördröjning på 200 millisekunder i den oändliga loopen. Deklarera en konstant \code{delayMillis} på lämpligt ställe i \code{Mole} och använd denna konstant som argument till \code{delay}.

\item Anropa \code{waitForKeyNonBlocking} i stället för \code{waitForKey} och kolla efter knapptryckning enligt nedan kodskelett. Fyll i de saknade delarna så att blockmullvaden rör sig ett steg i rätt riktning i varje looprunda.
\begin{Code}
      if      key == "w" then { dy = -1; dx = 0 }
      else if key == "a" then { ??? }
      else if key == "s" then { ??? }
      else if key == "d" then { ??? }
      else if key == "q" then { quit = true }
      y += ???
      x += ???
\end{Code}

%\item Anpassa fördröjningen efter din förmåga att hinna styra blockmullvaden.

% \item Lägg till så att man kan pausa grävandet om man trycker på blankstegstangenten.
%
% \item Gör så att om man ger en parameter \code{--non-blocking} till programmet när man startar det kan välja mellan om \code{main} anropar \code{dig} eller \code{keepOnDigging}.

\end{enumerate}

\Task\label{lab:blockmole:task:blockworm} \emph{Fånga blockmasken.}


{\raggedright
\begin{minipage}{0.27\textwidth}
\begin{figure}[H]%
  \includegraphics[width=\textwidth]{../img/blockworm.png}
  \label{lab:blockmole:fig:worm}
\end{figure}
\end{minipage}%
}%
\hfill\begin{minipage}{0.69\textwidth}
\noindent\textbf{Blockmask} (\textit{Lumbricus quadratus}) är ett fantasidjur i familjen daggmaskar. Den är känd för att kunna teleportera sig från en plats till en annan på ett ögonblick och är därför svårfångad. Den har i likhet med den verkliga daggmasken (\emph{Lumbricus terrestris}) RGB-färgen $(225, 100, 235)$, men är kvadratisk och exakt ett block stor. Blockmasken är ett eftertraktat villebråd bland blockmullvadar.

\end{minipage}



\Subtask Lägg till modulen \code{Worm} nedan i din kod och använd procedurerna i \code{keepOnDigging} så att blockmullvaden får en blockmask att jaga.


%\begin{figure}
\begin{Code}
object Worm:
  import BlockWindow.Pos

  def nextRandomPos(): Pos = 
    import scala.util.Random.nextInt
    val x = nextInt(BlockWindow.windowSize._1)
    val y = nextInt(BlockWindow.windowSize._2 - 7) + 7
    (x, y)

  var pos = nextRandomPos()

  def isHere(p: Pos): Boolean = pos == p

  def draw(): Unit  = BlockWindow.block(pos)(Color.worm)

  def erase(): Unit = BlockWindow.block(pos)(Color.soil)

  val teleportProbability = 0.02

  def randomTeleport(notHere: Pos): Unit =
    if math.random() < Worm.teleportProbability then
      erase()
      while
        pos = nextRandomPos()
        pos == notHere
      do ()
      draw()
    
end Worm
\end{Code}
%\end{figure}

\Subtask Koden i \code{Worm} förutsätter att himmel finns i fönstrets översta $7$ block. Hur många block som är himmel kan egentligen med fördel vara en konstant med ett bra namn på en bra plats. Denna konstant bör användas även i \code{drawWorld}. Fixa det!

\Subtask Gör så att texten \code{"WORM CAUGHT!"} skrivs ut i terminalen om blockmullvaden är på samma plats som blockmasken.

\Subtask Använd parametern \code{notHere} till att förhindra att blockmasken teleporterar sig till samma plats som blockmullvaden.

\Subtask Gör så att blockmullvaden får $1000$ poäng varje gång den fångar blockmasken.

\Subtask Gör så att spelet varar en bestämd, lagom lång tid, innan \code{Game Over}. Använd \code{System.currentTimeMillis} som ger aktuella antalet millisekunder sedan den förste januari 1970. När spelet är slut ska den totala poängen som blockmullvaden samlat skrivas ut i terminalen.

\Subtask Gör så att spelets hastighet ökar (d.v.s. att fördröjningen i spel-loopen minskar) efter en viss tid. I samband med det ska sannolikheten för att blockmasken teleporterar sig öka.
