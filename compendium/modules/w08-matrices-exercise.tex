
%!TEX encoding = UTF-8 Unicode
%!TEX root = ../exercises.tex

\ifPreSolution

\Exercise{\ExeWeekEIGHT}\label{exe:W08}

\begin{Goals}
\item Kunna skapa och använda matriser med nästlade strukturer av \code{Vector}.
\item Kunna iterera över elementen i en matris med nästlade \code{for}-satser och \code{for}-\code{yield}-uttryck, samt nästlad applicering av \code{map} respektive \code{foreach}.
\item Kunna skapa och använda funktioner som tar matriser som parametrar.
\item Kunna skapa en enkel generisk klass och enkla generiska funktioner med hjälp av en typparameter.
\item Kunna beskriva skillnader och likheter mellan Scala och Java vad gäller indexering och iterering i matriser implementerade med nästlade arrayer.
%\item Kunna skapa och använda matriser med hjälp inbyggda arrayer i Java.
%\item Kunna använda nästlade \code{for}-satser i Java för att iterera över elementen i en matris.
\end{Goals}

\begin{Preparations}
\item \StudyTheory{08}
\end{Preparations}

\BasicTasks

\else

\ExerciseSolution{\ExeWeekEIGHT}

\BasicTasks

\fi



\WHAT{Para ihop begrepp med beskrivning.}

\QUESTBEGIN

\Task \what

\vspace{1em}\noindent Koppla varje begrepp med den (förenklade) beskrivning som passar bäst:

\begin{ConceptConnections}
\input{generated/quiz-w08-concepts-taskrows-generated.tex}
\end{ConceptConnections}

\SOLUTION

\TaskSolved \what

\begin{ConceptConnections}
\input{generated/quiz-w08-concepts-solurows-generated.tex}
\end{ConceptConnections}

\QUESTEND




\WHAT{Skapa matriser med hjälp av nästlade samlingar.}

\QUESTBEGIN

\Task  \what~  Man kan i ett datorprogram, med hjälp av samlingar som innehåller samlingar, skapa nästlade strukturer som kan indexeras i två dimensioner och på så sätt representera en  \textbf{matris}.\footnote{\href{https://sv.wikipedia.org/wiki/Matris}{sv.wikipedia.org/wiki/Matris}}

\Subtask Rita minnessituationen efter tilldelningen på rad 1 nedan. Vad har \code{m} för typ och värde? Vad har \code{m} för dimensioner? Hur sker indexeringen i ett datorprogram jämfört med i matematiken?

\begin{REPL}
scala> val m = Vector((1 to 5).toVector, (3 to 7).toVector)
scala> m.apply(0).apply(1)
scala> m(1)
scala> m(1)(4)
\end{REPL}

\Subtask Vad ger uttrycken på raderna 2, 3 och 4 ovan för värden och typ?

\Subtask Man kan i ett datorprogram mycket väl skapa tvådimensionella, nästlade strukturer där raderna \emph{inte} innehåller samma antal element. Det blir då ingen äkta matris i strikt matematisk mening, men man kallar ofta ändå en sådan struktur för en ''matris''. Vilken typ har variablerna \code{m2}, \code{m3}, \code{m4} och \code{m5} nedan?

\begin{REPL}
scala> val m2 = Vector(Vector(1,2,3),Vector(4,5),Vector(42))
scala> val m3 = Vector(Vector(1,2), Vector(1.0, 2.0, 3.0))
scala> val m4 = m3(1) +: Vector("a") +: m3
scala> val m5 = Vector.fill(42){ m2(1).map(e => (e * math.random()).toInt) }
\end{REPL}

\Subtask Vilken av variablerna \code{m2}, \code{m3}, \code{m4} och \code{m5} ovan representerar en äkta matris i matematisk mening? Vilken är dess dimensioner?

\SOLUTION

\TaskSolved \what

\SubtaskSolved   \includegraphics{../img/w09-solutions/1a} \\
Typ: \code{Vector[Vector[Int]]}\\
Värde: \code{Vector(Vector(1, 2, 3, 4, 5), Vector(3, 4, 5, 6, 7))} \\
Dimensioner: $2 \times 5$\\
Inom matematiken sker indexering enligt konvention med 1 som lägsta index. I scala är lägsta index 0, man använder s.k. 0-indexering. \footnote{Detta är inte fallet i alla programmeringsspråk, vilket du kan läsa mer om på \url{https://en.wikipedia.org/wiki/Array\_data\_type\#Index\_origin}}

\SubtaskSolved
\begin{REPL}
scala> val m = Vector((1 to 5).toVector, (3 to 7).toVector)
m: Vector[Vector[Int]] = Vector(Vector(1, 2, 3, 4, 5), Vector(3, 4, 5, 6, 7))

scala> m.apply(0).apply(1)
res4: Int = 2

scala> m(1)
res5: Vector[Int] = Vector(3, 4, 5, 6, 7)

scala> m(1)(4)
res6: Int = 7
\end{REPL}

\SubtaskSolved  \\
m2: \code{Vector[Vector[Int]]}\\
m3: \code{Vector[Vector[Int | Double]]}\\
m4: \code{Vector[Vector[Int | Double | String]]}\\
m5: \code{Vector[Vector[Int]]}

\SubtaskSolved  m5, $42 \times 2$

\QUESTEND





\WHAT{Skapa och iterera över matriser.}

\QUESTBEGIN

\Task  \label{matrices:task:yatzy} \what~  Du ska skapa matriser där varje rad representerar 5 kast med en tärning i spelet Yatzy.\footnote{\href{https://sv.wikipedia.org/wiki/Yatzy}{sv.wikipedia.org/wiki/Yatzy}}


\Subtask Definiera i REPL en funktion \code{def throwDie: Int = ???} som returnerar ett slumptal mellan 1 och 6.

\Subtask Skapa nedan heltalsmatris i REPL. Vilken dimension får matrisen?
\begin{REPL}
scala> val ds1 = for (i <- 1 to 1000) yield 
            for (j <- 1 to 5) yield throwDie
          
\end{REPL}

\Subtask Man kan också använda nedan varianter för att skapa en heltalsmatris. Vilken av varianterna \code{ds1} ... \code{ds6} tycker du är lättast att läsa och förstå? Prova respektive variant i REPL och ange vilken typ på \code{ds1} ... \code{ds6} som härleds av kompilatorn.
\begin{REPL}
val ds2 = (1 to 1000).map(i => (1 to 5).map(j => throwDie))
val ds3 = (1 to 1000).map(i => Vector.fill(5)(throwDie))
val ds4 = for (i <- 1 to 1000) yield Vector.fill(5)(throwDie)
val ds5 = Vector.fill(1000)(Vector.fill(5)(throwDie))
val ds6 = Vector.fill(1000, 5)(throwDie)
\end{REPL}


\Subtask Definiera en funktion \\ \code{def roll(n: Int): Vector[Int] = ???}\\ som ger en heltalsvektor med $n$ stycken slumpvisa tärningskast. Kasten ska vara sorterade i växande ordning; använd för detta ändamål samlingsmetoden \code{sorted}.


\Subtask \label{matrices:subtask:isyatzyforall} Definera i REPL en funktion \code{isYatzy(xs: Vector[Int]): Boolean = ???} som testar om alla elementen i en heltalsvektor är samma. Använd samlingsmetoden \code{forall}.


\Subtask Skapa en funktion  \\ \code{def diceMatrix(m: Int, n: Int): Vector[Vector[Int]] = ???} \\ som med hjälp av funktionen \code{roll} skapar en matris med \code{m} st vektorer med vardera \code{n} slumpvisa tärningskast.


\Subtask \label{matrices:subtask:diceMatrixToString} Skapa en funktion som returnerar en utskriftsvänlig sträng \\ \code{def diceMatrixToString(xss: Vector[Vector[Int]]): String = ???} \\med hjälp av \code{map} och \code{mkString}, som fungerar enligt nedan.
\begin{REPL}
scala> val dm2s = diceMatrixToString(diceMatrix(4, 5))
val dm2s: String = 1 4 4 6 6
1 1 2 6 6
2 4 4 5 6
1 1 5 6 6

scala> println(dm2s)
1 4 4 6 6
1 1 2 6 6
2 4 4 5 6
1 1 5 6 6
\end{REPL}



\Subtask Implementera funktionen \\ \code{def filterYatzy(xss: Vector[Vector[Int]]): Vector[Vector[Int]]} \\ som filtrerar fram alla yatzy-rader i matrisen \code{xss} enligt nedan. Använd din funktion \code{isYatzy} och samlingsmetoden \code{filter}.
\begin{REPL}
scala> println(diceMatrixToString(filterYatzy(diceMatrix(10000, 5))))
4 4 4 4 4
6 6 6 6 6
4 4 4 4 4
6 6 6 6 6
4 4 4 4 4
4 4 4 4 4
2 2 2 2 2
\end{REPL}



\Subtask Implementera funktionen \\
\code{def yatzyPips(xss: Vector[Vector[Int]]): Vector[Int] = ???}\\
som ska ge en vektor med de tärningsvärden som gav yatzy, för kasten i matrisen \code{xss} enligt nedan. Använd din funktion \code{filterYatzy}.
\begin{REPL}
scala> val dm = Vector(Vector(1,2,3,4,5),Vector(4,4,4,4,4),Vector(3,3,3,3,3))
scala> yatzyPips(dm)
val res42: Vector[Int] = Vector(4, 3)
\end{REPL}

\SOLUTION

\TaskSolved \what

\SubtaskSolved
\begin{Code}
def throwDie: Int = (math.random() * 6).toInt + 1
\end{Code}
Eller:
\begin{Code}
def throwDie: Int = scala.util.Random.nextInt(6) + 1
\end{Code}

\SubtaskSolved  Matrisdimension i matematisk notation: $1000 \times 5$, vilket motsvarar en matris med 1000 rader och 5 kolumner.

\SubtaskSolved
\begin{Code}
ds1: IndexedSeq[IndexedSeq[Int]]
ds2: IndexedSeq[IndexedSeq[Int]]
ds3: IndexedSeq[Vector[Int]]
ds4: IndexedSeq[Vector[Int]]
ds5: Vector[Vector[Int]]
ds6: Vector[Vector[Int]]
\end{Code}
\code{IndexedSeq} och \code{Vector} ovan finns i paketet \code{scala.collection.immutable}

\SubtaskSolved  \begin{Code}
def roll(n: Int) = Vector.fill(n)(throwDie).sorted
\end{Code}

\SubtaskSolved  \begin{Code}
def isYatzy(xs: Vector[Int]): Boolean = xs.forall(_ == xs(0))
\end{Code}



%2.g)
\SubtaskSolved  \begin{Code}
def diceMatrix(m: Int, n: Int): Vector[Vector[Int]] =
  Vector.fill(m)(roll(n))
\end{Code}

\SubtaskSolved  \begin{Code}
def diceMatrixToString(xss: Vector[Vector[Int]]): String =
  xss.map(_.mkString(" ")).mkString("\n")
\end{Code}


%2.j)
\SubtaskSolved
\begin{Code}
def filterYatzy(xss: Vector[Vector[Int]]): Vector[Vector[Int]] =
  xss.filter(isYatzy)
\end{Code}



%2.m)
\SubtaskSolved  \begin{Code}
def yatzyPips(xss: Vector[Vector[Int]]): Vector[Int] =
  filterYatzy(xss).map(_.head)
\end{Code}

\QUESTEND








\WHAT{En oföränderlig, generisk matris-klass till veckans laboration \hyperref[section:lab:\LabWeekEIGHT]{\texttt{\LabWeekEIGHT}}.}

\QUESTBEGIN

\Task\label{exe:matrices:labprep}  \what~Under veckans laboration ska du simulera en enkel form av ''liv'' som består av celler i ett rutnät. För detta ändamål har vi nytta av en matris-klass som du ska implementera steg för steg i denna övning.
Skapa case-klassen nedan med en editor i filen \code{Matrix.scala}. Testa din lösning med hjälp av valfri \hyperref[appendix:ide]{IDE}, t.ex. \code{scalaide} eller \code{idea}.
\begin{Code}
case class Matrix(data: Vector[Vector[String]]){
  def apply(row: Int, col: Int): String = data(row)(col)
}
object Matrix {
  def fill(dim: (Int, Int))(value: String): Matrix =
    Matrix(Vector.fill(dim._1, dim._2)(value))
}
\end{Code}

\begin{REPLnonum}
scala> val m = Matrix.fill(3,4)("hej")
scala> val e = m(2, 2)
\end{REPLnonum}

\Subtask Vad får \code{m} ovan för typ?

\Subtask Vad får \code{e} ovan för typ?

\Subtask På hur många ställen måste du ändra i \code{Matrix} ovan för att den i stället ska representera en matris av heltal?

\Subtask Du ska nu med hjälp av en \textbf{typparameter} göra \code{Matrix} \textbf{generisk} \Eng{generic}, så att den blir en mer användbar matrisklass som kan innehålla element av vilken typ som helst. Genomför följande ändringar i \code{Matrix.scala}:

\begin{itemize}[noitemsep, nolistsep]
  \item Lägg till en typparameter \code{T} inom klammerparenteser efter namnet \code{Matrix} på alla ställen där det förekommer \emph{utom} efter namnet på kompanjonsobjektet\footnote{Singelobjekt kan inte ha typparametrar, men deras medlemmar kan.}.
  \item Byt ut \code{String} mot \code{T} på alla ställen där \code{String} förekommer.
  \item Lägg till en typparameter \code{T} inom klammerparenteser efter \code{def fill}.
\end{itemize}
Testa din generiska klass i REPL genom att skapa en boolesk matris:
\begin{REPLnonum}
scala> val bm = Matrix.fill(3,4)(false)
scala> val be = bm(0, 0)
\end{REPLnonum}

\Subtask Vad får \code{bm} ovan för typ?

\Subtask Vad får \code{be} ovan för typ?

\Subtask Lägg en kodrad i början av klasskroppen som med hjälp av \code{require} garanterar att alla rader i matrisen är lika långa.

\Subtask Lägg till en medlem \code{val dim: (Int, Int)} i klasskroppen efter \code{require}-satsen som ger ett par (alltså en 2-tupel) med antalet rader resp. kolumner i matrisen.

\Subtask Lägg till en metod \code{def updated(row: Int, col: Int)(value: T): Matrix[T]} som ger en ny matris där element på platsen \code{(row, col)} har uppdaterats till \code{value}.

\Subtask Lägg till en metod \code{def foreachIndex(f: (Int, Int) => Unit): Unit} som för varje index i \code{data} applicerar funktionen \code{f}.

\Subtask Lägg till en metod \code{override def toString} som så att en instans av \code{Matrix} visas enligt följande:
\begin{REPLnonum}
scala> val dm = Matrix.fill(3,4)(42.0)
val dm: Matrix[Double] =
Matrix of dim (3,4):
42.0 42.0 42.0 42.0
42.0 42.0 42.0 42.0
42.0 42.0 42.0 42.0
\end{REPLnonum}


\SOLUTION


\TaskSolved \what

\SubtaskSolved Typen på \code{m} blir \code{Matrix}.

\SubtaskSolved Typen på \code{e} blir \code{String}.

\SubtaskSolved Man behöver ändra på 3 ställen från \code{String} till \code{Int}.

\SubtaskSolved Generisk matris \code{Matrix[T]} för element av godtycklig typ \code{T}:

\begin{CodeSmall}
case class Matrix[T](data: Vector[Vector[T]]):
  def apply(row: Int, col: Int): T = data(row)(col)

object Matrix:
  def fill[T](dim: (Int, Int))(value: T): Matrix[T] =
    Matrix[T](Vector.fill(dim._1, dim._2)(value))
\end{CodeSmall}

\SubtaskSolved Tack vare kompilatorns typinferens så får \code{bm} typen \code{Matrix[Boolean]}.

\SubtaskSolved Typen på \code{be} blir \code{Boolean}.

\noindent \SubtaskSolved \SubtaskSolved \SubtaskSolved \SubtaskSolved \SubtaskSolved är alla implementerade i koden nedan: \vspace{-0.5em}
\begin{CodeSmall}
case class Matrix[T](data: Vector[Vector[T]]):
  require(data.forall(row => row.length == data(0).length))

  val dim: (Int, Int) = (data.length, data(0).length)

  def apply(row: Int, col: Int): T = data(row)(col)

  def updated(row: Int, col: Int)(value: T): Matrix[T] =
    Matrix(data.updated(row, data(row).updated(col, value)))

  def foreachIndex(f: (Int, Int) => Unit): Unit =
    for r <- data.indices; c <- data(r).indices do f(r, c)

  override def toString =
    s"""Matrix of dim $dim:\n${ data.map(_.mkString(" ")).mkString("\n") }"""

object Matrix:
  def fill[T](dim: (Int, Int))(value: T): Matrix[T] =
    Matrix[T](Vector.fill(dim._1, dim._2)(value))

\end{CodeSmall}

\QUESTEND


\clearpage

\ExtraTasks %%%%%%%%%%%%%%%%%%%%%%%%%%%%%%%%%%%%%%%%%%%%%%%%%


\WHAT{Imperativa matrisalgoritmer.}

\QUESTBEGIN

\Task  \what~Imperativa angreppssätt är nödvändiga att kunna när du stöter på samlingar och/eller språk som saknar funktionella metoder och/eller funktionsprogrammeringsmöjligheter. Genom att studera imperativa lösningar till de ofta mer koncisa funktionella lösningarna, får du träning i att skapa algoritmer som använder förändring genom tilldelning vid iterering.

\Subtask Implementera \code{isYatzy} från uppgift \ref{matrices:task:yatzy}\ref{matrices:subtask:isyatzyforall} igen, men nu med ett imperativt angreppssätt som använder en \code{while}-sats i stället för funktionella \code{forall}. Ta hjälp av en variabel \code{i} som håller reda på index och en variabel \code{foundDiff} som håller reda på om ett avvikande värde upptäcks. Funktionen kräver ca 9 rader, så det kan vara lämpligt att öppna en editor att skriva i medan du klurar ut lösningen. Börja med att skriva pseudokod, gärna med penna på papper. Prova genom att klistra in i REPL.

\Subtask En imperativ implementation av \code{diceMatrixToString} från uppgift \ref{matrices:task:yatzy}\ref{matrices:subtask:diceMatrixToString} med hjälp av förändringsbara  \code{StringBuilder}\footnote{\url{https://www.scala-lang.org/api/2.12.9/scala/collection/mutable/StringBuilder.html}} visas nedan. Förklara hur nedan kod fungerar. Vad händer om \code{xss} är tom? Vad händer om \code{xss} bara innehåller tomma vektorer? Nämn en fördel och en nackdel med att använda \code{val sb: StringBuilder} och \code{append}, jämfört med en vanlig, oföränderlig \code{var s: String} och \code{+} för tillägg i slutet.
\begin{Code}
def diceMatrixToString(xss: Vector[Vector[Int]]): String = 
  val sb = new StringBuilder()
  for(m <- xss.indices) do
    for(n <- xss(m).indices) do
      sb.append(xss(m)(n).toString)
      if n < xss(m).size - 1 then sb.append(" ")
      else if m < xss.size - 1 then sb.append("\n")
    end for
  end for
  sb.toString
\end{Code}

\Subtask Gör som träning en imperativ implementation av \code{filterYatzy} med en \code{for}-\code{do}-sats (alltså utan att använda \code{filter}, och utan att använda \code{yield}).


\Subtask Förklara hur nedan funktionella implementation av \code{filterYatzy} med \code{for}-\code{yield}-uttryck fungerar. Tycker du din imperativa lösning är lättare eller svårare att läsa och förstå jämfört nedan funktionella lösning?
\begin{CodeSmall}
def filterYatzy(xss: Vector[Vector[Int]]): Vector[Vector[Int]] = 
  (for i <- xss.indices if isYatzy(xss(i)) yield xss(i)).toVector
\end{CodeSmall}


\SOLUTION

\TaskSolved \what

\SubtaskSolved  \begin{Code}
def isYatzy(xs: Vector[Int]): Boolean = 
  var foundDiff = false
  var i = 0
  while (i < xs.size && !foundDiff) do
    foundDiff = xs(i) != xs(0)
    i += 1
  end while
  !foundDiff
\end{Code}


\SubtaskSolved  Funktionen går igenom varje matrisrad, där den i sin tur går igenom
varje element på raden och lägger till i \code{StringBuilder}-objektet. Om det inte är
det sista elementet på raden läggs även ett blanktecken till, annars läggs ett
nyradstecken till. Undantaget är sista raden, där inget nyradstecken läggs till.
Slutligen konverteras \code{StringBuilder}-objektet till en \code{String} som
returneras.


Är \code{xss} tom blir \code{xss.indices} en tom \code{Range} och den yttre \code{for}-loopen hoppas över och en tom sträng returneras.
Är alla rader tomma hoppas i stället de inre \code{for}-looparna över, med samma resultat.

\emph{Fördel:} \code{StringBuilder} är snabbare vid tillägg på slutet vid stora strängar (men här kommer det inte märkas eftersom strängen är så liten).

\emph{Nackdel:} StringBuilder-koden uppfattas av många som svårare att läsa.

\SubtaskSolved
\begin{Code}
def filterYatzy(xss: Vector[Vector[Int]]): Vector[Vector[Int]] = 
  var result: Vector[Vector[Int]] = Vector()
  for i <- xss.indices if isYatzy(xss(i)) do result = result :+ xss(i)
  result
\end{Code}

\SubtaskSolved  Varje looprunda ger en vektor \code{xss(i)} om filtervillkoret är uppfyllt och resultatet av \code{for}-uttrycket blir en vektor med vektorer som är yatzyslag.

\QUESTEND



\WHAT{Strängtabell med kolumnrubriker.}

\QUESTBEGIN

\Task  \what~  %Denna övning utgör en början på laboration \hyperref[section:lab:survey]{\texttt{survey}} i avsnitt \ref{section:lab:survey} på sidan \pageref{section:lab:survey}.

\Subtask Implementera case-klassen \code{Table} enligt specifikationen nedan. Du kan förutsätta att alla rader har lika många kolumner som antalet element i \code{headings}, samt att alla rubrikerna i \code{headings} är unika. Parametern \code{sep} anger det tecken som används för att separera kolumner. Detta förutsätts också gälla för indatafiler som läses in med \code{fromFile}.

\emph{Tips:}
\begin{itemize}%[nolistsep,noitemsep]
\item Värdet \code{indexOfHeading} kan skapas med hjälp av metoden \code{zipWithIndex} som fungerar på alla sekvenssamlingar, samt metoden \code{toMap} som fungerar på sekvenser av 2-tupler. Undersök först hur metoderna fungerar i REPL och sök upp deras dokumentation.
\item Skapa en indatafil som du kan använda för att testa att \code{Table} fungerar.
\end{itemize}


\begin{CodeSmall}
case class Table(
  data: Vector[Vector[String]],
  headings: Vector[String],
  sep: Char
):
  /** A 2-tuple with (number of rows, number of columns) in data */
  val dim: (Int, Int) = ???

  /** The element in row r and column c of data, counting from 0 */
  def apply(r: Int, c: Int): String = ???

  /** The row-vector r in data, counting from 0 */
  def row(r: Int): Vector[String]= ???

  /** The column-vector c in data, counting from 0 */
  def col(c: Int): Vector[String] = ???

  /** A map from heading to index counting from 0 */
  lazy val indexOfHeading: Map[String, Int] = ???

  /** The column-vector with heading h in data */
  def col(h: String): Vector[String] = ???

  /** A vector with the distinct, sorted values of col with heading h */
  def values(h: String): Vector[String] = ???

  /** Headings and data with columns separated by sep */
  override lazy val toString: String = ???

object Table:
  /** Creates a new Table from fileName with columns split by sep */
  def fromFile(fileName: String, sep: Char = ';'): Table = ???
\end{CodeSmall}

\Subtask Skapa med hjälp av \code{Table} ett program som kan köras från terminalen med \texttt{scala run infile.csv ';'} som ger en utskrift av antalet förekomster av olika värden i respektive kolumn (alltså en variant av registrering).



\SOLUTION

\TaskSolved \what

\SubtaskSolved  \begin{CodeSmall}
case class Table(
  data: Vector[Vector[String]],
  headings: Vector[String],
  sep: Char
):

  val dim: (Int, Int) = (data.size, headings.size)

  def apply(r: Int, c: Int): String = data(r)(c)

  def row(r: Int): Vector[String]= data(r)

  def col(c: Int): Vector[String] = data.map(r => r(c))

  lazy val indexOfHeading: Map[String, Int] = headings.zipWithIndex.toMap

  def col(h: String): Vector[String] = col(indexOfHeading(h))

  def values(h: String): Vector[String] = col(h).distinct.sorted

  override def toString: String =
    val s = sep.toString
    headings.mkString(s) + "\n" +data.map(_.mkString(s)).mkString("\n")

object Table:
  def fromFile(fileName: String, sep: Char = ';'): Table = 
    val lines = scala.io.Source.fromFile(fileName).getLines.toVector
    val matrix= lines.map(_.split(sep).toVector)
    new Table(matrix.tail, matrix.head, sep)
\end{CodeSmall}

\SubtaskSolved  \begin{CodeSmall}
@main 
def run(fileName: String, separator: String): Unit = 
  require(separator.length == 1, "separator ska vara exakt ett tecken")
  val t = Table.fromFile(fileName, separator.head)
  val counts: Vector[Vector[String]] =
    (0 until t.dim._2)
      .map(i => t.values(t.headings(i))
      .map(x => s"$x: ${t.col(i).count(_ == x)}"))
      .toVector
  for (i <- 0 until t.dim._2) do
    println(s"\nColumn: ${i + 1}, ${t.headings(i)}:")
    for (j <- 0 until counts(i).length) do
      println(counts(i)(j))
\end{CodeSmall}

\QUESTEND




\WHAT{Skapa ett yatzy-spel för användning i terminalen.}

\QUESTBEGIN

\Task  \what~%
% \Subtask Skapa en yatzy-matris enligt nedan specifikation. Läs om hur de olika predikaten för att kolla olika giltiga kombinationer i Yatzy ska fungera här: \href{https://en.wikipedia.org/wiki/Yahtzee}{en.wikipedia.org/wiki/Yahtzee}. Bygg ett huvudprogram som testar dina funktioner. Kompilera och testa i terminalen allteftersom du lägger till nya funktioner.
%
% \begin{CodeSmall}
% /** En skiss på en klass som kan användas till ett förenklat yatzy-spel */
% case class YatzyRows(val rows: Vector[Vector[Int]]) {
%   /** A new YatzyRows with a new row of 5 dice rolls appended to rows  */
%   def roll: YatzyRows = ???
%
%   /** A new YatzyRows with some indices of the last row re-rolled  */
%   def reroll(indices: Vector[Int]): YatzyRows = ???
% }
%
% object YatzyRows {
%   def isYatzy(xs: Vector[Int]): Boolean = ???
%   def isThreeOfAKind(xs: Vector[Int]): Boolean = ???
%   def isFourOfAKind(xs: Vector[Int]): Boolean = ???
%   def isFullHouse(xs: Vector[Int]): Boolean = ???
%   def isSmallStraight(xs: Vector[Int]): Boolean = ???
%   def isLargeStraight(xs: Vector[Int]): Boolean = ???
% }
% \end{CodeSmall}
%
%
% \Subtask Använd \code{YatzyRows} för att med hjälp av många tärningskast beräkna sannolikheter för några olika giltiga kombinationer. Använd, om du vill, möjligheten som reglerna ger att slå om tärningar i två ytterliggare kast, där de tärningar som slås om väljs slumpmässigt.
%
%\Subtask
Bygg ett förenklat yatzy-spel i terminalen där användaren kan bestämma vilka tärningar som ska slås om. Börja med något riktigt enkelt och bygg sedan vidare på ditt spel genom att införa fler och fler funktioner.

\SOLUTION


\TaskSolved \what
     %starts with: \emph{Skapa ett yatzy-spel för %%%

 --

% \SubtaskSolved   \begin{CodeSmall}
% /** En skiss på en klass som kan användas till ett förenklat yatzy-spel */
% case class YatzyRows(val rows: Vector[Vector[Int]]) {
%
%   private def throwDie: Int = (math.random() * 6).toInt + 1
%
%   /** A new YatzyRows with a new row of 5 dice rolls appended to rows */
%   def roll: YatzyRows = new YatzyRows(rows :+ Vector.fill(5)(throwDie))
%
%   /** A new YatzyRow with some indices of the last row re-rolled */
%   def reroll(indices: Vector[Int]): YatzyRows =
%     new YatzyRows(rows :+ rows(rows.length - 1).zipWithIndex.map {
%       case (x, i) => if (indices.contains(i)) throwDie else x
%     })
% }
% object YatzyRows {
%
%   def isYatzy(xs: Vector[Int]): Boolean = xs.forall(_ == xs(0))
%
%   def isThreeOfAKind(xs: Vector[Int]): Boolean =
%     xs.exists(x => xs.count(_ == x) >= 3)
%
%   def isFourOfAKind(xs: Vector[Int]): Boolean =
%     xs.exists(x => xs.count(_ == x) >= 4)
%
%   def isFullHouse(xs: Vector[Int]): Boolean =
%     xs.exists(x => xs.count(_ == x) == 3) &&
%     xs.exists(x => xs.count(_ == x) == 2)
%
%   def isSmallStraight(xs: Vector[Int]): Boolean =
%     xs.forall(x => xs.count(_ == x) == 1) && !xs.exists(_ == 6)
%
%   def isLargeStraight(xs: Vector[Int]): Boolean =
%     xs.forall(x => xs.count(_ == x) == 1) && !xs.exists(_ == 1)
% }
%
% \end{CodeSmall}
% Observera att fem stycken 2:or uppfyller kraven för Yatzy, men även för triss och fyrtal.
%
% \SubtaskSolved   Slumpen gör att utfallet inte kommer stämma exakt överens med teorin, men för ett stort antal kast bör resultaten hamna ganska nära. De teoretiska sannolikheterna (utan omkast) finns i \ref{yatzyProb}.
% \begin{table}[h]
% \centering
% \caption{Sannolikhet för olika Yatzy-resultat}
% \label{yatzyProb}
% \begin{tabular}{ll}
% Yatzy&  $0,077\%$  \\
% $\geq3$ av samma& $21\%$\\
% $\geq4$ av samma& $2,0\%$\\
% Kåk& $3,9\%$\\
% Liten stege& $1,5\%$\\
% Stor stege& $1,5\%$
% \end{tabular}
% \end{table}
%
% Kodexempel:
% \begin{CodeSmall}
% import YatzyRows._
%
% object YatzyStats extends App {
%   val n = 1000000.0
%   var yr = YatzyRows(Vector(Vector[Int]()))
%   for (i <- 1 to n.toInt) yr = yr.roll
%   println(s"Yatzy: ${yr.rows.count(isYatzy(_)) / n * 100}%")
%   println(s"Three of a kind: ${yr.rows.count(isThreeOfAKind(_)) / n * 100}%")
%   println(s"Four of a kind: ${yr.rows.count(isFourOfAKind(_)) / n * 100}%")
%   println(s"Full house: ${yr.rows.count(isFullHouse(_)) / n * 100}%")
%   println(s"Small straight: ${yr.rows.count(isSmallStraight(_)) / n * 100}%")
%   println(s"Large straight: ${yr.rows.count(isLargeStraight(_)) / n * 100}%")
% }
% \end{CodeSmall}
%
% \SubtaskSolved  --

\QUESTEND






\clearpage

\AdvancedTasks %%%%%%%%%%%%%%%%%


\WHAT{Generiska funktioner.}

\QUESTBEGIN

\Task  \what~  En generisk funktion har (minst) en typparameter inom klammerparenteser efter namnet, till exempel \code{[T]}. Denna typ förekommer sedan som typ på (någon av) parametrarna i parameterlistan. Kompilatorn härleder en konkret typ vid kompileringstid och ersätter typparametern med denna konkreta typ. På så sätt kan en funktion fungera för många olika typer.

\Subtask Förklara för varje rad nedan vad som händer.

\begin{REPL}
scala> def tnirp[T](x: T): Unit = println(x.toString.reverse)
scala> tnirp(42)
scala> tnirp("hej")
scala> case class Gurka(vikt: Int)
scala> tnirp(Gurka(42))
scala> tnirp[String](42)
scala> tnirp[Double](42)
\end{REPL}

\Subtask Man kan kombinera generiska funktioner med funktioner som tar funktioner som parametrar. Det är så \code{map} och \code{foreach} är implementerade. Förklara för varje rad nedan vad som händer.

\begin{REPL}
scala> def compose[A, B, C](f: A => B, g: B => C)(x: A): C = g(f(x))
scala> def inc(x: Int): Int = x + 1
scala> def half(x: Int): Double = x / 2.0
scala> compose(inc, half)(42)
scala> compose(half, inc)(42)
\end{REPL}

\Subtask Hur lyder felmeddelandet på sista raden ovan? Ändra \code{inc} och/eller \code{half} så att typerna passar.

\SOLUTION

\TaskSolved \what
     %starts with: \emph{Generiska funkioner.} En %%%

%4.a)
\SubtaskSolved   \begin{enumerate}
\item --
\item Strängrepresentationen av \code{42} spegelvänds
\item \code{"hej"} spegelvänds - \code{toString} av en sträng ger en likadan sträng
\item --
\item Gurk-objektets strängrepresentation spegelvänds
\item Funktionens typparameter matchar inte parameterns typ: \code{42} är ingen sträng
\item Implicit typkonvertering till \code{Double} sker för att stämma överens med typparametern, vilket ger en strängrepresentation med decimal
\end{enumerate}

%4.b)
\SubtaskSolved   \begin{enumerate}
\item En funktion definieras så att den tar emot två andra funktioner som argument, sätter ihop dem, och matar in ett tredje argument till den den sammansatta funktionen.
\item En funktion som inkrementerar ett heltal med 1 definieras.
\item En funktion som halverar ett flyttal definieras.
\item \code{42} matas in i \code{inc()} och resultatet (\code{43}) matas vidare till \code{half()}. Inuti \code{half()} sker implicit typkonvertering till \code{Double} då talet divideras med ett flyttal (\code{2.0}) och resultatet blir \code{43.0 / 2.0}, alltså \code{21.5}.
\item Resultatet från \code{half()} är av typ \code{Double}, medan \code{inc()} tar emot ett argument av typ \code{Int}. Då flyttal generellt inte kan konverteras till heltal utan informationsförlust sker ingen implicit konvertering, istället sker ett kompileringsfel.
\end{enumerate}

%4.c)
\SubtaskSolved  \begin{Code}
def inc(x: Double): Double = x + 1.0
\end{Code}
Nu ges kompileringsfel på rad 4 istället, vilket kan lösas med följande ändring:
\begin{Code}
def half(x: Double): Double = x / 2.0
\end{Code}

\QUESTEND




\WHAT{Generiska klasser.}

\QUESTBEGIN

\Task  \what~  Även klasser kan vara generiska. En generisk klass har (minst) en typparameter inom klammerparenteser efter klassens namn.

\Subtask Testa nedan generiska klass \code{Cell[T]} i REPL. Skapa instanser av klassen \code{Cell[T]} där typparametern \code{T} binds till olika konkreta typer och förklara vad som händer.

\begin{REPL}
scala> class Cell[T](var value: T):
         override def toString = "Cell(" + value + ")"
       
scala> new Cell(42)
scala> new Cell("hej")
scala> new Cell(new Cell(math.Pi))
scala> new Cell[String](42)
scala> new Cell[Double](42)
\end{REPL}

\Subtask Lägg till metoden \code{def concat[U](that: Cell[U]):Cell[String]} i klassen \code{Cell} som konkatenerar strängrepresentationerna av de båda cellvärdena.

\begin{REPL}
scala> val a = new Cell("hej")
scala> val b = new Cell(42)
scala> a concat b
\end{REPL}

\Subtask Vilken sorts celler kan du konkatenera om du tar bort typparameternamnet \code{U} i \code{concat} samtidigt som du använder \code{Cell[T]} som typ på värdeparametern \code{that}? Vad ger det för konsekvenser för celler av annan typ än \code{Cell[String]}?

\SOLUTION

\TaskSolved \what

%5.a)
\SubtaskSolved  --

%5.b)
\SubtaskSolved  \begin{Code}
class Cell[T](var value: T):
  override def toString = "Cell(" + value + ")"
  def concat[U](that: Cell[U]): Cell[String] = 
    Cell(s"$value${that.value}")
\end{Code}

%5.c)
\SubtaskSolved   Endast celler med samma typparameter kan nu konkateneras. Eftersom \code{concat()} returnerar ett objekt av typ \code{Cell[String]} kan ett ojämnt antal celler med någon annan typparameter än \code{String} alltså inte längre konkateneras. Är antalet jämnt går det att konkatenera dem parvis och sedan konkatenera de returnerade \code{Cell[String]}-objekten, men det är något omständigt.

\QUESTEND

\WHAT{Implementera fler generiska metoder i \code{Matrix[T]}.}

\QUESTBEGIN

\Task \what~ Bygg vidare på uppgift \ref{exe:matrices:labprep} och implementera nedan specifikation. Skapa egna tester som kontrollerar att alla metoder fungerar som förväntat.

\begin{ScalaSpec}{Matrix[T]}
/** En oföränderlig, generisk Matris-klass. */
case class Matrix[T](data: Vector[Vector[T]]):
  require(???)  // garantera att alla rader har lika många kolumner

  /** Ger ett par med antal rader och kolumner. */
  val dim: (Int, Int) = ???

  /** Ger elementet på plats (row, col). */
  def apply(row: Int, col: Int): T = ???

  /** Ger en ny matris där elementet på plats (row, col) har värdet value. */
  def updated(row: Int, col: Int)(value: T): Matrix[T] =  ???

  /** Applicerar f på alla element. */
  def foreach(f: T => Unit): Unit = ???

  /** Applicerar f på alla index. */
  def foreachIndex(f: (Int, Int) => Unit): Unit = ???

  /** Ger en ny matris med resultaten av elementvis applicering av f. */
  def map[U](f: T => U): Matrix[U] = ???

  /** Ger en ny matris med resultaten av applicering av f på varje index. */
  def mapIndex[U](f: (Int, Int) => U): Matrix[U] = ???

  /** Ger en utskriftsvänlig strängrepresentation av matrisen. */
  override def toString = ???

object Matrix:
  /** Ger en matris med dimension dim där alla element har värdet value. */
  def fill[T](dim: (Int, Int))(value: T): Matrix[T] = ???
\end{ScalaSpec}

\SOLUTION


\TaskSolved \what

\begin{CodeSmall}
case class Matrix[T](data: Vector[Vector[T]]):
  require(data.forall(row => row.size == data(0).size))

  val dim: (Int, Int) = (data.length, data(0).length)

  def apply(row: Int, col: Int): T = data(row)(col)

  def updated(row: Int, col: Int)(value: T): Matrix[T] =
    Matrix(data.updated(row, data(row).updated(col, value)))

  def foreach(f: T => Unit): Unit = data.foreach(_.foreach(f))

  def foreachIndex(f: (Int, Int) => Unit): Unit =
    for r <- data.indices; c <- data(r).indices do f(r, c)

  def map[U](f: T => U): Matrix[U] = Matrix(data.map(_.map(f)))

  def mapIndex[U](f: (Int, Int) => U): Matrix[U] =
    var result = Matrix.fill(dim)(f(0,0))
    for 
      r <- data.indices
      c <- data(r).indices 
    do
      result = result.updated(r, c)(f(r, c))
    end for
    result

  override def toString =
    s"""Matrix of dim $dim:\n${ data.map(_.mkString(" ")).mkString("\n") }"""

object Matrix:
  def fill[T](dim: (Int, Int))(value: T): Matrix[T] =
    Matrix[T](Vector.fill(dim._1, dim._2)(value))
\end{CodeSmall}


\QUESTEND





% \WHAT{Skapa en generisk, oföränderlig matrisklass.}
%
% \QUESTBEGIN
%
% \Task \label{task:generic-matrix} \what~   Med hjälp av en typparameter kan vi skapa en matrisklass som kan innehålla vilka element som helst. Implementera nedan specifikation. Testa din matrisklass i REPL för olika typer av element.
%
% \begin{ScalaSpec}{Matrix[T]}
% case class Matrix[T](data: Vector[Vector[T]]){
%
%   def foreachRowCol(f: (Int, Int, T) => Unit): Unit =
%     for (r <- 0 until data.size) {
%       for (c <- 0 until data(r).size) {
%         f(r, c, data(r)(c))
%       }
%     }
%
%   def map[U](f: T => U): Matrix[U] = Matrix(data.map(_.map(f)))
%
%   /** The element at row r and column c */
%   def apply(r: Int, c: Int): T = ???
%
%   /** Gives Some[T](element) at row r and column c
%    *  if r and c are within index bounds, else None */
%   def get(r: Int, c: Int): Option[T] = ???
%
%   /** The row vector of row r */
%   def row(r: Int): Vector[T] = ???
%
%   /** The column vector of column c */
%   def col(c: Int): Vector[T] = ???
%
%   /** A new Matrix with element at row r and col c updated */
%   def updated(r: Int, c: Int, value: T): Matrix[T] = ???
% }
% object Matrix {
%   def fill[T](rowSize: Int, colSize: Int)(init: T): Matrix[T] =
%     new Matrix(Vector.fill(rowSize)(Vector.fill(colSize)(init)))
% }
% \end{ScalaSpec}
%
% \SOLUTION
%
%
% \TaskSolved \what
%      %%%TODO number  8 %%%starts with: \label{task:generic-matrix} \em%%%
%
% \SubtaskSolved  -- %%%TODO in task 8 %%%
%
%
%
% \QUESTEND
%

% \clearpage
%
% \WHAT{Skapa en Sprite-editor.}
%
% \QUESTBEGIN
%
% \Task  \what~ Använd matrisklassen från uppgift \ref{task:generic-matrix} för att göra en SpriteEditor med JColorChoser enligt nedan skiss.
%
% \begin{Code}
% object ColorChooser {
%   import java.awt.Color
%   import javax.swing.JColorChooser
%
%   var title = "Pick Color"
%   private val chooser = new JColorChooser(Color.BLACK)
%   private val dialog = JColorChooser.
%     createDialog(null, title, true, jcs, null, null)
%
%   def getColor(initColor: Color = Color.BLACK): Color = {
%     chooser.setColor(initColor)
%     dialog.setVisible(true)
%     chooser.getColor
%   }
% }
%
% class Sprite(// en bild med många lager av pixlar i olika färger
%   val id: String,
%   val size: (Int, Int),
%   val pixels: Matrix[Int],   // färg i colors, -1 betyder genomskinlig
%   var scale: Int,            // uppskalning av storlek i pixlar
%   var colors: Vector[Color], // tillgängliga färger
%   var pos: (Int, Int, Int)   // (row, col, layer)
% ){
%   def row = pos._1
%   def col = pos._2
%   def layer = pos._3
% }
%
% class SpriteEditor(
%     rows: Int = 64, cols: Int = 64,
%     scale: Int = 16, nColors: Int = 16) {
%   private val w = new SimpleWindow(???)
%   def edit: Unit = ???
% }
%
% \end{Code}
%
%
%
% \SOLUTION
%
%
% \TaskSolved \what
%      %%%TODO number  9 %%%starts with: \TODO \emph{Klasser för täta oc%%%
%
% \SubtaskSolved  -- %%%TODO in task 9 %%%
%
% \SubtaskSolved  -- %%%TODO in task 9 %%%
%
% \SubtaskSolved  -- %%%TODO in task 9 %%%
%
% \SubtaskSolved  -- %%%TODO in task 9 %%%
%
% \SubtaskSolved  -- %%%TODO in task 9 %%%
%
% \SubtaskSolved  -- %%%TODO in task 9 %%%
%
%
%
% \QUESTEND




% \WHAT{Klasser för täta och glesa matematiska matriser med flyttal.}
%
% \QUESTBEGIN
%
% \Task  \what~   Läs om matrisräkning här: \href{https://sv.wikipedia.org/wiki/Matris}{sv.wikipedia.org/wiki/Matris}
%
% \Subtask Skapa en oföränderlig klass \code{DenseMatrix} för matematiska matriser med dubbelprecisionsflyttal. \code{DenseMatrix} ska internt lagra elementen i en privat \emph{endimensionell} array av flyttal av typen \code{Array[Double]}.
%
% Klassen ska inte vara en case-klass. Det ska gå att skapa matriser med uttryck så som  \code{DenseMatrix.ofDim(3,7)(1.0,42,3.2,1.0,2.2,3)} tack vare ett kompanjonsobjekt med lämplig fabriksmetod som anropar den privata konstruktorn.  Om antalet element är för litet i förhållande till den angivna dimensionen så fyll på med nollor.
%
% \Subtask Överskugga metoderna equals och hashcode och ge \code{DenseMatrix} innehållslikhet i stället för referenslikhet.
%
% \Subtask Implementera egna innehålllikhetsmetoder med namnet \code{===} på \code{DenseMatrix} som är typsäker, d.v.s. bara tillåter innehållsjämförelse mellan täta matriser.
%
% \Subtask Läs om glesa matriser här: \href{https://sv.wikipedia.org/wiki/Gles_matris}{https://sv.wikipedia.org/wiki/Gles\_matris} och implementera \code{SparseMatrix} med ett privat attribut av typen \\ \code{mutable.Map[(Int, Int), Double]} som bara lagrar index som inte är noll.
%
% \Subtask Skapa ett \code{trait Matrix} som både \code{DenseMatrix} och \code{SparseMatrix} ärver, med lämpliga abstrakta och konkreta medlemmar. Implementera addition, subtraktion och multiplikation av täta och glesa matriser.
%
% %\Task \emph{Matriser med \jcode{ArrayList} i Java.} Om man i Java inte vet antalet element i matrisen från början kan man använda en lista av typen \jcode{ArrayList}, där varje element i sin tur innehåller en lista av typen\jcode{ArrayList}. Javas \jcode{ArrayList} är en generisk samling som motsvaras av Scalas \code{ArrayBuffer}. Generiska samlingar i Java kan endast innehålla referenstyper; vill man ha en primitiv typ, t.ex. \jcode{int}, behöver man packa in denna i en s.k. wrapper-klass, t.ex.  klassen \jcode{Integer}. Det finns en wrapper-klass för varje primitiv typ i Java. Matristypen för en heltalstyp i Java skrivs \jcode{ArrayList<ArrayList<Integer>>} där alltså \code{<T>} motsvarar Scalas hakparenteser \code{[T]} för typparametern T.
% %
% %
%
% \SOLUTION
%
% \TaskSolved \what
%      %%%TODO number  10 %%%starts with: \emph{Matriser med \jcode{Array%%%
%
% \SubtaskSolved  -- %%%TODO in task 10 %%%
% \QUESTEND
