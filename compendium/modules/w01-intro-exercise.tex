%!TEX encoding = UTF-8 Unicode
%!TEX root = ../exercises.tex

\ifPreSolution
\Exercise{\ExeWeekONE}\label{exe:W01}

\begin{Goals}
%!TEX encoding = UTF-8 Unicode

\item Förstå vad som händer när satser exekveras och uttryck evalueras.
\item Känna till betydelsen av begreppen sekvens, alternativ, repetition och abstraktion.
\item Känna till litteralerna för enkla värden, deras typer och omfång.
\item Kunna deklarera och använda variabler och tilldelning, samt kunna rita bilder av minnessituationen då variablers värden förändras.
\item Förstå skillnaden mellan olika numeriska typer, kunna omvandla mellan dessa och vara medveten om noggrannhetsproblem som kan uppstå.
\item Förstå booleska uttryck och värdena \code{true} och \code{false}, samt kunna förenkla booleska uttryck.
\item Förstå skillnaden mellan heltalsdivision och flyttalsdivision, samt användning av rest vid heltalsdivision.
\item Förstå precedensregler och användning av parenteser i uttryck.
\item Kunna använda \code{if}-satser och \code{if}-uttryck.
\item Kunna använda \code{for}-satser och \code{while}-satser.
\item Kunna använda \code{math.random()} för att generera slumptal i olika intervaller.
\item Kunna beskriva skillnader och likheter mellan en procedur och en funktion.

\end{Goals}

\begin{Preparations}
\item \StudyTheory{01}
\item Du behöver en dator med Scala och Kojo, se appendix~\ref{appendix:compile} och  \ref{appendix:kojo}.
\end{Preparations}

\else

\ExerciseSolution{\ExeWeekONE}

\fi  %%% END \ifPreSolution


\BasicTasks




\WHAT{Para ihop begrepp med beskrivning.}

\QUESTBEGIN

\Task \what

\vspace{1em}\noindent Koppla varje begrepp med den (förenklade) beskrivning som passar bäst:

\begin{ConceptConnections}
\input{generated/quiz-w01-concepts-taskrows-generated.tex}
\end{ConceptConnections}

\SOLUTION

\TaskSolved \what

\begin{ConceptConnections}
\input{generated/quiz-w01-concepts-solurows-generated.tex}
\end{ConceptConnections}

\QUESTEND






\WHAT{Utskrift i Scala REPL.}

\QUESTBEGIN

\Task \what

\vspace{1em}\noindent Starta Scala REPL \Eng{Read-Evaluate-Print-Loop}.

\begin{REPLnonum}
> scala
Welcome to Scala 3.1.2 (17.0.2, Java OpenJDK 64-Bit Server VM).
Type in expressions for evaluation. Or try :help.
scala -version.
scala>
\end{REPLnonum}

\Subtask Skriv efter prompten \code{scala>} en sats som skriver ut en valfri (bruklig/knasig) hälsningsfras, genom anrop av proceduren \code{println} med något strängargument. Tryck på \textit{Enter} så att satsen kompileras och exekveras.

\Subtask Skriv samma sats igen (eller tryck pil-upp) men ''glöm bort'' att skriva högerparentesen efter argumentet innan du trycker på \textit{Enter}. Vad händer?

\begin{framed}
\noindent\emph{Tips inför fortsättningen:} Det finns många användbara kortkommandon och andra trix för att jobba snabbt i REPL. Be gärna någon som kan dessa trix att visa dig hur man kan jobba snabbare. Läs appendix \ref{appendix:compile:REPL} och prova sedan att kopiera och klistra in text. Använd piltangenterna för att bläddra i historiken, Ctrl+A för att komma till början av raden, Ctrl+K för att radera resten av raden, etc.
\end{framed}



\SOLUTION
\TaskSolved \what

\SubtaskSolved Till exempel:
\begin{REPLnonum}
scala> println("hejsan svejsan")
\end{REPLnonum}

\SubtaskSolved Om högerparentes fattas får man fortsätta skriva på nästa rad. Detta indikeras med vertikalstreck i början av varje ny rad:
\begin{REPLnonum}
scala> println("hejsan svejsan"
     | + "!"
     | )
hejsan svejsan!
\end{REPLnonum}

\QUESTEND



\WHAT{Konkatenering av strängar.}

\QUESTBEGIN

\Task \what

\Subtask Skriv ett uttryck som konkatenerar två strängar, t.ex. \code{"gurk"} och \code{"burk"}, med hjälp av operatorn \code{+} och studera resultatet. Vad har uttrycket för värde och typ? Vilken siffra står efter ordet \code{res} i variabeln som lagrar resultatet?

\Subtask Använd resultatet från konkateneringen, t.ex. \code{res0} (byt ev. ut \code{0}:an mot siffran efter \code{res} i utskriften från förra evalueringen), och skriv ett uttryck med hjälp av operatorn \code{*} som upprepar resultatet från förra deluppgiften 42 gånger.


\SOLUTION

\TaskSolved \what

\SubtaskSolved
\begin{REPLnonum}
scala> "gurk" + "burk"
res1: String = gurkburk
\end{REPLnonum}
värde: \code{"gurkburk"}, typ:  \code{String}

\SubtaskSolved
\begin{REPLnonum}
scala> res1 * 42
res2: String = gurkatomatgurkatomatgurkatomatgurkatomatgurkatomatgurkatomatgurkatomatgurkatomatgurkatomatgurkatomatgurkatomatgurkatomatgurkatomatgurkatomatgurkatomatgurkatomatgurkatomatgurkatomatgurkatomatgurkatomatgurkatomatgurkatomatgurkatomatgurkatomatgurkatomatgurkatomatgurkatomatgurkatomatgurkatomatgurkatomatgurkatomatgurkatomatgurkatomatgurkatomatgurkatomatgurkatomatgurkatomatgurkatomatgurkatomatgurkatomatgurkatomatgurkatomat
\end{REPLnonum}

\QUESTEND




\WHAT{När upptäcks felet?}

\QUESTBEGIN

\Task \what

\Subtask Vad har uttrycket \code{ "hej" * 3 } för typ och värde? Testa i REPL.

\Subtask Byt ut 3:an ovan mot ett så pass stort heltal så att minnet blir fullt, 
men inte så stort att talet inte får plats i det givna omfånget för grundtypen \code{Int}. Hur börjar felmeddelandet? Är detta ett körtidsfel eller ett kompileringsfel?

\Subtask Välj ett värde på argumentet efter operatorn \code{*} så att ett typfel genereras. Hur börjar felmeddelandet? Är detta ett körtidsfel eller ett kompileringsfel?

\begin{framed}
\noindent\emph{Tips inför fortsättningen:} Gör gärna fel när du kodar så lär du dig mer! Träna på att tolka olika felmeddelanden och fråga någon om hjälp om du inte förstår. Kompilatorns utskrifter kan vara till stor hjälp, men är ibland kryptiska. Om du kör fast och inte kommer vidare själv så be om hjälp, \emph{men be om tips snarare än färdiga lösningar} så att du behåller initiativet själv och tar kontroll över nästa steg i ditt lärande.
\end{framed}


\SOLUTION

\TaskSolved \what

\SubtaskSolved Typ: \code{String}, värde: \code{"hejhejhej"}

\SubtaskSolved Körtidsfel:
\begin{REPLnonum}
scala> "hej" * Int.MaxValue
java.lang.OutOfMemoryError: Java heap space
\end{REPLnonum}

\SubtaskSolved Kompileringsfel: (indikeras av texten \code{<console> ... error:})
\begin{REPLnonum}
scala> "hej" * true
<console>:12: error: type mismatch;
 found   : Boolean(true)
 required: Int
       "hej" * true
\end{REPLnonum}
Ett typfel innebär att kompilatorn inte kan få typerna att överensstämma i t.ex. ett funktionsanrop. I Scala får vi reda på typfel redan vid kompilering medan i andra språk (t.ex. Javascript) upptäcks sådana fel under exekveringen, i värsta fall genom svårhittade buggar som kanske först märks långt senare.

\QUESTEND




\WHAT{Litteraler och typer.}

\QUESTBEGIN

\Task \what

\Subtask Ta hjälp av REPL-kommadot \verb+:type+ (kan förkortas \code{:t}) vid behov för att para ihop nedan litteraler med rätt typ.

\begin{ConceptConnections}[0.35\textwidth]
\input{generated/quiz-w01-types-taskrows-generated.tex}
%\Connect{\code|1      |}  {\code|Int    |}
%\Connect{\code|1L     |}  {\code|Long   |}
%\Connect{\code|1.0    |}  {\code|Double |}
%\Connect{\code|1D     |}  {\code|Double |}
%\Connect{\code|1F     |}  {\code|Float  |}
%\Connect{\code|'1'    |}  {\code|Char   |}
%\Connect{\code|\"1\"  |}  {\code|String |}
%\Connect{\code|true   |}  {\code|Boolean|}
%\Connect{\code|false  |}  {\code|Boolean|}
%\Connect{\code|()     |}  {\code|Unit   |}
\end{ConceptConnections}

\Subtask Vad händer om du adderar 1 till det största möjliga värdet av typen \code{Int}?
\\\emph{Tips:} se snabbreferensen \footnote{\url{https://fileadmin.cs.lth.se/pgk/quickref.pdf/}} under rubriken ''The Scala type system'' avsnitt ''Methods on numbers''.

\Subtask Vad är skillnaden mellan typerna \code{Long} och \code{Int}?

\Subtask Vad är skillnaden mellan typerna \code{Double} och \code{Float}?


\SOLUTION

\TaskSolved \what

\SubtaskSolved

\begin{ConceptConnections}
\input{generated/quiz-w01-types-solurows-generated.tex}
%\ConnectSolved{\code|1      |}  {\code|Int    |}
%\ConnectSolved{\code|1L     |}  {\code|Long   |}
%\ConnectSolved{\code|1.0    |}  {\code|Double |}
%\ConnectSolved{\code|1D     |}  {\code|Double |}
%\ConnectSolved{\code|1F     |}  {\code|Float  |}
%\ConnectSolved{\code|'1'    |}  {\code|Char   |}
%\ConnectSolved{\code|\"1\"  |}  {\code|String |}
%\ConnectSolved{\code|true   |}  {\code|Boolean|}
%\ConnectSolved{\code|false  |}  {\code|Boolean|}
\end{ConceptConnections}

\SubtaskSolved Värdet går över gränsen för vad som får plats i ett 32 bitars heltal och ''börjar om'' på det minsta möjliga heltalet \code{Int.MinValue} eftersom det är så binär aritmetik med begränsat antal bitar fungerar i CPU:n.
\begin{REPL}
scala> Int.MaxValue + 1
res3: Int = -2147483648

scala> Int.MinValue
res4: Int = -2147483648
\end{REPL}

\SubtaskSolved Båda är heltal men \code{Long} kan representera större tal än \code{Int}.

\SubtaskSolved Båda är flyttal men \code{Double} har dubbel precision och kan representera större tal med fler decimaler.



\QUESTEND





\WHAT{Matematiska funktioner. Använda dokumentation.}

\QUESTBEGIN

\Task \what

\Subtask Antag att du har ett schackbräde med 64 rutor. Tänk dig att du börjar med att lägga ett enda riskorn på första rutan och sedan 
lägger dubbelt så många riskorn i en ny hög för varje efterföljande ruta: 1, 2, 4, 8, ...  etc. När du har gjort detta för alla rutor, 
hur många riskorn har du totalt lagt på schackbrädet?\footnote{\url{https://en.wikipedia.org/wiki/Wheat_and_chessboard_problem}}

\emph{Tips:} Du ska beräkna $2^{64} - 1$. Om du skriver \code{math.} i REPL och trycker TAB får du se inbyggda matematiska funktioner i Scalas standardbibliotek:
\begin{REPLnonum}
scala> math.    // Tryck TAB direkt efter punkten och betrakta listan
\end{REPLnonum}
Använd funktionen \code{math.pow} och lämpliga argument. Om du anger \code{math.pow} eller \code{math.pow()} utan argument får du se funktionshuvudet med 
parameterlistan.

Om du surfar till \url{http://www.scala-lang.org/api/current/} och skriver \code{math} i sökrutan och sedan, efter att du klickat på 
\textbf{\texttt{\small scala.math}}, skriver \textbf{\texttt{\small pow}} i rutan längre ner, så filtreras sidan och du hittar dokumentationen 
av \code{ def pow } som du kan klicka på och läsa mer om.

\Subtask Definiera funktionen \code{omkrets} nedan i REPL. Går det bra att utelämna returtyp-annoteringen? Varför? Finns det anledning att ha den kvar?
\begin{Code}
def omkrets(radie: Double): Double = 2 * math.Pi * radie
\end{Code}

\Subtask Jordens (genomsnittliga) diameter (vid ekvatorn) är ca $12 750$ $km$. Skriv ett uttryck som anropar funktionen \code{omkrets} ovan för att beräkna hur många kilometer per dag man ungefär måste färdas om man vill åka jorden runt på 80 dagar.

\SOLUTION

\TaskSolved \what

\SubtaskSolved Beräkning av $2^{64} - 1$ med \code{math.pow} enligt nedan ger ungefär $1.8 \cdot 10^{19}$
\begin{REPL}
scala> math.pow(2, 64) - 1
res0: Double = 1.8446744073709552E19
\end{REPL}

\SubtaskSolved Ja, returtyp-annoteringen \code{: Double} kan utelämnas.

\begin{itemize}
\item Varför kan returtyp utelämnas?\\Eftersom kompilatorns typhärledning kan härleda returtypen.
\item Varför kan man vilja utelämna den?\\Det blir kortare att skriva utan.
\item Anledningar att ange returtyp:
\begin{itemize}
\item  Med explicit returtyp får du hjälp av kompilatorn att redan under kompileringen kontrollera att uttrycket till höger om likhetstecknet har den typ som förväntas.

\item Genom att du anger returtypen explicit får de som enbart läser metodhuvudet (och inte implementationen)
 tydligt se vad som returneras.
\end{itemize}
\end{itemize}

\SubtaskSolved Ca $500$ $km$.
\begin{REPL}
scala> omkrets(12750 / 2) / 80
res0: Double = 500.6913291658733
\end{REPL}

\QUESTEND




\WHAT{Variabler och tilldelning. Förändringsbar och oföränderlig variabel.}

\QUESTBEGIN

\Task \what~

\Subtask Rita en \emph{ny} bild av datorns minne efter \emph{varje} exekverad rad 1--6 nedan. Varje bild ska visa alla variabler som finns i minnet och deras variabelnamn, typ och värde.

\begin{REPL}[numbers=left, numberstyle=\color{black}\ttfamily\scriptsize\selectfont]
scala> var a = 13
scala> val b = a + 1
scala> var c = (a + b) * 2.0
scala> b = 0
scala> a = 0
scala> c = c + 1
\end{REPL}
Efter första raden ser minnessituationen ut så här:

\MEM{a}{Int}{13}

\Subtask Varför blir det fel på rad 4? Är det ett kompileringsfel eller exekveringsfel? Hur lyder felmeddelandet?

\SOLUTION

\TaskSolved \what

\SubtaskSolved

\begin{tabular}{@{}l l l}
\MEM{{\it Efter rad 1:~~~~} a}{Int}{13}\\
\MEM{{\it Efter rad 2:~~~~} a}{Int}{13} & \MEM{b}{Int}{14}\\
\MEM{{\it Efter rad 3:~~~~} a}{Int}{13} & \MEM{b}{Int}{14} & \MEM{c}{Double}{54.0}\\
\MEM{{\it Efter rad 4:~~~~} a}{Int}{13} & \MEM{b}{Int}{14} & \MEM{c}{Double}{54.0}\\
\MEM{{\it Efter rad 5:~~~~} a}{Int}{0} & \MEM{b}{Int}{14} & \MEM{c}{Double}{54.0}\\
\MEM{{\it Efter rad 6:~~~~} a}{Int}{0} & \MEM{b}{Int}{14} & \MEM{c}{Double}{55.0}\\
\end{tabular}

\SubtaskSolved
Oföränderliga variabler deklareras med nyckelordet \code{val}. Det går inte att tilldela en oföränderlig variabel ett nytt värde; vid försök blir det kompileringsfel som lyder \texttt{\textbf{error: reassignment to val}}. Kompileringsfel känns igen med hjälp av texten \texttt{\textbf{error:}}, så som visas nedan:
\begin{REPLnonum}
scala> b = 0
<console>:12: error: reassignment to val
       b = 0
         ^
\end{REPLnonum}

\QUESTEND


\WHAT{Slumptal med \code{math.random()}.}

\QUESTBEGIN

\Task\label{exercise:expressions:roll} \what

\Subtask Vad ger funktionen \code{math.random()} för resultatvärde? Vilken typ? Vad är största och minsta möjliga värde?
\\\emph{Tips:} Sök här: \Scaladoc och prova i REPL.

\Subtask Deklarera den parameterlösa funktionen \code{def roll: Int = ???} som ska representera ett tärningskast och ge ett slumpmässigt heltal mellan 1 och 6. Testa funktionen genom att anropa den många gånger. \\\emph{Tips:} Använd \code{math.random()} och multiplicera och addera med lämpliga heltal. Omge beräkningen med parenteser och avsluta med \code{.toInt} för att avkorta decimaler och omvandla typen från \code{Double} till \code{Int}.

\SOLUTION

\TaskSolved \what

\SubtaskSolved Ur dokumentationen:
\begin{Code}
/** Returns a Double value with a positive sign,
 *  greater than or equal to 0.0 and less than 1.0.
 */
def random(): Double
\end{Code}
Dokumentationskommentarer, som börjar med \code{/**} och slutar med \code{*/}, ger oss en beskrivning av hur funktionen fungerar. Efter dokumentationskommentaren kommer funktionshuvudet, som här berättar att funktionen heter \code{random} och alltid kommer att returnera en \code{Double}. (Verktyget \code{scaladoc} kan med hjälp av  dokumentationskommentarerna automatiskt generera webbsajter med speciella  dokumentationssidor och sökfunktioner.)

\SubtaskSolved
\begin{REPL}
scala> def roll: Int = (math.random() * 6 + 1).toInt

scala> roll
res0: Int = 4

scala> roll
res1: Int = 1
\end{REPL}

\QUESTEND




\WHAT{Repetition med \code{for}, \code{foreach} och \code{while}.}

\QUESTBEGIN

\Task \what

\Subtask Så här kan en \code{for}-sats ser ut:
\begin{Code}
for i <- 1 to 10 do print(s"$i, ")
\end{Code}
Använd en \code{for}-sats för att skriva ut resultatet av 100 tärningskast med funktionen \code{roll} från uppgift \ref{exercise:expressions:roll}.

\Subtask Så här kan en \code{foreach}-sats ser ut:
\begin{Code}
(1 to 10).foreach(i => print(s"$i, "))
\end{Code}
Använd en \code{foreach}-sats för att skriva ut resultatet av 100 tärningskast med funktionen \code{roll} från uppgift \ref{exercise:expressions:roll}.

\Subtask Så här kan en \code{while}-sats se ut:
\begin{Code}
var i = 1
while i <= 10 do { print(s"$i, "); i = i + 1 }
\end{Code}
Använd en \code{while}-sats för att skriva ut resultatet av 100 tärningskast med funktionen \code{roll} från uppgift \ref{exercise:expressions:roll}. Vad händer om du glömmer \code{i = i + 1} ?


\SOLUTION

\TaskSolved \what

\SubtaskSolved
\begin{Code}
for i <- 1 to 100 do print(s"$roll, ")
\end{Code}

\SubtaskSolved
\begin{Code}
(1 to 100).foreach(i => print(s"$roll, "))
\end{Code}


\SubtaskSolved
\begin{Code}
var i = 1
while i <= 100 do { print(s"$roll, "); i = i + 1 }
\end{Code}

\begin{Code}
var i = 1
while i <= 100 do
    print(s"$roll, ") 
    i += 1
\end{Code}




\QUESTEND






\WHAT{Alternativ med \code{if}-sats och \code{if}-uttryck.}

\QUESTBEGIN

\Task \what

\Subtask Så här kan en \code{if}-sats se ut (notera dubbla likhetstecken):
\begin{Code}
if roll == 3 then println("TRE") else println("INTE TRE")
\end{Code}
Testa ovan i REPL. Skriv sedan en \code{for}-sats som kastar 100 tärningar och skriver ut strängen \code{"GRATTIS! "} om det blir en sexa, annars en ledsen smiley: \code{":("}

\Subtask Så här kan ett \code{if}-uttryck se ut:
\begin{Code}
if roll < 6 then 0 else 1
\end{Code}
Testa ovan i REPL. Skriv sedan en \code{while}-sats som kastar 100 tärningar och räknar antalet sexor. Skriv ut antalet efter \code{while}-satsen.

\SOLUTION

\TaskSolved \what

\SubtaskSolved
\begin{Code}
for i <- 1 to 100 do 
  if roll == 6 then print("GRATTIS! ") else print(":(")
\end{Code}
eller
\begin{Code}
for (i <- 1 to 100) if (roll == 6) print("GRATTIS! ") else print(":(")
\end{Code}

\SubtaskSolved
\begin{Code}
var i = 1
var n = 0
while i <= 100 do
  if roll == 6 then n = n + 1
  i = i + 1
println("Antalet sexor: " + n)
\end{Code}


\QUESTEND



\WHAT{Sekvens, sats och block.}

\QUESTBEGIN

\Task \what

\Subtask Vad gör dessa satser?
\begin{REPLnonum}
scala> def p = { print("san"); print("!"); println("hej")}
scala> p;p;p;p
\end{REPLnonum}

\Subtask
Använd pil-upp för att få tillbaka raden du skrev med definitionen av proceduren \code{p}. Byt plats på strängarna i utskriftsanropen i proceduren \code{p} så att utskriften blir:
\begin{REPLnonum}
hejsan!
hejsan!
hejsan!
hejsan!
\end{REPLnonum}

\Subtask Hur tolkar kompilatorn klammerparenteser och semikolon? Vad är ett block?

\SOLUTION

\TaskSolved \what

\SubtaskSolved
Satserna skapar denna utskrift:
\begin{REPLnonum}
san!hej
san!hej
san!hej
san!hej
\end{REPLnonum}

\SubtaskSolved
\begin{REPLnonum}
scala> def p = { print("hej"); print("san"); println("!")}
scala> p;p;p;p
\end{REPLnonum}

\SubtaskSolved
\begin{itemize}
\item Klammerparenteser används för att gruppera flera satser. 
Klammerparenteser behövs om man vill definiera en funktion som består av mer än en sats. 
Sedan scala 3 kan man istället använda indentering för att definera en funktion med flera rader och satser.

\item Semikolon särskiljer flera satser. Semikolon behövs om man vill skriva många satser på samma rad.


\end{itemize}

\QUESTEND




\WHAT{Heltalsdivision.}

\QUESTBEGIN

\Task \what~Vilket värde och vilken typ hör till vilket uttryck?  Är du osäker på svaret, testa i REPL.

\begin{ConceptConnections}[0.3\textwidth]
\input{generated/quiz-w01-intdiv-taskrows-generated.tex}
\end{ConceptConnections}

\SOLUTION

\TaskSolved \what

\begin{ConceptConnections}[0.3\textwidth]
\input{generated/quiz-w01-intdiv-solurows-generated.tex}
\end{ConceptConnections}

\QUESTEND





\WHAT{Booleska värden.}

\QUESTBEGIN

\Task \what~Vilket värde har dessa uttryck?  % Uppgift 13

\Subtask \code{true && true}

\Subtask \code{false && true}

\Subtask \code{true || true}

\Subtask \code{false || true}


\Subtask \code{false || false}

\Subtask \code{true == true}

\Subtask \code{true != false}


\Subtask \code{true > false}

\Subtask \code{true && (1 / 0 > 1)}

\Subtask \code{false && (1 / 0 > 1)}

\SOLUTION

\TaskSolved \what

\SubtaskSolved \code{true}

\SubtaskSolved \code{false}

\SubtaskSolved \code{true}

\SubtaskSolved \code{true}


\SubtaskSolved \code{false}

\SubtaskSolved \code{true}

\SubtaskSolved \code{true}


\SubtaskSolved \code{true}

\SubtaskSolved Undantag kastas: \code{java.lang.ArithmeticException: / by zero}

\SubtaskSolved \code{false}

\QUESTEND





\WHAT{Booleska variabler.}

\QUESTBEGIN

\Task \what~Vad skrivs ut på rad 2 och 4 nedan?

\begin{REPL}
scala> var monster = false
scala> if monster then println("akta dig!!!")
scala> monster = true
scala> if monster then println("akta dig!!!")
\end{REPL}

\SOLUTION

\TaskSolved \what

\begin{itemize}
\item[2:] Ingenting skrivs ut.
\item[4:] \code{akta dig!!!}
\end{itemize}


\QUESTEND






\WHAT{Turtle graphics med Kojo.}

\QUESTBEGIN

\Task \what~På veckans laboration ska du använda Kojo för att verifiera att du kan använda sekvens, alternativ, repetition och abstraktion. Med Kojo ska du skapa Scala-program som ritar färgglada figurer med hjälp av ett lättanvänt Scala-bibliotek för \emph{turtle graphics}\footnote{\url{https://en.wikipedia.org/wiki/Turtle_graphics}}. 

Om du använder Kojo som ett grafikbibliotek (rekommenderas) och kör med \texttt{scala-cli} (se Appendix \ref{appendix:kojo}) så kan du använda Scala 3. Men kör du Kojo Desktop så är det Scala 2 som gäller och även om det mesta i veckans labb fungerar lika i Scala 2 och Scala 3 så kräver Scala 2 den gamla syntaxen för kontrollstrukturer med nödvändiga parenteser runt villkorsuttryck, utan varken \code{do} eller \code{then}, och varken valfria klammerparenteser eller indenteringssyntax. 

Skriv in och kör nedan program med valfri metod enligt Appendix \ref{appendix:kojo}. Notera kopplingen mellan satsernas ordning och vad som händer i ritfönstret.

\begin{Code}
fram; höger
fram; vänster
färg(grön)
fram
\end{Code}
\noindent Om du kör Kojo Desktop är det bra att börjar programmet med \code{sudda} (varför det?\footnote{När du trycker på playknappen i Kojo Desktop så nollställs varken canvas i ritfönstret eller paddans tillstånd. Genom att börja dina Kojo Desktop-program med \code{sudda} så startar du exekveringen i exakt samma utgångsläge: en tom canvas där paddan pekar uppåt, pennan är nere och pennans färg är röd.}).

\Subtask Skriv kod som ritar en kvadrat enligt bilden nedan.
\vspace{1em}\\\includegraphics[width=0.47\textwidth]{../img/kojo/kvadrat}

\noindent Prova gärna olika sätt att skriva din kod \emph{utan} att resultatet ändras: skriv satser i sekvens på flera rader eller satser i sekvens på samma rad med semikolon emellan; använd blanktecken och blanka rader i koden. Hur vill du gruppera dina satser så att de är lätta för en människa att läsa?
%Prova att ändra på \emph{ordningen} mellan satserna och studera hur resultatet påverkas. Använd den \emph{gula} play-knappen  (programspårning) för att studera exekveringen i detalj. Vad händer du klickar på satser i ditt program och på rutor i programspårningen?


\Subtask Rita en trappa enligt bilden nedan.

\includegraphics[width=0.3\textwidth]{../img/kojo/stairs}

\Subtask Rita valfri bild på valfri bakgrund med hjälp av några av procedurerna i tabellen nedan. Du kan till exempel rita en rosa triangel med lila konturer mot svart bakgrund. % \ref{lab:kojo:kojo-procedures}.
Försök att underlätta läsbarheten av din kod med hjälp av lämpliga radbrytningar och gruppering av satser.


\begin{table}[H]
\begin{longtable}{l l}
\code|fram(100)| & Paddan går framåt 100 steg (25 om argument saknas).\\
\code|färg(rosa)| & Sätter pennans färg till rosa. \\
\code|fyll(lila)| & Sätter ifyllnadsfärgen till lila. \\
\code|fyll(genomskinlig)| & Gör så att paddan \emph{inte} fyller i något när den ritar. \\
\code|bredd(20)| & Gör så att pennan får bredden 20. \\
\code|bakgrund(svart)| & Bakgrundsfärgen blir svart. \\
%bakgrund2 ger ingen gradient i Kojo på webben http://kojo.lu.se/
%\code|bakgrund2(grön,gul)| & Bakgrund med övergång från grönt till gult. \\
\code|pennaNer|  & Sätter ner paddans penna så att den ritar när den går. \\
\code|pennaUpp|  & Sänker paddans penna så att den \emph{inte} ritar när den går. \\
\code|höger(45)|   & Paddan vrider sig 45 grader åt höger. \\
\code|vänster(45)| & Paddan vrider sig 45 grader åt vänster. \\
\code|hoppa|       & Paddan hoppar 25 steg utan att rita. \\
\code|hoppa(100)|  & Paddan hoppar 100 steg utan att rita. \\
\code|hoppaTill(100, 200)| & Paddan hoppar till läget (100, 200) utan att rita. \\
\code|gåTill(100, 200)|    & Paddan vrider sig och går till läget (100, 200). \\
\code|öster|   & Paddan vrider sig så att nosen pekar åt höger. \\
\code|väster|  & Paddan vrider sig så att nosen pekar åt vänster. \\
\code|norr|    & Paddan vrider sig så att nosen pekar uppåt. \\ 
\code|söder|   & Paddan vrider sig så att nosen pekar neråt. \\
%mot funkar inte i iKojo på webben http://kojo.lu.se/
%\code|mot(100,200)|   & Paddan vrider sig så att nosen pekar mot läget (100, 200) \\
\code|sättVinkel(90)| & Paddan vrider nosen till vinkeln 90 grader. \\
\end{longtable}
%\label{lab:kojo:kojo-procedures}
%\caption{Några användbara procedurer i Kojo.}
\end{table}

\begin{framed}
\noindent\emph{Tips inför fortsättningen:} Ha både REPL och en editor igång samtidigt. Då kan du undersöka hur olika kodfragment fungerar i REPL, medan du \emph{stegvis} skapar allt större program i editorn. Detta sätt att jobba har du stor nytta av under resten av kursen. Oavsett vilka andra verktyg du kör är det användbart att ha REPL igång i ett eget fönster som hjälp i den kreativa processen, medan du jagar buggar och medan du lär dig nya koncept. Så fort du undrar hur något fungerar i Scala: \textbf{fram med REPL och testa!}
\end{framed}


\SOLUTION

\TaskSolved \what

\SubtaskSolved Genom att börja din Kojo-program med \code{sudda} så startar du exekveringen i samma utgångsläge: en tom rityta \Eng{canvas} där paddan pekar uppåt, pennan är nere och pennans färg är röd.  Då blir det lättare att resonera om vad programmet gör från början till slut, jämfört med om exekveringen beror på resultatet av tidigare exekveringar.


\SubtaskSolved
\begin{Code}
sudda

fram; vänster
fram; vänster
fram; vänster
fram; vänster
\end{Code}


\SubtaskSolved
\begin{Code}
sudda

fram; vänster
fram; höger

fram; vänster
fram; höger

fram; vänster
fram; höger

fram; vänster
\end{Code}


\QUESTEND









\clearpage

\ExtraTasks %%%%%%%%%%%%%%%%%% EXTRAUPPGIFTER



\WHAT{Typ och värde.}

\QUESTBEGIN

\Task \what~Vilket värde och vilken typ hör till vilket uttryck?  Är du osäker på svaret, testa i REPL.

\begin{ConceptConnections}[0.3\textwidth]
\input{generated/quiz-w01-values-taskrows-generated.tex}
\end{ConceptConnections}

\SOLUTION

\TaskSolved \what

\begin{ConceptConnections}
\input{generated/quiz-w01-values-solurows-generated.tex}
\end{ConceptConnections}

%\Subtask \code{1.0 + 18}
%
%\Subtask \code{(41 + 1).toDouble}
%
%\Subtask \code{1.042e42 + 1}
%
%\Subtask \code{12E6.toLong}
%
%\Subtask \code{"gurk" + 'a'}
%
%\Subtask \code{32.toChar.toString}
%
%\Subtask \code{'A'.toInt}
%
%\Subtask \linebreak[0] \code{'0'.toInt}
%
%\Subtask \code{'0'.toInt}
%
%\Subtask \code{'9'.toInt}
%
%\Subtask \code{'A' + '0'}
%
%\Subtask \code{('A' + '0').toChar}
%
%\Subtask \code{"*!%#".charAt(0)}
%%%%%%%%%%%%%%%%%%%%%%%%%%%%%%%%%%%%%%%%%%%%%%%%
%\SubtaskSolved \code{Double, 19}
%
%\SubtaskSolved \code{Double, 42}
%
%\SubtaskSolved \code{Double, 1.042E42}
%
%\SubtaskSolved \code{Long, 12000000}
%
%\SubtaskSolved \code{String, gurka}
%
%\SubtaskSolved \code{String, " "}
%
%\SubtaskSolved \code{Int, 65}
%
%\SubtaskSolved \code{Int, 48}
%
%\SubtaskSolved \code{Int,49}
%
%\SubtaskSolved \code{Int,57}
%
%\SubtaskSolved \code{Int, 113}
%
%\SubtaskSolved \code{Char, 'q'}
%
%\SubtaskSolved \code{Char, '*'}


\QUESTEND




\WHAT{Satser och uttryck.}

\QUESTBEGIN

\Task \what

\Subtask Vad är det för skillnad på en sats och ett uttryck?

\Subtask Ge exempel på satser som inte är uttryck?

\Subtask Förklara vad som händer för varje evaluerad rad:
\begin{REPL}
scala> def värdeSaknas = ()
scala> värdeSaknas
scala> värdeSaknas.toString
scala> println(värdeSaknas)
scala> println(println("hej"))
\end{REPL}

\Subtask Vilken typ har litteralen \code{()}?

\Subtask Vilken returtyp har \code{println}?

\SOLUTION

\TaskSolved \what

\SubtaskSolved  Ett utryck kan evalueras och resulterar då i ett användbart värde. En sats \emph{gör} något (t.ex. skriver ut något), men resulterat inte i något användbart värde.

\SubtaskSolved \code{println()}

\SubtaskSolved

 \code{värdeSaknas} innehåller Unit

 Skriver ut \code{Unit}

 Skriver ut \code{"()"}

 Skriver ut \code{"()"}

 Skriver först ut hej med det innersta anropet och sen \code{()} med det yttre anropet

\SubtaskSolved  \code{Unit}

\SubtaskSolved  \code{Unit}

\QUESTEND



\WHAT{Procedur med parameter.}

\QUESTBEGIN

\Task \what~En procedur är en funktion som orsakar en effekt, till exempel en utskrift eller en variabeltilldelning, men som inte returnerar något intressant resultatvärde.%
\footnote{I Scala är procedurer funktioner som returnerar det \emph{tomma värdet}, vilket skrivs \code{()} och är av typen \code{Unit}. I Java och flera andra språk finns inget tomt värde och man har en specialsyntax för procedurer som använder nyckelordet \code{void}. }

\Subtask Deklarera en förändringsbar variabel \code{highscore} som initieras till 0.

\Subtask Deklarera en procedur \code{updateHighscore} som tar en parameter \code{points} och tilldelar \code{highscore} ett nytt värde om \code{points} är större än \code{highscore} och skriver ut strängen \code{"REKORD!"}. Om inte \code{points} är större än \code{highscore} ska strängen \code{"GE INTE UPP!"} skrivas ut. Testa proceduren i REPL.

\Subtask Gör en ny variant av \code{updateHighscore}, som \emph{inte} är en procedur utan i stället är en funktion som ger en sträng för senare utskrift. Testa funktionen i REPL.

\SOLUTION

\TaskSolved \what

\SubtaskSolved
\begin{Code}
var highscore = 0
\end{Code}

\SubtaskSolved
\begin{Code}
def updateHighscore(points: Int): Unit =
  if points > highscore then
    highscore = points
    println("REKORD!")
  else println("GE INTE UPP!")
\end{Code}

\SubtaskSolved
\begin{Code}
def updateHighscore(points: Int): String =
  if points > highscore then
    highscore = points
    "REKORD!"
  else "GE INTE UPP!"
\end{Code}



\QUESTEND


\WHAT{Flyttalsaritmetik.}

\QUESTBEGIN

\Task \what

\Subtask Vilket är det minsta positiva värdet av typen \code{Double}?

\Subtask Vad är värdet av detta uttryck? Varför blir det så?
\begin{REPL}
scala> Double.MaxValue + Double.MinPositiveValue == Double.MaxValue
\end{REPL}

\SOLUTION

\TaskSolved \what

\SubtaskSolved

\begin{REPL}
scala> Double.MinPositiveValue
res0: Double = 4.9E-324
\end{REPL}

\SubtaskSolved

\begin{REPL}
scala> Double.MaxValue + Double.MinPositiveValue == Double.MaxValue
res2: Boolean = true
\end{REPL}

\QUESTEND



\WHAT{\code{if}\textit{-sats}.}

\QUESTBEGIN

\Task \what~För varje rad nedan, beskriv vad som skrivs ut.  % Uppgift 18
\begin{REPL}
scala> if !true then println("sant") else println("falskt")
scala> if !false then println("sant") else println("falskt")
scala> def singlaSlant = if math.random() < 0.5 then "krona" else "klave"
scala> for i <- 1 to 5 do print(s"$i:$singlaSlant ")
\end{REPL}

\SOLUTION

\TaskSolved \what

\begin{enumerate}
\item Utskrift: \code{falskt}
\item Utskrift: \code{sant}
\item Inget skrivs ut, funktionen deklareras men körs ej.
\item Utskrift: \code{1:krona 2:klave 3:krona 4:krona 5:klave } eller liknande beroende på vilka slumptal \code{math.random()} ger.
\end{enumerate}

\QUESTEND




\WHAT{\code{if}\textit{-uttryck}.}

\QUESTBEGIN

\Task  Deklarera följande variabler med nedan initialvärden:

\begin{REPLnonum}
scala> var grönsak = "gurka"
scala> var frukt = "banan"
\end{REPLnonum}

Ange för varje rad nedan vad uttrycket har för värde och typ:
\begin{REPLnonum}
scala> if grönsak == "tomat" then "gott" else "inte gott"
scala> if frukt == "banan" then "gott" else "inte gott"
scala> if true then grönsak else 42
scala> if false then grönsak else 42
\end{REPLnonum}

\SOLUTION


\TaskSolved \what~Notera typen \code{Any} på de sista två uttrycken.

\begin{REPLnonum}
scala> if grönsak == "tomat" then "gott" else "inte gott"
res0: String = inte gott

scala> if frukt == "banan" then "gott" else "inte gott"
res1: String = gott

scala> if true then grönsak else 42
res2: Any = gurka

scala> if false then grönsak else 42
res3: Any = 42
\end{REPLnonum}


\QUESTEND





\WHAT{Modulo-operatorn {\tt \%} och Booleska värden.}

\QUESTBEGIN

\Task \what

\Subtask Deklarera en funktion \code{def isEven(n: Int): Boolean = ???} som ger \code{true} om talet \code{n} är jämnt, annars \code{false}.

\Subtask Deklarera en funktion \code{def isOdd(n: Int): Boolean = ???} som ger \code{false} om talet \code{n} är jämnt, annars \code{true}.

\SOLUTION


\TaskSolved \what

\SubtaskSolved
\begin{REPL}
scala> def isEven(n: Int): Boolean = n % 2 == 0

scala> isEven(42)
res0: Boolean = true

scala> isEven(43)
res1: Boolean = false

\end{REPL}


\SubtaskSolved
\begin{REPL}
scala> def isOdd(n: Int): Boolean = !isEven(n)

scala> isOdd(42)
res2: Boolean = false

scala> isOdd(43)
res3: Boolean = true
\end{REPL}


\QUESTEND





\WHAT{Skillnader mellan \code{var}, \code{val}, \code{def}.}

\QUESTBEGIN

\Task \what~

\Subtask
 Evaluera varje rad en i taget i tur och ordning i Scala REPL. För varje rad nedan: förklara för vad som händer och notera värde och ev fel. % Uppgift 15
\begin{REPL}
scala> var x = 30
scala> x + 1
scala> x = x + 1
scala> x == x + 1
scala> val y = 20
scala> y = y + 1
scala> var z = { println("hej z!"); math.random() }
scala> def w = { println("hej w!"); math.random() }
scala> z
scala> z
scala> z = z + 1
scala> w
scala> w
scala> w = w + 1
\end{REPL}


\Subtask Vad är det för skillnad på \code{var}, \code{val} och \code{def}?



\SOLUTION

\TaskSolved \what

\SubtaskSolved
\begin{REPL}
  scala> var x = 30
  x: Int = 30

  scala> x + 1
  res6: Int = 31

  scala> x = x + 1
  x: Int = 31

  scala> x == x + 1
  res7: Boolean = false

  scala> val y = 20
  y: Int = 20

  scala> y = y + 1
  <console>:12: error: reassignment to val
         y = y + 1
           ^

  scala> var z = { println("hej z!"); math.random() }
  hej z!
  z: Double = 0.3381365875903367

  scala> def w = { println("hej w!"); math.random() }
  w: Double

  scala> z
  res8: Double = 0.3381365875903367

  scala> z
  res9: Double = 0.3381365875903367

  scala> z = z + 1
  z: Double = 1.3381365875903368

  scala> w
  hej w!
  res10: Double = 0.06420209879434557

  scala> w
  hej w!
  res11: Double = 0.5777951341051852

  scala> w = w + 1
  <console>:12: error: value w_= is not a member of object
         w = w + 1
\end{REPL}


\SubtaskSolved
\begin{itemize}
\item \code{var namn = uttryck} används för att deklarera en förändringsbar variabel. Namnet kan med hjälp av en tilldelningssats referera till nya värden.

\item \code{val namn = uttryck} används för att deklarera en oföränderlig variabel som efter initialisering inte kan förändras med tilldelningssatser. Vid försök ges kompileringsfel.
\item \code{def namn = uttryck} används för att deklarera en funktion vars uttryck evalueras varje gång den anropas.
\end{itemize}

\QUESTEND




\WHAT{Skillnaden mellan \code{if} och \code{while}.}

\QUESTBEGIN

\Task \what~Vad blir resultatet av rad 3 och 4?

\begin{REPL}
scala> def lotto1 = if math.random() > 0.5 then print("vinst :) ")
scala> def lotto2 = while math.random() > 0.5 do print("vinst :) ")
scala> lotto1
scala> lotto2
\end{REPL}

\SOLUTION

\TaskSolved \what

\begin{itemize}
\item Rad 3: Har du tur (50\% chans) får du vinst en gång.

\item Rad 4: Har du tur får du många vinster i rad. Sannolikheten för $n$ vinster i rad är $(\frac{1}{2})^n$.
\end{itemize}
\QUESTEND












\clearpage

\AdvancedTasks   %%%%%%%%%%%%%%%%%%% FÖRDJUPNINGSUPPGIFTER





\WHAT{Logik och De Morgans Lagar.}

\QUESTBEGIN

\Task \what~Förenkla följande uttryck. Antag att \code{poäng} och \code{highscore} är heltalsvariabler medan \code{klar} är av typen \code{Boolean}.
  % Uppgift 24

\Subtask \code{poäng > 100 && poäng > 1000}

\Subtask \code{poäng > 100 || poäng > 1000}

\Subtask \code{!(poäng > highscore)}

\Subtask \code{!(poäng > 0 && poäng < highscore) }

\Subtask \code{!(poäng < 0 || poäng > highscore) }

\Subtask \code{klar == true}

\Subtask \code{klar == false}

\SOLUTION

\TaskSolved \what


\SubtaskSolved \code{poäng > 1000}

\SubtaskSolved \code{poäng > 100}

\SubtaskSolved \code{poäng <= highscore}

\SubtaskSolved \code{poäng <= 0 || poäng >= highscore }

\SubtaskSolved \code{poäng >= 0 && poäng <= highscore}

\SubtaskSolved \code{klar}

\SubtaskSolved \code{!klar}


\QUESTEND






\WHAT{Stränginterpolatorn \code{s}.}

\QUESTBEGIN

\Task \what~Med ett \code{s} framför en stränglitteral får man hjälp av kompilatorn att, på ett typsäkert sätt, infoga variabelvärden i en sträng.
Variablernas namn ska föregås med ett dollartecken , t.ex. \code{s"Hej $namn"}.
Om man vill evaluera ett uttryck placeras detta inom klammer direkt efter dollartecknet, t.ex.
\code/s"Dubbla längden: ${namn.size * 2}"/

\Subtask Vad skrivs ut nedan?
\begin{REPL}
scala> val f = "Kim"
scala> val e = "Finkodare"
scala> println(s"Namnet '$f $e' har ${f.size + e.size} bokstäver.")
\end{REPL}

\Subtask Skapa följande utskrifter med hjälp av stränginterpolatorn \code{s} och variablerna \code{f} och \code{e} i föregående deluppgift.
\begin{REPL}
Kim har 3 bokstäver.
Finkodare har 9 bokstäver.
\end{REPL}

\SOLUTION

\TaskSolved \what

\SubtaskSolved
\begin{REPLnonum}
Namnet 'Kim Finkodare' har 12 bokstäver.
\end{REPLnonum}

\SubtaskSolved
\begin{REPLnonum}
println(s"$f har  ${f.size} bokstäver.")
println(s"$e har  ${e.size} bokstäver.")
\end{REPLnonum}

\QUESTEND



\WHAT{Tilldelningsoperatorer.}

\QUESTBEGIN

\Task \what~Man kan förkorta en tilldelningssats som förändrar en variabel, t.ex. \code{x = x + 1}, genom att använda så kallade tilldelningsoperatorer och skriva \code{x += 1} som betyder samma sak. Rita en ny bild av datorns minne efter varje rad nedan. Bilderna ska visa variablers namn, typ och värde.

\begin{REPL}
scala> var a = 40
scala> var b = a + 40
scala> a += 10
scala> b -= 10
scala> a *= 2
scala> b /= 2
\end{REPL}

\SOLUTION

\TaskSolved \what

\begin{tabular}{l l}
\MEM{{\it Efter rad1:~~~~} a}{Int}{40}\\
\MEM{{\it Efter rad2:~~~~} a}{Int}{40} & \MEM{b}{Int}{80}\\
\MEM{{\it Efter rad3:~~~~} a}{Int}{50} & \MEM{b}{Int}{80}\\
\MEM{{\it Efter rad4:~~~~} a}{Int}{50} & \MEM{b}{Int}{70} \\
\MEM{{\it Efter rad5:~~~~} a}{Int}{100} & \MEM{b}{Int}{70} \\
\MEM{{\it Efter rad6:~~~~} a}{Int}{100} & \MEM{b}{Int}{35} \\
\end{tabular}

\QUESTEND






\WHAT{Stora tal.}

\QUESTBEGIN

\Task \what~Om vi vill beräkna $2^{64} -1$ som ett exakt heltal\footnote{\url{https://en.wikipedia.org/wiki/Wheat_and_chessboard_problem}} blir det större än \code{Int.MaxValue}, så vi kan tyvärr inte använda snabba \code{Int}. Till vår räddning: \code{BigInt}

\Subtask Läs om \code{BigInt} och \code{BigDecimal} på \Scaladoc \\ Notera vad de kan användas till.

\Subtask Du skapar ett \code{BigInt}-heltal med \code{BigInt(2)} och kan anropa funktionen \code{pow} på en \code{BigInt} med punktnotation. Beräkna $2^{64} -1$ som ett exakt heltal.

\Subtask Vilka nackdelar finns med \code{BigInt} och \code{BigDecimal}?

\SOLUTION

\TaskSolved \what

\SubtaskSolved \code{BigInt} kan användas i stället för \code{Int} vid mycket stora heltal. Det finns förståss även \code{Long} som har dubbelt omfång jämfört med \code{Int}, medan \code{BigInt} kan ha godtyckligt många siffror (ända tills minnet tar slut) och kan därmed representera ofantligt stora tal. \code{BigDecimal} kan användas i stället för \code{Double} vid mycket stora decimaltal.

\SubtaskSolved
\begin{REPL}
scala> BigInt(2).pow(64)
res0: scala.math.BigInt = 18446744073709551616
\end{REPL}

\SubtaskSolved Beräkningar går mycket långsammare och de är lite krångligare att använda.

\QUESTEND





\WHAT{Precedensregler}

\QUESTBEGIN

\Task \what~Evalueringsordningen kan styras med parenteser. Vilket värde och vilken typ har följande uttryck?

\Subtask \code{23 + 2 * 2 + (23 + 2) * 2}

\Subtask \code{(-(2 - 42)) / (1 + 1 + 1)}

\Subtask \code{(-(2 - 42)) / (-1)/(1 + 1 + 1)}

\SOLUTION

\TaskSolved \what

\SubtaskSolved \code{77:  Int}

\SubtaskSolved \code{13: Int}

\SubtaskSolved \code{-13: Int}

\QUESTEND



\WHAT{Dokumentation av paket i Java och Scala.}

\QUESTBEGIN

\Task \what


\Subtask Genom att trycka på tab tangenten kan man se vad som finns i olika paket. Vad heter konstanten $\pi$  i \code{java.lang.Math} (notera stort M) respektive \code{scala.math.}?

\begin{REPL}
scala> java.lang.Math.    //tryck TAB efter punkten
scala> scala.math.        //tryck TAB efter punkten
\end{REPL}

\Subtask Jämför dokumentationen för klassen \code{java.lang.Math} här: \\ \url{https://docs.oracle.com/javase/8/docs/api/} \\
med dokumentationen för paketet \code{scala.math} här: \\
\url{http://www.scala-lang.org/api} \\
Ge exempel på vad man kan göra på webbsidan med Scala-dokumentationen som man \emph{inte} kan göra i motsvarande webbsida Java-dokumentation.

\Subtask Vad gör metoden \code{hypot}? Vad är det som är bra med att använda \code{hypot} i stället för att själv implementera beräkningen med hjälp av kvadratrot, multiplikation och addition?

\SOLUTION

\TaskSolved \what

\SubtaskSolved Scala: \code{Pi}, Java: \code{PI}

\SubtaskSolved Man kan söka och filtrera fram alla förekomster av en viss teckenkombination.

\SubtaskSolved Räknar ut hypotenusan (Pythagoras sats) utan risk för avrundningsproblem i mellanberäkningar.


\QUESTEND





\WHAT{Noggrannhet och undantag i aritmetiska uttryck.}

\QUESTBEGIN

\Task \what~Vad blir resultatet av uttrycken nedan? Notera undantag \Eng{exceptions} och noggrannhetsproblem.

\Subtask \code{Int.MaxValue + 1}

\Subtask \code{1 / 0}

\Subtask \code{1E8 + 1E-8}

\Subtask \code{1E9 + 1E-9}

\Subtask \code{math.pow(math.hypot(3,6), 2)}

\Subtask\Uberkurs \code{1.0 / 0}

\Subtask\Uberkurs \code{(1.0 / 0).toInt}

\Subtask\Uberkurs \code{math.sqrt(-1)}

\Subtask\Uberkurs \code{math.sqrt(Double.NaN)}

\Subtask \code{throw new Exception("PANG!!!")}

\SOLUTION

\TaskSolved \what

\SubtaskSolved \code{-2147483648} vilket motsvarar \code{Int.MinValue}.

\SubtaskSolved Ett undantag kastas: \code{java.lang.ArithmeticException: / by zero}

\SubtaskSolved \code{1.0000000000000001E8} (som förväntat)

\SubtaskSolved Avrundas till \code{1E9} (flyttalsaritmetik med noggrannhetsproblem: ett stort flyttal plus ett (alltför) litet flyttal kan ge samma tal. Det lilla talet ''försvinner'').


\SubtaskSolved \code{45.00000000000001} (flyttalsaritmetik med noggrannhetsproblem: enligt ''normal'' aritmetik ska det bli exakt 45.)

\SubtaskSolved \code{Infinity} (som även ges av \code{Double.PositiveInfinity} och som representerar den positiva oändligheten).

\SubtaskSolved \code{2147483647} vilket motsvarar \code{Int.MaxValue}.

\SubtaskSolved \code{NaN} vilket betyder ''Not a Number''.

\SubtaskSolved \code{NaN} vilket betyder ''Not a Number''.

\SubtaskSolved Ett undantag kastas: \code{java.lang.Exception: PANG!!!}

\QUESTEND



\WHAT{Modulo-räkning med negativa tal.}

\QUESTBEGIN

\Task\Uberkurs \what~Läs om moduloräkning här: \\
 \href{https://en.wikipedia.org/wiki/Modulo\_operation}{en.wikipedia.org/wiki/Modulo\_operation} \\
 och undersök hur det blir med olika tecken (positivt resp. negativt) på moduloräkning med $dividend \% divisor$ i Scala.


\SOLUTION

\TaskSolved \what~I Scala har resultatet samma tecken som dividenden.
\begin{REPL}
scala> 1 % 2
res0: Int = 1

scala> -1 % 2
res1: Int = -1

scala> -1 % -2
res2: Int = -1

scala> 1 % -2
res3: Int = 1

\end{REPL}

\QUESTEND




\WHAT{Bokstavliga identifierare.}

\QUESTBEGIN

\Task\Uberkurs \what~Läs om identifierare i Scala och speciellt \emph{literal identifiers} här: \url{http://www.artima.com/pins1ed/functional-objects.html#6.10}.

\Subtask Förklara vad som händer nedan:
\begin{REPLnonum}
scala> val `bokstavlig val` = 42
scala> println(`bokstavlig val`)
\end{REPLnonum}

\Subtask Scala och Java har olika uppsättningar med reserverade ord. På vilket sätt kan ''backticks'' vara använbart med anledning av detta?

\SOLUTION

\TaskSolved \what

\SubtaskSolved Variabeln får namnet 'bokstavlig val', bakåt-apostrofer \Eng{backticks} gör att man kan namnge variabler till annars otillåtna namn, t.ex. med mellanrum eller nyckelord i sig.

\SubtaskSolved Backticks i Scala möjliggör alla möjliga tecken i namn. Exempel på användning: I java finns en metod som heter \jcode{java.lang.Thread.yield} men i Scala är yield ett nyckelord; för att komma runt det går det att i Scala skriva \jcode{java.lang.Thread.`yield`}

\QUESTEND












\WHAT{\code{java.lang.Integer}, hexadecimala litteraler, BigDecimal.}

\QUESTBEGIN

\Task\Uberkurs \what~

\Subtask Sök upp dokumentationen för \code{java.lang.Integer}.\\Använd metoderna \code{toBinaryString} och \code{toHexString} för att fylla i tabellen nedan.

\begin{table}[H]
\begin{tabular}{l | l | l}
decimalt heltal & binärt värde & hexadecimalt värde \\
\hline
$33$ &   &  \\
$42$ &   &  \\
$64$ &   &  \\
\end{tabular}
\end{table}

\Subtask Hur anger man det hexadecimala heltalsvärdet 10c (motsvarar 268 decimalt) som en litteral i Scala?

\Subtask Vad blir \code{0x10} upphöjt till $c =$ ljusets hastighet i $m/s$? \emph{Tips:} Använd \code{BigDecimal}.

\SOLUTION

\TaskSolved \what

\SubtaskSolved

\begin{REPL}
scala> import Integer.{toBinaryString => toBin, toHexString => toHex}

scala> for i <- Seq(33, 42, 64) do println(s"$i \t ${toBin(i)} \t ${toHex(i)}")
33 	 100001 	 21
42 	 101010 	 2a
64 	 1000000 	 40
\end{REPL}


\SubtaskSolved Det hexadecimala heltalet $10c$ kan anges med litteralen \code{0x10c} i Scala, Java och många andra språk: \footnote{\url{https://en.wikipedia.org/wiki/0x10c}}
\begin{REPL}
scala> 0x10c
res0: Int = 268
\end{REPL}

\SubtaskSolved \footnote{\url{https://c418.bandcamp.com/album/0x10c}}
\begin{REPL}
scala> val c = 299792458
c: Int = 299792458

scala> BigDecimal(0x10).pow(c)
res68: scala.math.BigDecimal = 2.124892963227906613060986110887672E+360986089
\end{REPL}


\QUESTEND









\WHAT{Strängformatering.}

\QUESTBEGIN

\Task\Uberkurs \what~Läs om \code{f}-interpolatorn här:\\
\url{http://docs.scala-lang.org/overviews/core/string-interpolation.html} \\
Hur kan du använda \code{f}-interpolatorn för att göra följande utskrift i REPL? Ändra rad 2 vid \code{???} så att flyttalet \code{g} avrundas till tre decimaler innan utskrift sker.
\begin{REPL}
scala> val g = 2 / 3.0
scala> val str = f"Jättegurkan är $g??? meter lång"
scala> println(str)
Jättegurkan är 0.667 meter lång
\end{REPL}

\SOLUTION

\TaskSolved \what

\begin{Code}
val str = f"Jättegurkan är $g%1.3f meter lång"
\end{Code}
(Om du tycker att \code{$g%1.3f}
ser kryptiskt ut, så kan du trösta dig med att du nu får chansen att föra vidare ett anrikt arv från det urgamla språket C och den sägenomspunna funktionen \code{printf} till kommande generationer av invigda kodmagiker.)

\QUESTEND




\WHAT{Multiplikationsvarning.}

\QUESTBEGIN

\Task \what~Sök upp dokumentationtionen för\\\code{java.lang.Math.multiplyExact} och läs om vad den metoden gör.

\Subtask Vad händer här?

\begin{REPLnonum}
scala> Math.multiplyExact(1, 2)
scala> Int.MaxValue * 2
scala> Math.multiplyExact(Int.MaxValue, 2)
\end{REPLnonum}

\Subtask Varför kan man vilja använda \code{java.lang.Math.multiplyExact} i stället för ''vanlig'' multiplikation?

\SOLUTION

\TaskSolved \what

\SubtaskSolved Den andra multiplikationen flödar över \Eng{overflow} gränsen för största möjliga värdet av en \code{Int}. I den tredje multiplikationen kastas i stället ett undantag \code{java.lang.ArithmeticException: integer overflow}


\begin{REPLnonum}
scala> Math.multiplyExact(1, 2)
res70: Int = 2

scala> Int.MaxValue * 2
res71: Int = -2

scala> Math.multiplyExact(Int.MaxValue, 2)
java.lang.ArithmeticException: integer overflow
  at java.lang.Math.multiplyExact(Math.java:867)
  ... 42 elided
\end{REPLnonum}

\SubtaskSolved Används då man vill vara helt säker på att overflow-buggar ''smäller'' direkt i stället för att generera felaktiga resultat vars konsekvenser kanske manifesterar sig långt senare. Dock är \code{multiplyExact} aningen långsammare än vanlig multiplikation.


\QUESTEND




\WHAT{Extra operatorer för exakt multiplikation.}

\QUESTBEGIN

\Task\Uberkurs \what~Kim Kodmagiker tycker att \\\code{Math.multiplyExact} är för krångligt att skriva och utökar därför typen \code{Int} med en extra operator:

\begin{Code}
extension (i: Int) def *!(j: Int) = Math.multiplyExact(i,j)
\end{Code}

\Subtask Klistra in koden ovan i REPL och prova den extra operatorn.

\Subtask Hjälp Kim Kodmagiker att lägga till fler operatorer på värden av typen \code{Int}, som gör att det även går att använda \code{Math.subtractExact} och \code{Math.addExact} smidigt.

\Subtask Testa ett sammansatt uttryck som använder alla extrametoder på \code{Int}. Tycker du det blev mer lättläst eller mer kryptiskt med de nya operatorerna?

\SOLUTION

\TaskSolved \what

\SubtaskSolved

\begin{REPL}
scala> Int.MaxValue *! 1
res0: Int = 2147483647

scala> Int.MaxValue *! 2
java.lang.ArithmeticException: integer overflow
  at java.lang.Math.multiplyExact(Math.java:867)
  at IntExtra.$times$bang(<console>:16)
  ... 32 elided

\end{REPL}


\SubtaskSolved

\begin{Code}
extension (i: Int) 
  def *!(j: Int) = Math.multiplyExact(i,j)
  def +!(j: Int) = Math.addExact(i,j)
  def -!(j: Int) = Math.subtractExact(i,j)
\end{Code}


\SubtaskSolved Det blir lätt väldigt kryptiskt med namn som består av flera specialtecken. Om du \emph{verkligen} vill ha sådana operatorer är det \emph{mycket} lämpligt att också erbjuda varianter i klartext:
\begin{Code}
extension (i: Int)
  def mulExact(j: Int) = Math.multiplyExact(i,j)
  def *!(j: Int) = i mulExact j

  def addExact(j: Int) = Math.addExact(i,j)
  def +!(j: Int) = i addExact j

  def subExact(j: Int) = Math.subtractExact(i,j)
  def -!(j: Int) = i subExact j

\end{Code}


\QUESTEND



%%%%%%%%%%%%%%%%%%%%%%%%%%%%%%%%%%%%%%%%%%%%%%%%
%%%%%%%%%%%%%%%%%%%%%%%%%%%%%%%%%%%%%%%%%%%%%%%%
%%%%%%%%%%%%%%%%%%%%%%%%%%%%%%%%%%%%%%%%%%%%%%%%
%%%%%%%%%%%%%%%%%%%%%%%%%%%%%%%%%%%%%%%%%%%%%%%%
%%%%%%%%%%%%%%%%%%%%%%%%%%%%%%%%%%%%%%%%%%%%%%%%
%%%%%%%%%%%%%%%%%%%%%%%%%%%%%%%%%%%%%%%%%%%%%%%%
%%%%%%%%%%%%%%%%%%%%%%%%%%%%%%%%%%%%%%%%%%%%%%%%
%%%%%%%%%%%%%%%%%%%%%%%%%%%%%%%%%%%%%%%%%%%%%%%%
%%%%%%%%%%%%%%%%%%%%%%%%%%%%%%%%%%%%%%%%%%%%%%%%


%% Saker som inte fick plats:



%\Task Läs om BigInt och BigDecimal här: \href{http://alvinalexander.com/scala/how-to-use-large-integer-decimal-numbers-in-scala-bigint-bigdecimal}{alvinalexander.com/scala/how-to-use-large-integer-decimal-numbers-in-scala-bigint-bigdecimal} och prova att skapa riktigt stora tal med hjälp av metoden \code{pow} på BigInt och tal med riktigt många decimaler med BigDecimal dess metod \code{pow}.


%
%
%\Subtask\Pen Sök med Ctrl+F i webbläsaren och efter förekomster av texten \textit{''overflow''} i javadoc för klassen \code{java.lang.Math} i JDK 8. Vad är ''overflow''? Vilka metoder finns i \code{java.lang.Math} som hjälper dig att upptäcka om det blir overflow?
%
%\Task Använda Scala REPL för att undersöka konstanterna nedan. Vilket av dessa värden är negativt? Vad kan man ha för praktisk nytta av dessa värden i ett program som gör flyttalsberäkningar?
%
%\Subtask \code{java.lang.Double.MIN_VALUE}
%
%\Subtask \code{scala.Double.MinValue}
%
%\Subtask \code{scala.Double.MinPositiveValue}
%
%\Task För typerna \code{Byte}, \code{Short}, \code{Char}, \code{Int}, \code{Long}, \code{Float}, \code{Double}: Undersök hur många bitar som behövs för att representera varje typs omfång? \\*
%\textit{Tips:} Några användbara uttryck: \\*
% \code{Integer.toBinaryString(Int.MaxValue + 1).size} \\*
% \code{Integer.toBinaryString((math.pow(2,16) - 1).toInt).size} \\*
% \code{1 + math.log(Long.MaxValue)/math.log(2)}
%Se även språkspecifikationen för Scala, kapitlet om heltalslitteraler: \\
%\url{http://www.scala-lang.org/files/archive/spec/2.11/01-lexical-syntax.html#integer-literals}
%
%\Subtask Undersök källkoden för paketobjektet \code{scala.math} här: \\
%\url{https://github.com/scala/scala/blob/v2.11.7/src/library/scala/math/package.scala} \\
%Hur många olika överlagrade varianter av funktionen \code{abs} finns det och för vilka parametertyper är den definierad?
