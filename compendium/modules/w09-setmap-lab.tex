%!TEX encoding = UTF-8 Unicode

%!TEX root = ../compendium2.tex


\Lab{\LabWeekNINE}
\begin{Goals}
%!TEX encoding = UTF-8 Unicode
%!TEX root = ../compendium2.tex

%\item Kunna använda en integrerad utvecklingsmiljö (IDE).
%\item Kunna använda färdiga funktioner för att läsa till, och skriva från, textfil.
%\item Kunna använda enkla case-klasser.
%\item Kunna skapa och använda enkla klasser med föränderlig data.
\item Kunna skapa och använda nyckel-värde-tabeller med samlingstypen \code{Map}.
\item Kunna skapa och använda mängder med samlingstypen \code{Set}.
\item Förstå skillnader och likheter mellan en sekvens och en mängd.
\item Förstå likheter och skillnader mellan en sekvens av par och en nyckel-värde-tabell. 
\item Kunna implementera algoritmer som använder nästlade strukturer. 
%\item Kunna skapa en ny samling från en befintlig samling.
%\item Förstå skillnaden mellan kompileringsfel och exekveringsfel.
%\item Kunna felsöka i små program med hjälp av utskrifter.
%\item Kunna felsöka i små program med hjälp av en debugger i en IDE.

\end{Goals}

\begin{Preparations}
\item \DoExercise{\ExeWeekNINE}{09}
%\item Läs om integrerade utvecklingsmiljöer i appendix \ref{appendix:ide}.
%\item Välj vilken IDE du vill använda på denna lab. %Om du inte vet vilken, välj \textbf{Eclipse} med ScalaIDE, som flest handledare känner väl till.
%\item Bekanta dig med utvecklingsmiljön genom att skapa ett nytt projekt och gör ett ''Hello World''-program.
%\item Ladda hem kursens \emph{workspace} enligt instruktioner i appendix \ref{subsubsection:download--import-workspace} och kontrollera så att du med \emph{Run} kan köra igång de båda ofärdiga \code{main}-metoderna i projektet \code{w04_pirates} inifrån din IDE. Om du inte får rätt på \emph{Run Configuration...} etc. så fråga någon om hjälp.
\item Läs igenom hela laborationen.
\item Hämta och läs given kod via \href{https://github.com/lunduniversity/introprog/tree/master/workspace/w09_words}{kursen github-plats} eller via \href{https://cs.lth.se/pgk/download/}{cs.lth.se/pgk/download}
%\item {\"O}ppna Scala IDE i Eclipse enligt intruktionerna XX.
%\item Skapa ett projekt och skapa ett \code{object Hello} med en \code{main}-metod enligt XY.
%\item Skriv ut en h{\"a}lsning till terminalen med \code{println("...")} och testk{\"o}r programmet genom att markera filnamnet i projektmenyn och trycka p{\aa} den gr{\"o}na pilen. Kontrollera att h{\"a}lsningen skrivs ut!
\end{Preparations}


\subsection{Bakgrund}

Denna uppgift handlar om analys av naurligt språk \Eng{Natural Language Processing, NLP}. Språkanalys bygger ofta på statistik över förekomsten av olika ord i långa texter. Du ska skriva kod, som utifrån en lång text, till exempel en bok, kan hjälpa dig att svara på denna typ av frågor:
\begin{itemize}[noitemsep]
\item Hur vanligt är ett visst ord i en given text?
\item Vilket är det vanligaste ordet som följer efter ett visst ord?
\item Hur kan man generera ordsekvenser som liknar ordföljden i en given text?
\end{itemize}

\noindent För att kunna svara på sådana frågor ska du skapa frekvenstabeller och även så kallade \emph{n-gram}; sekvenser av $n$ ord som förekommer i följd i en text. Exempel på några 2-gram (kallas även \emph{bigram}) som finns i föregående mening: (för, att), (att, kunna), (kunna, svara), (svara, på), (på, sådana), och så vidare.\footnote{Du kan undersöka olika n-gram i en stor mängd böcker med hjälp av Googles n-gram-viewer: \url{https://books.google.com/ngrams/}}

\subsection{Obligatoriska uppgifter}

Du ska bygga ditt program med en editor, t.ex. VS \texttt{code}, och kompilera din kod i terminalen med \code{scalac} eller med hjälp av \code{scala-cli}. Medan du steg för steg utvecklar ditt program, ska du parallellt göra experiment i REPL för att undersöka hur du kan använda samlingsmetoder för att lösa uppgifterna.

Kod att utgå ifrån finns på github här: \url{https://github.com/lunduniversity/introprog/tree/master/workspace/w09_words}

Dessa ofärdiga kodfiler ligger i paketet \code{nlp}:
\begin{itemize}
  \item \href{https://github.com/lunduniversity/introprog/blob/master/workspace/w09_words/FreqMapBuilder.scala}{\texttt{FreqMapBuilder.scala}} innehåller ett skelett till en klass för att, ord för ord, bygga en nyckel-värde-tabell som registrerar antalet förekomster av olika ord. Att implementera denna ingick i övningen du gjorde tidigare i veckan.

  \item \href{https://github.com/lunduniversity/introprog/blob/master/workspace/w09_words/Text.scala}{\texttt{Text.scala}} innehåller ett skelett till en klass som kan göra textbehandling genom att analysera ord i en text.

  \item \href{https://github.com/lunduniversity/introprog/blob/master/workspace/w09_words/Main.scala}{\texttt{Main.scala}} innehåller ett ofärdigt huvudprogram som du kan använda i laborationens senare del.
\end{itemize}

För att underlätta ditt arbetsflöde under det att du stegvis bygger upp din kod metod för metod, kan du med fördel använda verktyget \texttt{scala-cli} (se appendix \ref{appendix:build}) så här:

\begin{itemize}
  \item
    Med kommandot \code{scala-cli repl .}~ startar du REPL med dina klasser i aktuella katalogen (därav punkten efter blanksteg efter \code{scala repl}) automatiskt tillgängliga på classpath och du kan anropa dina metoder när du experimenterar inför nästa steg. 
  \item
    Du kan också ge annan sökväg som argument till kommandot, exempelvis \code{scala-cli repl minKatalog} och när du ändrat något i din editor och vill experimentera med nya versionen så stänger du ned REPL (exempelvis med Ctrl+D eller Ctrl+C eller \code{:q}) och startar om genom att återupprepa kommandot \code{scala-cli repl .}
  \item
    Med \texttt{scala-cli run .} sker kompilering och körning av \code{main}-metoden i aktuella katalogen. Du kan ange en annan katalog genom att skicka med sökvägen som argument till kommandot. 
  \item
    Om det finns flera \texttt{main}-metoder i aktuella katalogen, går det att specificera vilken av dessa som ska exekveras med optionen \verb|--main-class|, exempelvis\\ \verb|scala-cli run .  --main-class Hello| \\ Argument anges efter dubbla minustecken \verb|--| så här: \\\verb|scala-cli run . -- arg1 arg2 arg3|
\end{itemize}


\Task \emph{Skapa frekvenstabeller}. Du ska använda \code{FreqMapBuilder} från veckans övning för att skapa frekvenstabeller av typen \code{Map[String, Int]}, där nyckel-värde-paren i tabellen anger antalet förekomster av en viss sträng.

\Subtask Lägg klassen \code{FreqMapBuilder} i ett paket som heter \code{nlp} och kompilera.

\begin{figure}[H]
\scalainputlisting[numbers=left,basicstyle=\ttfamily\fontsize{10.5}{12.5}\selectfont]{../workspace/w09_words/FreqMapBuilder.scala}
%\caption{Den ofärdiga klassen \code{FreqMapBuilder}.}
%\label{data:fig-freqmap}
\end{figure}

\Subtask Testa noga så att din \code{FreqMapBuilder} fungerar korrekt. Exempel på test i REPL:
\begin{REPL}
scala> import nlp._

scala> val fmb = FreqMapBuilder("hej", "på", "dej")
fmb: nlp.FreqMapBuilder = nlp.FreqMapBuilder@458f85ef

scala> fmb.add("hej")

scala> fmb.toMap
res0: Map[String,Int] = Map(på -> 1, hej -> 2, dej -> 1)

scala> (1 to Short.MaxValue).foreach(i => fmb.add(i.toString))

scala> fmb.toMap.size
res1: Int = 32770

scala> fmb.toMap
res2: Map[String,Int] = Map(10292 -> 1, 19125 -> 1, 26985 -> 1, 29301 -> 1, 5451 -> 1, 4018 -> 1, 31211 -> 1, 17319 -> 1, 20778 -> 1, 25285 -> 1, 17079 -> 1, 9936 -> 1, 13172 -
\end{REPL}

\noindent I kommande uppgifter ska du steg för steg skapa och testa case-klassen \code{Text}. %figur \ref{data:fig-text}.

\begin{figure}[H]
\scalainputlisting[numbers=left,basicstyle=\ttfamily\fontsize{10.4}{12.5}\selectfont]{../workspace/w09_words/Text.scala}
%\caption{Den ofärdiga klassen \code{Text}.}
%\label{data:fig-text}
\end{figure}





\Task \emph{Dela upp en sträng i ord}. Du ska implementera medlemmen \code{words}. Den ska innehålla en vektor med alla ord i \code{source}, utan andra tecken än bokstäver.
Detta åstadkommer du genom att utgå ifrån strängen \code{source} och i tur och ordning göra följande:
\begin{enumerate}%[nolistsep, noitemsep]
\item byta ut alla tecken i \code{source} för vilka \code{isLetter} är falskt mot \code{' '}
\item dela upp strängen från föregående steg i en array av strängar med \code{split(' ')}
\item filtrera bort alla tomma strängar
\item gör om alla bokstäver i alla strängar till små bokstäver
\item gör om arrayen till en sekvens av typen \code{Vector[String]}.
\end{enumerate}

\noindent Testa så att \code{words}, och de värden som använder \code{words}, fungerar i REPL:
\begin{REPL}
scala> val t = Text("Gurka är ingen tomat, men gurka är en grönsak.")

scala> t.words
res1: Vector[String] =
  Vector(gurka, är, ingen, tomat, men, gurka, är, en, grönsak)

scala> t.distinct
res2: Vector[String] =
  Vector(gurka, är, ingen, tomat, men, en, grönsak)

scala> t.wordSet
res3: Set[String] = Set(grönsak, är, gurka, men, ingen, tomat, en)

scala> t.wordsOfLength(5)
res4: Set[String] = Set(gurka, ingen, tomat)

\end{REPL}



\Task Du ska nu skapa ordfrekvenstabellen \code{wordFreq} genom att registrera ordförekomster med hjälp av \code{FreqMapBuilder}. Tabellen \code{wordFreq} ska bestå av nyckelvärdepar \code{w -> f} där \code{f} är antalet gånger ordet \code{w} förekommer i \code{words}. Testa \code{wordFreq} genom att ladda ner boken ''Skattkammarön'' skriven av Robert Louis Stevenson\footnote{Copyright för denna bok har gått ut, så du gör dig inte skyldig till piratkopiering (i juridisk mening).} och undersök frekvensen för olika vanliga ord. Vilket ord är vanligast? Näst vanligast?

\begin{REPL}[basicstyle=\color{white}\ttfamily\fontsize{9}{11}\selectfont]
scala> val piratbok = Text.fromURL("https://fileadmin.cs.lth.se/pgk/skattkammaron.txt")
val piratbok: nlp.Text = Text(Herr Trelawney, doktor Livesey och de övriga herrarna har bett mig att skriva ner alla omständigheter kring Skattkammarön, ...

scala> piratbok.words.size
val res0: Int = 69438

scala> piratbok.wordFreq("pirat")
val res1: Int = 7
\end{REPL}
Länkar till böcker i UTF-8-format som du kan använda i dina tester:
\begin{itemize}%[nolistsep,noitemsep]
\item ''Skattkammarön'' av R. L. Stevenson: \\\url{https://fileadmin.cs.lth.se/pgk/skattkammaron.txt}
\item ''Inferno'' av August Stringberg: \\\url{https://fileadmin.cs.lth.se/pgk/inferno.txt}
\item ''Pride and Prejudice'' av Jane Austen: \\\url{https://fileadmin.cs.lth.se/pgk/prideandprejudice.txt}
\item Projekt Gutenberg med många fritt tillgängliga böcker i textformat: \\\url{https://www.gutenberg.org/}
\end{itemize}






\Task Implementera metoden \code{ngrams} som ger en sekvens med alla ordföljder i $n$ steg. \emph{Tips:} På veckans övning ingick att undersöka hur metoden \code{sliding} fungerar, med vilken du kan skapa $n$-gram. Gör \code{toVector} på resultatet från \code{sliding}. Testa noga så att \code{ngrams} och \code{bigrams} fungerar korrekt innan du går vidare.
\begin{REPL}
scala> piratbok.ngrams(3).take(2)
val res1: Vector[Vector[String]] =
  Vector(Vector(herr, trelawney, doktor), Vector(trelawney, doktor, livesey))

scala> piratbok.bigrams.take(2)
val res2: Vector[(String, String)] =
  Vector((herr,trelawney), (trelawney,doktor))
\end{REPL}

\Task Implementera \code{followFreq}, som ska innehålla en nyckel-värde-tabell där värdet i sin tur är en frekvenstabell över de ord som kommer efter nyckeln. \label{task-follow-freq}

Genom att analysera alla ordpar kan vi få fram vilket som är det vanligaste ordet som följer efter ett givet ord. Metoden \code{bigrams} ger oss alla ordpar \code{(w1, w2)} där \code{w2} följer efter \code{w1}. Vi kan spara statistiken över efterföljande ord i en nyckelvärdetabell med mappningarna \code{w -> f} där nyckeln \code{w} är ett ord  och värdet \code{f} är en frekvenstabell av typen \code{Map[String, Int]}. I frekvenstabellen lagrar vi frekvensen för alla de ord som följer efter \code{w}. Du ska alltså bygga en nästlad tabell av typen \code{Map[String, Map[String, Int]]}. Rita en bild av den nästlade strukturen.\Pen

Implementera metoden followFreq genom att utgå från nedan pseudokod:
\begin{Code}
val result = scala.collection.mutable.Map.empty[String, FreqMapBuilder]
for (key, next) <- bigrams do
  if /* key finns redan definierad i result */ then
    /* på "platsen" result(key): lägg till next i frekvenstabellen */
  else
    /* lägg till (key -> ny frekvenstabell med next) i result*/
end for
result.map(p => p._1 -> p._2.toMap).toMap // toMap ger oföränderlig Map
\end{Code}
Gör utskrifter för att ta reda på följande frågor. Skriv ner svaren och var redo att redovisa dem i samband med kontrollfrågorna (se avsnitt \ref{words-check}).\Pen

\Subtask Vilka ord brukar följa efter \emph{han} respektive \emph{hon} i Stevensons ''Skattkammarön''?

\Subtask Vilka ord brukar följa efter \emph{han} respektive \emph{hon} i Stringbergs ''Inferno''?

\Subtask Vilka ord brukar följa efter \emph{he} respektive \emph{she} i Austens ''Pride and Prejudice''?


\Task Skapa ett huvudprogram som rapporterar valfria, intressanta mått om orden i en text. Programmet ska ta textens källa som argument, givet som en URL eller ett filnamn. Skriv huvudprogrammet i filen \code{Main.scala} i ett singelobjekt med namnet \code{Main}. Exempel på en rapport som ditt huvudprogram kan generera finns nedan. Här ges även ett heltal som argument som styr topplistornas längd.
\begin{REPL}
> scala nlp.Main https://fileadmin.cs.lth.se/pgk/skattkammaron.txt 13

Källa: https://fileadmin.cs.lth.se/pgk/skattkammaron.txt

*** Antal ord: 69438

*** De 13 vanligaste orden och deras frekvens:
(och,3089), (jag,2007), (att,1594), (det,1382), (en,1262),
(i,1244), (som,1132), (på,1068), (han,1063), (var,990),
(med,854), (den,774), (av,740)

*** De 13 längsta orden och deras längd:
(besättningsmedlemmarnas,23), (befästningsanordningar,22),
(temperamentsuppvisning,22), (undsättningsexpedition,22),
(besättningsmedlemmarna,22), (försiktighetsåtgärder,21),
(undsättningsfartyget,20), (sjukdomsframkallande,20),
(husföreståndarinnans,20), (sjömansterminologin,19),
(parlamentärsflaggan,19), (bregravningsplatsen,19),
(tidvattenströmmarna,19)
\end{REPL}

\noindent Exempel på huvudprogram som kan skapa ovan utskrift:
\scalainputlisting[numbers=left,basicstyle=\ttfamily\fontsize{10.4}{12.5}\selectfont]{../workspace/w09_words/Main.scala}

\subsection{Kontrollfrågor}\label{words-check}

\begin{enumerate}[noitemsep, nolistsep]

\item Vilket är dina svar på uppgift \ref{task-follow-freq} a) b) c) på sidan \pageref{task-follow-freq}?

\item I vilken ordning hamnar elementen om man anropar \code{distinct} på en sekvens?

\item Om man itererar över en mängd, i vilken ordning behandlas elementen?

\item Ge exempel på när är det lämpligt att använda en mängd i stället för en sekvens av distinkta värden?

\item Är alla nycklar i en nyckel-värde-tabell garanterat unika?

\item Är alla värden i en nyckel-värde-tabell garanterat unika?

\item LTH-teknologen Oddput Clementin vill summera längden på varje sträng i en mängd och skriver:
\begin{REPL}
scala> Set("hej", "på", "dej").map(_.length).sum
res0: Int = 5
\end{REPL}
Varför blir det fel? Hur kan Oddput åtgärda problemet?
\end{enumerate}

\subsection{Frivilliga uppgifter}

\Task Bygg vidare på klassen \code{Text} och implementera nedan metod som ska ge ett slumpmässigt ord ur \code{wordSet}. Varje ord ska förekomma med lika stor sannolikhet.
\begin{Code}
def randomWord: String = ???
\end{Code}

\Task \label{task:words:randomSeq} Med NLP kan man generera slumpmässiga meningar som statistiskt sett liknar ''riktiga'', människoskapade meningar.

Implementera metoden \code{randomSeq(firstWord, n)} nedan i klassen \code{Text}. Den ska ge en sekvens $w_{1}, w_{2}, ..., w_{n}$  där $w_{1}$ är \code{firstWord} och $w_{i+1}$ är något slumpmässigt ord som är draget bland de ord som följer efter $w_{i}$. Detta kan du åstadkomma genom att varje efterföljande ord $w_{i+1}$ väljs ur \code{keys.toVector} för den \code{followFreq}-tabell som hör till $w_{i}$. Orden ska dras ur efterföljandemängden, med lika stor sannolikhet.
\begin{Code}
def randomSeq(firstWord: String, n: Int): Vector[String] = ???
\end{Code}
%\emph{Tips:} Ett sätt att garanterat välja slumpmässigt element med rektangelfördelning ur en sekvens är att använda metoden \code{scala.util.Random.shuffle} som tar en sekvens som argument och genererar en ny sekvens av samma typ, men med elementen ordnade i slumpmässig ordning på ett välblandat sätt, där varje möjlig ordning är lika sannolik.

\Task \label{task:words:mostCommonSeq} För att dina datorgenererade meningar verkligen ska likna mänskilgt språk kan vi skapa de mest sannolika meningarna av olika längder ur vår analys av ordfrekvenser.

Lägg till metoden \code{mostCommonSeq} i klassen \code{Text} enligt nedan:
\begin{Code}
def mostCommonSeq(firstWord: String, n: Int): Vector[String] = ???
\end{Code}
\Subtask Implementera metoden så att resultatet blir en sekvens med \code{n} ord. Sekvensen ska börja med \code{firstWord} och därefter följas av det ord som är det \emph{vanligaste} efterföljande ordet efter \code{firstWord}, och därpå det vanligaste efterföljande ordet efter det, etc. \emph{Tips:} Använd en lokal variabel \code{val result} som är en ArrayBuffer till vilken du i en \code{while}-loop lägger de efterföljande orden.

\Subtask Jämför de slumpmässiga sekvenserna med sekvenser genererade med \code{randomSeq} i uppgift \ref{task:words:randomSeq}. Vilka sekvenser liknar mest ''riktiga'' meningar?


\Task Använd befintliga samlingsmetoder i stället för \code{FreqMapBuilder} för att registrera efterföljande ord.

\Subtask Undersök i REPL hur metoden \code{groupBy(x => x)} fungerar då den appliceras på en samling med strängar. Sök efter och studera dokumentationen för \code{groupBy}.

\Subtask Inför värdet \code{lazy val wordFreq2}. Den ska ge samma resultat som \code{wordFreq} men men implementeras med hjälp av \code{groupBy} och \code{map} i stället för \code{FreqMapBuilder}.

\Subtask\Uberkurs Jämför prestanda mellan \code{wordFreq2} och \code{wordFreq}. Vilken är snabbast för stora texter? Är skillnaden stor?

\Subtask Inför värdet \code{lazy val followsFreq2}. Den ska ge samma resultat som \code{followsFreq} men implementeras med hjälp av \code{groupBy} och \code{map} i stället för \code{FreqMapBuilder}.
Denna uppgift är ganska knepig. Experimentera dig fram i REPL, och bygg upp en lösning steg för steg. \emph{Tips:}
\begin{Code}
bigrams
  .groupBy(???)
  .map(p => p._1 -> p._2.map(???).groupBy(???).map(???))
\end{Code}

\Subtask\Uberkurs Jämför prestanda mellan \code{followsFreq2} och \code{followsFreq}. Vilken är snabbast för stora texter? Är skillnaden stor?

\Task \emph{Gör \code{FreqMapBuilder} generisk.} Generiska strukturer, alltså sådana som har typparametrar, är ofta väsentligt mycket mer användbara. Om du gör \code{FreqMapBuilder} generisk genom att införa en typparameter i stället för att hårdkoda typen till \code{String} så kan du använda \code{FreqMapBuilder} med godtycklig elementtyp. 

\Subtask Studera \code{FreqMapBuilder} och identifiera allt i den klassen som är specifikt för typen \code{String}.

\Subtask Inför en typparameter \code{A} inom hakparenteser efter klassnamnet och använd sedan \code{A} i stället för \code{String} i alla metoder.

\Subtask Testa så att din generiska frekvenstabellbyggare fungerar på sekvenser som innehåller annat än strängar.

Detta funkar eftersom inget i \code{FreqMapBuilder} egentligen förutsätter att elementen som ska räknas är av sträng-typ (det räcker att det finns en vettig \code{equals} och \code{hashcode}).
