%!TEX encoding = UTF-8 Unicode
\documentclass[a4paper]{compendium}
%\usepackage{xr} %to crossreference to compendium1.tex
\externaldocument{compendium1}
\usepackage[swedish]{babel}


\addto\captionsswedish{%
  \renewcommand{\appendixname}{Appendix}%
}
%TODO: Glossary
%http://tex.stackexchange.com/questions/5821/creating-a-standalone-glossary/5837#5837

\setlength{\columnsep}{16mm}

\newcommand{\LibVersion}{1.3.1} % latest version of introlib at https://github.com/lunduniversity/introprog-scalalib
\newcommand{\LibJar}{\texttt{introprog\_3-\LibVersion.jar}}
\newcommand{\JDKApiUrl}{\url{https://docs.oracle.com/en/java/javase/17/docs/api/}}
\newcommand{\CurrentYear}{2024}
\newcommand{\VMName}{vm2020} %TODO: update vm
\newcommand{\VMPassword}{pgkBytMig2020}
\newcommand{\VirtualBoxVersion}{7.0} %https://www.virtualbox.org/wiki/Downloads
\newcommand{\UbuntuVersion}{22.04}
\newcommand{\ScalaVersion}{3.4.2} %https://www.scala-lang.org/
\newcommand{\SbtVersion}{1.10.0} %https://eed3si9n.com/category/tags/sbt
\newcommand{\JDKVersion}{17} %https://adoptium.net/temurin/releases/?version=17
\newcommand{\KojoVersion}{2.9.28} %https://www.kogics.net/kojo-download
\newcommand{\VSCodeVersion}{1.90} %https://code.visualstudio.com/updates
\newcommand{\MetalsVersion}{v1.35} %https://marketplace.visualstudio.com/items?itemName=scalameta.metals
\newcommand{\WindowsVersion}{10}
\newcommand{\ScalaIDEVersion}{4.7.0} %%DEPRECATED
\newcommand{\OmkontrollDatum}{Torsd. 14/11 kl 13:00-18:00, E:1147}%Tors W09; används i lect-w08 och lect-w09
\newcommand{\LastLectureDate}{onsdagen den 6:e december i E:A kl 10-12}





\title{
{\vspace{-3.0cm}\bf\sffamily\Huge\selectfont  Introduktion till programmering med Scala}
\\ \vspace{2em}%\hspace*{1.5cm}\inputgraphics[width=0.6\textwidth]{../img/gurka} \\
{\sffamily \textbf{Kompendium 2}\\Andra läsperioden: Modul 8 -- 14}\\\vspace{2cm}
\includegraphics[height=11cm]{../img/glider-blinker-block}
%\includegraphics[height=4cm]{../img/scala-logo.png}
%\includegraphics[height=4cm]{../img/java-logo.png}
%\includegraphics[height=12cm]{cover/gurka.jpg}
}

\author{Björn Regnell}
\date{\raggedbottom%
\vspace{1em}\begin{minipage}{1.0\textwidth}\centering
EDAA45, Lp1-2, HT \CurrentYear\\
Datavetenskap, LTH\\
Lunds universitet\\
~\\
Kompileringsdatum: \today \\
\url{https://lunduniversity.github.io/pgk}
\end{minipage}
}

\usepackage{multicol}

\usepackage{pgffor}  %% http://stackoverflow.com/questions/2561791/iteration-in-latex
%  allows:  \foreach \n in {1,...,4}{ do something with \n }

\usepackage{framed}  %  allows:   \begin{framed}\end{framed}
\FrameSep5pt
\OuterFrameSep0pt

% \newenvironment{Slide}[2][]{%
% \begin{oframed}\setlist{noitemsep}%
% {\vspace{-1.5\topsep}}%tighter frames
% \subsection{#2}%
% }%
% {\end{oframed}} 

\newenvironment{Slide}[2][]{%
%\noindent\rule{\textwidth}{0.4pt}%
\setlist{noitemsep}%
%{\vspace{-1.5\topsep}}%tighter frames
\subsection{#2}%
}%
{~\newline\noindent\rule{\textwidth}{0.4pt}}

% \newcommand{\SlideHeading}[1]{\section*{#1}}

% \usepackage[most]{tcolorbox}
% \newenvironment{Slide}[2][]
%   {\vspace{0.5em}\begin{tcolorbox}[left=1.5em,%width=1.05\textwidth,
%   grow to right by=0.05\textwidth,grow to left by=0.05\textwidth,%
%   %breakable,
%   %frame hidden,
%   colframe=gray!20,
%   enhanced]\setlist{noitemsep}\SlideHeading{#2}}
%   {\end{tcolorbox}\vspace{0.5em}}

\newcommand{\Subsection}[1]{} %ignore slide sections
\newcommand{\SlideOnly}[1]{} %ignore slide font size

\usepackage[framemethod=tikz]{mdframed}


\newif\ifkompendium  % to allow conditional text in slides only showing up in compendium
\kompendiumtrue      % in slides: \kompendiumfalse

\newif\ifPreSolution  % to allow tasks and solutions in same file
\PreSolutiontrue      % in solutions: \PreSolutionfalse

\let\QUESTBEGIN\ifPreSolution  % to mark formatting and numbering of exercises
\let\SOLUTION\else  % to mark solutions in the same file as questions
\let\QUESTEND\fi    % to mark end of exercise

%!TEX encoding = UTF-8 Unicode

\newcommand{\ModWeekONE}{Introduktion}
\newcommand{\ExeWeekONE}{expressions}
\newcommand{\LabWeekONE}{kojo}


\newcommand{\ModWeekTWO}{Program och kontrollstrukturer}
\newcommand{\ExeWeekTWO}{programs}
\newcommand{\LabWeekTWO}{--}


\newcommand{\ModWeekTHREE}{Funktioner och abstraktion}
\newcommand{\ExeWeekTHREE}{functions}
\newcommand{\LabWeekTHREE}{irritext}


\newcommand{\ModWeekFOUR}{Objekt och inkapsling}
\newcommand{\ExeWeekFOUR}{objects}
\newcommand{\LabWeekFOUR}{blockmole}


\newcommand{\ModWeekFIVE}{Klasser och datamodellering}
\newcommand{\ExeWeekFIVE}{classes}
\newcommand{\LabWeekFIVE}{blockbattle0}


\newcommand{\ModWeekSIX}{Mönster och felhantering}
\newcommand{\ExeWeekSIX}{patterns}
\newcommand{\LabWeekSIX}{blockbattle1}


\newcommand{\ModWeekSEVEN}{Sekvenser och enumerationer}
\newcommand{\ExeWeekSEVEN}{sequences}
\newcommand{\LabWeekSEVEN}{shuffle}


\newcommand{\ModWeekEIGHT}{Nästlade och generiska strukturer}
\newcommand{\ExeWeekEIGHT}{matrices}
\newcommand{\LabWeekEIGHT}{life}


\newcommand{\ModWeekNINE}{Mängder och tabeller}
\newcommand{\ExeWeekNINE}{lookup}
\newcommand{\LabWeekNINE}{words}


\newcommand{\ModWeekTEN}{Arv och komposition}
\newcommand{\ExeWeekTEN}{inheritance}
\newcommand{\LabWeekTEN}{snake0}


\newcommand{\ModWeekELEVEN}{Varians och kontextparametrar}
\newcommand{\ExeWeekELEVEN}{context}
\newcommand{\LabWeekELEVEN}{snake1}


\newcommand{\ModWeekTWELVE}{Valfri fördjupning, Projekt}
\newcommand{\ExeWeekTWELVE}{extra}
\newcommand{\LabWeekTWELVE}{Projekt0}


\newcommand{\ModWeekTHIRTEEN}{Repetition}
\newcommand{\ExeWeekTHIRTEEN}{examprep}
\newcommand{\LabWeekTHIRTEEN}{Projekt1}


\newcommand{\ModWeekFOURTEEN}{MUNTLIGT PROV}
\newcommand{\ExeWeekFOURTEEN}{Munta}
\newcommand{\LabWeekFOURTEEN}{Munta}


\begin{document}

\pagenumbering{roman}

\frontmatter
\maketitle
%!TEX encoding = UTF-8 Unicode
%!TEX root = ../compendium.tex

\clearpage\null\thispagestyle{empty}
\vfill

{
\setlength{\parindent}{0pt}
\emph{Editor}: Björn Regnell \\

%  LIST OF CONTRIBUTORS to https://github.com/lunduniversity/introprog
%    Please contact bjorn.regnell@cs.lth.se if you think you should be
%    on this list, or make a pull request with an update of file briefly
%    describing your contribution in the commit text.
%    This work is licenced under CC-BY-SA-4.0.
%!TEX encoding = UTF-8 Unicode
%!TEX root = compendium/compendium.tex
\hyphenation{Borg-lund Da-ne-bjer Grampp Palm-qvist Ravn-borg Ro-sen-qvist Schrei-ter Wih-lan-der}
\emph{Contributors} in alphabetical order:
Anders Buhl,
André Philipsson Eriksson,
Anna Axelsson,
Anna Palmqvist Sjövall,
Annie Predel,
Anton Andersson,
Benjamin Lindberg,
Björn Regnell,
Casper Schreiter,
Cecilia Lindskog,
Dag Hemberg,
Elias Åradsson,
Elliot Bräck,
Elsa Cervetti Ogestad,
Emelie Engström,
Emil Wihlander,
Erik Bjäreholt,
Erik Grampp,
Evelyn Beck,
Felix Ohrgren,
Fredrik Danebjer,
Fritjof Bengtsson,
Gustav Cedersjö,
Henrik Olsson,
Hjalmar Rutberg,
Hussein Taher,
Jakob Hök,
Jakob Sinclair,
Johan Ravnborg,
Jonas Danebjer,
Jos Rosenqvist,
Maj Stenmark,
Maria Kulesh,
Måns Magnusson,
Nicholas Boyd Isacsson,
Niklas Sandén,
Oliver Levay,
Oliver Persson,
Oscar Sigurdsson,
Oskar Berg,
Oskar Widmark,
Patrik Persson,
Per Holm,
Philip Sadrian,
Sandra Nilsson,
Sebastian Hegardt,
Simon Persson,
Stefan Jonsson,
Theodor Lundqvist,
Tim Borglund,
Tom Postema,
Valthor Halldorsson,
Viktor Claesson,
Wilhelm Wanecek,
William Karlsson.
\\ \newline

\emph{Home}: \url{https://cs.lth.se/pgk} \newline

\emph{Repo}: \url{https://github.com/lunduniversity/introprog} \\ \newline

This compendium is on-going work. \\ \textbf{Contributions are welcome!} \\
\emph{Contact}: \url{bjorn.regnell@cs.lth.se}
\\ \newline

%\emph{Cover art}: Björn Regnell (inspired by Poul Ströyer's illustration of Lennart Hellsing's lyrics to  the childrens song ''Herr Gurka'' with music by Knut Brodin)\\ \newline

\emph{Versions:} \\
Scala \ScalaVersion \\
JDK \JDKVersion\\
\href{https://github.com/lunduniversity/introprog-scalalib/}{introprog-scalalib} \LibVersion \\ 

\vfill

You can use this work if you respect this \textbf{LICENSE}: CC BY-SA 4.0 \\
\url{http://creativecommons.org/licenses/by-sa/4.0/} \\
\textbf{Do \emph{not} distribute solutions to lab assignments and projects.}
\\ \newline
Copyright \copyright~ 2015-\CurrentYear. \\
Björn Regnell, Dept. of Computer Science, LTH, Lund University.\\
}

%!TEX encoding = UTF-8 Unicode
%!TEX root = ../compendium.tex

\ChapterUnnum{Framstegsprotokoll}\label{progress-protocoll}


\section*{Genomförda övningar}

\vspace{1em}\noindent
{Till varje laboration hör en övning med uppgifter som utgör förberedelse inför labben. Du behöver minst behärska grunduppgifterna för att klara labben inom rimlig tid. Om du känner att du behöver öva mer på grunderna, gör då även extrauppgifterna. Om du vill fördjupa dig, gör fördjupningsuppgifterna som är på mer avancerad nivå. Kryssa för nedan vilka övningar du har gjort, så blir det lättare för din handledare att anpassa dialogen till de kunskaper du förvärvat hittills.}

%% COMPATIBILITY PROBLEM In latex --version 2022 \bottomrule \addlinespace \midrule \toprule etc gives error ! Misplaced \noalign.
%% Here they are replaced by \hline and \\[1.2em]  etc
%% The commands from the booktab chapter are thus cancelled here; can the booktab package be removed?

\newcommand{\TickBox}{\raisebox{-.50ex}{\Large$\square$}}
\newcommand{\ExeRow}[1]{\hyperref[section:exe:#1]{\texttt{#1}} & \TickBox  &  \TickBox &  \TickBox  \\ } %\addlinespace }

\begin{table}[h]
%\centering
\vspace{2em}
\begin{tabular}{lccc}
\hline \\ %\toprule 
%\addlinespace
{\sffamily Övning} &
{\sffamily Grund} &
{\sffamily Extra} &
{\sffamily Fördjupning} \\[1em] \hline  %\addlinespace \midrule \\[-0.7em]  
\\

\input{../compendium/generated/exercises-generated.tex}
\\\hline%\bottomrule 
\end{tabular}
\end{table}

\newpage

\section*{Godkända obligatoriska moment}

\vspace{1em}\noindent
För att bli godkänd på laborationsuppgifterna och projektuppgiften måste du lösa deluppgifterna och diskutera dina lösningar med en handledare. Denna diskussion är din möjlighet att få feedback på dina lösningar. Ta vara på den!
Se till att handledaren noterar nedan när du har blivit godkänd på respektive obligatoriska moment. Spara detta blad tills du fått slutbetyg i kursen.


\vspace{2.2em}\noindent Namn: \dotfill\\

\vspace{1em}\noindent Namnteckning: \dotfill\\

\newcommand{\LabRow}[1]{\\[-1.1em] \hyperref[section:lab:#1]{\texttt{#1}} & \dotfill &  \dotfill  \\[0.7em]}  %\addlinespace }

\begin{table}[h]
%\centering
\vspace{1em}
\begin{tabular}{lcc}
\hline%\toprule\addlinespace
\\
{\sffamily\bfseries\small Lab} & {\sffamily\small kompl+datum,gk+datum } &	
{\sffamily\small Handl. underskr. + namnförtydl.}\\ %\addlinespace 
%\midrule 
\\[-0.5em]
%!TEX encoding = UTF-8 Unicode
%!TEX root = ../compendium.tex
\LabRow{kojo}
\LabRow{irritext}
\LabRow{blockmole}
\LabRow{blockbattle0}
\LabRow{blockbattle1}
\LabRow{shuffle}
\LabRow{life}
\LabRow{words}
\LabRow{snake0}
\LabRow{snake1}
%\toprule
%\addlinespace
\\ 
%\midrule 
%\addlinespace\addlinespace
{\sffamily\small {\bfseries Projektuppgift}} & \dotfill &  \dotfill  \\
%\addlinespace\addlinespace %\midrule
\\
{\Large$\square$}\texttt{~~~\hyperref[section:proj:bank]{bank}} &
\multicolumn{2}{c}{\textit{Om egendef., ge kort beskrivning här:}}  \\[0.6em] %\addlinespace
%{\Large$\square$}\texttt{~~~\hyperref[section:proj:tabular]{tabular}} \\[0.6em] %\addlinespace
{\Large$\square$}\texttt{~~~\hyperref[section:proj:music]{music}} \\[0.6em] %\addlinespace
{\Large$\square$}\texttt{~~~\hyperref[section:proj:photo]{photo}}  \\[0.6em] %\addlinespace
{\Large$\square$}\texttt{~~~}\textit{egendefinerad}  \\
%\dotfill  \\
%\addlinespace\addlinespace
\\
%\midrule
% \addlinespace
\\
{\sffamily\small {\bfseries Muntligt prov}} &  & \\
%\addlinespace\addlinespace{}
\\
{\Large$\square$}\texttt{~~~} godkänd & \dotfill &  \dotfill \\
%\addlinespace\addlinespace
\\\hline%\bottomrule
\end{tabular}
\end{table}

%!TEX encoding = UTF-8 Unicode
%!TEX root = ../compendium2.tex


\ChapterUnnum{Förord}

Detta kompendium innehåller övningar och laborationer och övningslösningar för andra läsperioden i LTH:s grundkurs i programmering för civilingenjörsprogrammet Datateknik.

Det är viktigt att du använder dina lärdomar från första läsperioden om vad du behöver träna mer på och direkt gör upp en plan för hur du kan befästa din förståelse för begreppen i första läsperioden, så att du hänger med under kommande läsperiod.

Det övergripande målet för den andra läsperioden är att du ska kunna skapa egna program som löser mer omfattande problem än tidigare, genom att kombinera flera abstraktionsmekanismer och begrepp från läsperiod 1. Vi inför även nya abstraktionsmekanismer (t.ex. arv) och nya språkkonstruktioner (t.ex. mönstermatching). Läsperiod 2 avslutas med ett obligatoriskt, individuellt projektarbete. Du ska i slutet av kursen även genomföra ett muntligt prov som kontrollerar att du har de kunskaper som krävs för efterföljande kurs. Du erbjuds även en valfri tentamen som ger möjlighet till ett högre betyg.

Kompendiet distribueras som öppen källkod. Det får användas fritt så länge erkännande ges och eventuella ändringar publiceras under samma licens som ursprungsmaterialet. 

I kursens repo \href{http://github.com/lunduniversity/introprog}{github.com/lunduniversity/introprog} finns instruktioner om hur du kan bidra till kursmaterialet.

Välkommen till andra halvlek!

\vspace{1em}\noindent \textit{\hfill Lund, \today, Björn Regnell}


\setcounter{tocdepth}{2} % set headings level in table of contents
\tableofcontents
\mainmatter

\pagenumbering{arabic}


\part{Modulöversikt}

\begin{table}
\noindent\resizebox{1.0\columnwidth}{!}{
\renewcommand{\arraystretch}{2.0}
%!TEX encoding = UTF-8 Unicode
\begin{tabular}{l|l|l|l}
\textit{W} & \textit{Modul} & \textit{Övn} & \textit{Lab} \\ \hline \hline
W01 & Introduktion & expressions & kojo \\
W02 & Program och kontrollstrukturer & programs & -- \\
W03 & Funktioner och abstraktion & functions & irritext \\
W04 & Objekt och inkapsling & objects & blockmole \\
W05 & Klasser och datamodellering & classes & blockbattle0 \\
W06 & Mönster och felhantering & patterns & blockbattle1 \\
W07 & Sekvenser och enumerationer & sequences & shuffle \\
TP & -- & -- & -- \\
W08 & Nästlade och generiska strukturer & matrices & life \\
W09 & Mängder och tabeller & lookup & words \\
W10 & Arv och komposition & inheritance & snake0 \\
W11 & Varians och kontextparametrar & context & snake1 \\
W12 & Fördjupning, Projekt & extra & Projekt0 \\
W13 & Repetition & examprep & Projekt1 \\
W14 & MUNTLIGT PROV & Munta & Munta \\
TP & VALFRI TENTAMEN & -- & -- \\
\end{tabular}

}
\end{table}
\clearpage

\hyphenation{intro-duktion sekvens-algoritmer kod-strukturer data-strukturer}
{\fontsize{11}{12}\selectfont
\renewcommand{\arraystretch}{1.75}
\begin{longtable}{@{}p{.05\textwidth} | >{\hspace{0.1em}\raggedright\bfseries\sffamily}p{.15\textwidth}  >{\raggedleft\arraybackslash\hspace{0.0em}%\fontsize{10.5}{12}\selectfont
}p{0.735\textwidth}}
W01 & Introduktion & sekvens, alternativ, repetition, abstraktion, editera, kompilera, exekvera, datorns delar, virtuell maskin, litteral, värde, uttryck, identifierare, variabel, typ, tilldelning, namn, val, var, def, definiera och anropa funktion, funktionshuvud, funktionskropp, procedur, inbyggda grundtyper, println, typen Unit, enhetsvärdet (), stränginterpolatorn s, aritmetik, slumptal, logiska uttryck, de Morgans lagar, if, true, false, while, for, dod: operativsystem \\
W02 & Program och kontrollstrukturer & huvudprogram, program-argument, indata, scala.io.StdIn.readLine, kontrollstruktur, iterera över element i samling, for-uttryck, yield, map, foreach, samling, sekvens, indexering, Array, Vector, intervall, Range, algoritm, implementation, pseudokod, algoritmexempel: SWAP, SUM, MIN-MAX, MIN-INDEX, dod: versionshantering \\
W03 & Funktioner och abstraktion & abstraktion, funktion, parameter, argument, returtyp, default-argument, namngivna argument, parameterlista, funktionshuvud, funktionskropp, applicera funktion på alla element i en samling, uppdelad parameterlista, skapa egen kontrollstruktur, funktionsvärde, funktionstyp, äkta funktion, stegad funktion, apply, anonyma funktioner, lambda, predikat, aktiveringspost, anropsstacken, objektheapen, stack trace, värdeandrop, namnanrop, klammerparentes och kolon vid ensam parameter, rekursion, scala.util.Random, slumptalsfrö, dod: typsättning \\
W04 & Objekt och inkapsling & modul, singelobjekt, punktnotation, tillstånd, medlem, attribut, metod, paket, filstruktur, jar, classpath, dokumentation, JDK, import, selektiv import, namnbyte vid import, export, tupel, multipla returvärden, block, lokal variabel, skuggning, lokal funktion, funktioner är objekt med apply-metod, namnrymd, synlighet, privat medlem, inkapsling, getter och setter, principen om enhetlig access, överlagring av metoder, introprog.PixelWindow, initialisering, lazy val, typalias, dod: maskinkod \\
W05 & Klasser och datamodellering & applikationsdomän, datamodell, objektorientering, klass, instans, Any, isInstanceOf, toString, new, null, this, accessregler, private, private[this], klassparameter, primär konstruktor, fabriksmetod, alternativ konstruktor, förändringsbar, oföränderlig, case-klass, kompanjonsobjekt, referenslikhet, innehållslikhet, eq, == \\
W06 & Mönster och felhantering & mönstermatchning, match, Option, throw, try, catch, Try, unapply, sealed, flatten, flatMap, partiella funktioner, collect, wildcard-mönster, variabelbindning i mönster, sekvens-wildcard, bokstavliga mönster, implementera equals, hashcode \\
W07 & Sekvenser och enumerationer & översikt av Scalas samlingsbibliotek och samlingsmetoder, klasshierarkin i scala.collection, Iterable, Seq, List, ListBuffer, ArrayBuffer, WrappedArray, sekvensalgoritm, algoritm: SEQ-COPY, in-place vs copy, algoritm: SEQ-REVERSE, registrering, algoritm: SEQ-REGISTER, linjärsökning, algoritm: LINEAR-SEARCH, tidskomplexitet, minneskomplexitet, översikt strängmetoder, StringBuilder, ordning, inbyggda sökmetoder, find, indexOf, indexWhere, inbyggda sorteringsmetoder, sorted, sortWith, sortBy, repeterade parametrar \\
TP & \multicolumn{2}{l}{--} \\
W08 & Nästlade och generiska strukturer & matris, nästlad samling, nästlad for-sats, typparameter, generisk funktion, generisk klass, fri och bunden typparameter, generiska datastrukturer, generiska samlingar i Scala \\
W09 & Mängder och tabeller & innehållstest, mängd, Set, mutable.Set, nyckel-värde-tabell, Map, mutable.Map, hash code, java.util.HashMap, java.util.HashSet, persistens, serialisering, textfiler, Source.fromFile, java.nio.file \\
W10 & Arv och komposition & arv, komposition, polymorfism, trait, extends, asInstanceOf, with, inmixning supertyp, subtyp, bastyp, override, Scalas typhierarki, Any, AnyRef, Object, AnyVal, Null, Nothing, topptyp, bottentyp, referenstyper, värdetyper, accessregler vid arv, protected, final, trait, abstrakt klass \\
W11 & Varians och kontextparametrar & övre- och undre typgräns, varians, kontravarians, kovarians, typjoker, kontextgräns, typkonstruktor, egentyp, typjoker, givet värde (given), kontextparameter (using), ad hoc polymorfism, typklass, api, kodläsbarhet, granskningar \\
W12 & Fördjupning, Projekt & välj valfritt fördjupningsområde, påbörja projekt \\
W13 & Repetition & träna på extentor, redovisa projekt, träna inför muntligt prov \\
W14 & \multicolumn{2}{l}{MUNTLIGT PROV} \\
TP & \multicolumn{2}{l}{VALFRI TENTAMEN} \\
\end{longtable}
}

%\renewcommand{\SlideHeading}[1]{\subsection{#1}}  %numbering sections in compendium slides

\part{Moduler}

\setcounter{chapter}{7}

%!TEX encoding = UTF-8 Unicode

%!TEX root = ../compendium2.tex

\input{generated/w08-chaphead-generated.tex}
\clearpage\section{Teori}
%!TEX encoding = UTF-8 Unicode
%!TEX root = ../lect-w08.tex

%%%

\Subsection{Veckans labb: \texttt{life}}

\begin{Slide}{Veckans labb: \texttt{life}}
\begin{minipage}{0.52\textwidth}
  \setlength{\leftmargini}{0pt}

\begin{itemize}
  \SlideFontSmall
\item Universum är en binär matris av \Emph{celler} där \Emph{levande} celler representeras med \code{true} och \Alert{döda} med \code{false}.
\item Följande regler gäller för \Emph{nästa generation} celler i universum:
\begin{itemize}\SlideFontTiny
  \item \textbf{Fortlevnad}: en levande cell med 2 eller 3 grannar \Emph{lever vidare}
  \item \textbf{Död}: en levande cell med färre än 2 eller fler än 3 grannar \Alert{dör}
  \item \textbf{Födelse}: en död cell med exakt tre grannar föds
\end{itemize}
\item Övning \code{matrices} uppgift 5: skapa en generisk \code{case class Matrix[T]}
\item På labben: använd \code{Matrix[Boolean]}
\end{itemize}

\end{minipage}%
\begin{minipage}{0.5\textwidth}
  \includegraphics[width=1.0\textwidth]{../img/glider-blinker-block}

  \begin{itemize}\SlideFontTiny
  \item Du ska simulera \emph{Game of Life} i ett \code{introprog.PixelWindow}
  \item Fördjupning:\\{\SlideFontTiny\url{https://en.wikipedia.org/wiki/Conway%27s_Game_of_Life}}
  \end{itemize}
\end{minipage}%

\end{Slide}






\Subsection{Matriser}

\begin{Slide}{Vad är en matris?}\SlideFontSmall
\begin{itemize}

\item En \Emph{matris} inom \Alert{matematiken} innehåller \Emph{rader} och \Emph{kolumner}\footnote{även kallade \emph{kolonner}} med tal.

\item I en \Alert{matematisk} matris har alla rader \Emph{lika många} element och

\item även alla kolumner har \Emph{lika många} element.

\item En matris av dimension $2\times{}5$ har $2 \cdot 5 = 10$ stycken element.

\item Exempel på en matematisk matris av dimension $2\times{}5$:
\[
M_{2,5}=
  \begin{pmatrix}
    5 & 2 & 42 & 4 & 5 \\
    3 & 4 & 18 & 6 & 7
  \end{pmatrix}
\]
\end{itemize}
\end{Slide}

\begin{Slide}{Indexering i en matris}\SlideFontSmall
\begin{itemize}

  \item En matris av dimension $m\times{}n$ har $m \cdot n$ stycken element.

  \item En matris $A_{m,n}$ av dimension $m\times{}n$ ritas inom matematiken ofta så här:

  \[
  A_{m,n} =
   \begin{pmatrix}
    a_{1,1} & a_{1,2} & \cdots & a_{1,n} \\
    a_{2,1} & a_{2,2} & \cdots & a_{2,n} \\
    \vdots  & \vdots  & \ddots & \vdots  \\
    a_{m,1} & a_{m,2} & \cdots & a_{m,n}
   \end{pmatrix}
  \]


\item Matrisindexering inom matematiken sker ofta från $1$, men ofta från $0$ i datorprogram.

\item Vad har talet $42$ för index i matrisen $M_{2,5}$ nedan?
\begin{itemize}\SlideFontTiny
  \item[--] Inom matematiken?
  \item[--] I Scala och Java och många andra språk?

  \[
  M_{2,5}=
    \begin{pmatrix}
      5 & 2 & 42 & 4 & 5 \\
      3 & 4 & 18 & 6 & 7
    \end{pmatrix}
  \]
\end{itemize}
\end{itemize}
\end{Slide}

\begin{Slide}{Hur skapa matriser?}
  \setlength{\leftmargini}{0pt}

  \begin{itemize}
  \item Inom programmering används ordet \Emph{matris} ofta för att beteckna en \Alert{nästlad struktur} i två dimensioner. Exempel:
  \begin{itemize}
   \item \Emph{Oföränderliga} sekvenser, t.ex. \code{Vector[Vector[Int]]} \\
   \code{val xss = Vector(Vector(0, 0, 0), Vector(0, 0, 0))} eller enklare: \\
      \code{val xss = Vector.fill(2,3)(0)}

    \item \Alert{Föränderliga} sekvens, t.ex. \code{Array[Array[Int]]} \\
    \code{val yss = Array(Array(0, 0, 0), Array(0, 0, 0))} eller enklare: \\
       \code{val yss = Array.fill(2,3)(0)}

  \end{itemize}

\end{itemize}
\end{Slide}

\begin{Slide}{Hur indexera i matriser?}
En matris med array av arrayer:
\begin{REPL}
scala> val xss = Array(Array(5,2,42,4,5),Array(3,4,18,6,7))
xss: Array[Array[Int]] = Array(Array(5, 2, 42, 4, 5), Array(3, 4, 18, 6, 7))
\end{REPL}
\pause
Man indexerar i en nästlad sekvens med upprepad \code{apply}:
\begin{REPL}
scala> xss(0)(2)
res0: ???

scala> xss.apply(0).apply(2)
res1: ???

scala> xss(0)
res2: ???
\end{REPL}
Övning: Vad är typ och värde vid \code{???} ovan?
\end{Slide}

\begin{Slide}{Hur indexera i matriser?}
En matris med array av arrayer:
\begin{REPL}
scala> val xss = Array(Array(5,2,42,4,5),Array(3,4,18,6,7))
xss: Array[Array[Int]] = Array(Array(5, 2, 42, 4, 5), Array(3, 4, 18, 6, 7))
\end{REPL}

Man indexerar i en nästlad sekvens med upprepad \code{apply}:
\begin{REPL}
scala> xss(0)(2)
res0: Int = 42

scala> xss.apply(0).apply(2)
res1: Int = 42

scala> xss(0)
res2: Array[Int] = Array(5, 2, 42, 4, 5)
\end{REPL}
Övning: Rita en bild av minnet som referensen \code{xss} refererar till.

\end{Slide}

\begin{Slide}{Uppdatering av en förändringsbar nästlad struktur}
Man kan förändra en array av arrayer ''på plats'' med tilldelning:
\begin{REPL}
scala> val xss = Array(Array(5,2,42,4,5),Array(3,4,18,6,7))

scala> xss(0)(0) = 100

scala> xss
res0: ???

scala> xss(0)(2) = xss(0)(2) - 1

scala> xss
res1: ???

scala> xss(1) = Array.fill(5)(-1)

scala> xss
res2: ???
\end{REPL}
\end{Slide}

\begin{Slide}{Uppdatering av en förändringsbar nästlad struktur}
Man kan förändra en array av arrayer ''på plats'' med tilldelning:
\begin{REPL}
scala> val xss = Array(Array(5,2,42,4,5),Array(3,4,18,6,7))

scala> xss(0)(0) = 100

scala> xss
res0: Array[Array[Int]]=Array(Array(100, 2, 42, 4, 5), Array(3, 4, 18, 6, 7))

scala> xss(0)(2) = xss(0)(2) - 1

scala> xss
res1: Array[Array[Int]]=Array(Array(100, 2, 41, 4, 5), Array(3, 4, 18, 6, 7))

scala> xss(1) = Array.fill(5)(-1)

scala> xss
res2: Array[Array[Int]]=Array(Array(100, 2, 41, 4, 5), Array(-1,-1,-1,-1,-1))
\end{REPL}
\end{Slide}

\begin{Slide}{Några olika sätt att skapa förändringsbara matriser}\SlideFontSmall
Det jobbiga, primitiva sättet:
\begin{REPL}
scala> val xss = new Array[Array[Int]](2)
xss: Array[Array[Int]] = Array(null, null)

scala> for (i <- xss.indices) {xss(i) = new Array[Int](5)}

scala> xss
res0: Array[Array[Int]] = Array(Array(0, 0, 0, 0, 0), Array(0, 0, 0, 0, 0))

scala> println(xss)
[[I@196a99d0
\end{REPL}
Enklare sätt:
\begin{REPL}
scala> val xss = Array.ofDim[Int](2,5)
xss: Array[Array[Int]] = Array(Array(0, 0, 0, 0, 0), Array(0, 0, 0, 0, 0))
\end{REPL}
Enklare och tydligare sätt, där initialvärdet anges explicit:
\begin{REPL}
scala> val xss = Array.fill(2,5)(0)
xss: Array[Array[Int]] = Array(Array(0, 0, 0, 0, 0), Array(0, 0, 0, 0, 0))
\end{REPL}

\end{Slide}

\begin{Slide}{Exempel på skapande av oföränderlig nästlad struktur}\SlideFontSmall
Om du kan beräkna initialvärde direkt, använd \code{Vector.fill}:\\
{\SlideFontTiny\code{def fill[A](n1: Int, n2: Int)(elem: => A): Vector[Vector[A]]}}
\begin{REPL}
scala> Vector.fill(2,5)(scala.util.Random.nextInt(6) + 1)
res0:
  typ???
  värde???

\end{REPL}
Om du kan beräkna initialvärde ur index, använd \code{Vector.tabulate}:\\
{\SlideFontTiny\code{def tabulate[A](n1: Int, n2: Int)(f: (Int, Int) => A): Vector[Vector[A]]}}
\begin{REPL}
scala> Vector.tabulate(5,2)((x,y) => x + y + 1)
res1:
  typ???
  värde???

\end{REPL}
\end{Slide}

\begin{Slide}{Exempel på skapande av oföränderlig nästlad struktur}\SlideFontSmall
Om du kan beräkna initialvärde direkt, använd \code{Vector.fill}:\\
{\SlideFontTiny\code{def fill[A](n1: Int, n2: Int)(elem: => A): Vector[Vector[A]]}}
\begin{REPL}
scala> Vector.fill(2,5)(scala.util.Random.nextInt(6) + 1)
res0: Vector[Vector[Int]] =
  Vector(Vector(1, 2, 6, 2, 1), Vector(1, 4, 3, 3, 2))

\end{REPL}
Om du kan beräkna initialvärde ur index, använd \code{Vector.tabulate}:\\
{\SlideFontTiny\code{def tabulate[A](n1: Int, n2: Int)(f: (Int, Int) => A): Vector[Vector[A]]}}
\begin{REPL}
scala> Vector.tabulate(5,2)((x,y) => x + y + 1)
res1: Vector[Vector[Int]] =
  Vector(Vector(1,2), Vector(2,3), Vector(3,4), Vector(4,5), Vector(5,	6))

\end{REPL}
\end{Slide}



\begin{Slide}{Uppdatering av en oföränderlig nästlad struktur}\SlideFontSmall
Uppdatering av endimensionell struktur med \code{xs.updated}:\\
{\SlideFontTiny\code{def updated[A](index: Int, elem: A): Vector[A]} }
\begin{REPL}
scala> var xs = Vector.tabulate(5)(x => x + 1)
xs: typ??? = värde???

scala> xs = xs.updated(1, 42)
xs: typ??? = värde???
\end{REPL}

Uppdatering av nästlad struktur i två dimensioner:
\begin{REPL}
scala> var xss = Vector.tabulate(2, 5)((x,y) => x + y + 1)
xss:
  typ??? =
  värde???

scala> xss = xss.updated(0, xss(0).updated(1, 42))
xss:
  typ??? =
  värde???
\end{REPL}

\end{Slide}



\begin{Slide}{Uppdatering av en oföränderlig nästlad struktur}\SlideFontSmall
Uppdatering av endimensionell struktur med \code{xs.updated}:\\
{\SlideFontTiny\code{def updated[A](index: Int, elem: A): Vector[A]} }
\begin{REPL}
scala> var xs = Vector.tabulate(5)(x => x + 1)
xs: Vector[Int] = Vector(1, 2, 3, 4, 5)

scala> xs = xs.updated(1, 42)
xs: Vector[Int] = Vector(1, 42, 3, 4, 5)
\end{REPL}

Uppdatering av nästlad struktur i två dimensioner:
\begin{REPL}
scala> var xss = Vector.tabulate(2, 5)((x,y) => x + y + 1)
xss: Vector[Vector[Int]] =
  Vector(Vector(1, 2, 3, 4, 5), Vector(2, 3, 4, 5, 6))

scala> xss = xss.updated(0, xss(0).updated(1, 42))
xss:
  Vector[Vector[Int]] =
  Vector(Vector(1, 42, 3, 4, 5), Vector(2, 3, 4, 5, 6))
\end{REPL}

\end{Slide}


\begin{Slide}{Iterera över nästlad struktur}\SlideFontSmall
Behandling av nästlade strukturer kräver ofta algoritmer med nästlad iterering. \\
Exempel: iterera med nästlad \code{for}-sats för utskrift av denna matris\\
\code{val xss = Vector.tabulate(2,5)((x,y) => x + y + 1)}
\pause
\begin{REPL}
scala> for ??? do
         for ??? do 
           print(xss(i)(j))
           print(" ")
         println

1 2 3 4 5
2 3 4 5 6
\end{REPL}
Övning: \\Vad ska det stå vid \code{???} för att alla element ska skrivas ut?
\end{Slide}

\begin{Slide}{Iterera över nästlad struktur}\SlideFontSmall
  \vspace{1em}
  Behandling av nästlade strukturer kräver ofta algoritmer med nästlad iterering. \\
  Exempel: iterera med nästlad \code{for}-sats för utskrift av denna matris \\
  \code{val xss = Vector.tabulate(2,5)((x,y) => x + y + 1)}

  \begin{REPL}
scala> for xs <- xss do
         for x <- xs do 
           print(x)
           print(" ")
         end for
         println()
       end for

1 2 3 4 5
2 3 4 5 6
\end{REPL}
Övning: skriv ut matrisen med nästlad \code{foreach}\\
\pause
\begin{Code}
xss.foreach { xs => 
  xs.foreach { x => print(x); print(" ") }
  println()
}
\end{Code}
\end{Slide}


\begin{Slide}{Övningsexempel: Yatzy}\SlideFontSmall
Skapa en funktion \code{roll} som ger utfallet av n st tärningskast:
\begin{REPL}
scala> import scala.util.Random

scala> def roll(n: Int): Vector[Int] = ???
\end{REPL}

Skapa en funktion \code{isYatzy} som ger \code{true} om alla utfall är lika:
\begin{REPL}
scala> def isYatzy(xs: Vector[Int]): Boolean = ???
\end{REPL}
Du kan anta att xs.length > 0\\
Tips: använd metoden xs.forall: \\
\code{def forall[A](p: A => Boolean): Boolean }
\end{Slide}


\begin{Slide}{Övningsexempel: Yatzy}\SlideFontSmall
Skapa en funktion \code{roll} som ger utfallet av n st tärningskast:
\begin{REPL}
scala> import scala.util.Random

scala> def roll(n: Int): Vector[Int] = Vector.fill(n)(Random.nextInt(6) + 1)
\end{REPL}

Skapa en funktion \code{isYatzy} som ger \code{true} om alla utfall är lika:
\begin{REPL}
scala> def isYatzy(xs: Vector[Int]): Boolean = xs.forall(x => x == xs(0))
\end{REPL}
Du kan anta att xs.length > 0\\
Tips: använd metoden xs.forall: \\
\code{def forall[A](p: A => Boolean): Boolean }
\end{Slide}

\begin{Slide}{Iterera över nästlad struktur: for-sats}\SlideFontSmall
Iterera med nästlad for-sats: (vad har xss för typ?)
\begin{REPL}
scala> val xss = Vector.fill(100)(roll(5))

scala> for i <- ??? do 
         for j <- ??? do
           print(s"($i)($j): ${xss(i)(j)} ")
         println(s" YATZY: ${isYatzy(xss(i))}")

(0)(0): 3 (0)(1): 6 (0)(2): 4 (0)(3): 4 (0)(4): 6  YATZY: false
(1)(0): 4 (1)(1): 1 (1)(2): 5 (1)(3): 2 (1)(4): 6  YATZY: false
(2)(0): 1 (2)(1): 3 (2)(2): 5 (2)(3): 6 (2)(4): 2  YATZY: false
(3)(0): 2 (3)(1): 1 (3)(2): 1 (3)(3): 5 (3)(4): 4  YATZY: false
(4)(0): 4 (4)(1): 4 (4)(2): 1 (4)(3): 6 (4)(4): 5  YATZY: false
(5)(0): 3 (5)(1): 3 (5)(2): 2 (5)(3): 3 (5)(4): 6  YATZY: false
(6)(0): 3 (6)(1): 6 (6)(2): 1 (6)(3): 1 (6)(4): 4  YATZY: false
(7)(0): 6 (7)(1): 2 (7)(2): 4 (7)(3): 4 (7)(4): 3  YATZY: false
(8)(0): 1 (8)(1): 5 (8)(2): 4 (8)(3): 2 (8)(4): 4  YATZY: false
(9)(0): 1 (9)(1): 1 (9)(2): 3 (9)(3): 6 (9)(4): 6  YATZY: false
(10)(0): 2 (10)(1): 5 (10)(2): 2 (10)(3): 4 (10)(4): 5  YATZY: false
(11)(0): 3 (11)(1): 4 (11)(2): 2 (11)(3): 5 (11)(4): 6  YATZY: false
...
\end{REPL}
\end{Slide}

\begin{Slide}{Iterera över nästlad struktur: for-sats}\SlideFontSmall
Iterera med nästlad for-sats: (xss är en \code{Vector[Vector[Int]]})
\begin{REPL}
scala> val xss = Vector.fill(100)(roll(5))

scala> for i <- xss.indices do 
         for j <- xss(i).indices do
           print(s"($i)($j): ${xss(i)(j)} ")
         println(s" YATZY: ${isYatzy(xss(i))}")

(0)(0): 3 (0)(1): 6 (0)(2): 4 (0)(3): 4 (0)(4): 6  YATZY: false
(1)(0): 4 (1)(1): 1 (1)(2): 5 (1)(3): 2 (1)(4): 6  YATZY: false
(2)(0): 1 (2)(1): 3 (2)(2): 5 (2)(3): 6 (2)(4): 2  YATZY: false
(3)(0): 2 (3)(1): 1 (3)(2): 1 (3)(3): 5 (3)(4): 4  YATZY: false
(4)(0): 4 (4)(1): 4 (4)(2): 1 (4)(3): 6 (4)(4): 5  YATZY: false
(5)(0): 3 (5)(1): 3 (5)(2): 2 (5)(3): 3 (5)(4): 6  YATZY: false
(6)(0): 3 (6)(1): 6 (6)(2): 1 (6)(3): 1 (6)(4): 4  YATZY: false
(7)(0): 6 (7)(1): 2 (7)(2): 4 (7)(3): 4 (7)(4): 3  YATZY: false
(8)(0): 1 (8)(1): 5 (8)(2): 4 (8)(3): 2 (8)(4): 4  YATZY: false
(9)(0): 1 (9)(1): 1 (9)(2): 3 (9)(3): 6 (9)(4): 6  YATZY: false
(10)(0): 2 (10)(1): 5 (10)(2): 2 (10)(3): 4 (10)(4): 5  YATZY: false
(11)(0): 3 (11)(1): 4 (11)(2): 2 (11)(3): 5 (11)(4): 6  YATZY: false
...
\end{REPL}
\end{Slide}


% \begin{Slide}{Iterera över nästlad struktur med nästlad foreach}\SlideFontSmall
% Iterera med nästlad foreach-sats:
% \begin{REPL}
% scala> val xss = Vector.tabulate(2,5)((x,y) => x + y + 1)

% xss.foreach{ xs => ??? ; println }

% 1 2 3 4 5
% 2 3 4 5 6
% \end{REPL}
% \end{Slide}


% \begin{Slide}{Iterera över nästlad struktur med nästlad foreach}\SlideFontSmall
% Iterera med nästlad foreach-sats:
% \begin{REPL}
% scala> val xss = Vector.tabulate(2,5)((x,y) => x + y + 1)

% xss.foreach{ xs => xs.foreach{ x => print(x + " ") }; println }

% 1 2 3 4 5
% 2 3 4 5 6
% \end{REPL}
% \end{Slide}


\begin{Slide}{Nästlade for-uttryck}\SlideFontSmall
Iterera med \Emph{nästlad for-yield}:\\
%Statisk typ: \code{IndexedSeq[IndexedSeq[[Int]]} \\
%Dynamisk typ: \code{Vector[Vector[[Int]]}

\begin{REPL}
scala> val xss = for i <- 1 to 2 yield 
                   for j <- 1 to 5 yield i + j + 1
                 
val xss: IndexedSeq[IndexedSeq[Int]] =
      ???

\end{REPL}
\pause Om man skriver så här får man en endimensionell struktur:
\begin{REPL}
scala> val xs = for { i <- 1 to 2; j <- 1 to 5 } yield i + j + 1
val xs: IndexedSeq[Int] =
    ???

\end{REPL}
\end{Slide}

\begin{Slide}{Nästlade for-uttryck}\SlideFontSmall
Iterera med \Emph{nästlad for-yield}:\\
\begin{REPL}
scala> val xss = for i <- 1 to 2 yield 
                   for j <- 1 to 5 yield i + j + 1

val xss: IndexedSeq[IndexedSeq[Int]] =
    Vector(Vector(3, 4, 5, 6, 7), Vector(4, 5, 6, 7, 8))

\end{REPL}
\pause Om man skriver så här får man en endimensionell struktur:
\begin{REPL}
scala> val xs = for { i <- 1 to 2; j <- 1 to 5 } yield i + j + 1
val xs: IndexedSeq[Int] =
    Vector(3, 4, 5, 6, 7, 4, 5, 6, 7, 8)

\end{REPL}
\end{Slide}



\begin{Slide}{Nästlade map-uttryck}\SlideFontSmall
Iterera med \Emph{nästlade map-uttryck}:\\
\begin{REPL}
scala> val xss = (1 to 2).map(i => (1 to 5).map(j => i + j + 1))
xss: IndexedSeq[IndexedSeq[Int]] =
      ???
\end{REPL}
\end{Slide}

\begin{Slide}{Nästlade map-uttryck}\SlideFontSmall
Iterera med \Emph{nästlade map-uttryck}:\\
\begin{REPL}
scala> val xss = (1 to 2).map(i => (1 to 5).map(j => i + j + 1))
xss: IndexedSeq[IndexedSeq[Int]] =
      Vector(Vector(3, 4, 5, 6, 7), Vector(4, 5, 6, 7, 8))
\end{REPL}
\end{Slide}



\ifkompendium\else
\begin{Slide}{Fallgrop: likhet av array}
\begin{REPL}
scala> Vector.fill(5, 2)(42) == Vector.fill(5, 2)(42)
val res0: ???

scala> Array.fill(5, 2)(42) == Array.fill(5, 2)(42)
val res1: ???
\end{REPL}
\end{Slide}
\fi

\begin{Slide}{Fallgrop: likhet av array}
\begin{REPL}
scala> Vector.fill(5, 2)(42) == Vector.fill(5, 2)(42)
val res0: Boolean = true

scala> Array.fill(5, 2)(42) == Array.fill(5, 2)(42)
val res1: Boolean = false  // AAAARRGH!!! :(
\end{REPL}
Primitiva arrayer har en equals-metod som ger referenslikhet, \Alert{inte} innehållslikhet. Och det fungerar följaktligen ej heller på nästlade strukturer. 
\end{Slide}

\ifkompendium\else
\begin{Slide}{Kolla likhet mellan två heltalsmatriser (uppfinner hjulet)}
\begin{CodeSmall}
def isEqual(xss: Array[Array[Int]], yss: Array[Array[Int]]) = 
  if xss.length != yss.length then false else
    var i = 0
    var foundUnequal = false
    while ??? do
      if xss(i).length != yss(i).length then 
        ???
      else 
        var j = 0
        while ??? do
          if xss(i)(j) != yss(i)(j) then ???
          j += 1
        end while
      end if
      i += 1
    end while
    !foundUnequal
  end if
end isEqual
\end{CodeSmall}
\end{Slide}
\fi


\begin{Slide}{Kolla likhet mellan två heltalsmatriser (uppfinner hjulet)}
\begin{CodeSmall}
def isEqual(xss: Array[Array[Int]], yss: Array[Array[Int]]) = 
  if xss.length != yss.length then false else
    var i = 0
    var foundUnequal = false
    while i < xss.length && !foundUnequal do
      if xss(i).length != yss(i).length then 
        foundUnequal = true
      else 
        var j = 0
        while j < xss(i).length && !foundUnequal do
          if xss(i)(j) != yss(i)(j) then foundUnequal = true
          j += 1
        end while
      end if
      i += 1
    end while
    !foundUnequal
  end if
end isEqual
\end{CodeSmall}
\end{Slide}

\begin{Slide}{Använd INTE \texttt{sameElements} på nästlade arrayer}
\SlideFontSmall
I Scala kan du använda metoden \code{sameElements} för att kolla innehållslikhet mellan två arrayer, men det funkar \Alert{INTE} på djupet i nästlade strukturer.

\begin{REPL}
scala> val xs = Array(1,2,3)
xs: Array[Int] = Array(1, 2, 3)

scala> val ys = Array(1,2,3)
ys: Array[Int] = Array(1, 2, 3)

scala> xs.sameElements(ys)                            // xs, ys ej nästlade
res0: Boolean = true                                  // innehåll lika!

scala> Array(Array(1)) sameElements Array(Array(1))  
res1: Boolean = false                                 //AAAARGH!

\end{REPL}
Använd i stället: \\\code{java.util.Objects.deepEquals} eller \code{java.util.Arrays.deepEquals} \\Den senare kräver typkonvertering av argumenten med: \code{asInstanceOf[Array[Object]]}
\end{Slide}

\begin{Slide}{Kontroll av innehållslikhet mellan nästlade arrayer}\SlideFontTiny
\code{java.util.Objects.deepEquals} fungerar \Emph{på djupet} för godtyckliga referenstyper:
\begin{REPLsmall}
scala> java.util.Objects.deepEquals(
          Array(Array("a", Array("b"), 42)),
          Array(Array("a", Array("b"), 42)))
val res0: Boolean = true

scala> java.util.Objects.deepEquals(
          Array(Array("a", Array("b"), 42)),
          Array(Array("a", Array("b"), 43)))
val res1: Boolean = false
\end{REPLsmall}
\code{java.util.Objects.deepEquals} kontrollerar om argumenten är arrayer och anropar då i sin tur \code{java.util.Arrays.deepEquals} efter typkonvertering:
\begin{REPLsmall}
scala> java.util.Arrays.deepEquals(
          Array(Array("a", Array("b"), 42)).asInstanceOf[Array[Object]],
          Array(Array("a", Array("b"), 42)).asInstanceOf[Array[Object]])
val res3: Boolean = true

\end{REPLsmall}
\url{https://stackoverflow.com/questions/63686721/best-replacement-of-deep-method-in-scala-2-13}
\end{Slide}

% \begin{Slide}{Matris som Array med Array med heltal i Java}\SlideFontTiny
% \begin{CodeSmall}[language=Java]
% public class ArrayMatrix {

%     public static void showMatrix(int[][] m){
%         System.out.println("\n--- showMatrix ---");
%         for (int row = 0; row < m.length; row++){
%             for (int col = 0; col < m[row].length; col++) {
%                 System.out.print("[" + row + "]");
%                 System.out.print("[" + col + "] = ");
%                 System.out.print(m[row][col] + "; ");
%             }
%             System.out.println();
%         }
%     }

%     public static void main(String[] args) {
%         int[][] xss = new int[10][5];
%         showMatrix(xss);
%     }
% }
% \end{CodeSmall}
% \pause
% Övning: skriv en metod \code{fillRnd} som fyller en heltalsmatris med slumptal 1 till n:\\
% \pause
% \jcode|public static void fillRnd(int[][] m, int n){ /* ??? */ }| \\
% \pause
% Tips: använd en nästlad for-sats och detta uttryck: \\
% \jcode{(int) (Math.random() * n + 1) // (int) motsvarar Scalas asInstanceOf[Int]}

% \end{Slide}

\begin{Slide}{Om veckans övningar}\SlideFontSmall
\begin{itemize}
\item Träna på att iterera över nästlade strukurer

\item Fortsätt jobba med Yatzy-exemplet

\item träna på att skapa \Emph{imperativa} algoritmer: \\
lös \code{isYatzy} med \code{while}-sats 

\item Extrauppgift där du ska bygga ett enkelt yatzy-spel i terminalen (kunde varit en tentauppgift...)

\end{itemize}
\end{Slide}

% \begin{Slide}{Övning extrauppgift, utgör början på labb \code{survey}}\SlideFontSmall
%
% \begin{ScalaSpec}{Table}
% object Table {
%   /** Creates a new Table from fileName with columns split by sep */
%   def fromFile(fileName: String, separator: Char = ';'): Table = ???
% }
% case class Table(
%   data: Vector[Vector[String]],
%   headings: Vector[String],
%   sep: String){
%   /** A 2-tuple with (number of rows, number of columns) in data */
%   val dim: (Int, Int) = ???
%
%   /** The element in row r an column c of data, counting from 0 */
%   def apply(r: Int, c: Int): String = ???
%
%   /** The row-vector r in data, counting from 0 */
%   def row(r: Int): Vector[String]= ???
%
%   /** The column-vector c in data, counting from 0 */
%   def col(c: Int): Vector[String] = ???
%
%   /** A map from heading to index counting from 0 */
%   lazy val indexOfHeading: Map[String, Int] = ???
%
%   /** The column-vector with heading h in data */
%   def col(h: String): Vector[String] = ???
%
%   /** A vector with the distinct, sorted values of col with heading h */
%   def values(h: String): Vector[String] = ???
%
%   /** Headings and data with columns separated by sep */
%   override lazy val toString: String = ???
% }
% \end{ScalaSpec}
% \end{Slide}


% \begin{Slide}{Övn. fördjupn. uppg.: skapa en generisk matris-klass}\SlideFontSmall
% \vspace{-0.7em}
% \begin{Code}[basicstyle=\SlideFontSize{6}{6.8}\ttfamily\selectfont]
% case class Matrix[T](data: Vector[Vector[T]]){
%
%   def foreachRowCol(f: (Int, Int, T) => Unit): Unit =
%     for (r <- data.indices) {
%       for (c <- data(r).indices) {
%         f(r, c, data(r)(c))
%       }
%     }
%
%   def map[U](f: T => U): Matrix[U] = Matrix(data.map(_.map(f)))
%
%   /** The element at row r and column c */
%   def apply(r: Int, c: Int): T = ???
%
%   /** Gives Some[T](element) at index (r, c) if within index bounds, else None */
%   def get(r: Int, c: Int): Option[T] = ???
%
%   /** The row vector of row r */
%   def row(r: Int): Vector[T] = ???
%
%   /** The column vector of column c */
%   def col(c: Int): Vector[T] = ???
%
%   /** A new Matrix with element at row r and col c updated */
%   def updated(r: Int, c: Int, value: T): Matrix[T] = ???
% }
% object Matrix {
%   def fill[T](rowSize: Int, colSize: Int)(init: T): Matrix[T] =
%     new Matrix(Vector.fill(rowSize)(Vector.fill(colSize)(init)))
% }
% \end{Code}
% \end{Slide}

%!TEX encoding = UTF-8 Unicode
%!TEX root = ../lect-w08.tex

\Subsection{Typparametrar}



\begin{Slide}{Exempel: Icke-generisk case-klass med heltalsmatris}
  En \emph{icke-generisk} datastruktur har inga obundna typparametrar; alla typer är \Emph{konkreta} (alltså specifika). \\~\\ En icke-generisk case-class med en \code{Vector[Vector[Int]]}:
  \begin{Code}
  case class Matrix(data: Vector[Vector[Int]]):
    def apply(x: Int, y: Int): Int = data(x)(y)
  \end{Code}

  \begin{REPL}
  scala> Matrix(Vector(Vector(5, 2, 42, 4, 5),Vector(3, 4, 18, 6, 7)))
  res0: Matrix =
    Matrix(Vector(Vector(5, 2, 42, 4, 5), Vector(3, 4, 18, 6, 7)))
  \end{REPL}

\end{Slide}





\begin{Slide}{Exempel: Generisk case-klass med generell matris}
  En \emph{generisk} datastruktur har minst en \Alert{obunden} \Emph{typparameter} som vid användning ska bindas till ett \Alert{konkret} \Emph{typargument}.
  
  \begin{Code}
  case class Matrix[T](data: Vector[Vector[T]]):
    def apply(x: Int, y: Int): T = data(x)(y)
  \end{Code}
  \code{Matrix} i exemplet ovan är en \Emph{generisk} case-class där \code{T} är obunden, eftersom \code{T} är en typparameter deklarerad inom \code{[]} \Alert{efter} klassens namn men \Alert{före} klassparameterlistan. \\

  \vspace{0.5em} Användning där \code{T} binds till \code{Int} via kompilatorns typhärledning:
  \begin{REPL}
  scala> Matrix(Vector(Vector(5, 2, 42, 4, 5),Vector(3, 4, 18, 6, 7)))
  res1: Matrix[Int] =
    Matrix(Vector(Vector(5, 2, 42, 4, 5), Vector(3, 4, 18, 6, 7)))
  \end{REPL}

\end{Slide}




\begin{Slide}{Vad är en typparameter?}\SlideFontSmall
  \setlength{\leftmargini}{0pt}

\begin{itemize}
\item En \Emph{typparameter} gör det möjligt att ge ett \Emph{typargument}.
\item Detta kallas \Emph{parametrisk polymorfism} \Eng{paramteric polymorphism}.
\item Exempel: \Emph{generisk} \Alert{funktion}:
\begin{Code}
def tnirp[A](x: A):Unit = println(x.toString.reverse)
\end{Code}
\pause
\item En \Emph{fri} typparameter kan bindas till vilken typ som helst.
\item Bindingen av typargument till typparametrar sker vid \Alert{kompileringstid}.
\item En typparameter är \Emph{fri} om den \Alert{inte} fått något värde, annars \Emph{bunden}. 
\pause
\item Exempel: \Emph{generisk} \Alert{klass} med \Emph{generiska} \Alert{metoder}:
\begin{Code}
class Cell[A](   // [A] är fri (måste bindas vid användning) 
    var value: A):                              // A är bunden
  def update(a: A): Unit = value = a            // A är bunden
  def replaced[B](b: B): Cell[B] = new Cell(b)  // första [B] är fri
\end{Code}
\pause
\item \Alert{Skuggning kan förekomma}: Om \code{replaced} i \code{Cell} hade använt namnet A på sin typparameter hade den \Emph{skuggat} klassens typparameter och tolkats som en  fri typparameter, alltså en godtycklig typ och \Alert{inte} klassens typparameter. (jämför  namnöverskuggning vid \Emph{lokala} namn i nästlade block)
\end{itemize}

\end{Slide}

\ifkompendium\else
\begin{Slide}{Exempel: Generisk funktion}
Vad händer här?
\begin{REPL}

scala> def skrikBaklänges(x: T): String = x.toString.toUpperCase.reverse
???



scala> def skrikBaklänges[T](x: T): String = x.toString.toUpperCase.reverse

scala> skrikBaklänges("gurka är gott")
val res0: ???

\end{REPL}
\end{Slide}


\begin{Slide}{Exempel: Generisk funktion}
Vad händer här?
\begin{REPL}

scala> def skrikBaklänges(x: T): String = x.toString.toUpperCase.reverse
1 |def skrikBaklänges(x: T): String = x.toString.toUpperCase.reverse
  |                      ^
  |                      Not found: type T
                             ^

scala> def skrikBaklänges[T](x: T): String = x.toString.toUpperCase.reverse

scala> skrikBaklänges("gurka är gott")
val res0: ???
\end{REPL}
\end{Slide}
\fi

\begin{Slide}{Exempel: Generisk funktion}
Vad händer här?
\begin{REPL}

scala> def skrikBaklänges(x: T): String = x.toString.toUpperCase.reverse
1 |def skrikBaklänges(x: T): String = x.toString.toUpperCase.reverse
  |                      ^
  |                      Not found: type T
                             ^

scala> def skrikBaklänges[T](x: T): String = x.toString.toUpperCase.reverse

scala> skrikBaklänges("gurka är gott")
val res0: String = TTOG RÄ AKRUG
\end{REPL}
Om ingen typparameter deklareras inom hakparenteser efter funktionens namns så vet inte kompilatorn vad \code{T} är för en typ. Men med en typparameter \code{[T]} efter funktionsnamnet tolkar kompilatorn funktionen som \Emph{generisk} och typen \code{T} bestäms av argumentets typ \Alert{vid anrop} och \code{T} kan bindas till godtycklig typ.
\end{Slide}


\begin{Slide}{Exempel: Generisk case-klass}
\begin{itemize}\SlideFontTiny
\item En generisk klass har en eller flera typparametrar efter klassnamnet:
\begin{CodeSmall}
case class Box[A](value: A)  
\end{CodeSmall}
\item Kompilatorn härleder typparameterarnas typ utifrån givna värden. 
\begin{REPLsmall}
scala> Box("gurka")  
val res1: Box[String] = Box(gurka)
\end{REPLsmall}
\item Du kan också ge typpparametern en typ explicit:
\begin{REPLsmall}
scala> Box[Int](42) 
val res2: Box[Int] = Box(42)
\end{REPLsmall}
\item Om typen inte stämmer får du hjälp av kompilatorn att hitta felet:
\begin{REPLsmall}
scala> Box[String](42)
-- Error:
1 |Box[String](42)
  |            ^^ Found:    (42 : Int)  Required: String
\end{REPLsmall}
\item En generisk klass, här \code{Box}, kallas också \Emph{typkonstruktor} \Eng{type constructor} då den ''färdiga'' typen \code{Box[Int]} ''konstrueras'' på platsen där den används.
\end{itemize}
\end{Slide}




\begin{Slide}{Fallgrop: Typradering \Eng{type erasure}}\SlideFontSmall
Informationen om typerna i typparametrar raderas innan kodgenerering för JVM av prestandaskäl och \Alert{typparametrar saknas vid runtime} i bytekoden.
\vspace{-0.25em}\begin{REPL}
scala> def isIntVector[T](xs: Vector[T]) = xs.isInstanceOf[Vector[Int]]
-- Warning:
1 |def isIntVector[T](xs: Vector[T]) = xs.isInstanceOf[Vector[Int]]
  |                                    ^^^^^^^^^^^^^^^^^^^^^^^^^^^^
  |                the type test for Vector[Int] cannot be checked at runtime
def isIntVector[T](xs: Vector[T]): Boolean

scala> isIntVector(Vector("hej"))
res42: Boolean = true  // AAAARGHH!! :(
\end{REPL}
Måste ''packa upp'' samlingen och typtesta alla element:
\begin{REPL}
scala> def isIntVector[T](xs: Vector[T]) = xs.forall(_.isInstanceOf[Int])

scala> isIntVector(Vector("hej"))
res43: Boolean = false  // FUNKAR :)

\end{REPL}
Typkontroll vid körtid görs oftast hellre med \code{match}.

\end{Slide}

\Subsection{Upptäcka och åtgärda buggar}

\begin{Slide}{Testning och avlusning}
%\TODO 
\begin{itemize}
\item Läs om testning och avlusning \Eng{debugging} i Appendix D: ''Fixa buggar'' 
\item Träna på println-debugging
\item Prova debuggern i VS code
%\item Visa hur testramverket ska funka som du ska skapa på övning och använda på labb
%\item sbt testOnly och andra sätt att köra testfall
%\item Visa hur en fördröjning kan skapas med en s.k. thunk 
%\item Visa hur printlndebugging
%\item Visa hur debugga i vs code
\end{itemize}
\end{Slide}


% \ifkompendium\else


% \begin{Slide}{Typparametrar på tentan?}
% \begin{itemize}
% \item Det ingår att kunna använda färdiga generiska strukturer med specifika typer, t.ex. \code{Vector[Int]}

% \item Det ingår att kunna skapa abstraktioner med specifika typparametrar, t.ex. metoder eller klasser som tar en vektor med en specifik typ som parameter:\\
% \code{case class X(x: Vector[Int])}


% \item Det ingår \Alert{inte} på tentan att kunna skapa generiska metoder eller klasser, t.ex.: \\
% \code{def f[T](x: Vector[T]): Vector[T] = ???} \\
% Mer om generiska strukturer i fördjupningskursen!
% \end{itemize}
% \end{Slide}

% \fi


%\input{generated/w08-chaphead-generated.tex}

%!TEX encoding = UTF-8 Unicode
%!TEX root = ../exercises.tex

\ifPreSolution

\Exercise{\ExeWeekEIGHT}\label{exe:W08}

\begin{Goals}
\item Kunna skapa och använda matriser med nästlade strukturer av \code{Vector}.
\item Kunna iterera över elementen i en matris med nästlade \code{for}-satser och \code{for}-\code{yield}-uttryck, samt nästlad applicering av \code{map} respektive \code{foreach}.
\item Kunna skapa och använda funktioner som tar matriser som parametrar.
\item Kunna skapa en enkel generisk klass och enkla generiska funktioner med hjälp av en typparameter.
\item Kunna beskriva skillnader och likheter mellan Scala och Java vad gäller indexering och iterering i matriser implementerade med nästlade arrayer.
%\item Kunna skapa och använda matriser med hjälp inbyggda arrayer i Java.
%\item Kunna använda nästlade \code{for}-satser i Java för att iterera över elementen i en matris.
\end{Goals}

\begin{Preparations}
\item \StudyTheory{08}
\end{Preparations}

\BasicTasks

\else

\ExerciseSolution{\ExeWeekEIGHT}

\BasicTasks

\fi



\WHAT{Para ihop begrepp med beskrivning.}

\QUESTBEGIN

\Task \what

\vspace{1em}\noindent Koppla varje begrepp med den (förenklade) beskrivning som passar bäst:

\begin{ConceptConnections}
\input{generated/quiz-w08-concepts-taskrows-generated.tex}
\end{ConceptConnections}

\SOLUTION

\TaskSolved \what

\begin{ConceptConnections}
\input{generated/quiz-w08-concepts-solurows-generated.tex}
\end{ConceptConnections}

\QUESTEND




\WHAT{Skapa matriser med hjälp av nästlade samlingar.}

\QUESTBEGIN

\Task  \what~  Man kan i ett datorprogram, med hjälp av samlingar som innehåller samlingar, skapa nästlade strukturer som kan indexeras i två dimensioner och på så sätt representera en  \textbf{matris}.\footnote{\href{https://sv.wikipedia.org/wiki/Matris}{sv.wikipedia.org/wiki/Matris}}

\Subtask Rita minnessituationen efter tilldelningen på rad 1 nedan. Vad har \code{m} för typ och värde? Vad har \code{m} för dimensioner? Hur sker indexeringen i ett datorprogram jämfört med i matematiken?

\begin{REPL}
scala> val m = Vector((1 to 5).toVector, (3 to 7).toVector)
scala> m.apply(0).apply(1)
scala> m(1)
scala> m(1)(4)
\end{REPL}

\Subtask Vad ger uttrycken på raderna 2, 3 och 4 ovan för värden och typ?

\Subtask Man kan i ett datorprogram mycket väl skapa tvådimensionella, nästlade strukturer där raderna \emph{inte} innehåller samma antal element. Det blir då ingen äkta matris i strikt matematisk mening, men man kallar ofta ändå en sådan struktur för en ''matris''. Vilken typ har variablerna \code{m2}, \code{m3}, \code{m4} och \code{m5} nedan?

\begin{REPL}
scala> val m2 = Vector(Vector(1,2,3),Vector(4,5),Vector(42))
scala> val m3 = Vector(Vector(1,2), Vector(1.0, 2.0, 3.0))
scala> val m4 = m3(1) +: Vector("a") +: m3
scala> val m5 = Vector.fill(42){ m2(1).map(e => (e * math.random()).toInt) }
\end{REPL}

\Subtask Vilken av variablerna \code{m2}, \code{m3}, \code{m4} och \code{m5} ovan representerar en äkta matris i matematisk mening? Vilken är dess dimensioner?

\SOLUTION

\TaskSolved \what

\SubtaskSolved   \includegraphics{../img/w09-solutions/1a} \\
Typ: \code{Vector[Vector[Int]]}\\
Värde: \code{Vector(Vector(1, 2, 3, 4, 5), Vector(3, 4, 5, 6, 7))} \\
Dimensioner: $2 \times 5$\\
Inom matematiken sker indexering enligt konvention med 1 som lägsta index. I scala är lägsta index 0, man använder s.k. 0-indexering. \footnote{Detta är inte fallet i alla programmeringsspråk, vilket du kan läsa mer om på \url{https://en.wikipedia.org/wiki/Array\_data\_type\#Index\_origin}}

\SubtaskSolved
\begin{REPL}
scala> val m = Vector((1 to 5).toVector, (3 to 7).toVector)
m: Vector[Vector[Int]] = Vector(Vector(1, 2, 3, 4, 5), Vector(3, 4, 5, 6, 7))

scala> m.apply(0).apply(1)
res4: Int = 2

scala> m(1)
res5: Vector[Int] = Vector(3, 4, 5, 6, 7)

scala> m(1)(4)
res6: Int = 7
\end{REPL}

\SubtaskSolved  \\
m2: \code{Vector[Vector[Int]]}\\
m3: \code{Vector[Vector[Int | Double]]}\\
m4: \code{Vector[Vector[Int | Double | String]]}\\
m5: \code{Vector[Vector[Int]]}

\SubtaskSolved  m5, $42 \times 2$

\QUESTEND





\WHAT{Skapa och iterera över matriser.}

\QUESTBEGIN

\Task  \label{matrices:task:yatzy} \what~  Du ska skapa matriser där varje rad representerar 5 kast med en tärning i spelet Yatzy.\footnote{\href{https://sv.wikipedia.org/wiki/Yatzy}{sv.wikipedia.org/wiki/Yatzy}}


\Subtask Definiera i REPL en funktion \code{def throwDie: Int = ???} som returnerar ett slumptal mellan 1 och 6.

\Subtask Skapa nedan heltalsmatris i REPL. Vilken dimension får matrisen?
\begin{REPL}
scala> val ds1 = for (i <- 1 to 1000) yield 
            for (j <- 1 to 5) yield throwDie
          
\end{REPL}

\Subtask Man kan också använda nedan varianter för att skapa en heltalsmatris. Vilken av varianterna \code{ds1} ... \code{ds6} tycker du är lättast att läsa och förstå? Prova respektive variant i REPL och ange vilken typ på \code{ds1} ... \code{ds6} som härleds av kompilatorn.
\begin{REPL}
val ds2 = (1 to 1000).map(i => (1 to 5).map(j => throwDie))
val ds3 = (1 to 1000).map(i => Vector.fill(5)(throwDie))
val ds4 = for (i <- 1 to 1000) yield Vector.fill(5)(throwDie)
val ds5 = Vector.fill(1000)(Vector.fill(5)(throwDie))
val ds6 = Vector.fill(1000, 5)(throwDie)
\end{REPL}


\Subtask Definiera en funktion \\ \code{def roll(n: Int): Vector[Int] = ???}\\ som ger en heltalsvektor med $n$ stycken slumpvisa tärningskast. Kasten ska vara sorterade i växande ordning; använd för detta ändamål samlingsmetoden \code{sorted}.


\Subtask \label{matrices:subtask:isyatzyforall} Definera i REPL en funktion \code{isYatzy(xs: Vector[Int]): Boolean = ???} som testar om alla elementen i en heltalsvektor är samma. Använd samlingsmetoden \code{forall}.


\Subtask Skapa en funktion  \\ \code{def diceMatrix(m: Int, n: Int): Vector[Vector[Int]] = ???} \\ som med hjälp av funktionen \code{roll} skapar en matris med \code{m} st vektorer med vardera \code{n} slumpvisa tärningskast.


\Subtask \label{matrices:subtask:diceMatrixToString} Skapa en funktion som returnerar en utskriftsvänlig sträng \\ \code{def diceMatrixToString(xss: Vector[Vector[Int]]): String = ???} \\med hjälp av \code{map} och \code{mkString}, som fungerar enligt nedan.
\begin{REPL}
scala> val dm2s = diceMatrixToString(diceMatrix(4, 5))
val dm2s: String = 1 4 4 6 6
1 1 2 6 6
2 4 4 5 6
1 1 5 6 6

scala> println(dm2s)
1 4 4 6 6
1 1 2 6 6
2 4 4 5 6
1 1 5 6 6
\end{REPL}



\Subtask Implementera funktionen \\ \code{def filterYatzy(xss: Vector[Vector[Int]]): Vector[Vector[Int]]} \\ som filtrerar fram alla yatzy-rader i matrisen \code{xss} enligt nedan. Använd din funktion \code{isYatzy} och samlingsmetoden \code{filter}.
\begin{REPL}
scala> println(diceMatrixToString(filterYatzy(diceMatrix(10000, 5))))
4 4 4 4 4
6 6 6 6 6
4 4 4 4 4
6 6 6 6 6
4 4 4 4 4
4 4 4 4 4
2 2 2 2 2
\end{REPL}



\Subtask Implementera funktionen \\
\code{def yatzyPips(xss: Vector[Vector[Int]]): Vector[Int] = ???}\\
som ska ge en vektor med de tärningsvärden som gav yatzy, för kasten i matrisen \code{xss} enligt nedan. Använd din funktion \code{filterYatzy}.
\begin{REPL}
scala> val dm = Vector(Vector(1,2,3,4,5),Vector(4,4,4,4,4),Vector(3,3,3,3,3))
scala> yatzyPips(dm)
val res42: Vector[Int] = Vector(4, 3)
\end{REPL}

\SOLUTION

\TaskSolved \what

\SubtaskSolved
\begin{Code}
def throwDie: Int = (math.random() * 6).toInt + 1
\end{Code}
Eller:
\begin{Code}
def throwDie: Int = scala.util.Random.nextInt(6) + 1
\end{Code}

\SubtaskSolved  Matrisdimension i matematisk notation: $1000 \times 5$, vilket motsvarar en matris med 1000 rader och 5 kolumner.

\SubtaskSolved
\begin{Code}
ds1: IndexedSeq[IndexedSeq[Int]]
ds2: IndexedSeq[IndexedSeq[Int]]
ds3: IndexedSeq[Vector[Int]]
ds4: IndexedSeq[Vector[Int]]
ds5: Vector[Vector[Int]]
ds6: Vector[Vector[Int]]
\end{Code}
\code{IndexedSeq} och \code{Vector} ovan finns i paketet \code{scala.collection.immutable}

\SubtaskSolved  \begin{Code}
def roll(n: Int) = Vector.fill(n)(throwDie).sorted
\end{Code}

\SubtaskSolved  \begin{Code}
def isYatzy(xs: Vector[Int]): Boolean = xs.forall(_ == xs(0))
\end{Code}



%2.g)
\SubtaskSolved  \begin{Code}
def diceMatrix(m: Int, n: Int): Vector[Vector[Int]] =
  Vector.fill(m)(roll(n))
\end{Code}

\SubtaskSolved  \begin{Code}
def diceMatrixToString(xss: Vector[Vector[Int]]): String =
  xss.map(_.mkString(" ")).mkString("\n")
\end{Code}


%2.j)
\SubtaskSolved
\begin{Code}
def filterYatzy(xss: Vector[Vector[Int]]): Vector[Vector[Int]] =
  xss.filter(isYatzy)
\end{Code}



%2.m)
\SubtaskSolved  \begin{Code}
def yatzyPips(xss: Vector[Vector[Int]]): Vector[Int] =
  filterYatzy(xss).map(_.head)
\end{Code}

\QUESTEND








\WHAT{En oföränderlig, generisk matris-klass till veckans laboration \hyperref[section:lab:\LabWeekEIGHT]{\texttt{\LabWeekEIGHT}}.}

\QUESTBEGIN

\Task\label{exe:matrices:labprep}  \what~Under veckans laboration ska du simulera en enkel form av ''liv'' som består av celler i ett rutnät. För detta ändamål har vi nytta av en matris-klass som du ska implementera steg för steg i denna övning.
Skapa case-klassen nedan med en editor i filen \code{Matrix.scala}. Testa din lösning med hjälp av valfri \hyperref[appendix:ide]{IDE}, t.ex. \code{scalaide} eller \code{idea}.
\begin{Code}
case class Matrix(data: Vector[Vector[String]]){
  def apply(row: Int, col: Int): String = data(row)(col)
}
object Matrix {
  def fill(dim: (Int, Int))(value: String): Matrix =
    Matrix(Vector.fill(dim._1, dim._2)(value))
}
\end{Code}

\begin{REPLnonum}
scala> val m = Matrix.fill(3,4)("hej")
scala> val e = m(2, 2)
\end{REPLnonum}

\Subtask Vad får \code{m} ovan för typ?

\Subtask Vad får \code{e} ovan för typ?

\Subtask På hur många ställen måste du ändra i \code{Matrix} ovan för att den i stället ska representera en matris av heltal?

\Subtask Du ska nu med hjälp av en \textbf{typparameter} göra \code{Matrix} \textbf{generisk} \Eng{generic}, så att den blir en mer användbar matrisklass som kan innehålla element av vilken typ som helst. Genomför följande ändringar i \code{Matrix.scala}:

\begin{itemize}[noitemsep, nolistsep]
  \item Lägg till en typparameter \code{T} inom klammerparenteser efter namnet \code{Matrix} på alla ställen där det förekommer \emph{utom} efter namnet på kompanjonsobjektet\footnote{Singelobjekt kan inte ha typparametrar, men deras medlemmar kan.}.
  \item Byt ut \code{String} mot \code{T} på alla ställen där \code{String} förekommer.
  \item Lägg till en typparameter \code{T} inom klammerparenteser efter \code{def fill}.
\end{itemize}
Testa din generiska klass i REPL genom att skapa en boolesk matris:
\begin{REPLnonum}
scala> val bm = Matrix.fill(3,4)(false)
scala> val be = bm(0, 0)
\end{REPLnonum}

\Subtask Vad får \code{bm} ovan för typ?

\Subtask Vad får \code{be} ovan för typ?

\Subtask Lägg en kodrad i början av klasskroppen som med hjälp av \code{require} garanterar att alla rader i matrisen är lika långa.

\Subtask Lägg till en medlem \code{val dim: (Int, Int)} i klasskroppen efter \code{require}-satsen som ger ett par (alltså en 2-tupel) med antalet rader resp. kolumner i matrisen.

\Subtask Lägg till en metod \code{def updated(row: Int, col: Int)(value: T): Matrix[T]} som ger en ny matris där element på platsen \code{(row, col)} har uppdaterats till \code{value}.

\Subtask Lägg till en metod \code{def foreachIndex(f: (Int, Int) => Unit): Unit} som för varje index i \code{data} applicerar funktionen \code{f}.

\Subtask Lägg till en metod \code{override def toString} som så att en instans av \code{Matrix} visas enligt följande:
\begin{REPLnonum}
scala> val dm = Matrix.fill(3,4)(42.0)
val dm: Matrix[Double] =
Matrix of dim (3,4):
42.0 42.0 42.0 42.0
42.0 42.0 42.0 42.0
42.0 42.0 42.0 42.0
\end{REPLnonum}


\SOLUTION


\TaskSolved \what

\SubtaskSolved Typen på \code{m} blir \code{Matrix}.

\SubtaskSolved Typen på \code{e} blir \code{String}.

\SubtaskSolved Man behöver ändra på 3 ställen från \code{String} till \code{Int}.

\SubtaskSolved Generisk matris \code{Matrix[T]} för element av godtycklig typ \code{T}:

\begin{CodeSmall}
case class Matrix[T](data: Vector[Vector[T]]):
  def apply(row: Int, col: Int): T = data(row)(col)

object Matrix:
  def fill[T](dim: (Int, Int))(value: T): Matrix[T] =
    Matrix[T](Vector.fill(dim._1, dim._2)(value))
\end{CodeSmall}

\SubtaskSolved Tack vare kompilatorns typinferens så får \code{bm} typen \code{Matrix[Boolean]}.

\SubtaskSolved Typen på \code{be} blir \code{Boolean}.

\noindent \SubtaskSolved \SubtaskSolved \SubtaskSolved \SubtaskSolved \SubtaskSolved är alla implementerade i koden nedan: \vspace{-0.5em}
\begin{CodeSmall}
case class Matrix[T](data: Vector[Vector[T]]):
  require(data.forall(row => row.length == data(0).length))

  val dim: (Int, Int) = (data.length, data(0).length)

  def apply(row: Int, col: Int): T = data(row)(col)

  def updated(row: Int, col: Int)(value: T): Matrix[T] =
    Matrix(data.updated(row, data(row).updated(col, value)))

  def foreachIndex(f: (Int, Int) => Unit): Unit =
    for r <- data.indices; c <- data(r).indices do f(r, c)

  override def toString =
    s"""Matrix of dim $dim:\n${ data.map(_.mkString(" ")).mkString("\n") }"""

object Matrix:
  def fill[T](dim: (Int, Int))(value: T): Matrix[T] =
    Matrix[T](Vector.fill(dim._1, dim._2)(value))

\end{CodeSmall}

\QUESTEND


\clearpage

\ExtraTasks %%%%%%%%%%%%%%%%%%%%%%%%%%%%%%%%%%%%%%%%%%%%%%%%%


\WHAT{Imperativa matrisalgoritmer.}

\QUESTBEGIN

\Task  \what~Imperativa angreppssätt är nödvändiga att kunna när du stöter på samlingar och/eller språk som saknar funktionella metoder och/eller funktionsprogrammeringsmöjligheter. Genom att studera imperativa lösningar till de ofta mer koncisa funktionella lösningarna, får du träning i att skapa algoritmer som använder förändring genom tilldelning vid iterering.

\Subtask Implementera \code{isYatzy} från uppgift \ref{matrices:task:yatzy}\ref{matrices:subtask:isyatzyforall} igen, men nu med ett imperativt angreppssätt som använder en \code{while}-sats i stället för funktionella \code{forall}. Ta hjälp av en variabel \code{i} som håller reda på index och en variabel \code{foundDiff} som håller reda på om ett avvikande värde upptäcks. Funktionen kräver ca 9 rader, så det kan vara lämpligt att öppna en editor att skriva i medan du klurar ut lösningen. Börja med att skriva pseudokod, gärna med penna på papper. Prova genom att klistra in i REPL.

\Subtask En imperativ implementation av \code{diceMatrixToString} från uppgift \ref{matrices:task:yatzy}\ref{matrices:subtask:diceMatrixToString} med hjälp av förändringsbara  \code{StringBuilder}\footnote{\url{https://www.scala-lang.org/api/2.12.9/scala/collection/mutable/StringBuilder.html}} visas nedan. Förklara hur nedan kod fungerar. Vad händer om \code{xss} är tom? Vad händer om \code{xss} bara innehåller tomma vektorer? Nämn en fördel och en nackdel med att använda \code{val sb: StringBuilder} och \code{append}, jämfört med en vanlig, oföränderlig \code{var s: String} och \code{+} för tillägg i slutet.
\begin{Code}
def diceMatrixToString(xss: Vector[Vector[Int]]): String = 
  val sb = new StringBuilder()
  for(m <- xss.indices) do
    for(n <- xss(m).indices) do
      sb.append(xss(m)(n).toString)
      if n < xss(m).size - 1 then sb.append(" ")
      else if m < xss.size - 1 then sb.append("\n")
    end for
  end for
  sb.toString
\end{Code}

\Subtask Gör som träning en imperativ implementation av \code{filterYatzy} med en \code{for}-\code{do}-sats (alltså utan att använda \code{filter}, och utan att använda \code{yield}).


\Subtask Förklara hur nedan funktionella implementation av \code{filterYatzy} med \code{for}-\code{yield}-uttryck fungerar. Tycker du din imperativa lösning är lättare eller svårare att läsa och förstå jämfört nedan funktionella lösning?
\begin{CodeSmall}
def filterYatzy(xss: Vector[Vector[Int]]): Vector[Vector[Int]] = 
  (for i <- xss.indices if isYatzy(xss(i)) yield xss(i)).toVector
\end{CodeSmall}


\SOLUTION

\TaskSolved \what

\SubtaskSolved  \begin{Code}
def isYatzy(xs: Vector[Int]): Boolean = 
  var foundDiff = false
  var i = 0
  while (i < xs.size && !foundDiff) do
    foundDiff = xs(i) != xs(0)
    i += 1
  end while
  !foundDiff
\end{Code}


\SubtaskSolved  Funktionen går igenom varje matrisrad, där den i sin tur går igenom
varje element på raden och lägger till i \code{StringBuilder}-objektet. Om det inte är
det sista elementet på raden läggs även ett blanktecken till, annars läggs ett
nyradstecken till. Undantaget är sista raden, där inget nyradstecken läggs till.
Slutligen konverteras \code{StringBuilder}-objektet till en \code{String} som
returneras.


Är \code{xss} tom blir \code{xss.indices} en tom \code{Range} och den yttre \code{for}-loopen hoppas över och en tom sträng returneras.
Är alla rader tomma hoppas i stället de inre \code{for}-looparna över, med samma resultat.

\emph{Fördel:} \code{StringBuilder} är snabbare vid tillägg på slutet vid stora strängar (men här kommer det inte märkas eftersom strängen är så liten).

\emph{Nackdel:} StringBuilder-koden uppfattas av många som svårare att läsa.

\SubtaskSolved
\begin{Code}
def filterYatzy(xss: Vector[Vector[Int]]): Vector[Vector[Int]] = 
  var result: Vector[Vector[Int]] = Vector()
  for i <- xss.indices if isYatzy(xss(i)) do result = result :+ xss(i)
  result
\end{Code}

\SubtaskSolved  Varje looprunda ger en vektor \code{xss(i)} om filtervillkoret är uppfyllt och resultatet av \code{for}-uttrycket blir en vektor med vektorer som är yatzyslag.

\QUESTEND



\WHAT{Strängtabell med kolumnrubriker.}

\QUESTBEGIN

\Task  \what~  %Denna övning utgör en början på laboration \hyperref[section:lab:survey]{\texttt{survey}} i avsnitt \ref{section:lab:survey} på sidan \pageref{section:lab:survey}.

\Subtask Implementera case-klassen \code{Table} enligt specifikationen nedan. Du kan förutsätta att alla rader har lika många kolumner som antalet element i \code{headings}, samt att alla rubrikerna i \code{headings} är unika. Parametern \code{sep} anger det tecken som används för att separera kolumner. Detta förutsätts också gälla för indatafiler som läses in med \code{fromFile}.

\emph{Tips:}
\begin{itemize}%[nolistsep,noitemsep]
\item Värdet \code{indexOfHeading} kan skapas med hjälp av metoden \code{zipWithIndex} som fungerar på alla sekvenssamlingar, samt metoden \code{toMap} som fungerar på sekvenser av 2-tupler. Undersök först hur metoderna fungerar i REPL och sök upp deras dokumentation.
\item Skapa en indatafil som du kan använda för att testa att \code{Table} fungerar.
\end{itemize}


\begin{CodeSmall}
case class Table(
  data: Vector[Vector[String]],
  headings: Vector[String],
  sep: Char
):
  /** A 2-tuple with (number of rows, number of columns) in data */
  val dim: (Int, Int) = ???

  /** The element in row r and column c of data, counting from 0 */
  def apply(r: Int, c: Int): String = ???

  /** The row-vector r in data, counting from 0 */
  def row(r: Int): Vector[String]= ???

  /** The column-vector c in data, counting from 0 */
  def col(c: Int): Vector[String] = ???

  /** A map from heading to index counting from 0 */
  lazy val indexOfHeading: Map[String, Int] = ???

  /** The column-vector with heading h in data */
  def col(h: String): Vector[String] = ???

  /** A vector with the distinct, sorted values of col with heading h */
  def values(h: String): Vector[String] = ???

  /** Headings and data with columns separated by sep */
  override lazy val toString: String = ???

object Table:
  /** Creates a new Table from fileName with columns split by sep */
  def fromFile(fileName: String, sep: Char = ';'): Table = ???
\end{CodeSmall}

\Subtask Skapa med hjälp av \code{Table} ett program som kan köras från terminalen med \texttt{scala run infile.csv ';'} som ger en utskrift av antalet förekomster av olika värden i respektive kolumn (alltså en variant av registrering).



\SOLUTION

\TaskSolved \what

\SubtaskSolved  \begin{CodeSmall}
case class Table(
  data: Vector[Vector[String]],
  headings: Vector[String],
  sep: Char
):

  val dim: (Int, Int) = (data.size, headings.size)

  def apply(r: Int, c: Int): String = data(r)(c)

  def row(r: Int): Vector[String]= data(r)

  def col(c: Int): Vector[String] = data.map(r => r(c))

  lazy val indexOfHeading: Map[String, Int] = headings.zipWithIndex.toMap

  def col(h: String): Vector[String] = col(indexOfHeading(h))

  def values(h: String): Vector[String] = col(h).distinct.sorted

  override def toString: String =
    val s = sep.toString
    headings.mkString(s) + "\n" +data.map(_.mkString(s)).mkString("\n")

object Table:
  def fromFile(fileName: String, sep: Char = ';'): Table = 
    val lines = scala.io.Source.fromFile(fileName).getLines.toVector
    val matrix= lines.map(_.split(sep).toVector)
    new Table(matrix.tail, matrix.head, sep)
\end{CodeSmall}

\SubtaskSolved  \begin{CodeSmall}
@main 
def run(fileName: String, separator: String): Unit = 
  require(separator.length == 1, "separator ska vara exakt ett tecken")
  val t = Table.fromFile(fileName, separator.head)
  val counts: Vector[Vector[String]] =
    (0 until t.dim._2)
      .map(i => t.values(t.headings(i))
      .map(x => s"$x: ${t.col(i).count(_ == x)}"))
      .toVector
  for (i <- 0 until t.dim._2) do
    println(s"\nColumn: ${i + 1}, ${t.headings(i)}:")
    for (j <- 0 until counts(i).length) do
      println(counts(i)(j))
\end{CodeSmall}

\QUESTEND




\WHAT{Skapa ett yatzy-spel för användning i terminalen.}

\QUESTBEGIN

\Task  \what~%
% \Subtask Skapa en yatzy-matris enligt nedan specifikation. Läs om hur de olika predikaten för att kolla olika giltiga kombinationer i Yatzy ska fungera här: \href{https://en.wikipedia.org/wiki/Yahtzee}{en.wikipedia.org/wiki/Yahtzee}. Bygg ett huvudprogram som testar dina funktioner. Kompilera och testa i terminalen allteftersom du lägger till nya funktioner.
%
% \begin{CodeSmall}
% /** En skiss på en klass som kan användas till ett förenklat yatzy-spel */
% case class YatzyRows(val rows: Vector[Vector[Int]]) {
%   /** A new YatzyRows with a new row of 5 dice rolls appended to rows  */
%   def roll: YatzyRows = ???
%
%   /** A new YatzyRows with some indices of the last row re-rolled  */
%   def reroll(indices: Vector[Int]): YatzyRows = ???
% }
%
% object YatzyRows {
%   def isYatzy(xs: Vector[Int]): Boolean = ???
%   def isThreeOfAKind(xs: Vector[Int]): Boolean = ???
%   def isFourOfAKind(xs: Vector[Int]): Boolean = ???
%   def isFullHouse(xs: Vector[Int]): Boolean = ???
%   def isSmallStraight(xs: Vector[Int]): Boolean = ???
%   def isLargeStraight(xs: Vector[Int]): Boolean = ???
% }
% \end{CodeSmall}
%
%
% \Subtask Använd \code{YatzyRows} för att med hjälp av många tärningskast beräkna sannolikheter för några olika giltiga kombinationer. Använd, om du vill, möjligheten som reglerna ger att slå om tärningar i två ytterliggare kast, där de tärningar som slås om väljs slumpmässigt.
%
%\Subtask
Bygg ett förenklat yatzy-spel i terminalen där användaren kan bestämma vilka tärningar som ska slås om. Börja med något riktigt enkelt och bygg sedan vidare på ditt spel genom att införa fler och fler funktioner.

\SOLUTION


\TaskSolved \what
     %starts with: \emph{Skapa ett yatzy-spel för %%%

 --

% \SubtaskSolved   \begin{CodeSmall}
% /** En skiss på en klass som kan användas till ett förenklat yatzy-spel */
% case class YatzyRows(val rows: Vector[Vector[Int]]) {
%
%   private def throwDie: Int = (math.random() * 6).toInt + 1
%
%   /** A new YatzyRows with a new row of 5 dice rolls appended to rows */
%   def roll: YatzyRows = new YatzyRows(rows :+ Vector.fill(5)(throwDie))
%
%   /** A new YatzyRow with some indices of the last row re-rolled */
%   def reroll(indices: Vector[Int]): YatzyRows =
%     new YatzyRows(rows :+ rows(rows.length - 1).zipWithIndex.map {
%       case (x, i) => if (indices.contains(i)) throwDie else x
%     })
% }
% object YatzyRows {
%
%   def isYatzy(xs: Vector[Int]): Boolean = xs.forall(_ == xs(0))
%
%   def isThreeOfAKind(xs: Vector[Int]): Boolean =
%     xs.exists(x => xs.count(_ == x) >= 3)
%
%   def isFourOfAKind(xs: Vector[Int]): Boolean =
%     xs.exists(x => xs.count(_ == x) >= 4)
%
%   def isFullHouse(xs: Vector[Int]): Boolean =
%     xs.exists(x => xs.count(_ == x) == 3) &&
%     xs.exists(x => xs.count(_ == x) == 2)
%
%   def isSmallStraight(xs: Vector[Int]): Boolean =
%     xs.forall(x => xs.count(_ == x) == 1) && !xs.exists(_ == 6)
%
%   def isLargeStraight(xs: Vector[Int]): Boolean =
%     xs.forall(x => xs.count(_ == x) == 1) && !xs.exists(_ == 1)
% }
%
% \end{CodeSmall}
% Observera att fem stycken 2:or uppfyller kraven för Yatzy, men även för triss och fyrtal.
%
% \SubtaskSolved   Slumpen gör att utfallet inte kommer stämma exakt överens med teorin, men för ett stort antal kast bör resultaten hamna ganska nära. De teoretiska sannolikheterna (utan omkast) finns i \ref{yatzyProb}.
% \begin{table}[h]
% \centering
% \caption{Sannolikhet för olika Yatzy-resultat}
% \label{yatzyProb}
% \begin{tabular}{ll}
% Yatzy&  $0,077\%$  \\
% $\geq3$ av samma& $21\%$\\
% $\geq4$ av samma& $2,0\%$\\
% Kåk& $3,9\%$\\
% Liten stege& $1,5\%$\\
% Stor stege& $1,5\%$
% \end{tabular}
% \end{table}
%
% Kodexempel:
% \begin{CodeSmall}
% import YatzyRows._
%
% object YatzyStats extends App {
%   val n = 1000000.0
%   var yr = YatzyRows(Vector(Vector[Int]()))
%   for (i <- 1 to n.toInt) yr = yr.roll
%   println(s"Yatzy: ${yr.rows.count(isYatzy(_)) / n * 100}%")
%   println(s"Three of a kind: ${yr.rows.count(isThreeOfAKind(_)) / n * 100}%")
%   println(s"Four of a kind: ${yr.rows.count(isFourOfAKind(_)) / n * 100}%")
%   println(s"Full house: ${yr.rows.count(isFullHouse(_)) / n * 100}%")
%   println(s"Small straight: ${yr.rows.count(isSmallStraight(_)) / n * 100}%")
%   println(s"Large straight: ${yr.rows.count(isLargeStraight(_)) / n * 100}%")
% }
% \end{CodeSmall}
%
% \SubtaskSolved  --

\QUESTEND






\clearpage

\AdvancedTasks %%%%%%%%%%%%%%%%%


\WHAT{Generiska funktioner.}

\QUESTBEGIN

\Task  \what~  En generisk funktion har (minst) en typparameter inom klammerparenteser efter namnet, till exempel \code{[T]}. Denna typ förekommer sedan som typ på (någon av) parametrarna i parameterlistan. Kompilatorn härleder en konkret typ vid kompileringstid och ersätter typparametern med denna konkreta typ. På så sätt kan en funktion fungera för många olika typer.

\Subtask Förklara för varje rad nedan vad som händer.

\begin{REPL}
scala> def tnirp[T](x: T): Unit = println(x.toString.reverse)
scala> tnirp(42)
scala> tnirp("hej")
scala> case class Gurka(vikt: Int)
scala> tnirp(Gurka(42))
scala> tnirp[String](42)
scala> tnirp[Double](42)
\end{REPL}

\Subtask Man kan kombinera generiska funktioner med funktioner som tar funktioner som parametrar. Det är så \code{map} och \code{foreach} är implementerade. Förklara för varje rad nedan vad som händer.

\begin{REPL}
scala> def compose[A, B, C](f: A => B, g: B => C)(x: A): C = g(f(x))
scala> def inc(x: Int): Int = x + 1
scala> def half(x: Int): Double = x / 2.0
scala> compose(inc, half)(42)
scala> compose(half, inc)(42)
\end{REPL}

\Subtask Hur lyder felmeddelandet på sista raden ovan? Ändra \code{inc} och/eller \code{half} så att typerna passar.

\SOLUTION

\TaskSolved \what
     %starts with: \emph{Generiska funkioner.} En %%%

%4.a)
\SubtaskSolved   \begin{enumerate}
\item --
\item Strängrepresentationen av \code{42} spegelvänds
\item \code{"hej"} spegelvänds - \code{toString} av en sträng ger en likadan sträng
\item --
\item Gurk-objektets strängrepresentation spegelvänds
\item Funktionens typparameter matchar inte parameterns typ: \code{42} är ingen sträng
\item Implicit typkonvertering till \code{Double} sker för att stämma överens med typparametern, vilket ger en strängrepresentation med decimal
\end{enumerate}

%4.b)
\SubtaskSolved   \begin{enumerate}
\item En funktion definieras så att den tar emot två andra funktioner som argument, sätter ihop dem, och matar in ett tredje argument till den den sammansatta funktionen.
\item En funktion som inkrementerar ett heltal med 1 definieras.
\item En funktion som halverar ett flyttal definieras.
\item \code{42} matas in i \code{inc()} och resultatet (\code{43}) matas vidare till \code{half()}. Inuti \code{half()} sker implicit typkonvertering till \code{Double} då talet divideras med ett flyttal (\code{2.0}) och resultatet blir \code{43.0 / 2.0}, alltså \code{21.5}.
\item Resultatet från \code{half()} är av typ \code{Double}, medan \code{inc()} tar emot ett argument av typ \code{Int}. Då flyttal generellt inte kan konverteras till heltal utan informationsförlust sker ingen implicit konvertering, istället sker ett kompileringsfel.
\end{enumerate}

%4.c)
\SubtaskSolved  \begin{Code}
def inc(x: Double): Double = x + 1.0
\end{Code}
Nu ges kompileringsfel på rad 4 istället, vilket kan lösas med följande ändring:
\begin{Code}
def half(x: Double): Double = x / 2.0
\end{Code}

\QUESTEND




\WHAT{Generiska klasser.}

\QUESTBEGIN

\Task  \what~  Även klasser kan vara generiska. En generisk klass har (minst) en typparameter inom klammerparenteser efter klassens namn.

\Subtask Testa nedan generiska klass \code{Cell[T]} i REPL. Skapa instanser av klassen \code{Cell[T]} där typparametern \code{T} binds till olika konkreta typer och förklara vad som händer.

\begin{REPL}
scala> class Cell[T](var value: T):
         override def toString = "Cell(" + value + ")"
       
scala> new Cell(42)
scala> new Cell("hej")
scala> new Cell(new Cell(math.Pi))
scala> new Cell[String](42)
scala> new Cell[Double](42)
\end{REPL}

\Subtask Lägg till metoden \code{def concat[U](that: Cell[U]):Cell[String]} i klassen \code{Cell} som konkatenerar strängrepresentationerna av de båda cellvärdena.

\begin{REPL}
scala> val a = new Cell("hej")
scala> val b = new Cell(42)
scala> a concat b
\end{REPL}

\Subtask Vilken sorts celler kan du konkatenera om du tar bort typparameternamnet \code{U} i \code{concat} samtidigt som du använder \code{Cell[T]} som typ på värdeparametern \code{that}? Vad ger det för konsekvenser för celler av annan typ än \code{Cell[String]}?

\SOLUTION

\TaskSolved \what

%5.a)
\SubtaskSolved  --

%5.b)
\SubtaskSolved  \begin{Code}
class Cell[T](var value: T):
  override def toString = "Cell(" + value + ")"
  def concat[U](that: Cell[U]): Cell[String] = 
    Cell(s"$value${that.value}")
\end{Code}

%5.c)
\SubtaskSolved   Endast celler med samma typparameter kan nu konkateneras. Eftersom \code{concat()} returnerar ett objekt av typ \code{Cell[String]} kan ett ojämnt antal celler med någon annan typparameter än \code{String} alltså inte längre konkateneras. Är antalet jämnt går det att konkatenera dem parvis och sedan konkatenera de returnerade \code{Cell[String]}-objekten, men det är något omständigt.

\QUESTEND

\WHAT{Implementera fler generiska metoder i \code{Matrix[T]}.}

\QUESTBEGIN

\Task \what~ Bygg vidare på uppgift \ref{exe:matrices:labprep} och implementera nedan specifikation. Skapa egna tester som kontrollerar att alla metoder fungerar som förväntat.

\begin{ScalaSpec}{Matrix[T]}
/** En oföränderlig, generisk Matris-klass. */
case class Matrix[T](data: Vector[Vector[T]]):
  require(???)  // garantera att alla rader har lika många kolumner

  /** Ger ett par med antal rader och kolumner. */
  val dim: (Int, Int) = ???

  /** Ger elementet på plats (row, col). */
  def apply(row: Int, col: Int): T = ???

  /** Ger en ny matris där elementet på plats (row, col) har värdet value. */
  def updated(row: Int, col: Int)(value: T): Matrix[T] =  ???

  /** Applicerar f på alla element. */
  def foreach(f: T => Unit): Unit = ???

  /** Applicerar f på alla index. */
  def foreachIndex(f: (Int, Int) => Unit): Unit = ???

  /** Ger en ny matris med resultaten av elementvis applicering av f. */
  def map[U](f: T => U): Matrix[U] = ???

  /** Ger en ny matris med resultaten av applicering av f på varje index. */
  def mapIndex[U](f: (Int, Int) => U): Matrix[U] = ???

  /** Ger en utskriftsvänlig strängrepresentation av matrisen. */
  override def toString = ???

object Matrix:
  /** Ger en matris med dimension dim där alla element har värdet value. */
  def fill[T](dim: (Int, Int))(value: T): Matrix[T] = ???
\end{ScalaSpec}

\SOLUTION


\TaskSolved \what

\begin{CodeSmall}
case class Matrix[T](data: Vector[Vector[T]]):
  require(data.forall(row => row.size == data(0).size))

  val dim: (Int, Int) = (data.length, data(0).length)

  def apply(row: Int, col: Int): T = data(row)(col)

  def updated(row: Int, col: Int)(value: T): Matrix[T] =
    Matrix(data.updated(row, data(row).updated(col, value)))

  def foreach(f: T => Unit): Unit = data.foreach(_.foreach(f))

  def foreachIndex(f: (Int, Int) => Unit): Unit =
    for r <- data.indices; c <- data(r).indices do f(r, c)

  def map[U](f: T => U): Matrix[U] = Matrix(data.map(_.map(f)))

  def mapIndex[U](f: (Int, Int) => U): Matrix[U] =
    var result = Matrix.fill(dim)(f(0,0))
    for 
      r <- data.indices
      c <- data(r).indices 
    do
      result = result.updated(r, c)(f(r, c))
    end for
    result

  override def toString =
    s"""Matrix of dim $dim:\n${ data.map(_.mkString(" ")).mkString("\n") }"""

object Matrix:
  def fill[T](dim: (Int, Int))(value: T): Matrix[T] =
    Matrix[T](Vector.fill(dim._1, dim._2)(value))
\end{CodeSmall}


\QUESTEND





% \WHAT{Skapa en generisk, oföränderlig matrisklass.}
%
% \QUESTBEGIN
%
% \Task \label{task:generic-matrix} \what~   Med hjälp av en typparameter kan vi skapa en matrisklass som kan innehålla vilka element som helst. Implementera nedan specifikation. Testa din matrisklass i REPL för olika typer av element.
%
% \begin{ScalaSpec}{Matrix[T]}
% case class Matrix[T](data: Vector[Vector[T]]){
%
%   def foreachRowCol(f: (Int, Int, T) => Unit): Unit =
%     for (r <- 0 until data.size) {
%       for (c <- 0 until data(r).size) {
%         f(r, c, data(r)(c))
%       }
%     }
%
%   def map[U](f: T => U): Matrix[U] = Matrix(data.map(_.map(f)))
%
%   /** The element at row r and column c */
%   def apply(r: Int, c: Int): T = ???
%
%   /** Gives Some[T](element) at row r and column c
%    *  if r and c are within index bounds, else None */
%   def get(r: Int, c: Int): Option[T] = ???
%
%   /** The row vector of row r */
%   def row(r: Int): Vector[T] = ???
%
%   /** The column vector of column c */
%   def col(c: Int): Vector[T] = ???
%
%   /** A new Matrix with element at row r and col c updated */
%   def updated(r: Int, c: Int, value: T): Matrix[T] = ???
% }
% object Matrix {
%   def fill[T](rowSize: Int, colSize: Int)(init: T): Matrix[T] =
%     new Matrix(Vector.fill(rowSize)(Vector.fill(colSize)(init)))
% }
% \end{ScalaSpec}
%
% \SOLUTION
%
%
% \TaskSolved \what
%      %%%TODO number  8 %%%starts with: \label{task:generic-matrix} \em%%%
%
% \SubtaskSolved  -- %%%TODO in task 8 %%%
%
%
%
% \QUESTEND
%

% \clearpage
%
% \WHAT{Skapa en Sprite-editor.}
%
% \QUESTBEGIN
%
% \Task  \what~ Använd matrisklassen från uppgift \ref{task:generic-matrix} för att göra en SpriteEditor med JColorChoser enligt nedan skiss.
%
% \begin{Code}
% object ColorChooser {
%   import java.awt.Color
%   import javax.swing.JColorChooser
%
%   var title = "Pick Color"
%   private val chooser = new JColorChooser(Color.BLACK)
%   private val dialog = JColorChooser.
%     createDialog(null, title, true, jcs, null, null)
%
%   def getColor(initColor: Color = Color.BLACK): Color = {
%     chooser.setColor(initColor)
%     dialog.setVisible(true)
%     chooser.getColor
%   }
% }
%
% class Sprite(// en bild med många lager av pixlar i olika färger
%   val id: String,
%   val size: (Int, Int),
%   val pixels: Matrix[Int],   // färg i colors, -1 betyder genomskinlig
%   var scale: Int,            // uppskalning av storlek i pixlar
%   var colors: Vector[Color], // tillgängliga färger
%   var pos: (Int, Int, Int)   // (row, col, layer)
% ){
%   def row = pos._1
%   def col = pos._2
%   def layer = pos._3
% }
%
% class SpriteEditor(
%     rows: Int = 64, cols: Int = 64,
%     scale: Int = 16, nColors: Int = 16) {
%   private val w = new SimpleWindow(???)
%   def edit: Unit = ???
% }
%
% \end{Code}
%
%
%
% \SOLUTION
%
%
% \TaskSolved \what
%      %%%TODO number  9 %%%starts with: \TODO \emph{Klasser för täta oc%%%
%
% \SubtaskSolved  -- %%%TODO in task 9 %%%
%
% \SubtaskSolved  -- %%%TODO in task 9 %%%
%
% \SubtaskSolved  -- %%%TODO in task 9 %%%
%
% \SubtaskSolved  -- %%%TODO in task 9 %%%
%
% \SubtaskSolved  -- %%%TODO in task 9 %%%
%
% \SubtaskSolved  -- %%%TODO in task 9 %%%
%
%
%
% \QUESTEND




% \WHAT{Klasser för täta och glesa matematiska matriser med flyttal.}
%
% \QUESTBEGIN
%
% \Task  \what~   Läs om matrisräkning här: \href{https://sv.wikipedia.org/wiki/Matris}{sv.wikipedia.org/wiki/Matris}
%
% \Subtask Skapa en oföränderlig klass \code{DenseMatrix} för matematiska matriser med dubbelprecisionsflyttal. \code{DenseMatrix} ska internt lagra elementen i en privat \emph{endimensionell} array av flyttal av typen \code{Array[Double]}.
%
% Klassen ska inte vara en case-klass. Det ska gå att skapa matriser med uttryck så som  \code{DenseMatrix.ofDim(3,7)(1.0,42,3.2,1.0,2.2,3)} tack vare ett kompanjonsobjekt med lämplig fabriksmetod som anropar den privata konstruktorn.  Om antalet element är för litet i förhållande till den angivna dimensionen så fyll på med nollor.
%
% \Subtask Överskugga metoderna equals och hashcode och ge \code{DenseMatrix} innehållslikhet i stället för referenslikhet.
%
% \Subtask Implementera egna innehålllikhetsmetoder med namnet \code{===} på \code{DenseMatrix} som är typsäker, d.v.s. bara tillåter innehållsjämförelse mellan täta matriser.
%
% \Subtask Läs om glesa matriser här: \href{https://sv.wikipedia.org/wiki/Gles_matris}{https://sv.wikipedia.org/wiki/Gles\_matris} och implementera \code{SparseMatrix} med ett privat attribut av typen \\ \code{mutable.Map[(Int, Int), Double]} som bara lagrar index som inte är noll.
%
% \Subtask Skapa ett \code{trait Matrix} som både \code{DenseMatrix} och \code{SparseMatrix} ärver, med lämpliga abstrakta och konkreta medlemmar. Implementera addition, subtraktion och multiplikation av täta och glesa matriser.
%
% %\Task \emph{Matriser med \jcode{ArrayList} i Java.} Om man i Java inte vet antalet element i matrisen från början kan man använda en lista av typen \jcode{ArrayList}, där varje element i sin tur innehåller en lista av typen\jcode{ArrayList}. Javas \jcode{ArrayList} är en generisk samling som motsvaras av Scalas \code{ArrayBuffer}. Generiska samlingar i Java kan endast innehålla referenstyper; vill man ha en primitiv typ, t.ex. \jcode{int}, behöver man packa in denna i en s.k. wrapper-klass, t.ex.  klassen \jcode{Integer}. Det finns en wrapper-klass för varje primitiv typ i Java. Matristypen för en heltalstyp i Java skrivs \jcode{ArrayList<ArrayList<Integer>>} där alltså \code{<T>} motsvarar Scalas hakparenteser \code{[T]} för typparametern T.
% %
% %
%
% \SOLUTION
%
% \TaskSolved \what
%      %%%TODO number  10 %%%starts with: \emph{Matriser med \jcode{Array%%%
%
% \SubtaskSolved  -- %%%TODO in task 10 %%%
% \QUESTEND

%!TEX encoding = UTF-8 Unicode
%!TEX root = ../compendium2.tex

\Lab{\LabWeekEIGHT}

\begin{Goals}
\item Kunna skapa och använda matriser med hjälp av en generisk datatyp.
\item Kunna iterera över alla element i en matris.
\item Träna på algoritmkonstruktion.
\item Träna på hantering av både oföränderliga och förändringsbara objekt.
\item Prova på att använda en avlusare \Eng{debugger} i en integrerad utvecklingsmiljö (IDE), t.ex. VS code.
\end{Goals}

\begin{Preparations}
\item Gör övning {\tt \ExeWeekEIGHT} i kapitel \ref{chapter:W08}, speciellt uppgift \ref{exe:matrices:labprep}.

\item Läs igenom hela laborationen och studera den givna koden\footnote{\url{https://github.com/lunduniversity/introprog/tree/master/workspace/w08_life}}.
\item Läs appendix \ref{appendix:debug} om avlusning \Eng{debugging}.
\item Hämta given kod via \href{https://github.com/lunduniversity/introprog/tree/master/workspace/}{kursen github-plats}.

\end{Preparations}


\begin{figure}[H]
  \includegraphics[width=0.8\textwidth]{../img/glider-blinker-block}

  \vspace{-2em}\caption{\label{lab:life:glider-blinker-block}Ett binärt, mörkt datauniversum av dimension $15  \times 20$. Cellkolonin innehåller tre cellgrupper: ett rymdskepp av typen \emph{glider}, en \emph{blinker} och ett \emph{block}.}
\end{figure}


\subsection{Bakgrund}

\emph{Game of Life} simulerar en koloni av encelliga organismer som lever, förökar sig och dör i en matris, enligt några enkla men väl valda regler som konstruerades av matematikern John Horton Conway på 1970-talet. Spelet går ut på att simulera flera generationer utifrån en startkonfiguration, även kallad \emph{cellkoloni}, där varje enskild cells överlevnad beror på dess omgivning. Spelet har inga medvetna spelare och om reglerna följs så kommer slutresultatet enbart bero på startkonfigurationen.

I \emph{Game of Life} består universum av en matris med celler som är antingen levande eller döda. Varje cell har 8 stycken \emph{grannar}, som utgörs av de närmsta omgivande cellerna vertikalt, horisontellt och diagonalt. Varje cells tillstånd i nästa generation bestäms av följande regler:
\begin{enumerate}[nolistsep]
    \item \textbf{Fortlevnad}. Om en levande cell har två eller tre grannar så lever den vidare.
    \item \textbf{Död}. Om en levande cell har färre än två eller mer än tre grannar så dör den av underpopulation respektive överpopulation.
    \item \textbf{Födelse}. Om cellen är död och har exakt tre grannar så föds den och dess tillstånd ändras till levande, annars fortsätter den vara död.
\end{enumerate}

Flera cellkolonier uppvisar ett ''levande'' beteende där cellmatrisen koloniseras på intressanta vis när en sekvens av generationer visualiseras. Detta är ett exempel på \emph{emergent} beteende där komplexa, självorganiserade strukturer kan uppstå ur enkla förutsättningar.

Läs mer om \emph{Game of Life} på Wikipedia:
\begin{itemize}[noitemsep,topsep=0pt]
    	\item \url{https://en.wikipedia.org/wiki/Conway's_Game_of_Life}
    	\item \url{https://sv.wikipedia.org/wiki/Game_of_Life}
\end{itemize}


\subsection{Obligatoriska krav}

Följande funktionella krav ska uppfyllas av ditt program:
\begin{itemize}[nosep, label={$\square$},]
\item Levande celler ska ha den vackra rosa\footnote{\url{https://www.dsek.se/aktiva/grafiskprofil/farg.php}} RGB-färgen \code{(242, 128, 161)}.
\item Döda celler ska vara svarta som rymden.
\item Detta mörka universum med binära dataceller ska ritas i ett rutnät bestående av smala, stilfulla linjer, så som visas i fig. \ref{lab:life:glider-blinker-block}.
\item Tangenttryckningar och musklick ska fungera enligt följande hjälptext, som ska skrivas ut då programmet startas:
\begin{CodeSmall}
  val help = """
    Welcome to GAME OF LIFE!

    Click on cell to toggle.
    Press ENTER for next generation.
    Press SPACE to toggle play/stop.
    Press R to create random life.
    Press BACKSPACE to clear life.
    Close window to exit.
  """
\end{CodeSmall}
Då \emph{play} aktiveras med blankstegstangenten ska en kontinuerlig simulering av universum fortgå där varje ny generation visualiseras med en lagom fördröjning emellan generationer, tills simuleringen stoppas, t.ex. genom tryck ånyo på blankstegstangenten. Vid varje \emph{Enter}-tryck visas \emph{en} efterkommande generation och ev. pågående simulering stoppas. Vid musklick på en cell ska livstillståndet växlas från levande till död eller vice versa. Ett tryck på R ska ge slumpmässigt liv. Ett tryck på backstegstangenten ska rendera alla universums cellers död.

\end{itemize}

\vspace{1em}\noindent Din kod ska utformas enligt dessa design-krav:
\begin{itemize}[nosep, label={$\square$}]
\item Alla klasser och singelobjekt ska ligga i paketet \code{life}.
\item Det ska finnas en oföränderlig case-klass \code{Life} som representerar ett celluniversum med hjälp av en \code{Matrix[Boolean]} från uppgift \ref{exe:matrices:labprep} i veckans övning.
\item Det ska finnas en klass \code{LifeWindow} som visualiserar en  instans av klassen \code{Life} i ett  \code{introprog.PixelWindow} så som i fig. \ref{lab:life:glider-blinker-block}.
\end{itemize}


\subsection{Valbara krav -- välj minst ett}

Du ska implementera minst ett (gärna flera) av dessa krav:
\begin{itemize}[nosep, label={$\square$}]
\item Cellerna ska färgläggas i olika färger i enlighet med reglerna för nästa generation. Fortlevnad ska fortfarande vara vackert rosa och fortvarig död svart. Följande färger föreslås men välj andra om du tycker det blir finare:
\begin{CodeSmall}
  val UnderPopulated = java.awt.Color.cyan  // en giftig färg
  val OverPopulated  = java.awt.Color.red   // rödklämd av trängsel
  val WillBeBorn     = new java.awt.Color(40, 0, 0)  // snart levande
\end{CodeSmall}
Ge dessutom \code{LifeWindow} en klassparameter \code{isMultiColor} som gör det möjligt att välja om det ska bli mångfärgade celler eller om det bara ska finnas rosa och svart som i grundkraven.

\item Om man trycker på \code{S} för \emph{Save} ska \code{introprog.Dialog.file("Save Life")} visas och, om användaren inte trycker \Button{Cancel}, det aktuella livet sparas med hjälp av\newline \code{introprog.IO.saveString} i en textfil via metoden \code{toString} i \code{Life}.

\item Om man trycker på \code{O} för \emph{Open} ska \code{introprog.Dialog.file("Open Life")} anropas och ett nytt universum läsas in från textfil enligt lämpligt format. Inläsningen ska ske med hjälp av \code{introprog.IO.loadString} och tolkas till en \code{Life}-instans av en metod i kompanjonsobjektet med detta huvud:
\begin{CodeSmall}
def fromString(s: String, rowDelim: String="\n", alive: Char='0'): Life
\end{CodeSmall}
Testa med filen \texttt{glider-gun.txt} som ska ha följande innehåll på de första 11 raderna och totalt 32 rader där alla rader efter elfte raden innehåller tomt liv:
\begin{REPLnonum}
> head -11 glider-gun.txt
------------------------------------------
-------------------------0----------------
-----------------------0-0----------------
-------------00------00------------00-----
------------0---0----00------------00-----
-00--------0-----0---00-------------------
-00--------0---0-00----0-0----------------
-----------0-----0-------0----------------
------------0---0-------------------------
-------------00---------------------------
------------------------------------------
\end{REPLnonum}
\item Universum ska vara cirkulärt, d.v.s grannen vid kanten finns på andra sidan genom att indexeringen börjar om \Eng{wrapped} enligt modulo-räkning. Inför en klassparameter \code{isWrapped} i \code{Life} och en variabel \code{wrapped: Boolean} i kompanjonsobjektet \code{Life} som styr om fabriksmetoderna skapar ett universum som är cirkulärt eller ej, så att du lätt kan konfigurera detta. \emph{Tips:} Du har stor nytta av att använda \code|java.lang.Math.floorMod| i \code{apply}-metoden i \code{Life}; metoden \code{floorMod} räknar på lämpligt sätt med negativa värden, se dokumentationen för \code{Math}-paketet i JDK8.

\item Läs om varianter till \code{Game of Life} på Wikipedia och implementera alternativa regler som görs valbara genom konfigurering via \code{args}-parametern i \code{main}.

\item Skapa en klass \code{LifeStatistics} som genom väldigt många simuleringar ska ta reda på sannolikheten att en slumpmässig cellkoloni efter $n$ generationer fortfarande utvecklas, respektive är helt dött. Ingen visualisering med \code{PixelWindow} ska ske; endast antalet celler som lever vid generation $n$ och antalet celler som ändrades sedan generation $n - 1$ behöver registreras.

\end{itemize}




\subsection{Tips och förslag}

\begin{enumerate}[leftmargin=*]
\item Här är ett förslag på hur du kan utforma klassen \code{Life}:
\scalainputlisting[basicstyle=\ttfamily\fontsize{10}{12}\selectfont]{../workspace/w08_life/Life.scala}
% \begin{CodeSmall}
% package life
%
% case class Life(cells: Matrix[Boolean]) {
%
%   /** Ger true om cellen på plats (row, col) är vid liv annars false.
%     * Ger false om indexeringen är utanför universums gränser.
%     */
%   def apply(row: Int, col: Int): Boolean = ???
%
%   /** Sätter status på cellen på plats (row, col) till value. */
%   def updated(row: Int, col: Int, value: Boolean): Life = ???
%
%   /** Växlar status på cellen på plats (row, col). */
%   def toggled(row: Int, col: Int): Life = ???
%
%   /** Räknar antalet levande grannar till cellen i (row, col).*/
%   def nbrOfNeighbours(row: Int, col: Int): Int = ???
%
%   /** Skapar en ny Life-instans med nästa generation av universum.
%     * Detta sker genom att applicera funktionen rule på cellerna.
%     */
%   def evolved(rule: (Int, Int, Life) => Boolean = Life.defaultRule):Life = {
%     var nextGeneration = Life.empty(cells.dim)
%     cells.foreachIndex { (r,c) =>
%       ???
%     }
%     nextGeneration
%   }
%
%   override def toString =
%     cells.data.map(_.map(if (_) '0' else '-').mkString).mkString("\n")
% }
%
% object Life {
%   /** Skapar ett universum med döda celler. */
%   def empty(dim: (Int, Int)): Life = ???
%
%   /** Skapar ett unviversum med slumpmässigt liv. */
%   def random(dim: (Int, Int)): Life = ???
%
%   /** Implementerar reglerna enligt Conways Game of Life. */
%   def defaultRule(row: Int, col: Int, current: Life): Boolean = ???
% }
% \end{CodeSmall}
Du har nytta av metoden \code{nbrOfNeighbours} när du ska implementera \code{defaultRule}. Vid implementation av \code{random} är metoden \code{foreachIndex} i \code{Matrix[T]} smidig att använda.
Om du som i förslaget ovan låter \code{evolved} ta uppdateringsregeln som en funktionsparameter blir det lättare att konfigurera vilka regler som ska gälla och därmed blir det även lättare att skapa varianter av \emph{Game of Life} genom att införa nya regler i kompanjonsobjektet (se en av de valfria uppgifterna med vidare hänvisning till Wikipedia).

\item Här är ett förslag på hur du kan utforma klassen \code{LifeWindow}:
\scalainputlisting[basicstyle=\ttfamily\fontsize{10}{12}\selectfont]{../workspace/w08_life/LifeWindow.scala}

\item \textbf{Testa dina dellösningar noga}. Du undviker svåra följdfel om du testar dina dellösningar noga innan du går vidare. Exempelvis är det vanligt att blanda ihop rader och kolumner, men du upptäcker detta om du testar alla relevanta funktioner med ett rektangulärt universum där antalet rader och kolumner är olika.

\item \textbf{Dra nytta av din IDE}. Det finns många användbara finesser i en integrerad utvecklingsmiljö \Eng{Integrated Development Environment (IDE)}, så som Microsoft VS Code\footnote{\url{https://code.visualstudio.com/}} med tillägget Metals\footnote{\url{https://scalameta.org/metals/docs/editors/vscode}}. Läsa på nätet om din IDE och lär dig om sådant du inte kände till som verkar användbart. Sök speciellt upp listan med kortkommandon\footnote{\url{https://code.visualstudio.com/docs/getstarted/keybindings}} och lär dig några valfria kortkommandon som kan hjälpa dig att snabba upp sådant du gör ofta. 
\item Studera dokumentationen om avlusaren (debuggern) in din IDE. \footnote{\url{https://scalameta.org/metals/docs/editors/vscode\#running-and-debugging-your-code}}  \footnote{\url{https://code.visualstudio.com/docs/editor/debugging}} 
\end{enumerate}


%!TEX encoding = UTF-8 Unicode

%!TEX root = ../compendium2.tex

\input{generated/w09-chaphead-generated.tex}
\clearpage\section{Teori}
%!TEX encoding = UTF-8 Unicode
%!TEX root = ../lect-w09.tex


\Subsection{\texttt{scala.collection}} %%%%%%%%%%%%%%%%%%%%%%%%%%%%%%%%%%%



% \begin{Slide}{Typparameter möjliggör generiska samlingar}\SlideFontSmall
%
% \begin{itemize}
%   \item Med \Emph{generisk} \Eng{generic} kod menar man att koden kan hantera data av \Alert{godtycklig} typ.
%   \item Funktioner och klasser kan, förutom vanliga parametrar, även ha \Emph{typparametrar} som skrivs i en \Alert{egen} parameterlista med \Alert{hakparenteser} i stället för vanliga parenteser.
%
%   \item En typparameter gör så att funktioner och datastrukturer blir \Emph{generiska}.
%
%   \item Exempel: Funktionerna \code{baklänges} 1--4 nedan är ordnade från specifik typ till mer generell typ.
%
% \begin{Code}
% def baklänges1(xs: Vector[Int]): Vector[Int] = xs.reverse
%
% def baklänges2[T](xs: Vector[T]): Vector[T] = xs.reverse
%
% def baklänges3(xs: Seq[T]): Vector[T] = xs.reverse.toVector
%
% def baklänges4(xs: Seq[T]): Seq[T] = xs.reverse  //reverse avgör samling
% \end{Code}
% \item Mer om typparametrar i w08.
% \end{itemize}
% \end{Slide}



\begin{Slide}{Hierarki av samlingstyper i \texttt{scala.collection} v2.13}

\begin{multicols}{2}
\begin{tikzpicture}[sibling distance=5.0em,->,>=stealth', inner sep=3pt, %scale=0.5,
  every node/.style = {shape=rectangle, draw, align=center,font=\small\ttfamily},
  class/.style = {fill=blue!20},
  trait/.style = {rounded corners, fill=red!20}]
  \node[trait] {Iterable}
      child { node[trait] {Seq} }
      child { node[trait] {Set} }
      child { node[trait] {Map} }
  ;
\end{tikzpicture}

\columnbreak

{\SlideFontTiny

\code{Iterable} har metoder som är implementerade med hjälp av: \\
\code{def foreach[U](f: Elem => U): Unit}\\
\code{def iterator: Iterator[A] }

}

\begin{itemize}\SlideFontTiny
\item[] \code{Seq}: ordnade i sekvens
\item[] \code{Set}: unika element
\item[] \code{Map}: par av (nyckel, värde)
\end{itemize}


\end{multicols}

{\SlideFontSmall Samlingen \Emph{\texttt{Vector}} är en \code{Seq} som är en \code{Iterable}. \\ \vspace{0.5em}%\pause
De konkreta samlingarna är uppdelade i dessa paket:\\
\code{scala.collection.immutable} \hfill där flera är \Emph{automatiskt} importerade\\
\code{scala.collection.mutable}  \hfill som \Alert{måste importeras} explicit\\%\pause
(undantag: primitiva förändringsbara \code{scala.Array} är automatiskt synlig)
}
\end{Slide}




\begin{Slide}{Metoden \texttt{iterator} ger en ''engångs-iterator''}\SlideFontSmall
Med \code{iterator} kan man iterera med \code{while}, men endast \Alert{en   gång}; sedan är iteratorn ''förbrukad''. (Men man kan be om en ny.) Används ''under huven'' i samlingsbiblioteket för att implementera andra metoder.
\begin{REPL}
scala> val xs = Vector(1,2,3,4)
val xs: Vector[Int] = Vector(1, 2, 3, 4)

scala> val it = xs.iterator
val it: Iterator[Int] = <iterator>

scala> while it.hasNext do print(it.next)
1234

scala> it.hasNext
val res0: Boolean = false

scala> it.next
java.util.NoSuchElementException: next on empty iterator
\end{REPL}
\Emph{Normalt} behöver man \Alert{inte} använda \code{iterator}: det finns oftast färdiga metoder som gör det man vill, till exempel \code{foreach}, \code{map}, \code{sum}, \code{min} etc.
\end{Slide}




% \ifkompendium
% \else
% \begin{Slide}{Hierarki av samlingar i scala.collection v2.12}\SlideFontTiny
% \includegraphics[width=0.6\textwidth]{../img/collection/collection-traits}\\
% %\noindent Läs mer om Scalas samlingar här: \\
% \url{https://docs.scala-lang.org/overviews/collections/overview.html}
% \end{Slide}
% \fi



\begin{Slide}{Mer specifika samlingstyper i \texttt{scala.collection}}
Det finns \Alert{mer specifika} \Emph{subtyper} av \code{Seq}, \code{Set} och \code{Map}:
\\ \vspace{1em}

\begin{tikzpicture}[sibling distance=5.8em,->,>=stealth', inner sep=3pt, %scale=0.5,
  every node/.style = {shape=rectangle, draw, align=center,font=\small\ttfamily},
  class/.style = {fill=blue!20},
  trait/.style = {rounded corners, fill=red!20}]
  \node[trait] {Iterable}
      child { node[trait, xshift=-2.4cm] {Seq}
        child { node[trait] {IndexedSeq} }
        child { node[trait] {LinearSeq} }
       }
      child { node[trait, yshift=-0.0cm] {Set}
        child { node[trait] {SortedSet} }
        child { node[trait] {BitSet} }
      }
      child { node[trait, xshift=1.0cm] {Map}
        child { node[trait] {SortedMap} }
    };
\end{tikzpicture}

\pause\vspace{0.5em}
\Emph{\texttt{Vector}} är en \Alert{\texttt{IndexedSeq}} medan
\Emph{\texttt{List}} är en \Alert{\texttt{LinearSeq}}.

\vspace{1em}{\SlideFontSmall
\href
{https://docs.scala-lang.org/overviews/collections-2.13/overview.html}
{docs.scala-lang.org/overviews/collections-2.13/overview.html}
}
\end{Slide}

\begin{Slide}{Några oföränderliga och förändringsbara sekvenssamlingar}\SlideFontSmall
\begin{tabular}{r l l}
\texttt{scala.collection.\Emph{immutable}.Seq.} & & \\
 & \code|IndexedSeq.| & \\
 & & \Emph{\texttt{Vector}} \\
 & & \Emph{\texttt{Range}} \\
 & \code|LinearSeq.| & \\
 & & \Emph{\texttt{List}} \\
   & & \Emph{\texttt{Queue}} \\

\texttt{scala.collection.\Alert{mutable}.Seq.} & & \\
 & \code|IndexedSeq.| & \\
 & & \Alert{\texttt{ArrayBuffer}} \\
 & & \Alert{\texttt{StringBuilder}} \\
 & \code|LinearSeq.| & \\
 & & \Alert{\texttt{ListBuffer}} \\
   & & \Alert{\texttt{Queue}} \\
\end{tabular}

{\SlideFontTiny Fördjupning: Studera samlingars prestanda-egenskaper här:\\ \href{https://docs.scala-lang.org/overviews/collections-2.13/performance-characteristics.html}{docs.scala-lang.org/overviews/collections-2.13/performance-characteristics.html}}
\end{Slide}



\begin{Slide}{Några användbara metoder på samlingar}\SlideFontTiny
\begin{tabular}{r r l}\hline
\texttt{\Emph{Iterable}}
  & \code|xs.size| & antal elementet \\
  & \code|xs.head| & första elementet \\
  & \code|xs.last| & sista elementet \\
  & \code|xs.take(n)| & ny samling med de första n elementet \\
  & \code|xs.drop(n)| & ny samling utan de första n elementet \\
  & \code|xs.foreach(f)| & gör \code|f| på alla element, returtyp \code|Unit|\\
  & \code|xs.map(f)| & gör \code|f| på alla element, ger ny samling \\
  & \code|xs.filter(p)| & ny samling med bara de element där p är sant\\
  & \code|xs.groupBy(f)| & ger en \code|Map| som grupperar värdena enligt f\\
  & \code|xs.mkString(",")| & en kommaseparerad sträng med alla element\\ 
  & \code|xs.zip(ys)| & ny samling med par (x, y); ''zippa ihop'' xs och ys \\
  & \code|xs.zipWithIndex| & ger en \code|Map| med par (x, index för x) \\
  & \code|xs.sliding(n)| & ny samling av samlingar genom glidande ''fönster''\\ \hline

\texttt{\Emph{Seq}}
  & \code|xs.length| & samma som \code|xs.size| \\
  & \code|xs :+ x| & ny samling med x sist efter xs \\
  & \code|x +: xs| & ny samling med x före xs \\ \hline

\end{tabular}
Prova fler samlingsmetoder ur snabbreferensen: ~~\url{http://cs.lth.se/quickref}

\vspace{0.25em}\Emph{Minnesregel} för \code{+:} och \code{:+  } \Alert{Colon on the collection side}\\Digga denna: \url{https://youtu.be/Lm9JWlEMHjo?si=sNdn_ZDaORlGr3lt}

\end{Slide}



% \ifkompendium\else

% \begin{Slide}{scala.collection.immutable}
% \includegraphics[width=0.67\textwidth]{../img/collection/collection-immutable}~~%
% \includegraphics[width=0.3\textwidth]{../img/collection/collection-legend}
% \end{Slide}


% \begin{Slide}{scala.collection.mutable}
% \includegraphics[width=1.05\textwidth]{../img/collection/collection-mutable}
% \end{Slide}

% \fi



% \begin{Slide}{\texttt{Vector} eller \texttt{List}?}\SlideFontTiny
% {\href{http://stackoverflow.com/questions/6928327/when-should-i-choose-vector-in-scala}{stackoverflow.com/questions/6928327/when-should-i-choose-vector-in-scala}}
%
% \begin{enumerate}
% \item If we only need to transform sequences by operations like map, filter, fold etc: basically it does not matter, we should program our algorithm generically and might even benefit from accepting parallel sequences. For sequential operations List is probably a bit faster. But you should benchmark it if you have to optimize.
%
% \item If we need a lot of random access and different updates, so we should use vector, list will be prohibitively slow.
%
% \item If we operate on lists in a classical functional way, building them by prepending and iterating by recursive decomposition: use list, vector will be slower by a factor 10-100 or more.
%
% \item If we have an performance critical algorithm that is basically imperative and does a lot of random access on a list, something like in place quick-sort: use an imperative data structure, e.g. ArrayBuffer, locally and copy your data from and to it.
% \end{enumerate}
% {\href{http://stackoverflow.com/questions/20612729/how-does-scalas-vector-work}{stackoverflow.com/questions/20612729/how-does-scalas-vector-work}}\\
% Mer om tids- och minneskomplexitet i fördjupningskursen och senare kurser.
% \end{Slide}








\Subsection{Repetition: sekvens}

\begin{Slide}{Repetition: Vad är en sekvens?}
\begin{itemize}
\item En sekvens är en \Emph{följd av element} som
  \begin{itemize}
   \item är \Alert{numrerade} (t.ex. från noll), och
   \item är av en viss \Alert{typ} (t.ex. heltal).
  \end{itemize}
  \pause
\item En sekvens kan innehålla \Alert{dubbletter}.
\item En sekvens kan vara \Alert{tom} och ha längden noll.
\item Exempel på en icke-tom sekvens med dubbletter:
\begin{REPLnonum}
scala> val xs = Vector(42, 0, 42, -9, 0, 13, 7)
val xs: Vector[Int] = Vector(42, 0, 42, -9, 0, 13, 7)
\end{REPLnonum}
\pause
\item \Emph{Indexering} ger ett element via dess ordningsnummer:
\begin{REPL}
scala> xs(2)
val res0: Int = 42

scala> xs.apply(2)
val res1: Int = 42
\end{REPL}
\end{itemize}
\end{Slide}


\begin{Slide}{En sträng är också en \texttt{IndexedSeq[Char]}}\SlideFontSmall
Det sker vid behov \Emph{implicit konvertering} från \code{String} till \code{IndexedSeq[Char]}.
\begin{REPLnonum}
scala> val x: IndexedSeq[Char] = "hej"
val x: IndexedSeq[Char] = hej
\end{REPLnonum}

Detta gör att \Alert{alla samlingsmetoder på \texttt{Seq} även funkar på strängar} och även flera andra smidiga strängmetoder erbjuds \Alert{utöver} de som finns i \href{https://www.scala-lang.org/api/current/scala/collection/StringOps.html}{\code{java.lang.String}} genom klassen \href{http://www.scala-lang.org/api/current/scala/collection/immutable/StringOps.html}{\code{StringOps}}.

\vspace{0.5em}
\begin{REPLnonum}
scala> "hej".  //tryck på TAB och se alla strängmetoder
JLine: do you wish to see all 248 possibilities (42 lines)?
\end{REPLnonum}
Detta är en stor fördel med Scala jämfört med många andra språk, som har strängar som inte kan allt som andra sekvenssamlingar kan.
\end{Slide}
  

\begin{Slide}{Konvertera mellan olika samlingstyper}
\begin{itemize}
\item För vanligt förekommande konverteringar finns metoderna \code{toVector}, \code{toList},  \code{toArray}, \code{toBuffer}, \code{toMap}, \code{toSeq}, \code{toIndexedSeq}, \code{toSet},  \code{toString}
\item Metoden \texttt{to} (ny från Scala 2.13) tar ett \Emph{kompanjonsobjekt} ur samlingsbiblioteket som argument och kan användas för konvertering till godtycklig samlingstyp.
\item Detta kräver kopiering om underliggande representation är olika och samlingen är förändringsbar.
\item Kan användas för att t.ex. konvertera mellan oföränderlig och förändringsbar samling:
\end{itemize}
\begin{REPLnonum}
scala> val ms = Set(1,2,3).to(collection.mutable.Set)
val ms: scala.collection.mutable.Set[Int] = HashSet(1, 2, 3)
\end{REPLnonum}
\end{Slide}


\Subsection{Mängd} %%%%%%%%%%%%%%%%%%%%%%%%%%%%%%%%%%%%%%%%%%%%%%%%%%%%%%


\begin{Slide}{Vad är en mängd?}\SlideFontSmall
\begin{itemize}
\item En \Emph{mängd} är en samling \Alert{unika} element av en viss \Alert{typ}.
\item En mängd kan alltså inte innehålla dubbletter:
\begin{REPLnonum}
scala> Set(1,1,2,2,3,3,4,4,5,5)
val res0: Set[Int] = HashSet(5, 1, 2, 3, 4)
\end{REPLnonum}
\pause
\item En mängd är \Alert{inte}  en sekvens: du kan inte utgå från att elementen ligger i någon viss ordning, t.ex. den ordning som de ges vid konstruktion; en mängd har ej längd, men en \Emph{storlek}; metoden \code{size} ger antalet element men metoden \code{length} saknas.
\item En mängd kan vara \Alert{tom} och har då storleken \code{0}.
\pause
\item Man kan gå igenom element i \Emph{någon} ordning (exakt vilken är ej def.), med till exempel \code{xs.map(f)} eller \code{for (x <- xs) yield f(x)}
\pause
\item Det går \Alert{inte} att indexera i en mängd med \code{apply}, som i stället ger \Emph{innehållstest}: \code{Set(1,2,3).apply(3) == true}
\item En mängd \code{Set[T]} med element av typen \code{T} kan således ses som ett \Emph{predikat för innehållstest}: alltså en funktion \code{T => Boolean} som är \code{true} om elementet finns annars \code{false}
\end{itemize}
\end{Slide}


\begin{Slide}{Oföränderlig mängd}
\setlength{\leftmargini}{1em}
\begin{itemize}
\item \Emph{Skapa}:
\begin{REPLnonum}
scala> var xs = Set("gurka", "tomat", "banan", "pingvin")
\end{REPLnonum}

\item \Emph{Läsa}: avgöra medlemskap
\begin{REPLnonum}
scala> xs("gurka")
val res1: Boolean = true
\end{REPLnonum}

\item \Emph{Uppdatera}: lägg till element (händer inget om redan finns)
\begin{REPLnonum}
scala> xs = xs + "jordekorre" // en ny, delvis förändrad
\end{REPLnonum}

\item \Emph{Ta bort}: (om finns, annars händer inget)
\begin{REPLnonum}
scala> xs = xs - "gurka" // en ny, delvis förändrad
\end{REPLnonum}
\end{itemize}
{\SlideFontTiny\code{SLUT} = Skapa, Läsa, Uppdatera, Ta bort \hfill\code{CRUD} = Create, Read, Update, Delete}
\end{Slide}


\begin{Slide}{Mysteriet med de försvunna elementen}
Vad händer här?
\begin{REPLnonum}
scala> val xs1 = Vector(1,2,3,4,5,6)
scala> xs1.map(_ % 2).count(_ == 0)
val res0: Int = 3                          // antalet jämna tal
scala> val xs2 = Set(1,2,3,4,5,6)
scala> xs2.map(_ % 2).count(_ == 0)
val res1: Int = 1                          // varför?
\end{REPLnonum}
\pause
Mängdegenskaper ger att \code{xs2.map(_ % 2) == Set(0, 1)}\\
Fundera alltid noga på om du \Alert{riskerar att förlora duplikat} som du egentligen hade velat behålla!\\
\pause
Använd \code{toSeq} på mängd om du behöver sekvensegenskaper:
\begin{REPLnonum}
scala> xs2.toSeq.map(_ % 2).count(_ == 0)
val res1: Int = 3         // med toSeq blir det som vi ville
\end{REPLnonum}

\end{Slide}
  
  




\begin{Slide}{Förändringsbar mängd}\SlideFontSmall
Med en \Alert{förändringsbar} mängd kan man stegvis utöka på plats.
\begin{REPL}
scala> val mängd = scala.collection.mutable.Set.empty[Int]

scala> for i <- 1 to 1_000_000 do mängd.addOne(i)

scala> mängd.contains(-1)   // samma som mängd(-1) eller mängd.apply(-1)
\end{REPL}
En \Emph{mängd} är \Alert{snabb} på att avgöra om ett element \Alert{finns eller inte} i mängden. Ingen linjärsökning krävs eftersom den smarta implementationen av datastrukturen medger snabb uppslagning \Eng{lookup} av ett element.
\pause
\\\vspace{0.5em}Men i en sekvens krävs linjärsökning vid innehållstest:
\begin{REPL}
scala> val sekvens = (1 to 1_000_000).toVector

scala> sekvens.contains(-1)   // kräver linjärsökning ända till slutet
\end{REPL}
\pause\SlideFontTiny Övning: Testa själv att mäta tidsskillnaden med hjälp av:
\begin{Code}
def nanos(b: => Unit) = { val t0 = System.nanoTime; b; System.nanoTime - t0 }
\end{Code}

\end{Slide}


\begin{Slide}{Speciella metoder på förändringsbar mängd}\SlideFontSmall
Förändringsbara mängder har metoder som ändrar på plats:
\begin{REPLsmall}
scala> val s = scala.collection.mutable.Set.empty[Int]

scala> s.addOne(1)     // finns även under namnet += om du gillar operator-notion
val res0: scala.collection.mutable.Set[Int] = HashSet(1)

scala> s.addOne(2).addOne(3).addOne(3).addOne(42) // addOne returnerar this
val res1: scala.collection.mutable.Set[Int] = HashSet(1, 2, 3, 42)

scala> res0.eq(res1)  // samma instans av mutable.Set (ingen ny har skapats)
val res2: Boolean = true

scala> s.addAll(Vector(3, 4, 5))  // finns även ++= om du gillar operator-notion
val res3: scala.collection.mutable.Set[Int] = HashSet(1, 2, 3, 4, 5, 42)

scala> s.subtractOne(1).subtractAll(List(1,2,3))  // finns även -= och --=
val res4: scala.collection.mutable.Set[Int] = HashSet(4, 5, 42)

scala> s.filterInPlace(_ > 4)
val res5: scala.collection.mutable.Set[Int] = HashSet(5, 42)

\end{REPLsmall}
\code{addOne}, \code{addAll}, \code{subtractOne}, \code{subtractAll}, \code{filterInPlace} returnerar \code{this} så du kan ändra på plats med kedjade anrop med punktnotation.
\end{Slide}

  




\Subsection{Nyckel-värde-tabell} %%%%%%%%%%%%%%%%%%%%%%%%%%%%%%%%%%%%%%%%


\begin{Slide}{Vad är en nyckel-värde-tabell?}\SlideFontSmall
\begin{itemize}
\item En \Emph{nyckel-värde-tabell} är en samling element som är \Alert{par} med:\\
en \Emph{nyckel} av någon typ \code{K} och ett \Emph{värde} av någon typ \code{V}.
\item En sådan tabell kan skapas ur en sekvens av par \code{(k, v)}\\
där \code{k} är en nyckel och \code{v} är ett värde:
\begin{REPL}
scala> val ålder = Map("Björn" -> 42, "Sandra" -> 35, "Kim" -> 19)
val ålder: Map[String, Int] = Map(Björn -> 42, Sandra -> 35, Kim -> 19)
\end{REPL}
\item Tabellens nycklar utgör en mängd som ges av metoden \code{keySet};\\
nycklarna är \Alert{unika}.
\item Elementen utgör \Alert{inte en sekvens} och har ingen speciell ordning;
\\en nyckel-värde-tabell har ej längd, men en \Emph{storlek};\\metoden \code{size} ger antalet element.
\pause
\item En tabell kan ses som en uppslagsfunktion \Eng{dictionary}:\\alltså en funktion \code{K => V} som ger ett värde givet en nyckel.
\end{itemize}
\end{Slide}


\begin{Slide}{Den fantastiska nyckel-värde-tabellen \texttt{Map}}\SlideFontSmall
\begin{itemize}
\item En \Emph{nyckel-värde-tabell} \Eng{key-value table} är en slags generaliserad vektor där man kan ''indexera'' med godtycklig typ.

\item Kallas även \href{https://sv.wikipedia.org/wiki/Hashtabell}{\Emph{hashtabell}} \Eng{hash table}, \Emph{lexikon} \Eng{Dictionary} eller \Emph{mapp} \Eng{Map} (det blir lätt sammanblandning med metoden \code{map}).

\item Om man vet nyckeln kan man slå upp värdet \Alert{snabbt}, på liknande sätt som indexering sker snabbt i en vektor givet heltalsindex.

\item Denna datastruktur är \Alert{mycket användbar} och fungerar som en slags databas i kombination med filtrering, registrering, etc.
\end{itemize}
\end{Slide}


\begin{Slide}{Oföränderlig nyckel-värde-tabell}
\setlength{\leftmargini}{1em}
\begin{itemize}
\item \Emph{Skapa}: ge par till metoden \code{apply} 
\begin{REPLsmall}
scala> var födelse = Map("C" -> 1972,  "C++" -> 1983, "C#" -> 2000,
  "Scala" -> 2004, "Java" -> 1995, "Javascript" -> 1995, "Python" -> 1991)
\end{REPLsmall}

\item \Emph{Läsa}: slå upp ett värde med hjälp av en nyckel
\begin{REPLsmall}
scala> val year = födelse.apply("Scala")
val year: Int = 2004
\end{REPLsmall}

\item \Emph{Uppdatera}: lägga till ett par, ersätta ett par
\begin{REPLsmall}
scala> födelse = födelse + ("Kotlin" -> 2011)
födelse: Map[String, Int] = HashMap(Scala -> 2004, C# -> 2000, Python -> 1991, 
Javascript -> 1995, C -> 1972, C++ -> 1983, Kotlin -> 2011, Java -> 1995)
\end{REPLsmall}

\item \Emph{Ta bort} ett par via nyckeln (om finns, annars händer inget)
\begin{REPLsmall}
scala> födelse = födelse - "Python"
födelse: Map[String, Int] = HashMap(Scala -> 2004, C# -> 2000, 
Javascript -> 1995, C -> 1972, C++ -> 1983, Kotlin -> 2011, Java -> 1995)
\end{REPLsmall}
\end{itemize}
\end{Slide}


\begin{Slide}{Fler exempel nyckel-värde-tabell}\SlideFontSmall
Några ofta förekommande metoder på tabeller:
\begin{itemize}
\item \code{xs.keySet} ger en mängd av alla nycklar
\item \code{xs.map(f)} kör funktionen f på alla par (key, value) i \Alert{någon} ordning
\item \code{xs.map((k, v) => k -> f(v))} kör funktionen f på alla \Emph{värden}
\end{itemize}
\begin{REPLsmall}
scala> val färg = Map("gurka" -> "grön", "tomat"->"röd", "aubergine"->"lila")
val färg: Map[String, String] = 
  Map(gurka -> grön, tomat -> röd, aubergine -> lila)

scala> färg("gurka")
val res0: String = grön

scala> färg.keySet
val res1: Set[String] = Set(gurka, tomat, aubergine)

scala> val ärGrönSak = färg.map((k,v) => (k, v == "grön"))
val ärGrönSak: Map[String,Boolean] = 
  Map(gurka -> true, tomat -> false, aubergine -> false)

scala> val baklängesFärg = färg.map((k, v) => k -> v.reverse)
val baklängesFärg: Map[String,String] = 
  Map(gurka -> nörg, tomat -> dör, aubergine -> alil)
\end{REPLsmall}

\end{Slide}

\begin{Slide}{Från sekvens av par till tabell}
\begin{REPL}
scala> val xs = Vector(("Kim",42), ("Pam", 42), ("Kim", 50), ("Pam", 50))
val xs: Vector[(String, Int)] = 
  Vector((Kim,42), (Pam,42), (Kim,50), (Pam,50))

scala> xs.toMap
val res0: Map[String, Int] = 
  Map(Kim -> 50, Pam -> 50) // inga dublettnycklar

scala> val grupperaEfterNamn = xs.groupBy(_._1)
grupperaEfterNamn: Map[String,Vector[(String, Int)]] =
  Map(Kim -> Vector((Kim,42), (Kim,50)), Pam -> Vector((Pam,42), (Pam,50)))

scala> val grupperaEfterÅlder = xs.groupBy(_._2)
grupperaEfterÅlder: Map[Int,Vector[(String, Int)]] =
  Map(50 -> Vector((Kim,50), (Pam,50)), 42 -> Vector((Kim,42), (Pam,42)))
\end{REPL}
\end{Slide}
  
  
\begin{Slide}{Övning: Implementera en \texttt{Multimap}}
\begin{itemize}
\item Om du lägger till ett värde i en \emph{vanlig} \code{Map} så ersätts  värdet:
\begin{REPLsmall}
scala> val m = Map(1 -> 2, 1 -> 3, 2 -> 1, 2 -> 2) 
val m: Map[Int, Int] = Map(1 -> 3, 2 -> 2)   //senaste värdet gäller
\end{REPLsmall}
...men ibland vill vi i stället lagra alla tillagda värden.
\item En \Emph{multimap} är en speciell nyckel-värde-tabell där värdena utgör en samling (ofta en mängd).
\item En multimap samlar alla värden som har samma nyckel.
\begin{REPL}
  scala> val mm = Multimap(1 -> 2, 1 -> 3, 2 -> 1, 2 -> 2) 
  val mm: Multimap[Int, Int] = Multimap(1 -> Set(2, 3), 2 -> Set(1, 2))
\end{REPL}
\end{itemize}
Övning: Implementera en multimap som fungerar som ovan, med hjälp av en case-klass med attributet \code{toMap} som är en oföränderlig nyckel-värde-tabell där värdena är en mängd. \\Tips: Använd~\code{groupBy}
\end{Slide}

\begin{Slide}{Lösning: \texttt{Multimap}}
\begin{CodeSmall}
case class Multimap[K, V] private (toMap: Map[K,Set[V]]):
  def apply(k: K): Set[V] = toMap(k)
  
  def +(kv: (K, V)): Multimap[K, V] = kv match 
    case (k, v) if toMap.isDefinedAt(k) => Multimap(toMap.updated(k, toMap(k) + v))
    case (k, v) => Multimap(toMap + (k -> Set(v)))
  
  override def toString = toMap.mkString("Multimap(",", ",")")

object Multimap:
  def apply[K, V](kvs: (K,V)*): Multimap[K, V] = 
    new Multimap(kvs.groupBy(_._1).map((k,xs) => k -> xs.map(_._2).toSet))
\end{CodeSmall}
\end{Slide}


\begin{Slide}{Speciella metoder på förändringsbar tabell}\SlideFontSmall
Både \code{Set} och \code{Map} finns i \Alert{förändringsbara} varianter med extra metoder för uppdatering av innehållet ''på plats'' utan att nya samlingar skapas.
\begin{REPLsmall}
scala> import scala.collection.mutable
scala> val ms = mutable.Set.empty[Int]
val ms: scala.collection.mutable.Set[Int] = HashSet()

scala> ms += 42  // metoden += gör samma som addOne
val res0: scala.collection.mutable.Set[Int] = HashSet(42)

scala> ms ++= Seq(1, 2, 3, 1, 2, 3)  // metoden ++= gör samma som addAll 
val res1: scala.collection.mutable.Set[Int] = HashSet(1, 2, 3, 42)

scala> val mm = mutable.Map.empty[String, String]

scala> mm.addOne("hej" -> "svejs").addAll(Seq("abra" -> "kadabra", "ada" -> "lovelace"))
val res2: scala.collection.mutable.Map[String, String] = 
  HashMap(hej -> svejs, abra -> kadabra, ada -> lovelace

scala> mm("abra")
val res3: String = kadabra
\end{REPLsmall}
Metoden \code{++=} gör samma som addAll, används gärna med operator-notation:\\
\code{mm ++= Seq("hej" -> "san", "abra" -> "kada", "bra" -> "scala")}
\end{Slide}
  

\Subsection{Tips inför veckans uppgifter}



\begin{Slide}{Övning: Förändringsbar lokalt, returnera oföränderlig}
\SlideFontSmall
Om du vill implementera en imperativ algoritm med en föränderlig samling:\\
Gör gärna detta \Alert{lokalt} i en \Alert{förändringsbar} samling och returnera sedan en \Emph{oföränderlig} samling, genom att köra t.ex. \code{toSet} på en mängd, eller \code{toMap} på en hashtabell, eller \code{toVector} på en \code{ArrayBuffer} eller \code{Array}.\\~\\
Exempel där lösningen har nytta av lokal förändring på plats:
\begin{Code}
def kastaTärningTillsAllaUtfallUtomEtt(sidor: Int = 6): (Int, Set[Int]) = ???
\end{Code}
\begin{REPL}
scala> kastaTärningTillsAllaUtfallUtomEtt()
val res0: (Int, Set[Int]) = (13,HashSet(5, 1, 6, 2, 3))
\end{REPL}
\end{Slide}


\begin{Slide}{Övning: Förändringsbar lokalt, returnera oföränderlig}
\SlideFontSmall
Om du vill implementera en imperativ algoritm med en föränderlig samling:\\
Gör gärna detta \Alert{lokalt} i en \Alert{förändringsbar} samling och returnera sedan en \Emph{oföränderlig} samling, genom att köra t.ex. \code{toSet} på en mängd, eller \code{toMap} på en hashtabell, eller \code{toVector} på en \code{ArrayBuffer} eller \code{Array}.
\begin{Code}
def kastaTärningTillsAllaUtfallUtomEtt(sidor: Int = 6): (Int, Set[Int]) = 
  /* 
    låt s vara en tom förändringsbar heltalsmängd
    låt n vara noll
    så länge mängden s är mindre än sidor - 1 gör:
      lägg till ett nytt tärningskast i s
      uppdatera n så att vi räknar hur många slumptal som dragits
  */
  (n, s.toSet)   // notera toSet som ger oföränderlig mängd
\end{Code}
\begin{REPL}
scala> kastaTärningTillsAllaUtfallUtomEtt()
val res0: (Int, Set[Int]) = (13,HashSet(5, 1, 6, 2, 3))
\end{REPL}
\end{Slide}


\begin{Slide}{Lösning: Förändringsbar lokalt, returnera oföränderlig}
\SlideFontSmall
Om du vill implementera en imperativ algoritm med en föränderlig samling:\\
Gör gärna detta \Alert{lokalt} i en \Alert{förändringsbar} samling och returnera sedan en \Emph{oföränderlig} samling, genom att köra t.ex. \code{toSet} på en mängd, eller \code{toMap} på en hashtabell, eller \code{toVector} på en \code{ArrayBuffer} eller \code{Array}.
\begin{Code}
def kastaTärningTillsAllaUtfallUtomEtt(sidor: Int = 6): (Int, Set[Int]) = 
  val s = scala.collection.mutable.Set.empty[Int] //förändringsbar lokalt
  var n = 0
  while s.size < sidor - 1 do
    s += util.Random.nextInt(sidor) + 1 
    n += 1
  (n, s.toSet)   // notera toSet som ger oföränderlig mängd
\end{Code}
\begin{REPL}
scala> kastaTärningTillsAllaUtfallUtomEtt()
val res0: (Int, Set[Int]) = (13,HashSet(5, 1, 6, 2, 3))
\end{REPL}
I veckans uppgifter används detta i en s.k. \Emph{builder}: Först bygga upp en förändringsbar struktur i \code{FreqMapBuilder} steg för steg,  
och sedan, då alla tillägg är gjorda, övergå till oföränderlig struktur \code{Map[String, Int]}. 
\end{Slide}


\begin{Slide}{Metoden \texttt{sliding}}\SlideFontSmall
Metoden \code{sliding(n)} skapar med ett ''glidande fönster'' en sekvens av
delsekvenser av längd \code{n} genom att ''svepa fönstret'' från början till slut:
\begin{REPL}
scala> val xs = "fem myror är fler än fyra elefanter".split(' ').toVector
val xs: Vector[String] = Vector(fem, myror, är, fler, än, fyra, elefanter)

scala> xs.sliding(2).toVector
val res0: Vector[Vector[String]] =
  Vector(Vector(fem, myror), Vector(myror, är), Vector(är, fler),
      Vector(fler, än), Vector(än, fyra), Vector(fyra, elefanter))

scala> xs.sliding(3).toVector
val res1: Vector[Vector[String]] =
  Vector(Vector(fem, myror, är), Vector(myror, är, fler),
    Vector(är, fler, än), Vector(fler, än, fyra),
      Vector(än, fyra, elefanter))
\end{REPL}
Denna metod har du nytta av på veckans laboration!
\\(se fler exempel på övning)
\end{Slide}
  
  

\begin{Slide}{Metoderna zipWithIndex, groupBy}
\vspace{-0.5em}
\begin{REPL}
scala> val kort = Vector("Knekt", "Dam", "Kung", "Äss")

scala> val kortIndex = kort.zipWithIndex.toMap
kortIndex: Map[String,Int] = Map(Knekt -> 0, Dam -> 1, Kung -> 2, Äss -> 3)

scala> kortIndex("Kung") > kortIndex("Knekt")
res0: Boolean = true

scala> kortIndex.map(p => p._1 -> (p._2 + 11))

scala> val tärningskast = Vector(1,2,3,4,5,6,2,4,6)

scala> val grupperaStörreÄnFyra = tärningskast.groupBy(_ > 4)
grupperaStörreÄnFyra: Map[Boolean,Vector[Int]] =
  Map(false -> Vector(1, 2, 3, 4, 2, 4), true -> Vector(5, 6, 6))

scala> val grupperaLika = tärningskast.groupBy(x => x)
grupperaLika: Map[Int,Vector[Int]] = Map(5 -> Vector(5), 1 -> Vector(1),
  6 -> Vector(6, 6), 2 -> Vector(2, 2), 3 -> Vector(3), 4 -> Vector(4, 4))

scala> val frekvens = tärningskast.groupBy(x => x).map((k,v) => k -> v.size)
frekvens: Map[Int,Int] = Map(5 -> 1, 1 -> 1, 6 -> 2, 2 -> 2, 3 -> 1, 4 -> 2)

\end{REPL}
\end{Slide}

\begin{Slide}{Fler användbara samlingsmetoder}
Exempel att öva på: räkna bokstäver i ord.  \\
Undersök vad som händer i REPL:
\begin{Code}[basicstyle=\SlideFontSize{9}{13}\ttfamily]
val ord = "sex laxar i en laxask sju sjösjuka sjömän"
val uppdelad = ord.split(' ').toVector
val ordlängd = uppdelad.map(_.length)
val ordlängdMap = uppdelad.map(s => (s, s.size)).toMap
val grupperaEfterFörstaBokstav = uppdelad.groupBy(s => s(0))
val bokstäver = ord.toVector.filter(_ != ' ')
val antalX = bokstäver.count(_ == 'x')
val grupperade = bokstäver.groupBy(ch => ch)
val antal = grupperade.map(p => p._1 -> p._2.size)
//samma som ovan men utnyttjar "parameter untupling":
val antal2 = grupperade.map((k,v) => k -> v.size) 
val sorterat = antal.toVector.sortBy(_._2)
val vanligast = antal.maxBy(_._2)
\end{Code}
%https://dotty.epfl.ch/docs/reference/other-new-features/parameter-untupling.html
\end{Slide}
    
    

%!TEX encoding = UTF-8 Unicode
%!TEX root = ../lect-w09.tex


\Subsection{Serialisering och deserialisering}

\begin{Slide}{Serialisering och deserialisering}
\begin{itemize}
  \item Att \Emph{serialisera} innebär att \Alert{koda objekt} i minnet till en avkodningsbar \Alert{sekvens av symboler}, som kan lagras t.ex. i en fil på din hårddisk.
  \item Att \Emph{de-serialisera} innebär att \Alert{avkoda en sekvens av symboler}, t.ex. från en fil, och \Alert{återskapa objekt} i minnet.
\end{itemize}
\end{Slide}


\begin{Slide}{Läsa text från fil och URL}\SlideFontSmall
I paketet \code{scala.io} finns singelobjektet \code{Source} med metoderna \code{fromFile} och \code{fromUrl} för läsning från fil resp. från  URL, alltså Universal Resource Locator, som börjar t.ex. med \texttt{http://}
\begin{Code}
def läsFrånFil(filnamn: String): String = 
  val s = scala.io.Source.fromFile(filnamn)
  try s.mkString finally s.close // säkerställ stängning även vid krasch

def läsRaderFrånFil(filnamn: String): Vector[String] =
  val s = scala.io.Source.fromFile(filnamn)
  try s.getLines.toVector finally s.close 

def läsFrånWebbsida(url: String): String = 
  val s = scala.io.Source.fromURL(url)
  try s.mkString finally s.close

def läsRaderWebbsida(url: String, kodning: String = "UTF-8"): Vector[String] =
  val s = scala.io.Source.fromURL(url, kodning) // läs med given teckenkodning
  try s.getLines.toVector finally s.close 

\end{Code}
{\SlideFontTiny Se vidare veckans övning. Exempel på annan teckenkodning: \code{"ISO-8859-1"} }
\end{Slide}


\begin{Slide}{Serialisering i modulen \texttt{introprog.IO}}
\begin{itemize}
\item I kursens kodbibliotek \code{introprog} finns ett singelobjekt \code{IO} som samlar smidiga funktioner för serialisering och de-serialisering. 
\item Se api-dokumentation här: \\ \url{https://cs.lth.se/pgk/api/} \\ Sök på IO och klicka på singelobjektet.
\item Se koden här:\\
\url{https://github.com/lunduniversity/introprog-scalalib/blob/master/src/main/scala/introprog/IO.scala}
\item Om du vill får du gärna använda \code{introprog.IO} istället för \code{scala.io.Source} på labben.  
\end{itemize}
\end{Slide}


%\input{generated/w09-chaphead-generated.tex}
%!TEX encoding = UTF-8 Unicode
%!TEX root = ../exercises.tex

\ifPreSolution


\Exercise{\ExeWeekNINE}\label{exe:W09}

\begin{Goals}
\input{modules/w09-setmap-exercise-goals.tex}
\end{Goals}

\begin{Preparations}
\item \StudyTheory{09}
\end{Preparations}

\else

\ExerciseSolution{\ExeWeekNINE}

\fi



\BasicTasks %%%%%%%%%%%%%%%%




\WHAT{Para ihop begrepp med beskrivning.}

\QUESTBEGIN

\Task \what

\vspace{1em}\noindent Koppla varje begrepp med den (förenklade) beskrivning som passar bäst:

\begin{ConceptConnections}
\input{generated/quiz-w09-concepts-taskrows-generated.tex}
\end{ConceptConnections}

\SOLUTION

\TaskSolved \what

\begin{ConceptConnections}
\input{generated/quiz-w09-concepts-solurows-generated.tex}
\end{ConceptConnections}

\QUESTEND



\WHAT{Vad är en mängd?}
\QUESTBEGIN

\Task \what~ Förklara vad som händer nedan. Varför hamnar elementen i en ''konstig'' ordning? Varför ''försvinner'' det element?

\begin{REPL}
scala> val xs = Vector(1,2,3,1,2,3,4,5,7).toSet
xs: scala.collection.immutable.Set[Int] = Set(5, 1, 2, 7, 3, 4)
scala> xs.foreach(print)
512734
\end{REPL}

\SOLUTION

\TaskSolved \what~En mängd är en samling som snabbt kan ge svaret på frågan om ett visst element ingår i samlingen eller ej. Elementen i en mängd är unika. Tilläg av redan existerande element ignoreras. En mängd är inte en  sekvens, eftersom traversering med t.ex. \code{map} eller \code{foreach} inte (nödvändigtvis) sker i den ordning som elementen gavs när mängden konstruerades eller uppdaterades.

\QUESTEND


\WHAT{Använda mängder.}

\QUESTBEGIN

\Task \what

\vspace{1em}\noindent Para ihop varje uttryck till vänster med ett uttryck till höger som har samma värde:

\begin{ConceptConnections}
\input{generated/quiz-w09-setops-taskrows-generated.tex}
\end{ConceptConnections}

\SOLUTION

\TaskSolved \what

\begin{ConceptConnections}
\input{generated/quiz-w09-setops-solurows-generated.tex}
\end{ConceptConnections}

\QUESTEND


\WHAT{Räkna unika ord med hjälp av en mängd.}

\QUESTBEGIN

\Task \what~På veckans laboration ska vi göra automatisk språkbehandling av långa texter som vi delar upp i ord. Med metoden \code{s.split(' ').toVector} kan du dela upp en sträng \code{s} i en sekvens av ord, där \code{s} blivit uppdelad i många strängar vid varje blanktecken och alla blanktecken är borttagna.

\Subtask Använd metoderna \code{split} och \code{toSet} för skapa ett uttryck som beräknar hur många unika ord det finns i strängen \code{hej} nedan:
\begin{REPLnonum}
scala> val hej = "hej hej hemskt mycket hej"
\end{REPLnonum}

\Subtask Mängder är snabba på att kolla om ett element finns i mängden men du kan inte förvänta dig att elementen finns i någon viss ordning. Det finns en sekvenssamlingsmetod som skapar en sekvens med unika element ur en sekvens och behåller den ursprungliga ordningen. Vad heter metoden? \\\emph{Tips:} Leta i snabbreferensen eller sök på nätet. Metoden fungerar på alla samlingar som är av typen \code{Seq} och har ett namn som börjar med bokstäverna \code{di}.

\SOLUTION

\TaskSolved \what~

\SubtaskSolved
\begin{REPL}
scala> val hej = "hej hej hemskt mycket hej"
scala> val n = hej.split(' ').toSet.size
n: Int = 3
\end{REPL}

\SubtaskSolved Metoden \code{distinct} returnerar en sekvens med unika element och bibehållen ursprunglig ordning.

\QUESTEND




\WHAT{Skapa 2-tupler med metoden \code{->} som kan uttalas ''mappas till''.}

\QUESTBEGIN

\Task \what~Vi har tidigare sett hur två olika värden kan samlas i en 2-tupel, till exempel \code{(0, true)}. Par kan även skapas med hjälp av metoden \code{->} enligt nedan. Testa detta i REPL:
\begin{REPL}
scala> ("Skåne", "Lund")          // ett strängpar med vanlig 2-tupel
scala> "Skåne" -> "Lund"           // operatornotation med ->
scala> "Skåne".->("Lund")         // punktnotation med -> (inte alls vanligt)
\end{REPL}
Metoden \code{->} fungerar med alla typer och är en fabriksmetod för par. Metodnamnet liknar en högerpil och illustrerar en mappning från första till andra värdet.

\Subtask Fungerar det på par skapade med \code{->} att använda metoderna \code{_1} och \code{_2}?


\Subtask Deklarera en variabel \code{val huvudstad: Vector[(String, String)]} som innehåller mappningar mellan geografiska områden och deras huvudstäder enligt tabellen nedan.

\begin{table}[H]
  \renewcommand{\arraystretch}{1.2}
  \begin{tabular}{|l|l|}\hline
  Sverige & Stockholm \\\hline
  Danmark & Köpenhamn \\\hline
  Grönland & Nuuk \\\hline
  Skåne & Lund \\\hline
  \end{tabular}
\end{table}

\Subtask Skriv ett uttryck som plockar fram \code{"Lund"} ur \code{huvudstad}.

\SOLUTION


\TaskSolved \what

\SubtaskSolved Ja, fabriksmetoden returnerar ett helt vanligt par:
\begin{REPLnonum}
scala> val härBorJag = "Skåne" -> "Lund"
val härBorJag: (String, String) = (Skåne,Lund)

scala> härBorJag._1
val res0: String = Skåne

scala> härBorJag._2
val res1: String = Lund
\end{REPLnonum}


\SubtaskSolved

\begin{Code}
val huvudstad = Vector(
  "Sverige"  -> "Stockholm",
  "Danmark"  -> "Köpenhamn",
  "Grönland" -> "Nuuk",
  "Skåne"    -> "Lund"
)
\end{Code}

\SubtaskSolved
\begin{REPL}
scala> huvudstad(3)._2
val res2: String = Lund
\end{REPL}

\QUESTEND



\WHAT{Linjärsöka efter nyckel i sekvens av mappningar.}

\QUESTBEGIN

\Task \what~

\Subtask Implementera funktionen \code{lookupIndex} nedan med hjälp av samlingsmetoden \code{indexWhere} så att linjärsökning sker efter index för ett par i sekvensen där \code{key} finns på första platsen i paret.

\begin{Code}
def lookupIndex(xs: Vector[(String, String)])(key: String): Int = ???
\end{Code}

\Subtask Testa din funktion i REPL genom att slå upp index för Skånes huvudstad i sekvensen \code{huvudstad} från föregående uppgift.

\SOLUTION

\TaskSolved \what~

\SubtaskSolved
\begin{Code}
def lookupIndex(xs: Vector[(String, String)])(key: String): Int =
  xs.indexWhere(_._1 == key)
\end{Code}

\SubtaskSolved
\begin{REPL}
scala> val i = lookupIndex(huvudstad)("Skåne")
val i: Int = 3

scala> huvudstad(i)._2
val res2: String = Lund
\end{REPL}

\noindent Eller med funktioner som återanvändbara dellösningar:
\begin{REPL}
scala> val indexOf = lookupIndex(huvudstad) _

scala> def capital(key: String) = huvudstad(indexOf(key))._2

scala> capital("Skåne")
val res3: String = Lund

scala> capital("Sverige")
val res4: String = Stockholm
\end{REPL}

\QUESTEND



\WHAT{Nyckel-värde-tabell.}

\QUESTBEGIN

\Task \what~En nyckel-värde-tabell är en smart datastruktur som gör att du kan slå upp det värde som en nyckel mappar till \emph{utan} att linjärsökning behöver ske. Värdet plockas fram direkt på en konstant tid, d.v.s. tiden att slå upp ett värde beror \emph{inte} på antalet element i samlingen, utan sker med mycket liten fördröjning.

I Scala heter nyckelvärdetabeller \code{Map} med stort M och är praktiska att använda i många olika sammanhang. \code{Map} finns i både en oföränderlig och en förändringsbar variant. Det går med metoder på formen \code{toXXX} lätt att omvandla mellan en \code{Map} och en sekvens av par av typen \code{XXX[(Nyckeltyp, Värdetyp)]}.

\Subtask Deklarera mappen \code{telnr} nedan i REPL och använd \code{apply} för att ta reda på telefonnumret till Fröken Ur.

\Subtask Vad har \code{telnr} för typ?

\Subtask Vad har \code{telnr.toVector} för typ?

\begin{Code}
val telnr = Map(
  "Anna"     -> 46462229812L,
  "Björn"     -> 46462229009L,
  "Sandra"    -> 46462220368L,
  "Fröken Ur" -> 4690510L,
)
\end{Code}
En uppsättning \code{Map}-instanser, vid behov nästlade, kan med fördel användas för att bygga upp en i-minnet-databas där inbyggda samlingsmetoder, t.ex. \code{map}, \code{filter}, och \code{for}-\code{yield}-uttryck, ger flexibla och effektiva sökmöjligheter. På veckans laboration ska du göra detta.

Samlingen \code{Map} är en generalisering av en sekvens, där man kan ''indexera'', inte bara med ett heltal, utan med vilken typ av värde som helst, t.ex. en sträng. Datastrukturen \code{Map} kallas också \emph{associativ array}\footnote{\href{https://en.wikipedia.org/wiki/Associative_array}{https://en.wikipedia.org/wiki/Associative\_array}} och är implementerad som en s.k. \emph{hashtabell}\footnote{\href{https://en.wikipedia.org/wiki/Hash_table}{https://en.wikipedia.org/wiki/Hash\_table}}, men du får vänta till fördjupningskursen innan vi går igenom hur en sådan datastruktur implementeras.

\SOLUTION

\TaskSolved \what~

\begin{REPL}
scala> telnr("Fröken Ur")
val res0: Long = 464690510

scala> :type telnr
Map[String,Long]

scala> :type telnr.toVector
Vector[(String, Long)]
\end{REPL}

\QUESTEND



\WHAT{Använda nyckel-värdetabell.}

\QUESTBEGIN

\Task \what~

\Subtask Skapa nedan variabler i REPL.
\begin{Code}
val follow = for i <- 2 to 16 by 2 yield (i, i + 1)
val xs = follow.toMap
val ys = xs.toVector
\end{Code}
Hamnar mappningarna i \code{ys} i samma ordning som \code{follow}? Varför?

\Subtask Med \code{xs} och \code{ys} deklarerade i REPL enligt ovan, para ihop yttryck till vänster med rätt resultat till höger. Om du är osäker på de sammansatta uttrycken, prova enklare uttryck i REPL och undersök värde och typ hos delresultat.

\begin{ConceptConnections}
\input{generated/quiz-w09-mapops-taskrows-generated.tex}
\end{ConceptConnections}

\SOLUTION

\TaskSolved \what


\SubtaskSolved Nej nyckel-värde-paren lagras i någon speciell ordning som bestäms av en intern, smart lagringsprincip enligt en s.k. hashfunktion\footnote{\url{https://sv.wikipedia.org/wiki/Hashfunktion}}, för att åstadkomma snabba uppslagningar av värden från nycklar och vilket normalt inte sammanfaller med ordningen i den sekvens som de skapades ur.

\SubtaskSolved

\begin{ConceptConnections}
  \input{generated/quiz-w09-mapops-solurows-generated.tex}
\end{ConceptConnections}

%%% BELOW IS SOLVED IN SCALA 3 AND the err msg is better! :)
% \noindent \emph{Fördjupning}:  Felmeddelandet som rad 2 ovan orsakar är lurigt:

% \begin{REPL}
% scala> ys(2)
% val res22: (Int, Int) = (6,7)

% scala> ys(4)
% val res23: (Int, Int) = (12,13)

% scala> ys(2) + ys(4)
% <console>:13: error: type mismatch;
%  found   : (Int, Int)
%  required: String
%        ys(2) + ys(4)

% \end{REPL}
% Det går som förväntat inte att addera två tupler, men varför säger kompilatorn att en sträng krävs?!? Detta beror på att, i enlighet med hur det fungerar i Java, valde Scala-språkets konstruktörer att låta strängsammanfogning fungera med alla möjliga typer vilket gör att kompilatorn inte ger upp när metoden \code{+} inte finns för tupler, utan i stället gör ett misslyckat försök med strängsammanfogning.

% Det mest olyckliga med detta är inte att felmeddelanden ibland blir missvisande, utan att det i vissa situationer inte ens \emph{blir} något felmeddelande, trots att man av rent misstag råkat strängkonkatenera i stället för t.ex. lägga till ett element i en mängd eller en mappning i en tabell. Detta typosäkra beteendet av strängsammanfogning har kritiserats, men det är inte okontroversiellt att ändra detta nu när så många utvecklare skrivit så mycket Scala-kod som bygger på strängars förmåga att kunna lägga till vad som helst på slutet. Situationen i Scala är dock inte hopplös efter introduktionen av stränginterpolering i Scala 2.10, som möjliggör infogande av värden i strängar på ett typsäkert sätt.
\QUESTEND





\WHAT{Registrering i förändringsbar nyckel-värde-tabell.}

\QUESTBEGIN

\Task \what~I denna uppgift ska du implementera en hjälpklass för registrering i en frekvenstabell som du sedan ska använda på veckans laboration. Klassen ska heta  \code{FreqMapBuilder} som efter upprepade anrop av metoden \code{add(s: String): Unit} kan skapa frekvenstabeller av typen \code{Map[String, Int]}, där nyckel-värde-paren i tabellen anger antalet förekomster av en viss sträng. Du ska utgå från koden nedan.

Klassen använder en förändringsbar tabell internt. Efter att man har lagt till många strängar kan man med metoden \code{toMap} få en oföränderlig tabell för  uppslagning av frekvenser för specifika strängar. Läs i snabbreferensen om vilka extra metoder för uppdatering som erbjuds av \code{mutable.Map[K, V]}.

\begin{Code}
class FreqMapBuilder:
  private val register = collection.mutable.Map.empty[String, Int]
  def toMap: Map[String, Int] = register.toMap
  def add(s: String): Unit = ???

object FreqMapBuilder:
  def apply(xs: String*): FreqMapBuilder = ???
\end{Code}

\noindent Implementera och testa \code{FreqMapBuilder}. \emph{Tips:} Du kan t.ex. använda \code{mutable.Map}-metoderna \code{addOne} och \code{getOrElse}.

\SOLUTION

\TaskSolved \what~
\begin{Code}
class FreqMapBuilder:
  private val register = scala.collection.mutable.Map.empty[String,Int]
  def toMap: Map[String, Int] = register.toMap
  def add(s: String): Unit =
    register.addOne(s -> (register.getOrElse(s, 0) + 1))

object FreqMapBuilder:
  def apply(xs: String*): FreqMapBuilder = 
    val result = new FreqMapBuilder
    xs.foreach(result.add)
    result
\end{Code}

\QUESTEND



\WHAT{Metoden \code{sliding}.}

\QUESTBEGIN

\Task  \what~  I veckans laboration kommer du att ha nytta av metoden \code{sliding}, som ger en iterator för speciella delsekvenser av en sekvens, vilka kan liknas vid ''utsikten'' i ett ''glidande fönster''.

\Subtask Kör nedan i REPL och beskriv vad som händer.

\begin{REPL}
scala> val xs = Vector("fem", "gurkor", "är", "fler", "än", "fyra", "tomater")
scala> xs.sliding(2).toVector
scala> xs.sliding(3).toVector
scala> xs.sliding(10).toVector
\end{REPL}

\Subtask Använd \code{xs.sliding(2)} och omvandla varje element i resultatet till ett par. Gör sedan om sekvensen av par till en nyckel-värde-tabell. Vad kan tabellen användas till?

\SOLUTION

\TaskSolved \what

\SubtaskSolved
\begin{REPL}
scala> val xs = Vector("fem", "gurkor", "är", "fler", "än", "fyra", "tomater")
val xs: Vector[String] =
  Vector(fem, gurkor, är, fler, än, fyra, tomater)

scala> xs.sliding(2).toVector
val res9: Vector[Vector[String]] =
  Vector(Vector(fem, gurkor), Vector(gurkor, är), Vector(är, fler), Vector(fler, än), Vector(än, fyra), Vector(fyra, tomater))

scala> xs.sliding(3).toVector
val res10: Vector[Vector[String]] =
  Vector(Vector(fem, gurkor, är), Vector(gurkor, är, fler), Vector(är, fler, än), Vector(fler, än, fyra), Vector(än, fyra, tomater))

scala> xs.sliding(10).toVector
val res11: Vector[Vector[String]] =
  Vector(Vector(fem, gurkor, är, fler, än, fyra, tomater))

\end{REPL}
\code{xs.sliding(n).toVector} skapar en sekvens som innehåller sekvenser av längden \code{n} som bildas genom att ta varje element och dess \code{n - 1} efterföljande element.

\SubtaskSolved
\begin{REPL}
scala> xs.sliding(2).map(ys => ys(0) -> ys(1)).toMap
val res0: Map[String,String] =
  Map(är -> fler,
      än -> fyra,
      fyra -> tomater,
      gurkor -> är,
      fem -> gurkor,
      fler -> än
  )
\end{REPL}
Man kan använda tabellen till att slå upp vilket som är efterföljande ord. Det fungerar eftersom alla ord är unika. Om det funnits flera likadana ord med olika efterföljande ord så hade vi behövt skapa en tabell med nycklar som mappar till en samling som registrerar efterföljande ord. Detta ska vi göra på veckans laboration.

\QUESTEND




\WHAT{Läsa text från fil och webbservrar.}

\QUESTBEGIN

\Task \what~På laborationen ska du bygga upp tabeller från data i textformat. Då har du nytta av att kunna läsa text från filer och från webben. Testa detta i REPL:
\begin{REPL}
scala> val url = "https://fileadmin.cs.lth.se/pgk/europa.txt"
scala> val xs = io.Source.fromURL(url, "UTF-8").getLines.toVector
scala> val data = xs.map(_.split(';').toVector)
scala> data.head
scala> data.foreach(println)
\end{REPL}

\Subtask Skapa dessa tabeller ur sekvensen \code{data}:
\begin{Code}
val populationOf: Map[String, Int]    = ???  // länders invånarantal
val sizeOf:       Map[String, Int]    = ???  // länders yta i km^2
val capitalOf:    Map[String, String] = ???  // länders huvudstäder
\end{Code}
Testa tabellerna i REPL.

\Subtask Spara ner data i en textfil \code{europa.txt}. Läsa in data från filen med metoden\\ \code{Source.fromFile(filnamn, teckenkodning)} på liknande sätt som med  \code{fromURL} ovan. Om du kör i en Linux-terminal kan du enkelt ladda ner en fil så här (notera att det är stora bokstaven \code{O} och inte en nolla i optionen \code{-sLO}):
\begin{REPLnonum}
> curl -sLO https://fileadmin.cs.lth.se/pgk/europa.txt
\end{REPLnonum}
Skriv ut alla raderna i \code{europa.txt} med hjälp av \code{Source.fromFile} i REPL.

\SOLUTION

\TaskSolved \what~

\SubtaskSolved
\begin{CodeSmall}
val populationOf = data.tail.map(v => v(0) -> v(1).toInt).toMap
val sizeOf       = data.tail.map(v => v(0) -> v(2).toInt).toMap
val capitalOf    = data.tail.map(v => v(0) -> v(3)).toMap
\end{CodeSmall}

\begin{REPL}
scala> capitalOf("Sverige")
res2: String = Stockholm

scala> populationOf("Sverige")
res3: Int = 9223766

scala> sizeOf("Sverige")
res4: Int = 449964
\end{REPL}

\begin{REPL}
scala> val filename = "europa.txt"
scala> val xs = io.Source.fromFile(filename, "UTF-8").getLines.toVector
scala> val data = xs.map(_.split(';').toVector)
scala> data.map(_.map(_.take(15).padTo(15,' ')).mkString(" ")).foreach(println)
\end{REPL}
\QUESTEND





\ExtraTasks %%%%%%%%%%%%%%%%%%%%%%%%%%%%%%%%%%%%%%%%%%%%%%%%%%%%%%%%%%%%%%%%%%%%

\WHAT{Skapa ett textspel med hjälp av tabeller.}

\QUESTBEGIN

\Task \what~Gör ett enkelt spel för att träna på olika fakta om Europas länder och huvudstäder genom att läsa data från URL:en:\\ \url{https://fileadmin.cs.lth.se/pgk/europa.txt}
\\Där finns text kodad i UTF-8 med följande innehåll (endast de första raderna visas):
\begin{Code}
Land;Invånarantal;Storlek(km^2);Huvudstad
Albanien;3581655;28748;Tirana
Andorra;71201;468;Andorra la Vella
Belgien;10584534;30528;Bryssel
Bosnien-Hercegovina;4590310;51129;Sarajevo
Bulgarien;7385367;110910;Sofia
Cypern;854000;9250;Nicosia
Danmark;5475791;43094;Köpenhamn
Estland;1324333;45226;Tallinn
Finland;5315280;338145;Helsingfors
Frankrike;61538322;551695;Paris
Färöarna;48344;139574;Torshamn
Grekland;10964021;131940;Aten
// ... etcetera för alla Europas länder.
\end{Code}
Låt till exempel användaren svara på slumpvisa frågor av typen:
\begin{itemize}[noitemsep]
  \item Har Andorra fler invånare än Cypern?
  \item Vad heter huvudstaden i Bulgarien?
  \item Har Danmark större yta än Finland?
\end{itemize}
Använd oföränderliga tabeller med lämpliga nycklar och värden. Du kan använda en mängd med länder/huvudstäder som användaren hittills svarat rätt på för att kunna förhindra att dessa återkommer igen.
\SOLUTION

\TaskSolved --

\QUESTEND



\AdvancedTasks %%%%%%%%%%%%%%%%%%%%%%%%%%%%%%%%%%%%%%%%%%%%%%%%%%%%%%%%%%%%%%%%%


\WHAT{Registrering med \code{groupBy}.}

\QUESTBEGIN

\Task \what~Vi ska nu utnyttja ett riktigt listigt trick för att via en enda kodrad implementera registrering med hjälp av samlingsmetoderna \code{groupBy} och \code{map}.

\Subtask Läs om metoden \code{groupBy} i snabbreferensen. Du hittar den under rubriken \emph{''Methods in trait \code{Iterable[A]}''} eftersom \code{groupBy} fungerar på alla samlingar. Testa \code{groupBy} enligt nedan och beskriv vad som händer.

\begin{REPL}
scala> val xs = Vector(1, 1, 2, 2, 4, 4, 4).groupBy(x => x > 2)
scala> val ys = Vector(1, 1, 2, 2, 4, 4, 4).groupBy(x => x)
\end{REPL}

\Subtask Skapa en funktion \code{freq} med nedan funktionshuvud som returnerar en tabell med antalet förekomster av olika heltal i \code{xs}. Testa \code{freq} på en sekvens av 1000 slumpvisa tärningskast och förklara hur funktionen \code{freq} fungerar. \emph{Tips:} Gör först \code{groupBy(???)} och sedan \code{map(???)}.

\begin{Code}
def freq(xs: Vector[Int]): Map[Int, Int] = ???

def kasta(n: Int): Vector[Int] =
  Vector.fill(n)(scala.util.Random.nextInt(6) + 1)
\end{Code}

\SOLUTION

\TaskSolved \what~

\SubtaskSolved Metoden \code{groupBy} skapar en nyckel-värde-tabell där värdena i tabellen är en sekvens med elementen grupperade på ett speciellt sett.
Mer precist:

Resultatet av \code{xs.groupBy(f: K => V)} för en sekvens \code{xs} av typen \code{Vector[K]} blir en tabell av typen \code{Map[V,Vector[K]]} där varje element \code{e} i \code{xs} är grupperade i samma tabellvärde om de lika är enligt \code{f(e)}. Varje grupp får tabellnyckeln \code{f(e)}.

\emph{Listigt trick:} Om man låter funktionen \code{f} vara enhetsfunktionen som avbildar varje element på sig själv, alltså \code{x => x}, så grupperas värdena i samma sekvens om de är lika.

\begin{REPL}
scala> val xs = Vector(1, 1, 2, 2, 4, 4, 4).groupBy(x => x > 2)
val xs: Map[Boolean,Vector[Int]] =
  Map(false -> Vector(1, 1, 2, 2), true -> Vector(4, 4, 4))

scala> val ys = Vector(1, 1, 2, 2, 4, 4, 4).groupBy(x => x)
val ys: Map[Int,Vector[Int]] =
  Map(2 -> Vector(2, 2), 4 -> Vector(4, 4, 4), 1 -> Vector(1, 1))
\end{REPL}


\SubtaskSolved

\begin{Code}
def freq(xs: Vector[Int]): Map[Int, Int] =
  xs.groupBy(x => x).map(p => p._1 -> p._2.size)
\end{Code}
Förklaring: metoden \code{groupBy} skapar en tabell med par \code{k, v} där \code{v} är en sekvens med så många \code{k} som antalet gånger \code{k} förekommer i \code{xs}. Genom att omvandla alla värden \code{p._2} till storleken \code{p._2.size} får vi en frekvenstabell.

\begin{REPL}
scala> freq(kasta(1000))
val res0: Map[Int,Int] = 
  Map(5 -> 163, 1 -> 174, 6 -> 161, 2 -> 169, 3 -> 167, 4 -> 166)

scala> freq(kasta(1000)).toVector.sortBy(_._1).foreach(println)
(1,183)
(2,167)
(3,169)
(4,179)
(5,154)
(6,148)
\end{REPL}

\QUESTEND





\WHAT{Skriva till fil.}

\QUESTBEGIN

\Task \what~Som hjälp när du skapar egna intressanta applikationer eller bygger vidare på kursens laborationer och övningar med frivilliga extrauppgifter, kan du använda funktionerna i singelobjektet \code{IO} nedan, som finns i kursens scala-bibliotek \href{https://fileadmin.cs.lth.se/pgk/api}{introprog}.\footnote{Källkoden finns här och även på sidan \pageref{disk-access-code}:\\ \href{https://github.com/lunduniversity/introprog/blob/master/compendium/workspace/introprog/src/main/scala/introprog/IO.scala}{https://github.com/lunduniversity/introprog-scalalib/blob/master/src/main/scala/introprog/IO.scala}}

IO-modulen använder \code{scala.io.Source} för att serialisera och de-serialisera strängar till och från vanliga textfiler. IO-modulen använder även paketet \code{java.io} för att erbjuda funktioner som gör det enkelt att serialisera/de-serialisera godtyckliga objekt skapade med hjälp av serialserbara klasser till/från binärfiler. Case-klasser i Scala blir automatiskt serialiserbara.

I implementationen av \code{IO} används \code{try ... finally} för att säkerställa att filer inte lämnas öppnade även om något går fel under den läs/skriv-process som sköts av det underliggande operativsystemet.

\Subtask
Kompilera och resta nedan med \code{introprog} på classpath, t.ex. med hjälp av \code{sbt}.
\begin{Code}
import introprog.IO

case class Player(name: String)

@main def run(): Unit = 
  println("Test of output/input objects to/from disk:")
  val highscores = Map(Player("Sandra") -> 42, Player("Björn") -> 5)
  IO.saveObject(highscores,"highscores.ser")
  val highscores2 = IO.loadObject[Map[Player, Int]]("highscores.ser")
  val isSameContents = highscores2 == highscores
  val testResult = if (isSameContents) "SUCCESS :)" else "FAILURE :("
  println(testResult)
\end{Code}

\Subtask
Använd \code{IO}-modulen för att spara användarens poängresultat i ditt spel om Europas länder och städer, i extrauppgiften ovan. Implementationen av \code{introprog.IO} finns här: \url{https://github.com/lunduniversity/introprog-scalalib/blob/master/src/main/scala/introprog/IO.scala} 

% \begin{figure}
% %  \scalainputlisting[basicstyle=\ttfamily\fontsize{9.2}{11}\selectfont]{examples/IO.scala}
%   \scalainputlisting[basicstyle=\ttfamily\fontsize{9.2}{11}\selectfont]{../workspace/introprog/src/main/scala/introprog/IO.scala}
%   \label{disk-access-code}
% \end{figure}
\SOLUTION

\TaskSolved --

\QUESTEND



%
%
% \subsection{\TODO Värdera nedan gamla uppgifter}
%
%
%
% \WHAT{Objekt med attribut (fält).}
%
% \QUESTBEGIN
%
% \Task  \what~  Ett objekt kan samla data som hör ihop och på så sätt skapa en datastruktur. Data i ett objekt kallas \emph{attribut} eller \emph{fält}, \Eng{field}. Objekt som samlar enbart data kallas även \emph{post} \Eng{record}.
% \begin{REPLnonum}
% scala> object mittKonto { var saldo = 0; val nummer = 12345L }
% \end{REPLnonum}
% \Subtask Skriv en sats som sätter in ett slumpmässigt belopp mellan 0 och en miljon på \code{mittKonto} ovan med hjälp av punktnotation och tilldelning.
%
% \Subtask Vad händer om du försöker ändra attributet \code{nummer}?
%
% \SOLUTION
%
%
% \TaskSolved \what
%
%
% \SubtaskSolved   \code{mittKonto.saldo = (math.random() * 1000000).toInt}
%
% \SubtaskSolved   Går ej eftersom val är oföränderlig, man får alltså ett Error.
%
%
% \QUESTEND
%
%
%
%
% %%<AUTOEXTRACTED by mergesolu>%%      %Uppgift 2
%
%
%
%
% \WHAT{Klass med attribut.}
%
% \QUESTBEGIN
%
% \Task  \what~  Om du vill ha många objekt av samma typ, kan du använda en \textbf{klass}. På så sätt kan man skapa många datastrukturer av samma typ men med olika innehåll. Man skapar nya objekt med nyckelordet \code{new} följt av klassens namn. Klassen utgör en ''mall'' för objektet som skapas. Ett objekt som skapas med \code{new Klassnamn} kallas även en \textbf{instans} av klassen \code{Klassnamn}. Nedan skapas en datastruktur \code{Konto} som samlar data om ett bankonto. Instanser av typen \code{Konto} håller reda på hur mycket pengar det finns på kontot och vilket kontonumret är. Datavärden som sparas i varje objektinstans, så som \code{saldo} och \code{nummer}, kallas \textbf{attribut} \Eng{attribute} eller \textbf{fält} \Eng{field}.
%
% \begin{REPL}
% scala> class Konto {
%          var saldo = 0
%          var nummer = 0L
%        }
% scala> val k1 = new Konto
% scala> val k2 = new Konto
% scala> k1.saldo = 1000
% scala> k1.nummer = 12345L
% scala> k2.saldo = 2000
% scala> k2.nummer = 67890L
% scala> println("Konto: " + k1.nummer + " Saldo:" + k1.saldo)
% scala> println("Konto: " + k2.nummer + " Saldo:" + k2.saldo)
% \end{REPL}
%
% \Subtask\Pen Rita hur minnessituationen ser ut efter att ovan rader har exekverats.
%
% \Subtask\Pen Vad hade det fått för konsekvenser om attributet \code{nummer} vore oföränderligt i klassen ovan? (Jämför med objektet \code{mittKonto}.)
%
%
% \SOLUTION
%
%
% \TaskSolved \what
%
%
% \SubtaskSolved   \includegraphics[scale=0.5]{../img/w04-solutions/uppgift-3a}
%
% \SubtaskSolved
% Tilldelningen på rad 8 \code{k1.nummer = 12345L} ger felmeddelande eftersom variablen är oföränderlig.
%
%
% \QUESTEND
%
%
%
%
% %%<AUTOEXTRACTED by mergesolu>%%      %Uppgift 3
%
%
%
%
% \WHAT{Klass med attribut som parametrar.}
%
% \QUESTBEGIN
%
% \Task  \what~  Om man vill ge attributen initialvärden när objektet skapas med \code{new}, kan man placera attributen i en parameterlista till klassen. Koden som körs när objektet skapas och attributen tilldelas sina initialvärden, kallas \textbf{konstruktor} \Eng{constructor}.
%
% \begin{REPL}
% scala> class Konto(var saldo: Int, val nummer: Long)
% scala> val k = new Konto(0, 12345L)
% scala> println("Konto: " + k.nummer + " Saldo:" + k.saldo)
% scala> println(k)
% scala> k.toString
% \end{REPL}
%
% \Subtask Den två sista raderna ovan skriver ut den identifierare som JVM använder för att hålla reda på objektet i sina interna datastrukturer. Vad skrivs ut?
%
% \Subtask Skapa ännu en instans av klassen Konto  med samma saldo och nummer som \code{k} ovan och spara den i \code{val k2} och undersök dess objektidentifierare. Får objekten \code{k} och \code{k2} olika objektidentifierare?
%
% \Subtask Sätt in olika belopp på respektive konto.
%
% \Subtask Vad händer om du försöker ändra attributet \code{nummer}?
%
% \Subtask\Pen Ibland räcker det fint med en tupel, men ofta vill man ha en klass istället. Beskriv några fördelar med en Konto-klassen ovan jämfört med en tupel av typen \code{(Int, Long)}.
%
% \begin{REPLnonum}
% scala> var k3 = (0, 12345L)
% scala> k3 = (k3._1 + 100, k3._2)
% \end{REPLnonum}
%
% \SOLUTION
%
%
% \TaskSolved \what
%
%
% \SubtaskSolved   \code{String = Konto@cd576}, där \code{Konto@cd576} är ett unikt namn som identifierar instansen.
%
% \SubtaskSolved   Ja.
%
% \SubtaskSolved
% \begin{REPLnonum}
% scala> k.saldo = 42
% scala> k2.saldo = 67
% \end{REPLnonum}
%
% \SubtaskSolved   Eftersom variablen är oföränderlig ges ett felmeddelande.
%
% \SubtaskSolved   En fördel med klass är att man kan specificera att variablen ska kunna vara föränderlig. En till är att man kan inkludera metoder i klassen som man vill kunna använda på värdena.
%
%
% \QUESTEND
%
%
%
%
% %%<AUTOEXTRACTED by mergesolu>%%      %Uppgift 4
%
%
%
%
% \WHAT{Publikt eller privat attribut?}
%
% \QUESTBEGIN
%
% \Task  \what~  Man kan förhindra att ett attribut syns utanför klassen med hjälp av nyckelordet \code{private}.
%
% \begin{REPL}
% scala> class Konto1(val nummer: Long){ var saldo = 0 }
% scala> val k1 = new Konto1(12345678901L)
% scala> k1.nummer
% scala> k1.saldo += 1000
% scala> class Konto2(val nummer: Long){ private var saldo = 0 }
% scala> val k2 = new Konto2(12345678901L)
% scala> k2.nummer
% scala> k2.saldo += 1000
% \end{REPL}
%
% \Subtask Vad händer ovan?
%
% \Subtask Gör en ny version av klassen \code{Konto} enligt nedan:
%
% \begin{Code}
% class Konto(val nummer: Long){
%   private var saldo = 0
%   def in(belopp: Int): Unit = {saldo += belopp}
%   def ut(belopp: Int): Unit = {saldo -= belopp}
%   def show: Unit =
%     println("Konto Nr: " + nummer + " saldo: " + saldo)
% }
%
% object Main {
%   def main(args: Array[String]): Unit = {
%     val k = new Konto(1234L)
%     k.show
%     k.in(1000)
%     println("Uttag: " + k.ut(500))
%     println("Uttag: " + k.ut(1000))
%     k.show
%   }
% }
% \end{Code}
%
% \Subtask Spara koden i en fil, kompilera med \code{scalac} och kör. Testa även vad som händer om du försöker komma åt attributet \code{saldo} i main-metoden med t.ex. \code{println(k.saldo)} eller \code{k.saldo += 1000}.
%
% \Subtask Vi ska nu förhindra överuttag. Ändra i metoden \code{ut} så att den får signaturen \code{ut(belopp: Int): (Int, Int) = ???} och implementera \code{ut} så att den returnerar både beloppet man verkligen kan ta ut och kvarvarande saldo. Om man försöker ta ut mer än det finns på kontot så ska saldot bli 0 och man får bara ut det som finns kvar. Spara, kompilera, kör.
%
% \Subtask Förbättra metoderna \code{in} och \code{ut} så att man inte kan sätta in eller ta ut negativa belopp.
%
% \Subtask Vad är fördelen med att göra föränderliga attribut privata och bara påverka deras värden indirekt via metoder?
%
% \SOLUTION
%
%
% \TaskSolved \what
%
%
% \SubtaskSolved
% Det går bra att ändra på variablen saldo i instansen av Konto1 men inte av Konto2 där man får ett error på raden ''k2.saldo += 1000''
%
% \SubtaskSolved  -
%
% \SubtaskSolved
% ''println(k.saldo)'' och ''k.saldo += 1000'' ger båda error, pga privat attribut.
%
% \SubtaskSolved
% \begin{Code}
% def ut(belopp: Int): (Int, Int) = {
% 	if(saldo >= belopp) {
% 		saldo -= belopp
% 		(belopp, saldo)
% 	} else {
% 		val temp = saldo
% 		saldo = 0
% 		(temp, 0)
% 	}
% }
% \end{Code}
%
% \SubtaskSolved
% Lägg till en if-sats i båda funktionerna som omsluter den gamla koden.
% \begin{Code}
% def ut(belopp: Int): (Int, Int) = {
%   if(belopp >= 0) {
%     if(saldo >= belopp) {
%       saldo -= belopp
%       (belopp, saldo)
%     } else {
%       val temp = saldo
%       saldo = 0
%       (temp, 0)
%     }
%   }
% }
%
% def in(belopp: Int): Unit = {
%   if(belopp >= 0) {
%     saldo += belopp
%   }
% }
% \end{Code}
%
% \SubtaskSolved
% Genom att göra attributet privat och gör egna metoder kan man se till att attriuten endast ändras på säkra sätt. Så inte fel uppstår.
%
%
% \QUESTEND
%
%
%
%
% %%<AUTOEXTRACTED by mergesolu>%%      %Uppgift 5
%
%
%
%
% \WHAT{Vilken typ har ett objekt?}
%
% \QUESTBEGIN
%
% \Task  \what~  Objektets typ bestäms av klassen. Vid tilldelning måste typerna passa ihop.
%
% \Subtask Vilka rader nedan ger felmeddelande? Hur lyder felmeddelandet?
% \begin{REPL}
% scala> class Punkt(val x: Double, val y: Double)
% scala> val pt: Punkt = new Punkt(10.0, 10.0)
% scala> val i: Int = pt.x
% scala> val (x: Double, y: Double) = (pt.x, pt.y)
% scala> val p: Double = new Punkt(5.0, 5.0)
% scala> val p = new Punkt(5.0, 5.0): Double
% scala> val p = new Punkt(5.0, 5.0): Punkt
% scala> pt: Punkt
% \end{REPL}
%
%
% \Subtask Man kan undersöka om ett objekt är av en viss typ med metoden \\ \code{isInstanceOf[Typnamn]}. Vad ger nedan anrop av metoden \code{isInstanceOf} för värde?
% \begin{REPL}
% scala> class Punkt(val x: Double, val y: Double)
% scala> val pt: Punkt = new Punkt(1.0, 2.0)
% scala> pt.isInstanceOf[Punkt]
% scala> pt.isInstanceOf[Double]
% scala> pt.x.isInstanceOf[Punkt]
% scala> pt.x.isInstanceOf[Double]
% scala> pt.x.isInstanceOf[Int]
% \end{REPL}
%
% \SOLUTION
%
%
% \TaskSolved \what
%
%
% \SubtaskSolved
% ''val i: Int = pt.x'' error: type mismatch;
% Eftersom typen Int ej är kompatibel med ett värde av typen Double.
%
% ''val p: Double = new Punkt(5.0, 5.0)'' error: type mismatch;
% Eftersom typen Double ej är kompatibel med ett värde av typen Punkt.
%
% ''val p = new Punkt(5.0, 5.0): Double'' error: type mismatch;
% Eftersom typen Double ej är kompatibel med ett värde av typen Punkt.
%
% \SubtaskSolved
% Rad 3 till 7 i respektive ordning: true, false, false, true och false.
%
%
% \QUESTEND
%
%
%
%
% %%<AUTOEXTRACTED by mergesolu>%%      %Uppgift 6
%
%
%
%
% \WHAT{Topptypen \code{Any}.}
%
% \QUESTBEGIN
%
% \Task  \what~ Alla klasser är också av typen \code{Any}. Alla klasser får därmed med sig några gemensamma metoder som finns i den fördefinierade klassen \code{Any}, däribland metoderna  \code{isInstanceOf} och \code{toString}.  Vad blir resultatet av respektive rad nedan? Vilken rad ger ett felmeddelande?
%
%
% \begin{REPL}
% scala> class Punkt(val x: Double, val y: Double)
% scala> val pt: Punkt = new Punkt(1.0, 2.0)
% scala> pt.isInstanceOf[Punkt]
% scala> pt.isInstanceOf[Any]
% scala> pt.x.toString
% scala> println(pt.x)
% scala> val a: Any = pt
% scala> println(a.x)
% scala> a.toString
% scala> pt.y.toString
% scala> a.y.toString
% \end{REPL}
%
% \SOLUTION
%
%
% \TaskSolved \what
%
% \begin{enumerate}
% \item Definierar klassen Punkt.
% \item En variabel pt: Punkt skapas.
% \item true
% \item true
% \item String = 1.0
% \item skriver ut: 1.0
% \item En variabel med namnet a skapas med typen Any.
% \item error: value x is not a member of Any
% \item a ges nu typen String
% \item String = 2.0
% \item error: value y is not a member of Any
% \end{enumerate}
%
%
% \QUESTEND
%
%
%
%
% %%<AUTOEXTRACTED by mergesolu>%%      %Uppgift 7
%
%
%
%
% \WHAT{Byta ut metoden \code{toString}}.
%
% \QUESTBEGIN
%
% \Task  \what~ I klassen \code{Any} finns metoden \code{toString} som skapar en strängrepresentation av objektet. Du kan byta ut metoden \code{toString} i klassen \code{Any} mot din egen implementation. Man använder nyckelordet \code{override} när man vill byta ut en metodimplementation.
%
% \begin{REPL}
% scala> class Punkt(val x: Double, val y: Double) {
%          override def toString: String = "[x=" + x + ",y=" + y + "]"
%        }
% scala> val pt = new Punkt(1.0, 42.0)
% scala> pt.toString
% scala> println(pt)
% \end{REPL}
%
% \Subtask Vad händer egentligen på sista raden ovan?
%
% \Subtask Omdefiniera toString så att den ger en sträng på formen \code{Punkt(1.0, 42.0)}.
%
% \Subtask Vad händer om du utelämnar nyckelordet \code{override} vid omdefiniering?
%
% \SOLUTION
%
%
% \TaskSolved \what
%
%
% \SubtaskSolved
% ''println(pt)'' kallar på pt.toString, och eftersom metoden är överskriven kallas den nya version.
%
% \SubtaskSolved   \code{override def toString: String = ''Punkt('' + x + '', '' + y + '').''}
%
% \SubtaskSolved
% error: overriding method toString in class Object of type ()String;
%
%
% \QUESTEND
%
%
%
%
% %%<AUTOEXTRACTED by mergesolu>%%      %Uppgift 8
%
%
%
%
% \WHAT{Objektfabrik med \code{apply}-metod.}
%
% \QUESTBEGIN
%
% \Task  \what~  Man kan ordna så att man slipper skriva \code{new} med ett s.k. \emph{fabriksobjekt} \Eng{factory object}.
% \begin{Code}
% class Pt(val x: Double, y: Double) {
%   override def toString: String = "Pt(x=" + x + ",y=" + y + ")"
% }
% object Pt {
%   def apply(x: Double, y: Double): Pt = new Pt(x, y)
% }
% \end{Code}
%
% \Subtask Skriv satser som använder metoden \code{apply} i fabriksobjektet \code{object Pt} för att skapa flera olika punkter.
%
% \Subtask Ge applymetoden default-argument 0.0 för både x och y så att \code{Pt()} skapar en punkt i origo.
%
% \Subtask Skapa en klass \code{Rational} som representerar rationellt tal som en kvot mellan två heltal. Ge klassen två oföränderliga, publika klassparameterattribut med namnen \code{nom} för täljaren och \code{denom} för nämnaren.
%
% \Subtask Skapa ett fabriksobjekt med en \code{apply}-metod som tar två heltalsparametrar och skapar en instans av klassen \code{Rational}.
%
% \Subtask Skapa olika instanser av din klass \code{Rational} ovan med hjälp av fabriksobjektet.
%
%
% \SOLUTION
%
%
% \TaskSolved \what
%
%
% \SubtaskSolved
% \begin{REPL}
% scala> val pt = Pt(1.0, 2.0)
% pt: Pt = Pt(x=1.0,y=2.0)
%
% scala> Pt(4.0, 2.0)
% res0: Pt = Pt(x=4.0,y=2.0)
%
% scala> Pt(6.0, 3.0)
% res1: Pt = Pt(x=6.0,y=3.0)
%
% scala> Pt(666.0, 1337.0)
% res2: Pt = Pt(x=666.0,y=1337.0)
% \end{REPL}
%
% \SubtaskSolved  \code{def apply(): Pt = new Pt(0, 0)}
%
% \SubtaskSolved  \code{class Rational(val nom: Int, val denom: Int)}
%
% \SubtaskSolved
% \begin{REPLnonum}
% object Rational {
% def apply(nom: Int, denom: Int): Rational = new Rational(nom, denom)
% }
% \end{REPLnonum}
%
% \SubtaskSolved
% \begin{REPL}
% scala> Rational(2, 5)
% scala> Rational(2, 7)
% scala> Rational(7, 4)
% scala> Rational(666, 1337)
% \end{REPL}
%
%
% \QUESTEND
%
%
%
%
% %%<AUTOEXTRACTED by mergesolu>%%      %Uppgift 9
%
%
%
%
% \WHAT{Skapa en case-klass.}
%
% \QUESTBEGIN
%
% \Task  \what~  Med en case-klass får man \code{toString} och fabriksobjekt på köpet. Man behöver inte skriva \code{val} framför klassparametrar i case-klasser; klassparametrar blir publika, oföränderliga attribut automatiskt när man deklarerar en case-klass.
%
% \begin{REPL}
% scala> case class Pt(x: Double, y: Double)
% scala> val p = Pt(1.0, 42.0)
% scala> p.toString
% scala> println(p)
% scala> println(Pt(5,6))
% \end{REPL}
%
% \Subtask Implementera din klass \code{Rational} från föregående uppgift, men nu som en case-klass.
%
% \SOLUTION
%
%
% \TaskSolved \what
%
% \SubtaskSolved  \code{case class Rational(nom: Int, denom: Int)}
%
%
% \QUESTEND
%
%
%
%
% %%<AUTOEXTRACTED by mergesolu>%%      %Uppgift 10
%
%
%
%
% \WHAT{Metoder på datastrukturer.}
%
% \QUESTBEGIN
%
% \Task \label{task:point} \what~   En datastruktur blir mer användbar om det finns metoder som kan användas på datastrukturen. Metoder i Scala kan även ha (vissa) specialtecken som namn, t.ex. \code{+} enligt nedan.
% \begin{REPL}
% scala> case class Point(x: Double, y: Double) {
%          def distToOrigin: Double = math.hypot(x, y)
%          def add(p: Point): Point = Point(x + p.x, y + p.y)
%          def +(p: Point): Point = add(p)
%        }
% \end{REPL}
%
% \Subtask Använd metoden \code{distToOrigin} för att ta reda på vad punkten med koordinaterna (3, 4) har för avstånd till origo?
%
% \Subtask Skriv satser som skapar två punkter (3,4) och (5, 6) och låt variablerna p1 och p2 referera till respektive punkt. Låt variabeln p3 bli summan av p1 och p2 med hjälp av metoden \code{add}. Vad får uttrycken \code{p3.x} resp. \code{p3.y} för värden?
%
%
%
% \SOLUTION
%
%
% \TaskSolved \what
%
%
% \SubtaskSolved
% \begin{REPLnonum}
% scala> Point(3, 4).distToOrigin
% res0: Double = 5.0
% \end{REPLnonum}
%
% \SubtaskSolved
% p3.x = 8
% p3.y = 10
%
%
% \QUESTEND
%
%
%
%
% %%<AUTOEXTRACTED by mergesolu>%%      %Uppgift 11
%
%
%
%
% \WHAT{Operatornotation.}
%
% \QUESTBEGIN
%
% \Task  \what~  Vid punktnotation på formen: \\ \code{objekt.metod(argument)} \\ kan man skippa punkten och parenteserna och skriva:\\ \code{objekt metod argument}  \\
% Detta förenklade skrivsätt kallas \textbf{operatornotation}.
%
% \Subtask Använd klassen \code{Point} från uppgift \ref{task:point} och prova nedan satser. Vilka rader använder operatortnotation och vilka rader använder punktnotation? Vilka rader ger felmeddelande?
% \begin{REPL}
% scala> val p1 = Point(3,4)
% scala> val p2 = Point(3,4)
% scala> p1.add(p2)
% scala> p1 add p2
% scala> p1.+(p2)
% scala> p1 + p2
% scala> 42 + 1
% scala> 42.+(1)
% scala> 42.+ 1
% scala> 42 +(1)
% scala> 1.to(42)
% scala> 1 to 42
% scala> 1.to(42)
% \end{REPL}
%
% \Subtask Implementera metoderna \code{sub} och \code{-} i klassen \code{Point} och skriv uttryck som kombinerar add och sub, samt + och - i både punktnotation och operatornotation.
%
% \Subtask Operatornotation fungerar även med flera argument. Man använder då parenteser om listan med argumenten:
% \code{ objekt metod (arg1, arg2)}  \\
% Definiera en metod \\
% \code{def scale(a: Double, b: Double) = Point(x * a, y * b)} \\
% i klassen \code{Point} och skriv satser som använder metoden med punktnotation och operatornotation.
%
%
%
%
%
% \SOLUTION
%
%
% \TaskSolved \what
%
%
% \SubtaskSolved
% \\Operatornotation:	4, 6, 10, 12
% \\Punktnotation:		3, 5, 8, 9, 11, 13
% \\Felmeddelande:		9
%
% \SubtaskSolved
% \begin{Code}
% case class Point(x: Double, y: Double) {
%   def distToOrigin: Double = math.hypot(x, y)
%   def add(p: Point): Point = Point(x + p.x, y + p.y)
%   def +(p: Point): Point = add(p)
%   def sub(p: Point): Point = Point(x - p.x, y - p.y)
%   def -(p: Point): Point = sub(p)
% }
% \end{Code}
% \begin{REPL}
% scala> val p1: Point = Point(1, 9)
% scala> val p2: Point = Point(9, 6)
% scala> p1.sub(p2)
% scala> p1.-(p2)
% scala> p2 sub p1
% scala> p2 - p2
% scala> p1.add(p2.sub(p1))
% scala> p1 + (p2 - p1)
% \end{REPL}
%
% \SubtaskSolved
% \begin{Code}
% case class Point(x: Double, y: Double) {
%   def distToOrigin: Double = math.hypot(x, y)
%   def add(p: Point): Point = Point(x + p.x, y + p.y)
%   def +(p: Point): Point = add(p)
%   def sub(p: Point): Point = Point(x - p.x, y - p.y)
%   def -(p: Point): Point = sub(p)
%   def scale(a: Double, b: Double) = Point(x * a, y * b)
% }
% \end{Code}
% \begin{REPL}
% scala> val p: Point(13,  37)
% scala> p.scale(4, 2)
% scala> p scale (3, 7)
% \end{REPL}
%
%
% \QUESTEND
%
%
%
%
% %%<AUTOEXTRACTED by mergesolu>%%      %Uppgift 12
%
%
%
%
% \WHAT{Föränderlighet och oföränderlighet.}
%
% \QUESTBEGIN
%
% \Task  \what~  Oföränderliga och föränderliga objekt beter sig olika vid tilldelning.
%
% \Subtask\Pen Innan du kör nedan kod: Försök lista ut vad som kommer att skrivas ut. Rita minnessituationen efter varje tilldelning.
%
% \begin{Code}
% println("\n--- Example 1: mutable value assigmnent")
% var x1 = 42
% var y1 = x1
% x1 = x1 + 42
% println(x1)
% println(y1)
% \end{Code}
%
% \Subtask\Pen Innan du kör nedan kod: Försök lista ut vad som kommer att skrivas ut. Rita minnessituationen efter varje tilldelning.
%
% \begin{Code}
% println("\n--- Example 2: mutable object reference assignment")
% class MutableInt(private var i: Int) {
%   def +(a: Int): MutableInt = { i = i + a; this }
%   override def toString: String = i.toString
% }
% var x2 = new MutableInt(42)
% var y2 = x2
% x2 = x2 + 42
% println(x2)
% println(y2)
% \end{Code}
%
% \Subtask\Pen Innan du kör nedan kod: Försök lista ut vad som kommer att skrivas ut. Rita minnessituationen efter varje tilldelning.
%
% \begin{Code}
% println("\n--- Example 3: immutable object reference assignment")
% class ImmutableInt(val i: Int) {
%   def +(a: Int): ImmutableInt = new ImmutableInt(i + a)
%   override def toString: String = i.toString
% }
% var x3 = new ImmutableInt(42)
% var y3 = x3
% x3 = x3 + 42
% println(x3)
% println(y3)
% \end{Code}
%
% \Subtask\Pen Vad finns det för fördelar med oföränderliga datastrukturer?
%
%
% \SOLUTION
%
%
% \TaskSolved \what
%
%
% \SubtaskSolved   \includegraphics[scale=0.5]{../img/w04-solutions/uppgift-13a}
%
% \SubtaskSolved
% \begin{enumerate}
% \item \includegraphics[scale=0.5]{../img/w04-solutions/uppgift-13b-1}
% \item \includegraphics[scale=0.5]{../img/w04-solutions/uppgift-13b-2}
% \item \includegraphics[scale=0.5]{../img/w04-solutions/uppgift-13b-3}
% \end{enumerate}
%
% \SubtaskSolved
% \begin{enumerate}
% \item \includegraphics[scale=0.5]{../img/w04-solutions/uppgift-13c-1}
% \item \includegraphics[scale=0.5]{../img/w04-solutions/uppgift-13c-2}
% \item \includegraphics[scale=0.5]{../img/w04-solutions/uppgift-13c-3}
% \end{enumerate}
%
% \SubtaskSolved   En stor fördel är att vi till exempel kan skicka med en immutable som argument till en metod och vara säkra på att metoden inte ändrar på värdet.
%
%
% \QUESTEND
%
%
%
%
% %%<AUTOEXTRACTED by mergesolu>%%      %Uppgift 13
%
%
%
%
% \WHAT{Några användbara samlingar.}
%
% \QUESTBEGIN
%
% \Task  \what~  En \textbf{samling} \Eng{collection} är en datastruktur som samlar många objekt av samma typ. I \code{scala.collection} och \code{java.util} finns många olika samlingar med en uppsjö användbara metoder. De olika samlingarna i \code{scala.collection} är ordnade i en gemensam hierarki med många gemensamma metoder; därför har man nytta av det man lär sig om metoderna i en Scala-samling när man använder en annan samling. Vi har redan tidigare sett samlingen \code{Vector}:
%
% \begin{REPL}
% scala> val tärningskast = Vector.fill(10000)((math.random() * 6 + 1).toInt)
% scala> tä   // tryck TAB
% scala> tärningskast.  // tryck TAB
% \end{REPL}
%
% \Subtask Ungefär hur många metoder finns det som man kan göra på objekt av typen \code{Vector}? Det är svårt att lära sig alla dessa på en gång, så vi väljer ut några få i kommande uppgifter.
%
% \Subtask Jämför överlappet mellan metoderna i \code{Vector} och \code{List} och uppskatta hur stor andel av metoderna som är gemensamma:
% \begin{REPL}
% scala> val myntkast =
%          List.fill(10000)(if (math.random() < 0.5) "krona" else "klave")
% scala> my   // tryck TAB
% scala> myntkast.  // tryck TAB
% \end{REPL}
%
% \SOLUTION
%
%
% \TaskSolved \what
%
%
% \SubtaskSolved   Ungefär 150 metoder.
%
% \SubtaskSolved   Ungefär lika många.
%
%
% \QUESTEND
%
%
%
%
% %%<AUTOEXTRACTED by mergesolu>%%      %Uppgift 14
%
%
%
%
% \WHAT{Typparameter.}
%
% \QUESTBEGIN
%
% \Task  \what~  Vissa funktioner är generella för många typer och tar en så kallad \textbf{typparameter} inom hakparenteser. Ofta slipper man skriva typparametrar, då kompilatorn kan härleda typen utifrån argumenten. Om man anger typparametrar explicit så hjälper kompilatorn dig med att kolla att det verkligen är rätt typ i samlingen.
%
% \Subtask Vad händer nedan?
% \begin{REPL}
% scala> var xs = Vector.empty[Int]
% scala> xs = xs :+ "42"
% scala> xs = xs :+ 43 :+ 64 :+ 46
% scala> xs
% scala> xs :+= "42".toInt
% scala> var ys = Vector[Int]("ett", "två", "tre")
% scala> var ingenting = Vector.empty
% scala> ingenting = Vector(1,2,3)
% \end{REPL}
%
% \Subtask Samlingar är mer användbara om de är \emph{generiska}, vilket innebär att elementens typ avgörs av en typparameter och därför kan vara av vilken typ som helst. Man kan definiera egna funktioner som tar generiska samlingar som parametrar. Förklara vad som händer här:
% \begin{REPL}
% scala> val vego = Vector("gurka", "tomat", "apelsin", "banan")
% scala> val prim = Vector(2, 3, 5, 7, 11, 13)
% scala> def först[T](xs: Vector[T]): T = xs.head
% scala> def sist[T](xs: Vector[T]) = xs.last
% scala> def förstOchSist[T](xs: Vector[T]): (T, T) = (xs.head, xs.last)
% scala> först(vego)
% scala> sist(prim)
% scala> förstOchSist(vego)
% scala> förstOchSist(prim)
% scala> def wrap[T](pair: (T, T))(xs: Vector[T]) = pair._1 +: xs :+ pair._2
% scala> wrap("Odla", "och ät!")(vego)
% scala> wrap("Odla", "och ät!")(vego).mkString(" ")
% \end{REPL}
%
%
%
%
%
% \SOLUTION
%
%
% \TaskSolved \what
%
%
% \SubtaskSolved
% \\1. Instansierar en tom vektor med element av typen int och tilldelar värdet till en variabel xs.
% \\2. Error eftersom \code{xs :+ ''42''} ger en Vector[Any] när Vector[Int] krävs.
% \\3. xs tilldelas ett nytt värde av Vector(43, 64, 46)
% \\4. xs skrivs ut.
% \\5. Lägger till talet 42 i xs.
% \\6. Error: type mismatch
% \\7. Skapar en tom Vector i variablen ingenting
% \\8. error: type mismatch; found: Int(3), required: Nothing
%
% \SubtaskSolved
% Tre metoder skapas: den första för att få första elementet i en lista, och eftersom den definieras med specialtypen T går den att använda med alla vektorer oavsett typen av variabeln i vektorn. Den andra får fram sista elementet och den sista hämtar båda två.
%
% En till function definieras längre ner med  namnet ''wrap'', som tar en lista och lägger till ett element längst fram och ett längst bak.
%
%
% \QUESTEND
%
%
%
%
% %%<AUTOEXTRACTED by mergesolu>%%      %Uppgift 15
%
%
%
%
% \WHAT{Några viktiga samlingsmetoder.}
%
% \QUESTBEGIN
%
% \Task  \what~  Deklarera följande vektorer i REPL.
% \begin{REPL}
% scala> val xs = (1 to 10).toVector
% scala> val a = Vector("abra", "ka", "dabra")
% scala> val b = Vector( "sim", "sala", "bim", "sala", "bim")
% scala> val stor = Vector.fill(100000)(math.random())
% \end{REPL}
% Undersök i REPL vad som händer nedan. Alla dessa metoder fungerar på alla samlingar som är indexerbara sekvenser. Givet deklarationerna ovan: vad har uttrycken nedan för värde och typ? Förklara vad som händer hälp av denna  översikt: \href{http://docs.scala-lang.org/overviews/collections/seqs}{docs.scala-lang.org/overviews/collections/seqs}
%
% \Subtask \code{a(1) + xs(1)}
%
% \Subtask \code{a apply 0}
%
% \Subtask \code{a.isDefinedAt(3)}
%
% \Subtask \code{a.isDefinedAt(100)}
%
% \Subtask \code{stor.length}
%
% \Subtask \code{stor.size}
%
% \Subtask \code{stor.min}
%
% \Subtask \code{stor.max}
%
% \Subtask \code{a indexOf "ka"}
%
% \Subtask \code{b.lastIndexOf("sala")}
%
% \Subtask \code{"först" +: b   //minnesregel: colon on the collection side}
%
% \Subtask \code{a :+ "sist"    //minnesregel: colon on the collection side}
%
% \Subtask \code{xs.updated(2,42)}
%
% \Subtask \code{a.padTo(10, "!")}
%
% \Subtask \code{b.sorted}
%
% \Subtask \code{b.reverse}
%
% \Subtask \code{a.startsWith(Vector("abra", "ka"))}
%
% \Subtask \code{"hejsan".endsWith("san")}
%
% \Subtask \code{b.distinct}
%
%
%
% \SOLUTION
%
%
% \TaskSolved \what
%
%
% \SubtaskSolved   String = ''ka2''
%
% \SubtaskSolved   String = ''abra''
%
% \SubtaskSolved   false
%
% \SubtaskSolved   false
%
% \SubtaskSolved   100000
%
% \SubtaskSolved   100000
%
% \SubtaskSolved   minsta talet i listan
%
% \SubtaskSolved   största talet i listan
%
% \SubtaskSolved   1
%
% \SubtaskSolved   3
%
% \SubtaskSolved   Vektor b fast med ''först'' som första element
%
% \SubtaskSolved   Vektor a fast med ''sist'' som sista element.
%
% \SubtaskSolved   plats 3 i vektorn xs får värdet 42
%
% \SubtaskSolved   En ny vektor fylld med ''!'' från och med plats 4 till 10. Men de andra värdena samma som i a.
%
% \SubtaskSolved   b sorterad i bokstavsordning
%
% \SubtaskSolved   b baklänges
%
% \SubtaskSolved   true
%
% \SubtaskSolved   true
%
% \SubtaskSolved   en vektor med alla unika element i b.
%
%
% \QUESTEND
%
%
%
%
% %%<AUTOEXTRACTED by mergesolu>%%      %Uppgift 16
%
%
%
%
% \WHAT{Några generella samlingsmetoder.}
%
% \QUESTBEGIN
%
% \Task  \what~  Det finns metoder som går att köra på \emph{alla} samlingar även om de inte är indexerbara. Givet deklarationerna i föregående uppgift: vad har uttrycken nedan för värde och typ? Förklara vad som händer med hjälp av dessa översikter: \\ \href{http://docs.scala-lang.org/overviews/collections/trait-traversable}{docs.scala-lang.org/overviews/collections/trait-traversable} \\ \href{http://docs.scala-lang.org/overviews/collections/trait-iterable}{docs.scala-lang.org/overviews/collections/trait-iterable}
%
% \Subtask \code{a ++ b}
%
% \Subtask \code{a ++ stor}
%
% \Subtask \code{val ys = xs.map(_ * 5)}
%
% \Subtask \code{b.toSet     // En mängd har inga dubletter}
%
% \Subtask \code{a.head + b.last}
%
% \Subtask \code{a.tail}
%
% \Subtask \code{a.head +: a.tail == a}
%
% \Subtask \code{Vector(a.head) ++ Vector(b.last)}
%
% \Subtask \code{a.take(1) ++ b.takeRight(1)}
%
% \Subtask \code{a.drop(2) ++ b.drop(1).dropRight(2)}
%
% \Subtask \code{a.drop(100)}
%
% \Subtask \code{val e = Vector.empty[String]; e.take(100)}
%
% \Subtask \code{Vector(e.isEmpty, e.nonEmpty)}
%
% \Subtask \code{a.contains("ka")}
%
% \Subtask \code{"ka" contains "a"}
%
% \Subtask \code{a.filter(s => s.contains("k")) }
%
% \Subtask \code{a.filter(_.contains("k")) }
%
% \Subtask \code{a.map(_.toUpperCase).filterNot(_.contains("K")) }
%
% \Subtask \code{xs.filter(x => x % 2 == 0)}
%
% \Subtask \code{xs.filter(_ % 2 == 0)}
%
%
% \SOLUTION
%
%
% \TaskSolved \what
%
%
% \SubtaskSolved
% Metoden ger tillbaka en ny Vector[String] som nu består av alla element i a plus alla element i b. I samma ordning med elementen i a först.
%
% \SubtaskSolved
% Samma som i uppgift a fast vektorn som returnas är av typen Vector[Any]. Det är eftersom Any är den närmsta typen som String och Double delar. Elementen från vektor a är fortfarande först och uppföljt av elementen i stor.
%
% \SubtaskSolved
% Variablen ys får värdet av en Vector[Int] som innehåller alla talen från xs fast multiplicerade med 5. Alltså ys = 5, 10, 15..., osv.
%
% \SubtaskSolved
% Functionen tar alla värden från en Vektor och sätter in i ett Set (mängd). Eftersom en mängd ej har dubletter så försvinner ett ''sala'' och ett ''bim'', Vector[String] som returneras blir därför (''sim'', ''sala'', ''bim'').
%
% \SubtaskSolved
% Metoden head ger första elementet i en samling, och last sista. Därför blir kombinationen av a.head och b.last en ny Vector[String] som består av a:s första element, och b:s första element.
%
% \SubtaskSolved
% Ger en Vector[String] som innehåller alla element efter det första. Alltså i detta fallet ''ka'' och ''dabra''.
%
% \SubtaskSolved
% True, eftersom head ger första elementet och tail ger resten, sedan sätter metoden +: ihop dem till en vektor med samma värden som a.
%
% \SubtaskSolved
% Eftersom ++ sätter ihop alla värden från två vektorer måste vi först omvandla från en sträng till vektor. Resultatet blir en ny vektor av samma typ som innan med a:s första element och b:S sista.
%
% \SubtaskSolved
% Samma resultat som i h, metoden take börjar från vänster och tar så många element som man skickar med som parameter och gör till en ny lista. Med 1 som parameter motsvarar det att göra Vector(a.head). Metoden takeRight gör samma sak fast från höger.
%
% \SubtaskSolved
% Metoden drop är motsvarigheten till take fast exkluderar de specifierade elementen istället för att inkludera dem i vektorn.
%
% \SubtaskSolved
% Eftersom a endast innehåller 3 element returnerar drop(100) en tom vektor.
%
% \SubtaskSolved
% Returnerar en tom vektor med element typen String
%
% \SubtaskSolved
% returnerar Vector(true, false)
%
% \SubtaskSolved
% True, metoden contains kollar om en samling innehåller ett specifikt element.
%
% \SubtaskSolved
% True. Eftersom en sträng även kan ses som Vector[Char].
%
% \SubtaskSolved
% Filtrerar vektorn a till att endast innehålla strängar som innehåller k.
%
% \SubtaskSolved
% Exakt samma som i p
%
% \SubtaskSolved
% map(\_.toUpperCase) omvandlar alla strängar i a till stora bokstäver
% filterNot(\_.contains(''K'')) tar resultatet vi precis fick och tar bort alla strängar som innehåller stora K.
%
% \SubtaskSolved
% filtrerar så att endast jämna tal finns kvar.
%
% \SubtaskSolved
% Exakt samma som i s
%
%
%
%
% \QUESTEND
%
%
%
%
% %%<AUTOEXTRACTED by mergesolu>%%      %Uppgift 17
%
%
%
%
% \WHAT{NEEDS A TOPIC DESCRIPTION}
%
% \QUESTBEGIN
%
% \Task  \what~ De olika samlingarna i \code{scala.collection} används flitigt i andra paket, exempelvis \code{scala.util} och \code{scala.io}.
%
% \Subtask Vad händer här? (Metoden \code{shuffle} skapar en ny samling med elementen i slumpvis ordning.)
% \begin{REPL}
% val xs = Vector(1,2,3)
% def blandat = scala.util.Random.shuffle(xs)
% def test = if (xs == blandat) "lika" else "olika"
% (for(i <- 1 to 100) yield test).count(_ == "lika")
% \end{REPL}
%
%
% \Subtask Skapa en textfil med namnet \code{fil.txt} som innehåller lite text och läs in den med: \\\code{scala.io.Source.fromFile("fil.txt", "UTF-8").getLines.toVector}
% \begin{REPL}
% > cat > fil.txt
% hejsan
% svejsan
% > scala
% scala> val xs = scala.io.Source.fromFile("fil.txt", "UTF-8").getLines.toVector
% scala> xs.foreach(println)
% \end{REPL}
%
%
% \Subtask Vad händer här? (Metoden \code{trim} på värden av typen \code{String} ger en ny sträng med blanktecken i början och slutet borttagna.)
% \begin{REPL}
% scala> val pgk =
%   scala.io.Source.fromURL("https://lunduniversity.github.io/pgk","UTF-8").getLines.toVector
% scala> pgk.foreach(println)
% scala> pgk.map(_.trim).
%          filterNot(_.startsWith("<")).
%          filterNot(_.isEmpty).
%          foreach(println)
% \end{REPL}
%
%
%
% \SOLUTION
%
%
% \TaskSolved \what
%
%
% \SubtaskSolved
% Vi instansierar en vektor xs med talen 1, 2 och 3.
% sedan definierar vi en metod blandat som ger oss en randomiserad version av xs.
% sedan definierar vi en till metod som testar om xs är lika med resultatet från blandat. Om det är så returnerar den strängen ''lika'' annars ''olika''.
% Sist kör vi en for-loop där vi 100 gånger kör testet, samtidigt räknas hur många gånger ''lika'' returneras.
%
% Vårt resultat är en siffra på hur många gånger xs var samma som en blandad version av sig själv, eftersom det finns 6 permutationer med 3 variabler så borde det vara ungefär 1/6 chans.
%
% \SubtaskSolved  -
%
% \SubtaskSolved
% \\ \code{map(\_.trim)} tar bort alla onödiga mellanrum i början och slutet på varje rad
% \\ \code{filterNot(\_.startsWith(''<''))} filtrerar bort alla rader som börjar med strängen ''<''
% \\ \code{filterNot(\_.isEmpty)} filtrerar bort alla tomma rader.
% \\ \code{foreach(println)} skriver ut alla rader.
%
%
% \QUESTEND
%
%
%
%
% %%<AUTOEXTRACTED by mergesolu>%%      %Uppgift 18
%
%
%
%
% \WHAT{Jämföra List och Vector.}
%
% \QUESTBEGIN
%
% \Task  \what~  En indexerbar sekvens av värden kallas vektor eller lista. I Scala finns flera klasser som kan kan indexeras, däribland klasserna \code{Vector} och \code{List}.
%
% \Subtask \emph{Likheter mellan \code{Vector} och \code{List}.} Kör nedan rader i REPL. Prova indexera i båda och studera hur stor andel av metoderna som är gemensamma.
% \begin{REPL}
% scala> val sv = Vector("en", "två", "tre", "fyra")
% scala> val en = List("one", "two", "three", "four")
% scala> sv(0) + sv(3)
% scala> en(0) + en(3)
% scala> sv. //tryck TAB
% scala> en. //tryck TAB
% \end{REPL}
%
% \Subtask \emph{Skillnader mellan \code{Vector} och \code{List}.} Klassen \code{Vector} i Scala har ''under huven'' en avancerad datastruktur i form av ett s.k. självbalanserande träd, vilket gör att \code{Vector} är snabbare än \code{List} på nästan allt, \emph{utom} att bearbeta elementen i \emph{början} av sekvensen; vill man lägga till och ta bort i början av en \code{List} så kan det ibland gå ungefär dubbelt så fort jämfört med \code{Vector}, medan alla andra operationer är lika snabba eller snabbare med \code{Vector}. Det finns ett fåtal speciella metoder, som bara finns i \code{List}, för att skapa en lista och lägga till i början av en lista. Vad händer nedan?
%
% \begin{REPL}
% scala> var xs = "one" :: "two" :: "three" :: "four" :: Nil
% scala> xs = "zero" :: xs
% scala> val ys = xs.reverse ::: xs
% \end{REPL}
%
%
% \SOLUTION
%
%
% \TaskSolved \what
%
%
% \SubtaskSolved
% I princip alla metoder delas, en lista har några fler t. ex. ''::'', '':::'', ''mapConserve'' osv.
%
% \SubtaskSolved
% Först skapas en lista med 4 sträng värden och instansierar variablen xs med det värdet.
% sedan skapar vi en ny lista, som består av ''zero'' + den gamla listan och ger värdet till xs.
% Sist instansierar vi en ny variabel ys, som får värdet av xs omvänd plus xs.
%
%
% \QUESTEND
%
%
%
%
% %%<AUTOEXTRACTED by mergesolu>%%      %Uppgift 19
%
%
%
%
% \WHAT{Mängd.}
%
% \QUESTBEGIN
%
% \Task  \what~  En mängd är en samling som garanterar att det inte finns några dubbletter. Det går dessutom väldigt snabbt, även i stora mängder, att kolla om ett element finns eller inte i mängden. Elementen i samlingen \code{Set} hamnar ibland, av effektivitetsskäl, i en förvånande ordning.
% \begin{REPL}
% scala> val s = Set("Malmö", "Stockholm", "Göteborg", "Köpenhamn", "Oslo")
% s: scala.collection.immutable.Set[String] =
%      Set(Oslo, Malmö, Köpenhamn, Stockholm, Göteborg)
%
% scala> val t = Set("Sverige", "Sverige", "Sverige", "Danmark", "Norge")
% t: scala.collection.immutable.Set[String] = Set(Sverige, Danmark, Norge)
% \end{REPL}
% Givet ovan deklarationer: vad blir värde och typ av nedan uttryck?
%
% \Subtask \code{s + "Malmö" == s}
%
% \Subtask \code{s ++ t}
%
% \Subtask \code{Set("Malmö", "Oslo").subsetOf(s)}
%
% \Subtask \code{s subsetOf Set("Malmö", "Oslo")}
%
% \Subtask \code{s contains "Lund"}
%
% \Subtask \code{s apply "Lund"}
%
% \Subtask \code{s("Malmö")}
%
% \Subtask \code{s - "Stockholm"}
%
% \Subtask \code{t - ("Norge", "Danmark", "Tyskland")}
%
% \Subtask \code{s -- t}
%
% \Subtask \code{s -- Set("Malmö", "Oslo")}
%
% \Subtask \code{Set(1,2,3) intersect Set(2,3,4)}
%
% \Subtask \code{Set(1,2,3) & Set(2,3,4)}
%
% \Subtask \code{Set(1,2,3) union Set(2,3,4)}
%
% \Subtask \code{Set(1,2,3) | Set(2,3,4)}
%
%
% \SOLUTION
%
%
% \TaskSolved \what
%
%
% \SubtaskSolved
% true, Boolean
%
% \SubtaskSolved
% En samling av alla värden i s och t, Set[String]
%
% \SubtaskSolved
% true, Boolean
%
% \SubtaskSolved
% false, Boolean
%
% \SubtaskSolved
% false, Boolean
%
% \SubtaskSolved
% false, Boolean
%
% \SubtaskSolved
% true, Boolean
%
% \SubtaskSolved
% Samlingen s utan elementet ''Stockholm'', Set[String]
%
% \SubtaskSolved
% Samlingen t utan elementen ''Norge'' och ''Danmark'', Set[String]
%
% \SubtaskSolved
% returnerar s, Set[String]
%
% \SubtaskSolved
% Samlingen s utan ''Malmö'' och ''Oslo'', Set[String]
%
% \SubtaskSolved
% Set(2, 3), Set[Int]
%
% \SubtaskSolved
% se deluppgift l
%
% \SubtaskSolved
% Set(1, 2, 3 ,4), Set[Int]
%
% \SubtaskSolved
% se deluppgift n
%
%
% \QUESTEND
%
%
%
%
% %%<AUTOEXTRACTED by mergesolu>%%      %Uppgift 20
%
%
%
%
% \WHAT{Slå upp värden från nycklar med \code{Map}.}
%
% \QUESTBEGIN
%
% \Task  \what~  Samlingen \code{Map} är mycket användbar. Med den kan man snabbt leta upp ett värde om man har en nyckel. Samlingen \code{Map} är en generalisering av en vektor, där man kan ''indexera'', inte bara med ett heltal, utan med vilken typ av värde som helst, t.ex. en sträng. Datastrukturen \code{Map} är en s.k. \emph{associativ array}\footnote{\href{https://en.wikipedia.org/wiki/Associative_array}{https://en.wikipedia.org/wiki/Associative\_array}}, implementerad som en s.k. \emph{hashtabell}\footnote{\href{https://en.wikipedia.org/wiki/Hash_table}{https://en.wikipedia.org/wiki/Hash\_table}}.
% \begin{REPL}
% scala> var huvudstad =
%   Map("Sverige" -> "Stockholm", "Norge" -> "Oslo", "Skåne" -> "Malmö")
% \end{REPL}
% Givet ovan variabel \code{huvudstad}, förklara vad som händer nedan?
%
% \Subtask \code{huvudstad apply "Skåne"}
%
% \Subtask \code{huvudstad("Sverige")}
%
% \Subtask \code{huvudstad.contains("Skåne")}
%
% \Subtask \code{huvudstad.contains("Malmö")}
%
% \Subtask \code{huvudstad += "Danmark" -> "Köpenhamn"}
%
% \Subtask \code{huvudstad.foreach(println)}
%
% \Subtask \code{huvudstad getOrElse ("Norge", "???") }
%
% \Subtask \code{huvudstad getOrElse ("Finland", "???") }
%
% \Subtask \code{huvudstad.keys.toVector.sorted}
%
% \Subtask \code{huvudstad.values.toVector.sorted}
%
% \Subtask \code{huvudstad - "Skåne"}
%
% \Subtask \code{huvudstad - "Jylland"}
%
% \Subtask \code{huvudstad = huvudstad.updated("Skåne","Lund") }
%
%
%
% \SOLUTION
%
%
% \TaskSolved \what
%
%
% \SubtaskSolved
% Returnerar strängen ''Malmö'' eftersom det värdet är indexerat på platsen ''Skåne''.
%
% \SubtaskSolved
% Returnerar strängen ''Stockholm'' eftersom det värdet är indexerat på platsen ''Sverige''.
%
% \SubtaskSolved
% true, eftersom huvudstad innehåller indexet ''Skåne''
%
% \SubtaskSolved
% false, eftersom huvudstad ej innehåller indexet ''Malmö''. Notera att det är index och inte värden vi
% kollar om det finns.
%
% \SubtaskSolved
% Lägger till indexet ''Danmark'' med värdet ''Köpenhamn'' i samlingen.
%
% \SubtaskSolved
% Skriver ut alla 2-tupler.
%
% \SubtaskSolved
% Returnerar ''Oslo'', Note: Om indexet ''Norge'' inte hade funnits hade ''???'' returnerats istället.
%
% \SubtaskSolved
% Returnerar ''???''
%
% \SubtaskSolved
% Returnerar en sorterar vektor med alla index.
%
% \SubtaskSolved
% Returnerar en sorterar vektor med alla värden.
%
% \SubtaskSolved
% Returnerar en ny mängd men med ''Skåne'' -> ''Malmö'' borttaget.
%
% \SubtaskSolved
% Returnerar huvudstad mängden eftersom det inte finns ett ''Jylland'' index att ta bort.
%
% \SubtaskSolved
% Uppdaterar indexet ''Skåne'' till att istället leda till värdet ''Lund''
%
%
% \QUESTEND
%
%
%
%
% %%<AUTOEXTRACTED by mergesolu>%%      %Uppgift 21
%
%
%
%
% \WHAT{Skapa Map från en samling.}
%
% \QUESTBEGIN
%
% \Task  \what~
%
% \Subtask Definiera denna vektor och undersök dess typ:
% \begin{Code}
% val pairs = Vector(
%   ("Björn", 46462229009L),
%   ("Maj", 46462221667L),
%   ("Gustav", 46462224906L))
% \end{Code}
%
% \Subtask Vad har variablen \code{telnr} nedan för typ: \\ \code{var telnr = pairs.toMap}
%
% \Subtask Använd \code{telnr} för att slå upp telefonnummer för Maj och Kim med hjälp av metoderna \code{apply} och \code{get}.
%
% \Subtask Använd metoden \code{getOrElse} vid upplagningar av \code{telnr} och ge \code{-1L} som telefonnummer i händelse av att ett nummer inte finns.
%
% \Subtask Lägg till \code{("Fröken Ur", 464690510L)} i \code{telnr}-mappen.
%
% \Subtask Skapa en \code{Vector[(String, String)]} enligt nedan, så att telefonnumret blir en sträng utan inledande landsnummer men med en nolla i riktnumret. Byt ut \code{???} mot lämpligt uttryck.
% \begin{REPL}
% scala> telnr.toVector.map(p => ???)
% res85: Vector[(String, String)] = Vector(("Björn", "0462229009"), ("Maj",
% "0462221667"), ("Gustav", "0462224906"), ("Fröken Ur", 04690510"))
%
% \end{REPL}
%
% \Subtask Använd vektorn i resultatet ovan för att skapa en ny \code{Map[String, String]} med nationella telefonnumer. Slå upp numret till Fröken Ur.
%
% \SOLUTION
%
%
% \TaskSolved \what
%
%
% \SubtaskSolved
% \begin{REPLnonum}
% pairs: scala.collection.immutable.Vector[(String, Long)] =
% 					Vector((Björn,444), (Maj,441), (Lucy,666))
% \end{REPLnonum}
%
% \SubtaskSolved
% Map[String, Long]
%
% \SubtaskSolved
% \begin{REPLnonum}
% scala> telnr(''Maj'')
% res0: Long = 441
%
% scala> telnr.get(''Maj'')
% res1: Option[Long] = Some(441)
%
% scala> telnr(''Kim'')
% java.util.NoSuchElementException: key not found: 'Kim
%   at scala.collection.MapLike$class.default(MapLike.scala:228)
%   at scala.collection.AbstractMap.default(Map.scala:59)
%   at scala.collection.MapLike$class.apply(MapLike.scala:141)
%   at scala.collection.AbstractMap.apply(Map.scala:59)
%   ... 32 elided
%
% scala> telnr.get(''Kim'')
% res2: Option[Long] = None
% \end{REPLnonum}
%
% \SubtaskSolved
% \begin{REPLnonum}
% scala> telnr.getOrElse(''Maj'', -1L)
% res0: Long = 441
%
% scala> telnr.getOrElse(''Kim'', -1L)
% res1: Long = -1
% \end{REPLnonum}
%
% \SubtaskSolved
% telnr += ''Fröken Ur'' -> 464690510L
%
% \SubtaskSolved
% telnr.toVector.map(p => p.\_1 -> (''0'' + p.\_2.toString.substring(2)))
%
% \SubtaskSolved
% Använd metoden toMap och apply.
%
%
%
%
% \QUESTEND
%
%
%
%
% %%<AUTOEXTRACTED by mergesolu>%%      %Uppgift 22
%
%
%
%
% \WHAT{Samlingsmetoden \code{maxBy}.}
%
% \QUESTBEGIN
%
% \Task  \what~  Med samlingsmetoden \code{maxBy} kan man själv definiera vad som ska maximeras. (Denna metod kommer du att behöva i veckans laboration.)
%
% \Subtask Förklara vad som händer nedan.
% \begin{REPL}
% scala> val xs = Vector((2,3), (1,5), (-1, 1), (7, 2))
% scala> xs.maxBy(x => x._1)
% scala> xs.maxBy(x => x._2)
% \end{REPL}
%
% \Subtask Om man bara använder en parameter i en anonym funktion, till exempel parametern \code{x} i lambdauttrycket \code{x => x + 1} \emph{en enda} gång, och kompilatorn kan gissa alla typer, kan man använda understreck som ''platshållare'' för att förkorta lambdauttrycket så här: \code{ _ + 1}
%
% Skriv uttrycken på raderna 2 och 3 i föregående deluppgift på ett kortare sätt med hjälp platshållarsyntax \Eng{place holder syntax}.
%
% \Subtask På motsvarande sätt kan man använda \code{minBy} för att välja vilken funktion som definierar minimum. Prova \code{minBy} på motsvarande sätt som i föregående deluppgifter.
%
% \SOLUTION
%
%
% \TaskSolved \what
%
%
% \SubtaskSolved   Metoden maxBy hämtar det element som är ''störst'', på rad två gör \code{x => x._1} att första värdet i tuplerna används för att bestämma vilken som är störst. Likt gör \code{x => x._2} på rad tre att istället det andra värdet används.
%
% \SubtaskSolved
% \begin{REPLnonum}
% scala> xs.maxBy(_._1)
% scala> xs.maxBy(_._2)
% \end{REPLnonum}
%
% \SubtaskSolved
% \begin{REPLnonum}
% scala> xs.minBy(_._1)
% scala> xs.minBy(_._2)
% \end{REPLnonum}
%
%
%
% \QUESTEND
%
%
%
%
%
%
%
%
% \WHAT{NEEDS A TOPIC DESCRIPTION}
%
% \QUESTBEGIN
%
% \Task  \what~ Skriv nedan program med en editor och kompilera från terminalen. Lägg till kod i huvudprogrammet som testar klassen \code{Account} och kompilera och kör. Utvidga sedan klassen \code{Account} med fler attribut och funktioner som du väljer själv.
%
% \begin{Code}
% class Account(val number: Long, val maxCredit: Int){
%   private var balance = 0
%
%   def deposit(amount: Int): Int = {
%     if (amount > 0) {balance += amount}
%     balance
%   }
%
%   def withdraw(amount: Int): (Int, Int) = if (amount > 0) {
%     val allowedWithdrawal =
%       if (amount < balance + maxCredit) amount
%       else balance + maxCredit
%     balance = balance - allowedWithdrawal
%     (allowedWithdrawal, balance)
%   } else (0, balance)
%
%   def show: Unit =
%     println("Account Nbr: " + number + " balance: " + balance)
% }
%
% object Main {
%   def main(args: Array[String]): Unit = {
%     ???
%   }
% }
% \end{Code}
%
%
%
% \SOLUTION
%
%
% \QUESTEND
%
%
%
%
%
%
% \WHAT{NEEDS A TOPIC DESCRIPTION}
%
% \QUESTBEGIN
%
% \Task \label{task:keno-set} \what~  Läs om reglerna för spelet Keno här: \\ \url{https://sv.wikipedia.org/wiki/Keno} och gör deluppgifterna nedan.
%
% \Subtask Skapa en klass \code{Keno} som kan användas för att genomföra en Kenodragning. Låt klassen ha ett privat attribut \code{balls} som är en föränderlig mängd med heltal och som från början är tom. Implementera lämpliga metoder i klassen för att användaren av klassen ska kunna dra nya slumpmässiga bollar som inte redan är dragna.
%
% \Subtask Skapa en \code{case class KenoBet(bet: Set[Int])} för att hålla reda vilka 11 bollar en viss person satsar på. Definiera en metod \\ \code{def numberOfHits(keno: Keno): Int = ???}\\ i case-klassen \code{KenoBet} som givet en kenodragning räknar ut hur många bollar som satsats rätt.
%
% \Subtask Skriv ett huvudprogram som simulerar en enkel Kenodragning. Låt två personer satsa på 11 slumpmässiga bollar, genomför en dragning av 20 bollar ur 70 möjliga och kontrollera sedan hur många bollar som personerna har prickat rätt.
%
%
%
%
%
% \SOLUTION
%
%
% \QUESTEND
%
%
%
%
%
%
% \WHAT{Dokumentationen för \code{Any}.}
%
% \QUESTBEGIN
%
% \Task  \what~  Undersök vilka metoder som finns i klassen Any här: \href{http://www.scala-lang.org/api/current/scala/Any.html}{http://www.scala-lang.org/api/current/scala/Any.html}. Prova några av metoderna i REPL.
%
% \SOLUTION
%
%
% \QUESTEND
%
%
%
%
%
%
% \WHAT{Dokumentationen för samlingar.}
%
% \QUESTBEGIN
%
% \Task  \what~  Leta upp metoden \code{tabulate} i dokumentationen för objektet \code{Vector} nästan längst ner i listan här: \\ \href{http://www.scala-lang.org/api/current/scala/collection/immutable/Vector.html}{http://www.scala-lang.org/api/current/scala/collection/immutable/Vector.html} \\Leta upp den variant av \code{tabulate} som har signaturen:\\ \code{def tabulate[A](n: Int)(f: (Int) => A): Vector[A] }\\ Klicka på den gråfyllda trekanten till vänster om signaturen som fäller ut beskrivningen
%
% \Subtask Förklara vad som händer här:
% \begin{REPLnonum}
% scala> Vector.tabulate(10)(i => i % 3)
% \end{REPLnonum}
%
% \Subtask Klicka på det blåa stora o-et överst på sidan, för att växla till klass-vyn och studera listan med alla metoder  i klassen \code{Vector}.
%
%
% \SOLUTION
%
%
% \QUESTEND
%
%
%
%
%
%
% \WHAT{Fler metoder på indexerbara sekvenser.}
%
% \QUESTBEGIN
%
% \Task  \what~  Deklarera följande vektorer i REPL.
% \begin{REPL}
% scala> val xs = (1 to 10).toVector
% scala> val a = Vector("abra", "ka", "dabra")
% scala> val b = Vector( "sim", "sala", "bim", "sala", "bim")
% \end{REPL}
% Undersök i REPL vad som händer nedan. Alla dessa metoder fungerar på alla samlingar som är indexerbara sekvenser. Vad har uttrycken för värde och typ? Förklara vad metoden gör. Studera även denna  översikt: \href{http://docs.scala-lang.org/overviews/collections/seqs}{docs.scala-lang.org/overviews/collections/seqs}
%
% \Subtask \code{b.indexWhere(s => s.startsWith("b"))}  % advanced
%
% \Subtask \code{a.indices}  % advanced
%
% \Subtask \code{xs.patch(1, Vector(42,43,44), 7)} % advanced
%
% \Subtask \code{xs.segmentLength(_ < 8, 2)} % advanced
%
% \Subtask \code{b.sortBy(_.reverse)}  % advanced
%
% \Subtask \code{b.sortWith((s1, s2) => s1.size < s2.size)} % advanced
%
% \Subtask \code{a.reverseMap(_.size)}	% advanced
%
% \Subtask \code{a intersect Vector("ka", "boom", "pow")} % advanced
%
% \Subtask \code{a diff Vector("ka")} % advanced
%
% \Subtask \code{a union Vector("ka", "boom", "pow")} % advanced
%
%
%
% \SOLUTION
%
%
% \QUESTEND
%
%
%
%
% \WHAT{NEEDS A TOPIC DESCRIPTION}
%
% \QUESTBEGIN
%
% \Task  \what~ För samlingen \code{List} finns en alternativ metod till \code{+:} som heter \code{::} och kallas ''cons'' och som i kombination med objektet \code{Nil} kan användas för att med alternativ syntax bygga listor. Läs om detta här: \\ \href{http://alvinalexander.com/scala/how-create-scala-list-range-fill-tabulate-constructors}{alvinalexander.com/scala/how-create-scala-list-range-fill-tabulate-constructors} \\ och hitta på några egna övningar för att undersöka hur cons och Nil fungerar. Metoder som slutar med kolon är högerassociativa. Läs mer om detta här: \href{http://www.artima.com/pins1ed/basic-types-and-operations.html#5.8}{http://www.artima.com/pins1ed/basic-types-and-operations.html\#5.8}\SOLUTION
%
%
% \QUESTEND

%!TEX encoding = UTF-8 Unicode

%!TEX root = ../compendium2.tex


\Lab{\LabWeekNINE}
\begin{Goals}
%!TEX encoding = UTF-8 Unicode
%!TEX root = ../compendium2.tex

%\item Kunna använda en integrerad utvecklingsmiljö (IDE).
%\item Kunna använda färdiga funktioner för att läsa till, och skriva från, textfil.
%\item Kunna använda enkla case-klasser.
%\item Kunna skapa och använda enkla klasser med föränderlig data.
\item Kunna skapa och använda nyckel-värde-tabeller med samlingstypen \code{Map}.
\item Kunna skapa och använda mängder med samlingstypen \code{Set}.
\item Förstå skillnader och likheter mellan en sekvens och en mängd.
\item Förstå likheter och skillnader mellan en sekvens av par och en nyckel-värde-tabell. 
\item Kunna implementera algoritmer som använder nästlade strukturer. 
%\item Kunna skapa en ny samling från en befintlig samling.
%\item Förstå skillnaden mellan kompileringsfel och exekveringsfel.
%\item Kunna felsöka i små program med hjälp av utskrifter.
%\item Kunna felsöka i små program med hjälp av en debugger i en IDE.

\end{Goals}

\begin{Preparations}
\item \DoExercise{\ExeWeekNINE}{09}
%\item Läs om integrerade utvecklingsmiljöer i appendix \ref{appendix:ide}.
%\item Välj vilken IDE du vill använda på denna lab. %Om du inte vet vilken, välj \textbf{Eclipse} med ScalaIDE, som flest handledare känner väl till.
%\item Bekanta dig med utvecklingsmiljön genom att skapa ett nytt projekt och gör ett ''Hello World''-program.
%\item Ladda hem kursens \emph{workspace} enligt instruktioner i appendix \ref{subsubsection:download--import-workspace} och kontrollera så att du med \emph{Run} kan köra igång de båda ofärdiga \code{main}-metoderna i projektet \code{w04_pirates} inifrån din IDE. Om du inte får rätt på \emph{Run Configuration...} etc. så fråga någon om hjälp.
\item Läs igenom hela laborationen.
\item Hämta och läs given kod via \href{https://github.com/lunduniversity/introprog/tree/master/workspace/w09_words}{kursen github-plats} eller via \href{https://cs.lth.se/pgk/download/}{cs.lth.se/pgk/download}
%\item {\"O}ppna Scala IDE i Eclipse enligt intruktionerna XX.
%\item Skapa ett projekt och skapa ett \code{object Hello} med en \code{main}-metod enligt XY.
%\item Skriv ut en h{\"a}lsning till terminalen med \code{println("...")} och testk{\"o}r programmet genom att markera filnamnet i projektmenyn och trycka p{\aa} den gr{\"o}na pilen. Kontrollera att h{\"a}lsningen skrivs ut!
\end{Preparations}


\subsection{Bakgrund}

Denna uppgift handlar om analys av naurligt språk \Eng{Natural Language Processing, NLP}. Språkanalys bygger ofta på statistik över förekomsten av olika ord i långa texter. Du ska skriva kod, som utifrån en lång text, till exempel en bok, kan hjälpa dig att svara på denna typ av frågor:
\begin{itemize}[noitemsep]
\item Hur vanligt är ett visst ord i en given text?
\item Vilket är det vanligaste ordet som följer efter ett visst ord?
\item Hur kan man generera ordsekvenser som liknar ordföljden i en given text?
\end{itemize}

\noindent För att kunna svara på sådana frågor ska du skapa frekvenstabeller och även så kallade \emph{n-gram}; sekvenser av $n$ ord som förekommer i följd i en text. Exempel på några 2-gram (kallas även \emph{bigram}) som finns i föregående mening: (för, att), (att, kunna), (kunna, svara), (svara, på), (på, sådana), och så vidare.\footnote{Du kan undersöka olika n-gram i en stor mängd böcker med hjälp av Googles n-gram-viewer: \url{https://books.google.com/ngrams/}}

\subsection{Obligatoriska uppgifter}

Du ska bygga ditt program med en editor, t.ex. VS \texttt{code}, och kompilera och köra din kod i terminalen med hjälp av \code{scala-cli} i \textit{watch mode} med det arbetssätt som beskrivs i appendix \ref{appendix:build} avsnitt \ref{appendix:build-scala-cli-watch-mode}. Medan du steg för steg utvecklar ditt program, ska du parallellt göra experiment i REPL för att undersöka hur du kan använda samlingsmetoder för att lösa uppgifterna.
Kod att utgå ifrån finns här: \url{https://github.com/lunduniversity/introprog/tree/master/workspace/w09_words}

Dessa ofärdiga kodfiler ligger i paketet \code{nlp}:
\begin{itemize}
  \item \href{https://github.com/lunduniversity/introprog/blob/master/workspace/w09_words/FreqMapBuilder.scala}{\texttt{FreqMapBuilder.scala}} innehåller ett skelett till en klass för att, ord för ord, bygga en nyckel-värde-tabell som registrerar antalet förekomster av olika ord. Att implementera denna ingick i övningen du gjorde tidigare i veckan.

  \item \href{https://github.com/lunduniversity/introprog/blob/master/workspace/w09_words/Text.scala}{\texttt{Text.scala}} innehåller ett skelett till en klass som kan göra textbehandling genom att analysera ord i en text.

  \item \href{https://github.com/lunduniversity/introprog/blob/master/workspace/w09_words/Main.scala}{\texttt{Main.scala}} innehåller ett ofärdigt huvudprogram som du kan använda i laborationens senare del.
\end{itemize}

\Task \emph{Skapa frekvenstabeller}. Du ska använda \code{FreqMapBuilder} från veckans övning för att skapa frekvenstabeller av typen \code{Map[String, Int]}, där nyckel-värde-paren i tabellen anger antalet förekomster av en viss sträng.

\Subtask Lägg klassen \code{FreqMapBuilder} i ett paket som heter \code{nlp} och kompilera.

\begin{figure}[H]
\scalainputlisting[numbers=left,basicstyle=\ttfamily\fontsize{10.5}{12.5}\selectfont]{../workspace/w09_words/FreqMapBuilder.scala}
%\caption{Den ofärdiga klassen \code{FreqMapBuilder}.}
%\label{data:fig-freqmap}
\end{figure}

\Subtask Testa noga så att din \code{FreqMapBuilder} fungerar korrekt. Exempel på test i REPL:
\begin{REPL}
scala> import nlp._

scala> val fmb = FreqMapBuilder("hej", "på", "dej")
fmb: nlp.FreqMapBuilder = nlp.FreqMapBuilder@458f85ef

scala> fmb.add("hej")

scala> fmb.toMap
res0: Map[String,Int] = Map(på -> 1, hej -> 2, dej -> 1)

scala> (1 to Short.MaxValue).foreach(i => fmb.add(i.toString))

scala> fmb.toMap.size
res1: Int = 32770

scala> fmb.toMap
res2: Map[String,Int] = 
  Map(10292 -> 1, 19125 -> 1, 26985 -> 1, 29301 -> 1, 5451 -> 1, 4018 -> 1, 31211 -> 1, ...
\end{REPL}

\noindent I kommande uppgifter ska du steg för steg skapa och testa case-klassen \code{Text}. %figur \ref{data:fig-text}.

\begin{figure}[H]
\scalainputlisting[numbers=left,basicstyle=\ttfamily\fontsize{10.4}{12.5}\selectfont]{../workspace/w09_words/Text.scala}
%\caption{Den ofärdiga klassen \code{Text}.}
%\label{data:fig-text}
\end{figure}





\Task \emph{Dela upp en sträng i ord}. Du ska implementera medlemmen \code{words}. Den ska innehålla en vektor med alla ord i \code{source}, utan andra tecken än bokstäver.
Detta åstadkommer du genom att utgå ifrån strängen \code{source} och i tur och ordning göra följande:
\begin{enumerate}%[nolistsep, noitemsep]
\item byta ut alla tecken i \code{source} för vilka \code{isLetter} är falskt mot \code{' '}
\item dela upp strängen från föregående steg i en array av strängar med \code{split(' ')}
\item filtrera bort alla tomma strängar
\item gör om alla bokstäver i alla strängar till små bokstäver
\item gör om arrayen till en sekvens av typen \code{Vector[String]}.
\end{enumerate}

\noindent Testa så att \code{words}, och de värden som använder \code{words}, fungerar i REPL:
\begin{REPL}
scala> val t = Text("Gurka är ingen tomat, men gurka är en grönsak.")

scala> t.words
res1: Vector[String] =
  Vector(gurka, är, ingen, tomat, men, gurka, är, en, grönsak)

scala> t.distinct
res2: Vector[String] =
  Vector(gurka, är, ingen, tomat, men, en, grönsak)

scala> t.wordSet
res3: Set[String] = Set(grönsak, är, gurka, men, ingen, tomat, en)

scala> t.wordsOfLength(5)
res4: Set[String] = Set(gurka, ingen, tomat)

\end{REPL}



\Task Du ska nu skapa ordfrekvenstabellen \code{wordFreq} genom att registrera ordförekomster med hjälp av \code{FreqMapBuilder}. Tabellen \code{wordFreq} ska bestå av nyckelvärdepar \code{w -> f} där \code{f} är antalet gånger ordet \code{w} förekommer i \code{words}. Testa \code{wordFreq} genom att ladda ner boken ''Skattkammarön'' skriven av Robert Louis Stevenson\footnote{Copyright för denna bok har gått ut, så du gör dig inte skyldig till piratkopiering (i juridisk mening).} och undersök frekvensen för olika vanliga ord. Vilket ord är vanligast? Näst vanligast?

\begin{REPL}[basicstyle=\color{white}\ttfamily\fontsize{9}{11}\selectfont]
scala> val piratbok = Text.fromURL("https://fileadmin.cs.lth.se/pgk/skattkammaron.txt")
val piratbok: nlp.Text = Text(Herr Trelawney, doktor Livesey och de övriga herrarna har bett mig att skriva ner alla omständigheter kring Skattkammarön, ...

scala> piratbok.words.size
val res0: Int = 69438

scala> piratbok.wordFreq("pirat")
val res1: Int = 7
\end{REPL}
Länkar till böcker i UTF-8-format som du kan använda i dina tester:
\begin{itemize}%[nolistsep,noitemsep]
\item ''Skattkammarön'' av R. L. Stevenson: \\\url{https://fileadmin.cs.lth.se/pgk/skattkammaron.txt}
\item ''Inferno'' av August Stringberg: \\\url{https://fileadmin.cs.lth.se/pgk/inferno.txt}
\item ''Pride and Prejudice'' av Jane Austen: \\\url{https://fileadmin.cs.lth.se/pgk/prideandprejudice.txt}
\item Projekt Gutenberg med många fritt tillgängliga böcker i textformat: \\\url{https://www.gutenberg.org/}
\end{itemize}






\Task Implementera metoden \code{ngrams} som ger en sekvens med alla ordföljder i $n$ steg. \emph{Tips:} På veckans övning ingick att undersöka hur metoden \code{sliding} fungerar, med vilken du kan skapa $n$-gram. Gör \code{toVector} på resultatet från \code{sliding}. Testa noga så att \code{ngrams} och \code{bigrams} fungerar korrekt innan du går vidare.
\begin{REPL}
scala> piratbok.ngrams(3).take(2)
val res1: Vector[Vector[String]] =
  Vector(Vector(herr, trelawney, doktor), Vector(trelawney, doktor, livesey))

scala> piratbok.bigrams.take(2)
val res2: Vector[(String, String)] =
  Vector((herr,trelawney), (trelawney,doktor))
\end{REPL}

\Task Implementera \code{followFreq}, som ska innehålla en nyckel-värde-tabell där värdet i sin tur är en frekvenstabell över de ord som kommer efter nyckeln. \label{task-follow-freq}

Genom att analysera alla ordpar kan vi få fram vilket som är det vanligaste ordet som följer efter ett givet ord. Metoden \code{bigrams} ger oss alla ordpar \code{(w1, w2)} där \code{w2} följer efter \code{w1}. Vi kan spara statistiken över efterföljande ord i en nyckelvärdetabell med mappningarna \code{w -> f} där nyckeln \code{w} är ett ord  och värdet \code{f} är en frekvenstabell av typen \code{Map[String, Int]}. I frekvenstabellen lagrar vi frekvensen för alla de ord som följer efter \code{w}. Du ska alltså bygga en nästlad tabell av typen \code{Map[String, Map[String, Int]]}. Rita en bild av den nästlade strukturen.\Pen

Implementera metoden followFreq genom att utgå från nedan pseudokod:
\begin{Code}
val result = collection.mutable.Map.empty[String, FreqMapBuilder]
for (key, next) <- bigrams do
  if /* key finns redan definierad i result */ then
    /* på "platsen" result(key): lägg till next i frekvenstabellen */
  else
    /* lägg till (key -> ny frekvenstabell med next) i result*/
end for
result.map(p => p._1 -> p._2.toMap).toMap // toMap ger oföränderlig Map
\end{Code}
Gör utskrifter för att ta reda på följande frågor. Skriv ner svaren och var redo att redovisa dem i samband med kontrollfrågorna (se avsnitt \ref{words-check}).\Pen

\Subtask Vilka ord brukar följa efter \emph{han} respektive \emph{hon} i Stevensons ''Skattkammarön''?

\Subtask Vilka ord brukar följa efter \emph{han} respektive \emph{hon} i Stringbergs ''Inferno''?

\Subtask Vilka ord brukar följa efter \emph{he} respektive \emph{she} i Austens ''Pride and Prejudice''?


\Task Skapa ett huvudprogram som rapporterar valfria, intressanta mått om orden i en text. Programmet ska ta textens källa som argument, givet som en URL eller ett filnamn. Skriv huvudprogrammet i filen \code{Main.scala} i ett singelobjekt med namnet \code{Main}. Exempel på en rapport som ditt huvudprogram kan generera finns nedan. Här ges även ett heltal som argument som styr topplistornas längd.
\begin{REPL}
> scala run . -- https://fileadmin.cs.lth.se/pgk/skattkammaron.txt 13

Källa: https://fileadmin.cs.lth.se/pgk/skattkammaron.txt

*** Antal ord: 69438

*** De 13 vanligaste orden och deras frekvens:
(och,3089), (jag,2007), (att,1594), (det,1382), (en,1262),
(i,1244), (som,1132), (på,1068), (han,1063), (var,990),
(med,854), (den,774), (av,740)

*** De 13 längsta orden och deras längd:
(besättningsmedlemmarnas,23), (befästningsanordningar,22),
(temperamentsuppvisning,22), (undsättningsexpedition,22),
(besättningsmedlemmarna,22), (försiktighetsåtgärder,21),
(undsättningsfartyget,20), (sjukdomsframkallande,20),
(husföreståndarinnans,20), (sjömansterminologin,19),
(parlamentärsflaggan,19), (bregravningsplatsen,19),
(tidvattenströmmarna,19)
\end{REPL}

\noindent Exempel på huvudprogram som kan skapa ovan utskrift:
\scalainputlisting[numbers=left,basicstyle=\ttfamily\fontsize{10.4}{12.5}\selectfont]{../workspace/w09_words/Main.scala}

\Task Para ihop dig med en annan student och planera hur ni tillsammans kan med hjälp av \url{https://cs.lth.se/pgk/muntabot} kan träna inför det muntliga provet där ni ömsesidigt agerar ''låtsasexaminator''. Gör en plan för när ni ska testa varandra på vilka veckor. Visa er plan för handledare och diskutera vad det innebär att vara en bra ''låtsasexaminator''.


\subsection{Kontrollfrågor}\label{words-check}

\begin{enumerate}[noitemsep, nolistsep]

\item Vilket är dina svar på uppgift \ref{task-follow-freq} a) b) c) på sidan \pageref{task-follow-freq}?

\item I vilken ordning hamnar elementen om man anropar \code{distinct} på en sekvens?

\item Om man itererar över en mängd, i vilken ordning behandlas elementen?

\item Ge exempel på när är det lämpligt att använda en mängd i stället för en sekvens av distinkta värden?

\item Är alla nycklar i en nyckel-värde-tabell garanterat unika?

\item Är alla värden i en nyckel-värde-tabell garanterat unika?

\item LTH-teknologen Oddput Clementin vill summera längden på varje sträng i en mängd och skriver:
\begin{REPL}
scala> Set("hej", "på", "dej").map(_.length).sum
res0: Int = 5
\end{REPL}
Varför blir det fel? Hur kan Oddput åtgärda problemet?
\end{enumerate}

\subsection{Frivilliga uppgifter}

\Task Bygg vidare på klassen \code{Text} och implementera nedan metod som ska ge ett slumpmässigt ord ur \code{wordSet}. Varje ord ska förekomma med lika stor sannolikhet.
\begin{Code}
def randomWord: String = ???
\end{Code}

\Task \label{task:words:randomSeq} Med NLP kan man generera slumpmässiga meningar som statistiskt sett liknar ''riktiga'', människoskapade meningar.

Implementera metoden \code{randomSeq(firstWord, n)} nedan i klassen \code{Text}. Den ska ge en sekvens $w_{1}, w_{2}, ..., w_{n}$  där $w_{1}$ är \code{firstWord} och $w_{i+1}$ är något slumpmässigt ord som är draget bland de ord som följer efter $w_{i}$. Detta kan du åstadkomma genom att varje efterföljande ord $w_{i+1}$ väljs ur \code{keys.toVector} för den \code{followFreq}-tabell som hör till $w_{i}$. Orden ska dras ur efterföljandemängden, med lika stor sannolikhet.
\begin{Code}
def randomSeq(firstWord: String, n: Int): Vector[String] = ???
\end{Code}
%\emph{Tips:} Ett sätt att garanterat välja slumpmässigt element med rektangelfördelning ur en sekvens är att använda metoden \code{scala.util.Random.shuffle} som tar en sekvens som argument och genererar en ny sekvens av samma typ, men med elementen ordnade i slumpmässig ordning på ett välblandat sätt, där varje möjlig ordning är lika sannolik.

\Task \label{task:words:mostCommonSeq} För att dina datorgenererade meningar verkligen ska likna mänskligt språk kan vi skapa de mest sannolika meningarna av olika längder ur vår analys av ordfrekvenser.

Lägg till metoden \code{mostCommonSeq} i klassen \code{Text} enligt nedan:
\begin{Code}
def mostCommonSeq(firstWord: String, n: Int): Vector[String] = ???
\end{Code}
\Subtask Implementera metoden så att resultatet blir en sekvens med \code{n} ord. Sekvensen ska börja med \code{firstWord} och därefter följas av det ord som är det \emph{vanligaste} efterföljande ordet efter \code{firstWord}, och därpå det vanligaste efterföljande ordet efter det, etc. \emph{Tips:} Använd en lokal variabel \code{val result} som är en ArrayBuffer till vilken du i en \code{while}-loop lägger de efterföljande orden.

\Subtask Jämför de slumpmässiga sekvenserna med sekvenser genererade med \code{randomSeq} i uppgift \ref{task:words:randomSeq}. Vilka sekvenser liknar mest ''riktiga'' meningar?


\Task Använd befintliga samlingsmetoder i stället för \code{FreqMapBuilder} för att registrera efterföljande ord.

\Subtask Undersök i REPL hur metoden \code{groupBy(x => x)} fungerar då den appliceras på en samling med strängar. Sök efter och studera dokumentationen för \code{groupBy}.

\Subtask Inför värdet \code{lazy val wordFreq2}. Den ska ge samma resultat som \code{wordFreq} men men implementeras med hjälp av \code{groupBy} och \code{map} i stället för \code{FreqMapBuilder}.

\Subtask\Uberkurs Jämför prestanda mellan \code{wordFreq2} och \code{wordFreq}. Vilken är snabbast för stora texter? Är skillnaden stor?

\Subtask Inför värdet \code{lazy val followsFreq2}. Den ska ge samma resultat som \code{followsFreq} men implementeras med hjälp av \code{groupBy} och \code{map} i stället för \code{FreqMapBuilder}.
Denna uppgift är ganska knepig. Experimentera dig fram i REPL, och bygg upp en lösning steg för steg. \emph{Tips:}
\begin{Code}
bigrams
  .groupBy(???)
  .map(p => p._1 -> p._2.map(???).groupBy(???).map(???))
\end{Code}

\Subtask\Uberkurs Jämför prestanda mellan \code{followsFreq2} och \code{followsFreq}. Vilken är snabbast för stora texter? Är skillnaden stor?

\Task \emph{Gör \code{FreqMapBuilder} generisk.} Generiska strukturer, alltså sådana som har typparametrar, är ofta väsentligt mycket mer användbara. Om du gör \code{FreqMapBuilder} generisk genom att införa en typparameter i stället för att hårdkoda typen till \code{String} så kan du använda \code{FreqMapBuilder} med godtycklig elementtyp. 

\Subtask Studera \code{FreqMapBuilder} och identifiera allt i den klassen som är specifikt för typen \code{String}.

\Subtask Inför en typparameter \code{A} inom hakparenteser efter klassnamnet och använd sedan \code{A} i stället för \code{String} i alla metoder.

\Subtask Testa så att din generiska frekvenstabellbyggare fungerar på sekvenser som innehåller annat än strängar.

Detta funkar eftersom inget i \code{FreqMapBuilder} egentligen förutsätter att elementen som ska räknas är av sträng-typ (det räcker att det finns en vettig \code{equals} och \code{hashcode}).


\input{modules/w10-inheritance-chapter.tex}
%\input{generated/w10-chaphead-generated.tex}

%!TEX encoding = UTF-8 Unicode
%!TEX root = ../exercises.tex

\ifPreSolution

\Exercise{\ExeWeekTEN}\label{exe:W10}

\begin{Goals}
\input{modules/w10-inheritance-exercise-goals.tex}
\end{Goals}

\begin{Preparations}
\item \StudyTheory{10}
\end{Preparations}

\BasicTasks

\else

\ExerciseSolution{\ExeWeekTEN}

\BasicTasks

\fi



\WHAT{Para ihop begrepp med beskrivning.}

\QUESTBEGIN

\Task \what

\vspace{1em}\noindent Koppla varje begrepp med den (förenklade) beskrivning som passar bäst:

\begin{ConceptConnections}
\input{generated/quiz-w10-concepts-taskrows-generated.tex}
\end{ConceptConnections}

\SOLUTION

\TaskSolved \what

\begin{ConceptConnections}
\input{generated/quiz-w10-concepts-solurows-generated.tex}
\end{ConceptConnections}

\QUESTEND





\WHAT{Gemensam bastyp.}

\QUESTBEGIN

\Task  \what~  Man vill ofta lägga in objekt av olika typ i samma samling.
\begin{REPL}
scala> class Gurka(val vikt: Int)
scala> class Tomat(val vikt: Int)
scala> val gurkor = Vector(Gurka(100), Gurka(200))
scala> val grönsaker = Vector(Gurka(300), Tomat(42))
\end{REPL}

\Subtask Om en samling innehåller objekt av flera olika typer försöker kompilatorn härleda den mest specifika typen som objekten har gemensamt. Vad blir det för typ på värdet \code{grönsaker} ovan?

\Subtask Försök ta reda på summan av vikterna enligt nedan. Vad ger andra raden för felmeddelande? Varför?

\begin{REPL}
scala> gurkor.map(_.vikt).sum     // fungerar
scala> grönsaker.map(_.vikt).sum  // fungerar inte
\end{REPL}

\Subtask Du ska nu göra så att du kan komma åt vikten på alla grönsaker genom att ge gurkor och tomater en gemensam bastyp som de olika konkreta grönsakstyperna utvidgar med nyckelordet \code{extends}. Det heter att subtyperna \code{Gurka} och \code{Tomat} \textbf{ärver} egenskaperna hos supertypen \code{Grönsak}.

Skapa en bastyp \code{Grönsak} med ett abstrakt attribut \code{vikt}. Låt sedan de konkreta grönsakerna ärva bastypen:

\begin{REPL}
scala> trait Grönsak { val vikt: Int }
scala> class Gurka(val vikt: Int) extends Grönsak
scala> class Tomat(val vikt: Int) extends Grönsak
scala> val gurkor = Vector(Gurka(100), Gurka(200))
scala> val grönsaker = Vector(Gurka(300), Tomat(42))
\end{REPL}
När sker initialisering av attributet \code{vikt}?

\Subtask Vad blir det nu för typ på variabeln \code{grönsaker} ovan?

\Subtask Går det nu att summera vikterna i \code{grönsaker} med uttrycket nedan? Varför?\\ \code{grönsaker.map(_.vikt).sum}


\Subtask En trait liknar en klass, men man kan inte instansiera den direkt. Vad blir det för felmeddelande om du försöker skapa en instans av en trait enligt nedan?
\begin{REPL}
scala> trait Grönsak { val vikt: Int }
scala> new Grönsak
\end{REPL}


\Subtask Traiten \code{Grönsak} har en abstrakt medlem \code{vikt}. Den sägs vara abstrakt eftersom den saknar implementation -- medlemmen har bara ett namn och en typ men inget värde. Du kan instansiera den abstrakta traiten \code{Grönsak} om du fyller i det som ''fattas'', nämligen ett värde på \code{vikt}. Man kan fylla på det som fattas i genom att ''hänga på'' ett block efter typens namn vid instansiering. Man får då vad som kallas en \textbf{anonym klass}, i detta fall en ganska konstig grönsak som inte är någon speciell sorts grönsak med som ändå har en vikt.

Vad får \code{anonymGrönsak} nedan för typ och strängrepresenation?
\begin{REPL}
scala> val anonymGrönsak = new Grönsak { val vikt = 42 }
\end{REPL}

\Subtask Vad blir felmeddelandet om du skapar en anonym klass \code{Grönsak} med en kropp som saknar definition av vikt?

\SOLUTION


\TaskSolved \what


\SubtaskSolved  \code{Vector[Object]}. Typen \code{Object} i JVM är motsvarar typen \code{AnyRef} som är bastyp för alla referenstyper.

\SubtaskSolved  Felmeddelande:
\begin{REPLnonum}
scala> grönsaker.map(_.vikt).sum  
-- Error:                                                                                 
1 |grönsaker.map(_.vikt).sum
  |              ^^^^^^
  |             value vikt is not a member of Object - did you mean wait?
-- Error:
1 |grönsaker.map(_.vikt).sum
  |                         ^
  |ambiguous implicit arguments: both object DoubleIsFractional in object Numeric and object ShortIsIntegral in object Numeric match type Numeric[B] of parameter num of method sum in trait IterableOnceOps
\end{REPLnonum}
Det första felmeddelandet beror på att vektorns element är av typen \code{Object} och medlemmen \code{vikt} är inte definierat för denna typ. Det andra felmeddelandet är ett följdfel som beror på att en sekvens med element av typen \code{Object} inte kan summeras eftersom kompilatorn inte kan härleda att elementtypen är numerisk.

\SubtaskSolved  Attributet \code{vikt} initialiseras vid konstruktion av \code{Gurka} resp. \code{Tomat}. Värdet ges av resp. klassparameter.

\SubtaskSolved  \code{Vector[Grönsak]}.

\SubtaskSolved  Ja. Eftersom den statiska typen för elementen i sekvensen är \code{Grönsak} (den dynamiska typen kan vara godtycklig subtyp av \code{Grönsak}) och alla instanser av denna typ garanterat har attributet \code{vikt} som är av typen \code{Int} så kan kompilatorn vid \emph{kompileringstid} dra slutsatsen att summeringen är giltig och därmed kan kompilatorn kompilera koden till körbar maskinkod.

\SubtaskSolved  
\begin{REPLnonum}
scala> new Grönsak
-- Error:
1 |new Grönsak
  |    ^^^^^^^
  |    Grönsak is a trait; it cannot be instantiated
\end{REPLnonum}

\SubtaskSolved  
\begin{REPLnonum}
scala> val anonymGrönsak = new Grönsak { val vikt = 42 }
val anonymGrönsak: Grönsak = anon$1@1edde8b6
scala> anonymGrönsak.toString                                                                                      
val res0: String = anon$1@1edde8b6
\end{REPLnonum}
Typen är \code{Grönsak} och blir här en s.k. \emph{anonym klass}, eftersom vi inte har använt en namngiven klass med \code{extends}, utan bara ''hängt på'' en klasskropp inom klammerparenteser direkt vid konstruktion. När du skapar anonyma klasser måste du använda nyckelordet \code{new}.

Kompilatorn hittar på ett unikt klassnamn, här anon\$1, för att hålla reda på den anonyma klassen under kompilering till maskinkod. Strängrepresentationen innehåller ett hexadecimalt heltal som är unikt för instansen, här \code{1edde8b6}.

\SubtaskSolved  

\begin{REPLsmall}
scala> new Grönsak { }
-- Error:
1 |new Grönsak { }
  |^
  |object creation impossible, since val vikt: Int in trait Grönsak is not defined 

\end{REPLsmall}


\QUESTEND






\WHAT{Polymorfism vid arv, s.k. subtypspolymorfism.}

\QUESTBEGIN

\Task  \what~  Polymorfism betyder ''många skepnader''. I samband med arv  innebär det att flera subtyper, till exempel \code{Ko} och \code{Gris}, kan hanteras gemensamt som om de vore instanser av samma supertyp, så som \code{Djur}. Subklasser kan implementera en metod med samma namn på olika sätt. Vilken metod som exekveras bestäms vid körtid beroende på vilken subtyp som instansieras. På så sätt kan djur komma i många skepnader.

\Subtask Implementera funktionen \code{skapaDjur} nedan så att den returnerar antingen en ny \code{Ko} eller en ny \code{Gris} med lika sannolikhet.

\begin{REPL}
scala> trait Djur { def väsnas: Unit }
scala> class Ko   extends Djur { def väsnas = println("Muuuuuuu") }
scala> class Gris extends Djur { def väsnas = println("Nöffnöff") }
scala> def skapaDjur(): Djur = ???
scala> val bondgård = Vector.fill(42)(skapaDjur())
scala> bondgård.foreach(_.väsnas)
\end{REPL}

\Subtask Lägg till ett djur av typen Häst som väsnas på lämpligt sätt och modifiera \code{skapaDjur} så att det skapas kor, grisar och hästar med lika sannolikhet.


\SOLUTION


\TaskSolved \what


\SubtaskSolved
\begin{Code}
def skapaDjur(): Djur = 
  if math.random() > 0.5 then Ko() else Gris()
\end{Code}

\SubtaskSolved
\begin{Code}
class Häst extends Djur: 
  def väsnas = println("Gnääääägg") 

def skapaDjur(): Djur = 
   math.random() match
    case r if r < 0.33 => Ko() 
    case r if r < 0.67 => Gris() 
    case _             => Häst()
\end{Code}


\QUESTEND





\WHAT{Olika typer av heltalspar till laborationen \hyperref[section:lab:\LabWeekTEN]{\texttt{\LabWeekTEN}}.}


\QUESTBEGIN


\Task\label{exe:inheritance:labprep-pair}  \what~\textbf{OBS! Gör denna uppgift \textit{innan} du kollar på given kod i labben så att du inte spojlar uppgiften.}

Under veckans laboration ska du använda olika typer av par som representerar riktning och position på en tvådimensionell spelplan, samt spelplanens storlek. I stället för att använda en vanlig 2-tupel till dessa tre olika typer av par ska du skapa egna, specifika  typer som alla ärver bastypen \code{Pair[T]}. Dessa typer ska alla ligga i filen \code{pairs.scala} i \code{package snake}.
\begin{Code}
// detta är en skiss på filen pairs.scala
package snake

trait Pair[T]:
  def x: T
  def y: T
  // uppgift a) lägg till den konkreta metoden tuple

// efterföljande deluppgifterna implementerar dessa subtyper till Pair:
//   case klass Dim beskriver en 2-dimensionell ytas storlek
//   case klass Pos beskriver en position på en yta av Dim storlek
//   enum Dir beskriver förflyttning mot North, South, East, West
\end{Code}
Skillnaden mellan \code{Pair[T]} och en vanlig 2-tupel är att medlemmarna \code{x} och \code{y} garanterat är av \emph{samma} typ, medan en 2-tupel kan innehålla element av olika typ.

I fig. \ref{snake:fig:pairs-uml} visas en bild av klasshierarkin som du steg-för-steg ska utveckla i efterföljande  uppgifter. Fördelen med att ha olika typer av par är att det är mer typsäkert \Eng{type safe}: vi får hjälp av kompilatorn att upptäcka om vi av misstag förväxlar t.ex. en position med en riktning.

\begin{figure}[H]
\begin{center}
\newcommand{\TextBox}[1]{\raisebox{0pt}[1em][0.5em]{#1}}
\tikzstyle{umlclass}=[rectangle, draw=black,  thick, anchor=north, text width=2cm, rectangle split, rectangle split parts = 3]
\begin{tikzpicture}[inner sep=0.5em,scale=1.2, every node/.style={transform shape}]

  \node [umlclass, rectangle split parts = 1, xshift=0cm, yshift=4.5cm] (BaseType1)  {
              \textit{\textbf{\centerline{\TextBox{\code{Pair[T]}}}}}
%              \nodepart[align=left]{second}\code{def x: T} \newline \code{def y: T}
          };


  \node [umlclass, rectangle split parts = 1, xshift=-3cm, yshift=2.5cm] (SubType1)  {
              \textit{\textbf{\centerline{\TextBox{\code{Dim}}}}}
%              \nodepart[align=left]{second}\code{val x: Int} \newline \code{val y: Int}
          };

\node [umlclass, rectangle split parts = 1, xshift=0cm, yshift=2.5cm] (SubType2)  {
            \textit{\textbf{\centerline{\TextBox{\code{Pos}}}}}
%            \nodepart[]{second}\TextBox{\code{val dim: Int}}
        };

\node [umlclass, rectangle split parts = 1, xshift=3cm, yshift=2.5cm] (SubType3)  {
            \textit{\textbf{\centerline{\TextBox{\code{Dir}}}}}
%            \nodepart[]{second}\TextBox{\code{val dim: Int}}
        };


\draw[umlarrow] (SubType1.north) -- ++(0,0.5) -| (BaseType1.south);
\draw[umlarrow] (SubType2.north) -- ++(0,0.5) -| (BaseType1.south);
\draw[umlarrow] (SubType3.north) -- ++(0,0.5) -| (BaseType1.south);

\end{tikzpicture}
\end{center}
\caption{Arvshierarki med \code{Pair[T]} som bastyp.}
\label{snake:fig:pairs-uml}
\end{figure}

\Subtask Öppna en editor och koda \code{trait Pair[T]} i en fil \code{pairs.scala}. Lägg dessutom till en konkret metod \code{tuple} i \code{Pair[T]} som returnerar en 2-tupel med de båda elementen i paret, så att det vid behov går att omvandla \code{Pair}-instanser till 2-tupler. Använd REPL för att testa att detta fungerar:
\begin{REPLnonum}
scala> val p = new Pair[Int] { override val x = 10; override val y = 20 }
p: Pair[Int]{val x: Int; val y: Int} = $anon$1@784223e9

scala> p.tuple
val res0: (Int, Int) = (10,20)
\end{REPLnonum}

\Subtask Fungerar koden ovan även utan nyckelordet \code{override} (testa i REPL)? Varför? När \emph{måste} \code{override} användas? Vad är fördelen resp. nackdelen med att använda \code{override} även när det inte är nödvändigt? 

\Subtask Skapa en case-klass \code{Dim} som ärver \code{Pair[Int]}. Instanser av denna klass kommer du att använda under veckans laboration för att representera en spelplans storlek genom att låta \code{x} ange antalet horisontella positioner och \code{y} antalet vertikala positioner.

Lägg även till ett kompanjonsobjekt \code{Dim} med en \code{apply}-metod som kan skapa \code{Dim}-instanser givet en 2-tupel.
Testa i REPL enligt nedan.
\begin{REPLnonum}
scala> Dim(50, 60)
val res1: Dim = Dim(50,60)

scala> Dim((60, 50))
val res2: Dim = Dim(60,50)

scala> res2.tuple
val res3: (Int, Int) = (60,50)
\end{REPLnonum}

\Subtask Lägg till en case-klass \code{Pos} som ärver \code{Pair[Int]} som representerar en position med en \code{x}-koordinat och en \code{y}-koordinat, båda klassparametrar. Kordinaterna ska hållas inom en spelplansstorlek som ges av klassparametern \code{dim} av typen \code{Dim}. Kordinatpositionerna är heltal och räknas från \code{0} till (men inte med) \code{dim.x} resp. \code{dim.y}.

Gör primärkonstruktorn i case-klassen \code{Pos} \textbf{privat}, genom att skriva nyckelordet \code{private} efter klassnamnet men före klassparameterlistan, så att det inte går att skapa instanser via primärkonstruktorn utanför klasskroppen och kompanjonsobjektet. 

Implementera metoderna \code{+} och \code{-} i case-klassen \code{Pos}. Båda metoderna ska ta en parameter \code{p} av typen \code{Pair[Int]} och returnera en ny \code{Pos}, där \code{p.x} resp. \code{p.y} är adderat resp. subtraherat från aktuell position. Observera att du inte ska skriva \code{new} när du skapar en ny instans, eftersom dessa alltid ska skapas via kompanjonsobjektets \code{apply}-metod, som är en ''smart'' fabriksmetod som garanterar håller koordinaterna inom spelplanen. 

Lägg till ett kompanjonsobjekt \code{Pos} med en \code{apply}-metod som skapar en ny \code{Pos}-instans som ser till att koordinaterna alltid är inom \code{dim}. Aritmetiken ska ske modulo storleken \code{dim}, d.v.s en position ska aldrig kunna hamna utanför spelplanen; i stället så börjar man om på andra sidan (se exempel i REPL nedan). \\ \emph{Tips:} Använd  \code|java.lang.Math.floorMod| som hanterar negativa argument så att resultatet blir positivt (till skillnad från modulo-operatorn \%).

Lägg även till fabriksmetoden \code{random} som kan skapa nya slumpmässiga positioner inom \code{dim}. \emph{Tips:} Använd \code{scala.util.Random.nextInt}.

Testa att det fungerar enligt nedan:
\begin{REPLnonum}
scala> Pos(-1,20,Dim(10,20))
val res4: Pos = Pos(9,0,Dim(10,20))

scala> new Pos(-1,20,Dim(10,20))  // förbjuds med privat primärkonstruktor
-- Error:
1 |new Pos(-1,20,Dim(10,20))
  |    ^^^
  |constructor Pos cannot be accessed as a member of Pos

scala> Pos(0,0,Dim(5,5)) + Pos(6,12, Dim(5,5))                                                                     
val res5: Pos = Pos(1,2,Dim(5,5))

scala> Pos(0,0,Dim(5,5)) - Pos(1,2, Dim(5,5))                                                                     
val res6: Pos = Pos(4,3,Dim(5,5))

scala> for (_ <- 1 to 3) yield Pos.random(Dim(10,10))
val res7: IndexedSeq[Pos] = 
  Vector(Pos(8,8,Dim(10,10)), Pos(2,6,Dim(10,10)), Pos(3,7,Dim(10,10)))
\end{REPLnonum}

\Subtask Vad händer om du glömmer skriva \code{new} när du anropar den privata konstruktorn i din \code{apply}-metod? Varför finns inte detta problem i \code{apply}-metoden för \code{Dim}?

\Subtask Lägg till en \code{enum Dir} som ärver \code{Pair[Int]} och har två \code{val}-parametrar \code{x} och \code{y}. Lägg också till fyra fall med \code{case} som alla ärver \code{Dir} och som representerar en enstegsförflyttning i de fyra väderstrecken, genom att ge parametrarna \code{x} resp. \code{y} något av värden $1$, $-1$ eller $0$. Norrut ska anges med x-koordinaten $0$ och y-koordinaten $-1$, etc. Verifiera i REPL att enumerationen fungerar.

Lägg till en \code{export} som gör så att det räcker att importera \code{snake.*} för att få alla fyra riktningar synliga direkt (annars behövs även import av \code{Dir.*} på alla ställen där riktning används i och utanför paketet \code{snake})


\SOLUTION


\TaskSolved \what

\SubtaskSolved
\begin{CodeSmall}
trait Pair[T]:
  def x: T
  def y: T
  def tuple: (T, T) = (x, y)

\end{CodeSmall}

\SubtaskSolved 
\begin{itemize}
  \item Fungerar koden ovan även utan nyckelordet \code{override}? Varför? 
  \item[] Ja den fungerar eftersom \code{override} ej måste anges när ärvda \emph{abstrakta} medlemmar implementeras i en subtyp. Abstrakta medlemmar saknar implementation och det finns inget som behöver överskuggas. 
  \item När \emph{måste} \code{override} användas? 
  \item[] Det krävs \code{override} om du vill ge en ärvd medlem en \emph{annan} implementation i subtypen, om denna medlemmen redan \emph{har} en implementation i supertypen. Din nya implementation överskuggar (ersätter) den ärvda medlemmens implementation. 
  \item Vad är fördelen resp. nackdelen med att använda \code{override} även när det inte är nödvändigt?
  \item[] \emph{Fördel:} du får hjälp av kompilatorn att kontrollera att du verkligen implementerar en ärvd medlem och inte t.ex. råkat stava medlemmens namn fel.
  \item[] \emph{Nackdel:} mer att skriva och därmed även längre att läsa.
\end{itemize}

\SubtaskSolved
\begin{CodeSmall}
case class Dim(x: Int, y: Int) extends Pair[Int]
object Dim:
  def apply(dim: (Int, Int)): Dim = Dim(dim._1, dim._2)  
\end{CodeSmall}

\SubtaskSolved
\begin{CodeSmall}
case class Pos private (x: Int, y: Int, dim: Dim) extends Pair[Int]:
  def +(p: Pair[Int]): Pos = Pos(x + p.x, y + p.y, dim)
  def -(p: Pair[Int]): Pos = Pos(x - p.x, y - p.y, dim)

object Pos:
  def apply(x: Int, y: Int, dim: Dim): Pos = 
    import java.lang.Math.floorMod as mod
    new Pos(mod(x, dim.x), mod(y, dim.y), dim) //OBS: new nödvändig här!

  def random(dim: Dim): Pos = 
    import scala.util.Random.nextInt as rni
    Pos(rni(dim.x), rni(dim.y), dim)
\end{CodeSmall}

\SubtaskSolved Om du glömmer skriva \code{new} explicit i kompanjonsobjektets \code{apply}-metod så blir det ett rekursivt anrop som resulterar i en oändlig loop vid körtid. Med \code{new} så är det garanterat den privata primärkonstruktorn för \code{Pos} som anropas. 

I \code{Dim.apply} så skiljer sig parametertyperna åt mellan fabriksmetoden och primärkonstruktorn och kompilatorn väljer då primärkonstruktorn eftersom den passar med de givna två separata heltalen och inte med en 2-tupel.

\SubtaskSolved
\begin{CodeSmall}
enum Dir(val x: Int, val y: Int) extends Pair[Int]:
  case North extends Dir( 0, -1)
  case South extends Dir( 0,  1)
  case East  extends Dir( 1,  0)
  case West  extends Dir(-1,  0)
export Dir.*  // gör så att North etc blir synliga i paketet snake
\end{CodeSmall}

\QUESTEND






\WHAT{Supertyp med parameter.}

\QUESTBEGIN

\Task  \what~  Utbildningsdepartementet vill med sitt nya datasystem hålla koll på vissa personer och skapar därför en klasshierarki enligt nedan. Skriv in koden i en editor och testa i REPL med \code{sbt}.
\begin{Code}
class Person(val namn: String)

class Akademiker(
  namn: String,
  val universitet: String) extends Person(namn)

class Student(
  namn: String,
  universitet: String,
  program: String) extends Akademiker(namn, universitet)

class Forskare(
  namn: String,
  universitet: String,
  titel: String) extends Akademiker(namn, universitet)
\end{Code}


\Subtask Deklarera fyra olika \code{val}-variabler med lämpliga namn som refererar till olika instanser av alla olika klasser ovan och ge attributen valfria initialvärden genom olika parametrar till konstruktorerna.

\Subtask Skriv två satser: en som först stoppar in instanserna i en \code{Vector} och en som sedan loopar igenom vektorn och skriv ut alla instansers \code{toString} och \code{namn}.

\Subtask Utbildningsdepartementet vill att det inte ska gå att instansiera objekt av typerna \code{Person} och \code{Akademiker}. Det kan åstadkommas genom att placera nyckelordet \code{abstract} före \code{class}. Uppdatera koden i enlighet med detta. Vilket blir felmeddelande om man försöker instansiera en \code{abstract class}? Går det lika bra med en \code{trait}?

\Subtask Utbildningsdeparetementet vill slippa implementera \code{toString}. Gör därför om typerna \code{Student} och \code{Forskare} till case-klasser. \emph{Tips:} För att undkomma ett kompileringsfel (vilket?) behöver du använda \code{override val} på lämpligt ställe.
Skapa instanser av de nya case-klasserna \code{Student} och \code{Forskare} och skriv ut deras \code{toString}. 

\Subtask 
%Eftersom \code{Person} och \code{Akademiker} nu är abstrakta, vill utbildningsdepartementet att du gör om dessa typer till traits med abstrakta attribut istället för klasser. 
Använd abstrakta attribut i stället för parametrar för typerna som är abstrakta, så att du inte behöver skriva \code{override val} i klassparametrarna till de konkreta case-klasserna.
Du ska också införa en case-klass \code{IckeAkademiker} som ska användas i olika statistiska medborgarundersökningar.
Dessutom förser man alla personer med ett personnummer representerat som en \code{Long}.
Hur ser utbildningsdepartementets kod ut nu, efter alla ändringar? Skriv ett testprogram som skapar några instanser och skriver ut deras attribut.

\SOLUTION


\TaskSolved \what


\SubtaskSolved
\begin{Code}
val person = new Person("Person1")
val akademiker = new Akademiker("Person2", "LTH")
val student = new Student("Person3", "LTH", "D")
val forskare = new Forskare("Person4", "LTH", "Doktorand")
\end{Code}

\SubtaskSolved
\begin{Code}
val vec = Vector(person, akademiker, student, forskare)
for(i <- vec){ print(i.toString + i.namn) }
\end{Code}

\SubtaskSolved  
Felmeddelande vid instansiering av \code{abstract class Akademiker}:\\
\texttt{Akademiker is abstract; it cannot be instantiated}

Det går \emph{inte} lika bra med en \code{trait} i det speciella fallet \code{Akademiker}, eftersom en trait inte får skicka vidare parametrar till en supertyp. Felmeddelande:\\
\texttt{trait Akademiker may not call constructor of trait Person}
\begin{Code}
trait Person(val namn: String)

abstract class Akademiker(
  namn: String,
  val universitet: String) extends Person(namn)

class Student(
  namn: String,
  universitet: String,
  program: String) extends Akademiker(namn, universitet)

class Forskare(
  namn: String,
  universitet: String,
  titel: String) extends Akademiker(namn, universitet)
\end{Code}



\SubtaskSolved  
\begin{REPLnonum}
scala>  
     |trait Person(val namn: String)                                                                              
     | 
     | abstract class Akademiker(
     |   namn: String,
     |   val universitet: String) extends Person(namn)
     | 
     | case class Student(
     |   namn: String,
     |   universitet: String,
     |   program: String) extends Akademiker(namn, universitet)
     | 
     | case class Forskare(
     |   namn: String,
     |   universitet: String,
     |   titel: String) extends Akademiker(namn, universitet)
-- Error:     
8 |  namn: String,
  |  ^
  |  error overriding value namn in trait Person of type String;
  |    value namn of type String needs `override` modifier
-- Error:
9 |  universitet: String,
  |  ^
  |  error overriding value universitet in class Akademiker of type String;
  |    value universitet of type String needs `override` modifier
-- Error:
13 |  namn: String,
   |  ^
   |  error overriding value namn in trait Person of type String;
   |    value namn of type String needs `override` modifier
-- Error:
14 |  universitet: String,
   |  ^
   |  error overriding value universitet in class Akademiker of type String;
   |    value universitet of type String needs `override` modifier
\end{REPLnonum}

\begin{Code}
trait Person(val namn: String)

abstract class Akademiker(
  namn: String,
  val universitet: String) extends Person(namn)

case class Student(
  override val namn: String,
  override val universitet: String,
  program: String) extends Akademiker(namn, universitet)

case class Forskare(
  override val namn: String,
  override val universitet: String,
  titel: String) extends Akademiker(namn, universitet)
\end{Code}

\begin{REPLsmall}
scala> val ps = Vector(Student("Kim", "Lund", "D"), Forskare("Herz", "Lund", "Dr"))
val ps: Vector[Akademiker] = Vector(Student(Kim,Lund,D), Forskare(Herz,Lund,Dr))
scala> ps :+ new Person("Abstrakt") {}
val res0: Vector[Person] = 
  Vector(Student(Kim,Lund,D), Forskare(Herz,Lund,Dr), anon1@1941bbf3)
\end{REPLsmall}

\SubtaskSolved
\begin{Code}
trait Person: 
  val namn: String 
  val nbr: Long

trait Akademiker extends Person:
  val universitet: String

case class Student(
  namn: String,
  nbr: Long,
  universitet: String,
  program: String) extends Akademiker

case class Forskare(
  namn: String,
  nbr: Long,
  universitet: String,
  titel: String) extends Akademiker

case class IckeAkademiker(
    namn: String,
    nbr: Long) extends Person
\end{Code}



\QUESTEND




%\clearpage




\ExtraTasks %%%%%%%%%%%%%%%%%





%\WHAT{Uppräknade värden.}

%\QUESTBEGIN

% \Task  \what~  Ett sätt att hålla reda på uppräknade värden, t.ex. färgen på olika kort i en kortlek, är att använda olika heltal som får representera de olika värdena, till exempel så här:\footnote{Om namnkonventioner för konstanter i Scala: läs under rubriken ''Constants, Values, Variable and Methods'' här \href{http://docs.scala-lang.org/style/naming-conventions.html}{docs.scala-lang.org/style/naming-conventions.html}}
% \begin{Code}
% object Färg {
%   val Spader = 1
%   val Hjärter = 2
%   val Ruter = 3
%   val Klöver = 4
% }
% \end{Code}
% Dessa kan sedan användas som parametrar till denna case-klass vid skapande av kortobjekt:
% \begin{lstlisting}[language=,keywords={case,class}]
% case class Kort(färg: Int, valör: Int)
% \end{lstlisting}
% Man kan hålla reda på färgen med en variabel av typen \code{Int} och tilldela den en viss färg med ovan konstanter. Och när du skapar ett kort kan du använda färgnamnet och du slipper därmed att behöva komma ihåg vilket heltal som representerar färgen.
% \begin{REPL}
% scala> val f = Färg.Spader
% scala> import Färg._
% scala> Kort(Ruter, 7)
% \end{REPL}
% En annan fördelen med detta är att man lätt kan iterera över alla färger:
% \begin{REPL}
% scala> val kortlek = for (f <- 1 to 4; v <- 1 to 13) yield Kort(f, v)
% \end{REPL}
% Men den stora nackdelen med detta är att kompilatorn vid kompileringstid inte kollar om variablerna av misstag råkar ges något värde utanför det giltiga intervallet, eftersom alla heltal är möjliga. Detta får vi själv hålla koll på vid körtid, till exempel med funktionen \code{require} eller \code{if}-satser, etc.

% Istället kan man använda uppräknade värden med hjälp av case-objekt enligt nedan deluppgifter och därmed få hjälp av kompilatorn att hitta eventuella fel vid kompileringstid.  Ett case-objekt är som ett vanligt singelton-objekt men det får bl.a. automatiskt en \code{toString} som är samma som namnet. Case-objekt kan dessutom användas som värden i mönstermatchningar (mer om detta i kapitel \ref{chapter:W10}).

% \Subtask Deklarera följande uppräknade värden som singelton-objekt med gemensam bastyp. Med nyckelordet \code{sealed} så ''förseglas'' klassen och inga andra direkta subtyper tillåts förutom de som finns i samma kod-fil eller block. I detta exempel  med kortfärger vet vi ju att det inte finns fler än dessa fyra färger.
% \begin{Code}
% sealed trait Färg
% case object Spader extends Färg
% case object Hjärter extends Färg
% case object Ruter extends Färg
% case object Klöver extends Färg
% \end{Code}
% Dessa kan sedan användas som parametrar till denna case-klass vid skapande av kortobjekt:
% \begin{lstlisting}[language=,keywords={case,class}]
% case class Kort(färg: Färg, valör: Int)
% \end{lstlisting}
% Skapa därefter några exempelinstanser av klassen \code{Kort}. Vad är fördelen med ovan angreppssätt jämfört med att använda heltalskonstanter?

% \Subtask Om man vill kunna iterera över alla värden är det bra om de finns i en samling med alla värden. Vi kan lägga en sådan i ett kompanjonsobjekt till bastypen enligt nedan. Skriv ut alla färgvärden med en \code{for}-sats.

% \begin{Code}
% sealed trait Färg
% object Färg {
%   val values = Vector(Spader, Hjärter, Ruter, Klöver)
% }
% case object Spader extends Färg
% case object Hjärter extends Färg
% case object Ruter extends Färg
% case object Klöver extends Färg
% \end{Code}
% Skapa en kortlek med 52 olika kort och blanda den sedan med \code{Random.shuffle} enligt nedan. Använd en dubbel \code{for}-sats och \code{yield}.
% \begin{REPL}
% scala> val kortlek: Vector[Kort] = ???
% scala> val blandad = scala.util.Random.shuffle(kortlek)
% \end{REPL}

% \Subtask Skriv en funktion \code{ def blandadKortlek: Vector[Kort] = ???} som ger en ny blandad kortlek varje gång den anropas med metoden i föregående uppgift.

% \Subtask Om man även vill ha en heltalsrepresentation med en medlem \code{toInt} för alla värden, kan man ge bastypen en parameter och i stället för en trait (som inte kan ha några parametrar) använda en abstrakt klass.

% \begin{Code}
% sealed abstract class Färg(final val toInt: Int)
% object Färg {
%   val values = Vector(Spader, Hjärter, Ruter, Klöver)
% }
% case object Spader  extends Färg(0)
% case object Hjärter extends Färg(1)
% case object Ruter   extends Färg(2)
% case object Klöver  extends Färg(3)
% \end{Code}
% Skapa en funktion \code{färgPoäng} som räknar ut summan av heltalsrepresentationen av alla färger hos en vektor med kort, och använd den sedan för att räkna ut \code{färgPoäng} för de första fem korten.
% \begin{REPL}
% scala> def blandadKortlek: Vector[Kort] = ???
% scala> def färgPoäng(xs: Vector[Kort]): Int = ???
% scala> färgPoäng(blandadKortlek.take(5))
% \end{REPL}


% \SOLUTION

% \TaskSolved \what

% \SubtaskSolved  Sättet är säkrare då man inte kan tilldela korten en färg som inte finns. Med heltalskonstanterna kan man till exempel ge ett kort färgen 5, vilken inte korresponderar till någon riktig färg.

% \SubtaskSolved  \code{for (f <- Färg.values; v <- 1 to 13) yield Kort(f,v)}

% \SubtaskSolved
% \begin{Code}
% def blandadKortlek: Vector[Kort] = {
%   val kortlek =
%     for (f <- Färg.values; v <- 1 to 13) yield Kort(f,v)
%   scala.util.Random.shuffle(kortlek)
% }
% \end{Code}

% \SubtaskSolved  \code{def färgPoäng(xs: Vector[Kort]): Int = xs.map(_.färg.toInt).sum}

% \QUESTEND







\WHAT{Bastypen \code{Shape} och subtyperna \code{Rectangle} och \code{Circle}.}

\QUESTBEGIN

\Task  \what~  Du ska i denna uppgift skapa ett litet bibliotek för geometriska former med oföränderliga objekt implementerade med hjälp av case-klasser. De geometriska formerna har en gemensam bastyp kallad \code{Shape}. Utgå från koden nedan.

\begin{CodeSmall}
case class Point(x: Double, y: Double):
  def move(dx: Double, dy: Double): Point = Point(x + dx, y + dy)

trait Shape:
  def pos: Point
  def move(dx: Double, dy: Double): Shape

case class Rectangle(pos: Point, width: Double, height: Double) extends Shape:
  def move(dx: Double, dy: Double): Rectangle = copy(pos = pos.move(dx, dy))

case class Circle(pos: Point, radius: Double) extends Shape:
  def move(dx: Double, dy: Double): Circle = copy(pos = pos.move(dx, dy))

\end{CodeSmall}

\Subtask Instansiera några cirklar och rektanglar och gör några relativa förflyttningar av dina instanser genom att anropa \code{move}.

\Subtask Lägg till en konkret metod \code{moveTo} i \code{Point} som gör en absolut förflyttning till koordinaterna \code{x} och \code{y}. Lägg till en abstrakt metod \code{moveTo} \code{Shape} som implementeras i subklasserna. Testa med REPL på några instanser av \code{Rectangle} och \code{Circle}.

\Subtask Lägg till metoden \code{distanceTo(that: Point): Double } i case-klassen \code{Point} som räknar ut avståndet till en annan punkt med hjälp av \code{math.hypot}. Klistra in i REPL och testa på några instanser av \code{Point}.

\Subtask Lägg till en konkret metod \code{distanceTo(that: Shape): Double } i traiten \code{Shape} som räknar ut avståndet till positionen för en annan Shape. Testa i REPL på några instanser av \code{Rectangle} och \code{Circle}.

\Subtask Gör så att \code{distanceTo} kan anropas med operatornotation.

\SOLUTION


\TaskSolved \what


\SubtaskSolved
\begin{CodeSmall}
val c1 = Circle(Point(1, 1), 42)
val r1 = Rectangle(Point(3, 3), 20, 30)
c1.move(2, 3)
r1.move(3, 2)
\end{CodeSmall}

\SubtaskSolved  
\begin{CodeSmall}
case class Point(x: Double, y: Double):
  def move(dx: Double, dy: Double): Point = Point(x + dx, y + dy)
  def moveTo(x: Double, y: Double): Point = Point(x, y)

trait Shape:
  def pos: Point
  def move(dx: Double, dy: Double): Shape
  def moveTo(x: Double, y: Double): Shape

case class Rectangle(pos: Point, width: Double, height: Double) extends Shape:
  def move(dx: Double, dy: Double): Shape = copy(pos = pos.move(dx, dy))
  def moveTo(x: Double, y: Double): Shape = copy(pos.moveTo(x, y))

case class Circle(pos: Point, radius: Double) extends Shape:
  def move(dx: Double, dy: Double): Shape = copy(pos = pos.move(dx, dy))
  def moveTo(x: Double, y: Double): Shape = copy(pos.moveTo(x, y))
\end{CodeSmall}


\SubtaskSolved \code{def distanceTo(that: Point): Double = math.hypot(that.x - x, that.y - y)}

\SubtaskSolved \code{def distanceTo(that: Shape): Double = pos.distanceTo(that.pos)}.

\SubtaskSolved  
\begin{CodeSmall}
case class Point(x: Double, y: Double):
  def move(dx: Double, dy: Double): Point = Point(x + dx, y + dy)
  def moveTo(x: Double, y: Double): Point = Point(x, y)
  infix def distanceTo(that: Point): Double = math.hypot(that.x - x, that.y - y)

trait Shape:
  def pos: Point
  def move(dx: Double, dy: Double): Shape
  def moveTo(x: Double, y: Double): Shape
  infix def distanceTo(that: Shape): Double = pos.distanceTo(that.pos)

case class Rectangle(pos: Point, width: Double, height: Double) extends Shape:
  def move(dx: Double, dy: Double): Shape = copy(pos = pos.move(dx, dy))
  def moveTo(x: Double, y: Double): Shape = copy(pos.moveTo(x, y))

case class Circle(pos: Point, radius: Double) extends Shape:
  def move(dx: Double, dy: Double): Shape = copy(pos = pos.move(dx, dy))
  def moveTo(x: Double, y: Double): Shape = copy(pos.moveTo(x, y))
\end{CodeSmall}

\QUESTEND






% \WHAT{Regler för \code{override}, \code{private} och \code{final}.}

% \QUESTBEGIN

% \Task  \what~

% \Subtask \label{subtask:overriderules} Undersök överskuggningning av abstrakta, konkreta, privata och finala medlemmar genom att skriva in raderna nedan en i taget i REPL. Vilka rader ger felmeddelande? Varför? Vid felmeddelande: notera hur felmeddelandet lyder och ändra deklarationen av den felande medlemmen så att koden blir kompilerbar (eller om det är enda rimliga lösningen: ta bort den felaktiga medlemmen), innan du provar efterkommande rad.

% \begin{REPL}
% trait Super1 { def a: Int; def b = 42; private def c = "hemlis" }
% class Sub2 extends Super1 { def a = 43; def b = 43; def c = 43 }
% class Sub3 extends Super1 { def a = 43; override def b = 43 }
% class Sub4 extends Super1 { def a = 43; override def c = "43" }
% trait Super5 { final def a: Int; final def b = 42 }
% class Sub6 extends Super5 { override def a = 43; def b = 43 }
% class Sub7 extends Super5 { def a = 43; override def b = 43 }
% class Sub8 extends Super5 { def a = 43; override def c = "43" }
% trait Super9 { val a: Int; val b = 42; lazy val c: String = "lazy" }
% class Sub10 extends Super9 { override def a = 43; override val b = 43 }
% class Sub11 extends Super9 { val a = 43; override lazy val b = 43 }
% class Sub12 extends Super9 { val a = 43; override var b = 43 }
% class Sub13 extends Super9 { val a = 43; override lazy val c = "still lazy" }
% class SubSub extends Sub13 { override val a = 44}
% trait Super14 { var a: Int; var b = 42; var c: String }
% class Sub15 extends Super14 { def a = 43; override var b = 43; val c = "?" }
% \end{REPL}

% \Subtask Skapa instanser av klasserna \code{Sub3}, \code{Sub13} och \code{SubSub} från ovan deluppgift och undersök alla medlemmarnas värden för respektive instans. Förklara varför de har dessa värden.

% %\Subtask Läs igenom reglerna i kapitel  \ref{slideW07:overriderules} om vad som gäller vid arv och överskuggning av medlemmar. Gör några egna undersökningar i REPL som försöker bryta mot någon regel som inte testades i deluppgift \ref{subtask:overriderules}.

% \SOLUTION


% \TaskSolved \what


% \SubtaskSolved  2. Måste ha \code{override} framför \code{b} för att kunna ändra på metoden. \\
% 4. \code{c} är \code{private}, vilket betyder att den är gömd för subklasserna. Därför kan den inte överskuggas. Genom att ta bort \code{override} fungerar klassen. \\
% 5. En \code{final}-medlem måste ha ett bestämt värde. Kan lösas genom att tilldela \code{final a} ett värde eller ta bort \code{final}. \\
% 6. En \code{final}-medlem kan inte överskuggas, varken med eller utan \code{override}. Här får konflikterna tas bort.  \\
% 7. Se 6. \\
% 8. Eftersom \code{c} inte finns i \code{Super5} kan den inte överskuggas. Genom att ta bort \code{override} fungerar klassen. \\
% 10. Överskuggningen av \code{val} måste vara oföränderlig (immutable); detta är inte nödvändigtvis \code{def}. Löses genom att byta ut \code{def a} mot \code{val a} hos \code{Sub10}.  \\
% 11. Samma problem som i 10.; \code{lazy val} kan vara föränderlig. Löses genom att ta bort \code{lazy}. \\
% 12. Samma problem igen! \code{var} är föränderlig, vilket bryter mot typsäkerheten när man försöker överskugga en \code{val}. Löses genom att ändra \code{var} till \code{val}. \\
% 15.\code{def a = 43} och \code{val c = "?"} täcker inte allt som \code{var} kräver. Det behövs en setter för att kunna uppfylla kraven för överskuggning för en \code{var}. Dessutom finns det ingen anledning för en \code{val} att överskuggas; man kan ju ändra på den lite hur man vill!

% \SubtaskSolved  Sub3: a = 43, b = 43 eftersom medlemmen är överskuggad. c hittas inte eftersom den är \code{private}.

% Sub13: a = 43, b = 42, c = "still lazy" eftersom medlemmen överskuggas.

% SubSub: a = 44 eftersom medlemmen överskuggas, b = 42, c = "still lazy".

% \SubtaskSolved  -.


% \QUESTEND





%\clearpage





\AdvancedTasks %%%%%%%%%%%%%%%%%

% \WHAT{Använda \code{trait} eller \code{class}?}

% \QUESTBEGIN

% \Task \what~ I vilka sammanhang är det nödvändigt att använda en \code{trait} respektive en \code{class}? Läs här för fördjupning:\\  \href{http://www.artima.com/pins1ed/traits.html\#12.7}{http://www.artima.com/pins1ed/traits.html\#12.7}.


% \SOLUTION


% \TaskSolved \what~Man måste använda en klass om man behöver klassparametrar. Man måste använda en trait om man vill göra in-mixning med \code{with}. \\

%  \QUESTEND



\WHAT{Inmixning.}

\QUESTBEGIN

\Task \label{task:fyle} \what~   Man kan utvidga en klass med multipla traits med en kommaseparerad lista. På så sätt kan man fördela medlemmar i olika traits och återanvända gemensamma koddelar genom så kallad \textbf{inmixning}, så som nedan exempel visar.

En alternativ fågeltaxonomi, speciellt populär i Skåne, delar in alla fåglar i två specifika kategorier: Kråga respektive Ånka. Krågor kan flyga men inte simma, medan Ånkor kan simma och oftast även flyga. Fågel i generell, kollektiv bemärkelse kallas på gammal skånska för Fyle.%
\footnote{\href{http://www.klangfix.se/ordlista.htm}{www.klangfix.se/ordlista.htm}}

\begin{CodeSmall}
trait Fyle:
  val läte: String
  def väsnas: Unit = print(läte * 2)
  val ärSimkunnig: Boolean
  val ärFlygkunnig: Boolean

trait KanSimma       { val ärSimkunnig = true }
trait KanInteSimma   { val ärSimkunnig = false }
trait KanFlyga       { val ärFlygkunnig = true }
trait KanKanskeFlyga { val ärFlygkunnig = math.random() < 0.8 }

class Kråga extends Fyle, KanFlyga, KanInteSimma:
  val läte = "krax"

class Ånka extends Fyle, KanSimma, KanKanskeFlyga: 
  val läte = "kvack"
  override def väsnas = print(läte * 4)
\end{CodeSmall}

\Subtask En flitig ornitolog hittar 42 fåglar i en perfekt skog där alla fågelsorter är lika sannolika, representerat av vektorn \code{fyle} nedan. Skriv i REPL ett uttryck som undersöker hur många av dessa som är flygkunniga Ånkor, genom att använda metoderna \code{filter}, \code{isInstanceOf}, \code{ärFlygkunnig} och \code{size}.

\begin{REPL}
scala> val fyle =
         Vector.fill(42)(if math.random() > 0.5 then new Kråga else new Ånka)
scala> fyle.foreach(_.väsnas)
scala> val antalFlygånkor: Int = ???
\end{REPL}

\Subtask \label{subtask:fyle:sound} Om alla de fåglar som ornitologen hittade skulle väsnas exakt en gång var, hur många krax och hur många kvack skulle då höras? Använd metoderna \code{filter} och \code{size}, samt predikatet \code{ärSimkunnig} för att beräkna antalet krax respektive kvack.
\begin{REPL}
scala> val antalKrax: Int = ???
scala> val antalKvack: Int = ???
\end{REPL}

\SOLUTION


\TaskSolved \what


\SubtaskSolved
Det finns många olika sätt, några exempellösningar:
\begin{Code}
val antalFlygånkor: Int = 
  fyle.count(f => f.isInstanceOf[Ånka] && f.ärFlygkunnig)
\end{Code}

\begin{Code}
val antalFlygånkor: Int = 
  fyle.filter(f => f.isInstanceOf[Ånka] && f.ärFlygkunnig).size
\end{Code}

\begin{Code}
val antalFlygånkor: Int = 
  fyle.collect{case f: Ånka if f.ärFlygkunnig}.size
\end{Code}

\begin{Code}
val antalFlygånkor: Int = fyle.map(_ match
  case f: Ånka if f.ärFlygkunnig => 1
  case _ => 0
).sum
\end{Code}

\SubtaskSolved
\begin{Code}
val antalKrax: Int = fyle.filter(f => !f.ärSimkunnig).size * 2
val antalKvack: Int = fyle.filter(f => f.ärSimkunnig).size * 4
\end{Code}


\QUESTEND











\WHAT{Finala klasser.}

\QUESTBEGIN

\Task  \what~  Om man vill förhindra att man kan göra \code{extends} på en klass kan man göra den final genom att placera nyckelordet \code{final} före nyckelordet \code{class}.

\Subtask Eftersom klassificeringen av fåglar i uppgiften ovan i antingen Ånkor eller Krågor är fullständig och det inte finns några subtyper till varken Ånkor eller Krågor är det lämpligt att göra dessa finala. Ändra detta i din kod.

\Subtask Testa att ändå försöka göra en subklass \code{Simkråga extends Kråga}. Vad ger kompilatorn för felmeddelande om man försöker utvidga en final klass?


\SOLUTION


\TaskSolved \what


\SubtaskSolved  Sätt \code{final} framför \code{class} i klasserna.

\SubtaskSolved  error: illegal inheritance from final class Kråga.


\QUESTEND






\WHAT{Accessregler vid arv och nyckelordet \code{protected}.}

\QUESTBEGIN

\Task  \what~  Om en medlem i en supertyp är privat så kan man inte komma åt den i en subklass. Ibland vill man att subklassen ska kunna komma åt en medlem även om den ska vara otillgänglig i annan kod.

\begin{Code}
trait Super:
  private val minHemlis = 42
  protected val vårHemlis = 42

class Sub extends Super:
  def avslöja = minHemlis
  def kryptisk = vårHemlis * math.Pi

\end{Code}

\Subtask Vad blir felmeddelandet när klassen \code{Sub} försöker komma åt \code{minHemlis}?

\Subtask Deklarera \code{Sub} på nytt, men nu utan den förbjudna metoden \code{avslöja}. Vad blir felmeddelandet om du försöker komma åt \code{vårHemlis} via en instans av klassen \code{Sub}? Prova till exempel med \code{(new Sub).vårHemlis}

\Subtask Fungerar det att anropa metoden \code{kryptisk} på instanser av klassen \code{Sub}?

\SOLUTION


\TaskSolved \what


\SubtaskSolved  
\begin{REPL}
2 |  def avslöja = minHemlis
  |                ^^^^^^^^^
  |                Not found: minHemlis
\end{REPL}

\SubtaskSolved  
\begin{REPL}
scala> class Sub extends Super:
         def kryptisk = vårHemlis * math.Pi
scala> (new Sub).vårHemlis
-- Error:
1 |(new Sub).vårHemlis
  |^^^^^^^^^^^^^^^^^^^
  |value vårHemlis in trait Super cannot be accessed as a member of Sub.
  | Access to protected value vårHemlis not permitted because enclosing object 
  | is not a subclass of trait Super where target is defined
\end{REPL}

\SubtaskSolved  Ja.


\QUESTEND






\WHAT{Använding av \code{protected}.}

\QUESTBEGIN

\Task  \what~  Den flitige ornitologen från uppgift \ref{task:fyle} ska ringmärka alla 42 fåglar hen hittat i skogen. När hen ändå håller på bestämmer hen att även försöka ta reda på hur mycket oväsen som skapas av respektive fågelsort. För detta ändamål apterar den flitige ornitologen en Linuxdator på allt infångat fyle. Du ska hjälpa ornitologen att skriva programmet.

\Subtask Inför en \code{protected var räknaLäte} i traiten \code{Fyle} och skriv kod på lämpliga ställen för att räkna hur många läten som respektive fågelinstans yttrar.

\Subtask Inför en metod \code{antalLäten} som returnerar antalet krax respektive kvack som en viss fågel yttrat sedan dess skapelse.

\Subtask Varför inte använda \code{private} i stället for \code{protected}?

\Subtask Varför är det bra att göra räknar-variabeln oåtkomlig från ''utsidan''?



\SOLUTION


\TaskSolved \what


\SubtaskSolved  I Fyle:
\begin{Code}
protected var räknaLäte: Int = 0
def väsnas: Unit = { print(läte * 2); räknaLäte += 2 }
\end{Code}

I Ånka: \code| override def väsnas = { print(läte * 4); räknaLäte += 4 }|

\SubtaskSolved  \code{ def antalLäten: Int = räknaLäte }

\SubtaskSolved  Om en klass som representerar en fågel som skulle ge ifrån sig fler/färre läten än en vanlig \code{Fyle}, behöver \code{väsnas} ändras. Denna metod behöver tillgång till \code{räknaLäte}, vilken inte får vara \code{private}.

\SubtaskSolved  Räknar-variabeln ska inte kunna påverkas i någon annan del av programmet.


\QUESTEND





\WHAT{Inmixning av egenskaper.}

\QUESTBEGIN

\Task  \what~ Det visar sig att vår flitige ornitolog från uppgift \ref{task:fyle} på sidan \pageref{task:fyle} sov på en av föreläsningarna i zoologi när hen var nolla på Natfak, och därför helt missat fylekategorin \code{Pjodd}. Hjälp vår stackars ornitolog så att fylehierarkin nu även omfattar Pjoddar. En Pjodd kan flyga som en Kråga men den \code{ÄrLiten} medan en Kråga \code{ÄrStor}. En Pjodd kvittrar dubbelt så många gånger som en Ånka kvackar. En Pjodd \code{KanKanskeSimma} där simkunnighetssannolikheten är $0.2$. Låt ornitologen ånyo finna 42 slumpmässiga fåglar i skogen och filtrera fram lämpliga arter. Undersök sedan hur dessa väsnas, i likhet med deluppgift \ref{task:fyle}\ref{subtask:fyle:sound}.


\SOLUTION

\TaskSolved \what


\begin{Code}
trait Fyle:
  val läte: String
  def väsnas: Unit = { print(läte * 2); räknaLäte += 2 }
  protected var räknaLäte: Int = 0
  val ärSimkunnig: Boolean
  val ärFlygkunnig: Boolean
  val ärStor : Boolean
  def antalLäten: Int = räknaLäte

trait KanSimma { val ärSimkunnig = true }
trait KanInteSimma { val ärSimkunnig = false }
trait KanFlyga { val ärFlygkunnig = true }
trait KanKanskeFlyga { val ärFlygkunnig = math.random() < 0.8 }
trait KanKanskeSimma { val ärSimkunnig = math.random() < 0.2 }
trait ÄrStor { val ärStor = true }
trait ÄrLiten { val ärStor = false }

final class Kråga extends Fyle, KanFlyga, KanInteSimma, ÄrStor:
  val läte = "krax"

final class Ånka extends Fyle, KanSimma, KanKanskeFlyga, ÄrStor:
  val läte = "kvack"
  override def väsnas = { print(läte * 4); räknaLäte += 4 }

final class Pjodd extends Fyle, KanFlyga, KanKanskeSimma, ÄrLiten:
  val läte = "kvitter"
  override def väsnas = { print(läte * 8); räknaLäte += 8 }
\end{Code}

I REPL:
\begin{REPL}
val fyle = Vector.fill(42)(
  if math.random() < 0.33 then Kråga()
  else if math.random() < 0.5 then Ånka()
  else Pjodd()
)
fyle.filter(f => f.isInstanceOf[Kråga]).size * 2
fyle.filter(f => f.isInstanceOf[Ånka]).size * 4
fyle.filter(f => f.isInstanceOf[Pjodd]).size * 8
\end{REPL}

\QUESTEND





% \WHAT{Typtest och typkonvertering.}

% \QUESTBEGIN

% \Task  \what~I Scala kan man testa körtidstyp och samtidigt konvertera till en mer specifik typ på ett typsäkert sätt med hjälp av \emph{mönstermatchning} i \code{match}-uttryck som vi ska se i kommande övning \texttt{\ExeWeekTEN}. För att underlätta interoperabilitet med Java finns  Scala-metoderna \code{isInstanceOf} och \code{asInstanceOf}, som motsvarar hur typtest och typkonvertering görs i Java.\footnote{\code{isInstanceOf} och \code{asInstanceOf} används sällan i Scala eftersom \code{match} är kraftfullare och säkrare.}

% Gör nedan deklarationer.
% \begin{REPL}
% scala> trait A; trait B extends A; class C extends B; class D extends B
% scala> val (c, d) = (new C, new D)
% scala> val a: A = c
% scala> val b: B = d
% \end{REPL}

% \Subtask Rita en bild över vilka typer som ärver vilka.

% \Subtask Vilket resultat ger dessa typtester? Varför?
% \begin{REPL}
% scala> c.isInstanceOf[C]
% scala> c.isInstanceOf[D]
% scala> d.isInstanceOf[B]
% scala> c.isInstanceOf[A]
% scala> b.isInstanceOf[A]
% scala> b.isInstanceOf[D]
% scala> a.isInstanceOf[B]
% scala> c.isInstanceOf[AnyRef]
% scala> c.isInstanceOf[Any]
% scala> c.isInstanceOf[AnyVal]
% scala> c.isInstanceOf[Object]
% scala> 42.isInstanceOf[Object]
% scala> 42.isInstanceOf[Any]
% \end{REPL}

% \Subtask Vilka av dessa typkonverteringar ger felmeddelande? Vilket och varför?
% \begin{REPL}
% scala> c.asInstanceOf[B]
% scala> c.asInstanceOf[A]
% scala> d.asInstanceOf[C]
% scala> a.asInstanceOf[B]
% scala> a.asInstanceOf[C]
% scala> a.asInstanceOf[D]
% scala> a.asInstanceOf[E]
% scala> b.asInstanceOf[A]
% \end{REPL}



% \SOLUTION


% \TaskSolved \what


% \SubtaskSolved  B ärver A. C och D ärver B.

% \SubtaskSolved  1. True eftersom c är av typen C. \\
% 2. False eftersom c inte är av typen D. \\
% 3. True eftersom d är av typen D som är en subtyp av B. \\
% 4. True eftersom c är av typen C som är en subtyp av B, som i sin tur är en subtyp av A. \\
% 5. True eftersom b är av typen D, som är en subtyp av B, som i sin tur är en subtyp av A. \\
% 6. True eftersom b är av typen D. \\
% 7. True eftersom a är av typen C som är en subtyp av B. \\
% 8. True eftersom c är av typen C som är en subtyp av AnyRef. \\
% 9. True eftersom c är av typen C som är en subtyp av Any. \\
% 10. Error eftersom \code{isInstanceOf} inte kan använda sig av \code{AnyVal}.  \\
% 11. True eftersom c är av typen C som är en subtyp av Object (Object är java-representationen av AnyRef). \\
% 12. Error eftersom \code{isInstanceOf} inte kan testa om värdetyper (i detta fallet \code{42}) är referenstyper. \\
% 13. True eftersom \code{42} är av typen \code{Int} som är en subtyp av Any. \\

% \SubtaskSolved  3. Går inte eftersom c inte är av typen D, utan typen C. \\
% 6. Går inte eftersom a inte är av typen D, utan typen C. \\
% 7. Går inte eftersom typen E inte finns. \\


% \QUESTEND













% \WHAT{Saknad referens med \texttt{null} och bottentypen \texttt{Nothing}.}

% \QUESTBEGIN

% \Task  \what~ Hitta på en egen fördjupningsuppgift inspirerat av denna artikel på Stackoverflow: \url{http://stackoverflow.com/questions/16173477/usages-of-null-nothing-unit-in-scala}

% \SOLUTION


% \QUESTEND






\WHAT{Arvshierarki med matematiska tal.}

\QUESTBEGIN

\Task  \what~ Studera den djupa arvshierarkin i paketet \code{numbers} i koden på efterföljande sidor. Paketet  \code{numbers} modellerar olika sorters tal i matematiken, med syftet att erbjuda ett s.k. DSL \footnote{\url{https://en.wikipedia.org/wiki/Domain-specific_language}}, alltså ett specialspråk för en viss applikationsdomän\footnote{\url{https://stackoverflow.com/questions/49216312/what-is-dsl-in-scala}}, här: domänen matematiska tal.

Du kan ladda ner koden för \code{numbers} här: \\
\href{https://github.com/lunduniversity/introprog/blob/master/compendium/examples/numbers.scala}{github.com/lunduniversity/introprog/blob/master/compendium/examples/numbers.scala}
\\ Notera speciellt metoden \code{reduce} som reducerar ett tal till sin enklaste form. Metoden \code{reduce} överskuggas på lämpliga ställen med relevant reduktion.

\Subtask Rita en bild över typhierarkin, t.ex. som ett upp-och-nedvänt träd med bastypen  \code{Number} som rot.

\Subtask Skriv kod som använder de olika konkreta klasserna i \code{package numbers}. 
\begin{REPL}
scala> numbers.  // Tryck Tab
AbstractComplex   AbstractNatural    AbstractReal   Frac    Nat      Polar
AbstractInteger   AbstractRational   Complex        Integ   Number   Real

scala> numbers.Integ(12)
res0: numbers.Integ = Integ(12)

scala> import numbers.Syntax._
import numbers.Syntax._

scala> 42.j
res1: numbers.Complex = Complex(Real(0),Real(42))

scala> 42.42.j
res2: numbers.Complex = Complex(Real(0),Real(42.42))

\end{REPL}

\Subtask Ändra på metoden \code{+} i \code{trait Number} så att den blir abstrakt och implementera den i alla konkreta klasser.

\Subtask Implementera fler räknesätt och bygg vidare på koden så som du finner intressant.

\Subtask Gör så att metoden \code{reduce} i klassen \code{AbstractRational} använder algoritmen Greatest Common Divisor (GCD)\footnote{\url{https://sv.wikipedia.org/wiki/St\%C3\%B6rsta\_gemensamma\_delare}} så som beskrivs här: \\ \href{http://www.artima.com/pins1ed/functional-objects.html#6.8}{www.artima.com/pins1ed/functional-objects.html\#6.8} \\ så att täljare och nämnare blir så små som möjligt.

%\clearpage

\scalainputlisting[numbers=left, basicstyle=\ttfamily\fontsize{9.1}{12.2}\selectfont]{examples/numbers.scala}\SOLUTION


\QUESTEND

%!TEX encoding = UTF-8 Unicode
%!TEX root = ../compendium2.tex

\Teamlab{\LabWeekTEN}

\begin{Goals}
\input{modules/w10-inheritance-lab-goals.tex}
\end{Goals}

\begin{Preparations}
\item Gör övning {\tt \ExeWeekNINE} i kapitel \ref{exe:W09}, speciellt uppgift \ref{exe:inheritance:labprep-pair}.
\item Läs dokumentationen för \code{introprog.BlockGame}.
\item Hämta given kod via \href{https://github.com/lunduniversity/introprog/tree/master/workspace/}{kursen github-plats}.
\item Läs igenom hela laborationen och förbered dig inför första gruppmötet.
\input{team-lab-prep-items.tex}
\item Träffas i din samarbetsgrupp och diskutera ert arbetssätt utifrån följande frågor:
\begin{itemize}[nolistsep]
  \item Vilka krav ska ni implementera?
  \item Hur ska ni jobba med gemensamma koddelar?
  \item Hur ska ni dela med er av de koddelar som ni utvecklar var för sig? Det är lärorikt att prova att göra manuell ändringssammanfogning till att börja med.
  \begin{itemize}[nolistsep]
  \item Om ni inte använder git: Fundera ut när och hur ni ska sammanfoga era ändringar. Använd t.ex. USB-minne eller någon gemensam lagringsplats på nätet som bara din grupp har åtkomst till. Berätta när det är dags att synka via chatt eller mejl.  
  \item Om ni använder git: Hur ska ni använda grenar? Hur ska er \code{.gitignore} se ut?
  \end{itemize}
\end{itemize}

\end{Preparations}

\subsection{Bakgrund}

Spelet \emph{Snake}\footnote{Även kallat ''masken''. \url{https://sv.wikipedia.org/wiki/Snake}} blev mäkta populärt i Sverige redan på 1980-talet, ofta spelat på den legendariska datorn ABC80. Spelet finns i flera varianter, både för en spelare och som duell mellan två spelare. Varje spelare styr en mask med huvud och svans som hela tiden rör sig framåt. Det gäller att undvika att köra in i en masksvans och att samla poäng t.ex. genom att äta äpple.

Figur~\ref{fig:snake-game} visar en startdialog där man kan välja antal spelare, samt ett exempel på spel med en och två spelare. I varianten för en spelare närmar sig maskens huvud äpplet och lyckas kanske äta det om spelaren styr rätt.  I varianten för två spelare vinner grön mask eftersom den blåa masken råkade köra in i den gröna maskens svans.
\begin{figure}[H]
\begin{minipage}{0.45\textwidth}
\includegraphics[width=1.0\textwidth]{../img/snake-start}

\includegraphics[width=1.0\textwidth]{../img/snake-oneplayer}
\end{minipage}
\begin{minipage}{0.5\textwidth}
\includegraphics[width=1.0\textwidth]{../img/snake-twoplayer}
\end{minipage}
\caption{Spelet snake för en spelare med äpple och för två spelare utan äpple. \label{fig:snake-game}}
\end{figure}

\subsection{Obligatoriska funktionella krav}

Följande funktionella krav ska uppfyllas av ert program om ni är sex personer i gruppen. Om ni är färre ingår de obligatoriska krav som visas i tabell \ref{lab:snak:table-reqt}.
%\footnote{Om någon student, p.g.a. långvarig sjukdom eller annat giltigt skäl, genomför laborationen själv i efterhand som en individuell laboration ska följande krav implementeras på egen hand: \code{Player}, \code{OnePlayerGame}, \code{Snake}, \code{Apple}.}
\begin{itemize}[nosep, label={$\square$},]
\item \textbf{\texttt{Player}}. Det ska finnas spelare som motsvarar mänskliga användare och som har ett namn och fyra tangenter som den kan spela med. Varje spelare har en egen orm som den kan styra med sina tangenter.

\item \textbf{\texttt{Snake}}. Det ska finnas ormar. En orm består av ett antal block, där det främsta blocket kallas huvud och resten av blocken kallas svans. Huvudet har en ljusare färg än kroppen. Svansens längd ökar under spelets gång. En orm rör sig i en viss riktning och varje spelare kan ändra riktningen på sin orm med sina tangenter, i en av fyra riktningar \code{North}, \code{South}, \code{East} eller \code{West}.

\item \textbf{\texttt{Apple}}. Det ska finnas (minst ett) äpple. Ett äpple består av ett rött block och finns på en slumpvis position. Ett äpple kan ätas av en orm om ormens huvud träffar äpplet. Varje gång ett äpple äts upp av en orm så teleporteras äpplet till en ny position och kan ätas igen.

\item \textbf{\texttt{Banana}}. Det ska finnas (minst en) banan. En banan består av tre vertikala gula block och finns på en slumpvis position. En banan äts upp av en orm om ormens huvud träffar bananen. Varje gång en banan äts upp av en orm så teleporteras bananen till en ny slumpvis position och kan ätas igen.

\item \textbf{\texttt{Monster}}. Det ska finnas (minst ett) monster. Ett monster består av fem rosa block i kryssform.  Ett monster föds på en slumpvis position och rör sig i en riktning som bestäms vid monstrets födelse. Ett orm blir uppäten och dör om ormens huvud nuddar ett monsterblock .

\item \textbf{\texttt{OneplayerGame}}. Det ska gå att spela ensam. I varianten med en spelare finns en orm och minst ett äpple (och ev. även bananer och monster). Varje gång användarens orm lyckas äta en frukt får användaren poäng. När ormen ätit ett visst antal äpplen, eller om ormen blivit uppäten av ett monster, är spelet slut och poängen visas. En ormsvans ska bli längre vid jämna tidsintervall eller om den äter frukt.

\item \textbf{\texttt{TwoplayerGame}}. Det ska gå att spela två och två. I varianten med två spelare finns två ormar. Det finns också äpplen, bananer och monster. Om en orm äter en banan blir dess svans längre. När ormen ätit ett visst antal äpplen, eller om ormen blivit uppäten av ett monster, är spelet slut och poängen visas. En ormsvans ska bli längre vid jämna tidsintervall eller om den äter frukt.

\item \textbf{\texttt{Settings}}. Inställningar för spelet ska vara konfigurerbara genom en textfil som laddas i början av spelet. Inställningar ska vara en kontextparameter.  

\end{itemize}
\begin{table}[H]
  \centering
  \caption{Krav som minst ska implementeras vid respektive gruppstorlek. Om du har särskilda skäl kan du efter godkännande från kursansvarig göra labben enskilt.  \label{lab:snak:table-reqt}}

\begin{tabular}{r | c c c c c c}
  Krav / Antal personer & 1       & 2       & 3       & 4       & 5       & 6 \\ \hline
  \texttt{Player}       & $\surd$ & $\surd$ & $\surd$ & $\surd$ & $\surd$ & $\surd$ \\
  \texttt{OnePlayerGame}& $\surd$ &         &         &         & $\surd$ & $\surd$ \\
  \texttt{TwoPlayerGame}&         & $\surd$ & $\surd$ & $\surd$ & $\surd$ & $\surd$ \\
  \texttt{Snake}        & $\surd$ & $\surd$ & $\surd$ & $\surd$ & $\surd$ & $\surd$ \\
  \texttt{Apple}        & $\surd$ &         & $\surd$ & $\surd$ & $\surd$ & $\surd$ \\
  \texttt{Banana}       &         &         &         & $\surd$ &         & $\surd$ \\
  \texttt{Monster}      &         &         & $\surd$ &         &         & $\surd$ \\
  \texttt{Settings}     & $\surd$ & $\surd$ & $\surd$ & $\surd$ & $\surd$ & $\surd$ \\
\end{tabular}
\end{table}

\subsection{Obligatoriska design-krav}

\begin{enumerate}[label={$\square$}, leftmargin=*]

\item Snake-spel ska gå att starta med huvudprogrammet nedan. Huvudprogrammet får ändras vid behov i enlighet med minimikrav vad gäller gruppstorlek i tabell \ref{lab:snak:table-reqt}, samt valbara extrakrav i avsnitt \ref{lab:snake:extra-reqts}, och era egna ideer.
\scalainputlisting{../workspace/w10_snake/Main.scala}

\item Spelet ska bygga vidare på \code{introprog.BlockGame} enligt typhierarkin i fig.~\ref{snake:fig:game-hierarchy}.

\begin{figure}[H]
\begin{center}
\newcommand{\TextBox}[1]{\raisebox{0pt}[1em][0.5em]{#1}}
\tikzstyle{umlclass}=[rectangle, draw=black,  thick, anchor=north, text width=3cm, rectangle split, rectangle split parts = 3]
\begin{tikzpicture}[inner sep=0.5em,scale=1.0, every node/.style={transform shape}]

  \node [umlclass, rectangle split parts = 1, xshift=0cm, yshift=4.5cm] (BaseType)  {
              \textit{\textbf{\centerline{\TextBox{\code{BlockGame}}}}}
%              \nodepart[align=left]{second}\code{def x: T} \newline \code{def y: T}
          };


  \node [umlclass, rectangle split parts = 1, xshift=0cm, yshift=3.0cm] (SubType)  {
              \textit{\textbf{\centerline{\TextBox{\code{SnakeGame}}}}}
%              \nodepart[align=left]{second}\code{val x: Int} \newline \code{val y: Int}
          };

\node [umlclass, rectangle split parts = 1, xshift=-3cm, yshift=1.0cm] (SubSubType1)  {
            {\textbf{\centerline{\TextBox{\code{OnePlayerGame}}}}}
%            \nodepart[]{second}\TextBox{\code{val dim: Int}}
        };

\node [umlclass, rectangle split parts = 1, xshift=3cm, yshift=1.0cm] (SubSubType2)  {
            {\textbf{\centerline{\TextBox{\code{TwoPlayerGame}}}}}
%            \nodepart[]{second}\TextBox{\code{val dim: Int}}
        };

\draw[umlarrow] (SubType.north) -- ++(0,0.5) -| (BaseType.south);
\draw[umlarrow] (SubSubType1.north) -- ++(0,0.5) -| (SubType.south);
\draw[umlarrow] (SubSubType2.north) -- ++(0,0.5) -| (SubType.south);

\end{tikzpicture}
\end{center}
\caption{Arvshierarki med klassen \code{introprog.BlockGame} som bastyp.}
\label{snake:fig:game-hierarchy}
\end{figure}


\item Ormar, monster och frukt ska utgå från bastypen \code{Entity} enligt typhierarkin i ~\ref{snake:fig:entity-hierarchy}.

\begin{figure}[H]
\begin{center}
\newcommand{\TextBox}[1]{\raisebox{0pt}[1em][0.5em]{#1}}
\tikzstyle{umlclass}=[rectangle, draw=black,  thick, anchor=north, text width=2.5cm, rectangle split, rectangle split parts = 3]
\begin{tikzpicture}[inner sep=0.5em,scale=1.0, every node/.style={transform shape}]

  \node [umlclass, rectangle split parts = 1, xshift=0.0cm, yshift=4.5cm] (BaseType)  {
              \textit{\textbf{\centerline{\TextBox{\code{Entity}}}}}
%              \nodepart[align=left]{second}\code{def x: T} \newline \code{def y: T}
          };


  \node [umlclass, rectangle split parts = 1, xshift=-3.0cm, yshift=2.5cm] (SubType1)  {
              \textit{\textbf{\centerline{\TextBox{\code{CanMove}}}}}
%              \nodepart[align=left]{second}\code{val x: Int} \newline \code{val y: Int}
          };

\node [umlclass, rectangle split parts = 1, xshift=-4.75cm, yshift=0.5cm] (SubSubType01)  {
            {\textbf{\centerline{\TextBox{\code{Snake}}}}}
%            \nodepart[]{second}\TextBox{\code{val dim: Int}}
};

\node [umlclass, rectangle split parts = 1, xshift=-1.5cm, yshift=0.5cm] (SubSubType02)  {
            {\textbf{\centerline{\TextBox{\code{Monster}}}}}
%            \nodepart[]{second}\TextBox{\code{val dim: Int}}
};


\node [umlclass, rectangle split parts = 1, xshift=3.0cm, yshift=2.5cm] (SubType2)  {
            \textit{\textbf{\centerline{\TextBox{\code{CanTeleport}}}}}
%            \nodepart[]{second}\TextBox{\code{val dim: Int}}
        };

\node [umlclass, rectangle split parts = 1, xshift=1.75cm, yshift=0.5cm] (SubSubType1)  {
            {\textbf{\centerline{\TextBox{\code{Apple}}}}}
%            \nodepart[]{second}\TextBox{\code{val dim: Int}}
        };

\node [umlclass, rectangle split parts = 1, xshift=5.0cm, yshift=0.5cm] (SubSubType2)  {
            {\textbf{\centerline{\TextBox{\code{Banana}}}}}
%            \nodepart[]{second}\TextBox{\code{val dim: Int}}
        };


\draw[umlarrow] (SubType1.north) -- ++(0,0.5) -| (BaseType.south);
\draw[umlarrow] (SubType2.north) -- ++(0,0.5) -| (BaseType.south);
\draw[umlarrow] (SubSubType1.north) -- ++(0,0.5) -| (SubType2.south);
\draw[umlarrow] (SubSubType2.north) -- ++(0,0.5) -| (SubType2.south);
\draw[umlarrow] (SubSubType01.north) -- ++(0,0.5) -| (SubType1.south);
\draw[umlarrow] (SubSubType02.north) -- ++(0,0.5) -| (SubType1.south);

\end{tikzpicture}
\end{center}
\caption{Arvshierarki med klassen \code{Entity} som bastyp.}
\label{snake:fig:entity-hierarchy}
\end{figure}


\item \code{Entity} representerar en varelse i ett spel och ska se ut så här:
\scalainputlisting{../workspace/w10_snake/Entity.scala}
% \begin{Code}
% trait Entity {
%   def draw():   Unit
%   def erase():  Unit
%   def update(): Unit
%   def reset():  Unit
% }
% \end{Code}
Metoderna \code{draw} resp. \code{erase} anropas vid ritning resp. radering. Metoden \code{reset} återställer ursprungstillståndet. Metoden \code{update} anropas en gång i varje runda i spel-loopen. Predikatet \code{isOccupyingBlockAt} ger sant om positionen \code{p} finns bland de block som varelsen ockuperar på skärmen.

\item \code{CanMove} representerar en entitet som kan röra sig i en viss hastighet, enligt:
\scalainputlisting{../workspace/w10_snake/CanMove.scala}

% \begin{Code}
% trait MovingEntity extends Entity {
%   def move(): Unit
%
%   var movesPerSecond: Double = 20.0
%
%   final def millisBetweenMoves(): Int =
%     (1000 / movesPerSecond).round.toInt max 1
%
%   private var _timestampLastMove: Long = System.currentTimeMillis
%   final def timestampLastMove = _timestampLastMove
%
%   override final def update(): Unit =
%     if (System.currentTimeMillis > _
%           timestampLastMove + millisBetweenMoves) {
%       _timestampLastMove = System.currentTimeMillis
%       move()
%     }
% }
% \end{Code}

\item \code{CanTeleport} representerar en entitet som finns på en viss plats men som efter ett visst antal uppdateringar utan förvarning teleporterar sig till en ny position:
\scalainputlisting{../workspace/w10_snake/CanTeleport.scala}

\item Det ska finnas en enumeration \code{State} i singelobjektet \code{SnakeGame} som representerar spelets övergripande tillstånd enligt följande:
\begin{Code}
package snake 

object SnakeGame:
  enum State:
    case Starting, Playing, GameOver, Quitting
  export State.* // gör alla tillstånd synliga i SnakeGame
\end{Code}

\item Vid varje runda i spelloopen ska följande logik exekveras. Denna kod placeras förslagsvis i \code{gameLoopAction}, se vidare \code{SnakeGame} i avsnitt \ref{lab:snake:tips}.
\begin{Code}
    if state == Playing && !isPaused then
      _iterationsSinceStart += 1
      entities.foreach(_.erase())
      entities.foreach(_.update())
      entities.foreach(_.draw())
      onIteration()
      if isGameOver then enterGameOverState()
\end{Code}

\item Det ska finnas ett singelobjekt \code{Colors} där alla färger som används i spelet samlas.

\item Filen \code{pairs.scala} ska enligt laborationsförberedelser i övningsuppgift   \ref{exe:inheritance:labprep-pair} på sidan \pageref{exe:inheritance:labprep-pair} innehålla
\code{Pair[T]}, \code{Dim}, \code{Pos}, \code{Dir}, \code{North}, \code{South}, \code{East}, \code{West}. Se workspace här:\\
\url{https://github.com/lunduniversity/introprog/tree/master/workspace/}

\item Klassen \code{Player} ska se ut som följer:

\end{enumerate}

\scalainputlisting[basicstyle=\ttfamily\fontsize{10.5}{13}\selectfont]
{../workspace/w10_snake/Player.scala}




\subsection{Valbara krav -- varje person ska välja minst ett}\label{lab:snake:extra-reqts}

Varje person i gruppen ska implementera \emph{minst ett} (gärna flera) av kraven nedan. Vid implementation av flera av dessa krav blir spelet väsentligt roligare.
\begin{itemize}[nosep, label={$\square$}]

\item \textbf{\code{Points}}. Inför ett poängsystem, där poängen beror på t.ex. längden på svansen, antalet steg, antalet svängar, antal uppätna äpplen, etc.

\item \textbf{\code{Highscore}}. Spelet ska visa en lista med de spelare som fått flest poäng.

\item \texttt{\textbf{Äpple}}. Om inte redan ingår bland obl. krav enl.~ \ref{lab:snak:table-reqt}.

\item \textbf{\code{Monster}}. Om inte redan ingår bland obl. krav enl. 
\ref{lab:snak:table-reqt}.

\item \textbf{\code{Banan}}. Om inte redan ingår bland obl. krav enl. 
\ref{lab:snak:table-reqt}.

\item \textbf{\code{SelfTailCrash}}. Om en spelare kör in i sin egen orms svans så är spelet förlorat. (Om detta krav ej implementeras så \emph{får} man köra igenom sin egen svans utan att något händer.)

\item \textbf{\code{BoundaryCrash}}. Om en spelare kör utanför spelplanen så är spelet förlorat. (Om detta krav ej implementeras så ska ormen fortsätta på andra sidan spelplanen när man når kanten.)

\item \textbf{\code{EnterPlayerName}}. Spelare kan ange sitt namn, t.ex. via en dialog eller genom argument till \code{main}. Namnet används i meddelandefältet vid poängräkning och i meddelanden om vem som vunnit.

\item \textbf{\code{OnePlayerGame}}. Du kan välja att implementera \code{OnePlayerGame} om det inte redan ingår i de obligatoriska kraven.

\item \textbf{\code{TwoPlayerComp extends Competition}}. Två spelare ska kunna tävla i en bäst-av-$n$-matcher-tävling i en sekvens av \code{TwoPlayerGame.play}, där den som vinner flest matcher blir blir totalvinnare.

\item \textbf{\code{SinglePlayerComp extends Competition}}. Flera spelare ska kunna tävla i en-persons-Snake, där den som får flest poäng av $n$ \code{OnePlayerGame}-spel blir totalvinnare.

\item \textbf{\code{Tournament extends Competition}}. Många spelare ska kunna spela en turnering.\footnote{\url{https://en.wikipedia.org/wiki/Tournament}} Namnen på spelarna läses in från en textfil. Valbara varianter:

\begin{itemize}[nosep, label={$\square$}]
\item \textbf{\code{KnockOut extends Tournament}}. Det ska gå att spela en utslagsturnering, som avslutas med final efter semi-final, etc., beroende på antal spelare.
\item \textbf{\code{RoundRobin extends Tournament}}. Det ska gå att spela en alla-möter-alla-turnering, där alla möjliga par av spelare möts i slumpvis ordning.
\end{itemize}

\end{itemize}


\subsection{Tips och förslag}\label{lab:snake:tips}

I detta stycke presenteras skisser till några av de klasser som behövs i enlighet med designkraven. Det är tillåtet att ändra, ta bort och lägga till, så länge de obligatoriska designkraven uppfylls. Koden finns här: \\
\url{https://github.com/lunduniversity/introprog/tree/master/workspace/}

% Här följer en skiss på klassen \code{Apple}:
% \scalainputlisting%[basicstyle=\ttfamily\fontsize{9.1}{12.2}\selectfont]
% {../workspace/w10_snake/Apple.scala}
% %
% Här följer en skiss på klassen \code{Banana}:
% \scalainputlisting%[basicstyle=\ttfamily\fontsize{9.1}{12.2}\selectfont]
% {../workspace/w10_snake/Banana.scala}
% Bananens ''kropp'' består av tre vertikalt ordnade blockpositioner i stället för en. Låt \code{pos}-attributet t.ex. betyda det översta av de tre bananblocken.

% Här följer en skiss på klassen \code{Banana}:
% \scalainputlisting%[basicstyle=\ttfamily\fontsize{9.1}{12.2}\selectfont]
% {../workspace/w10_snake/Monster.scala}
% Monsterkroppen består av fem blockpositioner ordnade som ett kryss. Låt \code{pos}-attributet t.ex. betyda det mittersta av de fem monsterblocken.


Här följer en skiss på klassen \code{Snake}:
\scalainputlisting[basicstyle=\ttfamily\fontsize{9}{12}\selectfont]
{../workspace/w10_snake/Snake.scala}


Här följer en skiss på den abstrakta klassen \code{SnakeGame} med de abstrakta metoderna \code{isGameOver} och \code{play} som överskuggas i de efterföljande underklasserna \code{OnePlayerGame} och \code{TwoPlayerGame}:

\scalainputlisting[basicstyle=\ttfamily\fontsize{9}{11.6}\selectfont]
{../workspace/w10_snake/SnakeGame.scala}

\vspace{1em}

Om gruppen funderar på att använda git och github: 
\begin{itemize}[nolistsep]
  \item Diskutera i gruppen om alla har kunskaper nog för att köra git och github, samt för- och nackdelar med det. 
  \item Om inte alla är bekväma med git och github så överväg om ni vill göra manuell versionshantering med kopiering av nya filer via USB-minne, ssh eller upp- och nedladdning via molnlagring. Efter en konkret upplevelse av manuell versionshantering så får du en djupare förståelse för behovet av verktygsstöd för versionshantering och det blir extra motiverande att lära sig git.
  \item Diskutera arbetssätt. Hur ska ni använda github issues, git branch, etc? Eller ska alla pusha till main branch? Ska ni använda github pull requests, github reviews, etc.?
  \item Kolla så att du har en \texttt{.gitignore} innan du gör push, så att inte t.ex. maskinkodsfiler hamnar i ert repo, vilket kan medföra knepigt städjobb och onödiga merge-konflikter. Exempel på en lämplig \code{.gitignore} finns här: \\\url{https://github.com/lunduniversity/introprog/blob/master/workspace/w10_snake/.gitignore} 
  \item \textbf{Var noga med att göra ert github-repo privat!} Det är inte tillåtet att dela labblösningar på internet -- då kan du efter disciplinärende dömas som skyldig till medhjälp till fusk och du kan bli avstängd från dina studier. 
\end{itemize}



\input{modules/w11-context-chapter.tex}
%%!TEX encoding = UTF-8 Unicode
\chapter{Varians och kontextparametrar}\label{chapter:W11}
Begrepp som ingår i denna veckas studier:
\begin{itemize}[noitemsep,label={$\square$},leftmargin=*]
\item övre- och undre typgräns
\item varians
\item kontravarians
\item kovarians
\item typjoker
\item kontextgräns
\item typkonstruktor
\item egentyp
\item typjoker
\item givet värde (given)
\item kontextparameter (using)
\item ad hoc polymorfism
\item typklass
\item api
\item kodläsbarhet
\item granskningar\end{itemize}


%!TEX encoding = UTF-8 Unicode
%!TEX root = ../exercises.tex

\ifPreSolution

\Exercise{\ExeWeekELEVEN}\label{exe:W11}

\begin{Goals}
\item Kunna förklara vad en kontextparameter är. 
\item Kunna förklara nyttan med kontextparametrar jämfört med en globala variabler och defaultargument vid lösning av konfigurationsproblemet.
%\item Känna till hur givna sorteringsordningar används för egendefinierade typer.
\item Kunna använda enkla kontextuella abstraktioner med \code{given} och \code{using}.
%\item Känna till existensen av, funktionen hos, och relationen mellan klasserna \code{Ordering} och \code{Comparator}, samt  \code{Ordered} och \code{Comparable}.

\end{Goals}

\begin{Preparations}
\item \StudyTheory{11}
\end{Preparations}

\BasicTasks %%%%%%%%%%%%%%%%

\else

\ExerciseSolution{\ExeWeekELEVEN}

\BasicTasks %%%%%%%%%%%

\fi

\WHAT{Kontextparameter.}

\QUESTBEGIN

\Task  \what~Deklarera följande funktioner som tar ett heltal som kontextparameter. Skapa även en \code{given}-deklaration som erbjuder det givna heltalsvärdet noll:
\begin{REPLnonum}
scala> def f(using i: Int) = i + 1

scala> def g(x: Int)(using y: Int) = x + y
\end{REPLnonum}

\Subtask Anropa funktionerna \code{f} och \code{g} med ett explicit givet argument som skiljer sig från det givna heltalsvärdet med hjälp av \code{using} i anropet. Vad händer om du utelämnar \code{using}?

\Subtask Anropa funktionerna \code{f} och \code{g} utan att ange \code{using}-argument. Förklara vad som händer. 

\Subtask Går det att blanda vanliga parametrar och kontextparametrar i samma parameterlista? Om inte vad händer?

\SOLUTION

\TaskSolved \what

\SubtaskSolved
\begin{REPLnonum}
scala> given Int = 0
val res0: Int = 83

scala> f(using 41)
val res1: Int = 42

scala> g(41)(using 42)
val res2: Int = 83
\end{REPLnonum}
\noindent 
Om man glömmer \code{using} vid explicit kontextargument blir det kompileringsfel. Kompilatorn blir ''förvirrad'' och tror att du försöker ge ett ''vanligt'' argument till en (i detta fallet) icke-existerande ''vanlig'' parameterlista.  
\begin{REPLnonum}
scala> f(41)
-- [E050] Type Error: ----------------------------------------------
1 |f(41)
  |^
  |method f does not take more parameters
  |
  | longer explanation available when compiling with `-explain`
1 error found

scala> :setting -explain

scala> f(41)
-- [E050] Type Error: ----------------------------------------------
1 |f(41)
  |^
  |method f does not take more parameters
  |-----------------------------------------------------------------
    | Explanation (enabled by `-explain`)
  |- - - - - - - - - - - - - - - - - - - - - - - - - - - - - - - - -
  | You have specified more parameter lists than defined in the 
    method definition(s).
   -----------------------------------------------------------------

scala> g(41)(42)
-- [E050] Type Error: ----------------------------------------------
1 |g(41)(42)
  |^^^^^
  |method g does not take more parameters
  |-----------------------------------------------------------------
    | Explanation (enabled by `-explain`)
  |- - - - - - - - - - - - - - - - - - - - - - - - - - - - - - - - -
  | You have specified more parameter lists than defined in the 
    method definition(s).
   -----------------------------------------------------------------
\end{REPLnonum}
Det är inte vanligt att ange \code{using}-parametrar explicit; det vanligaste är att låta kompilatorn framkalla ett givet värde.

\SubtaskSolved Det givna värdet \code{0} binds till motsvarande kontextparameter, som ska vara deklarerad i en egen parameterlista som börjar med \code{using}.
\begin{REPLnonum}
scala> f
val res3: Int = 1

scala> g(42)
val res4: Int = 42
\end{REPLnonum}

\SubtaskSolved Nej, det blir kompileringsfel om man försöker blanda vanliga parametrar och kontextparametrar i en och samma parameterlista:
\begin{REPLnonum}
scala> def h(i: Int, using j: Int) = i + j
-- [E040] Syntax Error: ---------------------------------------------
1 |def h(i: Int, using j: Int) = i + j
  |                    ^
  |                    ':' expected, but identifier found
\end{REPLnonum}
Det är ett medvetet val att kräva separata parameterlistor, så att det inte ska uppstå förvirring om huruvida en vanlig parameter eller kontextparameter avses. 

\QUESTEND



\WHAT{Flera olika givna värden i lokal kontext.}

\QUESTBEGIN

\Task \what~Olika värden beroende på kontext.
\begin{Code}
case class Delta(value: Int)
object Delta:
  given default: Delta = Delta(1)

def inc(x: Int)(using dx: Delta) = x + dx.value

object Context1:
  val a = inc(1)

object Context2:
  given Delta = Delta(42)
  val a = inc(1)

\end{Code}

\Subtask Vilket värde har \code{Context1.a}? 

\Subtask Vilket värde har \code{Context2.a}? 

\Subtask Förklara vad som händer.

\SOLUTION

\TaskSolved \what

\SubtaskSolved 2

\SubtaskSolved 43

\SubtaskSolved När kompilatorn försöker framkalla ett givet värde att automatiskt använda som argument till \code{using}-parametern \code{dx}, så letar den i den kontext som är närmast anropet först. Om det finns ett givet värdet i kompanjonsobjektet för parametertypen så tar kompilatorn detta i sista hand, om inget annat givet värde hittas närmare anropet.

\QUESTEND




\WHAT{Lösning på konfigurationsproblemet med hjälp av givna värden.}

\QUESTBEGIN

\Task  \what~ Antag att vi vill kunna konfigurera beteendet hos en funktion för att göra den mer flexibel. Nedan visas tre principiellt olika sätt att göra detta på för en funktion \code{greet} som skriver ut en hälsning: 1) en globalt åtkomlig variabel, 2) defaultargument, samt 3) kontextuell abstraktion med \code{given} och \code{using}.

\begin{Code}
object GlobalVar:
  case class GreetConfig(greeting: String, receiver: String)
  object GreetConfig:
    val default = GreetConfig(greeting = "Hello", receiver = "World")
    var config = default
  
  def greetMsg = 
    s"${GreetConfig.config.greeting} ${GreetConfig.config.receiver}!"

object DefaultArgs:
  case class GreetConfig(greeting: String, receiver: String)
  object GreetConfig:
    val default = GreetConfig(greeting = "Hello", receiver = "World")
  
  def greetMsg(config: GreetConfig = GreetConfig.default) =
    s"${config.greeting} ${config.receiver}!"

object GivenVal:
  case class GreetConfig(greeting: String, receiver: String)
  object GreetConfig:
    given default: GreetConfig = GreetConfig("Hello", "World")
  
  def greetMsg(using g: GreetConfig) = s"${g.greeting} ${g.receiver}"
\end{Code}

\Subtask Skriv kod som testar de olika varianterna ovan. Visa speciellt hur du kan använda default-konfigurationen och därefter ge en konfiguration som skiljer sig från \code{default}. 

\Subtask Vad är för- och nackdelar med de olika varianterna ovan? Diskutera speciellt vilken/vilka lösningar som medger flera lokala konfigurationer utan att de påverkar varandra.

\Subtask Förklara vad som händer vid anrop av \code{summon[GivenVal.GreetConfig]}. 

\Subtask Vad händer om du försöker framkalla ett givet värde för en typ som inte har något sådant?

\Subtask Måste det givna värdet vara unikt?

\SOLUTION


\TaskSolved \what\\

\noindent Nedan visas test av de tre olika lösningarna som givits i uppg. \SubtaskSolved

\noindent Efter varje test diskuteras tillhörande för- och nackdelar, som efterfrågas i uppg. \SubtaskSolved

\begin{Code}
def testGlobalVar(useDefault: Boolean = true) = 
  import GlobalVar.*
  if useDefault then println(greetMsg) else 
    GreetConfig.config = GreetConfig("Godmorgon", "världen")
    println(greetMsg)
\end{Code}
Eftersom \code{config} här är en förändringsbar variabel, så kan en ändring på ett ställe påverka helt andra delar av programmet, vilket ibland kan vara en fördel, men ofta en nackdelen eftersom det kan vara svårt att förstå vad som händer bara genom att läsa en enskild del av programmet -- en förändring av \code{config} kan ju ske varsomhelst. Det är lätt att glömma ändra till baka till default-värdet, om det är det som förväntas.
\begin{REPL}
scala> testGlobalVar(); testGlobalVar(false); testGlobalVar()
Hello World!
Godmorgon världen!
Godmorgon världen!
\end{REPL}

\begin{Code}[numbers=left]
def testDefaultArgs(useDefault: Boolean = true) =
  import DefaultArgs.*
  if useDefault then println(greetMsg()) else 
    println(greetMsg(GreetConfig("Godmorgon","världen")))
\end{Code}
Här sker ingen tillståndsförändring och default-användning är enkel, men det går inte enkelt att göra avsteg från default som  gäller i en lokal kontext; vid \emph{varje} enskilt anrop behöver du explicit ange alla de argument som inte ska vara default, så som visas på rad 4 ovan. Ändring av default har bara lokal påverkan. Om alla argument ska följa default, så gäller det att inte glömma anropa med tomt parentespar: \code{greetMsg()}. (Vad händer annars?)
\begin{REPL}
scala> testDefaultArgs(); testDefaultArgs(false); testDefaultArgs()
Hello World!
Godmorgon världen!
Hello World!
\end{REPL}
Med kontextparametrar är flexibiliteten större; \code{using}-parametrar låter användaren själv styra vad som gäller i olika sammanhang och själva anropet blir enkelt oavsett om det är default-värdet eller andra, i den lokala kontexten, givna värden som önskas. Ändring av default har bara lokal påverkan, men den har påverkan på godtyckligt många anrop i den lokala kontexten --  argument som skiljer sig kan alltså vara givna en gång utan att behöva upprepas vid varje anrop. Vid anrop där man vill låta kompilatorn framkallar givna värden för kontextparametern ska inga parenteser användas, och anropen bli därmed korta och enkla.
\begin{Code}
def testGivenVal(using g: GivenVal.GreetConfig) = println(g.greetMsg)
\end{Code}

\begin{REPL}
scala> testGivenVal
Hello World

scala> def localContext =
         import GivenVal.*
         given GreetConfig = GreetConfig("Godmorgon","världen")
         testGivenVal

scala> localContext
Godmorgon världen

scala> testGivenVal
Hello World
\end{REPL}


\SubtaskSolved  Kompilatorn framkallar ett givet värde i den lokala kontexten:
\begin{REPL}
scala> summon[GivenVal.GreetConfig]
val res0: GivenVal.GreetConfig = GreetConfig(Hello,World)
\end{REPL}
Kompilatorn följer denna prioritetsordning i sökandet efter ett unikt givet värde:
\begin{enumerate}[nolistsep,noitemsep]
\item \textbf{Explicita} argument till kontextparametrar märkta med \code{using}
\item \code{given} och \code{import given ...} i aktuell namnrymd \Eng{current scope} 
\item \code{given}-värden i \textbf{kompanjonsobjekt} för den använda typen.
\end{enumerate}
Om flera givna värden kan framkallas för typer som ingår i en gemensam arvshierarki så väljer kompilatorn det givna värdet som är av den \emph{mest specifika} typen.\\

\SubtaskSolved Det blir kompileringsfel om kompilatorn inte hittar ett givet värde för den typ som avses.

\begin{REPL}
scala> summon[Long]
-- Error: ----------------------------------------------------------------
1 |summon[Long]
  |            ^
  |            no given instance of type Long was found for parameter x of 
               method summon in object Predef
1 error found
\end{REPL}

\SubtaskSolved Ja! Det får \emph{inte} vara tvetydigt vilket givet värde som ska framkallas:
\begin{REPL}
scala> def tvetydigt =
     |   given a: Int = 42
     |   given b: Int = 43
     |   summon[Int]
     | 
-- Error: ------------------------------------------------------------------
4 |  summon[Int]
  |             ^
  |ambiguous given instances: both given instance b and given instance a 
  |match type Int of parameter x of method summon in object Predef
1 error found

\end{REPL}
Läs mer om kontexuella abstraktioner här:\\\url{https://docs.scala-lang.org/scala3/reference/contextual/}


\QUESTEND


\ExtraTasks


\WHAT{Kontextparameter och givet värde.}

\QUESTBEGIN

\Task  \what~Prova nedan i REPL.
\begin{REPL}
scala> def add(x: Int)(using y: Int) = x + y
scala> add(1)(using 2)
scala> add(1)
scala> given ngtNamn = 42
scala> add(1)
\end{REPL}
\Subtask Vad blir felmeddelandet på rad 3 ovan? 

\Subtask Varför fungerar det på rad 5 utan fel?

\Subtask Definiera och testa en motsvarande funktion \code{sub} som kan subtrahera ett givet värde.

\SOLUTION


\TaskSolved \what

\SubtaskSolved 
\begin{REPL}
scala> add(1)
-- Error: ------------------------------------------------------------------
1 |add(1)
  |      ^
  |    no given instance of type Int was found for parameter y of method add
1 error found
\end{REPL}

\SubtaskSolved Nu finns ett givet värde som kompilatorn automatiskt kan fylla i på platsen vid anropet.

\SubtaskSolved \code{def sub(x: Int)(using y: Int) = x - y}

\QUESTEND

\AdvancedTasks %%%%%%%%%%%


\WHAT{Varians och typgränser.}

\QUESTBEGIN

\Task  \what~ Koden nedan är en modell av husdjur med följande innebörd: Husdjur kan vara friska eller sjuka och föds i normalfallet friska. Det kan finnas många katter och hundar, vilka alla är olika slags husdjur.

\begin{Code}
trait Pet(var isHealthy: Boolean = true)
class Cat extends Pet()
class Dog extends Pet()

\end{Code}

\Subtask Förändra koden nedan så att efterföljande REPL-sats \emph{inte} ger kompileringsfel?
\begin{Code}
case class Box[A](x: A)
\end{Code}
\begin{REPLnonum}
scala> val b: Box[Any] = Box[Cat](Cat())
\end{REPLnonum}

\Subtask Prova nedan i REPL och förklara vad som händer.      
\begin{REPLnonum}
scala> val v: Vector[Pet] = Vector[Cat](Cat())

scala> val s: Set[Pet] = Set[Cat](Cat())

scala> :settings -explain

scala> val s: Set[Pet] = Set[Cat](Cat())
\end{REPLnonum} 
\emph{Ledtråd:} I Scalas standardbibliotek så är ärver \code{Set[T]} funktionstypen \code{T => Boolean} som är deklarerad kontravariant i sin inparameter.

\Subtask Det ska finnas veterinärer som kan behandla husdjur och göra dem friska. Varför fungerar inte nedan kod? Är det ett kompileringsfel eller körtidsfel?

\begin{Code}
class Vet[-A]:
  def treat(x: A): Unit = x.isHealthy = true
\end{Code}

\Subtask Inför en typgräns i veterinärens typparametern som åtgärdar felet.

\Subtask Skriv valfri kod som visar 1) att kompilatorn tillåter kattveterinärer att behandla katter men 2) förhindrar att kattveterinärer får behandla godtyckliga husdjur och att 3) en veterinär som har komptens att behandla godtyckliga husdjur kan behandla både katter och hundar. Förklara varför kompilatorn tillåter/förhindrar detta.

\SOLUTION


\TaskSolved \what

\SubtaskSolved Gör lådan flexibel i sin typparameter med ett \code{+} före typparametern enligt nedan. 
\begin{Code}
case class Box[+A](x: A)
\end{Code}
Kompilatorn tillämpar reglerna för kovarians eftersom typparametern har ett plustecken framför sig: \code{Box[Cat]} är en suptyp till \code{Box[Any]} om \code{Cat} är en subtyp till \code{Any}, vilket den ju är eftersom alla typer är subtyp till \code{Any}. 

\SubtaskSolved Förklaringen till beteendet har med olika varians att göra:
\begin{itemize}
  \item Samlingen \code{Vector} är kovariant och därmed flexibel i sin typparameter (liksom andra oföränderliga sekvenser i Scalas standardbibliotek). Kompilatorn betraktar därmed \code{Vector[Cat]} som en subtyp till \code{Vector[Pet]} eftersom \code{Cat} är en subtyp till \code{Pet}. På platser i koden där en \code{Vector[Pet]} krävs så anses \code{Vector[Cat]} överensstämma med \Eng{conforms to} \code{Vector[Pet]} och får därmed duga på dessa platser.
  \item En mängd har en apply-metod från elemttypen till \code{Boolean} som ger innehållstest. Av det skälet har man låtit \code{Set[T]} ärva \code{Function1[T, Boolean]} som är deklarerad kontravariant i \code{T}, så att en mängd kan användas där en \code{T => Boolean} förväntas. Även om det skulle vara praktiskt om Set[T] vore kovariant i \code{T}, i likhet med \code{Vector}, \code{List}, \code{Seq} etc, så kan inte \code{T} vara både kovariant och kontravariant på en och samma gång. Man har därför valt att göra \code{Set} invariant och därmed är mängder ej flexibla i sin typparameter. \code{Set[Cat]} är alltså \emph{inte} en subtyp till \code{Set[Pet]} \emph{även} om \code{Cat} är en subtyp till \code{Pet}, vilket ger kompileringsfel i uppgiftens exempel. 
  Se även \url{https://stackoverflow.com/questions/676615/why-is-scalas-immutable-set-not-covariant-in-its-type}  
  \item Med \code{:settings -explain} ger kompilatorn en längre utskrift som förklarar den bevisföring som skedde under kompileringens typkontroll.
\end{itemize}



\SubtaskSolved Det blir kompileringsfel då metoden \code{isHealthy} ej existerar för godtycklig typ.

\SubtaskSolved Lägg till en övre gräns som garanterar att metoden \code{isHealthy} finns för alla typer som kan bindas till typparametern \code{A}:
\begin{Code}
class Vet[-A <: Pet]:
  def treat(x: A): Unit = x.isHealthy = true
\end{Code}
Kompilatorn garanterar alltså att typparametern \code{A} är ''mindre än eller lika med'' \code{Pet}.

\SubtaskSolved Veterinären \code{Vet} är flexibel i sin typparameter och minustecknet anger kontravarians och därmed att \code{Vet[Pet]} är en subtyp till \code{Vet[Cat]} då \code{Cat} är en subtyp till \code{Pet}. Detta kan demonstreras med nedan exempel:
\begin{REPL}
scala> val pinkPanther = Cat()
val pinkPanther: Cat = Cat@33e7ece5

scala> val somePet: Pet = Cat()
val somePet: Pet = Cat@57f1cb96

scala> val catVet = Vet[Cat]()
val catVet: Vet[Cat] = Vet@1060e784

scala> pinkPanther.isHealthy = false

scala> catVet.treat(pinkPanther)

scala> pinkPanther.isHealthy
val res2: Boolean = true

scala> somePet.isHealthy = false

scala> catVet.treat(somePet)
-- [E007] Type Mismatch Error: --------------
1 |catVet.treat(somePet)
  |             ^^^^^^^
  |             Found:    (somePet : Pet)
  |             Required: Cat

scala> val powerVet = Vet[Pet]()
val powerVet: Vet[Pet] = Vet@2eb90ae9

scala> pinkPanther.isHealthy = false

scala> powerVet.treat(pinkPanther)

scala> pinkPanther.isHealthy
val res3: Boolean = true

scala> val pluto = Dog()
val pluto: Dog = Dog@6f27db5d

scala> pluto.isHealthy = false

scala> powerVet.treat(pluto)

scala> pluto.isHealthy
val res4: Boolean = true

\end{REPL}


\QUESTEND




\WHAT{Typklasser och kontextparametrar.}

\QUESTBEGIN

\Task  \what~  I Scala finns möjligheter till avancerad funktionsprogrammering med s.k. \textbf{typklasser} (ä.k. \emph{ad hoc polymorfism}). En typklass definierar generella beteenden som fungerar för godtyckliga befintliga typer utan att implementationen av dessa behöver ändras. Vi nosar i denna uppgift på hur kontextuella abstraktioner kan användas för att skapa typklasser i Scala, illustrerat med hjälp av givna ordningarna vid sortering.

Genom att kombinera koncepten givna värden, generiska klasser och kontextparametrar får man möjligheten till ad hoc polymorfism, exemplifierat med typklassen \code{CanCompare} nedan, som vi kan få att fungera för befintliga typer \emph{utan} att de behöver ändras. Speciellt så har vi ju inte möjligheten att lägga till metoder på befintliga typer i standardbiblioteket, eftersom det inte är våran egen kod.


\Subtask 
Vad händer nedan? Vilka rader ger felmeddelande? Varför?

\begin{REPL}
scala> trait CanCompare[T]:
         def compare(a: T, b: T): Int

scala> def sort[T](a: T, b: T)(using cc: CanCompare[T]): (T, T) =
         if cc.compare(a, b) > 0 then (b, a) else (a, b)

scala> sort(42, 41)

scala> given intComparator: CanCompare[Int] with
         override def compare(a: Int, b: Int): Int = a - b

scala> sort(42, 41)

scala> sort(42.0, 41.0)
\end{REPL}

\Subtask Definiera ett givet värde som gör så att \code{sort} fungerar för värden av typen \code{Double}.

\Subtask Definiera ett givet värde som gör så att \code{sort} fungerar för värden av typen \code{String}. \emph{Tips:} Du har nytta av de befintliga jämförelseoperatorerna på strängar, men tänk på att \code{compare} fortfarande måste returnera ett heltal även vid jämförelse av strängar.


\SOLUTION


\TaskSolved \what

\SubtaskSolved 

\begin{itemize}
  \item Först deklarerar vi en \code{trait}, \code{CanCompare}, med en generisk typparameter \code{T}. Den innehåller en abstrakt metod \code{compare} som tar två parametrar av typen \code{T} och returnerar en \code{Int}.
  \item Sedan definieras en metod \code{sort} som också tar en generisk typparameter \code{T}. Metoden tar två parametrar, a och b av typen T, samt en \code{using} parameter cc som måste vara en instans av \code{CanCompare[T]}. Inuti metoden används compare-metoden från CanCompare för att bestämma om a och b ska byta plats eller inte. 
  \item När vi försöker köra \code{sort(42, 41)} så får vi felmeddelande av kompilatorn. Anledning till detta är att det inte finns en given instans av CanCompare[Int].
  \item Vi löser detta på nästa rad med \code{given intComparator} som är av typen \linebreak CanCompare[Int]. Vi definierar även vår abstrakta metod \code{compare} från CanCompare med \code{override def compare}... När vi kör \code{sort(42,41)} på nästa rad fungerar det nu som det ska och vi får tillbaka \code{(Int, Int) = (41, 42)}
  \item När vi försöker köra sort med argument av typen \code{Double} får vi ett liknande felmeddelande som vi fick tidigare, och av samma anledning att det inte finns en CanCompare för typen Double.
\end{itemize}

\SubtaskSolved 
\begin{REPL}
scala> given doubleComparator: CanCompare[Double] with
         override def compare(a: Double, b: Double): Int = (a - b).toInt
\end{REPL}

\SubtaskSolved 
\begin{REPL}
scala> given stringComparator: CanCompare[String] with
         override def compare(a: String, b: String): Int = a.compareTo(b)
\end{REPL}

\QUESTEND





\WHAT{Användning av given ordning.}

\QUESTBEGIN

\Task \label{task:implicit-ordering} \what~  Vi ska nu skapa en funktion \code{isSorted} som är generellt användbar genom att göra givna ordningsfunktioner tillgängliga för olika typer. Funktionen  \code{def isSorted(xs: Vector[Int]): Boolean = ???} fungerar bara för samlingar av typen \code{Vector[Int]}.

Om vi i stället använder
\code{def isSorted(xs: Seq[Int]): Boolean = ???} fungerar den för olika samlingar med heltal, även \code{Vector} och \code{List}. 

\Subtask  Testa nedan funktion i REPL med heltalssekvenser av olika typ.
\begin{Code}
def isSorted(xs: Seq[Int]): Boolean = xs == xs.sorted
\end{Code}

\Subtask Det blir problem med nedan försök att göra \code{isSorted} generisk. Hur lyder felmeddelandet? Vad saknas enligt felmeddelandet?
\begin{REPLnonum}
scala> def isSorted[T](xs: Seq[T]): Boolean = xs == xs.sorted
\end{REPLnonum}

\Subtask Vi vill gärna att \code{isSorted} ska fungera för godtyckliga typer T som har en ordningsdefinition. Detta kan göras med nedan funktion där den speciella typparametern \code{[T:Ordering]} betyder att \code{isSorted} är definierad för alla samlingar där typen \code{T} har en given ordning \code{Ordering[T]}. Speciellt gäller detta för alla grundtyperna \code{Int}, \code{Double}, \code{String}, etc., som alla har specifika implementationer av typklassen \code{Ordering}.
\begin{Code}
def isSorted[T:Ordering](xs: Seq[T]): Boolean = xs == xs.sorted
\end{Code}
Testa metoden ovan i REPL enligt nedan.
\begin{REPL}
scala> isSorted(Vector(1,2,3))
scala> isSorted(List(1,2,3,1))
scala> isSorted(Vector("A","B","C"))
scala> isSorted(List("A","B","C","A"))
scala> case class Person(firstName: String, familyName: String)
scala> val persons = Vector(Person("Kim", "Finkodare"), Person("Voldemort","Fulhackare"))
scala> isSorted(persons)
\end{REPL}
Vad ger sista raden för felmeddelande? Varför?


\Subtask \emph{Implicita ordningar.} En typparameter på formen \code{[T:Ordering]} kallas kontextgräns \Eng{context bound} och föranleder kompilatorn att automatiskt expandera funktionshuvudet för \code{isSorted} med en kontextparameter. I stället för att använda \code{[T:Ordering]} kan vi själva lägga till en kontextparameter som motsvarar kontextgränsen. Då får vi också tillgång till ett namn, här nedan \code{ord}, på den implicita ordningen och kan använda det namnet i funktionskroppen och anropa metoder som är medlemmar av typklassen \code{Ordering}. (Namnet på kontextparametern kan också utelämnas, men då får vi istället gå omvägen via inbyggda funktionen \code{summon[T]} för att be kompilatorn leta upp den givna instansen för den typparameter som ges vid anropet.)

\begin{CodeSmall}
def isSorted[T](xs: Seq[T])(using ord: Ordering[T]): Boolean =
  xs.zip(xs.tail).forall(x => ord.lteq(x._1, x._2))
\end{CodeSmall}

Objekt av typen \code{Ordering} har jämförelsemetoder som t.ex. \code{lteq} (förk. \emph{less than or equal}) och \code{gt} (förk. \emph{greater than}).

Det finns givna ordningar för alla grundtyper i standardbiblioteket, alltså t.ex. \code{Ordering[Int]}, \code{Ordering[String]}, etc.
Testa så att kompilatorn hittar ordningen för samlingar med värden av några grundtyper. Kontrollera även enligt nedan att det fortfarande blir problem för egendefinierade klasser, t.ex. \code{Person}  (detta ska vi råda bot på i uppgift \ref{task:custom-ordering}).
\begin{REPL}
scala> isSorted(Vector(1,2,3))
scala> isSorted(Array(1,2,3,1))
scala> isSorted(Vector("A","B","C"))
scala> isSorted(List("A","B","C","A"))
scala> class Person(firstName: String, familyName: String)
scala> val persons = Vector(Person("Kim", "Finkodare"), Person("Robin","Fulhack"))
scala> isSorted(persons)
\end{REPL}

\Subtask \emph{Importera implicita ordningsoperatorer från en \code{Ordering}.} Om man gör import på ett \code{Ordering}-objekt får man tillgång till implicita konverteringar som gör att jämförelseoperatorerna fungerar. Testa nedan variant av \code{isSorted} på olika sekvenstyper och verifiera att \code{<=}, \code{>}, etc., nu fungerar enligt nedan.
\begin{CodeSmall}
def isSorted[T](xs: Seq[T])(given ord: Ordering[T]): Boolean = {
  import ord._
  xs.zip(xs.tail).forall(x => x._1 <= x._2)
}
\end{CodeSmall}


\SOLUTION

\TaskSolved \what

\SubtaskSolved 

\SubtaskSolved 
Exempel på tester:
\begin{REPL}
scala> isSorted(Vector(1,2,3))
val res0: Boolean = true

scala> isSorted(Vector(1,2,4,3))
val res1: Boolean = false

scala> isSorted(List(1,2,3))
val res2: Boolean = true

scala> isSorted(List(1,2,4,3))
val res3: Boolean = false
\end{REPL}

\SubtaskSolved 
\begin{REPL}
scala> given stringComparator: CanCompare[String] with
         override def compare(a: String, b: String): Int = a.compareTo(b)
\end{REPL}

\QUESTEND






\WHAT{Skapa egen implicit ordning med \code{Ordering}.}

\QUESTBEGIN

\Task \label{task:custom-ordering} \what~

\Subtask Ett sätt att skapa en egen, specialanpassad ordning för dina egna klasser är att mappa dina objekt till typer som redan har en implicit ordning. Med hjälp av metoden \code{by} i objektet \code{scala.math.Ordering} kan man skapa ordningar genom att bifoga en funktion \code{T => S} där \code{T} är typen för de objekt du vill ordna och \code{S} är någon annan typ, t.ex. \code{String} eller \code{Int}, där det redan finns en given ordning.
\begin{REPL}
scala> case class Team(name: String, rank: Int)
scala> val xs =
         Vector(Team("fnatic", 1499), Team("nip", 1473), Team("lumi", 1601))
scala> xs.sorted  // Hur lyder felmeddelandet? Varför blir det fel?
scala> val teamNameOrdering: Ordering[Team] = Ordering.by(t => t.name)
scala> xs.sorted(using teamNameOrdering)   //explicit ordning
scala> given Ordering[Team] = Ordering.by(t => t.rank)
scala> xs.sorted   // Varför funkar det nu?
\end{REPL}

\Subtask Vill man sortera i omvänd ordning kan man använda
\code{Ordering.fromLessThan} som tar en funktion \code{(T, T) => Boolean} vilken ska ge \code{true} om första parametern ska komma före, annars \code{false}. Om vi vill sortera efter \code{rank} i omvänd ordning kan vi göra så här:
\begin{REPL}
scala> val highestRankFirst: Ordering[Team] =
         Ordering.fromLessThan((t1, t2) => t1.rank > t2.rank)
scala> xs.sorted(using highestRankFirst)
\end{REPL}

\Subtask Om du har en case-klass med flera fält och vill ha en fördefinierad implicit sorteringsordning samt \emph{även} erbjuda en alternativ sorteringsordning, så kan du placera en default ordningsdefinition i ett kompanjonsobjekt; detta är nämligen ett av de ställen där kompilatorn söker sist efter eventuella implicita värden innan den ger upp att leta.
\begin{Code}
case class Team(name: String, rank: Int)
object Team:
  given highestRankFirst: Ordering[Team] = 
    Ordering.fromLessThan((t1, t2) => t1.rank > t2.rank)
  val nameOrdering: Ordering[Team] = Ordering.by(t => t.name)
\end{Code}
\begin{REPL}
scala> val xs =
         Vector(Team("fnatic", 1499), Team("nip", 1473), Team("lumi", 1601))
scala> xs.sorted
scala> xs.sorted(Team.nameOrdering)
\end{REPL}



\Subtask Det går också med kompanjonsobjektet ovan att få jämförelseoperatorer att fungera med din case-klass, genom att importera medlemmarna i lämpligt ordningsobjekt. Verifiera att så är fallet enligt nedan:
\begin{REPL}
scala> Team("fnatic",1499) < Team("gurka", 2)  // Vilket fel? Varför?
scala> import Team.highestRankFirst.given
scala> Team("fnatic",1499) < Team("gurka", 2)  // Inget fel? Varför?
\end{REPL}


\SOLUTION


\TaskSolved \what 

\SubtaskSolved \TODO

\SubtaskSolved \TODO

\SubtaskSolved \TODO

\SubtaskSolved \TODO


\QUESTEND






\WHAT{Specialanpassad ordning genom att ärva från \code{Ordered}}

\QUESTBEGIN

\Task  \what~  Om det finns \emph{en} väldefinierad, specifik ordning som man vill ska gälla för sina case-klass-instanser kan man göra den ordnad genom att låta case-klassen mixa in traiten \code{Ordered} och implementera den abstrakta metoden \code{compare}. (Detta illustrerar användning av subtypspolymorfism (d.v.s arv) i stället för ad hoc polymorfism med typklasser.)

\begin{Background}
En trait som används på detta sätt kallas \textbf{gränssnitt} \Eng{interface}, och om man \emph{implementerar} ett gränssnitt så uppfyller man ett ''kontrakt'', som i detta fall innebär att man implementerar det som krävs av ordnade objekt, nämligen att de har en konkret \code{compare}-metod. Du lär dig mer om gränssnitt i kommande kurser.
\end{Background}

\Subtask Implementera case-klassen \code{Team} så att den är en subtyp till \code{Ordered} enligt nedan skiss. Högre rankade lag ska komma före lägre rankade lag. Metoden \code{compare} ska ge ett heltal som är negativt om \code{this} kommer före \code{that}, noll om de ordnas lika, annars positivt.

\begin{Code}
case class Team(name: String, rank: Int) extends Ordered[Team]:
  override def compare(that: Team): Int = ???
\end{Code}
\emph{Tips:} Du kan anropa metoden \code{compare} på alla grundtyper, t.ex. \code{Int}, eftersom de implementerar gränssnittet \code{Oredered}. Genom att negera uttrycket blir ordningen den omvända. 

\begin{REPLnonum}
scala> -(2.compare(1))
\end{REPLnonum}

\Subtask Testa att  din case-klass nu uppfyller det som krävs för att vara ordnad.
\begin{REPLnonum}
scala> Team("fnatic",1499) < Team("gurka", 2)
\end{REPLnonum}


\Subtask Diskutera med handledare eller kursare skillnader och likheter mellan gränssnitt och typklasser, med ledning av denna och föregående uppgifter.
\SOLUTION


\TaskSolved \what

\SubtaskSolved

\begin{Code}
case class  Team(name: String, rank: Int) extends Ordered[Team]:
  override def compare(that: Team): Int = -rank.compare(that.rank)
\end{Code}

\SubtaskSolved

\begin{REPLnonum}
scala> Team("fnatic",1499) < Team("gurka", 2)
val res1: Boolean = true
\end{REPLnonum}

\SubtaskSolved Ad hoc polymorfism är mer flexibel. \TODO{mer diskussion om likheter och skillnader här...}

\QUESTEND



% \WHAT{Sortering med inbyggda sorteringsmetoder.}

% \QUESTBEGIN

% \Task  \what~  För grundtyperna (\code{Int}, \code{Double}, \code{String}, etc.) finns en fördefinierad ordning som gör så att färdiga sorteringsmetoder fungerar på alla samlingar i \code{scala.collection}. Även jämförelseoperatorerna i uppgift \ref{task:string-order-operators} fungerar enligt den fördefinierade ordningsdefinitionen för alla grundtyper. Denna ordningsdefinition är \textit{implicit tillgänglig} vilket betyder att kompilatorn hittar ordningsdefinitionen utan att vi explicit måste ange den. Detta fungerar i Scala även med primitiva \code{Array}.

% \Subtask Testa metoden \code{sorted} på några olika samlingar. Förklara vad som händer. Hur lyder felmeddelandet på sista raden? Varför blir det fel?

% \begin{REPL}
% scala> Vector(1.1, 4.2, 2.4, 42.0, 9.9).sorted
% scala> val xs = (100000 to 1 by -1).toArray
% scala> xs.sorted
% scala> xs.map(_.toString).sorted
% scala> xs.map(_.toByte).sorted.distinct
% scala> case class Person(firstName: String, familyName: String)
% scala> val ps = Vector(Person("Robin", "Finkodare"), Person("Kim","Fulhack"))
% scala> ps.sorted
% \end{REPL}

% \Subtask Om man har en samling med egendefinierade klasser eller man vill ha en annan sorteringsordning får man definiera ordningen själv. Ett helt generellt sätt att göra detta på  illustreras i uppgift \ref{task:custom-ordering}, men de båda hjälpmetoderna \code{sortWith} och \code{sortBy} räcker i de flesta fall. Hur de används illustreras nedan. Metoden \code{sortBy} kan användas om man tar fram ett värde av grundtyp och är nöjd med den inbyggda sorteringsordningen.

% Metoden \code{sortWith} används om man vill skicka med ett eget jämförelsepredikat som ordnar två element; funktionen ska returnera \code{true} om det första elementet ska vara först, annars \code{false}.

% \begin{REPL}
% scala> case class Person(firstName: String, familyName: String)
% scala> val ps = Vector(Person("Robin", "Finkodare"), Person("Kim","Fulhack"))
% scala> ps.sortBy(_.firstName)
% scala> ps.sortBy(_.familyName)
% scala> ps.sortBy  // tryck TAB två gånger för att se signaturen
% scala> ps.sortWith((p1, p2) => p1.firstName > p2.firstName)
% scala> ps.sortWith  // tryck TAB två gånger för att se signaturen
% scala> Vector(9,5,2,6,9).sortWith((x1, x2) => x1 % 2 > x2 % 2)
% \end{REPL}
% Vad har metoderna \code{sortWith} och \code{sortBy} för signaturer?

% \Subtask Lägg till attributet \code{age: Int} i case-klassen \code{Person} ovan och lägg till fler personer med olika namn och ålder i en vektor och sortera den med \code{sortBy} och \code{sortWith} för olika attribut. Välj själv några olika sätt att sortera på.



% \SOLUTION


% \TaskSolved \what


% \SubtaskSolved
% \begin{enumerate}
% \item Returnerar en sorterad \code{Vector} av \code{double}-värden
% \item Skapar en variabel xs och sparar en \code{Array} med \code{Int}-värden mellan 100000 till 1.
% \item Sorterar \code{xs = 1,2,3...}
% \item Konverterar xs till en \code{Array} av \code{String}-värden och sorterar dem lexikografiskt: \code{xs = "1", "10", "100"} etc.
% \item Konverterar xs till en \code{Array} av \code{Byte}-värden (max 127, min -128) och sorterar dem, samt tar bort dubletter: \code{xs = -128, -127, -1...}
% \item Skapar en ny klass \code{Person} som tar 2 \code{String}-argument i konstruktorn
% \item Sparar en Vector med två \code{Person}-objekt i en variabel ps
% \item Försöker kalla på \code{sorted}-metoden för klassen \code{Person}. Eftersom vi skrivit denna klass själva och inte berättat för Scala hur \code{Person}-objekt ska sorteras, resulterar detta i ett felmeddelande.
% \end{enumerate}

% \SubtaskSolved

% \begin{enumerate}
% \item ---
% \item ---
% \item Sorterar \code{Person}-objekten i ps med avseende på värdet i \code{firstName}
% \item Sorterar \code{Person}-objekten i ps med avseende på värdet i \code{familyName}
% \item \code{sortBy} tar en funktion f som argument. f ska ta ett argument av typen \code{Person} och returnera en generisk typ B.
% \item Sortera \code{Person}-objekten i ps med avseende på \code{firstName} i sjunkande ordning (omvänt från tidigare alltså)
% \item \code{sortWith} tar en funktion lt som argument. lt ska i sin tur ta två argument av typen \code{Person} och returnera ett booleskt värde.
% \item Sorterar en vektor så att värdena som är minst delbara med 2 hamnar först, och de mest delbara med 2 hamnar sist. Detta delar alltså upp udda och jämna tal.
% \end{enumerate}

% \SubtaskSolved
% Klassens signatur blir då:
% \begin{REPLnonum}
% case class Person(firstName: String, lastName: String, age: Int)
% \end{REPLnonum}

% Lägg in dem i en vektor:
% \begin{REPLnonum}
% val ps2 = Vector(Person("a", "asson", 34), Person("asson", "assonson", 1234),
% Person("anna", "Book", 2))
% \end{REPLnonum}

% Sortera dem på olika sätt:
% \begin{enumerate}
% \item
% Vektorn blir sorterad med avseende på personernas ålder i stigande ordning
% \begin{REPLnonum}
% scala> ps2.sortWith((p1, p2) => p1.age < p2.age)
% res40: scala.collection.immutable.Vector[Person] = Vector(Person(anna,Book,2),
% Person(a,asson,34), Person(asson,assonson,1234))
% \end{REPLnonum}

% \item
% Sorterar vektorn med avseende på namn, men också med avseende på ålder (i sjunkande ordning). För att komma före någon i ordningen måste alltså både namnet komma före, och åldern vara högre.
% \begin{REPLnonum}
% scala> ps2.sortWith((p1, p2) => (p1.firstName < p2.firstName) &&
% (p1.age > p2.age))
% res42: scala.collection.immutable.Vector[Person] = Vector(Person(a,asson,34),
% Person(asson,assonson,1234), Person(anna,Book,2))
% \end{REPLnonum}
% \end{enumerate}



% \QUESTEND




\input{modules/w11-context-lab.tex}

\input{modules/w12-extra-chapter.tex}
%%!TEX encoding = UTF-8 Unicode
\chapter{Fördjupning, Projekt}\label{chapter:W12}
Begrepp som ingår i denna veckas studier:
\begin{itemize}[noitemsep,label={$\square$},leftmargin=*]
\item välj valfritt fördjupningsområde
\item påbörja projekt\end{itemize}


%!TEX encoding = UTF-8 Unicode
%!TEX root = ../exercises.tex

\ifPreSolution

\Exercise{\ExeWeekTWELVE}\label{exe:W12}

\begin{Goals}
\item Denna veckas övning innehåller valfri fördjupning.
\item Sökning och sortering: 
\begin{itemize}
\item Kunna använda inbyggda sökmetoder.
\item Förstå när binärsökning är lämplig och möjlig.
\item Kunna implementera binärsökning.
\item Kunna implementera urvalssortering, både till ny samling och på plats.
\end{itemize}
\item Trådar och jämlöpande exekvering:
\begin{itemize}
\item Känna till vad en tråd är och kunna förklara begreppet jämlöpande exekvering.
\item Känna till vad metoderna \code{run} och \code{start} gör i klassen \code{Thread}.
\item Kunna skapa och starta en tråd med överskuggad \code{run}-metod.
\item Kunna skapa ett enkelt program som från två trådar tävlar om att uppdatera en variabel och förklara varför beteendet kan bli oförutsägbart.
\item Kunna använda en \code{Future} för att köra igång flera parallella beräkningar.
\item Kunna registrera en callback på en \code{Future} med metoden \code{onComplete}.
%\item Känna till att webbsidor beskrivs av HTML-kod och kunna skapa en minimal webbsida.
%\item Kunna ladda ner en webbsida med \code{scala.io.Source.fromURL}.
\end{itemize}
\end{Goals}

% \begin{Preparations}
% \item \StudyTheory{12}
% \end{Preparations}

%\BasicTasksNoLab %%%%%%%%%%%%%%%%

\subsection{Uppgifter om sökning och sortering}

\else

\ExerciseSolution{\ExeWeekTWELVE}

\subsection{Uppgifter om sökning och sortering}
%\BasicTasksNoLab

\fi




\WHAT{Tidmätning.}

\QUESTBEGIN

\Task \label{task:timed} \what~  I kommande uppgifter kommer du att ha nytta av funktionen \code{timed} enligt nedan.
\begin{Code}
def timed[T](code: => T): (T, Long) = 
  val now = System.nanoTime
  val result = code
  val elapsed = System.nanoTime - now
  println("\ntime: " + (elapsed / 1e6) + " ms")
  (result, elapsed)
\end{Code}
\Subtask Klistra in \code{timed} i REPL och testa så att den fungerar, genom att mäta hur lång tid nedan uttryck tar att exekvera.
\begin{REPL}
scala> val (v, t1) = timed{ (1 to 1000000).toVector.reverse }
scala> val (s, t2) = timed{ v.toSet }
scala> timed{ v.find(_ == 1) }
scala> timed{ s.find(_ == 1) }
scala> timed{ s.contains(1) }
\end{REPL}
\Subtask\Pen Försök förklara skillnaderna i exekveringstid mellan de olika sätten att söka reda på  talet $1$ i samlingen. Ungefär hur många gånger behöver man använda \code{contains} på heltalsmängden \code{s} för att det ska löna sig att skapa \code{s} i stället för att linjärsöka i \code{v} med \code{find} i ovan exempel?


\SOLUTION


\TaskSolved \what


\SubtaskSolved
Exekvera koden och du bör finna att det tar längre tid att hitta värdet 1 i vårt Set s än i vektorn v.

\SubtaskSolved

En vektor har en sekventiell ordning som find kan använda, medan \code{Set} är internt ordnad  på ett annat sätt för att innehållskontroll ska gå extra snabbt. Anledningen att det tar tid för \code{find} på \code{Set} är att det först måste skapas en iterator innan vår mängd kan gås igenom från början till slut. Metoden \code{contains} på \code{Set} däremot är rasande snabb beroende på den interna strukturen hos objekt av typen \code{Set} (som är smart designad med s.k. hash-koder, där det går lika snabbt att hitta ett element oavsett vart det befinner sig).



\QUESTEND




\WHAT{Sökning med inbyggda sökmetoder.}

\QUESTBEGIN

\Task  \what~

\Subtask \emph{Linjärsökning framifrån med \code{indexOfSlice}}. Studera dokumentationen för Scalas samlingsmetod \code{indexOfSlice}\footnote{\href{http://docs.scala-lang.org/overviews/collections/seqs.html}{docs.scala-lang.org/overviews/collections/seqs.html}} och skriv 8 olika uttryck i REPL som, både med en sträng och med en vektor med heltal, provar 4 olika fall: (1) finns i början, (2) finns någonstans i mitten, (3) finns i slutet, samt (4) finns ej.

\Subtask \emph{Linjärsökning bakifrån med \code{lastIndexOfSlice}}. Studera dokumentationen för Scalas samlingsmetod \code{lastIndexOfSlice}\footnote{\href{http://docs.scala-lang.org/overviews/collections/seqs.html}{docs.scala-lang.org/overviews/collections/seqs.html}} och skriv 8 olika uttryck i REPL som, både med en sträng och med en vektor med heltal, provar 4 olika fall: (1) finns i början, (2) finns någonstans i mitten, (3) finns i slutet, samt (4) finns ej.

\Subtask \emph{Sökning med inbyggd binärsökning.} Om en samling är sorterad kan man utnyttja detta för att göra snabbare sökning. Vid \textbf{binärsökning} \Eng{binary search}\footnote{\label{footnote:binarysearch}\href{https://en.wikipedia.org/wiki/Binary_search_algorithm}{en.wikipedia.org/wiki/Binary\_search\_algorithm}} börjar man på mitten och kollar vilken halva att  söka vidare i; sedan delar man upp denna halva på mitten och kollar vilken fjärdedel att söka vidare i, etc.

I objektet \code{scala.collection.Searching}\footnote{\href{http://www.scala-lang.org/api/current/scala/collection/Searching\$.html}{http://www.scala-lang.org/api/current/scala/collection/Searching\$.html}} finns en metod \code{search} som, om den importeras, erbjuder binärsökning för alla sorterade sekvenssamlingar. Om samlingen är sorterad ger den ett objekt av case-klassen \code{Found} som innehåller indexet för platsen där elementet först hittats; alternativt om det som eftersöks ej finns, ges ett objekt av case-klassen \code{InsertionPoint} som innehåller indexet där elementet borde ha varit placerad om det funnits i samlingen. Observera att om samlingen inte är sorterad är resultatet ''odefinierat'', d.v.s. något returneras men det är \emph{inte} att lita på; man måste alltså först sortera samlingen eller vara helt säker på att den är sorterad.

Undersök hur \code{search} fungerar genom att förklara vad som händer nedan. Vilken är snabbast av \code{lin} och \code{bin} nedan? Använd \code{timed} från uppgift \ref{task:timed}.

\begin{REPL}
scala> val udda = (1 to 1000000 by 2).toVector
scala> import scala.collection.Searching._
scala> udda.search(udda.last)
scala> udda.search(udda.last + 1)
scala> udda.reverse.search(udda(0))
scala> def lin(x: Int, xs: Seq[Int]) = xs.indexOf(x)
scala> def bin(x: Int, xs: Seq[Int]) = xs.search(x) match 
         case Found(i) => i
         case InsertionPoint(i) => -i
scala> timed{ lin(udda.last, udda) }
scala> timed{ bin(udda.last, udda) }
\end{REPL}

\Subtask Om en samling innehåller $n$ element, hur många jämförelser behövs då vid binärsökning i värsta fall? \emph{Tips:} Läs om algoritmen på Wikipedia\textsuperscript{\ref{footnote:binarysearch}}.


\SOLUTION


\TaskSolved \what


\SubtaskSolved
Förslag på test av \code{indexOfSlice}:
\begin{REPLnonum}
scala> List(1,2,3,35,1,23).indexOfSlice(List(35,1,23))
res73: Int = 3
scala> List(1,2,3,35,1,23).indexOfSlice(List(35,1,3))
res74: Int = -1
\end{REPLnonum}

\SubtaskSolved
Förslag på test av \code{lastIndexOfSlice}:
\begin{REPLnonum}
Vector(1,2,3,4,1,2).lastIndexOfSlice(Vector(1,2))
res2: Int = 4
Vector("apa", "banan", "majs", "banan").lastIndexOfSlice(Vector("banan"))
res3: Int = 3
Vector("apa", "banan", "majs", "banan").lastIndexOfSlice(Vector("banand"))
res4: Int = -1
\end{REPLnonum}

\SubtaskSolved
Observera att metoden \code{search} antar att samlingen är sorterad i stigande ordning. När vi inverterar ordningen kan \code{search} oftast inte hitta det vi letar efter, eftersom den kommer leta i fel halva av samlingen.

\begin{REPLnonum}
scala> val udda = (1 to 1000000 by 2).toVector
scala> import scala.collection.Searching._
scala> udda.search(udda.last)
res18: collection.Searching.SearchResult = Found(499999)
//Search hittar det sista elementet på plats 499999 i samlingen.

scala> udda.search(udda.last + 1)
res19: collection.Searching.SearchResult = InsertionPoint(500000)
//Search kan inte hitta udda.last + 1 eftersom det inte existerar i samlingen
//och returnerar således ett objekt av typen InsertionPoint med värdet 500000.
//Vårt element udda.last + 1 hade alltså legat på plats 500000 om det funnits.

scala> udda.reverse.search(udda(0))
res20: collection.Searching.SearchResult = InsertionPoint(0)
//Som förklarat innan så förutsätter search att listan är sorterad i stigande
//ordning, så den kan inte hitta elementet udda(0) = 1 när listan är inverterad.

scala> def lin(x: Int, xs: Seq[Int]) = xs.indexOf(x)
scala> def bin(x: Int, xs: Seq[Int]) = xs.search(x) match 
	case Found(i) => i
	case InsertionPoint(i) => -i

//Definierar en metod bin som använder sig av metoden search på en sekvens.
//Den ser sedan till med hjälp av "pattern matching" att bara returnera positionen
//i, och inte ett objekt av typen Found eller InsertionPoint.

scala> timed{ lin(udda.last, udda) }
time: 42.294821 ms
res22: (Int, Long) = (499999,42294821)
//För att hitta udda.last = 499999 med linjärsökning tog det ca 42ms.

scala> timed{ bin(udda.last, udda) }
time: 0.147314 ms
res23: (Int, Long) = (499999,147314)
//Binärsökning för att hitta värdet 499999 tog extremt mycket kortare tid.
//Detta för att vid varje steg i binärsökningen halveras mängden tal som
//sökningen måste kolla i. Detta är dock ett extremfall eftersom vi söker
//talet längst bak i listan. Om vi istället gjort en linjärsökning efter
//det första talet 1, hade detta gått minst lika snabbt som binärsökning.
\end{REPLnonum}

\SubtaskSolved
Det behövs $log_2(n)$ jämförelser. Detta eftersom att vi hela tiden halverar antalet element i listan vi behöver söka igenom. Så efter första jämförelsen har vi $\frac{n}{2}$ element kvar. Efter andra jämförelsen har vi $\frac{n}{2*2}$ element kvar etc. När vi bara har ett element kvar har vi hittat det vi söker efter, och har då gjort $b$ antal jämförelser. Ekvationen ser då ut på följande vis:
\begin{equation*}
\frac{n}{2^b} = 1
\end{equation*}
Enligt lagarna för logaritmer kan vi nu komma fram till vad b är:
\begin{equation*}
log_2(n) = b
\end{equation*}

\QUESTEND




\WHAT{Sök bland LTH:s kurser med linjärsökning.}

\QUESTBEGIN

\Task \label{task:linsearch-lth}\what~

\Subtask Via denna URL kan du ladda ner en tab-separerad lista med alla kurser som ges på LTH under innevarande läsår: \url{http://cs.lth.se/pgk/kurser} \\Vilken data finns i filen? Du kan undersöka detta t.ex. med:
\begin{REPLnonum}
scala> import scala.io.Source.fromURL
scala> val url = "https://fileadmin.cs.lth.se/pgk/lthkurser201819.txt"
scala> val data = fromURL(url,"UTF-8").getLines.mkString("\n")
\end{REPLnonum}

\Subtask \label{subtask:download-lthcourses} Klistra in objektet \code{courses} på sidan \pageref{lth-courses} i REPL.\footnote{Du kan ladda ner koden från: \\ \href{https://raw.githubusercontent.com/lunduniversity/introprog/master/compendium/examples/lth-courses/courses.scala}{github.com/lunduniversity/introprog/tree/master/compendium/examples/lth-courses/courses.scala}} Vad gör koden? Hur många kurser innehåller \code{courses.lth}?

\begin{figure}[h]
  \scalainputlisting[basicstyle=\ttfamily\fontsize{10.9}{14}\selectfont]{examples/lth-courses/courses.scala}
  \caption{Kod för att ladda ner data om alla kurser på LTH.}
  \label{lth-courses}
\end{figure}


\Subtask \emph{Linjärsökning med find.} Teknologen Oddput Clementina vill gå första bästa datavetenskapskurs som är på G2-nivå. Hjälp Oddput med att söka upp första förekommande kurs genom linjärsökning med samlingsmetoden \code{find}. Kurskoder vid datavetenskap börjar på EDA eller ETS\footnote{Detta är en förenklad bild av LTH:s kurskodnamnsystem. Några kurser från EIT-institutionen  kommer att slinka med, men det bortser vi ifrån i denna uppgift.}. \emph{Tips:} Du har nytta av att definiera predikatet \code{def isCS(s: String): Boolean} som i sin tur lämpligen nyttjar strängmetoden \code{startsWith}.

\Subtask \emph{Implementera linjärsökning.} Som träning ska du nu implementera en egen linjärsökningsfunktion med signaturen: \\ \code{def linearSearch[T](xs: Seq[T])(p: T => Boolean): Int = ???}
\\ Funktionen ska ta en sekvenssamling \code{xs} och ett predikat \code{p} som är en funktion som tar ett element och returnerar ett booleskt värde. Typen \code{Seq} är supertyp till alla sekvenssamlingar, så om vi använder den som parametertyp för parametern \code{xs} så fungerar funktionen för \code{Vector}, \code{Array}, \code{List}, etc. Genom typparametern \code{T} blir funktionen generisk och fungerar för godtycklig typ.
Funktionen \code{p} ska ge \code{true} om parametern är ett eftersökt element. Funktionen \code{linearSearch} ska returnera index för första hittade elementet i \code{xs} där \code{p} gäller. Om det inte finns något element som uppfyller predikatet ska -1 returneras. Skriv först pseudokod för funktionen med penna och papper. Du ska använda \code{while}.



\Subtask \label{subtask:linsearch-rndCode} Implementera en funktion \code{def rndCode: String} som genererar slumpmässiga kurskoder som består av 4 bokstäver mellan A och Z följt av 2 siffror mellan 0 och 9. \emph{Tips:} Använd REPL i kombination med en editor för att stegvis skapa och testa hjälpfunktioner som löser lämpliga delproblem.


\Subtask Använd \code{rndCode} från föregående deluppgift för att fylla en vektor kallad \code{xs} med en halv miljon slumpmässiga kurskoder. För varje slumpkod i \code{xs} sök med din funktion \code{linearSearch} efter index i vektorn \code{courses.lth} från deluppgift \ref{subtask:download-lthcourses}. Mät totala tiden för de $500000$ linjärsökningarna med hjälp av funktionen \code{timed} från uppgift \ref{task:timed}. Hur många av de slumpmässiga kurskoderna hittades bland de verkliga kurskoderna på LTH?



\Subtask Hur kan du implementera \code{linearSearch} med den inbyggda samlingsmetoden \code{indexWhere}?



\SOLUTION


\TaskSolved \what


\SubtaskSolved
Första raden innehåller kolumnnamnen \code{Kurskod KursSve KursEng Hskpoang Niva}. Därefter kommer en rad för varje kurs med kursdata enligt kolumnnamnen.

\SubtaskSolved
Koden laddar ner data och skapar en vektor med instanser av case-klassen \code{Course} med hjälp av metoden \code{fromLine}. Eftersom variabeln \code{lth} är deklarerad som \code{lazy} kommer inte \code{download()} bli anropad förrän första gången som variablen \code{lth} refereras. Antalet kurser ges av:
\begin{REPLnonum}
scala> val n = courses.lth.length
n: Int = 1104
\end{REPLnonum}

\SubtaskSolved
\begin{REPL}
scala> def isCS(s: String) = s.startsWith("EDA") || s.startsWith("ETS")
scala> val x = courses.lth.find(c => isCS(c.code) && c.level == "G2")
x: Option[courses.Course] = Some(Course(EDAF05,Algoritmer, datastrukturer och
   komplexitet,Algorithms, Data Structures and Complexity,5.0,G2))
\end{REPL}

\SubtaskSolved
\begin{Code}
def linearSearch[T](xs: Seq[T])(p: T => Boolean): Int = 
   var i = 0
   while(i < xs.length && !p(xs(i))) i += 1
   if (i < xs.length) i else -1
\end{Code}

\SubtaskSolved

\begin{Code}
def rndCode: String = 
   //randomizes from 0 to n (inclusive)
   def rnd(n: Int) = (math.random() * (n + 1)).toInt

   def letter = (rnd('Z' - 'A') + 'A').toChar
   def dig = ('0' + rnd(9)).toChar
   Seq(letter, letter, letter, letter, dig, dig).mkString
\end{Code}

\SubtaskSolved

\begin{Code}
val xs = Vector.fill(500000)(rndCode)
val(ixs, elapsedLin) =
  timed { xs.map(x => linearSearch(courses.lth)(_.code == x)) }
val found = ixs.filterNot(_== -1).size
\end{Code}

\SubtaskSolved

\begin{Code}
def linearSearch[T](xs: Seq[T])(p: T => Boolean): Int = xs.indexWhere(p)
\end{Code}



\QUESTEND

%%%%%% GAMLA VARIANTEN AV OVAN UPPGIFT
%%%%%% -- Funkar ej längre URL-api till LTH:S databas
% \WHAT{Sök bland LTH:s kurser med linjärsökning.}
%
% \QUESTBEGIN
%
% \Task \label{task:linsearch-lth} \what~ OBS! Använd \code{https} och \emph{inte} \code{http} i webbadresserna i denna och nästa uppgift, för att det ska fungera.
%
% \Subtask Surfa till denna URL:\\{%\nolinebreak[4]
% \footnotesize\url{https://kurser.lth.se/lot/?lasar=17_18&soek_text=&sort=kod&val=kurs&soek=t}}
% \\
% och inspektera HTML-koden i din webbläsare genom att trycka \emph{Ctrl+U} (fungerar i Firefox och Chrome). Rulla ner till rad 171 och framåt. Var finns antalet poäng för respektive kurs i HTML-koden?
%
% \Subtask \label{subtask:download-lthcourses} Klistra in objektet \code{courses} på sidan \pageref{lth-courses} med kommandot \code{:paste} i REPL.\footnote{Du kan ladda ner koden från: \\ \href{https://raw.githubusercontent.com/lunduniversity/introprog/master/compendium/examples/lth-courses/courses.scala}{github.com/lunduniversity/introprog/tree/master/compendium/examples/lth-courses/courses.scala}} Vad gör koden? Hur många kurser innehåller \code{lth2017}?
%
% \begin{figure}
%   \scalainputlisting[basicstyle=\ttfamily\fontsize{10.9}{14}\selectfont]{examples/lth-courses/courses.scala}
%   \caption{Kod för att söka bland kurser från LTH:s webbsida.}
%   \label{lth-courses}
% \end{figure}
%
%
% \Subtask \emph{Linjärsökning med find.} Teknologen Oddput Clementina vill gå första bästa datavetenskapskurs som är på G2-nivå. Hjälp Oddput med att söka upp första bästa kurs genom linjärsökning med samlingsmetoden \code{find}. Kurskoder vid datavetenskap börjar på EDA eller ETS\footnote{Detta är en förenklad bild av LTH:s kurskodnamnsystem. Några kurser från EIT-institutionen  kommer att slinka med, men det bortser vi ifrån i denna uppgift.}. \emph{Tips:} Du har nytta av att definiera predikatet \code{def isCS(s: String): Boolean} som i sin tur lämpligen nyttjar strängmetoden \code{startsWith}.
%
% \Subtask \emph{Implementera linjärsökning.} Som träning ska du nu implementera en egen linjärsökningsfunktion med signaturen: \\ \code{def linearSearch[T](xs: Seq[T])(p: T => Boolean): Int = ???}
% \\ Funktionen ska ta en sekvenssamling \code{xs} och ett predikat \code{p} som är en funktion som tar ett element och returnerar ett booleskt värde. Funktionen \code{p} ska ge \code{true} om parametern är ett eftersökt element. Funktionen \code{linearSearch} ska returnera index för första hittade elementet i \code{xs} där \code{p} gäller. Om det inte finns något element som uppfyller predikatet ska -1 returneras. Skriv först pseudokod för funktionen med penna och papper. Använd \code{while}.
%
% Typen \code{Seq} är supertyp till alla sekvenssamlingar, så om vi använder den som parametertyp för parametern \code{xs} så fungerar funktionen för \code{Vector}, \code{Array}, \code{List}, etc. Genom typparametern \code{T} blir funktionen generisk och fungerar för godtycklig typ.
%
%
%
% \Subtask \label{subtask:linsearch-rndCode} Implementera en funktion \code{def rndCode: String} som genererar slumpmässiga kurskoder som består av 4 bokstäver mellan A och Z följt av 2 siffror mellan 0 och 9. \emph{Tips:} Använd REPL  för att stegvis bygga upp hjälpfunktioner som du, när de fungerar som de ska, klistrar in i ett editorfönster som lokala funktioner där du utvecklar den slutliga koden för en lättläst, koncis och fungerande \code{rndCode}.
%
%
% \Subtask Använd \code{rndCode} från föregående deluppgift för att fylla en vektor kallad \code{xs} med en halv miljon slumpmässiga kurskoder. För varje slumpkod i \code{xs} sök med din funktion \code{linearSearch} efter index i vektorn \code{courses.lth2017} från deluppgift \ref{subtask:download-lthcourses}. Mät totala tiden för de $500000$ linjärsökningarna med hjälp av funktionen \code{timed} från uppgift \ref{task:timed}. Hur många av de slumpmässiga kurskoderna hittades bland de verkliga kurskoderna på LTH?
%
%
%
% \Subtask\Pen Hur kan du implementera \code{linearSearch} med den inbyggda samlingsmetoden \code{indexWhere}?
%
%
%
% \SOLUTION
%
%
% \TaskSolved \what
%
%
% \SubtaskSolved
% Den finns som värde för en \emph{td} tagg, på följande vis: \code{<td class="mitt">2</td>}.
%
% \SubtaskSolved
% Koden laddar ner html-koden för sidan \\ \mbox{\small\url{https://kurser.lth.se/lot/?lasar=17_18&soek_text=&sort=kod&val=kurs&soek=t}} och sparar den i en vektor. Sedan filtreras ut endast de rader som innehåller strängen ”kurskod” så att all onödig HTML-kod försvinner. Sedan konverteras detta, för varje rad, till \code{Course}-objekt med hjälp av metoden \code{fromHtml}. Eftersom variabeln \code{lth2017} är deklarerad som \code{lazy} kommer inte \code{download()} bli anropad förrän vi vill komma åt variabeln. Vi startar alltså processen genom att referera variabeln \code{lth2017} i objektet \code{courses}:
%
% \begin{REPLnonum}
% courses.lth2017
% \end{REPLnonum}
% Detta generarar en lång lista med \code{Course}-objekt. Antalet kurser är således lika med storleken på vektorn \code{lth2017}.
%
% \begin{REPLnonum}
% courses.lth2017.size
% res38: Int = 1101
% \end{REPLnonum}
%
% \SubtaskSolved
% \begin{REPL}
% scala> def isCS(s: String) = s.startsWith("EDA") || s.startsWith("ETS")
% scala> val x = courses.lth2017.find(c => isCS(c.code) && c.level == "G2").get
% x: courses.Course = Course(EDAF05,Algoritmer, datastrukturer och komplexitet,Algorithms, Data Structures and Complexity,5.0,G2)
% \end{REPL}
% Obs: metoden \code{find} returnerar ett objekt av typen \code{Option}. För att få värdet som är lagrat i detta objekt krävs det att man kallar på \code{get}.
%
% \SubtaskSolved
% \begin{Code}
% def linearSearch[T](xs: Seq[T])(p: T => Boolean): Int = {
%    var i = 0
%    while(i < xs.size && !p(xs(i))) i += 1
%    if (i < xs.size) i else -1
% }
% \end{Code}
%
% \SubtaskSolved
%
% \begin{Code}[language=Scala]
% def rndCode: String = {
%    //randomizes from 0 to n (inclusive)
%    def rnd(n: Int) = (math.random() * (n + 1)).toInt
%
%    def letter = (rnd('Z' - 'A') + 'A').toChar
%    def dig = ('0' + rnd(9)).toChar
%    Seq(letter, letter, letter, letter, dig, dig).mkString
% }
% \end{Code}
%
% \SubtaskSolved
%
% \begin{Code}
% val lthCourses = courses.lth2017 //avoid including download time
% val xs = Vector.fill(500000)(rndCode)
% val(ixs, elapsedLin) = timed{
% xs.map(x => linearSearch(lthCourses)(_.code == x))}
% val found = ixs.filterNot(_== -1).size
% \end{Code}
%
% \SubtaskSolved
%
% \begin{Code}
% def linearSearch[T](xs: Seq[T])(p: T => Boolean): Int =
%   xs.indexWhere(p)
% \end{Code}
%
%
%
% \QUESTEND






\WHAT{Sök bland LTH:s kurser med binärsökning.}

\QUESTBEGIN

\Task  \what~Sökalgoritmen BINSEARCH kan formuleras med nedan pseudokod:

\begin{algorithm}[H]
 \SetKwInOut{Input}{Indata}\SetKwInOut{Output}{Utdata}

 \Input{En växande sorterad sekvens $xs$ med $n$ heltal och \\ ett eftersökt heltal $key$}
 \Output{Ett heltal $i \geq 0$ som anger platsen där $x$ finns, eller ett negativt tal $i$ där $-i$ motsvarar platsen där $x$ ska sättas in i sorterad ordning om $x$ ej finns i samlingen.}
 sätt intervallet ($low$, $high$) till ($0$, $n - 1$) \\
 $found \leftarrow \bf{false}$ \\
 $mid \leftarrow -1$\\
 \While{$low \leq high$~$\bf{and}~\bf{not}$ $found$}{
   $mid \leftarrow $ platsen mitt emellan $low$ och $high$\\
   \eIf{$xs(mid)$ == $key$}{$found \leftarrow \bf{true}$}{
     \eIf{$xs(mid) < key$}{$low \leftarrow mid + 1$}{$high \leftarrow mid - 1$}
    }
 }
 \eIf{$found$}{$mid$}{$-(low + 1)$}
\end{algorithm}

\Subtask Prova algoritmen ovan med penna och papper på en sorterad sekvens med mindre än 10 heltal. Prova om algoritmen fungerar med ett jämnt antal tal, ett udda antal tal, en sekvens med ett heltal och en tom sekvens. Prova både om talet du letar efter finns och om det inte finns.

\Subtask Implementera binärsökning i en funktion med signaturen\\
\code{def binarySearch(xs: Seq[String], key: String): Int = ??? }\\
och testa i REPL för olika fall. Vad händer om sekvensen inte är sorterad?

\Subtask Använd \code{binarySearch} för att leta efter LTH-kurser enligt nedan. Använd \code{rndCode}, \code{timed} och \code{courses} från tidigare uppgifter.
\begin{Code}
def binarySearch(xs: Seq[String], key: String): Int = ???

val lthCodesSorted = courses.lth.map(_.code).sorted
val xs = Vector.fill(500000)(rndCode)
val (_, elapsedBin) =
  timed{xs.map(x => binarySearch(lthCodesSorted, x))}
val (_, elapsedLin) =
  timed{xs.map(x => linearSearch(lthCodesSorted)(_ == x))}
println(elapsedLin / elapsedBin)
\end{Code}


\Subtask Hur mycket snabbare blev binärsökningen jämfört med linjärsökningen?\footnote{Vid en körning på en i7-4970K med 4.0GHz tog \code{elapsedLin} cirka $3000~ms$ och \code{elapsedBin} cirka $60~ms$. Binärsökning var alltså i detta fall ungefär $50$ gånger snabbare än linjärsökning.}


\SOLUTION


\TaskSolved \what


\SubtaskSolved ---

\SubtaskSolved
\begin{Code}
def binarySearch(xs: Seq[String], key: String): Int = 
  var (low, high) = (0, xs.size - 1)
  var found = false
  var mid = -1

  while (low <= high && !found) do
    mid = (low + high) / 2
    if xs(mid) == key then found = true
    else if xs(mid) < key then low = mid + 1
    else high = mid - 1
  end while
  if found then  mid else -(low + 1)
\end{Code}

\SubtaskSolved
Med en i7-3770K @ 3.50Hz tog sökningarna följande tid:

\begin{itemize}
\item Binärsökning: \code{time: 142.6 ms}
\item Linjärsökning: \code{time: 3316.5 ms}
\end{itemize}

Med en i7-8700T @ 2.40GHz tog sökningarna följande tid:
\begin{itemize}
\item Binärsökning: \code{time: 81.5 ms}
\item Linjärsökning: \code{time: 5138.6 ms}
\end{itemize}




\SubtaskSolved
Binärsökningen var ca 23 gånger snabbare på en i7-3770K @ 3.50Hz och ca 63 gånger snabbare på en i7-8700T CPU @ 2.40GHz.



\QUESTEND





\WHAT{Insättningssortering.}

\QUESTBEGIN

\Task  \what~ Implementera sortering av en heltalssekvens till en  sekvens med \textbf{insättningssortering} \Eng{insertion sort} i en funktion med följande signatur:
\begin{Code}
def insertionSort(xs: Seq[Int]): Seq[Int] = ???
\end{Code}

\emph{Lösningsidé:} Skapa en ny, tom sekvens som ska bli vårt sorterade resultat. För varje element i den osorterade sekvensen: Sätt in det på rätt plats i den nya sorterade sekvensen.

\Subtask \emph{Pseudokod:} Kör nedan pseudokod med papper och penna t.ex. på sekvensen 5 1 4 3 2 1. Rita minnessituationen efter varje runda i loopen. Här använder vi internt i funktionen föränderliga \code{ArrayBuffer} som är snabb på insättning och avslutar med \code{toVector} så att vi lämnar ifrån oss en oföränderlig sekvens.

\begin{algorithm}[H]
    $result \leftarrow$ en ny, tom ArrayBuffer \\
    \ForEach{element $e$ \bf{in} $xs$}{
      $pos \leftarrow$  leta upp rätt position i $result$ \\
      stoppa in $e$ på plats $pos$ i $result$
    }
    $result$.toVector
\end{algorithm}


\Subtask Implementera \code{insertionSort}. Använd en \code{while}-loop för att implementera rad 3 i pseudokoden. Sök upp dokumentationen för metoden \code{insert} på \code{ArrayBuffer}. Testa  \code{insert} på \code{ArrayBuffer} i REPL och verifiera att den kan användas för att stoppa in på slutet på den ''oanvända'' positionen som är precis efter sista positionen. Vad händer om man gör \code{insert} på positionen \code{size + 2}?

Klistra in din implementation av \code{insertionSort} i REPL och testa så att allt fungerar:
\begin{REPL}
scala> insertionSort(Vector())
res0: Seq[Int] = Vector()

scala> insertionSort(Vector(42))
res1: Seq[Int] = Vector(42)

scala> insertionSort(Vector(1,2,3))
res2: Seq[Int] = Vector(1, 2, 3)

scala> insertionSort(Vector(5,1,4,3,2,1))
res3: Seq[Int] = Vector(1, 1, 2, 3, 4, 5)
\end{REPL}


\SOLUTION

\TaskSolved \what


\SubtaskSolved ---

\SubtaskSolved

\begin{Code}
def insertionSort(xs: Seq[Int]): Seq[Int] = 
  val result = scala.collection.mutable.ArrayBuffer.empty[Int]
  for e <- xs do
    var pos = 0
    while pos < result.size && result(pos) < e do pos += 1
    result.insert(pos,e)
  end for
  result.toVector
\end{Code}

\QUESTEND





\WHAT{Sortering på plats.}

\QUESTBEGIN

\Task  \what~ Implementera sortering på plats \Eng{in-place} i en \code{Array[String]} med urvalssortering \Eng{selection sort}

\emph{Lösningsidé:} För alla index $i$: sök $minIndex$ för ''minsta'' strängen från plats $i$ till sista plats och byt plats mellan strängarna på plats $i$ och plats $minIndex$. Se även animering här: \href{https://sv.wikipedia.org/wiki/Urvalssortering}{sv.wikipedia.org/wiki/Urvalssortering}

Implementera enligt nedan skiss.  \emph{Tips:} Du har nytta av en modifierad variant av lösningen till uppgift \ref{task:minindex} i kapitel \ref{chapter:W02}.
\begin{Code}
def selectionSortInPlace(xs: Array[String]): Unit = 
  def indexOfMin(startFrom: Int): Int = ???
  def swapIndex(i1: Int, i2: Int): Unit = ???
  for i <- 0 to xs.size - 1 do swapIndex(i, indexOfMin(i))
\end{Code}




\SOLUTION


\TaskSolved \what


\begin{Code}
def selectionSortInPlace(xs: Array[String]): Unit = 
  def indexOfMin(startFrom: Int): Int = 
    var minPos = startFrom
    var i = startFrom + 1
    while (i < xs.size) do
      if (xs(i) < xs(minPos)) minPos = i
      i += 1
    end while
    minPos
  end indexOfMin

  def swapIndex(i1: Int, i2: Int): Unit = 
    val temp = xs(i1)
    xs(i1) = xs(i2)
    xs(i2) = temp
  end swapIndex  

  for i <- 0 to xs.size - 1 do swapIndex(i, indexOfMin(i))
end selectionSortInPlace
\end{Code}


\QUESTEND


\clearpage

%\ExtraTasks %%%%%%%%%%%%%%%%%%%




\WHAT{Undersök om en sekvens är sorterad.}

\QUESTBEGIN

\Task \label{task:isSorted} \what~   Ett enkelt och lättläst sätt att undersöka om en sekvens är sorterad visas nedan.
\begin{REPL}
scala> def isSorted(xs: Vector[Int]): Boolean = xs == xs.sorted
\end{REPL}


\Subtask\Pen  Om \code{xs} har $10^6$ element, hur många jämförelser kommer i värsta fall att ske med \code{isSorted} enligt ovan? Metoden \code{sorted} använder algoritmen Timsort\footnote{\href{http://stackoverflow.com/questions/14146990/what-algorithm-is-used-by-the-scala-library-method-vector-sorted}{stackoverflow.com/questions/14146990/what-algorithm-is-used-by-the-scala-library-method-vector-sorted}}. Sök upp antalet jämförelser i värstafallet på Wikipedia.

Denna lösning är dock relativt långsam för stora samlingar. Man behöver ju inte först sortera  för att avgöra om det är sorterat (om man inte ändå hade tänkt sortera av andra skäl), det räcker att kolla att elementen är i växande ordning.

\Subtask\label{subtask:issorted} Implementera en effektivare variant av \code{isSorted} som använder en \code{while}-sats och kollar att elementen är i växande ordning. Din algoritm ska sluta söka så fort osorterade element hittats.

\Subtask\Pen Vad blir antalet jämförelser i värstafallet med metoden i deluppgift \ref{subtask:issorted} om du har $n$ element?


\Subtask \label{subtask:isSorted-zip} Man kan kolla om en sekvens är sorterad med det listiga tricket att först zippa sekvensen med sin egen svans och sedan kolla om alla element-par uppfyller sorteringskriteriet, alltså \code{xs.zip(xs.tail).forall(???)} där \code{???} byts ut mot lämpligt predikat. Vilken typ har 2-tupeln \code{xs.zip(xs.tail))} om \code{xs} är av typen \code{Vector[Int]}? Implementera \code{isSorted} med detta listiga trick. 

\SOLUTION


\TaskSolved \what



\SubtaskSolved Det tar i värsta fall $O(n*log(n))$ för timsort att sortera listan med $n$ element. Sedan krävs $n$ stycken jämförelser mellan den sorterade och osorterade listan. Det totala antalet jämförelser i värstafallet uppgår därför till max $n + n*log(n)$. För $10^6$ element blir det ca $10^7$ jämförelser.
\begin{REPLnonum}
scala> val n = 1E6
val n: Double = 1000000.0

scala> def worstCase(n: Double) = n + n * math.log(n)
def worstCase(n: Double): Double

scala> println(s"i värsta fall med n=$n så blir det ${worstCase(n)} jämförelser")
i värsta fall med n=1000000.0 så blir det 1.4815510557964273E7 jämförelser
\end{REPLnonum}

\SubtaskSolved En mer effektiv version av \code{isSorted} som avbryter sökningen när ett osorterat element upptäcks:
\begin{Code}
def isSorted(xs: Vector[Int]): Boolean = 
  if xs.length > 1 then
    var i = 0
    var result = true
    while i < xs.length-1 && result do 
      if xs(i) > xs(i+1) then result = false
      i += 1
    end while
    result
  else true
end isSorted
\end{Code}

\SubtaskSolved I värsta fall behöver man göra $n - 1$ parvisa jämförelser, om alla ligger i sorterad ordning utom den sista.


\SubtaskSolved 2-tupeln är av typen \code{(Int, Int)}.

\begin{Code}
def isSorted(xs: Vector[Int]): Boolean =
  xs.zip(xs.tail).forall(x => x._1 <= x._2)
\end{Code}



\QUESTEND






\WHAT{Insättningssortering på plats.}

\QUESTBEGIN

\Task  \what~ Implementera och testa sortering på plats i en array med heltal med \footnote{\href{https://en.wikipedia.org/wiki/Insertion_sort}{en.wikipedia.org/wiki/Insertion\_sort}}.

Implementera och testa funktionen nedan i Scala med följande signatur:
\begin{Code}
  def insertionSort(xs: Array[Int]): Unit
\end{Code}
Placera metoden i ett objekt med lämpligt namn, samt skapa ett huvudprogram med testkod. Kompilera och kör från terminalen. Börja med att skriva sorteringsalgoritmen i pseudokod.

% \Subtask Implementera och testa metoden nedan i Java med följande signatur:
% \begin{Code}[language=Java]
%   public static void insertionSort(int[] xs)
% \end{Code}
% Placera metoden i en klass med lämpligt namn, samt skapa ett huvudprogram med testkod. Börja med att skriva sorteringsalgoritmen i pseudokod.

\SOLUTION


\TaskSolved \what


\begin{Code}
def insertionSort(xs: Array[Int]): Unit = 
  for elem <- 1 until xs.length if xs.length > 0 do
    var pos = elem
    while pos > 0 && xs(pos) < xs(pos - 1) do
      val temp = xs(pos -1)
      xs(pos -1) = xs(pos)
      xs(pos) = temp
      pos -= 1
    end while
  end for
end insertionSort
\end{Code}

% \SubtaskSolved

% \begin{Code}[language=Java]
% public static void insertionSort(int[] xs) {

%     if (xs.length < 1)
%         return;

%     for (int i = 1; i < xs.length; i++) {
%         int pos = i;

%         for (; pos > 0 && xs[pos] < xs[pos - 1]; pos--) {
%             int temp = xs[pos - 1];
%             xs[pos - 1] = xs[pos];
%             xs[pos] = temp;
%         }
%     }
% }
% \end{Code}



\QUESTEND



\clearpage

%\AdvancedTasks


\WHAT{Sortering till ny sekvens med urvalssortering.}

\QUESTBEGIN

\Task  \what~ Implementera och testa sortering till ny sekvens med urvalssortering\footnote{\href{https://en.wikipedia.org/wiki/Selection_sort}{en.wikipedia.org/wiki/Selection\_sort}} i Scala, enligt nedan skiss.  Du har nytta av lösningen till uppgift \ref{task:minindex} i kapitel \ref{chapter:W02}.
\begin{Code}
def selectionSort(xs: Seq[String]): Seq[String] = 
  def indexOfMin(xs: Seq[String]): Int = ???
  val unsorted = xs.toBuffer
  val result = scala.collection.mutable.ArrayBuffer.empty[String]
  /*
  så länge unsorted inte är tom 
    minPos = indexOfMin(unsorted)
    elem   = unsorted.remove(minPos)
    result.append(elem)
  */
  result.toVector
end selectionSort
\end{Code}



\SOLUTION


\TaskSolved \what


\begin{Code}
def selectionSort(xs: Seq[String]): Seq[String] = 
  def indexOfMin(xs: Seq[String]): Int = xs.indexOf(xs.min)
  val unsorted = xs.toBuffer
  val result = scala.collection.mutable.ArrayBuffer.empty[String]
  while !unsorted.isEmpty do
    val minPos = indexOfMin(unsorted)
    val elem = unsorted.remove(minPos)
    result.append(elem)
  end while
  result.toVector
end selectionSort
\end{Code}


\QUESTEND














%
%
% \WHAT{NEEDS A TOPIC DESCRIPTION}
%
% \QUESTBEGIN
%
% \Task  \what~ Fördjupa dig inom webbteknologi.
%
% \Subtask Lär dig om HTML här: \url{http://www.w3schools.com/html/}
%
% \Subtask Lär dig om Javascript här: \url{http://www.w3schools.com/js/}
%
% \Subtask Lär dig om CSS här: \url{http://www.w3schools.com/css/}
%
% \Subtask Lär dig om Scala.JS här: \url{http://www.scala-js.org/}\SOLUTION
%
%
% \TaskSolved \what
%
% \QUESTEND


\subsection{Uppgifter om trådar och jämlöpande exekvering}

\WHAT{Trådar.}

\QUESTBEGIN

\Task  \what~   Klassen \code{java.lang.Thread} används för att skapa  \textbf{trådar} med jämlöpande exekvering \Eng{concurrent execution}. På så sätt kan man få olika koddelar att köra samtidigt.

Klassen \code{Thread} definierar en tom \code{run}-metod. Vill man att tråden ska göra något vettigt får man överskugga \code{run} med det man vill ska göras.

En tråd körs igång med metoden \code{start} och då anropas automatiskt \code{run}-metoden och tråden exekverar koden i \code{run} jämlöpande med övriga trådar. Om man anropar \code{run} direkt blir det \emph{inte} jämlöpande exekvering.

\Subtask Skapa en tråd som gör något som tar lite tid och kör med \code{run} resp. \code{start}.
\begin{REPL}
def zzz = { print("zzzzzz"); Thread.sleep(5000); println(" VAKEN!")}
zzz
val t2 = new Thread{ override def run = zzz }
t2.run; println("Gomorron!")
t2.start; println("Gomorron!")
t2.start
\end{REPL}

\Subtask Vad händer om man anropar \code{start} mer än en gång på samma tråd?

\Subtask Skapa två trådar med överskuggade \code{run}-metoder och kör igång dem samtidigt enligt nedan. Vilken ordning skrivs hälsningarna ut efter rad 3 resp. rad 4 nedan? Förklara vad som händer.
\begin{REPL}
val g = new Thread{ override def run = for i <- 1 to 100 do print("Gurka ") }
val t = new Thread{ override def run = for i <- 1 to 100 do print("Tomat ") }
g.run; t.run
g.start; t.start
\end{REPL}

\Subtask Använd \code{Thread.sleep} enligt nedan. Är beteendet helt förutsägbart (deterministiskt)? Förklara vad som händer. Du kan avbryta REPL med CTRL+C eller SHIFT+CTRL+C, beroende på din terminalinställningar.%
\footnote{\href{http://stackoverflow.com/questions/6248884/can-i-stop-the-execution-of-an-infinite-loop-in-scala-repl}{stackoverflow.com/questions/6248884/can-i-stop-the-execution-of-an-infinite-loop-in-scala-repl}}.
\begin{REPL}
def ibland(block: => Unit) = new Thread {
  override def run = while(true) { block; Thread.sleep(600) }
}.start
ibland(print("zzz ")); ibland(print("snark ")); ibland(println("hej!"))
\end{REPL}


\SOLUTION


\TaskSolved \what
     %%%TODO number  1 %%%starts with: \emph{Trådar.}  %%%

\SubtaskSolved   -

\SubtaskSolved  \code {java.lang.IllegalThreadStateException}. Det går inte att starta en tråd mer än en gång. Tråden kan därför inte startas om när den redan har exekverats.

\SubtaskSolved   När \code {start} anropas exekveras koden i \code{run} parallellt. Därför skrivs \code{Gurka} och \code{Tomat} ut omlöpande. Om istället \code{run} anropas direkt blir det inte jämnlöpande exekvering och \code{Gurka} skrivs ut 100 gånger, sedan skrivs \code{Tomat} ut 100 gånger.

\SubtaskSolved   \code{Thread.sleep} pausar inte tråden i exakt den tiden som angets. Alltså kommer det skrivas ut \code{zzz snark hej!} i de flesta fall, men det är inte garanterat.



\QUESTEND






\WHAT{Jämlöpande variabeluppdatering.}

\QUESTBEGIN

\Task \label{task:racecondition} \what~   Skriv klasserna \code{Bank} och \code{Kund} i en editor och klistra sedan in koden i REPL.

\begin{Code}
class Bank:
  private var _saldo = 0;
  def saldo: Int = _saldo
  def sättIn(): Unit = _saldo += 1 
  def taUt(): Unit   = _saldo -= 1 
end Bank

class Kund(bank: Bank):
  def slösaSpara(): Unit = 
    bank.taUt()
    Thread.sleep(1)
    bank.sättIn()
  end slösaSpara
end Kund
\end{Code}

\Subtask Använd funktionen \code{ibland} från föregående uppgift och kör nedan rader i REPL. Resultatet av jämlöpande variabeluppdatering blir här heltokigt och leder till mycket upprörda bankkunder och -ägare. Förklara vad som händer.

\begin{REPL}
val bank = new Bank
println(bank.saldo)
bank.sättIn()
println(bank.saldo)
bank.taUt()
println(bank.saldo)

val bamse = new Kund(bank)
val skutt = new Kund(bank)

bamse.slösaSpara()
skutt.slösaSpara()
println(bank.saldo)

def ofta(block: => Unit) = new Thread { // varje millisekund
  override def run = while true do { block;  Thread.sleep(1)} 
}.start

ofta(bamse.slösaSpara()); ofta(skutt.slösaSpara())

def ibland(block: => Unit) = new Thread {  // varje 600 ms
  override def run = while(true) do { block; Thread.sleep(600) }
}.start

ibland(println(bank.saldo))
\end{REPL}


\SOLUTION


\TaskSolved \what
     %%%TODO number  2 %%%starts with: \emph{Jämlöpande variabeluppdat%%%

\SubtaskSolved  I \code{slösaSpara} hämtas saldot, ändras och placeras tillbaka i minnet -  fördröjs -  upprepas. Om \code{bamse} blir klar med att ladda, ändra och lagra innan skutt gör detsamma blir det problem, då de tävlar om vem som får uppdatera \Eng{race contiion}. Problemet innan en tråd kan lagra det förändrade värdet laddar den andra tråden det gamla värdet. Bara en av dessa trådar vinner racet och får lagra sitt ändrade tal och den andra ändringen går förlorad. \code{skutt} och \code{bamse} blir alltså upprörda för att inte alla dess uttag och insättningar registreras.


\QUESTEND






\WHAT{Trådsäkra \code{AtomicInteger}.}

\QUESTBEGIN

\Task  \what~  Det finns stöd i JVM för att åstadkomma uppdateringar som inte kan avbrytas av andra trådar under pågånde minnesskrivning. En operation som inte kan avbrytas kallas \textbf{atomär} \Eng{atomic}. Studera dokumentationen för \code{AtomicInteger}\footnote{\href{https://docs.oracle.com/javase/8/docs/api/java/util/concurrent/atomic/AtomicInteger.html}{docs.oracle.com/javase/8/docs/api/java/util/concurrent/atomic/AtomicInteger.html}} och prova nedan kod. Förklara vad som händer.

Använd funktionerna \code{ofta} och \code{ibland} från tidigare uppgifter.
\begin{Code}
class SäkerBank:
  import java.util.concurrent.atomic.AtomicInteger
  private var _saldo = new AtomicInteger
  def saldo: Int = _saldo.get
  def sättIn(): Unit = _saldo.incrementAndGet()
  def taUt(): Unit   = _saldo.decrementAndGet()
end SäkerBank

class SäkerKund(bank: SäkerBank):
  def slösaSpara = 
    bank.taUt()
    Thread.sleep(1)
    bank.sättIn()
  end slösaSpara
end SäkerKund

\end{Code}
\begin{REPL}
val sb = new SäkerBank
val farmor = new SäkerKund(sb)
val vargen = new SäkerKund(sb)

ofta(farmor.slösaSpara); ofta(vargen.slösaSpara)

ibland(println(sb.saldo))
\end{REPL}


\SOLUTION


\TaskSolved \what
     %%%TODO number  3 %%%starts with: \emph{Jämlöpande exekvering med%%%

Nu är \code{farmor}-tråden garanterad att kunna ladda saldot, ta ut pengar/ändra och lagra innan \code{vargen}-tråden kan skriva över resultatet. I \code{slösaSpara} pausas tråden i en millisekund så \code{vargen}-tråden kan hinna ta ut pengar innan \code{farmor}-sätter hinner sätta in pengar igen och saldot blir negativt. Dock kommer alla uttag och insättningar registreras eftersom operationerna är atomära och saldot kommer återställas till noll, utan att insättningar går förlorade.


\QUESTEND






\WHAT{Jämlöpande exekvering med \code{scala.concurrent.Future}.}

\QUESTBEGIN

\Task \label{task:future} \what~   Att skapa och hålla reda på trådar kan bli ganska omständligt och knepigt att få rätt på.
Med hjälp av \code{scala.concurrent.Future} kan man på ett enklare sätta skapa jämlöpande exekvering.

\begin{Background}
Med en \code{Future} skapas jämlöpande exekvering som ''under huven'' använder ett ramverk som heter Akka\footnote{\url{http://akka.io/}}, skrivet i Scala och Java. Akka erbjuder automatisk  multitrådning med s.k. trådpooler och möjliggör avancerad parallellprogrammering på en hög  abstraktionsnivå, där man själv slipper skapa instanser av klassen \code{Thread}. I stället kan man helt enkelt placera sin kod inramad med \code|Future{ "körs parallellt" }| efter att man importerat det som behövs.
\end{Background}

\Subtask För att skapa jämlöpande exekvering med \code{Future} behöver man först göra import enligt nedan; då skapas ett exekveringssammanhang med trådpooler redo för användning. Starta om REPL och studera felmeddelandet efter rad 1 nedan. Importera därefter enligt nedan. Vad har \code{f} för typ?
\begin{REPL}
scala> concurrent.Future { Thread.sleep(1000); println("En sekund senare!") }
scala> import scala.concurrent._
scala> import ExecutionContext.Implicits.global
scala> val f = Future { Thread.sleep(1000); println("En sekund senare!") }
\end{REPL}

\Subtask Skapa en procedur \code{printLater} enligt nedan som skriver ut argumentet efter slumpmässig tid. Förklara vad som händer nedan.
\begin{REPL}
scala> def printLater(a: Any): Unit =
         Future { Thread.sleep((math.random() * 10000).toInt); print(a + " ") }
scala> (1 to 42).foreach(i => printLater(i)); println("alla är igång!")
\end{REPL}

\Subtask Skapa enligt nedan en \code{Future} som räknar ut hur många siffror det är i ett väldigt stort tal. Med \code{onComplete} kan man ange vad som ska göras när den tunga beräkningen är färdig; detta kallas att ''registrera en callback''. Vilken returtyp har \code{big}? Hur många siffror har det stora talet? Vad har \code{r} för typ? Justera argumentet till \code{big} om du inte orkar vänta på resultatet...

\begin{REPL}
scala> BigInt(10).pow(100)
scala> BigInt(10).pow(100).toString.size
scala> def big(n: Int) = Future { BigInt(n).pow(n).toString.size }
scala> big(1234567).onComplete{r => println(r + " siffror") }
\end{REPL}

\Subtask Den stora vinsten med \code{Future} är att man kan köra vidare under tiden, varför anropet av \code{Future} kallas \textbf{icke-blockerande} \Eng{non-blocking}. Det händer ibland att man ändå vill blockera exekveringen i väntan på ett resultat. Man kan då använda objektet \code{scala.concurrent.Await} och dess metod \code{result} enligt nedan. Använd \code{big} från föregående uppgift och gör en blockerande väntan på resultatet enligt nedan. Vad händer? Vad händer om du väntar för kort tid?

\begin{REPL}
scala> import scala.concurrent.duration._
scala> Await.result(big(1234567), 20.seconds)
\end{REPL}



\SOLUTION


\TaskSolved \what
     %%%TODO number  4 %%%starts with: TODO  %%%%%%%%%%%%%%%%%%%\Advan%%%

\SubtaskSolved  error: Cannot find an implicit ExecutionContext. Future behöver en ExecutionContext för att kunna köras. \code{f} är av typen Future[Unit].

\SubtaskSolved  Funktionen \code{printLater} har en Future, vilket innebär att när både \code{printLater} och \code{println} anropas i foreach-loopen exekveras de jämnlöpande. Eftersom det tar längre tid att starta upp en Future för datorn är \code{println} snabbare och skriver ut att alla är igång först. Sedan skrivs siffrorna från 1 - 42 ut med oregelbundna mellanrum eftersom tråden pausas olika länge.

\SubtaskSolved  \code{big} är en Future[Int]. Det stora talet har 7 520 383 siffror. \code{r} är av typen Try[Int] (se dokumentationen för Future om du är osäker)

\SubtaskSolved  Eftersom exekveringen blockas tills den har fått ett resultat går det inte att fortsätta skriva i REPL medan uträkningen pågår. Väntar man för kort tid får man ett TimeOutException och uträkningen avbryts.


\QUESTEND






\WHAT{Använda \code{Future} för att göra flera saker samtidigt.}

\QUESTBEGIN

\Task  \what~
I denna uppgift ska du ladda ner webbsidor parallellt med hjälp av \code{Future}, så att en nedladdning kan avslutas under tiden en annan dröjer.

\Subtask Koden för en minimal webbsida ser ut som nedan. Du kan beskåda sidan här: \url{http://fileadmin.cs.lth.se/pgk/mini.html} eller skriva in nedan kod i en fil som heter något som slutar på \texttt{.html} och öppna filen i din webbläsare.

\begin{verbatim}
<!DOCTYPE html>
<html>
<body>
HELLO WORLD!
</body>
</html>
\end{verbatim}

\Subtask För att simulera slöa webbservrar kan man ladda ner en sida via sajten \\\texttt{http://deelay.me/} \\ Ladda ner ovan sida med 2 sekunders fördröjning:\\
\url{http://deelay.me/2000/http://fileadmin.cs.lth.se/pgk/mini.html}

\Subtask Man kan ladda ner webbsidor med \code{scala.io.Source}. Vad händer nedan? Försök, med ledning av hur \code{delay} beräknas, uppskatta hur lång tid du måste vänta i medeltal, i bästa fall, respektive värsta fall, innan du kan se första webbsidan i vektorn \code{laddningar} nedan?

\begin{REPL}
scala> def ladda(url: String) = scala.io.Source.fromURL(url).getLines.toVector
scala> def slöladda(url: String) = 
         val delay = (math.random() * 1000 + 2000).toInt
         val delaySite = s"http://deelay.me/$delay/"
         ladda(delaySite+url)
       end slöladda
scala> ladda("http://fileadmin.cs.lth.se/pgk/mini.html")
scala> def seg = slöladda("http://fileadmin.cs.lth.se/pgk/mini.html")
scala> val laddningar = Vector.fill(10)(seg)
scala> laddningar(0)
\end{REPL}

\Subtask Innan vi kan köra igång en \code{Future} så måste vi, som visats i uppgift \ref{task:future} importera den underliggande exekveringsmiljön som är redo att parallelisera ditt program i trådar utan att du själv måste skapa dem. Vad händer nedan?
\begin{REPL}
scala> import scala.concurrent._
scala> import ExecutionContext.Implicits.global
scala> val f = Future(seg)
scala> f   // kolla om den är klar annars prova igen senare
scala> f
\end{REPL}

\Subtask Ladda indata utan att blockera \Eng{non-blocking input}. Förklara vad som händer nedan.
\begin{REPL}
scala> val nonBlocking = Future(Vector.fill(10)(seg))
scala> nonBlocking   // kolla igen senare om ej klar
scala> nonBlocking
\end{REPL}

\Subtask Ladda indata separat i olika parallella trådar. Förklara vad som händer nedan. Kör uttrycket på rad 3 nedan upprepade gånger i snabb följd efter varandra med pil-upp+Enter i REPL.
\begin{REPL}
scala> val para = Vector.fill(10)(Future(seg))
scala> para
scala> para.map(_.isCompleted)
scala> para.map(_.isCompleted) // studera hur de blir färdiga en efter en
scala> para(0)
\end{REPL}

\Subtask Registrera en callback med metoden \code{onComplete}. Förklara vad som händer nedan.

\begin{REPL}
scala> val action = Vector.fill(10)(Future(seg))
scala> action(0).onComplete(xs => println(s"ready:$xs"))
scala> // vänta tills laddning på plats 0 är klar
\end{REPL}

\Subtask Registrera en callback för felhantering i händelse av undantag med metoden \code{onFailure}. Förklara vad som händer nedan.
\begin{REPL}
scala> def lycka  = { Thread.sleep(3000); println(":)") }
scala> def olycka = { Thread.sleep(3000); 42 / 0; lycka }
scala> Future(lycka ).onFailure{ case e => println(s":( $e") }
scala> Future(olycka).onFailure{ case e => println(s":( $e") }
\end{REPL}



\SOLUTION


\TaskSolved \what
     %%%TODO number  5 %%%starts with: Sök upp och studera dokumentati%%%

\SubtaskSolved  -

\SubtaskSolved  -

\SubtaskSolved  Varje sida fördröjs med mellan 2 upp till 3 sekunder (2000-3000 millisekunder). Så i medeltal tar det 2.5 sekunder för varje sida att laddas. Vektorn måste fyllas innan exekveringen kan fortsätta. Därför laddas alla 10 stycken sidor in innan man kan se första websidan. Det tar därför i medeltal 2.5 x 10 = 25 sekunder.

\SubtaskSolved  \code{f} ger en Vektor fylld med strängar där varje element ges av en rad på hemsidan. Då \code{f} körs i bakgrunden kan programmet fortlöpa medan innehållet räknas ut. Du kan därför skriva \code{f} i REPL:n men det är inte säkert att processen är klar och det slutgiltiga resultatet visas.

\SubtaskSolved  Samma som ovan, förutom att det blir en vektor där varje element är i sig en vektor med strängar.

\SubtaskSolved  Ladda data parallellt så att nedladdningen sker samtidigt, men det går olika snabbt pga metoden seg.

\SubtaskSolved  Eftersom datan laddas i parallella trådar utan blockering blir de inte klara i ordning, utan i den ordningen tråden körs klart. Till slut blir alla klara och resultatet visar en vektor med \code{true} värden.

\SubtaskSolved  Metoden \code{lycka} är väldefinerad och kastar därför inga undantag. Den skriver alltid ut \code{:)}. Metoden \code{olycka} är inte väldefinierad då division med 0 ger \\\code{java.lang.ArithmeticException}. Detta fångas upp vid callbacken och det skrivs ut \code{:(} samt det specificerade undantaget.

\ExtraTasks %%%%%%%%%%%%


\QUESTEND






\WHAT{}

\QUESTBEGIN

\Task  \what~ Räkna ut stora primtal parallellt genom att använda nedan funktioner. Implementera \code{isPrime} enligt pseudokod från den engelska wikipediasidan om primtalstest\footnote{\href{https://en.wikipedia.org/wiki/Primality_test}{en.wikipedia.org/wiki/Primality\_test}} med den s.k. ''naiva algoritmen''.  Räkna ut 10 st slumpvisa primtal med 16 siffror vardera. Gör beräkningarna parallellt med hjälp av \code{Future}.

\begin{Code}
def isPrime(n: BigInt): Boolean = ???

def nextPrime(start: BigInt): BigInt = 
  var i = start
  while !isPrime(i) do i += 1 
  i
end nextPrime

def randomBigInt(nDigits: Int): BigInt = 
   def rndChar = ('0' + (math.random() * 10).toInt).toChar
   val str = Array.fill(nDigits)(rndChar).mkString
   BigInt(str)
randomBigInt
\end{Code}

\SOLUTION


\TaskSolved \what
  %%%TODO number  6 %%%

\begin{Code}
def isPrime(n: BigInt): Boolean = n match 
  case _ if (n <= 1) => false
  case _ if (n <= 3) => true
  case _ if n % 2 == 0 || n % 3 == 0 => false
  case _ =>
    var i = BigInt(5)
    while i * i < n do
      if (n % i == 0 || n % (i + 2) == 0) false
      i += 6
    end while
    true
end isPrime

import scala.concurrent.*
import ExecutionContext.Implicits.global

val primes = Vector.fill(10)(Future{nextPrime(randomBigInt(16))})
primes.foreach(_.onSuccess{case i => println(i)})
\end{Code}


\QUESTEND






\WHAT{Svara på teorifrågor.}

\QUESTBEGIN

\Task  \what~\Pen

\Subtask Vad är en tråd?

\Subtask Hur skapar man en tråd med klassen \code{Thread}?

\Subtask Hur startar man en tråd?

\Subtask Vilka problem kan man råka ut för om man uppdaterar samma resurs i flera olika trådar?

\Subtask Vad innbär det att kod är \emph{trådsäker}?

\Subtask Nämn några fördelar med att använda Future jämfört med att använda trådar direkt.


\SOLUTION


\TaskSolved \what
 %%%TODO number  7 %%%

\SubtaskSolved  Stackoverflow ger följande förklaring:

A thread is an independent set of values for the processor registers (for a single core). Since this includes the Instruction Pointer (aka Program Counter), it controls what executes in what order. It also includes the Stack Pointer, which had better point to a unique area of memory for each thread or else they will interfere with each other.

\SubtaskSolved

\begin{Code}
val thread = new Thread(new Runnable{
	def run(){println(''Det här är en tråd'')}
})
\end{Code}

\SubtaskSolved  \code{thread.start}

\SubtaskSolved  Det kan bli kapplöpning(race conditions) om vilken tråds resurser blir sparade. Vilket leder till att de andra trådarnas ändringar blir ignorerade.

\SubtaskSolved  Trådsäkerhet innebär att flera trådar kan köras parallellt utan felaktigheter i resultatet. Exempelvis får man vara väldigt försiktig om man vill ha en muterbar variabel som alla trådar ska ändra samtidigt.

\SubtaskSolved  Till exempel slipper man skapa instanser av klassen Thread eftersom man kan placera koden i en Future istället. Den löser även mycket under huven för kodaren.


\QUESTEND






\WHAT{Klasser med atomär uppdatering.}

\QUESTBEGIN

\Task  \what~ Läs om och testa klasserna AtomicBoolean, AtomicDouble och AtomicReference för atomär uppdatering i paketet \\ \code{java.util.concurrent.atomic}.

Använd några av dessa tillsammans med \code{scala.concurrent.Future}.


\SOLUTION

\TaskSolved --

\QUESTEND





\WHAT{Skapa din egen multitrådade webbserver.}

\QUESTBEGIN

\Task  \what~

\Subtask Skriv in\footnote{Eller ladda ner här: \href{https://github.com/lunduniversity/introprog/blob/master/compendium/examples/simple-web-server/webserver.scala}{github.com/lunduniversity/introprog/blob/master/compendium/examples/simple-web-server/webserver.scala}} nedan kod i en editor och spara i en fil med namn \texttt{webserver.scala} och kompilera och kör med \texttt{scala-cli run webserver.scala} och beskriv vad som händer när du med din webbläsare surfar till adressen: \\ \url{http://localhost:8089/abbasillen}

\scalainputlisting[numbers=left,basicstyle=\ttfamily\fontsize{11}{12}\selectfont]{examples/simple-web-server/webserver.scala}

\Subtask Du ska nu skapa en webbserver som gör något lite mer intressant. Den ska svara med det 13:e Fibonacci-talet\footnote{\href{https://sv.wikipedia.org/wiki/Fibonaccital}{https://sv.wikipedia.org/wiki/Fibonaccital}} om du surfar till \url{http://localhost:8089/fib/13}.
Spara din webbserver från föregående deluppgift under det nya namnet \texttt{fibserver.scala} och använd koden nedan och lägg till och ändra så att din server kan svara med Fibonaccital. Vi börjar med att räkna ut Fibonaccital i funktionen \code{compute.fib} nedan på ett onödigt processorkrävande sätt med exponentiell tidskomplexitet så att webbservern verkligen får jobba, för att i senare deluppgifter implementera \code{compute.fib} med linjär tidskomplexitet och därmed undvika onödig planetuppvärmning.
\begin{CodeSmall}
// lägg till nedan i webserver.scala från 
//    https://github.com/lunduniversity/introprog/blob/master/compendium/examples/simple-web-server/webserver.scala

object compute:
  def fib(n: BigInt): BigInt = 
    if n < 0 then 0 else
    if n == 1 || n == 2 then 1
    else fib(n - 1) + fib(n -2)
  end fib
end compute

def fibResponse(num: String) = 
  num.toIntOption match 
    case Some(n) => html.page(s"fib($n) == " + compute.fib(n))
    case None    => html.page(s"FEL: skriv ett heltal, inte $num")

def errorResponse(uri:String) = html.page(s"Error: $uri </br> use /fib/heltal")


// ändra handleRequest i start i webserver.scala till
  def handleRequest(cmd: String, uri: String, socket: Socket): Unit = 
    val os = socket.getOutputStream
    val afterSlash = uri.toString.drop(1) // skip initial slash
    println(s"afterSlash:$afterSlash")
    val response: String = 
      if afterSlash.startsWith("fib/") then fibResponse(afterSlash.stripPrefix("fib/"))
      else errorResponse(uri)
    os.write(html.header(response.size).getBytes("UTF-8"))
    os.write(response.getBytes("UTF-8"))
    os.close
    socket.close
  end handleRequest
\end{CodeSmall}

Kör i terminalen med \texttt{scala-cli run webserver.scala} och beskriv vad som händer i din webbläsare när du surfar till servern.


%%%\textbf{KOD TILL FACIT:}
%%%\scalainputlisting[numbers=left,basicstyle=\ttfamily\fontsize{11}{12}\selectfont]{examples/simple-web-server/fibserver.scala}


\Subtask Surfa efter flera stora Fibonacci-tal samtidigt i olika flikar i din browser. Hur märks det att servern bara kör i en enda tråd?

\Subtask Gör din server multitrådad med hjälp av den nya server-loopen nedan.

\begin{CodeSmall}
import scala.concurrent._
import ExecutionContext.Implicits.global

  def serverLoop(server: ServerSocket): Unit = {
    println(s"http://localhost:${server.getLocalPort}/hej")
		while (true) {
  		Try {
  		  var socket = server.accept  // blocks thread until connect
	  	  val scan = new Scanner(socket.getInputStream, "UTF-8")
		    val (cmd, uri) = (scan.next, scan.next)
			  println(s"Request: $cmd $uri")
		    Future { handleRequest(cmd, uri, socket) }.onFailure {
		      case e => println(s"Reqest failed: $e")
		    }
		  }.recover{ case e: Throwable => s"Connection failed: $e" }
		}
  }
\end{CodeSmall}

\Subtask Surfa efter flera stora Fibonacci-tal samtidigt i olika flikar i din browser. Hur märks det att servern är multitrådad?


\Subtask Det är onödigt att räkna ut samma Fibonacci-tal flera gånger. Med hjälp av en cache i form av en föränderlig \code{Map} kan du spara undan redan uträknade värden. Det funkar dock inte med en vanlig \code{scala.collection.mutable.Map} i vår multitrådade webbserver, eftersom den inte är \textbf{trådsäker} \Eng{thread-safe}. Med trådosäkra föränderliga datastrukturer blir det samma besvärliga beteende som i uppgift \ref{task:racecondition}.

Du ska i stället använda \code{java.util.concurrent.ConcurrentHashMap}. Sök upp  dokumentationen för \code{ConcurrentHashMap} och försök förstå koden nedan. Hur fungerar metoderna \code{containsKey}, \code{put} och \code{get}?
\begin{Code}
object compute {
  import java.util.concurrent.ConcurrentHashMap
  val memcache = new ConcurrentHashMap[BigInt, BigInt]

  def fib(n: BigInt): BigInt =
    if (memcache.containsKey(n)) {
      println("CACHE HIT!!! no need to compute: " + n)
      memcache.get(n)
    } else {
      println("cache miss :( must compute fib:  " + n)
      val f = fastFib(n)
      memcache.put(n, f)
      f
    }

  private def fastFib(n: BigInt): BigInt = {
    if (n < 0) 0 else
    if (n == 1 || n == 2) 1
    else fib(n - 1) + fib(n -2)
  }
}
\end{Code}

\Subtask Använd ovan \code{fib}-objekt i en ny version av din webserver. Spara den i en ny kodfil med namnet \texttt{fibserver-memcached.scala}. Undersök hur snabbt det går med stora Fibonaccital med den nya varianten. Hur stora tal kan du räkna ut? Kan servern fortsätta efter överflödad stack? Förklara varför.

\Subtask Nu när vi kan få väldigt stora Fibonacci-tal kan det vara användbart att stoppa in radbrytningar på webbsidan. Html-taggen \texttt{</br>} ger en radbrytning.
\begin{Code}
  def insertBreak(s: String, n: Int = 80): String = {
    if (s.size < n) s
    else s.take(n) + "</br>" + insertBreak(s.drop(n),n)
  }
\end{Code}
Använd den rekursiva funktionen ovan för att pilla in radbrytningstaggar på var $n$:te position i långa strängar. Testa hur det ser ut på webbsidan med ovan funktion när din server svarar med väldigt stora tal.

\Subtask Vi ska nu använda det större heap-minnet i stället för stack-minnet och därmed inte begränsas av stackens max-storlek. Skriv om \code{fastFib} så att den använder en \code{while}-sats i stället för ett rekursivt anrop. Denna uppgift är ganska klurig, men om du kör fast kan du snegla i lösningarna i Appendix för inspiration.

Hur stora tal klarar din server nu? Vad händer med servern när minnet tar slut? Hur kan du skydda servern så att den inte kan hänga sig?

\SOLUTION


\TaskSolved \what
 %%%TODO number  9 %%%

\SubtaskSolved  \code{abbasillen} skrivs ut baklänges till \code{nellisabba}.

\SubtaskSolved

\SubtaskSolved

\SubtaskSolved

\SubtaskSolved

\SubtaskSolved

\SubtaskSolved

\SubtaskSolved

\SubtaskSolved

Lösningsförslag:
\scalainputlisting[numbers=left,basicstyle=\ttfamily\fontsize{11}{12}\selectfont]{examples/simple-web-server/fibserver-threaded-memcached-while.scala}


\QUESTEND






% \WHAT{}

% \QUESTBEGIN

% \Task  \what~ Utöka din server med fler beräkningsintensiva funktioner. Exempelvis primtalsberäkningar eller beräkningar av valfritt antal decimaler av $\pi$ eller $e$. Utnyttja gärna det du lärt dig i  matematiken om summor och serieutvecklingar.

% \SOLUTION


% \TaskSolved \what
%  %%%TODO number  10 %%%

% ---


% \QUESTEND






% \WHAT{}

% \QUESTBEGIN

% \Task  \what~ Läs mer om \code{Future} och jämlöpande exekvering i Scala här:\\
% \href{http://alvinalexander.com/scala/future-example-scala-cookbook-oncomplete-callback}{alvinalexander.com/scala/future-example-scala-cookbook-oncomplete-callback}

% \SOLUTION


% \TaskSolved \what
%  %%%TODO number  11 %%%

% ---


% \QUESTEND






% \WHAT{}

% \QUESTBEGIN

% \Task  \what~ Läs mer om jämlöpande exekvering och multitrådade program i Java här: \href{http://www.tutorialspoint.com/java/java_multithreading.htm}{www.tutorialspoint.com/java/java\_multithreading.htm}  \\
% \noindent När man skriver program med jämlöpande exekvering finns det många fallgropar; det kan bli kapplöpning \Eng{race conditions} om gemensamma resurser och dödläge \Eng{deadlock} där inget händer för att trådar väntar på varandra. Mer om detta i senare kurser.


% \SOLUTION


% \TaskSolved \what
%  %%%TODO number  12 %%%

% ---


% \QUESTEND






% \WHAT{Studera dokumentationen i \code{scala.concurrent}.}

% \QUESTBEGIN

% \Task  \what~\Pen

% \Subtask Studera dokumentationen för \code{scala.concurrent.Future}\footnote{\href{http://www.scala-lang.org/api/current/scala/concurrent/Future.html}{http://www.scala-lang.org/api/current/scala/concurrent/Future.html}}. Hur samverkar \code{Future} med \code{Try} och \code{Option}? Vilka vanliga samlingsmetoder känner du igen?

% \Subtask Studera dokumentationen för \code{scala.concurrent.duration.Duration}\footnote{\href{http://www.scala-lang.org/api/current/scala/concurrent/duration/Duration.html}{www.scala-lang.org/api/current/scala/concurrent/duration/Duration.html}}. Vilka tidsenheter kan användas?

% \Subtask Vid import av \code{scala.concurrent.duration.* } dekoreras de numeriska klasserna med metoder för att skapa instanser av klassen \code{Duration}. Detta möjligörs med hjälp av klassen \code{scala.concurrent.duration.DurationConversions}. Studera dess dokumentation och testa att i REPL skapa några tidsperioder med metoderna på \code{Int}.



% \SOLUTION


% \TaskSolved \what
%  %%%TODO number  13 %%%

% \SubtaskSolved

% \SubtaskSolved

% \SubtaskSolved


% \QUESTEND






% \WHAT{}

% \QUESTBEGIN

% \Task  \what~ Fördjupa dig inom webbteknologi.

% \Subtask Lär dig om HTML, CSS och JavaScript här: \url{https://developer.mozilla.org/en-US/docs/Learn}

% \Subtask Lär dig om Scala.JS här: \url{http://www.scala-js.org/}\SOLUTION


% \TaskSolved \what
%  %%%TODO number  14 %%%

% \SubtaskSolved  ---

% \SubtaskSolved  ---

% \SubtaskSolved  ---

% \SubtaskSolved  ---
% \QUESTEND

%!TEX encoding = UTF-8 Unicode
%!TEX root = ../compendium2.tex

\Assignment{bank}

\subsection{Fokus}
\begin{itemize}[nosep,label={$\square$},leftmargin=*]
\item Kunna implementera ett helt program efter given specifikation
\item Kunna sätta samman olika delar från olika moduler
\item Förstå hur Java-klasser kan användas i Scala
\item Förstå och bedöma när immutable/mutable såväl som var/val bör användas i större sammanhang
\item Kunna använda sig av kompanjonsobjekt
\item Kunna läsa och skriva till fil
\item Kunna söka i olika datastrukturer på olika sätt
\end{itemize}

\subsection{Bakgrund}

I detta projekt ska du skriva ett program som håller reda på bankkonton och kunder i en bank. Programmet ska utöver att hålla reda på bankens nuvarande tillstånd även föra historik över alla tillståndsändringar. Historiken ska vara så pass detaljerad att det nuvarande tillståndet kan återskapas genom att återuppspela alla ändringar som finns lagrade i historiken.

Programmet ska vara helt textbaserat, man ska alltså interagera med programmet via terminalen där en meny skrivs ut och input görs via tangentbordet.

Du ska skriva större delen av programmet själv, utan någon färdig kod. Programmet ska dock följa de specifikationer som ges i uppgiften, såväl som de objektorienterade principer du lärt dig i kursen.

\subsection{Krav}

Kraven för bankapplikationen återfinns här nedan. För att bli godkänd på denna uppgift måste samtliga krav uppfyllas:

\begin{itemize}
\item Programmet ska ha följande menyval:

\begin{itemize}
\item 1. Hitta konton för en viss kontoinnehavare med angivet ID.
\item 2. Söka efter kunder på (del av) namn.
\item 3. Sätta in pengar på ett konto.
\item 4. Ta ut pengar på ett konto.
\item 5. Överföra pengar mellan två olika konton.
\item 6. Skapa ett nytt konto.
\item 7. Ta bort ett befintligt konto.
\item 8. Skriva ut bankens alla konton, sorterade i bokstavsordning efter innehavare.
\item 9. Skriva ut historiken över alla ändringar av bankens tillstånd.
\item 10. Återställa banken till tillståndet den hade vid ett givet datum. För enkelhetens skull får du permanent kassera all historik som skapades efter det datum banken återställs till.
\item 11. Avsluta.
\end{itemize}

\item När något av följande sker ska programmet notera det i historiken:
\begin{itemize}
\item Pengar sätts in på ett konto.
\item Pengar tas ut från ett konto.
\item Pengar överförs mellan två konton.
\item Ett konto skapas.
\item Ett konto tas bort.
\end{itemize}
\item Historiken ska sparas både i minnet och i en fil.
\item Då programmet startas ska det läsa in historikfilen för att återskapa tillståndet som banken hade tidigare.
\item Formatet för historikfilen ska vara detsamma som i denna exempelfil: \\\url{https://github.com/lunduniversity/introprog/blob/master/workspace/w13_bank_proj/bank_log.txt}
\item Allt som berör användargränssnittet (såsom utskrifter till terminalen och inläsning från terminalen) ska ske i \code{BankApplication} eller hjälpklasser till \code{BankApplication}, inte i någon annan av klasserna som specificeras i uppgiften.
\item Alla metoder och attribut ska ha lämplig synlighet, så att interna, förändringsbara delar inte i onödan exponeras.
\item Valen av val/var och immutable/mutable måste vara lämpliga.
\item Programmet ska fungera som i de bifogade exemplen på körning av programmet.
\item Rimlig felhantering ska finnas. Det är alltså önskvärt att programmet inte kraschar då användaren matar in felaktig input, utan istället säger till användaren att input är ogiltig. Du kan dock anta att historikfilen alltid är i rätt format.
\item Programdesignen ska följa de specifikationer som är angivna nedan.
\item Det räcker med att banken ska kunna hantera heltal, men detta ska göras med klassen \code{BigInt} för att tillåta stora belopp. Om din bank hanterar decimaltal ska detta ske med \code{BigDecimal} för att undvika avrundningsfel.
\item Klassen \code{BankAccount} ska generera ett unikt kontonummer för varje konto. Dessa ska återställas om bankens tillstånd återställs till ett tidigare datum, d.v.s. att om en återställning av banken tar bort ett konto så ska dess kontonummer återigen bli tillgängligt.
\item Det enda sättet att förändra tillståndet för en \code{Bank} ska vara (förutom att anropa \code{returnToState}) att anropa \code{doEvent} med en \code{BankEvent} som beskriver tillståndsförändringen. Vid en första anblick kan detta kan verka lite väl bökigt, men när ändringshistoriken ska implementeras kommer det vara till stor hjälp att det finns en \code{BankEvent} som representerar varje ändring.
\item \input{modules/w12-assignment-add-docs-task.tex}
\end{itemize}

\subsection{Design}
Nedan följer beskriving av medlemmar som de olika klasserna bankapplikationen måste innehålla. Dessa påbörjade klasser finns i kursens workspace, tillsammans med de färdigskrivna klasserna \code{HistoryEntry} och \code{Date} samt \code{BankEvent} med tillhörande subtyper: \url{https://github.com/lunduniversity/introprog/tree/master/workspace/w13_bank_proj}

\scalainputlisting[basicstyle=\ttfamily\fontsize{10}{13}\selectfont]{../workspace/w13_bank_proj/src/main/scala/bank/Customer.scala}

\scalainputlisting[basicstyle=\ttfamily\fontsize{10}{13}\selectfont]{../workspace/w13_bank_proj/src/main/scala/bank/BankAccount.scala}

\scalainputlisting[basicstyle=\ttfamily\fontsize{10}{13}\selectfont]{../workspace/w13_bank_proj/src/main/scala/bank/Bank.scala}


\subsection{Tips}

\begin{itemize}
\item Använd ett \code{match}-uttryck för att hantera de olika subtyperna av \code{BankEvent} när du implementerar \code{doEvent}.
\begin{Code}
event match {
  case Deposit(account, amount) => ???
  case Withdraw(account, amount) => ???
  case Transfer(accFrom, accTo, amount) => ???
  case NewAccount(id, name) => ???
  case DeleteAccount(account) => ???
}
\end{Code}

\item För att skriva till fil på ett enkelt sätt kan man t.ex. använda sig av statiska metoder i klassen \code{Files} som finns tillgänglig i \code{java.nio.file}. För att undvika portabilitetsproblem kan man då använda sig av ett bestämt \code{Charset}, t.ex. \code{UTF_8}, som finns tillgänglig i \code{java.nio.charset.StandardCharsets.UTF_8}.

\item För att läsa ifrån en fil kan du använda \code{introprog.IO}. Studera speciellt metoden \code{appensString} och hur ny-rad-tecken hanteras i \code{appendLines}\\\url{https://github.com/lunduniversity/introprog-scalalib/blob/master/src/main/scala/introprog/IO.scala}

\item Var noggrann med att dina tester innehåller fler fall än de som givits i exempel (se \ref{bank:exempel}), vilka kan behövas för mer omfattande testning och avlusning och efterfrågas på redovisningen.
\end{itemize}

\subsection{Obligatoriska uppgifter}

\Task Implementera klassen \code{Customer}. Testa så att den fungerar REPL.

\Task Implementera klassen \code{BankAccount}. Testa så att den fungerar i REPL.

\Task Skapa ett huvudprogram i singelobjektet \code{BankApplication}. Gör så att huvudprogrammet skriver ut menyn korrekt och kan ta input från tangentbordet som motsvarar de menyval som ska finnas. Låt val av menyerna ge ett meddelande som berättar för användaren att att de ännu ej är implementerade.

\Task Implementera klassen \code{Bank}.

\Subtask Implementera menyval 6. När användaren väljer att skapa ett nytt konto ska \code{BankApplication} skapa ett \code{NewAccount}-objekt som den sedan använder som argument i ett anrop till \code{doEvent} i \code{Bank}. Det är i \code{doEvent} (eller en privat funktion som anropas från \code{doEvent}) som kontot faktiskt ska skapas.

\Subtask Implementera menyval 8. Kontrollera att både menyval 6 och 8 fungerar rätt.

\Subtask Implementera menyval 9. Varje gång \code{doEvent} exekveras utan fel ska dess \code{BankEvent}-argument läggas till i historiken tillsammans med det nuvarande datumet.

\Subtask Implementera alla andra menyval, förutom menyval 10. Testa de nya menyvalen noga efterhand som du implementerar dem, i synnerhet så att ändringshistoriken fungerar korrekt. Gör de utökningar du anser behövs.

\Task Implementera säkerhetskopiering av historiken.

\Subtask När något läggs till i historiken ska det också skrivas till en historikfil omedelbart. Banken ska ej behöva avslutas för att utskriften ska hamna på fil, om så vore fallet kan information gå förlorad om banken kraschar. Använd \code{toLogString}-metoden i \code{HistoryEntry} för att få utskrifter i rätt format.

\Subtask När programmet startar ska det läsa in alla händelser från historikfilen och återuppspela dem en efter en. På så sätt kan bankens tillstånd återställas, fastän vi bara har sparat ändringshistoriken och inte själva tillståndet. Använd \code{fromLogString}-metoden i \code{HistoryEntry} när du läser in strängar från filen.

\Task Implementera menyval 10 genom att först nollställa bankens tillstånd och sedan återuppspela allt i historiken som hände före det givna datumet. Resten av historiken bör tas bort permanent, både i minnet och i historikfilen.


\subsection{Frivilliga extrauppgifter}

Gör först klart projektets obligatoriska delar. Därefter kan du, om du vill, utöka ditt program enligt följande.

\Task Implementera ett nytt menyalternativ som skriver ut all kontohistorik för en given person. I historiken ska finnas typ av händelse med tillhörande parametrar, dåvarande saldo vid händelsen, såväl som datumet för händelsen. (Du kan ha nytta av denna funktion när du testar ditt program.)

\Task Skriv en eller flera av klasserna \code{Customer} och \code{BankAccount} i Java istället och använd dig av dessa i din Scala-kod. (Detta är en nyttig uppgift som förberedelse inför efterkommande fördjupningskurs, som har Java som huvudspråk.)

\subsection{Exempel på historikfil}

I workspace-katalogen för denna projektuppgift medföljer en historikfil. Inläsning och utskrift ska ske med dess format. Varje rad representerar en händelse, och formatet för en rad är: \textbf{År  Månad  Dag  Timme  Minut  BankEventTyp  Argument}. De olika sorternas \code{BankEvent} representeras med följande bokstäver: D för \code{Deposit}, W för \code{Withdraw}, T för \code{Transfer}, N för \code{NewAccount} och E för \code{DeleteAccount}.

\subsection{Exempel på körning av programmet}\label{bank:exempel}

Nedan visas möjliga exempel på körning av programmet. Data som matas in av användaren är markerad i fetstil.
Ditt program måste inte se identiskt ut, men den övergripande strukturen såväl som resultat av körningen ska vara densamma.
När det första exemplet börjar förutsätts det att banken inte har några konton.

Listan över val, som är markerad i kursiv stil i det första exemplet, är inte utskriven i senare exempel för att spara plats på pappret. Ditt program ska alltid skriva ut listan över val före användaren ska mata in ett val.

% This environment uses minipage to prevent column breaks from occurring in the middle of an example
\newenvironment{exampleblock}
	{\begin{minipage}{\columnwidth}
	 - - - - - - - - - - - - - - - - - - - - - - - - - - -\\}
	{\end{minipage}}

\begin{multicols}{2}
\noindent
\begin{exampleblock}
\textit{
1.   Hitta konton för en given kund\\
2.   Sök efter kunder på (del av) namn\\
3.   Sätt in pengar\\
4.   Ta ut pengar\\
5.   Överför pengar mellan konton\\
6.   Skapa nytt konto\\
7.   Radera existerande konto\\
8.   Skriv ut alla konton i banken\\
9.   Skriv ut ändringshistoriken\\
10.  Återställ banken till ett tidigare datum\\
11.  Avsluta\\
}
Val: \textbf{6}\\
Namn: \textbf{Adam Asson}\\
Id: \textbf{6707071234}\\
Nytt konto skapat med kontonummer: 1000\\
10:03 14/5-2016\\
\end{exampleblock}
\begin{exampleblock}
Val: \textbf{1}\\
Id: \textbf{6707071234}\\
Konto 1000 (Adam Asson, id 6707071234) 0 kr\\
10:04 14/5-2016\\
\end{exampleblock}
\begin{exampleblock}
Val: \textbf{6}\\
Namn: \textbf{Berit Besson}\\
Id: \textbf{8505255678}\\
Nytt konto skapat med kontonummer: 1001\\
10:12 14/5-2016\\
\end{exampleblock}
\begin{exampleblock}
Val: \textbf{8}\\
Konto 1000 (Adam Asson, id 6707071234) 0 kr\\
Konto 1001 (Berit Besson, id 8505255678) 0 kr\\
10:13 14/5-2016\\
\end{exampleblock}
\begin{exampleblock}
Val: \textbf{2}\\
Namn: \textbf{adam}\\
Adam Asson, id 6707071234\\
10:15 14/5-2016\\
\end{exampleblock}
\begin{exampleblock}
Val: \textbf{6}\\
Namn: \textbf{Berit Besson}\\
Id: \textbf{8505255678}\\
Nytt konto skapat med kontonummer: 1002\\
13:56 14/5-2016\\
\end{exampleblock}
\begin{exampleblock}
Val: \textbf{2}\\
Namn: \textbf{erit}\\
Berit Besson, id 8505255678\\
14:01 14/5-2016\\
\end{exampleblock}
\begin{exampleblock}
Val: \textbf{3}\\
Kontonummer: \textbf{1000}\\
Summa: \textbf{5000}\\
Transaktionen lyckades.\\
14:36 14/5-2016\\
\end{exampleblock}
\begin{exampleblock}
Val: \textbf{5}\\
Kontonummer att överföra ifrån: \textbf{1000}\\
Kontonummer att överföra till: \textbf{1001}\\
Summa: \textbf{1000}\\
Transaktionen lyckades.\\
14:37 14/5-2016\\
\end{exampleblock}
\begin{exampleblock}
Val: \textbf{8}\\
Konto 1000 (Adam Asson, id 6707071234) 4000 kr\\
Konto 1001 (Berit Besson, id 8505255678) 1000 kr\\
Konto 1002 (Berit Besson, id 8505255678) 0 kr\\
14:52 14/5-2016\\
\end{exampleblock}
\begin{exampleblock}
Val: \textbf{7}\\
Ange konto att radera: \textbf{1002}\\
Transaktionen lyckades.\\
14:54 14/5-2016\\
\end{exampleblock}
\begin{exampleblock}
Val: \textbf{8}\\
Konto 1000 (Adam Asson, id 6707071234) 4000 kr\\
Konto 1001 (Berit Besson, id 8505255678) 1000 kr\\
14:55 14/5-2016\\
\end{exampleblock}
\begin{exampleblock}
Val: \textbf{9}\\
10:03 14/5-2016: Skapade ett konto tillhörandes Adam Asson, id 6707071234\\
10:12 14/5-2016: Skapade ett konto tillhörandes Berit Besson, id 8505255678\\
13:56 14/5-2016: Skapade ett konto tillhörandes Berit Besson, id 8505255678\\
14:36 14/5-2016: Satte in 5000 kr i konto 1000\\
14:37 14/5-2016: Överförde 1000 kr från konto 1000 till konto 1001\\
14:54 14/5-2016: Raderade konto 1002\\
14:58 14/5-2016\\
\end{exampleblock}
\begin{exampleblock}
Val: \textbf{10}\\
Vilket datum vill du återställa banken till?\\
År: \textbf{2016}\\
Månad: \textbf{5}\\
Datum (dag): \textbf{14}\\
Timme: \textbf{10}\\
Minut: \textbf{5}\\
Banken återställd.\\
15:00 14/5-2016\\
\end{exampleblock}
\begin{exampleblock}
Val: \textbf{9}\\
10:03 14/5-2016: Skapade ett konto tillhörandes Adam Asson, id 6707071234\\
15:00 14/5-2016\\
\end{exampleblock}
\begin{exampleblock}
Val: \textbf{8}\\
Konto 1000 (Adam Asson, id 6707071234) 0 kr\\
15:01 14/5-2016\\
\end{exampleblock}
\begin{exampleblock}
Val: \textbf{3}\\
Kontonummer: \textbf{1001}\\
Summa: \textbf{5000}\\
Transaktionen misslyckades. Inget sådant konto hittades.\\
15:06 14/5-2016\\
\end{exampleblock}
\begin{exampleblock}
Val: \textbf{4}\\
Kontonummer: \textbf{1000}\\
Summa: \textbf{1000}\\
Transaktionen misslyckades. Otillräckligt saldo.\\
15:23 14/5-2016\\
\end{exampleblock}

\end{multicols}

%\input{modules/w12-assignment-tabular.tex}
%!TEX encoding = UTF-8 Unicode

%!TEX root = ../compendium2.tex

\Assignment{music}

\begin{Preparations}
\item Testa så att datorn du ska använda på redovisningen kan spela upp ljud med \code{javax.sound.midi} genom att köra igång \code{Main} i den givna koden.\footnote{I skrivande stund fungerar inte ljud under WSL (Windows Subsystem for Linux). Sök gärna på nätet efter ev. lösningar på problem med ljud i WSL.}
\item Det är bra om du kan ta med hörlurar till datorsalen så att du inte stör andra.
\item Hämta given kod via \href{https://github.com/lunduniversity/introprog/tree/master/workspace/}{kursen github-plats} eller via kurshemsidan under \href{httphttps://lunduniversity.github.io/pgk/#l%C3%A4nkar}{Länkar}.
\end{Preparations}

\subsection{Bakgrund}
När man skriver program skapar man ofta modeller av en viss verklig \emph{domän}, som kan vara t.ex. försäkringskassans regelverk eller en fysiksimulering i ett datorspel. För att kunna skapa sådana modeller behöver man ofta skaffa sig  \emph{domänkunskap} genom att noga sätta sig in i vad olika koncept i domänen innebär och hur de är relaterade. Med denna kunskap kan du skapa kod som modellerar domänen, utifrån noga valda förenklingar av den komplexa verkligheten. Förmåga att kunna skapa domänmodeller utgör en viktig grund för konsten att utveckla bra programvarusystem, och du kommer lära dig mer om detta i kommande kurser.

I denna laboration ska du skapa ett program baserat på en förenklad modell av domänen \emph{musik}. Du får färdig kod som modellerar hur toner är uppbyggda, samt hur olika stränginstrument fungerar.
Med denna domänmodell ska du skapa ditt eget musikprogram som använder \emph{ackord} som är uppbyggda av flera toner som spelas tillsammans.

\subsection{Domänmodell}


\subsubsection{Tonhöjd}

En \textbf{ton} \Eng{note} som spelas på ett instrument, t.ex. ett piano eller en gitarr, har en \textbf{tonhöjd} \Eng{pitch} som är relaterad till den specifika  grundfrekvens som tonens ljud har. I vår modell av musikdomänen tillordnar vi olika distinkta tonhöjder ett unikt heltal. En tonhöjd kan då beskrivas av en \code{case class Pitch(nbr: Int)} där vi använder \code{nbr} i intervallet \code{(0 to 127)}. Heltalet \code{60} motsvarar en viss ton, som även har namnet  \code{"C5"}, och som ligger ungefär i mitten av tangentbordet på ett piano.

Inom (västerländsk) musik utgår man från 12 olika \emph{tonklasser} \Eng{pitch classes}.
Dessa tolv tonklasser är ordnade i en sekvens av så kallade \emph{halva tonsteg} och har följande \textbf{tonklassnamn}:
\begin{Code}
  val pitchClassNames: Vector[String] =
    Vector("C","C#","D","D#","E","F","F#","G","G#","A","A#","B")
\end{Code}
Efter tonklassen med namnet \code{B} återkommer tonklassen med namnet \code{C}.
Symbolen \code{#} representerar en höjning ett halvt tonsteg. Tonklassen \code{C#} uttalas \emph{siss} på svenska, och \emph{see sharp} på engelska.\footnote{Man använder även b-förtecknet $\flat$, som uttalas \emph{flat} på engelska, för sänkning av en ton ett halvt tonsteg, men för enkelhetens skull bortser vi i vår modell från detta sätt att namnge toner.}
 Notera att det är ett halvt tonsteg mellan \code{E} och \code{F}, samt mellan \code{B} och \code{C} (det finns därför varken \code{E#} eller \code{B#} i listan med tonklassnamn.\footnote{Varför det är på detta viset kan du läsa mer om på t.ex. Wikipedia, men du kan också nöja dig med att det helt enkelt är så på grund av historiska skäl.})

På ett piano motsvaras de vita tangenterna av tonklassnamen \code{C D E F G A B} och de svarta tangenterna motsvaras av tonklassnamnen \code{C# D# F# G# A#}.

En s.k. \textbf{tonklass} är ett positivt heltal i intervallet \code{0 until 12} som motsvaras av index för tonklassnamnet i \code{pitchClassNames}. En tonhöjd  \code{Pitch(nbr)} tillhör tonklassen \code{nbr % 12}.

Med hjälp av heltalsdivision med 12 får man fram tonhöjdens så kallade \textbf{oktav}, alltså \code{nbr / 12}. Ett piano har normalt toner som spänner över 7 eller 8 oktaver.
En tonhöjd \code{Pitch(nbr)} kan även namnges med en kombination av tonklassnamnet och tonens oktav, t.ex. \code{"C5"}.

Med denna domänbeskrivning kan vi skapa en mer detaljerad modell av konceptet tonhöjd med hjälp av en case-klass och tillhörande kompanjonsobjekt:

\scalainputlisting[basicstyle=\ttfamily\fontsize{10}{13}\selectfont]{../workspace/w13_music_proj/src/main/scala/music/Pitch.scala}

\noindent Kompanjonsobjektet har två fabriksmetoder som kan skapa \code{Pitch}-objekt från en strängrepresentation av en tonhöjd.

\begin{itemize}[noitemsep]
  \item Metoden \code{fromString} omvandlar en sträng till en \code{Option[Pitch]}.

  \item Metoden \code {apply} kastar ett undantag om det inte går att omvandla en sträng till ett \code{Pitch}-objekt.
\end{itemize}

\Task\label{music:exceptions}\Pen Vilka två uttryck i \code{Try}-blocket kan ge undantagen \code{NumberFormatException} respektive \code{NoSuchElementException}? Undersök liknande uttryck i REPL som ger dessa undantag. Hur kan fabriksmetoden \code{fromString} skrivas om så att den använder \code{toIntOption} i stället för \code{toInt} på strängen och \code{get} i stället för \code{apply} på nyckelvärde-tabellen och utan att använda \code{scala.util.Try}? Vad finns det för nackdelar med att gå omvägen via \code{scala.util.Try} i stället för metoder som direkt ger \code{Option}?


\Task Undersök klassen \code{Pitch} i REPL.

\begin{REPL}
> sbt
sbt> console
Welcome to Scala 2.13.3 (OpenJDK 64-Bit Server VM, Java 11.0.8).
Type in expressions for evaluation. Or try :help.

scala> import music._
import music._

scala> Pitch("C#5").nbr
res0: Int = 61

scala> Pitch("C") + 1
res1: music.Pitch = Pitch("C#5")
\end{REPL}


\Subtask\Pen Ge ett exempel på argument till \code{Pitch.apply} som gör att undantag kastas.

\Subtask\Pen Ge ett exempel på argument till \code{Pitch.fromString} som ger \code{None}.

\Subtask\Pen Ge ett exempel på argument till \code{Pitch.+} som gör att undantag kastas.


\Task Ändra i implementationen av \code{fromString} så att du i stället för \code{.toOption} gör en mönstermatchning med \code{match} på \code{Try}-resultaten \code{Success} och \code{Failure} i varsin case-klausul på formen \code{case Success(x) => } och \code{case Failure(e: ???) => ???} och returnera lämpligt värde. Ta endast hand om de två förväntade undantagstyperna som du identifierade i uppgift \ref{music:exceptions}. Gör så att alla övriga eventuella undantag kastas genom denna klausul: \code{case Failure(e) => throw e}

\Subtask Testa så att din lösning fungerar i både normalfall och vid felaktigt tonhöjdsnamn.

\Subtask\Pen Undersök vad som händer om du kommenterar bort olika case-klausuler. När ger kompilatorn varning? Varför?

\Subtask\Pen Finns det någon fördel resp. nackdel med att bara fånga vissa undantag?

\subsubsection{Ackord}

Ett ackord består av flera toner som spelas tillsammans. Man kan spela ett ackord på ett stränginstrument genom att slå an en mängd toner samtidigt eller en sekvens av toner i snabb följd. Man väljer att kalla en av tonerna i ackordet (oftast den lägsta/första tonen) för \textbf{grundton} \Eng{root}.

Ett \textbf{intervall} är en tons relativa tonhöjdsavstånd från grundtonen. Ackord har olika namn beroende på vilka intervall som ingår i ackordet. Det finns väldigt många olika ackordnamn, men här begränsar vi oss för enkelhetens skull till fyra olika typer av ackord: \footnote{Om du vill veta mer om ackordnamn läs här: \url{https://en.wikipedia.org/wiki/Chord_(music)}}
\begin{itemize}
  \item dur-ackord, betecknas t.ex. \code{"C"},
  \item moll-ackord, betecknas t.ex. \code{"Cm"}
  \item sju-ackord, betecknas t.ex. \code{"C7"}
  \item maj-sju-ackord som betecknas t.ex. \code{"Cmaj7"}.
\end{itemize}

I case-klassen \code{Chord} nedan finns en metod \code{name} som definerar vilka intervall som ingår i de olika ackordtyperna ovan, utom maj-sju-ackord. Den krångliga modulo-12-omräkningen innan matchningen gör så att intervall i olika oktaver behandlas lika, även för negativa intervall.

\scalainputlisting[basicstyle=\ttfamily\fontsize{10}{13}\selectfont]{../workspace/w13_music_proj/src/main/scala/music/Chord.scala}

\Task

\Subtask Maj-sju-ackord har samma intervall som sju-ackord, förutom att det fjärde intervallet ska vara \code{11} halva tonsteg från grundtonen i stället för \code{10}. Lägg till en case-klausul i \code{Chord.name} så att maj-sju-ackord ges namn som slutar med ändelsen \code{"maj7"}.

\Subtask Testa din kod och kontrollera så att ackordet \code{Chord("D4","F#4","A4","C#5")} får namnet \code{"Dmaj7"}

\begin{REPL}
scala> Chord("D4","F#4","A4","C#5").name()
res2: String = "Dmaj7"
\end{REPL}

\Subtask Vilka fyra toner har ett \code{Cmaj7}-ackord med grundtonen \code{"C5"}?

\subsubsection{Stränginstrument}

Ett stränginstrument, t.ex. ett piano eller en gitarr, kännetecknas av att det kan spela ackord genom att flera strängar kan sättas i svängning så att många toner spelas tillsammans. I vår modell fångar vi denna egenskap med en trait \code{StringInstrument} som har en metod \code{toChordOpt} som ger något ackord om minst en sträng spelas.

Gitarr och ukulele är exempel på stränginstrument som har en greppbräda \Eng{fret board}. Man spelar på ett stränginstrument med greppbräda \Eng{fretted instrument} genom att trycka strängar mot greppbrädan med en hand, samtidigt som man knäpper på strängarna med den andra handen. Olika instanser av dessa  instrument kan skilja sig åt vad gäller antalet strängar och hur dessa strängar är stämda. En normal gitarr har 6 strängar, medan en normal ukulele bara har 4 strängar. Dessa egenskaper modelleras i koden nedan.

Varje sträng har en stämskruv med vilken kan man ändra strängens spänning,  strängens s.k. \textbf{stämning} \Eng{tuning}.  Om man knäpper på alla lösa strängarna på en gitarr med standardstämning spelas tonerna E3, A3, D4, G4, B4, E5, räknat från den tjockaste till den tunnaste strängen.

\scalainputlisting[basicstyle=\ttfamily\fontsize{10}{12.9}\selectfont]{../workspace/w13_music_proj/src/main/scala/music/instruments.scala}

Om man trycker på greppbrädans olika positioner får man olika toner, beroende på vilken position man trycker på. Positionerna på greppbrädan räknas från ett och uppåt. Position \code{0} motsvarar \textbf{lös sträng}, alltså att man slår an en sträng utan att trycka på greppbrädan över denna sträng. En negativ position, tex. \code{-1}, anger att en sträng inte spelas alls; många gitarrackord spelas genom att bara en delmängd av strängarna slås an.
Ett exempel på ett gitarrackord  visas i figur \ref{music:fig:guitar-chord}.

\Task\Pen Studera modellen av stränginstrument ovan och använd REPL för att svara på dessa frågor:

\Subtask Vad är namnet på detta pianoackord om vi väljer att lägsta tonen i ackordet är grundton: \code{Piano(Set(60, 64, 67, 70))}

\Subtask Vad heter tonerna som ingår i ackordet \code{Guitar(3,3,2,0,1,0)}.

\Subtask Vad heter detta ackord om vi väljer ett A som grundton: \code{ Ukulele(0,2,1,2)}


\begin{figure}
  \centering
  \includegraphics{../img/chords/guitar-C-major-chord.jpg}
  \caption{Ett C-dur-ackord på en gitarr motsvarande \code{Guitar(3,3,2,0,1,0)}.}
  \label{music:fig:guitar-chord}
\end{figure}


\subsubsection{Elektroniska instrument}

Ett elektroniskt instrument syntetiserar ljud med hjälp av analog och/eller digital elektronik, och kallas därför \textbf{synthesizer}, ofta förkortat \emph{synt} \Eng{synth}.

De flesta moderna PC-operativsystem inkluderar mjukvaruimplementerade syntar som följer den så kallade MIDI-standarden. Java-paketet \code{javax.sound.midi} innehåller klasser som kan få en sådan MIDI-synt att spela musik.

MIDI-standarden baseras på en modell av ett pianotangentbord där olika toner kan vara ''på'' eller ''av'' beroende på om en tangent är nedtryckt eller ej. Dessa toners höjd är modellerade på samma sätt som i vår klass \code{Pitch}, där alltså tonhöjden \code{60} motsvarar tonen \code{"C5"}, etc. En tangent kan tryckas ner olika hårt, vilket representeras av ett heltalsvärde i \code{Range(0,128)} kallat \code{velocity}. Ett högt värde ger en stark ton, medan ett litet värde motsvarar en svag (tyst) ton.

En synt som följer MIDI-standarden kan spela upp ljud via 16 olika så kallade \textbf{kanaler} \Eng{channel}, numrerade \code{(0 until 16)},  där varje kanal kan ställas in så att den spelar ett ljud som t.ex. liknar ett visst verkligt instrument, så som piano eller gitarr.

I kursens workspace i paketet \code{music} finns en \code{Synth}-modul som förenklar användningen av Java-paketet \code{javax.sound.midi}. I modulen \code{Synth} finns metoden \code{playBlocking} som kan spela flera toner under en viss tid med hjälp av synten på ditt ljudkort. Exekveringen av ditt program  blockeras tills tonerna spelats klart, därav \emph{''blocking''} i namnet.

Metoden \code{playBlocking} har följande parametrar, default-argument och returtyp:
\footnote{Om du är nyfiken kan du studera implementationen av \code{Synth}-modulen här:
\\\url{https://github.com/lunduniversity/introprog/tree/master/workspace/w13_music_proj}
 Koden blir lättare att förstå om du samtidigt läser api-dokumentationen av paketet \code{javax.sound.midi} och även lära dig mer om MIDI-standarden med hjälp av t.ex. wikipedia.}

\begin{Code}
def playBlocking(
  noteNumbers: Seq[Int] = Seq(60), // en sekvens av tonhöjder
  velocity: Int         = 60,      // hur hårt anslag i Range(0, 128)
  duration: Long        = 300,     // hur länge i millisekunder
  spread:   Long        = 50,      // millisekunder mellan tonerna
  after:    Long        = 0,       // millisekunder innan första tonen
  channel:  Int         = 0        // MIDI-kanal som spelar tonerna
): Unit
\end{Code}


\Task Anropa \code{playBlocking()} i REPL och undersök om din dator kan spela tonen \code{"C5"}. Använd gärna lurar så att du inte stör dina labbkamrater. Prova vad som händer när du ger olika argument till \code{playBlocking}.

\Task Gör klart modulen \code{ChordPlayer} enligt nedan så att metoden \code{play} kan spela ett ackord. Case-klassen \code{Strike} representerar ett ackordanslag.

\scalainputlisting[basicstyle=\ttfamily\fontsize{10}{13}\selectfont]{../workspace/w13_music_proj/src/main/scala/music/ChordPlayer.scala}

\Task Implementera ett singelobjekt med namnet \code{Test} med en \code{main}-metod som med hjälp av din \code{play}-metod från föregående uppgift spelar några olika ackord.

% \Task\Checkpoint Inför redovisningen: förbered en förklaring av koden du skrivit, med fokus på hur mönstermatchningen och undantagshanteringen fungerar.


%\clearpage

% \subsection{Frivilliga extrauppgifter}

\Task Gör en terminalapp som kan spela ackord. I kursens workspace i \code{w13_music_proj} finns en påbörjad terminalapp som du kan bygga vidare på. Den har redan en \code{Main.main}-metod som startar en loop där användaren kan ge kommando \Eng{Command Line Interface, CLI}. Kommandot \code{?} ger hjälp och kommandot \code{:q} avslutar.

\begin{REPL}
*** Welcome to music!
music> ?
?         print help
:q        quit this app
!         play chord TODO
music> !
play chord TODO
music> :q
Goodbye music!
\end{REPL}

Det finns, som syns ovan, också ett påbörjat kommando \code{!} som är tänkt att spela ett ackord, men som än så länge bara skriver ut ett TODO-meddelande. Gör så att användaren med \code{!} kan spela ackord från olika instrument enligt nedan:

\begin{REPL}
music> ! p 60 64 67
Play Piano(Set(60, 64, 67)) Chord(C5,E5,G5)
music> ! g 0 2 2 0 0 0
Play Guitar((0,2,2,0,0,0)) Chord(E3,B3,E4,G4,B4,E5)
\end{REPL}

% \noindent\emph{Tips och förslag:} Du kan i stället för \code{scala.io.StdIn.readLine} använda \code{jline} och då får du kommandohistorik med pil upp samt Ctrl+A, Ctrl+E etc. helt automatiskt. Gör helt enkelt så här i \code{Main} i stället för vanliga \code{readLine}:
% \begin{CodeSmall}
%   val console = new jline.console.ConsoleReader // skapa kommandoläsare
%   console.setExpandEvents(false) // stäng av hantering av specialtecken
%   def readLine(): String = console.readLine("music> ")
% \end{CodeSmall}
% Du behöver då lägga till jar-filen\footnote{\url{https://repo1.maven.org/maven2/jline/jline/2.14.6/jline-2.14.6.jar}} med \code{jline} till ditt bygge. Om du använder sbt kan du göra det enkelt med denna rad i filen \code{build.sbt}:
% \begin{CodeSmall}
% libraryDependencies += "jline" % "jline" % "2.14.6"
% \end{CodeSmall}
% Lägg till nedan rader i din \code{build.sbt} så att ditt program körs i en separat JVM, annars blir det konstiga initialiseringsfel av MIDI-systemet om du kör med \code{sbt run}.

% \begin{CodeSmall}
% fork                := true // https://stackoverflow.com/questions/18676712
% connectInput        := true // http://www.scala-sbt.org/1.x/docs/Forking.html
% outputStrategy      := Some(StdoutOutput)
% \end{CodeSmall}


\Task Skapa ett kommando som låter användare definierar egna namn på kommandon som sedan enkelt kan köras med hjälp av det definierade namnet. Vid definition med tidigare existerande namn så ska den gamla definitionen ersättas
\begin{REPL}
music> def Em = ! g 0 2 2 0 0 0
defined Em = ! g 0 2 2 0 0 0
music> Em
Play Guitar((0,2,2,0,0,0)) Chord(E3,B3,E4,G4,B4,E5)
\end{REPL}
Det ska fungera att göra nästlade def-kommando, alltså att kroppen för en def innehåller namnet på en annan def. Testa att definiera en hel låt som i sin tur består av definierade ackord.

\Task \input{modules/w12-assignment-add-docs-task.tex}


\subsection{Valfria uppgifter}

\Task Gör så att definitioner sparas mellan körningar.

\Task Implementera fler valfria kommandon. Du kan t.ex. 
\begin{itemize} 
\item skapa kommando som ritar en grepptabell för gitarrackord eller fingersättning på piano med \code{introprog.PixelWindow}.
\item skapa kommando för att göra drumbeats, t.ex. \code{drum mygrove x x o x} för 2 hihat + basatrumma + 1 hihat och \code{start mygrove} och \code{stop mygrove} för uppspelning av beat i bakgrunden med \code{playConcurrently}. Se i \code{Synth.scala} för vilka instrument som ger trumljud med ledning av deras namn \\\code{music.Synth.instruments.map(_.getName)}, t.ex \code{Standard Kit} eller \code{Synth Drum}. Använd kanal 10 för att spela upp trumljuden.
\end{itemize}

\Task Använd öppen-källkodsprokjektet \code{jline} i stället för \code{scala.io.StdIn.readLine} för att automatiskt få pil-upp-historik, Ctrl+A Ctrl+K, TAB-completion, etc. Se exempel på användning av \code{jline} här: \url{https://github.com/bjornregnell/termut}


% ...
%
% \vspace{7em}{\TODO OLD TEXT FROM HERE:}
%
%
% \Task ChordDraw
%
% \Subtask Rita upp en greppbräda liknande bilden nedan (kryssen läggs till i kommande uppgifter). Antalet strängar ska variera beroende på instrument.
%
% \includegraphics[width=0.5\textwidth]{../img/chords/ChordDraw}
%
% \Subtask Skapa en hjälpmetod \code{cross} som tar in två heltal $x$ och $y$. Metoden ska rita upp ett kryss som är 20x20 pixlar och har sitt centrum i den angivna koordinaten.
%
% \Subtask Rita ut ett kryss där en sträng trycks ner. \textbf{Tänk på att -1 och 0 anger att en sträng inte trycks ner}.
%
% \Subtask Implementera metoden \code{play} som börjar med att vänta på ett event från \code{SimpleWindow}, sedan kollar om eventet är av typen \code{SimpleWindow.MOUSE_EVENT}. Sedan ska man kolla om användaren tryckte på någon sträng (ett intervall på -10 till +10 i förhållande till strängens x-koordinat kan anses vara på strängen). Om användaren tryckt på en sträng ska denna spelas med hjälp av \code{SimpleNotePlayer}. Metoden \code{play} ska köras tills användaren kryssar ner fönstret, vilket motsvarar \code{SimpleWindow.CLOSE_EVENT}.
%
% \Subtask Lägg till menyvalet \code{draw} i \code{textui}. Använd \code{match} för att ta hand om felfallen att inget argument eller fler än ett argument angivits. Argumentet motsvarar ackordets plats i den filtrerade litan. Använd \code{Try} och \code{match} för att ta hand om felet att användaren anger något annat än en siffra. Använd \code{ChordDraw} för att rita upp ackord. \textbf{Kom ihåg att lägga till kommandot i listan med kommandon i \code{doCommand}}

%!TEX encoding = UTF-8 Unicode
%!TEX root = ../compendium2.tex

\Assignment{photo}

\subsection{Bakgrund}
Detta projekt innebär att du ska implementera en egen bildbehandlingsapplikation, en mycket förenklad variant av \emph{Phtotoshop} eller \emph{Gimp}. 

En digital bild består av ett rutnät, en s.k. matris \Eng{matrix}, av pixlar, var och en med en viss färg. Om man har många små pixlar bredvid varandra i ett rutnät, så flyter de samman för ögat och betraktaren upplever en bild.

Bilder kan manipuleras genom applicering av olika s.k. \emph{filter}, som förändrar bildens pixlar på ett intressanta sätt. Du ska, utifrån given matematisk teori, implementera olika filter med hjälp av speciella matrisoperationer.


Det finns olika system för hur man färgsätter pixlar. T.ex. så används CMYK-systemet (cyan, magenta, gul, svart) vid blandning av färg som ska tryckas på papper eller annat material. På en dator, däremot, används vanligtvis RGB-systemet, som har de tre grundfärgerna röd, grön och blå. Mättnaden av varje grundfärg anges av ett heltal som vi i fortsättningen förutsätter ligger i intervallet [0, 255]. 0 anger ''ingen färg'' och 255 anger ''maximal färg''. Man kan därmed representera 256 × 256 × 256 = 16 777 216 olika färgnyanser. Man kan också representera gråskalor; det gör man med färger som har samma värde på alla tre 
grundfärgerna: (0, 0, 0) är helt svart, (255, 255, 255) är helt vitt. 

I detta projekt kommer vi skapa matriser av heltal för att beräkna intressanta egenskaper hos en bild, till exempel intensiteten för varje pixel. 
För att spara plats vid bearbetning av stora bilder så använder vi, heltalsmatriser med typen \code{Short}, som använder 16 bitar i minnet, i ståället för \code{Int}, som använder 32 bitar i minnet. 

\subsection{Förberedelser}

I detta projekt har du nytta av följande delar av \href{https://github.com/lunduniversity/introprog-scalalib}{\texttt{introprog-scalalib}} och \code{java.awt}:

\begin{itemize}
\item \code{introprog.Image} för bildhantering.
\item \code{introprog.PixelWindow} och \code{introprog.Dialog} för användarinteraktion.
\item \code{introprog.IO} för filhantering.
\item \code{java.awt.Color} för hantering av pixelfärger.
\end{itemize}
Läs noga dokumentationen för klasserna i introprog här och gör egna experiment i REPL så du förstår hur de kan användas: 
\url{https://fileadmin.cs.lth.se/pgk/api}\\
Läs om klassen \code{java.awt.Color} här:\\\url{https://docs.oracle.com/en/java/javase/11/docs/api/java.desktop/java/awt/Color.html}
Hämta och studera noga den kod som är given för detta projekt här:\\
\url{https://github.com/lunduniversity/introprog/tree/master/workspace/}

\subsection{Matris med värden av typen \code{Short}}

\Task \textbf{\code{Matrix}}.
I den givna kodfilen \code{Matrix.scala} finns hjälp-funktioner för att skapa och uppdatera matriser med värden av typen \code{Short}, för att spara minne vid stora bilder.  

Gör klart saknade implementationer och testa noga i REPL så att allt fungerar som det ska innan du går vidare. \emph{Tips:} Du har nytta av \code{Array.tabulate}. 
\begin{REPLsmall}
scala> import photo.*

scala> val m = Matrix(3,3)(1,2,3,4,5,6,7,8,9) // en 3x3-matiris med Short-värden
val m: photo.Matrix = Array(Array(1, 4, 7), Array(2, 5, 8), Array(3, 6, 9))

scala> m(0,1)
val res0: Short = 4

scala> m(1,0) = 42

scala> m
val res1: photo.Matrix = Array(Array(1, 4, 7), Array(42, 5, 8), Array(3, 6, 9))

scala> m.row(0)
val res2: Array[Short] = Array(1, 42, 3)
\end{REPLsmall}

\subsection{Användargränssnitt}

När appen startar så visas ett fönster enligt fig. \ref{photo:fig:main-window}, implementerat av givna koden i \code{Main.scala}. Med hjälp av givna \code{Button.scala} skapas en kolumn med knappar som är klickbara. Studera koden i \code{Main.scala} och \code{Button.scala} noga så att du förstår vad som händer. Ännu öppnas inget ImageEditor-fönster, men det ingår i näst uppgift.

\begin{figure}[H]
\centering
\includegraphics[width=0.4\textwidth]{../img/w12-assignment-photo/photo-main.png}
\caption{Photo-applikationens startfönster.}
\label{photo:fig:main-window}
\end{figure}

\Task \textbf{\code{ImageEditor}}. Följande krav ska implementeras: 

\begin{itemize}
\item Du ska skapa en kodfil \code{ImageEditor.scala} som innehåller en klass med samma namn som implementerar ett bildredigeringsfönster med en kolumn med knappar till vänster och en bild inläst från fil till höger, så som visas i fig. \ref{photo:fig:editor-one-filter}. 

\item Vid tryck på Exit-knappen ska en varningsfråga ''Ok to Exit without save?'' ges med \code{introprog.Dialog.isOK} och användaren ska kunna ångra avslut. Om avslut ändå väljs så ska detta ske med \code{System.exit(0)} så att alla ev. aktiva fönster och tillhörande trådar avbryts direkt.

\item Vid tryck på Open-knappen ska en fil väljas med hjälp av \code{introprog.Dialog.file}. Om det i aktuell katalog finns en underkatalog vid namn \code{images} så ska filbläddringen börja där, annars i aktuell katalog. 

\item Efter OK på filöppningen ska en bild öppnas i ett bildredigeringsfönster enligt fig. \ref{photo:fig:editor-one-filter} med knapparna Save, Undo, Close, plus en knapp för varje filter, till vänster om bilden. Fönstrets höjd och bredd ska avpassas så att hela bilden och alla knappar får plats.

\begin{figure}
\centering
\includegraphics[width=0.8\textwidth]{../img/w12-assignment-photo/photo-duck.png}
\caption{Bildredigeringsfönstret, innan fler filter (utöver identitetsfiltret) implementerats. Varje filter som implementeras ska ha en motsvarande knapp.}
\label{photo:fig:editor-one-filter}
\end{figure}

\item Huvudfönstret och alla bildredigeringsfönster ska fungera parallellt. Detta ska du åstadkomma genom att händelseloopen i \code{ImageEditor}-klassen körs som argument till metoden \code{runInParallell} enligt nedan: 
\begin{CodeSmall}
  def runInParallell(block: => Unit) = 
    new Thread{ override def run(): Unit = block }.start

  def startEventLoop(): Unit = runInParallell:
    // initialisering och händelseloop här
\end{CodeSmall}

\item Det ska gå att göra Undo i flera steg och återställa alla bilder före applicering av filter i tur och ordning. \emph{Tips:} Inför i attributet \code{var history: Vector[Image]} som från början innehåller den ursprungliga bilden.

\item Fönstrets titel ska innehålla namnet \code{ImageEditor} och de två sista delarna av den sökväg \Eng{path} som öppnats, enligt exempel i fig. \ref{photo:fig:editor-one-filter}, eftersom en fullständig sökväg t.ex. \texttt{/home/userxyz/workspace/photo/images/duck.jpg} riskerar att inte få plats i fönstrets titelbalk.

\item Vid Save ska fråga om filnamn ställas med \code{introprog.Dialog.file} och kontroller göras om filen redan finns eller ej, och om den finns ska en fråga med \code{introprog.Dialog.isOK} ställas om den ska skrivas över eller ej.

\item Alla implementerade filter ska ha en knapp som applicerar filtret och sparar resultatet i historiken så att filtret kan ångras med Undo. Om filtret har argument så ska en informativ dialog öppnas där användaren kan ange argument via \code{introprog.Dialog.input}. Filterknapparna ska vara sorterade i bokstavsordning efter filtrets namn, se fig. \ref{photo:fig:photo-jay-sobel}.
\end{itemize}




\subsection{Filter}

Du ska bygga vidare på givna koden i \code{Filter.scala} som visas nedan. Du ska implementera och testa ett antal olika filter som ändrar bilder på intressanta sätt med hjälp av olika matrisalgoritmer.

\scalainputlisting[basicstyle=\ttfamily\fontsize{9.5}{11}\selectfont]{../workspace/w13_photo_proj/Filter.scala}

\noindent \code{Filter.scala} innehåller en bastyp för alla filter med ett antal medlemmar som alla filter ska implementera, enligt nedan krav. Det finns ett färdigimplementerat filter, \code{Identity}, som kan användas för testsyften; detta filter gör inget annat än kopierar alla bildpunkter till en ny bild och har således ingen editerande effekt.

\begin{itemize}
\item Metoden \code{apply} ska returnera en ny bilden där filtret applicerats, utan att förändra inparameter-bilden. 

\item Ett filter ska kunna ha noll eller flera argument av typen \code{Double} som kan påverka vad som händer när filtret appliceras. Varje sådant argument ska i tur och ordning ha en kort, instruktiv beskrivning i sekvensen \code{argDescription}.

\item Metoden \code{intensity} ska beräkna en s.k. \emph{intensitetsmatris} och behövs vid implementeringen av gråskale-, Gauss- och Sobel-filtren. Hur en intensitetsmatris beräknas beskrivs nedan. 

\item Metoden \code{convolve} ska göra en s.k. faltning (medelvärdesbildning i matriser) och behövs vid implementering av Gauss- och Sobel-filtren. Hur en faltning görs beskrivs nedan.

\item Alla implementerade filter ska finnas i sekvensen \code{byIndex}, som används i tabellen \code{byName}. Dessa behövs för att skapa alla filterknappar och applicera respektive filter.
\end{itemize}

\noindent Du ska implementera och testa alla filter i uppgifterna nedan, ett i taget. Uppgifterna är ordnade i stigande svårighetsgrad.

\begin{figure}
\centering
\includegraphics[width=0.8\textwidth]{../img/w12-assignment-photo/photo-jay-sobel.png}
\caption{Bildredigeringsfönstret med alla filter implementerade. Ett kontruförstärkande så kallat Sobel-filter är applicerat med tröskelvärde 150.}
\label{photo:fig:photo-jay-sobel}
\end{figure}


\Task \textbf{Blåfilter.} Skapa ett filter i ett singelobjekt med namn \code{Blue} som vid applicering ger en blå version av bilden, där varje pixel bara innehåller den blå komponenten. \emph{Tips:} Du har nytta av metoden \code{getBlue} i klassen \code{java.awt.Color}. 

\Task \textbf{Negativ.} Skapa ett filter \code{Invert} som inverterar en bild, dvs skapar en ''negativ'' kopia av bilden. Ljusa färger ska alltså bli mörka och mörka färger ska bli ljusa.
Fundera över vad som kan menas med en inverterad eller negativ kopia. \emph{Tips:} Även de nya RGB-värdena ska vara i heltal i intervallet 0 -- 255. De nya RGB-värdena beräknas \emph{inte} med något divisionsuttryck över de gamla värdena (då skulle de nya värdena bli decimaltal och inte heltal i intervallet 0 -- 255). 

\Task \textbf{Gråskalefilter.} Skapa ett filter \code{GrayScale} som gör om bilden till en gråskalebild. Implementera först \code{intensity}-metoden i \code{trait Filter} genom att bilda medelvärdet av alla tre RGB-komponenterna. Använd sedan intensiteten för varje pixel för att bestämma gråskalenivån. Om intensiteten i en pixel till exempel är 105 så ska den nya gråskale-pixeln var ett \code{Color}-objekt med RGB-värdena (105, 105, 105).

\Task \textbf{Kryptering.} Skapa ett filter \code{XOrCrypto} som krypterar bilden med xor-operatorn ˆ. Denna operator gör binär xor mellan bitarna i ett heltal. Exempelvis ger 8 ˆ 127 värdet 119. Om man gör xor igen med 127, alltså 119 ˆ 127, får man tillbaka värdet 8. Varje pixel krypteras genom att använda xor-operatorn med ursprungsvärdena för rött, grönt och blått tillsammans med slumpmässiga heltalsvärden som genereras ur en ny instans av \code{scala.util.Random}. Tre nya slumptal ska dras för varje pixels RGB-komponent ur samma Random-instans. Låt användaren ge ett argument som du använder som slumptalsfrö vid skapande av \code{Random}-instansen. På så sätt kan du återskapa bilden genom att applicera krypteringsfiltret igen, med samma argument, på den numera krypterade bilden.

Om filtrets \code{argDescriptions}-sekvens är icke-tom så ska \code{ImageEditor} fråga efter varje argument i tur och ordning och visa varje beskrivning i dialogrutan. Användarens indata görs om till ett decimaltal av typen \code{Double} före att argumenten används i metoden \code{apply}. Bestäm själv hur du vill hantera defaultvärden och felhantering om användaren anger en sträng som inte går att göra om till en \code{Double}. \emph{Tips:} Du har nytta av \code{toDoubleOption} och \code{getOrElse}.

\begin{CodeSmall}
  object XOrCrypto extends Filter:
    val name = "Encrypt"
    override val argDescriptions = Seq("Encryption key")

    def apply(im: Image, args: Double*): Image = ???
\end{CodeSmall}

\begin{REPLnonum}
scala> Seq(8, 127, 8 ^ 127).map(_.toBinaryString)
val res0: Seq[String] = List(1000, 1111111, 1110111)

scala> 8 ^ 127
val res1: Int = 119

scala> 119 ^ 127
val res2: Int = 8
\end{REPLnonum}

\begin{figure}
\centering
\includegraphics[width=0.85\textwidth]{../img/w12-assignment-photo/photo-xor.png}
\caption{Krypteringsfilter före applicering, under pågående inmatning av nyckel. }
\label{photo:fig:photo-xor}
\end{figure}
  

\Task \textbf{Gaussfilter.} Ett Gaussfilter gör bilden lite mindre skarp. Gaussfiltrering är ett exempel på så kallad \emph{faltningsfiltrering}. Faltning \Eng{convolution} är en slags lokal medelvärdesbildning. Nya pixlar skapas genom att kombinera varje pixel med dess omgivande pixlar enligt en speciell matrisalgoritm.

För att åstadkomma detta utnyttjar man en så kallad \emph{faltningskärna} K som är en liten kvadratisk heltalsmatris. Man placerar K över varje element i intensitetsmatrisen och multiplicerar varje element i K med motsvarande element i intensitetsmatrisen. Man summerar produkterna och dividerar summan med summan av elementen i K för att få det nya värdet på intensiteten i punkten (alltså ett slags medelvärde). Divisionen görs för att den nya intensiteten ska hamna i rätt intervall (0 -- 255).
Exempel:

\vspace{1em}
\begin{minipage}{5cm}
\begin{displaymath}
\mathit{intensity} = \left(
\begin{array}{ccccc}
5 & 4 & 2 & 8 & \ldots \\
4 & 3 & 4 & 9 & \ldots \\
9 & 8 & 7 & 7 & \ldots \\
8 & 6 & 6 & 5 & \ldots \\
\vdots & \vdots & \vdots & \vdots & \ddots
\end{array}
\right)
\end{displaymath}
\end{minipage}\hspace{2cm}
\begin{minipage}{5cm}
\begin{displaymath}
K = \left(
\begin{array}{ccc}
0 & 1 & 0 \\
1 & 4 & 1 \\
0 & 1 & 0
\end{array}
\right)
\end{displaymath}
\end{minipage}

\vspace{2em}\noindent Här är summan av elementen i $K$ $1+1+4+1+1 = 8$. För att räkna ut det nya värdet på intensiteten i punkten \code{(1, 1)} med det nuvarande värdet är 3, beräknar man följande:

\begin{displaymath}
\mathit{newintensity} = \frac{0 \cdot 5 + 1 \cdot 4 + 0 \cdot 2 + 1 \cdot 4 + 4 \cdot 3 + 1 \cdot 4 + 0 \cdot 9 + 1 \cdot 8 + 0 \cdot 7}{8} = \frac{32}{8} = 4
\end{displaymath}


\noindent Man fortsätter med att flytta K ett steg åt höger och beräknar på motsvarande sätt ett nytt värde för elementet med index \code{(1)(2)} (där det nuvarande värdet är 4 och det nya värdet blir 5). Därefter gör man på samma sätt för alla element utom för ”ramen” dvs elementen i matrisens ytterkanter.

Implementera och testa noga först metoden\\
\code{convolve(p: Matrix, x: Int, y: Int, kernel: Matrix, weight: Int): Short}\\ i \code{trait Filter} som alltså ska ge den normerade produktsumman av \code{kernel} och punkterna i närheten av \code{(x,y)} i matrisen \code{p} normerat med \code{weight}. \code{Tips:} Du har nytta av metoderna \code{round} och \code{toShort}.

\begin{REPLnonum}
scala> import photo.*

scala> val p = Matrix(4,4)(5,4,2,8,4,3,4,9,9,8,7,7,8,6,6,5)
val p: photo.Matrix = Array(Array(5, 4, 9, 8), Array(4, 3, 8, 6), Array(2, 4, 7, 6), Array(8, 9, 7, 5))

scala> val K = Matrix(3,3)(0,1,0,1,4,1,0,1,0)
val K: photo.Matrix = Array(Array(0, 1, 0), Array(1, 4, 1), Array(0, 1, 0))

scala> Filter.convolve(p, 1, 1, K, K.flatten.sum)
val res0: Short = 4
\end{REPLnonum}

Skapa därefter ett filter \code{Gauss} som gör en faltning med hjälp av \code{convolve} för varje färgkomponent separat. Gör på följande sätt:
\begin{enumerate}
	\item Bilda tre \code{short}-matriser och lagra pixlarnas red-, green- och blue-komponenter i matriserna.
	\item Utför faltningen av de tre komponenterna för varje element och uppdatera \code{result} med de uträknade värdena.
	\item Elementen i ramen behandlas inte, men i \code{result} måste också dessa element få värden. Enklast är att flytta över dessa element oförändrade från \code{im} till \code{result}. (Man kan också sätta dem till \code{Color.BLACK}, men då kommer den filtrerade bilden att se något mindre ut.)
\end{enumerate}

Använd \code{kernel} $K$ enligt ovan och låt \code{weight} vara summan av alla element i $K$. 

Det kan vara intressant att prova med andra värden än 4 i mitten av faltningsmatrisen. Med värdet 0 får man en större utjämning eftersom man då inte alls tar hänsyn till den aktuella pixelns värde. Låt användaren mata in argument för mittvärdet, mellan 0 och 50, och beskriv detta i \code{argDescriptions}. \footnote{Det kan ibland vara svårt att se någon skillnad mellan den Gauss-filtrerade bilden och originalbilden. Om man vill ha en riktigt suddig bild så måste man använda en större matris som faltningskärna. Prova gärna detta som extrauppgift. }


\Task  \textbf{Sobelfilter.} Sobelfiltrering är, precis som Gaussfiltrering, en typ av faltningsfiltrering. Med Sobelfiltrering får man dock motsatt effekt i jämförelse med Gaussfiltrering, dvs man förstärker konturer i en bild. I princip deriverar man bilden i x- och y-led och sammanställer resultatet.

\begin{figure}[H]
\includegraphics[width=0.9\textwidth]{../img/w12-assignment-photo/derivatabild2.pdf}
\caption { En funktion (heldragen linje) och dess derivata (streckad linje).}
\label{fig:photo:sobelfilter:derivatabild}
\end{figure}

I figur~\ref{fig:photo:sobelfilter:derivatabild} visas en funktion $f$ (heldragen linje) och funktionens derivata $f'$ (streckad linje). Vi ser att där funktionen gör ett ''hopp'' så får derivatan ett stort värde. Om funktionen representerar intensiteten hos pixlarna längs en linje i x-led eller y-led så motsvarar ''hoppen'' en kontur i bilden. Om man sedan bestämmer sig för att pixlar där derivatans värde överstiger ett visst tröskelvärde ska vara svarta och andra pixlar vita så får man en bild med starka konturer.

Nu är ju intensiteten hos pixlarna inte en kontinuerlig funktion som man kan derivera enligt vanliga matematiska regler. Men man kan approximera derivatan, till exempel med följande formel:

\begin{displaymath}
f'(x) \approx \frac{f(x+h) - f(x-h)}{2h}
\end{displaymath}

Om man låter $h$ gå mot noll så får man definitionen av derivatan.
Efter ytterligare teoretiska utredningar så kan man visa att det går att uttrycka derivering i en matris med hjälp av faltning enligt följande:
% 
% Uttryckt i Scala och matrisen \code{m: Matrix} så får man:

% \begin{Code}
% val derivative = (m(x, y + 1) - m(x, y-1)) / 2
% \end{Code}
\begin{enumerate}
	\item Beräkna intensitetsmatrisen med metoden \code{intensity}.
	\item För varje punkt i intensitetsmatrisen gör två faltningar med dessa kärnor:
$$
SobelX =
\begin{pmatrix}
  -1 & 0 & 1 \\
  -2 & 0 & 2 \\
  -1 & 0 & 1 \\
\end{pmatrix}
~\hspace{3em}~
SobelY =
\begin{pmatrix}
  -1 & -2 & -1 \\
  0 & 0 & 0 \\
  1 & 2 & 1 \\
\end{pmatrix}
$$
	Använd metoden \code{convolve} med vikten 1. Koefficienterna i matrisen $SobelX$ uttrycker derivering i x-led, medan $SobelY$ uttrycker derivering i y-led. För att förklara varför koefficienterna ibland är 1, ibland 2, ibland positiva och ibland negativa, måste man studera den bakomliggande teorin noggrant, men det gör vi inte här.
	\item Om resultaten av faltningen i en punkt betecknas med \code{sx} och \code{sy} så får man en indikator på närvaron av en kontur med \code{math.abs(sx) + math.abs(sy)}. Absolutbelopp behöver man eftersom man har negativa koefficienter i faltningsmatriserna.
	\item  Sätt pixeln till svart om indikatorn är större än tröskelvärdet, till vit annars. Låt tröskelvärdet bestämmas av ett argument som användaren kan ange.
\end{enumerate}
\noindent Skapa ett filter \code{Sobel} som implementerar konturförstärkning med ovan algoritm. Se exempel i fig. \ref{photo:fig:photo-jay-sobel}. Du ska låta användaren ge tröskelvärdet med argumentbeskrivningen \code{"Threshold (0.0 - 255.0)"}.


\Task \input{modules/w12-assignment-add-docs-task.tex}

\subsection{Frivilliga extrauppgifter}

\Task \textbf{Kortkommando}. Gör så att det blir möjligt att applicera filter med hjälp av tangenttryck. Utvidga \code{trait Filter} så att alla filter kan ha kortkommando. Skriv på knappen vad kortkommandot är så att användaren kan upptäcka det.

\Task \textbf{Kontrastfilter.} Om man applicerar kontrastfiltrering på en färgbild så kommer bilden att konverteras till en gråskalebild. (Man kan naturligtvis förbättra kontrasten i en färgbild och få en färgbild som resultat. Då behandlar man de tre färgkanalerna var för sig.) Många bilder lider av alltför låg kontrast. Det beror på att bilden inte utnyttjar hela det tillgängliga området 0–255 för intensiteten. Man får en bild med bättre kontrast om man ''töjer ut'' intervallet enligt följande formel (linjär interpolation):

\begin{Code}
val newIntensity = 255 * (intensity - 45) / (225 - 45)
\end{Code}

Som synes kommer en punkt med intensiteten 45 att få den nya intensiteten 0 och en punkt med intensiteten 225 att få den nya intensiteten 255. Mellanliggande punkter sprids ut jämnt över intervallet \code{[0, 255]}. För punkter med en intensitet mindre än 45 sätter man den nya intensiteten till 0, för punkter med en intensitet större än 225 sätter man den nya intensiteten till 255. Vi kallar intervallet där de flesta pixlarna finns för \code{[lowCut, highCut]}. De punkter som har intensitet mindre än \code{lowCut} sätter man till 0, de som har intensitet större än \code{highCut} sätter man till 255. För de övriga punkterna interpolerar man med formeln ovan (45 ersätts med \code{lowCut}, 225 med \code{highCut}).

Det återstår nu att hitta lämpliga värden på \code{lowCut} och \code{highCut}. Detta är inte något som kan göras enkelt, eftersom värdena beror på intensitetsfördelningen hos bildpunkterna. Man börjar därför med att först beräkna bildens intensitetshistogram, dvs hur många punkter i bilden som har intensiteten 0, hur många som har intensiteten 1, . . . , till och med 255.

I de flesta bildbehandlingsprogram kan man sedan titta på histogrammet och interaktivt bestämma värdena på \code{lowCut} och \code{highCut}. Så ska vi dock inte göra här. I stället bestämmer vi oss för ett procenttal \code{cutOff}, som användaren kan ange som argument från terminalen, och som  beräknar \code{lowCut} så att \code{cutOff} procent av punkterna i bilden har en intensitet som är mindre än \code{lowCut} och \code{highCut} så att \code{cutOff} procent av punkterna har en intensitet som är större än \code{highCut}.

\vspace{1em}

\noindent \textbf{Exempel}: antag att en bild innehåller 100 000 pixlar och att \code{cutOff} är 1.5. Beräkna bildens intensitetshistogram genom registrering av varje intensitet i en heltals-array \\  \code{val histogram = Array.ofDIm[Int](256)}\\och beräkna \code{lowCut} så att \\\code{histogram(0)} + \ldots + \code{histogram(lowCut)} = 0.015 * 100000 \\så nära det går att komma, det blir troligen inte exakt likhet. Beräkna \code{highCut} på liknande sätt.

\vspace{1em}


\noindent Sammanfattning av algoritmen:
\begin{enumerate}
	\item Beräkna intensitetsmatrisen.
	\item Beräkna bildens intensitetshistogram.
	\item Argument från användaren användas som \code{cutOff}.
	\item Beräkna \code{lowCut} och \code{highCut} enligt exempel ovan.
	\item Beräkna den nya intensiteten för varje pixel enligt interpolationsformeln och lagra de nya pixlarna i \code{result}.
\end{enumerate}
Skapa ett filter \code{Contrast} som implementerar algoritmen. I katalogen \emph{images} kan bilden \emph{moon.jpg} vara lämplig att testa, eftersom den har låg kontrast. Anmärkning: om \code{cutOff} sätts = 0 så får man samma resultat av denna filtrering som man får av \code{GrayScale}. Detta kan man se genom att studera interpolationsformeln.

\Task \textbf{Eget filter}. Skapa ett eget valfritt filter. Till exempel så kan du skapa ett filter som tar fem argument, där de två första värdena representerar ett intensitetsintervall och de tre sista värdena representerar röd-, grön- och blå-komponenterna till en pixel
som ska ersättas med denna färg då intensiteten ligger utanför det givna intervallet. 

\Task \textbf{Egna interaktiva verktyg}. Skapa valfria interaktiva redigeringsverktyg med mus- och tangentinput. Börja med ett markeringsverktyg som gör så att en rektangelformad del av bilden kan markeras med hjälp av musen. Gör det möjligt att applicera filter på den markerade delen av bilden. Du kan också göra så att argument till t.ex. Gauss-filtret kan ställas in med ett skjutreglage som du ritar under knappen och som kan regleras med mus eller piltangenter.


\input{modules/w13-examprep-chapter.tex}
%\input{generated/w13-chaphead-generated.tex}
\input{modules/w13-examprep-exercise.tex}

\input{modules/w14-extra-chapter.tex}
%\input{generated/w14-chaphead-generated.tex}
\input{modules/w14-extra-exercise.tex}
\input{modules/w14-extra-lab.tex}


\part{Appendix}
\appendix

%\setcounter{chapter}{3} %next after 3 is D in \Alph
%!TEX encoding = UTF-8 Unicode
%!TEX root = ../compendium2.tex

\chapter{Kojo}\label{appendix:kojo}

\section{Vad är Kojo?}

Kojo%
\footnote{\href{https://en.wikipedia.org/wiki/Kojo_(programming_language)}{en.wikipedia.org/wiki/Kojo\_(programming\_language)}}
 är en integrerad utvecklingsmiljö för Scala som är speciellt anpassad för programmeringsundervisning i grundskolan. Kojo används i LTH:s Science Center Vattenhallen för utbildning av grundskolelärare i programmering och vid skolbesök och annan besöksverksamhet, i vilken lärare och studenter vid LTH arbetar som handledare. 
 
 Kojo är öppen källkod och utvecklingsgemenskapen leds av Lalit Pant från Indien. I Kojo finns även lättillgängliga bibliotek som gör tröskeln lägre att programmera rörlig grafik och enkla spel.

Under kursens första laboration använder vi grafikbiblioteket i Kojo för att illustrera grundläggande begrepp, så som sekvens, alternativ, repetition och abstraktion.  


\begin{figure}[H]
\centering
\includegraphics[width=0.8\textwidth]{../img/kojo/kojo.png}
\caption{Den nybörjarvänliga utvecklingsmiljön Kojo för Scala på svenska.}
\label{fig:appendix:ide:kojo}
\end{figure}

\section{Använda grafikbiblioteket i Kojo}\label{appendix:ide:kojo:install}

Kojo bygger på den beprövade pedagogiska idén med sköldpaddsgrafik \Eng{turtle graphics}\footnote{\url{https://en.wikipedia.org/wiki/Turtle_graphics}}, där du skriver program som styr en sköldpadda med en penna under magen. När sköldpaddan rör sig bildas ett streck av valfri färg på skärmen. Beroende på hur du bestämmer att sköldpaddan ska röra sig och vilken färg du bestämmer att pennan ska ha, kan du skapa olika intressanta bilder och samtidigt lära dig om programmeringens grunder.

Under kursens första laboration ska du använda grafikbiblioteket i Kojo tillsammans med editorn VS \code{code} och \code{scala-cli} i terminalen (se appendix \ref{appendix:terminal} och \ref{appendix:compile}). Ladda ner filen \texttt{kojolib.scala} från \url{https://fileadmin.cs.lth.se/kojolib.scala} och spara i en ny katalog med hjälp av din webbläsare, eller via dessa kommandon (notera att det är stora bokstaven \code{O} och inte en nolla i optionen \code{-sLO}):

\begin{REPLnonum}
> mkdir w01-kojo
> cd w01-kojo
> curl -sLO https://fileadmin.cs.lth.se/kojolib.scala
\end{REPLnonum}

Nu kan du starta Scala REPL och rita med Kojo så här:

\begin{REPLnonum}
> scala-cli repl .
Welcome to Scala 3.1.2 (17.0.2, Java OpenJDK 64-Bit Server VM).
Type in expressions for evaluation. Or try :help.
                                                                                                                               
scala> fram; höger; fram; vänster

\end{REPLnonum}

Du kan starta VS \code{code} i aktuellt bibliotek så här:
\begin{REPLnonum}
> code .
\end{REPLnonum}

Skriv nedan progam i VS \code{code} och spara det i samma katalog som den tidigare nedladdade filen, under ett nytt valfritt filnamn, t.ex. \code{rita.scala}:

\begin{Code}
@main def rita = { fram; höger; fram; vänster }
\end{Code}

Kör ditt fristående program med:
\begin{REPLnonum}
> scala-cli run .
\end{REPLnonum}

Du ska nu få upp ett fönster som heter Kojo Canvas med en sköldpadda som ritat två streck. När du stänger fönstret så avslutas programmet. Prova fler sköldpaddsfunktioner enligt tabell \ref{table:kojo:functions}.

I stället för att ladda ned filen \code{kojolib.scala} så kan du placera dess innehåll på lämpligt ställe i ditt program enligt nedan. Observera att raden som börjar med \code{//> using lib} ska vara en enda lång rad utan radbrytningar.%\code{export} gör Kojos kommandon tillgängliga utan prefix:
\lstinputlisting[breaklines=true,basicstyle=\ttfamily\fontsize{9}{11}\selectfont]{../workspace/w01_kojo/kojo.scala}

\noindent Scala-koden för den svenska paddans api finns här: \\
%\href{https://github.com/litan/kojo/blob/master/src/main/scala/net/kogics/kojo/lite/i18n/svInit.scala}{github.com/litan/kojo/blob/master/src/main/scala/net/kogics/kojo/lite/i18n/svInit.scala} \\
\href{https://github.com/litan/kojo-lib/blob/main/src/main/scala/net/kogics/kojo/i18n/Swedish.scala}{github.com/litan/kojo-lib/blob/main/src/main/scala/net/kogics/kojo/i18n/Swedish.scala}


%Kojo kräver (numera) \emph{inte} att \texttt{java} finns på din dator utan kommer med en egen JVM. 
%Eftersom du behöver tillgång till JDK i kursen, är det lika bra att installera hela JDK direkt (och inte bara JRE, så som beskrivs å länken ovan); se vidare hur du gör detta i avsnitt \ref{appendix:compile:install-jdk}.
%\href{http://www.kogics.net/kojo-download}{www.kogics.net/kojo-download}



\section{Kojo Desktop}

Kojo finns som fristående skrivbordsapplikation, kallad Kojo Desktop. Kojo Desktop innehåller en egen editor med syntaxfärgning för Scala, men fungerar ännu så länge bara för Scala 2. En av de synligaste skillnaderna mellan Scala 2 och Scala 3 är att klammerparenteser vid flerradiga funktioner är nödvändiga i Scala 2, medan Scala 3 har valfria klammerparenteser. Så om du använder Kojo Desktop behöver du komma ihåg att omgärda sekvenser av rader som hör ihop med \code|{| och \code|}|. 

Kojo Desktop är förinstallerad på LTH:s datorer och körs igång med terminalkommandot \texttt{kojo} eller via applikationsmenyn.  För instruktioner om hur du installerar Kojo Desktop på din egen dator se här: \href{http://www.lth.se/programmera/installera/}{lth.se/programmera/installera}

När du startar Kojo första gången, välj ''Svenska'' i språkmenyn och starta om Kojo. Därefter fungerar grafikfunktionerna på svenska enligt tabell \ref{table:kojo:functions} på sidan \pageref{table:kojo:functions}. När du startat om Kojo inställt på svenska ser programmet ut ungefär som i figur \ref{fig:appendix:ide:kojo} på sidan \pageref{fig:appendix:ide:kojo}.

Det finns ett antal användbara kortkommando som du hittar i menyerna i Kojo Desktop. Undersök speciellt Ctrl+Alt+Mellanslag som ger autokomplettering baserat på det du börjat skriva.

%\section{Kojo i Webbläsaren}

%En begränsad variant av Kojo finns tillgänglig för programmering direkt i din webbläsare här: \url{http://kojo.lu.se/}

%När du trycker på play-knappen så kompileras din kod på en server till Javascript via ScalaJS och därefter körs Javascript-koden i din webbläsare. 
%Kojo på webben är också ännu så länge begränsad till Scala 2 och kräver att du omgärdar sekvenser av rader som hör ihop med \code|{| och \code|}|.


\section{Mer om Kojo}

I detta dokument finns en enkel introduktion till Kojo: \\ ''Introduction to Kojo'' \url{http://www.kogics.net/kojo-ebooks#intro}

\noindent I tabell \ref{table:kojo:functions}, som fortsätter på efterföljande sidor, finns ett urval av kommando i Kojo på svenska och engelska.

{\small\renewcommand{\arraystretch}{1.4}
\begin{longtable}{@{}p{0.42\textwidth} p{0.55\textwidth}}

\caption{Ett urval av funktioner i Kojo. Se även \href{http://lth.se/programmera}{lth.se/programmera}}\label{table:kojo:functions}\\

\emph{Svenska/Engelska} & \emph{Vad händer?}  \\ \hline
\input{postchapters/kojo-commands.tex}
\end{longtable}
}%end small

%!TEX encoding = UTF-8 Unicode
%!TEX root = ../compendium2.tex

\chapter{Terminalfönster}\label{appendix:terminal}

\section{Vad är ett terminalfönster?}

I ett terminalfönster kan man skriva kommandon som kör program och hanterar filer. När man programmerar använder man ofta terminalkommandon för att kompilera och exekvera sina program.  
 
\subsubsection{Terminal i Linux}

    \begin{figure}[!b]
    \centering
    \includegraphics[width=1.0\textwidth]{../img/linux-terminal.png}
    \caption{Terminalfönster i Ubuntu öppnas med Ctrl+Alt+T.}
    \label{fig:terminal:linux}
    \end{figure}

I Ubuntu trycker du lättast \textbf{Ctrl+Alt+T} eller sök efter ''terminal'' i app-menyn.  Då öppnas ett fönster med en blinkande markör som visar att det är redo att ta emot dina textkommando. Ett exempel på kommando är \texttt{ls} som skriver ut en lista med filer i den aktuella katalogen, så som visas i fig. \ref{fig:terminal:linux}.

Det som visas i ett terminalfönster sköts av ett \textbf{kommandoskal} \Eng{command shell}, som är redo att ta emot kommando efter en prompt som slutar med ett \texttt{\$}-tecken. När du skriver ett kommando och trycker Enter anropar kommandoskalet en kommandotolk som tolkar och utför dina kommandon. Om ett kommando inte kan tolkas, skrivs ett felmeddelande. 

Det finns många användbara kortkommando, varav de viktigaste visas i tabell \ref{fig:terminal:shortcuts}. Det är bra om du lär dig dessa kortkommandon utantill så att ditt arbete i terminalen går snabbt och smidigt.

\begin{table}[H]
\renewcommand{\arraystretch}{1.15}
\begin{tabular}{@{}r | l}
pil upp/ner & bläddra i kommandohistoriken \\
Tab & ''auto-complete'', fyll i resten baserat på vad du skrivit hittills \\
Tab Tab & två tryck på Tab listar flera alternativ, om så finnes \\
Ctrl+A & ''ahead'', flytta markören till början av raden \\
Ctrl+E & ''end'', flytta markören till slutet av raden \\
Ctrl+K & ''kill'', ta bort tecken från markören till radens slut\\
Ctrl+U & ''undo'', ta bort tecken från markören till början av raden \\
Ctrl+Y & ''yank'', sätt in det som senast togs bort\\
Ctrl+Z & ''zleep'', stoppa pågående process, skriv sedan \texttt{bg} för bakgrundskörning\\
Ctrl+L & rensa terminalfönstret\\
Ctrl+D & avsluta kommandoskalet \\
\end{tabular}
    \caption{Viktiga kortkommandon i Linux terminalfönster.}
    \label{fig:terminal:shortcuts}
\end{table}

\noindent Ctrl+C orsakar normalt ett avbrott av pågående process och istället är \emph{paste} kopplat till Shift+Ctrl+C, men om du vill tvärtom att Ctrl+C ska vara ''Copy'' som vanligt för att kopiera markerad text och göra avbrott med Shift+Ctrl+C , så kan du ställa om detta med terminalfönstrets meny ''Edit $\rightarrow$ Keyboard Shortcuts'', eller liknande.



\subsubsection{PowerShell, Cmd och Linux i Microsoft Windows}
Det finns flera olika sätt att köra terminalkommando i Windows:

\begin{itemize}
\item \textbf{Powershell}. I Microsoft Windows finns kommandotolken \textit{Powershell} med speciell kommandosyntax. Den är inte Linux-baserad men det finns alias definierade för några vanliga Linux-kommandon, inkluderat \texttt{ls}, \texttt{cd} och \texttt{pwd}. Du startar Powershell t.ex. genom att trycka på Windows-knappen och skriva \texttt{powershell}. 
Du kan också, medan du bläddrar bland filer, klicka på filnamnsraden överst i filbläddraren och skriva \texttt{powershell} och tryck Enter; då startas Powershell i aktuellt katalog. %Ändra gärna typsnitt och bakgrundsfärg med hjälp av fönstrets menyer, så att det blir lättare för dig att läsa vad som skrivs.

\item \textbf{Cmd}. Det finns även i Windows den ursprungliga, gamla kommandotolken \textit{Cmd} med helt andra kommandon. Till exempel skriver man i Cmd kommandot \texttt{dir} i stället för \texttt{ls} för att lista filer. 

\item \textbf{WSL}. I både Windows 10 och 11 kan du även köra Ubuntu-terminalen med hjälp av Windows Linux Subsystem (WSL), vilket rekommenderas, speciellt om du inte har möjlighet att göra s.k. dual boot\footnote{Läs mer om dual boot här och be gärna någon om hjälp som gjort det förr:\\ \href{https://www.linuxtechi.com/dual-boot-ubuntu-22-04-and-windows-11/}{https://www.linuxtechi.com/dual-boot-ubuntu-22-04-and-windows-11/}}. 




\begin{itemize}[nolistsep]
\item Se vidare här om hur du kan installera WSL under Windows, (WSL2 rekommenderas före WSL1 om din maskin klarar det): 

\url{https://docs.microsoft.com/en-us/windows/wsl/install}

\item Det finns även ett smidigt tillägg till VS Code som heter Remote-WSL som gör att du kan editera filer i Windows som finns i WSL, se vidare här: 

\url{https://code.visualstudio.com/docs/remote/wsl-tutorial}

\end{itemize}

\item \textbf{Windows Terminal}. Den nya Microsoft-appen \textit{Windows Terminal} rekommenderas oavsett om du använder Powershell, Cmd eller WSL. Läs mer här om hur du installerar Windows Terminal: \\
  \url{https://docs.microsoft.com/en-us/windows/terminal/}

\end{itemize}







% \url{https://ubuntu.com/wsl} 

% Läs mer här: \href{https://www.omgubuntu.co.uk/2020/03/windows-10-linux-kernel-update}{www.omgubuntu.co.uk/2020/03/windows-10-linux-kernel-update}



\subsubsection{Terminal i Apple macOS/OS X}


Apple OS X och macOS är Unix-baserade operativsystem. De flesta vanliga terminalkommandon som fungerar i Linux fungerar också under Apple OS X och macOS. Du startar ett terminalfönster i Apples operativsystem genom att klicka på förstoringsglaset uppe till höger, skriva \texttt{terminal}, och trycka Enter.

\section{Vad är en path/sökväg?}\label{terminal:path}

När du skriver ett kommando i terminalen, eller kör vilket program som helst på din dator, behöver operativsystemet identifera i vilken fil programmets maskinkod ligger innan programmet kan köras. 

Lokaliseringen av filer sker med hjälp av en \textbf{sökväg} \Eng{path}, som anger en position i filsystemet. Ofta betraktas filsystemet som ett upp-och-ned-vänt träd, och kallas därför även ''filträdet''. Den ''översta'' positionen kallas ''rot'' \Eng{root} och betecknas med ett enkelt snedstreck \texttt{/}. Kataloger som ligger i kataloger utgör förgreningar i trädet. En sökväg pekar ut vägar genom trädet som behövs för att nå ''löven'', som utgörs av själva filerna.

Du kan se var ett program ligger i Linux med hjälp av kommandot \texttt{which} enligt nedan.\footnote{Skriv \texttt{ gcm ls } i Windows Powershell för motsvarighet till \texttt{ which ls } \\ Eller skriv \texttt{ New-Alias which get-command } för tillgång till kommandot \texttt{which} i Powershell. \\ \href{http://stackoverflow.com/questions/63805/equivalent-of-nix-which-command-in-powershell}{stackoverflow.com/questions/63805/equivalent-of-nix-which-command-in-powershell}} Listan med kataloger i sökvägen avskiljs med snedstreck.
\begin{REPLnonum}
$ which java
/usr/lib/jvm/oracle_jdk8/bin/java
$ which ls
/bin/ls
\end{REPLnonum}

En sökväg kan vara \textbf{absolut} eller \textbf{relativ}. En absolut sökväg utgår från roten och visar hela vägen från rot till destination, t.ex. \texttt{/usr/bin/firefox}, medan en relativ sökväg utgår från aktuellt katalog (där du ''står'') och börjar \textit{inte} med ett snedstreck.

Alla operativsystem håller reda på en mängd olika sökvägar för att kunna hitta speciella filer i filträdet. Dessa sökvägar lagras i s.k. \textbf{miljövariabler} \Eng{environment variables}. Det finns en \textit{speciell} miljövariabel som heter kort och gott \textbf{PATH}, i vilken alla sökvägar till de program finns, som ska vara tillgängliga för din användaridentitet direkt för exekvering genom sina filnamn, \textit{utan} att man behöver ange absoluta sökvägar. 

Du kan i Linux se vad som ligger i din PATH med kommandot \code{ echo $PATH } medan man i Windows Powershell skriver \code{$env:Path} där det bara är första bokstaven som ska vara en versal. I Linux separeras katalogerna i sökvägen med kolon, medan Windows använder semikolon.

Ibland kan du behöva uppdatera din PATH för att program som du installerat och ska bli allmänt tillgängliga. Detta görs på lite olika sätt i olika operativsystem, för Linux se t.ex. här:
\href{http://stackoverflow.com/questions/14637979/how-to-permanently-set-path-on-linux}{stackoverflow.com/questions/14637979/how-to-permanently-set-path-on-linux}

När man anger sökvägar finns några tecken med speciell betydelse:

\begin{tabular}{r  p{0.8\textwidth}}
\code|~| & ''tilde'', din hemkatalog \\
\code|/| & ''slash'', snedstreck anger filträdets rot om det finns i början av sökvägen, men utgör katalogsavskiljare inuti sökvägen \\
\code|.| & en punkt anger aktuell katalog, där du ''står'' \\
\code|..| & två punkter anger ett steg ''upp'' i filträdet \\
\code|"| & omgärda en sökväg med citationstecken, först och sist, om den innehåller annat än engelska bokstäver, t.ex. blanktecken\\
\code|\ | & \textit{backslash+blanktecken} används för att beteckna mellanslag i sökvägar som \textit{inte} omgärdas av citationstecken\\
\end{tabular}

\section{Några viktiga terminalkommando}

I tabell \ref{fig:terminal:commands} finns en lista med några viktiga terminalkommando som är bra att lära sig utantill.

En introduktion till LTH:s datorer med exempel på hur du använder vanliga Linux-kommandon finns i denna skrift \url{http://www.ddg.lth.se/perf/unix/} som används i introduktionsveckan för nybörjare på datateknikprogrammet vid LTH.

På sajten \url{http://ss64.com/} finns en mer omfattande lista med användbara terminalkommando och tillhörande förklaringarför för Linux (Bash), Windows (Powershell, Cmd) och Apple OS X (Bash).  

\begin{table}[H]
\renewcommand{\arraystretch}{1.25}
   
\begin{tabular}{@{}r | l}
\texttt{ls} & lista filer i aktuell katalog (alltså där du ''står'')\\
\texttt{ls} \textit{p}  & lista filer i katalogen  \textit{p} \\
\texttt{ls -A} & lista alla filer i aktuell katalog, även gömda \\
\texttt{man ls} & manual för kommandot \texttt{ls}; testa även \texttt{man} för andra kommandon! \\
\texttt{cd} \textit{p} & ''change directory'', ändra aktuell katalog till \textit{p}\\
\texttt{pwd} & ''print working directory'', skriv ut sökväg för aktuell katalog \\
\texttt{cp} \textit{p1 p2} & ''copy'', kopiera filen med path \textit{p1} till en ny fil kallad \textit{p2} \\
\texttt{mv} \textit{p1 p2} & ''move'', byt namn på filen \textit{p1} till \textit{p2}  \\
\texttt{rm} \textit{p} & ''remove'', ta bort filen \textit{p}\\
\texttt{rm -r} \textit{p} & ''remove recursive'', ta bort katalogen \textit{p} med allt innehåll; var försiktig!\\
\texttt{mkdir} \textit{p} & ''make dir'', skapa ett en katalog \textit{p}\\
\texttt{cat} \textit{p1 p2}& ''concatenate'', skriv ut hela innehållet i en eller flera filer \textit{p1 p2 etc.}\\
\texttt{less} \textit{p}& skriv ut innehållet i filen \textit{p}, en skärm i taget\\
\texttt{wget} \textit{url}&ladda ner \textit{url}, t.ex. \texttt{ wget http://cs.lth.se/pgk/ws -o ws.zip}\\
\texttt{unzip} \textit{p}& packa upp \textit{p}, t.ex. \texttt{ unzip ws.zip}\\
\end{tabular}

    \caption{Några viktiga terminalkommando i Linux. Med \textit{p}, \textit{p1}, \textit{p2}, etc.  avses en absolut eller relativ sökväg \Eng{path}, se avsnitt \ref{terminal:path}.}
    \label{fig:terminal:commands}

\end{table}


%!TEX encoding = UTF-8 Unicode
%!TEX root = ../compendium2.tex

\chapter{Editera, kompilera och exekvera}\label{appendix:compile}

\section{Vad är en editor?}

En editor används för att redigera programkod. Det finns många olika editorer att välja på. Erfarna utvecklare lägger ofta mycket energi på att lära sig att använda favoriteditorns kortkommandon och specialfunktioner, eftersom detta påverkar stort hur snabbt kodredigeringen kan göras.

En bra editor har \textbf{syntaxfärgning} för språket du använder, så att olika delar av koden visas i olika färger. Då går det mycket lättare att läsa och hitta i koden.

Nedan listas några viktiga funktioner som man använder många gånger dagligen när man kodar:

\begin{itemize}
\item \textbf{Navigera}. Det finns flera olika sätt att flytta markören och bläddra genom koden. Alla editorer erbjuder sökmöjligheter, och de flesta editorer har även mer avancerade sökfunktioner där kodmönster kan identifieras och multipla sökträffar markeras över flera kodfiler.

\item \textbf{Markera}. Att markera kod kan göras på många sätt: med piltangenter+Shift, med olika speciella menyalternativ, med mus + dubbelklick eller trippelklick, etc. I vissa editorer finns även möjlighet att ha multipla markörer så att flera rader kan editeras samtidigt.

\item \textbf{Kopiera}. Genom Copy-Paste slipper du du skriva samma sak många gånger. Kortkommandona Ctrl+C för Copy och Ctrl+V för Paste sitter i fingrarna efter ett tag. Man ska dock vara medveten om att det lätt blir fel när man kopierar en stor del som sedan ska ändras lite; många Copy-Paste-buggar kommer av att man inte är tillräckligt noggrann och ofta är det bättre att skriva från grunden i stället för att kopiera så att du hinner tänka efter medan du skriver.

\item \textbf{Klipp ut}. Genom Ctrl+X för Cut och Ctrl+V för Paste, kan du lätt flytta kod. Att skriva kod är en stegvis process där man gör många förändringar under resans gång för att förbättra och vidareutveckla koden. Att flytta på kod för att skapa en bättre struktur är mycket vanligt.

\item \textbf{Formatering}. Med indragningar, radbrytningar och nästlade block i flera nivåer får koden struktur. Många editorer kan hjälpa till med detta och har speciella kortkommandon för att ändra indragningsnivå inåt eller utåt.

\item \textbf{Parentesmatchning}. Olika former av parenteser, \code+ ( { [ ) } ] +,  behöver matchas för att koden ska fungera; annars går kompilatorn ofta helt vilse och konstiga felmeddelanden kan peka på helt fel plats i koden. En bra kodeditor kan hjälpa dig att markera vilka parentespar som hör ihop så att du undviker att spendera för mycket tid med att leta efter en parentes som saknas eller står i vägen.

\end{itemize}

\subsection{Välj editor}\label{appendix:compile:edit}

I tabell \ref{edit:popular-editors} visas en lista med några populära editorer. Det är en stor fördel om din favoriteditor finns på flera plattformar så att du har nytta av dina förvärvade färdigheter när du behöver växla mellan olika operativsystem.

I denna kurs rekommenderas Visual Studio \textbf{\texttt{code}}, eftersom den är öppen, gratis och finns för Linux, Windows och Mac, och har bra stöd för Scala och Java. Men om du redan är van vid någon annan av editorerna i tabell \ref{edit:popular-editors} så fungerar de också bra. 

En integrerad utvecklingsmiljö \Eng{integrated development environment, IDE}, se appendix \ref{appendix:ide}, erbjuder många avancerade funktioner som hjälper dig att koda effektivt när du väl lärt dig handgreppen. VS \texttt{code} har numera flera IDE--funktioner, och gränsen mellan en renodlad editor och en IDE, så som IntelliJ och Eclipse, är inte längre lika tydlig som förr.  %Men även när du lärt dig använda en IDE kommer du fortfarande ha stor nytta av en ''vanlig'' editor. Ofta har man flera terminalfönster igång, tillsammans med flera editorfönster och en IDE.

%Om du jobbar i Linux och hellre vill börja med en enklare editor, kan du prova \texttt{gedit}. När du behöver mer avancerade funktioner kan du gå över till \texttt{code}.

%Det kan vara bra att lära sig de allra mest basala kommandona (starta, spara, ändra text och avsluta) i editorerna \texttt{vim} och \texttt{nano}, eftersom dessa kan köras direkt i terminalen, även vid fjärrinloggning utan fönstersystem, och finns förinstallerade i de flesta Linux-system.


\begin{table}

\renewcommand{\arraystretch}{2.0}\small

    \caption{Några populära editorer. I kursen rekommenderas VS Code.}
    \label{edit:popular-editors}

\begin{longtable}{@{}r | p{0.8\textwidth}}
\textit{Editor} & \textit{Beskrivning} \\ \hline

VS Code & Öppen, fri och gratis. Finns för Linux, Windows, \& Mac. Är förinstallerad på LTH:s Linux-datorer och startas med kommandot \verb+code+. Öppenkällkodsprojektet startades av Microsoft och har en aktiv gemenskap med många utvecklare och många användbara tillägg \Eng{extensions}. Sök efter tillägget \texttt{scalameta.metals} och installera så får du syntaxfärgning och många andra IDE-funktioner för Scala.
\newline \url{https://code.visualstudio.com/} \newline \url{https://scalameta.org/metals/docs/editors/vscode/#installation}\\

Gedit & Öppen, fri och gratis. Lätt att lära men inte så avancerad. Är förinstallerad på LTH:s Linux-datorer och startas med kommandot \verb+gedit+. \newline \url{https://wiki.gnome.org/Apps/Gedit} \\

Nano & Öppen, fri och gratis. En simpel editor för enkla småjobb i terminalen. Är förinstallerad på de flesta Linux-system på planeten Jorden. Startas med kommandot \verb+nano+. \newline \url{https://www.nano-editor.org/}\\

Notepad++ & Öppen, fri och gratis. Utvecklad speciellt för Windows men finns även för Linux. \newline \url{https://notepad-plus-plus.org/} \newline \url{https://snapcraft.io/notepad-plus-plus}\\

Vim & Öppen, fri och gratis. Hög inlärningströskel. Finns för Linux, Windows, \& Mac. Är förinstallerad på LTH:s Linux-datorer och startas med kommandot \verb+vim+. Med Scala Metals (se länk nedan) får du IDE-liknande funktioner. Du avslutar vim genom att trycka Escape och sedan skriva :q och trycka Enter.\newline \url{http://www.vim.org/} \newline \url{https://scalameta.org/metals/docs/editors/vim.html}\\

Emacs & Öppen, fri och gratis. Hög inlärningströskel. Finns för Linux, Windows, \& Mac. Är förinstallerad på LTH:s Linux-datorer och startas med kommandot \verb+emacs+. Med Scala Metals (se länk nedan) får du IDE-liknande funktioner. \newline \url{http://www.gnu.org/software/emacs/} \newline \url{https://scalameta.org/metals/docs/editors/emacs.html}\\

Sublime Text& Sluten källkod. Gratis att prova på, men programmet föreslår då och då att du köper en licens. Finns för Linux, Windows, \& Mac. Med Scala Metals (se länk nedan) får du IDE-funktioner. \newline \url{http://www.sublimetext.com/3} \newline \url{https://scalameta.org/metals/docs/editors/sublime.html} \\



% Textwrangler & Sluten källkod. Gratis. Lätt att lära men inte så avancerad. Finns endast för Mac.
% \newline \url{http://www.barebones.com/products/textwrangler/} \\

\end{longtable}

\end{table}

\section{Vad är en kompilator?}

En \textbf{kompilator} \Eng{compiler} är ett program som läser programtext och översätter den till exekverbar maskinkod, så som visas i figur \ref{fig:appendix:compiler}. Programtexten som kompileras kallas källkod och utgörs av text som följer reglerna för ett programmeringsspråk, till exempel Scala eller Java.

\begin{figure}[H]
\centering
\begin{tikzpicture}[node distance=1.8cm, scale=1.5]
\node (input) [startstop] {\bf\sffamily Källkod};
\node(inptext) [right of=input, text width=2cm, xshift=1.5cm]{För\\människor};
\node (compile) [process, below of=input] {\bf\sffamily Kompilator};
%\node(explain) [right of=compile, text width=5cm, xshift=3.0cm]{Översätter från källkod till maskinkod};
\node (output) [startstop, below of=compile] {\bf\sffamily Maskinkod};
\node(outtext) [right of=output, text width=2cm, xshift=1.5cm]{För\\maskiner};
\draw [arrow] (input) -- (compile);
\draw [arrow] (compile) -- (output);
\end{tikzpicture}
    \caption{En kompilator översätter från källkod till maskinkod.}
    \label{fig:appendix:compiler}
\end{figure}




Vissa kompilatorer genererar kod som kan köras av en processor direkt, medan andra kompilatorer genererar ett mellanformat som tolkas under exekveringen. Det senare är fallet med Java och Scala, vilket möjliggör att programmet kan kompileras en gång för alla plattformar och sedan kan programmet köras på all de processorer till vilka det finns en s.k. virtuell maskin för Java \Eng{Java Virtual Machine, JVM}. Den kod som genereras av en kompilator för JVM kallas \textbf{bytekod}.

Om kompileringen inte lyckas skriver kompilatorn ut ett felmeddelande och ingen maskinkod genereras. Det är inte lätt att bygga en kompilator som ger bra felmeddelanden i alla lägen, men felmeddelandet ger oftast goda ledtrådar till felorsaken efter att man lärt sig tolka det programmeringsspråksspecifika vokabulär som kompilatorn använder.

Även om programmet kompilerar utan felmeddelande och genererar exekverbar maskinkod, är det vanligt att programmet ändå inte fungerar som det är tänkt. Ibland är det mycket svårt att lista ut vad problemet beror på och man kan behöva göra omfattande undersökningar av vad som händer under körningen, genom att t.ex. skriva ut olika variablers värden eller på annat sätt ändra koden och se vad som händer. Denna process kallas felsökning eller avlusning \Eng{debugging}, och är en väsentlig del av all systemutveckling. Läs mer om debugging i Appendix \ref{appendix:debug}.

En uttömmande testning av ett större program, som kör programmets \textit{alla} möjliga exekveringsvägar, är i praktiken omöjlig att genomföra inom rimlig tid, då antalet kombinationsmöjligheter växer mycket snabbt med storleken på programmet.
Därför är kompilatorn ett mycket viktigt hjälpmedel. Med hjälp av den analys och de kontroller som görs av kompilatorn kan många buggar, som annars vore mycket svåra att hitta, undvikas och åtgärdas i kompileringsfasen, redan \textit{innan} man exekverar programmet.


\section{Java JDK}

Scala, Java och flera andra språk använder Java-plattformen som exekveringsmiljö. Om man inte bara vill köra program som andra har utvecklat, utan även utveckla egna program som fungerar i denna miljö, behöver man installera Java Develpment Kit (JDK). Detta utvecklingspaket innehåller flera delar, bland annat:

\begin{itemize}

\item Kompilatorn \texttt{javac} kompilerar Java-program till bytekod som lagras i klassfiler med filnamnsändelsen \texttt{.class}.

\item Exekveringsmiljön Java Runtime Enviroment (JRE) med kommandot \texttt{java} som drar igång den virtuella javamaskinen (Java Virtual Machine) som kan ladda och exekvera bytekod lagrade i klassfiler.

\item Programmet \texttt{jar} som packar ihop många sammanhörande klassfiler till en enda jar-fil som lätt kan distribueras via nätet och sedan köras med \texttt{java}-kommandot på alla maskiner med JRE.

\item Programmet \texttt{javap} som läser klassfiler och skriver ut vad de innehåller i ett format som kan läsas av människor (ett sådant program kallas disassembler).

\item I JDK ingår också en mycket stor mängd färdiga programbibliotek med stöd för nätverkskommunikation, filhantering, grafik, kryptering och en massa annat som behövs när man bygger moderna system.

\end{itemize}

\noindent Du kan läsa mer om Java och dess historik här: \\
\href{https://en.wikipedia.org/wiki/Java_(programming_language)}{https://en.wikipedia.org/wiki/Java\_(programming\_language)}

\subsection{Kontrollera om du har JDK installerat}\label{appendix:compile:check-jdk}

Öppna ett terminalfönster (se appendix \ref{appendix:terminal}) och skriv (observera det avslutande c:et i \texttt{javac}):
\begin{REPLnonum}
javac -version
\end{REPLnonum}
Då ska något som liknar följande skrivas ut, där \texttt{x} och \texttt{y} är siffror:\\
\texttt{javac \JDKVersion.x.y}\\
Om utskriften säger att \texttt{javac} saknas, installera JDK enl. nedan.

Vi använder alltså JDK \JDKVersion~i kursen. Det går också bra att använda de äldre versionerna JDK 8 och JDK 11, men JDK 9 eller 10 fungerar inte med alla verktyg vi använder och senare versioner än \JDKVersion~ kan också ge problem. Läs mer under ''Verktyg'' på kurshemsidan.

%Du kanske redan har enbart Java Runtime Environment (JRE) installerad, men inte JDK. Då saknar du Javakompilatorn \texttt{javac} m.m. och behöver installera JDK, se nedan. Du kan kolla om du har JRE genom att skriva \texttt{java -version} (alltså utan \texttt{c} efter \texttt{java}). Eller så har du redan JDK installerad men inte rätt katalog i din PATH. 





\subsection{Installera JDK}\label{appendix:compile:install-jdk}

Det finns flera JDK-distributioner att välja mellan, varav OpenJDK och Oracle JDK är två exempel. Vi använder OpenJDK i kursen, som kan installeras via \\ \url{https://adoptium.net/temurin/releases/?version=21}. 

Om du installerar alla Scala-verktyg med hjälp av Coursier enligt instruktioner på kurshemsidan under ''Verktyg'', \url{https://lunduniversity.github.io/pgk/#verktyg} så kommer JDK att installeras automatiskt (om du inte redan har JDK). % För att installera JDK på din egen dator behöver du gå igenom flera steg, varav vissa behöver anpassas efter det operativsystem du kör, enligt nedan.

%
%
%
% Din användaridentitet behöver ha administratörsrättigheter för att du ska kunna genomföra installationen.
%
%
%
% \subsubsection{Linux}
% För Ubuntu: läs igenom och följ sedan dessa instruktioner noga: \\ \href{http://www.webupd8.org/2012/09/install-oracle-java-8-in-ubuntu-via-ppa.html}{www.webupd8.org/2012/09/install-oracle-java-8-in-ubuntu-via-ppa.html}
%
% För andra Linux-distributioner, kör detta i terminalen (funkar även i Ubuntu, men du får med detta kommando inte Oracles aningen snabbare JVM): \\ \texttt{sudo apt-get install openjdk-8-jdk}
%
% \subsubsection{Windows/macOS}
%
% \begin{enumerate}
% \item Installera senaste JDK från Oracle. Om du inte har installerat JDK förr på din dator så be gärna någon kurskamrat med erfarenhet av detta att assistera dig medan du följer stegen nedan.
%
% \begin{enumerate}
% \item Surfa till Oracles hemsida för Java SE här: \\ \url{http://www.oracle.com/technetwork/java/javase/downloads/}
%
% \item Klicka på rubriken ''Java SE 8u101 / 8u102'' och på nästa sida klicka på knappen ''Accept License Agreement'' i listan under rubriken ''Java SE Development Kit 8u101''. (Siffrorna 101 eller 102 kan vara annorlunda om senare versioner tillkommit.)
%
% \item Välj rätt version av operativsystem (Windows x64 eller Mac OS X). Det är viktigt att du väljer x64, d.v.s 64-bitarsvarianten som gäller för alla moderna datorer.
%
% \item Klicka på länken och en stor fil kommer laddas ner till din dator.
%
% \item Installera när filen laddats färdigt.
%
% \end{enumerate}
%
% \item Uppdatera PATH, så att du får tillgång till alla kommando i terminalen:
% \begin{itemize}
% \item För Windows görs detta enklast genom att ladda ner och sedan köra denna fil genom att dubbelklicka på den: \\ \mbox{\href{https://github.com/lunduniversity/introprog/raw/master/tools/windows-jdk-set-path.bat}{github.com/lunduniversity/introprog/raw/master/tools/windows-jdk-set-path.bat}}
% \item För macOS, läs här: \\ \href{https://docs.oracle.com/javase/8/docs/technotes/guides/install/mac_jdk.html}{docs.oracle.com/javase/8/docs/technotes/guides/install/mac\_jdk.html}
% \\ Ge bästa rådet att sätta path på mac; HOMEBREW!!!
%
% \item Om något krånglar, be om hjälp. Om du behöver mer detaljer om PATH-uppdatering för java, läs här:  \href{https://java.com/sv/download/help/path.xml}{java.com/sv/download/help/path.xml} \\
% Om du kör engelska menyer byt \texttt{sv} mot \texttt{en} i adressen ovan.  Du kan ta reda på vilken katalog som ska läggas in sist i din PATH genom att bläddra bland dina systemfiler och undersöka var JDK har installerats; i Windows antagligen något liknande detta (kolla exakt vilket versionsnummer du har): \code|C:\Program Files\Java\jdk1.8.0_101\bin|
% \end{itemize}
%
% \item Starta om datorn. Det är först efter att en ny användarinloggning initierats, som PATH-tilldelningen får effekt.
%
% \item Kontrollera att \texttt{javac} fungerar enligt avsnitt \ref{appendix:compile:check-jdk}.
% \end{enumerate}


\section{Scala}

Scala använder Java Virtual Machine (JVM) som exekveringsmiljö, men går även att köra i browsern med hjälp av ScalaJS-kompilatorn som kompilerar från Scala till JavaScript. I denna kurs använder vi i Scala på JVM.
I en Scala-installation ingår bl.a. kompilatorn \texttt{scalac} och även ett interaktivt kommandoskal kallat Scala REPL (se nedan \ref{appendix:compile:REPL}) där du kan testa din Scala-kod rad för rad och se vad som händer direkt.

%De flesta av kursens övningar görs i Scala REPL (förk. \textit{read-evaluate-print-loop}), medan laborationerna kräver kompilering av lite större program.

Den officiella hemsidan för Scala finns här: \url{http://www.scala-lang.org/}

Du hittar mer om Scalas historik och annan bakgrundsinformation här:\\\mbox{%
 \href{https://en.wikipedia.org/wiki/Scala_(programming_language)}{en.wikipedia.org/wiki/Scala\_(programming\_language)}
}

\subsection{Installera Scala}

Scala finns förinstallerat på LTH:s datorer. På kurshemsidan under ''Verktyg'' finns detaljerade instruktioner om hur du installerar Scala på din egen dator:  \\ \url{https://lunduniversity.github.io/pgk/#verktyg}

% \begin{enumerate}
% \item Kontrollera att du har JDK installerad enligt avsnitt \ref{appendix:compile:check-jdk} och installera vid behov enligt avsnitt \ref{appendix:compile:install-jdk}.
% \item Surfa till denna hemsida för nedladdning av Scala 2.11.8: \\ \url{http://scala-lang.org/download/2.11.8.html}
% \item Klicka på ''Download'' av den variant som är relevant för ditt operativsystem och spara filen:
%
% \begin{enumerate}
% \item \textbf{Linux Ubuntu}: Filen heter \texttt{scala-2.11.8.deb} och installeras genom att dubbelklicka på filen eller via terminalkommandot:\\ \code{sudo apt install ~/Downloads/scala-2.11.8.deb} \\ Anpassa sökvägen ovan efter var du sparade filen.
% \item \textbf{Windows}: Filen heter \texttt{scala-2.11.8.msi} och installationen startas med ett dubbelklick. Följ instruktionerna. Installationsprogrammet uppdaterar även din PATH åt dig och kommandot \texttt{scala} bör fungera efter omstart.
% \item \textbf{Mac}: Filen heter \texttt{scala-2.11.8.tgz} och kan packas upp på lämpligt ställe med terminalkommandot \texttt{tar -xvzf scala-2.11.8.tgz} och sedan är det underkatalogen \texttt{bin} som ska inkluderas i din PATH. \TODO klura ut säkraste rådet för PATH-uppdatering på mac -- enklast är nog att visa hur man installerar via homebrew
% \end{enumerate}
% \end{enumerate}
% Kontrollera, efter ev. omstart, att terminalkommandot \texttt{scala} nu kan användas för att starta Scala REPL på din dator:
% \begin{REPLnonum}
% > scala
% Welcome to Scala 2.11.8 (Java HotSpot(TM) 64-Bit Server VM, Java 1.8.0_101).
% Type in expressions for evaluation. Or try :help.
%
% scala> val msg = "hej"
% msg: String = hej
%
% scala> println(msg)
% hej
%
% scala>
% \end{REPLnonum}


\subsection{Scala Read-Evaluate-Print-Loop (REPL)}\label{appendix:compile:REPL}

För många språk, t.ex. Scala och Python, finns det ett interaktivt program ämnat för terminalen som gör det möjligt att exekvera enstaka programrader och direkt se effekten. Ett sådant program kallas \textit{Read-Evaluate-Print-Loop} (REPL), eftersom det läser  och tolkar en rad i taget. Resultatet av evalueringen av din kod skrivs ut i terminalen och därefter är kommandoskalet redo för nästa kodrad.

Kursens övningar bygger till stor del på att du använder Scala REPL för att undersöka principer och begrepp som ingår i kursen genom dina egna kodexperiment. Även när du på labbarna utvecklar större program med en editor och en IDE, är det bra att ha Scala REPL till hands. Då kan du klistra in delar av programmet du håller på att utveckla i Scala REPL och stegvis utveckla delprogram, som till slut fungerar så som du vill.

I Scala REPL får du se typinformation för variabler och metoder, vilket är till stor hjälp när man försöker lista ut vad en kodrad innebär. Genom att öva upp din förmåga att dra nytta av Scala REPL, kommer din produktivitet öka.

Du startar Scala REPL med kommandot \texttt{scala} och skriver Scala-kod efter prompten \texttt{scala>} och kompilering+exekvering sker när du trycker Enter.
\begin{REPLnonum}
> scala
Welcome to Scala 3.1.2 (17.0.2, Java OpenJDK 64-Bit Server VM).
Type in expressions for evaluation. Or try :help.
                                                                                                                               
scala> 41 + 1
val res0: Int = 42
\end{REPLnonum}

Varje evaluerat värde sparas i en ny variabel, här \code{res0}.

Om du skriver en ofullständig rad fortsätter editeringen på nästa rad. Du kan navigera mellan raderna med pil-upp- och pil-ner-tangenterna. När du avslutar med en rad som gör din kod fullständig så kompileras och exekveras alla raderna. Du kan avbryta flerradsediteringen i förtid genom skriva ett semikolon \texttt{;} och sen trycka Enter. Vill du fortsätta editeringen med en ny rad och förhindra för tidig evaluering så tryck Esc+Enter. Escape-tangenten finns överst till vänster på tangentbordet. Se exempel nedan:

\begin{REPLnonum}
scala> def fleraRader = 42  // Esc+Enter ger ny rad
     |   + "ny rad".length  // fortsättningsrad, avsluta med Enter
\end{REPLnonum}

Beroende på vilket operativsystem du kör så kan även andra tangentkombinationer fungera för att starta ny rad i REPL; prova t.ex. Linux: Left Alt+Enter, Windows: Left Alt + Shift + Enter, VS Code Terminal i Windows Left Alt + Enter, VS Code Terminal i MacOS: Option + Enter.

Många av de kortkommandon som fungerar i terminalens kommandoskal, fungerar också i Scala REPL. Gå gärna igenom listan i tabell \ref{fig:terminal:shortcuts} på sidan \pageref{fig:terminal:shortcuts}, och testa vad som händer i Scala REPL. Om du tränar upp din fingerfärdighet med dessa kortkommandon, går ditt arbete i Scala REPL väsentligt snabbare.

Med kommandot \texttt{:help} får du se en lista med specialkommandon för Scala REPL:

\begin{REPLsmall}
The REPL has several commands available:

:help                    print this summary
:load <path>             interpret lines in a file
:quit                    exit the interpreter
:type <expression>       evaluate the type of the given expression
:doc <expression>        print the documentation for the given expression
:imports                 show import history
:reset [options]         reset the repl to its initial state, forgetting all session entries
:settings <options>      update compiler options, if possible


\end{REPLsmall}

Du kan också starta Scala REPL med hjälpa av kommandot \code{scala-cli repl . } ~med ett blanktecken och en punkt på slutet. Punkten gör att alla \code{.scala}-filer som finns i aktuell katalog kompileras av Scala CLI och görs tillgänglig för användning i REPL.  

\subsection{Kompilera och kör med Scala Command Line Interface}\label{appendix:compile:scala-cli}

Det finns sedan 2022 ett nytt smidigt kommandoradsgränssnitt \Eng{command line interface} för att kompilera, exekvera och paketera Scala-program som kallas \emph{Scala CLI}. Om du installerar Scala-verktygen enligt instruktioner på kurshemsidan under ''Verktyg'', \url{https://lunduniversity.github.io/pgk/#verktyg} så medföljer Scala CLI. 

Här finns några användbara kommandon:
\begin{itemize}
\item Första gången du kör en nyinstallerad Scala CLI-installation så kör detta kommando så att du får tillgång till smidiga kompletteringar med TAB-tangenten:\\
\texttt{scala-cli install completions}

\item Med hjälp av detta kommando kan du förbereda VS Code för samverkan med Scala CLI (notera blanktecken och avslutande punkt):\\ 
\texttt{scala-cli setup-ide .}\\
Kör ovan kommando innan du startar VS Code första gången med \texttt{code .} i aktuell katalog, eller avsluta VS Code och kör ovan kommando och starta VS Code igen med \texttt{code .}  i aktuell katalog.


\item Scala CLI kan köra igång REPL i aktuell katalog med dina Scala- och Java-program automatiskt kompilerade och tillgängliggjorda i REPL med hjälp av nedan kommando. Med optionen \code{-S} anger du vilken version av Scala du vill köra:\\
\code{scala-cli repl . -S 3}

\item I stället för att ange Scala-version med optionen \code{-S} på kommandoraden kan du inuti ditt program, på första raden, skriva denna ''magiska'' kommentar:\\ 
\code{//> using scala "3.1.2"} \\
Då kommer Scala CLI att automatiskt välja (och vid behov ladda ned) önskad version av Scala-kompilatorn (notera \code{>} efter \texttt{//}):

\item Kompilera alla Scala- och Java-program i aktuell katalog och se eventuella felmeddelanden. Med hjälp av \code{--watch} (kan förkortas till \code{-w}) så kompileras alla filer automatiskt om så fort ändringar sparas i VS Code (kortkommando Ctrl+S):\\
\code{scala-cli compile . --watch}

\item Kör Scala- och Java-program i aktuell katalog med start av den topp-nivå-\code{def} som är märkt \code{@main} (om det finns flera får du en frågan om vilken \code{@main def} du vill köra).:\\
\code{scala-cli run .}

\item Skapa en exekverbar fil:\\
\texttt{scala-cli package .}

\item Skapa en kopia av ditt projekt med katalogstruktur och filer anpassade för byggverktyget \code{sbt} (se Appendix \ref{appendix:build}):\\
\verb|scala-cli export . --sbt --output ../nameofnewprojdir|\\
Ändra katalognamnet \code{nameofnewprojdir} till valfritt nytt namn på en katalog som inte existerar. Notera de dubbla punkterna som gör att nya katalogen hamnar på samma nivå som ditt nuvarande projekt, och \emph{inte} i din aktuella katalog (för att undvika att dubbletter av dina scala-filer ger kompileringsfel).

\item Om du skriver \texttt{scala-cli help} så får du se vad du mer kan göra.
\end{itemize}


\noindent
Läs mer om Scala CLI i Appendix \ref{appendix:build:scala-cli} och här: \\ \url{https://scala-cli.virtuslab.org/}


% \begin{table}
% \renewcommand{\arraystretch}{1.25}\centering
%     \caption{Några vanliga kommandon i Scala REPL.}
%     \label{fig:repl:shortcuts}
% \begin{tabular}{r | c | l}
% \textit{Kommando} & \textit{Förk.} & \textit{Beskrivning} \\ \hline
%  \texttt{:help}     & \texttt{:he} & visa lista med kommando och förklaringar\\
%  \texttt{:load} \textit{path}    & \texttt{:load} \textit{path} & klistra in en hel fil, t.ex. \code|:load util/mio.scala|\\
%  \texttt{:quit} & \texttt{:q}  & avsluta Scala REPL \\
%  \texttt{:require} \textit{path} & \texttt{:req} \textit{path} & jar-fil till classpath, t.ex. \texttt{:req lib/cslib.jar}\\

%  \texttt{:type} & \texttt{:t}  & visa typ med -v för ''verbose'', t.ex. \code|:t -v 42.0| \\

%  \texttt{:warnings} & \texttt{:w}  & visa beskrivning av ev. varningar \\

% \end{tabular}

% \end{table}

\input{postchapters/debug.tex}
%!TEX encoding = UTF-8 Unicode
%!TEX root = ../compendium.tex

\chapter{Dokumentation}\label{appendix:doc}

Dokumentation hjälper andra att använda din kod, men underlättar även för dig själv när du vid ett senare tillfälle ska erinra dig hur den fungerar och hur du ska använda och bygga vidare på din kod. Modern systemutveckling baseras ofta på öppen källkod och färdiga api \Eng{application programming interface}, där kvaliteten på dokumentationen är avgörande för hur lätt det är att komma igång med att använda koden.

Nedan listas exempel på olika typer av  dokumentation\footnote{\href{https://en.wikipedia.org/wiki/Software_documentation}{en.wikipedia.org/wiki/Software\_documentation}}:

\begin{itemize}
\item \textbf{Kravdokumentation} beskriver det övergripande målet med mjukvaran, samt funktionella krav och kvalitetskrav som uppfylls av systemet.
\item \textbf{Designdokumentation} beskriver arkitekturen, hur koden är organiserad i moduler, och den interna systemstrukturen t.ex. i form av klasser, objekt och deras relation.
\item \textbf{Slutanvändardokumentation} kan t.ex. vara manualer för användning av systemet och installationsanvisningar.
\item \textbf{Teknisk dokumentation} kan t.ex. vara api-dokumentation som beskriver vilka funktioner som ingår i ett programbibliotek. Sådan dokumentation genereras ofta med hjälp av ett \textbf{dokumentationsverktyg} (se avsnitt \ref{appendix:buildtool}).  Andra typer av teknisk dokumentation är instruktioner om hur man bygger koden med eventuellt tillhörande byggverktygskonfigurationsfiler; ofta beskrivs byggförfarandet steg för steg i en textfil med namnet \code{README}. (Läs mer om byggverktyg i appendix \ref{appendix:build}.)
\end{itemize}

\noindent Det är en stor utmaning att hålla dokumentationen uppdaterad allteftersom koden utvecklas. Även om man får hjälp att generera en navigerbar sajt av ett dokumentationsverktyg, måste själva \textit{innehållet} i de manuellt författade dokumentationskommentarerna vara i överensstämmelse med den aktuella versionen av koden. Uppdateras koden, måste man alltså vara noga med att uppdatera dokumentationskommentarerna, annars uppstår stor förvirring.

Detta problem är så pass allvarligt att man ska tänka sig noga för hur man kan formulera  dokumentationskommentarerna på ett framtidssäkert sätt, och hur omfattande de ska vara i förhållande till den framtida arbetsinsatsen med att hålla dem uppdaterade. Desto mer omfattande kommentarer desto mer jobb att hålla dem uppdaterade.

Det är i praktiken svårt att uppnå en optimal balans mellan bra och många kommentarer som \textit{hjälper} användaren, och å andra sidan svårunderhållna och föråldrade kommentarer som \textit{stjälper} användare.


\section{Vad gör ett dokumentationsverktyg?}\label{appendix:buildtool}

Ett dokumentationsverktyg genererar teknisk dokumentation av koden baserat på speciella \textbf{dokumentationskommentarer} som skrivs i koden omedelbart före deklarationer av det som ska dokumenteras. Dessa dokumentationskommentarer skrivs enligt en speciell syntax som dokumentationsverktyget kan tolka.

Utdata från ett dokumentationsverktyg utgörs typiskt av en webbsajt med ändamålsenlig formatering och navigationslänkar, se figur \ref{fig:appendix:doctool}.

\begin{figure}[H]
\centering
\begin{tikzpicture}[node distance=1.8cm, scale=1.5]
\node (input) [startstop] {\bf\sffamily Källkod};
\node(inptext) [right of=input, text width=6cm, xshift=4.2cm]{med speciella dokumentationskommentarer före deklarationer};
\node (compile) [process, below of=input] {\bf\sffamily Dokumentationsverktyg};
%\node(explain) [right of=compile, text width=5cm, xshift=3.0cm]{Översätter från källkod till maskinkod};
\node (output) [startstop, below of=compile] {\bf\sffamily Dokumentation};
\node(outtext) [right of=output, text width=6cm, xshift=4.2cm]{t.ex. en webbsajt med dokumentation och navigationslänkar};
\draw [arrow] (input) -- (compile);
\draw [arrow] (compile) -- (output);
\end{tikzpicture}
    \caption{Ett dokumentationsverktyg läser koden och dokumentationskommentarer och genererar dokumentation, t.ex. i form av en webbsajt.}
    \label{fig:appendix:doctool}
\end{figure}



\section{scaladoc}
\newcommand{\scaladoc}{\texttt{scaladoc}}

Med Scala-installationen följer dokumentationsverktyget \scaladoc, som genererar en webbsajt med ändamålsenlig layout och specialfunktioner för att söka, filtrera och navigera i dokumentationen.

Dokumentationen av stora bibliotek kan bli omfattande och det krävs träning i att använda dokumentationssajter för att få maximal nytta av dem. I efterföljande avsnitt beskrivs först hur du använder dokumentation som är genererad med \scaladoc. Därefter visas hur du själv kan generera dokumentation för din egen kod.


\subsection{Använda dokumentation från scaladoc}

Dokumentationen av Scalas standardbiliotek är genererad med \scaladoc\ och att navigera i denna ger bra träning i hur man använder avancerad api-dokumentation. Du hittar dokumentationen för Scalas standardbibliotek här: \\
\url{http://scala-lang.org/api/current}


När du surfar dit möts du av dokumentationen för \textit{root package}, som ger en översikt av olika paket i standardbiblioteket. I sökrutan uppe till vänster kan du skriva början på namnet på klasser, traits, eller objekt som du letar efter, så som visas i figure \ref{fig:scaladoc:root-package}.

\begin{figure}[H]
\centering
\includegraphics[width=1.0\textwidth]{../img/scaladoc/scaladoc-root}

     \caption{Förstasidan med dokumentationen av standardbiblioteket i Scala.}
    \label{fig:scaladoc:root-package}
\end{figure}

Genom att skriva text i sökrutan får du en filtrerad lista på allt som har ett namn som börjar på det söker efter. I figur \ref{fig:scaladoc:vec} visas en sökning efter \code{vec}.

\begin{figure}[H]
\centering
\includegraphics[width=1.0\textwidth]{../img/scaladoc/scaladoc-vec}

     \caption{ Sökning efter innehåll som börjar på \code{vec}.}
    \label{fig:scaladoc:vec}
\end{figure}

Om du klickar på \code{Vector} får du se dokumentationen i figur  \ref{fig:scaladoc:vector}

\begin{figure}[H]
\centering
\includegraphics[width=0.9\textwidth]{../img/scaladoc/scaladoc-vector}

     \caption{En del av dokumentationen av klassen \code{Vector}}
    \label{fig:scaladoc:vector}
\end{figure}

Genom att skriva text i den nedre, mörkgrå sökrutan kan du filtrera listan med klassmedlemmar. Om klickar på länken \code{object Vector}, eller på den runda, gröna ikonen med ett stort C, får du se kompanjonsobjektets medlemmer.
Du kan få fram en filtreringsruta med fler möjligheter genom att klicka på expanderingspilen i den mörkgrå sökrutan så som visas i figur \ref{fig:scaladoc:filter}.



\begin{figure}[H]
\centering
\includegraphics[width=0.75\textwidth]{../img/scaladoc/scaladoc-filter}

     \caption{Expanderad sökruta med extra filtreringsmöjligheter.}
    \label{fig:scaladoc:filter}
\end{figure}

%Det finns fler sökmöjligheter och finesser i \scaladoc. Prova att klicka på olika länkar och symboler och upptäck vad du kan få fram för information. Träning i att använda \scaladoc~gör dig till en mer produktiv Scala-utvecklare.

\clearpage

\subsection{Skriva dokumentationskommentarer}


Verktyget \scaladoc\ läser kommentarer som börjar med \verb|/**| och slutar med \verb|*/| och associeras till efterföljande deklaration. Notera de dubbla asteriskerna. Alla rader som följer efter \verb|/**| ska, enligt konventionen för Scalas dokumentationskommentarer, börja med en asterisk \code|*| med indrag med flera blanksteg så att den hamnar under \textit{andra} asterisken i öppningskommentaren, som nedan:
\begin{Code}
/** Först kommer en sammanfattning på en enda rad.
  *
  * Sedan kommer eventuellt en mer detaljerad beskrivning,
  * som kan vara flera rader lång.
  */
\end{Code}
Dokumentationskommentaren slutar med \code|*/| rakt under asterisk-kolumnen.

\begin{itemize}
\item Med \verb|@constructor| i början på en rad ges en speciell kommentar om konstruktorer. 
\item Med \verb|@param| i början på en rad ges en speciell kommentar angående parametrar. 
\item Med \verb|@return| i början av en rad ges en speciell kommentar angående vad som returneras vid metodanrop.
\item Länkar skrivs inom dubbla hakparenteser, enligt exempel nedan.
\end{itemize}

\begin{Code}
/** A person who uses our application.
 *
 *  @constructor create a new person with a name and age.
 *  @param name the person's name
 *  @param age the person's age in years
 */
class Person(name: String, age: Int)  

/** Factory for [[mypackage.Person]] instances. */
object Person:
  /** Creates a person with a given name and age.
   *
   *  @param name their name
   *  @param age the age of the person to create
   */
  def apply(name: String, age: Int) = ???

  /** Creates a person with a given name and birthdate
   *
   *  @param name their name
   *  @param birthDate the person's birthdate
   *  @return a new Person instance with the age determined by the
   *          birthdate and current date.
   */
  def apply(name: String, birthDate: java.util.Date) = ???

\end{Code}


Läs mer här om hur du skriver dokumentationskommentarer: \\
\url{https://docs.scala-lang.org/scala3/guides/scaladoc/}

Läs mer om dokumentation här:\\
\url{https://docs.scala-lang.org/scala3/scaladoc.html}\\
\url{https://scala-cli.virtuslab.org/docs/commands/doc}



\subsection{Generera dokumentation}


Du genererar dokumentation enklast med hjälp av körverktyget \code{scala-cli}. 
I terminalen skriv \\\code{scala-cli doc . -o api} 

När \code{scala-cli} är färdig med att generera dokumentationen så meddelas vilken katalog som dokumentationen ligger i. För att länkarna inom dokumentationen ska fungera krävs antingen att du kör en lokal webbserver i katalogen eller att du använder ett program för att konvertera länkarna till lokala sådana.

\subsubsection{Använda en lokal webbserver}
Om du har Python 3 installerat på din dator har du en inkluderad webbserver. Du startar denna i terminalen med \\\code{python -m http.server} när du står i dokumentationens katalog. För att öppna dokumentationen besöker du sedan \url{http://localhost:8000} i din webbläsare.

\subsubsection{Använda ett program för att konvertera länkar}
Om du inte har Python installerat kan du köra ett Scala-program som byter ut alla länkar till lokala sådana, vilket gör att de går att öppna direkt. Detta kan laddas ner från \\
\url{https://github.com/dixine55/Scaladoc-Local-Version-Patcher/blob/main/scaladocPatch.scala}. Placera programmet i dokumentationsmappen och kör det med \\\code{scala-cli run .} när du är i dokumentationens mapp. Du öppnar sedan dokumentationen genom filen \\\texttt{index.html} i din webbläsare.

Mer att läsa om att generera dokumentation finns här: \\
\url{https://scala-cli.virtuslab.org/docs/commands/doc}


% med terminalkommandot \scaladoc\ följt av en eller flera källkodsfiler. Med optionen \code{-d} anger du i vilken katalog sajten ska sparas. Du visar sajten genom att öppna filen \code{index.html} i en webbläsare. Nedan visas hur dokumentationen genereras för källkodsfilen i figur \ref{fig:scaladoc:mio}.
% \begin{REPLnonum}
% $ scaladoc mio.scala -d apidoc
% $ firefox apidoc/index.html
% \end{REPLnonum}

% I figur \ref{fig:scaladoc:webpage} på sidan \pageref{fig:scaladoc:webpage} visas delar av en webbsida som genererats utifrån koden i figur \ref{fig:scaladoc:mio} på sidan \pageref{fig:scaladoc:mio}. För de publika metoder där ingen dokumentationskommentar finns, visas ändå metodens signatur med parametrar, parametertyper, och returtyp. %Medlemmar som deklareras \code{private} visas inte, men om man klickar på knappen \Button{All} bredvid rubriken \textbf{Visibility} visas medlemmar som är deklarerade \code {protected}.

% %Om du klickar på symbolen \Forward\ till vänster om metodsignaturen, ändras den till symbolen \MoveDown\ som indikerar att den mer detaljerade beskrivningen av parametrar etc. har vecklats ut (i den mån detaljerade kommentarer finns).

% Om du vill ha övergripande dokumentation om ett paket \code{x}, ges det speciella objektet \code{package object x} en dokumentationskommentar med sådan information. Ofta innehåller \code{package object} medlemmar som man vill ska bli synliga vid import av paketet, så som variabler, metoder och implicita medlemmar som inte har någon annan naturlig hemvist.

% \begin{figure}[b]
% \scalainputlisting[numbers=left, basicstyle=\ttfamily\fontsize{9}{11}\selectfont]{../util/mio.scala}
%     \caption{Dokumentationskommentarer som kan kan användas för att generera en dokumentations-webbsajt. Sådana kommentarer börjar  med snedstreck och dubbla asterisker, se bl.a. raderna 8--13 ovan.}
%     \label{fig:scaladoc:mio}
% \end{figure}

% \begin{figure}[t]
% \includegraphics[width=1.0\textwidth]{../img/scaladoc/scaladoc-mio}
%     \caption{Delar av en webbsida genererad med hjälp av \scaladoc. %Mer detaljerade beskrivningar kan i förekommande fall vecklas ut eller in om man växlar mellan \Forward\ och \MoveDown.
%     }
%     \label{fig:scaladoc:webpage}
% \end{figure}


%\subsection{Lära mer om scaladoc}

% Fler tips om \scaladoc:
% \begin{itemize}%[leftmargin=*]

% %\item En video med tips om hur du söker och navigerar i \scaladoc-dokumentation:
% %\\
% %\url{http://docs.scala-lang.org/overviews/scaladoc/interface.html}

% \item
% Riktlinjer för hur du skriver dokumentationskommentarer: \\
% \url{http://docs.scala-lang.org/style/scaladoc.html}

% % \item Länksida till mer detaljerade beskrivningar: \\
% % \url{https://wiki.scala-lang.org/display/SW/Writing+Documentation} \\
% % inkluderande bland annat:
% %
% % \begin{itemize}[nolistsep]
% % \item En beskrivning av syntaxen för formatering: \\
% % \url{https://wiki.scala-lang.org/display/SW/Syntax}
% %
% % \item En beskrivning av speciella annoteringar, t.ex. \code{@param}: \\
% % \url{https://wiki.scala-lang.org/display/SW/Tags+and+Annotations}
% %
% % \end{itemize}

% \item Kör kommandot \texttt{ scaladoc -help } för att se användbara optioner.

% \item Kommandot \texttt{sbt doc} är ett smidigt sätt att automatiskt generera api-dokumentation för alla dina kodfiler om du placerat dem i underbiblioteket \code{src/main/scala} resp. \code{src/main/java}.
% Läs mer om \texttt{sbt} och api-dokumentation här: \\
% \url{http://www.scala-sbt.org/1.x/docs/Howto-Scaladoc.html}


%\end{itemize}

% \clearpage

% \section{javadoc}
% \newcommand{\javadoc}{\texttt{javadoc}}


% Med Java JDK följer dokumentationsverktyget \javadoc, som utifrån dokumentationskommentarer i Java-kod genererar en webbsajt med navigationslänkar. Webbsidor genererade med \javadoc\ erbjuder inte samma funktioner för sökning och filtrering som \scaladoc, men det fungerar bra att hitta det man söker om navigationslänkarna används tillsammans med webbläsarens inbyggda sökfunktion (Ctrl+F).

% \subsection{Använda dokumentation genererad med javadoc}

% I figur \ref{fig:javadoc:overview} visas exempel på \javadoc\ för biblioteket \code{cslib}. Om du klickar på ett paket kan du navigera till en översikt av innehållet i paketet. Om du klickar på en klass får du en översikt av klassens medlemmar, så som visas i \ref{fig:javadoc:class}.  Om du t.ex. klickar på ett metodnamn får du se mer detaljerade kommentarer.

% Ramarna till vänster på webbsidorna innehåller länkar till paket och klasser. Om du klickar på länken \textit{All Classes} överst till vänster får du en lista med navigationslänkar till alla tillgängliga klasser. De gulmarkerade rubrikerna visar vilken vy som är aktiv och navigationslänkar skrivs med blå text.

% \begin{figure}
% \includegraphics[width=1.0\textwidth]{../img/javadoc/javadoc-overview}
%     \caption{Delar av en webbsida genererad med hjälp av \javadoc.}
%     \label{fig:javadoc:overview}
% \end{figure}



% \begin{figure}
% \includegraphics[width=1.0\textwidth]{../img/javadoc/javadoc-class}
%     \caption{Delar av en webbsida med klassdokumentation genererad med hjälp av \javadoc.}
%     \label{fig:javadoc:class}
% \end{figure}






% \subsection{Skriva dokumentationskommentarer för javadoc}

% Kommentarer för \javadoc\ och \scaladoc\ ser ganska lika ut, även om det finns några skillnader. Det finns t.ex. inte lika många styrtecken för layouten i \javadoc\ som i \scaladoc, och konventionen i Java är fyra blankstegs indrag och att fortsättningsrader i dokumentationskommentarer börjar asterisken under \textit{första} asterisken i öppningskommentaren.

% Nedan visas delar av \javadoc-kommentarerna för klassen \code{SimpleWindow} och dess konstruktor:
% \begin{Code}[language=Java]
% package cslib.window;

% /** A simple window to draw in */
% public class SimpleWindow {
%    /**
%     * Creates a window and makes it visible.
%     *
%     * @param width   the width of the window
%     * @param height  the height of the window
%     * @param title   the title of the window
%     */
%     public SimpleWindow(int width, int height, String title) {
%         ...
% \end{Code}
% Annoteringen \verb|@param| i början på en rad ger en speciell kommentar angående en parameter. Vid dokumentation av metoder kan annoteringen \verb|@return| användas i början av en rad för att skapa en speciell kommentar angående vad som returneras.

% Övergripande dokumentation om innehållet i ett paket läggs i en textfil i paketets katalog med namnet \texttt{package-info.java}, se till exempel här: \\ {\href{https://github.com/lunduniversity/introprog/tree/master/workspace/cslib/src/main/java/cslib/window}{\small github.com/lunduniversity/introprog/tree/master/workspace/cslib/src/main/java/cslib/window}


% Du kan läsa mer om hur man skriver \code{javadoc}-kommentarer här:\\
% \href{http://www.oracle.com/technetwork/java/javase/documentation/index-137868.html}{www.oracle.com/technetwork/java/javase/documentation/index-137868.html}

% % conversation of diffs between javadoc and scaladoc:
% %  https://groups.google.com/forum/#!msg/scala-user/q-Vw03zcIVs/CaTR5XL-BQAJ

% \subsection{Generera dokumentationskommentarer för javadoc}

% Om du står i den katalog där din källkod finns, kan du med nedan kommando i terminalen gå igenom alla paket och underpaket och generera \javadoc-webbsidor i katalogen \texttt{doc}. Du kan därefter öppna dokumentationen i en webbläsare.
% \begin{REPLnonum}[basicstyle=\color{white}\ttfamily\fontsize{9}{11}\selectfont]
% $ javadoc -d doc -encoding UTF-8 -charset UTF-8 -sourcepath . -subpackages . *
% $ firefox doc/index.html
% \end{REPLnonum}
% Ett smidigt sätt att generera både \scaladoc\ och \javadoc\ är att använda \texttt{sbt}; det är bara att skriva \code{ sbt doc } i terminalen så genereras alla dokumentation för både Scala och Java i den katalog som \texttt{sbt} meddelar i sin resultatutskrift.

% Om du lägger in nedan i \code{settings} i din \code{build.sbt} fungerar även svenska bokstäver och andra specialtecken på alla plattformar.
% \begin{Code}
%   javacOptions in (Compile, doc) ++= Seq(
%     "-encoding", "UTF-8",
%     "-charset", "UTF-8",
%     "-docencoding", "UTF-8")
% \end{Code}

% \noindent Du kan också använda din IDE för att köra \javadoc. I Eclipse, använd menyn \MenuArrow{Project}\Menu{Generate Javadoc...}, medan du i IntelliJ hittar motsvarande i menyn \MenuArrow{Tools}\Menu{Generate Javadoc...}

%!TEX encoding = UTF-8 Unicode
%!TEX root = ../compendium2.tex

\newcommand{\sbt}{\texttt{sbt}}

\chapter{Byggverktyg}\label{appendix:build}

\section{Vad gör ett byggverktyg?}

Ett \textbf{byggverktyg} \Eng{build tool} används för att
\begin{itemize}
\item ladda ner,
\item kompilera,
\item testköra,
\item paketera och
\item distribuera
\end{itemize}
programvara. Ett stort utvecklingsprojekt kan innehålla många hundra kodfiler och under utvecklingens gång vill man kontinuerligt testköra systemet för att kontrollera att allt fortfarande fungerar; även den kod som inte ändrats, men som kanske ändå påverkas av ändringen. Ett byggverktyg används för att \textit{automatisera} denna process.

Ett viktigt begrepp i byggsammanhang är \textbf{beroende} \Eng{dependency}. Om koden X behöver annan kod Y för att fungera, sägs kod X ha ett beroende till kod Y.

I konfigurationsfiler, som är skrivna i ett format som byggverktyget kan läsa, specificeras de beroenden som finns mellan olika koddelar. Byggverktyget analyserar dessa beroenden och, baserat på ändringstidsmarkeringar för kodfilerna, avgör byggverktyget vilken delmängd av kodfilerna som behöver \textbf{omkompileras} efter en ändring. Detta snabbar upp kompileringen avsevärt jämfört med en total omkompilering från grunden, som för ett stort projekt kan ta många minuter eller till och med timmar. Efter omkompilering av det som ändrats, kan byggverktyget instrueras att köra igenom testprogram och rapportera om testernas utfall, men även ladda upp körbara programpaket till t.ex. en webbserver.


En vanlig typ av beroende är färdiga programbibliotek som utnyttjas av systemet under utveckling, vilket i praktiken ofta innebär att en sökväg till den kompilerade koden för programbiblioteket behöver göras tillgänglig. I JVM-sammanhang innebär detta att sökvägen till alla nödvändiga jar-filer behöver finnas på sökvägslistan kallad \textbf{classpath}.

Många byggverktyg kan utföra så kallad \textbf{beroendeupplösning} \Eng{dependency resolution}, vilket innebär att nätverket av beroenden analyseras och rätt uppsättning programpaket görs tillgänglig under bygget. Detta kan även innebära att programpaket som är tillgängliga via nätet automatiskt laddas ned inför bygget, t.ex. via lagringsplatser för öppen källkod.

Även om man bara har ett litet kodprojekt med några få kodfiler, är det ändå smidigt att använda ett byggverktyg. Man kan nämligen göra så att byggverktyget är aktivt i bakgrunden och, så fort man sparar en ändring av koden, gör omkompilering och rapporterar eventuella kompileringsfel.

Det är klokt att kompilera om ofta, helst vid varje liten ändring, och rätta eventuella fel \textit{innan} nya ändringar görs, eftersom det är mycket lättare att klura ut ett enskilt problem efter en mindre ändring, än att åtgärda en massa svåra följdfel, som beror på en sekvens av omfattande ändringar, där misstaget begicks någon gång långt tidigare.

En integrerad utvecklingsmiljö, så som VS Code eller IntelliJ IDEA, bygger om koden kontinuerligt och kan ofta konfigureras att kommunicera med flera olika byggverktyg. Exempelvis kan du med VS Code välja om du vill att Scala CLI eller \sbt~ ska användas för att bygga ditt projekt.

Det finns många olika byggverktyg. Några allmänt kända byggverktyg listas nedan så att du ska känna igen vilket byggverktyg som används i öppen-källkods-projekt som du stöter på, t.ex. på GitHub.

\begin{itemize}

\item \texttt{Scala CLI}. Verktyget Scala CLI (Command Line Interface) är öppen källkod utvecklad av VirtusLab\footnote{\url{https://scala-cli.virtuslab.org/}} för att kompilera och köra Scala- och Java-program och innehåller också grundläggande byggverktygsfunktioner, så som att köra testfall, paketera jar-filer och skapa dokumentation. Kommandot \code{scala-cli} övertog år 2023 rollen som det officiella \code{scala}-kommandot. Detta är det enklaste och rekommenderade sättet att bygga system med Scala-kod. Grundläggande användning av Scala CLI beskrivs i Appendix \ref{appendix:compile:scala-cli}, medan en mer utförlig beskrivning återfinns nedan i avsnitt \ref{appendix:build:scala-cli}.

\item \sbt. Även kallad \textit{Scala Build Tool}. Används för att bygga Java- och Scala-program i samexistens, men även för att automatisera en mängd andra saker. \sbt~är utvecklat i Scala och konfigurationsfilerna, som heter \texttt{build.sbt}, innehåller Scala-kod som styr byggprocessen. \sbt~är avancerat och klarar bygga system som består av många projekt \Eng{multi-project build}. \sbt~är det i särklass vanligaste byggverktyget för Scala och många öppen-källkodsprojekt använder \sbt. Läs mer om \sbt~i avsnitt \ref{appendix:build:sbt} nedan.

Efter kritik om att \sbt~är komplicerat så har flera alternativa byggverktyg för Scala utvecklats, däribland \code{bleep}\footnote{\url{https://bleep.build/docs/}} och \code{mill}\footnote{\url{https://mill-build.com/mill/Intro_to_Mill.html}}. 


\item Apache Maven, \texttt{mvn} är också skriven i Java och är en efterföljare till \texttt{ant}. Maven används av många Java-utvecklare. Konfigurationsfilerna heter \texttt{pom.xml} och innehåller en s.k. projektobjektmodell specificerad i XML enligt  speciella regler.

\item \texttt{gradle} bygger vidare på idéerna från \texttt{ant} och \texttt{maven} och är skrivet i Java och Groovy.  Konfigurationsfilerna skrivs i Groovy och heter \texttt{build.gradle}.

\item Apache \texttt{ant}. Detta byggverktyg är utvecklat i Java som ett alternativ till \texttt{make} och används fortfarande i många Java-projekt, även om Maven och Gradle är vanligare numera. Konfigurationsfilerna heter \texttt{build.xml} och skrivs i det standardiserade språket XML enligt  speciella regler.

\item \texttt{make}. Detta anrika byggverktyg har varit med ända sedan 1970-talet och används fortfarande för att bygga många system under Linux, och är populärt vid utveckling med programspråken C och C++. En konfigurationsfil för \texttt{make} heter \texttt{Makefile} och har en egen, speciell syntax.
\end{itemize}

\section{Scala Command Line Interface \texttt{scala-cli}}\label{appendix:build:scala-cli}

Utvecklingen av Scala CLI\footnote{CLI är en förkortning för \textit{command line interface}. Läs mer om Scala CLI här: \url{https://scala-cli.virtuslab.org/docs/overview}} påbörjades 2022 och planeras under 2024 bli det officiella bygg- och körverktyget. Det kommer då medfölja den officiella installationen av Scala via \url{https://www.scala-lang.org}. Scala CLI kan även installeras separat från \url{https://scala-cli.virtuslab.org} och köras med kommandot \code{scala-cli}, men någon gång under 2024 när Scala 3.5 släpps så blir kommandot \code{scala} liktydigt med \code{scala-cli}.\footnote{I skrivande stund har gamla \code{scala}-kommandot ännu inte uppgraderats, men det förväntas ske i början av hösten. När så väl sker kan \code{scala-cli} ersättas med \code{scala} om du uppdaterar till Scala 3.5.0 eller senare.}

Efter nyinstallation av Scala CLI kan du ange följande kommando för att, en gång för alla, få tillgång till kompletteringar av optioner med Tab-tangenten i terminalen:
\begin{REPLsmall}
scala-cli install-completions 
\end{REPLsmall}

Innan du börjar skriva källkod i en ny katalog i VS Code kan du göra VS Code redo för att använda Scala CLI som byggverktyg i aktuell katalog med följande kommando (vänta med att öppna VS Code till efter att du kört kommandot): 
\begin{REPLsmall}
mkdir minNyaKatalog
cd minNyaKatalog
scala-cli setup-ide . 
code .
\end{REPLsmall}
I senare versionerna av VS Code med Metals och Scala 3.4 behövs ej \code{setup-ide} då Metals kör scala-cli som default.

\subsection{Exempel på användning av Scala CLI}\label{appendix:build-scala-cli-watch-mode}

Nedan beskrivs de viktigaste Scala-CLI-kommandona för att stegvis bygga upp din kod med många små steg och experimenterande med dellösningar. 

\begin{itemize}
  \item Med kommandot \code{scala-cli compile . -w} i ett eget terminalfönster bredvid din editor startar du Scala CLI i så kallad \textit{watch mode}. Då bevakas alla filändringar och omkompilering sker direkt när någon ny ändring sparats och du kan se eventuella kompileringsfel direkt. Åtgärda helst ett kompileringsfel innan du bygger vidare på din kod, då följdfel kan vara svåra att lösa speciellt om de är många. Dela upp stora kodändringar i små steg och försök att så fort som möjligt få den senaste ändringen att kompilera felfritt. 
  \item
    Med kommandot \code{scala-cli repl .}~ i ett annat terminalfönster startar du REPL med dina klasser i aktuella katalogen  automatiskt tillgängliga på classpath (därav punkten efter blanksteg efter \code{scala repl}) och du kan anropa dina metoder, efter ev. \code{import} när du gör experiment inför nästa steg. 
    
    På så sätt kan du skapa och testa små funktioner och få dem att att fungera innan inför dem i ditt program och sätter samman dessa med redan skapade funktioner. Det är ofta lättare att felsöka och bygga upp ett större program om du har många små funktioner som samverkar, snarare än få väldigt stora funktioner. Och det är oftast lättare att testköra nya lösningsidéer i REPL innan du skapar ''färdig'' kod i ditt program.
  \item
    Med \texttt{scala-cli run .} sker kompilering och körning av \code{main}-metoden i aktuella katalogen. Du kan ange en annan katalog genom att skicka med sökvägen som argument till kommandot.  Du kan också ange optionen \code{-w} för \textit{watch mode} vid körning. Då kommer ditt program att köras om vid varje ändring. Watch mode vid körning är användbart om programmet ger resultat utan att vänta på input, men inte så smidigt om varje körning kräver att användaren skriver indata eller om fönster måste stängas.
  \item
    Om det finns flera \texttt{main}-metoder i aktuella katalogen, går det att specificera vilken av dessa som ska exekveras med optionen \verb|--main-class|, exempelvis\\ \verb|scala-cli run .  --main-class hello| 
  \item Argument till ditt huvudprogam anges efter dubbla minustecken \verb|--| så här: \\\verb|scala-cli run . -- arg1 arg2 arg3|
\end{itemize}



\subsection{Grundläggande byggfunktioner i Scala CLI}

De grundläggande funktionerna sammanfattas nedan (se även Appendix \ref{appendix:compile:scala-cli}):

\begin{table}[H]
\begin{tabular}{l p{6.5cm}}
\texttt{scala-cli repl} & Starta Scala REPL.  Det går även bra med enbart \texttt{scala-cli}\\
\texttt{scala-cli repl hello.scala} & Starta Scala REPL med kompilerade koden i \texttt{hello.scala} på classpath.  \\
\texttt{scala-cli repl .} & Starta repl med kodfiler i aktuell katalog tillgängliga på classpath. \\
\texttt{scala-cli compile hello.scala} & Kompilera koden i filen \texttt{hello.scala}  \\
\texttt{scala-cli compile .} & Kompilera alla kodfiler i aktuell katalog. \\
\texttt{scala-cli run hello.scala} & Kompilera koden i filen \texttt{hello.scala} och kör igång eventuellt huvudprogram om kompileringen gick bra. \\
\texttt{scala-cli run .} & Kompilera och kör alla kodfiler i aktuell katalog. \\
\texttt{scala-cli run . -{}-list-main-class} & Lista alla huvudprogram. \\
\texttt{scala-cli run . -M mypkg.myMain} & Kör ett specifikt huvudprogram. Förk. \texttt{-M} kan även skrivas \texttt{-{}-main-class}\\
\texttt{scala-cli package .} & Paketera all kompilerade kodfiler i en körbar fil. \\
\\
\end{tabular}
\end{table}

\subsection{Använda optioner för att styra Scala CLI}

\noindent Det finns en mängd olika optioner som du kan lägga till för att styra vad Scala CLI ska göra. Se exempel nedan och förklaringar i efterföljande tabell:
\begin{REPLsmall}
scala-cli run . -S 3.3 -O -unchecked --dep se.lth.cs::introprog::1.3.1 -w
\end{REPLsmall}

\noindent Här förklaras några vanliga optioner som kan användas vid både kompilering och exekvering: 
\begin{table}[H]
\begin{tabular}{l p{6.5cm}}
\texttt{-{}-scala 3.3} & Använd version 3.3 av Scala. Optionen \texttt{-{}-scala} kan förkortas med \texttt{-S} \\
\texttt{-{}-watch} & Upprepa kommando vid sparad ändring. Optionen \texttt{-{}-watch} förkortas med \texttt{-w} \\
\texttt{-{}-jar introprog.jar} & Lägg till en jar-fil på classpath. \\
\texttt{-{}-dep se.lth.cs::introprog::1.3.1} & Lägg till ett beroende på classpath. \\
\texttt{-{}-scalac-option -unchecked} & Lägg till en kompilator-option som ger extra varningar vid osäker kod.  Optionen \texttt{-{}-scalac-option} kan förkortas med \texttt{-O}\\
\end{tabular}
\end{table}

\noindent Fördelen med att explicit ange en viss Scala-version är att byggprocessen blir \emph{upprepningsbar} även på en annan dator som kanske råkar har en annan Scala-version installerad. Det går att ''spika fast'' Scala-versionen till en ännu mer precis version, t.ex. \code{3.2.2}. Om inte den versionen av kompilatorn finns installerad på datorn så kommer Scala CLI att automatiskt ladda ner och använda den explicit efterfrågade versionen under byggandet.

Det går också att be om den absolut mest rykande färska kompilatorversionen om man vill använda det allra senaste i Scala-språkets utveckling med \code{3.nigthly}. Speciellt kräver s.k. experimentella funktioner att du använder \code{nightly}-versionen\footnote{\url{https://stackoverflow.com/questions/40622878/how-do-i-use-a-nightly-build-of-scala}}. 

\subsection{Generera dokumentation med Scala CLI}

Scala CLI kan också skapa dokumentation baserat på dokumentationskommentarer (se vidare Appendix \ref{appendix:doc}), enligt nedan. Med optionen \code{--ouput} kan du ange destinationskatalog och med \code{--force} så skrivs ev. gammal dokumentation över.
\begin{REPLsmall}
scala-cli doc . --output apidoc --force
\end{REPLsmall}

\subsection{Paketering av exekverbar fil med Scala CLI}

\noindent Scala CLI kan paketera din kod i en exekverbar fil så här:
\begin{REPLsmall}
scala-cli package . --force --standalone --output myapp
\end{REPLsmall}

\noindent Här förklaras några vanliga optioner som kan användas vid paketering: 
\begin{table}[H]
\begin{tabular}{l p{8.5cm}}
\texttt{-{}-output} & Ange namn på utfilen med paketerad kod. Optionen \texttt{-{}-output} kan förkortas med \texttt{-o}\\
\texttt{-{}-force} & Skriv över utfilen om den redan finns. Optionen \texttt{-{}-force} kan förkortas med \texttt{-f} \\
\texttt{-{}-standalone} & Skapa en självständig, exekverbar jar-fil med din kod och dess beroenden.\\
\texttt{-{}-library} & Skapa en jar-fil med din kod för användning av andra program.\\
\texttt{-{}-assembly} & Skapa en fet jar-fil med din kod och alla dess beroenden för användning av andra program.\\
\end{tabular}
\end{table}

\subsection{Optioner som användningsdirektiv i ''magiska'' kommentarer}

\noindent I stället för att använda optioner i terminalen så kan du ge dessa som s.k. användningsdirektiv \Eng{using directives} i ''magiska'' kommentarer som börjar med \code{//> using} i början av valfri kod-fil. 

Om du har flera kodfiler i samma katalog brukar man skapa en speciell fil som vanligtvis kallas \code{project.scala} och i den samla alla användningsdirektiv som styr byggandet. Här visas ett exempel hur det kan se ut:
\begin{Code}
//> using scala 3.3
//> using option -unchecked -deprecation 
//> using option -Wunused:all -Wvalue-discard -Wsafe-init
//> using dep se.lth.cs::introprog::1.3.1
\end{Code}

\noindent De kompilatoroptioner som föreslås för att få extra varningar ovan har följande betydelser:
\begin{table}[H]
\begin{tabular}{l p{8.5cm}}
\texttt{-unchecked} & Extra varningar vid flera fall av osäker kod. \\
\texttt{-deprecation} & Förklaring vid användning av utgående funktioner. \\
\texttt{-Wunused:all} & Varning om deklarationer ej används. \\
\texttt{-Wvalue-discard} & Varning vid förlorat värde. \\
\texttt{-Wsafe-init} & Varna vid risk för ej initialiserade värden. \\
\end{tabular}
\end{table}

\noindent Du hittar mer information om Scala CLI här: \url{https://scala-cli.virtuslab.org/}



\section{Scala Build Tool \texttt{sbt}}\label{appendix:build:sbt}

Byggverktyget \sbt\ är skrivet i Scala och är det mest använda byggverktyget bland Scala-utvecklare. Med \sbt\ kan du skriva byggkonfigurationsfiler i Scala och även styra byggprocessen via ett interaktivt kommandoskal i terminalfönstret. Med inkrementell (stegvis) kompilering och parallellkörning av byggprocessens olika delar, kan den snabbas upp avsevärt.


\subsection{Installera sbt}

\sbt\ finns förinstallerat på LTH:s datorer och körs igång med kommandot \sbt\ i terminalen.

Om du vill installera \sbt\ på din egen dator,
säkerställ först att du har \code{java} på din dator med terminalkommandot \code{java -version}. Om \code{java} saknas, följ instruktionerna i avsnitt \ref{appendix:compile:install-jdk} på sidan \pageref{appendix:compile:install-jdk}.
Följ sedan instruktionerna här för att installera \sbt: \url{http://www.scala-sbt.org/download.html}

\begin{itemize}

\item \textbf{Linux}. Om du surfar till ovan sida från en Linux-dator syns några terminalkommando som du använder för att installera \sbt\ i terminalen.

\item \textbf{Windows}. Om du surfar till ovan sida från en Windows-dator visas en länk till en \code{.msi}-fil. Ladda ner och dubbelklicka på den. Innan du kör igång med sbt i en Windowsterminal är det bra att skriva \code{chcp 65001} för att särskilda tecken (t.ex. ÅÄÖ) ska fungera som de ska.

\item \textbf{macOS}. Följ instruktionerna under rubriken \textit{Manual Installation}.

\end{itemize}

\noindent När du kör sbt första gången kommer ytterligare filer att laddas ner och installeras och delar av denna process kan ta lång tid. Ha tålamod och avbryt inte körningen, även om inget speciellt ser ut att hända på ett bra tag.

%% Below is problematic for some libs noty compiled for 2.11.x as it causes dependency problems...
%\subsection{Anpassa sbt}
%För att följa de versioner av \sbt\ och Scala som vi använder i kursen, skapa med hjälp av editor en textfil med namnet \code{global.sbt} i katalogen \code{.sbt} som ligger i din hemkatalog efter att du installerat klart \sbt. Fråga vid behov någon om hjälp om hur man hittar dolda filer i ditt operativsystem, då filer som börjar med punkt ibland inte syns i filbläddraren. Filen ska ha följande innehåll:
%\begin{Code}
%scalaVersion := "2.11.8"
%
%sbtVersion := "0.13.12"
%\end{Code}
%
%\noindent När du kör igång \sbt\ igen kommer ovan inställningar eventuellt medföra vissa nedladdningar, men när det är gjort har du rätt versioner tillgängliga och \sbt\ kommer att starta snabbt nästa gång.


\subsection{Använda sbt}
\sbt\ är konstruerat för att klara mycket stora projekt, men det är enkelt att använda \sbt\ även om du bara har ett litet projekt med någon enstaka kodfil. Med \sbt\ installerat, är det bara att köra igång \sbt och skriva \texttt{run} enligt nedan
\begin{REPLnonum}
> sbt
sbt> run
\end{REPLnonum}
i terminalen i den katalog där dina kodfiler ligger. \sbt\ letar då upp och kompilerar alla de \code{.scala}-filer som ligger i katalogen och, om det bara finns ett objekt med main-metod, kör \sbt\ igång denna main-metod direkt, förutsatt att kompileringen kan avlutas utan fel. Även \code{.java}-filer kompileras automatiskt om de ligger i samma katalog.

Om du enbart skriver \sbt\ körs det interaktiva kommandoskalet igång, där du kan köra kommando så som \code{compile} och \code{run}. Om du skriver ett \code{~} före kommandot \code{run}, enligt nedan kommer \sbt\ vara aktivt i bakgrunden medan du editerar och så fort du sparar en ändring kommer omkompilering av ändrade kodfiler ske, varefter main-metoden exekveras om kompileringen lyckades.

\begin{REPLnonum}
> sbt
[info] Set current project to hello (in build file:/home/bjornr/hello/)
> ~run
[info] Running hello
Hello, World!
[success] Total time: 0 s, completed Aug 9, 2016 9:50:16 PM
1. Waiting for source changes... (press enter to interrupt)
[info] Compiling 1 Scala source to /home/bjornr/hello/target/scala-2.10/classes...
[info] Running hello
Hello again, World!
[success] Total time: 1 s, completed Aug 9, 2016 9:50:45 PM
2. Waiting for source changes... (press enter to interrupt)
\end{REPLnonum}

\noindent I ovan körning gör \sbt\ en omkompilering, efter att en ändring av utskriftssträngen sparats.
\begin{figure}[H]
\begin{Code}
// in file hello.scala

@main def run = println("Hello again, World!") // add 'again'; Ctrl+S
\end{Code}
\end{figure}

\subsubsection{Katalogstruktur}

Om man har kod i underkataloger förutsätter \sbt\ att du följer en viss, specifik katalogstruktur. Denna katalogstruktur används även av andra byggverktyg, så som Maven, och fungerar även i många utvecklingsmiljöer så som Eclipse och IntelliJ.

Det blir också mindre rörigt och lättare för alla att hitta i projektets kataloger om dina kodfiler placeras i en given struktur som är allmänt accepterad.
Placera därför gärna dina kodfiler i underkataloger enligt strukturen som visas i figur \ref{fig:sbt:dir-structure}.

\begin{figure}[H]
\centering

\begin{lstlisting}[frame=none, backgroundcolor=]
					src/
					  main/
					    resources/
					       <files to include in main jar here>
					    scala/
					       <main Scala sources>
					    java/
					       <main Java sources>
					  test/
					    resources
					       <files to include in test jar here>
					    scala/
					       <test Scala sources>
					    java/
					       <test Java sources>
\end{lstlisting}

\caption{Katalogstrukturen i ett \sbt-projekt. Bara de kataloger som har något innehåll behöver finnas.}
\label{fig:sbt:dir-structure}
\end{figure}

\noindent Lägg enligt denna struktur dina \code{.scala}-filer i underkatalogen \code{src/main/scala/} och dina \code{.java}-filer i underkatalogen \code{src/main/java/}. Om du lägger kod i någon av katalogerna \code{src/test/scala/} respektive \code{src/test/java/} kommer denna kod köras när du skriver \sbt-kommandot \code{test}. Om du lägger filer i underkatalogen \code{src/main/resources/} kommer dessa att paketeras med i jar-filen som skapas när du kör \sbt-kommandot \texttt{package}.

Om du använder t.ex. \code{package x.y.z;} i din Java-kod, måste även strukturer på underkataloger matcha och kodfilen alltså ligga i  \code{src/main/java/x/y/z/}.

I Scala är det egentligen inte nödvändigt att koden ligger i samma katalog som de kompilerade \texttt{.class}-filerna, men det kan vara bra att följa paketstrukturen även för Scala-källkoden; speciellt om du senare vill kunna köra din kod med Eclipse, som kräver denna överensstämmelse mellan paket och källkodskataloger, inte bara för Java, utan även för Scala.

\subsubsection{Konfigurera dina byggen i filen \code{build.sbt}}

Om du vill göra inställningar och även hjälpa andra att kunna återskapa dina byggen, så skapa en konfigurationsfil med namnet \code{build.sbt} och placera den i projektets baskatalog. Figur \ref{fig:sbt:build-file} visar en byggkonfigurationsfil som specificerar vilken version av Scala-kompilatorn du använder, så att andra ska kunna bygga din kod under samma förutsättningar som du.

\begin{figure}[H]
\centering
\begin{Code}
scalaVersion := "3.2.2"
\end{Code}
\caption{Exempel på konfigurationsfil för \sbt. Filen ska ha namnet \code{build.sbt} och vara placerad i projektets baskatalog.}
\label{fig:sbt:build-file}
\end{figure}

\noindent Här är ett exempel på en mer omfattande \code{build.sbt}:
\begin{CodeSmall}
scalaVersion   := "3.2.2"
scalacOptions  := Seq("-unchecked", "-deprecation") //mer info vid kompilering

fork           := true   // kör i en egen JVM, bra om ljud och grafik används 
connectInput   := true   // koppla indata till rätt JVM vid fork
outputStrategy := Some(StdoutOutput)  // koppla utdata till rätt JVM vid fork

ThisBuild / useSuperShell := false // stänger av rör(l)ig progressinformation
\end{CodeSmall}

\noindent Du kan läsa mer om alla möjligheter med \sbt\ och hur man skapar mer avancerade byggkonfigurationsfiler här: \url{http://www.scala-sbt.org}

% Du hittar ett exempel på en avancerad byggdefinition i kursens repo, som har många aggregerade underprojekt, bl.a. för att bygga detta kompedium med \code{pdflatex}. I byggdefinitionen instrueras även \sbt\ att bygga kursens workspace, samt att generera de speciella projektfiler som Eclipse+ScalaIDE kräver med en \sbt-plugin. Filen finns här: \\
% \url{https://github.com/lunduniversity/introprog/blob/master/build.sbt}

\subsubsection{Fixera versionen för sbt i \code{project/build.properties}}
Om du skapar en katalog \code{project} (om den inte redan finns) kan du i en fil med namnet \code{build.properties} fixera versionen av sbt genom att låta filen ha detta innehåll (notera punkten och avsaknaden av citationstecken):
\begin{Code}
sbt.version=1.8.2
\end{Code}
På så sätt riskerar du inga inkonsekvenser mellan en gammal \code{build.sbt} vid framtida uppdatering av sbt, ovan inställning garanterar att ditt bygge alltid kommer att byggas med denna version av sbt, och andra kan bygga din kod under samma förutsättningar som du.

\subsubsection{Lägga till kursbiblioteket \texttt{introprog} som ett beroende}

Med följande text i \code{build.sbt} får du automatisk nedladdning och tillgång till kursens Scala-bibliotek \texttt{introprog} med bl.a. klassen \code{PixelWindow} för grafiska fönster:

\begin{Code}
scalaVersion := "3.2.2"
libraryDependencies += "se.lth.cs" %% "introprog" % "1.3.1"
\end{Code}
Ändra ev. versionsnummer till senaste versionen. Notera de dubbla procent-tecknen före biblioteksnamnet, som används för Scala-bibliotek som kors-publicerats för olika versioner av Scala, t.ex. 3, 2.12 och 2.13, vilket gör att rätt biblioteksversion för rätt kompilatorversion laddas ned.

Du kan läsa mer om \code{introprog} här: 
\begin{itemize}
  \item Kod: \url{https://github.com/lunduniversity/introprog-scalalib}
	\item Dokumentation: \url{http://cs.lth.se/pgk/api}
\end{itemize}


\subsubsection{Lägga till andra beroenden}

I filen \texttt{build.sbt} kan man lägga till många beroenden till flera olika kodbibliotek. Det finns på nätplatsen \textit{Maven Central} en mycket omfattande koddatabas, som är sökbar här \url{http://search.maven.org}, med en massa användbara öppenkällkodsprojekt. Du kan be \sbt\ att ladda ner den färdigkompilerade koden till vilket som helst av projekten på \textit{Maven Central} och automatiskt lägga till jar-filen till \code{classpath} så att koden blir tillgänglig direkt i ditt program.

Till exempel kan du lägga till Java-biblioteket \code{jline} som gör det möjligt att göra terminalinläsning från tangentbordet med många bra finesser, t.ex. kommandohistorik med pil-upp, bara genom att lägga till denna rad i din \code{build.sbt} och den specifika version du önskar (notera enkla procent-tecken för Java-bibliotek):

\begin{Code}
libraryDependencies += "org.jline" % "jline" % "3.20.0"
\end{Code}
Du kan läsa mer om \code{jline} här: \url{https://jline.github.io/}


%!TEX encoding = UTF-8 Unicode
%!TEX root = ../compendium2.tex


\chapter{Versionshantering och kodlagring}

\section{Vad är versionshantering?}

\textbf{Versionshantering}\footnote{\href{https://en.wikipedia.org/wiki/Version_control}{en.wikipedia.org/wiki/Version\_control}} \Eng{version control \textup{eller} revision control} av mjukvara innebär att hålla koll på olika versioner av koden i ett utvecklingsprojekt allteftersom koden ändras. Versionshantering är en deldisciplin inom \textbf{konfigurationshantering} \Eng{software configuration management} som inbegriper allt i processen för att identifiera, besluta, genomföra och följa upp ändringar.

En viktig del av versionshantering är att \textit{lagra} olika versioner av koden allt eftersom den utvecklas, så att tidigare versioner kan \textit{återskapas} vid behov. Ett bra verktygsstöd och en väldefinierad arbetsprocess för versionshanteringen, som alla i utvecklingsprojektet följer, möjliggör att flera utvecklare kan \textit{arbeta parallellt} med att sammanfoga \Eng{merge} varandras tillägg och ändringar i den gemensamma kodbasen utan att det blir kaos och förvirring.

God versionshantering är helt avgörande för utvecklarnas produktivitet, speciellt för stora projekt med många utvecklare som jobbar parallellt mot en omfattande kodbas med många olika interna och externa komponenter. 
Men även ett litet projekt med en enda utvecklare kan ha god nytta av ett versionshanteringsverktyg och ett disciplinerat förfarande för att namnge versioner, t.ex. för att kunna återskapa tidigare versioner av projektets olika kodfiler när en ändring visar sig mindre lyckad.   

Det finns flera olika modeller för hur kodlagringen sker:
\begin{itemize}
\item \textbf{lokal}; alla utvecklare jobbar i samma, lokala filsystem där alla olika versioner lagras.
\item \textbf{centraliserad}; ett repositorium (förk. repo), alltså en databas med koden, finns centralt på en server som alla jobbar mot med hjälp av en versionshanteringsklient.
\item \textbf{distribuerad}; alla utvecklare har sitt eget lokala repo och varje utvecklare initierar enskilt delning av ändringar mellan olika repo. 
\end{itemize}


\section{Versionshanteringsverktyget Git}

Det finns många olika versionshanteringsverktyg\footnote{\href{https://en.wikipedia.org/wiki/List_of_version_control_software}{https://en.wikipedia.org/wiki/List\_of\_version\_control\_software}}
 som använder olika modeller för kodlagring; lokal, centraliserad, distribuerad eller kombinationer därav. 
På senare tid har verktyget \textbf{Git}\footnote{\href{https://en.wikipedia.org/wiki/Git_(software)}{https://en.wikipedia.org/wiki/Git\_(software)}} fått en stark ställning, speciellt i öppenkällkodsvärlden. Git utvecklades ursprungligen av Linus Torvalds för att versionshantera Linuxkärnan, men har växt till ett omfattande öppenkällkodsprojekt med stor spridning och många användare och bidragsgivare. 

Git är skapad för \textbf{distribuerad} versionshantering där var och en kan jobba snabbt och smidigt i sitt eget lokala repo, utan att behöva vänta på att en klient ska synkronisera koden med ett centralt repo på en server över nätverket. Ändringar delas mellan repo på begäran av enskilda utvecklare. 

Varje ny version av koden lagras som en avgränsad mängd ändringar sedan förra versionen, en s.k. \textbf{commit}%
\footnote{På svenska kan t.ex. ''inlämning'' användas, men låneordet commit är redan etablerat.}%
, och hanteras internt av Git i en lokal databas i katalogen \code{.git} som ligger överst i din projektkatalog. Genom olika kommandon i terminalen, eller via en klient med ett grafiskt användargränssnitt, kan din kod överföras till och från den lokala koddatabasen, alternativt delas med andra repon via nätet. 

Det finns en välskriven bok kallad \textit{''Pro Git''} som förklarar Git på djupet och är tillgänglig fritt här: 
\url{https://git-scm.com/book/en/v2}.
Läs kapitel 1 och 2 så får du en bra grund att stå på. 

Dessa termer är bra att kunna utantill innan du kör igång med Git:
\newcommand{\TermItem}[3]{\item \textbf{#1} (\textit{substantiv}: #2, \textit{verb}: #3).}
\begin{itemize}

\item \textbf{repo} (\textit{substantiv}: ett repositorium, \textit{eng. a repository}) En koddatabas med ändringshistorik. 

\TermItem{commit}{en inlämning}{att lämna in} 
  En avgränsad mängd nya ändringar lämnas in i det lokala repot. Repots ändringshistorik utgörs av sekvensen av alla inlämningar.

\TermItem{push}{en leverans}{att leverera, att trycka upp} En eller flera inlämningar trycks upp till ett annat repo.

\TermItem{pull}{en hämtning}{att hämta, att dra ner} En eller flera inlämningar dras ner från ett annat repo.

\TermItem{merge}{en sammanfogning}{att sammanfoga} En eller flera inlämningar slås samman till en ny inlämning. 

\item \textbf{merge conflict} (\textit{substantiv}: en sammanfogningskonflikt, \textit{eng. a merge conflict}) Problem vid sammanfogning; ändringar kan inte enkelt sammanfogas på ett entydigt sätt.

\item \textbf{pull request} (förk. PR, \textit{substantiv}: en hämtningsbegäran, \textit{verb}: att begära en hämtning). Utvecklare A ber en annan utvecklare B att hämta en eller flera inlämningar från A:s repo och sammanfoga med B:s repo.

\end{itemize}

\subsection{Installera git}\label{subsection:install-git}

Git finns förinstallerat på LTH:s Linuxdatorer. Du kan kolla om Git redan finns på din maskin genom att skriva \code{git help} i terminalen. 

Det finns bra instruktioner om hur du installerar Git på din egen maskin här: \url{https://git-scm.com/book/en/v2/Getting-Started-Installing-Git}

VS Code har speciellt stöd för git och du kan inne ifrån VS Code göra t.ex. add, commit, push och pull via editorns grafiska gränssnitt. Läs mer här: \url{https://code.visualstudio.com/docs/editor/versioncontrol}

Det finns även många andra grafiska användargränssnitt till git, t.ex. \href{https://desktop.github.com/}{GitHub Desktop (Windows/Mac) eller \href{https://www.gitkraken.com/}{GitKraken (Linux/Windows/Mac)}}. Se fler exempel här: \url{https://git-scm.com/downloads/guis} 

%Om du inte vet vilken du ska välja, prova GitKraken som är gratis (men stängd) och finns för alla plattformar: \url{https://www.gitkraken.com/}.


\subsection{Anpassa Git}

Innan du börjar använda git, konfigurera ditt namn och din email med nedan terminalkommando, där du anger ditt namn i stället för \code{Förnamn Efternamn} och din mejladress i stället för \code{mejladr@plats.se}. Namnet och mejladressen kommer lagras i varje commit som du gör så att det går att se vem som har gjort en given ändring.
\begin{REPLnonum}
> git config --global user.name "Förnamn Efternamn"
> git config --global user.email mejladr@plats.se
\end{REPLnonum}

Läs mer om hur du gör andra inställningar här, t.ex. hur du anger vilken editor som git startar när du ska skriva commit-beskrivningar: \\ \url{https://git-scm.com/book/en/v2/Getting-Started-First-Time-Git-Setup}


\subsection{Använda git}

Nedan listas några vanliga terminalkommandon i Git.

\begin{itemize}[leftmargin=*]

\item Skapa ett repo i en katalog:
\begin{REPLnonum}
> cd myproject
> git init
\end{REPLnonum} 

\item Se vilka filer som ändrats och ännu ej lämnats in:
\begin{REPLnonum}
> git status
> git status -s
\end{REPLnonum} 

\item Se vilka ändringar som gjorts i filer som ännu ej lämnats in:
\begin{REPLnonum}
> git diff 
\end{REPLnonum} 

\item Se vilka inlämningar som finns i ändringshistoriken:
\begin{REPLnonum}
> git log 
> git log --oneline -5
\end{REPLnonum} 

\item Lägg till filer som ska ingå i nästa inlämning och gör sedan inlämningen; ge inlämningen en bra beskrivning som förklarar vad inlämningen omfattar:
\begin{REPLnonum}
> git add *.scala
> git commit -m 'initial project version'
\end{REPLnonum} 

\item Ångra alla tillägg inför inlämning (ändringarna finns kvar och kan läggas till igen om du vill):
\begin{REPLnonum}
> git reset 
\end{REPLnonum} 

\item Du kan skippa de senaste, ännu ej commitade, ändringar i filen \code{filename}, och göra ''\textit{undo}'', med kommandot \code{git checkout} på filen enligt nedan. Gör bara detta om du är helt säker på att du vill ångra dina senaste ändringar.
\\ \mbox{\colorbox{red!30}{VARNING!} Dina senaste ändringar i filen förloras för alltid; kan ej ångras!}   
\begin{REPLnonum}
> git checkout filename 
\end{REPLnonum} 

\item Man vill förhindra versionshantering av vissa filer, t.ex. binärkodsfiler så som \code{.class}-filer och andra genererade filer. Detta gör du genom att skapa en fil med namnet \code{.gitignore} och lägga in filändelser enligt nedan syntax, där \code{**/} avser alla kataloger och underkataloger och \code{*} kan vara vilken början på ett filnamn som helst. Symbolen \code{#} föregår en kommentarsrad.
\begin{Code}[language=]
# this is my .gitignore

# Java / Scala
**/*.class

# Sbt
**/target

\end{Code} 


\end{itemize}
 

\clearpage 
  
\section{Kodlagringsplatser på nätet}\label{section:code-hosting}

Många utvecklare använder kodlagringsplatser på nätet (''i molnet'') \Eng{code hosting} för att underlätta samarbete kring kod och för att dela med sig av öppen källkod. Det finns många olika kodlagringsplatser som kan användas gratis under vissa förutsättningar eller mot betalning med tillhörande extratjänster. 

\begin{oframed}
  \noindent \textbf{OBS!} Du får \emph{inte} lagra dina lösningar på kursens laborationer i ett öppet repo. Om du vill använda en kodlagringsplats måste du säkerställa att dina lösningar förblir i ett stängt repo utan att någon annan kan komma åt det.
\end{oframed}

Nedan beskrivs några vanliga nätplatser för öppen och sluten kodlagring, som alla är Git-baserade:

\begin{itemize}
\item  \textbf{GitHub}, \url{https://github.com}, är en av de mest populära kodlagringsplatserna för öppen källkod, men har även blivit en populär plats för jobbsökande utvecklare att visa upp sina  kodarbetsprover för framtida arbetsgivare. GitHub är gratis att använda för dig som privatperson. Många företag betalar GitHub för att lagra sin stängda kod med tilläggstjänster för att testa, bygga och driftsätta kod etc. Koden som styr själva kodlagringsplatsen GitHub är stängd, till skillnad från GitLab. GitHub köptes \href{https://computersweden.idg.se/2.2683/1.703485/microsoft-kop-github}{2018} av Microsoft för 65 miljarder kronor.

\item \textbf{GitLab}, \url{https://gitlab.com}, erbjuder gratis kodlagring för öppen källkod, men det är även gratis för privatpersoner och gemenskapsprojekt att ha stängda repo. Företag kan betala för stängd kodlagring med extratjänster för att testa, bygga och driftsätta kod etc. GitLab är i sig ett öppenkällkodsprojekt och koden som styr kodlagringsplatsen är öppen och fri. Detta innebär att du själv kan ladda ner koden och starta en kodlagringsplats. LTH har en GitLab-baserad kodlagringsplats här: \url{https://git.cs.lth.se}

\item \textbf{BitBucket}, \url{https://bitbucket.org}, är en populär kodlagringsplats både för öppen och stängd källkod och drivs av det australiensiska företaget Atlassian. Det är gratis för privatpersoner och små team att ha både öppna och slutna repon, men bara om det är få bidragsgivare. Kostnader tillkommer om antalet bidragsgivare kommer över en viss nivå. Universitetsanställda och studenter kan få mer gynnsamma villkor efter ansökan. Atlassian erbjuder en hel verktygssvit för att hantera buggar och samarbeta över nätet. BitBucket stödjer, förutom Git, även andra versionshanteringsverktyg.

\end{itemize}

\subsubsection{Använda kodlagringsplatser}

Om du inte redan gjort det är det bra om du registrerar ditt användarnamn, förslagsvis \code{fornamnefternamn} som ett ord utan svenska tecken med små bokstäver, på någon eller alla av ovan sajter, dels för att paxa ditt namn och dels för börja samarbeta med utvecklare världen över. Det är bra att välja \textit{ett} användarnamn för \textit{alla} kodlagringsplatser på nätet, speciellt om du jobbar med öppen källkod där ditt namn kommer associeras med alla de kodbidrag du gör under ditt yrkesliv.

Om du inte vet vilken sajt du ska välja, börja med \url{https://github.com}. Om du vill att även kodlagringssajten ska drivas av öppen källkod, testa \url{https://gitlab.com}.

Med en Git-baserad kodlagringsplats får du möjlighet att synka ditt lokala repo mot en server på nätet med hjälp av \code{git}-kommandon i terminalen eller via en Git-klient med grafiskt användargränssnitt, se avsnitt \ref{subsection:install-git}. 

Innan du börjar använda en kodlagringsplats är det bra att sätta sig in i begreppen nedan.

\begin{itemize}
\TermItem{clone}{en klon är kopia av ett (nätlagrat) repo}{att klona, att skapa en kopia} Genom att klona ett repo som ligger på en nätlagringsplats kan du bygga, undersöka och vidareutveckla koden lokalt på din dator. Om du har rättigheter att lämna in kod till det centrala originalet kan du pusha dina commits direkt via terminalkommando eller Git-klient.

\TermItem{fork}{en förgrening av ett helt repo}{att förgrena ett repo, att ''forka''} Genom att förgrena ett repo skapar du en kopia, normalt även den nätlagrad på en kodlagringsplats, som du kan utveckla separat från originalet. Det blir då möjligt för dig att lämna in ändringar och trycka upp dem, även om du inte har rättigheter att leverera (''pusha'') till originalet. Gör en ändringsbegäran (Pull Request, PR) om du vill bidra med dina ändringar, så kan ägaren av originalet sedan välja att sammanfoga (''merga'') dina ändringar med originalet. Många nätlagringsplatser, så som GitHub, har en speciell knapp som du trycker på för att enkelt skapa en fork av ett repo under din användare. 

\item \textbf{upstream} (\textit{preposition}: uppströms, \textit{substantiv}: uppströmsrepo) Ett uppströmsrepo utgör original till ett förgrenat repo (en ''fork''). 
\begin{itemize}[noitemsep,nolistsep]

\item Här beskrivs hur du länkar en förgrening uppströms: \\ 
{\small\url{https://help.github.com/articles/configuring-a-remote-for-a-fork/}}

\item Här beskrivs hur du synkar en förgrening uppströms:\\
{\small\url{https://help.github.com/articles/syncing-a-fork/}}

\end{itemize}

\end{itemize}

Om du vill bidra till ett öppenkällkodsprojekt, börja med att forka repot på kodlagringsplatsen och sedan klona repot till din lokala dator. Därefter kan du commita ändringar och pusha till din fork och slutligen göra en pull request från din fork till upstream. Läs om hur ett bidrag kan gå till i avsnitt \ref{section:OSS-contribution-example}.

Här följer några användbara kommandon:

\begin{itemize}
\item Skapa en lokal kopia av ett fjärran \Eng{remote} repo; här visas hur du klonar kursens repo från GitHub:
\begin{REPLnonum}
$ git clone --depth 1 https://github.com/lunduniversity/introprog
\end{REPLnonum} 

\item Dra ner nya inlämningar från ett fjärran repo:
\begin{REPLnonum}
$ git pull 
\end{REPLnonum} 

\item Trycka upp nya lokala inlämning till ett fjärran repo:
\begin{REPLnonum}
$ git push 
\end{REPLnonum} 

\end{itemize}



%!TEX encoding = UTF-8 Unicode
%!TEX root = ../compendium2.tex

\chapter{Integrerad utvecklingsmiljö}\label{appendix:ide}

\section{Vad är en integrerad utvecklingsmiljö?}

En integrerad utvecklingsmiljö \Eng{integrated development environment, IDE} samlar ett flertal verktyg för att skapa, köra och testa program, inklusive en avancerad \textbf{editor} och speciella felsökningsverktyg. % (se appendix \ref{appendix:compile})
 Det finns flera utvecklingsmiljöer att välja mellan, som kan användas för både Scala och Java.

En IDE ger stöd för \textbf{kodkomplettering} \Eng{code completion} där tillgängliga metoder visas i en lista och resten av ett namn kan fyllas i automatiskt efter att du skrivit de första bokstäverna i namnet. En IDE kan hjälpa dig med formatering och även skapa skelettkod utifrån \textbf{kodmallar} \Eng{code templates}. Med \textbf{felindikering} \Eng{error highlighting} får du understrykning av vissa fel direkt i koden och ibland kan du även få hjälp med förslag på åtgärder för att rätta till enkla fel. Funktioner för \textbf{avlusning} \Eng{debugging} hjälper dig att felsöka medan du kör din kod. Med funktioner för \textbf{omstrukturering} \Eng{refactoring} av kod får du hjälp av editorn i samarbete med kompilatorn att göra omfattande strukturförändringar i många kodfiler samtidigt, t.ex. namnbyten med hänsyn taget till språkets synlighetsregler.

Alla dessa avancerade funktioner kan öka produktiviteten avsevärt, men samtidigt tar de tid att lära sig och en IDE kan kräva mycket datorkraft och viss väntetid jämfört med en vanlig, fristående editor. I början kan all funktionalitet upplevas som överväldigande och det kan vara svårt att hitta i alla menyer och inställningar. Det är därför många som föredrar en fristående, snabbstartad kodeditor före en fullfjädrad, tungrodd IDE, speciellt om det rör ett mindre program. Å andra sidan kan en IDE med kodkomplettering vara till stor hjälp när man ska lära sig ett nytt api och experimentera med en okänd kodmassa. Med tiden har hanliga editorer, så som VS Code, fått allt fler  funktioner som tidigiare bara fanns i en fullfjädrad IDE\footnote{Se t.ex. LSP: \url{https://en.wikipedia.org/wiki/Language_Server_Protocol}}, och den praktiska skillnaden allt mindre mellan en ''vanlig'' editor och en IDE blir allt mindre.

I kursen använder vi flera utvecklingsmiljöer. På första labben använder vi Kojo (se appendix \ref{appendix:kojo}) som är en IDE speciellt anpassad på nybörjare. Sedan använder vi en editor t.ex. VS Code, gärna i kombination med byggverktyget \code{sbt}. Du kan under andra halvan av kursen välja att gå över från VS Code till att använda en (annan) IDE, men det går utmärkt att fortsätta med VS Code som numera har en bra debugger för Scala genom tillägget Metals. Om du vill använda en IDE i stället för VS Code så rekommenderas IntelliJ IDEA med Scala-plugin. 
Om du redan lärt dig Eclipse på djupet och verkligen vill fortsätta med denna IDE, välj då ScalaIDE -- dock har denna IDE inte hängt med i den tekniska utvecklingen och ligger kvar på Scala 2.12. 

%!TEX encoding = UTF-8 Unicode
%!TEX root = ../compendium2.tex

\section{Visual Studio Code med tillägget Scala Metals}\label{appendix:ide:vscode}

Visual Studio Code\footnote{\href{https://en.wikipedia.org/wiki/Visual\_Studio\_Code}{en.wikipedia.org/wiki/Visual\_Studio\_Code}}, förkortat VS Code eller bara \code{code}, är en gratis utvecklingsmiljö som är mestadels öppen källkod\footnote{Varianten VS Codium \url{https://vscodium.com/} är helt fri från stängd källkod och telemetri.}. Projektet startades och leds av Microsoft och har en aktiv gemenskap med många utvecklare och många användbara tillägg \Eng{extensions}.

VS Code kallas ofta för ''bara'' en editor, men har genom åren utvecklats till en fullfjädrad IDE med bl.a. inbyggd debugger och stöd för många olika språk via ett omfattande bibliotek av tillägg.%

\begin{itemize}
\item 
Läs mer om hur man använder VS Code här: \\
\url{https://code.visualstudio.com/docs}

\item
Läs mer om hur du använder Scala i VS Code här: \\
\url{https://scalameta.org/metals/docs/editors/vscode}

\end{itemize}

Det finns många användbara kortkommandon som gör dig snabbare och snabbare när du kodar, allteftersom du lär dig nya kortkommandon. Ett bra tips är att du lär dig minst ett nytt kortkommando om dagen och efter ett tag kan du riktigt många. Här finns en sammanfattning av de viktigaste kortkommandona för VS Code för Linux:\\
\url{https://code.visualstudio.com/shortcuts/keyboard-shortcuts-linux.pdf}\\
Byt ut \code{linux} mot \code{windows} eller \code{macos} i adressen ovan för motsvarande plattform.

\subsection{Installera VS Code och Metals}\label{appendix:ide:vscode:install}

VS Code är förinstallerad på LTH:s datorer, men du behöver själv installera Scala-tillägget \textbf{Metals} första gången du kör igång VS Code på LTH:s datorer. Läs om installation av Metals här: \\
\url{https://marketplace.visualstudio.com/items?itemName=scalameta.metals} 

Läs mer om hur du installerar VS Code på din egen dator här: \\\url{https://code.visualstudio.com}

Mer information om installation av verktyg finns på kursens hemsida: \\
\url{https://lunduniversity.github.io/pgk/#verktyg}

\begin{figure}
\centering
\includegraphics[width=1.0\textwidth]{../img/vscode-run}
\caption{Kör program genom att klicka på \textsf{run} ovanför huvudprogrammet. \label{appendix-ide:vscode-run}}
\end{figure}

\subsection{Köra program i VS Code}

Det finns olika sätt att köra igång huvudprogrammet i ett projekt i VS Code:

\begin{enumerate}
  \item Använd \texttt{scala-cli run .} i ett separat terminalfönster. Läs mer om \texttt{scala-cli} i Appendix \ref{appendix:compile:scala-cli}.
  \item Kör igång \texttt{sbt} i ett separat terminalfönster och kör kommandot \texttt{run} inifrån \texttt{sbt}. Detta kräver att du har en giltig \texttt{build.sbt}, se Appendix \ref{appendix:build}.
  \item Köra igång program inifrån VS Code. Detta kräver att du öppnat katalogen med din kod med File-menyns ''Open Folder'', eller genom att du startar VS Code med \texttt{code .} överst i din projektkatalog (du ser att detta är gjort om nedre meddelandefältet är blått i stället för lila). Du behöver \textit{innan} du startar VS Code en första gång köra \texttt{scala-cli setup-ide .} (se Appendix \ref{appendix:compile:scala-cli}), eller skapa en giltig \code{build.sbt}-fil (se Appendix \ref{appendix:build}) som du importerar i VS Code när frågan dyker upp i nedre högra hörnet.
  \item Kombinera Scala CLI eller \code{sbt} och VS Code. Kör detta kommando i ett separat terminalfönster: \texttt{scala-cli compile . -w}~~ där \texttt{-w} betyder \emph{watch} och gör så att ändringar bevakas. Om du istället använder \texttt{sbt} kör \code{sbt ~compile} i ett separat terminalfönster (notera tilde-tecknet som gör att ändringar bevakas). Vid ändringsbevakning kommer kompileringsfel visas där varje gång du sparar en ändring i VS Code med Ctrl+S. När alla kompileringsfel är åtgärdade och du är redo att testköra så klickar du på \textsf{run}.
\end{enumerate}

Du ser att VS Code är beredd att köra igång ditt program genom att det (efter ett tag) kommer upp en extra rad ovanför ditt huvudprogram med texten \textsf{run|debug} och då kan du klicka på \textsf{run} för att köra ditt program. Utdata från körningen visas i en flik under koden. Observera att det kan ta lite tid för VS Code att förbereda allt som behövs för att kunna köra ditt program. Håll koll på om VS Code håller på med dessa förberedelser i det blåa meddelandefältet längst ned till höger. När allt är klart efter att du startat VS Code står det ''Index complete!'' bredvid en raketsymbol i meddelandefältet.

Om något krånglar och du inte får fram \textsf{run|debug} ovanför din \code{@main}-funktion, trots du har startat VS code enligt ovan, så prova att under Metals-fliken (ikonen med det stiliserade M:et i det gröna verktygsfältet) klicka på någon av ''Restart build server'' eller ''Import build'' (den senare tar längre tid men börjar om helt) och vänta tills det står ''Index Complete!'' i det blå meddelandefältet och då ska \textsf{run|debug} synas ovanför din \code{@main}-funktion.

\begin{figure}
\centering
\includegraphics[width=1.0\textwidth]{../img/vscode-debug}
\caption{Debuggern i VS Code. Nederkanten är orange när debuggern kör. \label{appendix-ide:vscode-debug}}
\end{figure}

\subsection{Använda debuggern i VS Code}

Innan du börjar använda debuggern, läs först om allmän felhantering i Appendix \ref{appendix:debug}.

Du kan aktivera debuggern i VS Code för dina Scala-program genom att klicka på ''debug'' ovanför din \code{main}-metod, förutsatt att du har tillägget Metals installerad i VS Code. Du behöver även köra \texttt{scala-cli setup-ide .} en första gång (se Appendix \ref{appendix:compile:scala-cli}), eller ha en giltig \code{build.sbt}-fil (se Appendix \ref{appendix:build}) som du importerar i VS Code när frågan dyker upp i nedre högra hörnet. 

Figur \ref{appendix-ide:vscode-debug} på sidan \pageref{appendix-ide:vscode-debug} visar hur det kan se ut när debuggern i VS Code är aktiverad. När debuggern är igång får det nedersta meddelandefältet en orange färg (istället för blå). Till vänster om radnummerkolumnen kan du klicka för att aktivera och avaktivera brytpunkter. Aktiverade brytpunkter visas som en röd prick i marginalen till vänster. Den ihåliga gula pilen i marginalen pekar på den rad som kommer att exekveras härnäst. Notera panelen med olika knappar i överkanten av editorfönstret. Med dessa knappar kan du styra exekveringen enligt följande (lär dig gärna kortkommandona så blir du snabbare):
\begin{itemize}
  \item \textbf{Fortsätt}. Den blåa play-knappen kör vidare till nästa brytpunkt eller tills programmet är klart om brytpunkt ej påträffas. Kortkommando ''Continue'': F5.
  \item \textbf{Stega över}.Den blåa böjda framåtpilen kör en rad i taget \emph{utan} att hoppa in i funktioner.  Kortkommando ''Step Over'': F10.
  \item \textbf{Stega in}. Den blåa nedåtpilen kör vidare en rad i taget och hoppar in i funktioner om raden innehåller funktionsanrop. Kortkommando ''Step Into'': F11.
  \item \textbf{Stega ut}. Den blåa uppåtpilen kör klar innevarande funktion. Kortkommando ''Step Out'': Shift+F11.
  \item \textbf{Kör igen}. Den gröna återstartsikonen kör om ditt program. Kortkommando ''Restart'': Ctrl+Shift+F5.
  \item \textbf{Avbryt}. Den röda stoppknappen avbryter denna debuggingsession. Kom ihåg att avbryta innan du startar en ny debuggingsession, annars kan det lätt bli förvirrande med många samtidigt pågående körningar. Kortkommando ''Stop'': Shift+F5. 
\end{itemize}

Figur \ref{appendix-ide:vscode-trace} på sidan \pageref{appendix-ide:vscode-trace} visar hur VS Code presenterar anropsstacken och värdet på de variabler som syns där exekveringen befinner sig för tillfället. Du får fram detta genom att klicka på ikonen med en lus och en playknapp i det vertikala, gröna verktygsfältet längst till vänster. I en blå ring står en etta om du har startat en debuggingsession. Om det står en tvåa eller mer så har du flera sessioner igång och då kan det vara klokt att avsluta alla utom en, så att inte förvirring uppstår om vilken session som är den aktuella. 

\begin{figure}
\centering
\includegraphics[width=1.0\textwidth]{../img/vscode-trace}
\caption{Anropsstack och variabler i VS Code.\label{appendix-ide:vscode-trace}}
\end{figure}

Mycket av konsten i debugging handlar om att undersöka variablers värde under exekveringen för att ta reda på om din hypotes om vad som händer under exekvering verkligen stämmer, eller om något egentligen inte fungerar så som du antar. Detta kan du med fördel göra genom att placera brytpunkter på relevanta ställen. Även vid användning av en debugger kan du ha stor nytta av att göra \code{println} av intressanta uttryck för att i detalj undersöka vad som egentligen händer. Läs mer om debugging i Appendix \ref{appendix:debug}.
 

%!TEX encoding = UTF-8 Unicode
%!TEX root = ../compendium2.tex

\section{JetBrains IntelliJ IDEA med Scala-plugin}\label{appendix:ide:intellij}

IntelliJ IDEA%
\footnote{\href{https://en.wikipedia.org/wiki/IntelliJ_IDEA}{en.wikipedia.org/wiki/IntelliJ\_IDEA}}
 är en professionell IDE som stödjer många olika programmeringsspråk. IntelliJ är skriven i Java och utvecklas av det tjeckiska företaget JetBrains.

IntelliJ IDEA finns i två varianter: en gratis gemenskapsvariant med öppenkällkodslicens \Eng{Community edition}, samt en betalvariant med sluten källkod och support-tjänster.

\begin{figure}
\centering
\includegraphics[width=1.0\textwidth]{../img/intellij/idea-hello}
\caption{Den integrerade utvecklingsmiljön Intellij IDEA.\label{appendix-ide:intellij-hello}}
\end{figure}

IntelliJ IDEA är en omfattande och avancerad programmeringsmiljö med många funktioner och inställningar. Det finns även en omfattande uppsättning insticksmoduler och tilläggsprogram som underlättar utveckling av t.ex. mobilappar, webbprogram, databaser och mycket annat.

Till IntelliJ IDEA finns en insticksmodul \Eng{plug-in} som stöd för Scala med tillhörande standardbibliotek och byggverktyget \code{sbt}, med mera. Scala-insticksmodulen kan inkluderas genom att välja Scala i en av de dialoger som visas vid första körningen, enligt instruktioner nedan.

I detta avsnitt ges länkar till installation samt tips om hur du kommer igång med att använda IntelliJ IDEA med Scala. Det går ganska snabbt att lära sig grunderna, men det kräven en viss ansträngning att lära sig de mer avancerade funktionerna. Det finns omfattande resurser på nätet som hjälper dig vidare.

Google tillkännagav 2013 att företaget övergår från Eclipse till IntelliJ som den officiellt understödda utvecklingsmiljön för Android och 2014 lanserades utvecklingsmiljön Android Studio%
\footnote {\href{https://en.wikipedia.org/wiki/Android_Studio}{en.wikipedia.org/wiki/Android\_Studio}}
 som bygger vidare på IntelliJ.

\subsection{Installera IntelliJ IDEA}\label{appendix:ide:intellij:install}

IntelliJ med Scala-plug-in är förinstallerat på LTH:s datorer och startas med kommandot \texttt{idea} i ett terminalfönster.

Du kan installera IntelliJ på din egen dator genom att följa instruktionerna för ditt operativsystem (Windows/macOS/Linux) här: \\
\url{https://www.jetbrains.com/help/idea/run-for-the-first-time.html}


Du behöver Scala-plugin som du kan välja under installationen av IntelliJ, men det går också att installera plugin för Scala i efterhand, se vidare här:\\
\url{https://www.jetbrains.com/help/idea/discover-intellij-idea-for-scala.html} 




%\input{postchapters/scalajs.tex} %TODO!!
\setcounter{chapter}{9} %next after 9 is J in \Alph
\input{postchapters/java.tex} %TODO!!
%%!TEX encoding = UTF-8 Unicode
%!TEX root = ../compendium.tex

\chapter{Virtuell maskin}\label{appendix:vbox}

\section{Vad är en virtuell maskin?}

Du kan köra alla kursens verktyg i en så kallad \textbf{virtuell maskin} (förk. vm, eng. \textit{virtual machine}). 
Detta är ett enkelt och säkert sätt köra ett annat operativsystem i en ''sandlåda'' som inte påverkar din dators ursprungliga operativsystem. Figur \ref{fig:vm} visar kursens virtuella maskin \texttt{pgk-\VMName}. Exekveringen av en vm sker på en \textbf{värddator} \Eng{host}. I figur \ref{fig:vm} körs kursens vm i en Windows-värd med virtualiseringsapplikationen \textit{VirtualBox}\footnote{\href{https://en.wikipedia.org/wiki/VirtualBox}{/en.wikipedia.org/wiki/VirtualBox}}, som är öppen och gratis och finns för Linux-, Windows- och macOS-värdar. 



\begin{figure}[H]
\centering
\includegraphics[width=0.75\textwidth]{../img/\VMName.png}
\caption{Den virtuella maskinen pgk-\VMName~med Ubuntu \UbuntuVersion~under Windows \WindowsVersion~ i VirtualBox med version \VirtualBoxVersion~eller senare.}
\label{fig:vm}
\end{figure}


\section{Vad innehåller kursens vm?}

Kursens virtuell maskin kör en minimal installation av Ubuntu Linux och har verktyg för programmering med Scala förinstallerade. Detta ingår i kursens vm:

\begin{itemize}
\item \texttt{java} och \texttt{javac} med OpenJDK \JDKVersion
\item \texttt{scala} och \texttt{scalac} version \ScalaVersion
\item Kojo \KojoVersion
\item VS \texttt{code} version \VSCodeVersion~med Scala (Metals) version \MetalsVersion
\item \texttt{sbt} version \SbtVersion
\end{itemize}

Du kan själv uppdatera dessa applikationer till dess senaste versioner, och även installera fler applikationer, när du väl startat den virtuella maskinen. Se vidare kursens hemsida under ''Verktyg''.

\section{Installera kursens vm}

Det går lite långsammare att köra i en virtuell maskin jämfört med att köra direkt ''på metallen'', då det sker vissa översättningar och kontroller under virtualiseringsprocessen som annars inte behövs. Den virtuella maskinen behöver dessutom få en rejäl andel av din dators minne. Så för att köra en virtuell maskin utan att det ska bli segt behövs en ganska snabb processor, gärna över 2 GHz, och ganska mycket minne, gärna minst 8~GB. 

Även om det går lite segt är en virtuell maskin ett utmärkt sätt att prova på Linux och Ubuntu. Eftersom man lätt kan spara undan en hel maskin är det ett bra sätt att experimentera med olika inställningar och installationer utan att din normala miljö påverkas. Du kan lätt klona maskinen för att spara undan den i ett visst läge. Och kör du terminalfönster och en enkel editor brukar svag prestanda och lite minne inte vara ett stort problem. Om du tycker det går alltför segt kan du istället installera Linux direkt på din dator jämsides ditt andra operativsystem -- fråga någon som vet om hur man gör detta. 

Gör så här för att installera VirtualBox och köra kursens virtuella maskin:
\begin{enumerate}
\item  Ladda ner VirtualBox \VirtualBoxVersion~eller senare version för ditt operativsystem (t.ex. '''Platform Packages for Windows Hosts'') här och installera: \\ \url{https://www.virtualbox.org/wiki/Downloads}

%\item Ladda även ner \textit{''VirtualBox Oracle VM VirtualBox Extension Pack''}  och installera enligt instruktionerna här:\\ \url{https://www.virtualbox.org/wiki/Downloads} \\ Om du stöter på problem eller undrar hur, fråga någon om hjälp.

\item     Ladda ner filen \texttt{pgk-\VMName.ova} här: \\ \url{https://cs.lth.se/pgk/vm/} \\ OBS! Då filen är mer än 5~GB kan nedladdningen ta \textit{mycket} lång tid, kanske flera timmar beroende på din internetuppkoppling. Har du problem med nedladdningstider kan du prova att ladda ner filen till ett USB-minne på skolans datorer, så att överföringen sker lokalt i E-huset.

\item     Öppna VirtualBox och välj \MenuArrow{File}\Menu{Import appliance...} och välj filen \texttt{pgk-\VMName.ova} och klicka \Button{Next} och sedan \Button{Import}. Själva importen kan ta lång tid, kanske flera tiotals minuter beroende på hur snabbt din dator läser från disk.

\item Starta maskinen \textbf{pgk-\VMName} med ett dubbelklick. Ha lite tålamod innan maskinen är igång. Du kan behöva justera skärmstorleken i värdmaskinsmenyn \Menu{View}. Du  behöver lösenordet~\textbf{\texttt{\VMPassword}} för att logga in och för att installera nya program. 

\item Det kan hända att du får ett felmeddelande som innehåller något som liknar ''Intel VT-x'' eller ''Hyper-V'', så som beskrivs här:
\\ \href{http://www.howtogeek.com/213795/how-to-enable-intel-vt-x-in-your-computers-bios-or-uefi-firmware/}{www.howtogeek.com/213795/how-to-enable-intel-vt-x-in-your-computers-bios-or-uefi-firmware}\\
Då behöver du tillåta virtualiseringsfunktioner i BIOS på din dator. Om du inte vet hur du ska göra detta, be någon som vet om hjälp.

\item Börja med att öppna ett terminalfönster och uppdatera mjukvaran på din virtuella maskin med detta terminalkommando:
\begin{REPLnonum}
sudo apt update && sudo apt dist-upgrade
\end{REPLnonum}

\item Byt lösenord genom att trycka på windowsknappen och skriva ''users'', klicka på \textit{Users}-ikonen och trycka \Button{Password}.

\item Skriv \texttt{scala} i ett terminalfönster och du är igång och kan börja göra övningarna i detta kompendium!

%\item För att dra ner de senaste inlämningarna i kursrepot och uppdatera kompendiet och workspace, kör följande %terminalkommando:
%\begin{REPLnonum}
%$ cd ~/git/lunduniversity/introprog
%$ git pull
%$ sbt build
%$ sbt eclipse
%\end{REPLnonum}

\item Om allt verkar fungera fint kan du nu prova att öka minnet och även sätta på 3D-accelereringen för snabbare grafikrendering så här: 

\begin{enumerate}
\item Stäng maskinen genom att välja \Menu{Shut Down...} i systemmenyn. 

\item Markera maskinen \textbf{pgk-\VMName} och välj menyn \MenuArrow{Machine}\Menu{Settings...} (eller tryck Ctrl+S) och undersök inställningarna. Se speciellt under fliken \textbf{System} och \textbf{Motherboard} där det står hur mycket \textbf{Base memory} du tilldelar. Om din värddator har gott om minne (undersök exakt hur mycket) så kan du med fördel öka minnet till minst 4096~MB, speciellt om du tänker köra IntelliJ. I din virtuella maskin kan du undersöka hur stor andel av maskinens minne som är ockuperat genom att trycka på windowsknappen, skriva ''syst'', klicka på \Button{System Monitor} och välja fliken \Button{Resources}.

\item Ändra inställningar i menyn \MenuArrow{Settings...}\Menu{Display} genom att i fliken \textbf{Acceleration} under \textbf{Screen} markera \FramedCheckmark{Enable 3D acceleration}. Stara maskinen. Om det fungerar så blir animeringar avsevärt snyggare och smidigare. Om det inte fungerar, stäng av maskinen med \Menu{Power off} och avmarkera \FramedUnchecked{Enable 3D acceleration} igen.%\footnote{Du kan också prova att genomföra stegen som visas här, för att ominstallera vissa saker som kan ha uppdaterats sedan detta skrevs: \url{https://www.linuxbabe.com/virtualbox/how-to-install-virtualbox-guest-additions-on-ubuntu-step-by-step}}


\end{enumerate}

\end{enumerate}  %TODO!!

\part{Lösningar}

\setcounter{chapter}{11} %next is L in \Alph
\chapter{Lösningar till övningarna}\label{chapter:solutions}
\setcounter{section}{7}

\PreSolutionfalse

\let\QUESTBEGIN\ifPreSolution  % to mark formatting and numbering of exercises
\let\SOLUTION\else  % to mark solutions in the same file as questions
\let\QUESTEND\fi    % to mark end of exercise


%!TEX encoding = UTF-8 Unicode
%!TEX root = ../exercises.tex

\ifPreSolution

\Exercise{\ExeWeekEIGHT}\label{exe:W08}

\begin{Goals}
\item Kunna skapa och använda matriser med nästlade strukturer av \code{Vector}.
\item Kunna iterera över elementen i en matris med nästlade \code{for}-satser och \code{for}-\code{yield}-uttryck, samt nästlad applicering av \code{map} respektive \code{foreach}.
\item Kunna skapa och använda funktioner som tar matriser som parametrar.
\item Kunna skapa en enkel generisk klass och enkla generiska funktioner med hjälp av en typparameter.
\item Kunna beskriva skillnader och likheter mellan Scala och Java vad gäller indexering och iterering i matriser implementerade med nästlade arrayer.
%\item Kunna skapa och använda matriser med hjälp inbyggda arrayer i Java.
%\item Kunna använda nästlade \code{for}-satser i Java för att iterera över elementen i en matris.
\end{Goals}

\begin{Preparations}
\item \StudyTheory{08}
\end{Preparations}

\BasicTasks

\else

\ExerciseSolution{\ExeWeekEIGHT}

\BasicTasks

\fi



\WHAT{Para ihop begrepp med beskrivning.}

\QUESTBEGIN

\Task \what

\vspace{1em}\noindent Koppla varje begrepp med den (förenklade) beskrivning som passar bäst:

\begin{ConceptConnections}
\input{generated/quiz-w08-concepts-taskrows-generated.tex}
\end{ConceptConnections}

\SOLUTION

\TaskSolved \what

\begin{ConceptConnections}
\input{generated/quiz-w08-concepts-solurows-generated.tex}
\end{ConceptConnections}

\QUESTEND




\WHAT{Skapa matriser med hjälp av nästlade samlingar.}

\QUESTBEGIN

\Task  \what~  Man kan i ett datorprogram, med hjälp av samlingar som innehåller samlingar, skapa nästlade strukturer som kan indexeras i två dimensioner och på så sätt representera en  \textbf{matris}.\footnote{\href{https://sv.wikipedia.org/wiki/Matris}{sv.wikipedia.org/wiki/Matris}}

\Subtask Rita minnessituationen efter tilldelningen på rad 1 nedan. Vad har \code{m} för typ och värde? Vad har \code{m} för dimensioner? Hur sker indexeringen i ett datorprogram jämfört med i matematiken?

\begin{REPL}
scala> val m = Vector((1 to 5).toVector, (3 to 7).toVector)
scala> m.apply(0).apply(1)
scala> m(1)
scala> m(1)(4)
\end{REPL}

\Subtask Vad ger uttrycken på raderna 2, 3 och 4 ovan för värden och typ?

\Subtask Man kan i ett datorprogram mycket väl skapa tvådimensionella, nästlade strukturer där raderna \emph{inte} innehåller samma antal element. Det blir då ingen äkta matris i strikt matematisk mening, men man kallar ofta ändå en sådan struktur för en ''matris''. Vilken typ har variablerna \code{m2}, \code{m3}, \code{m4} och \code{m5} nedan?

\begin{REPL}
scala> val m2 = Vector(Vector(1,2,3),Vector(4,5),Vector(42))
scala> val m3 = Vector(Vector(1,2), Vector(1.0, 2.0, 3.0))
scala> val m4 = m3(1) +: Vector("a") +: m3
scala> val m5 = Vector.fill(42){ m2(1).map(e => (e * math.random()).toInt) }
\end{REPL}

\Subtask Vilken av variablerna \code{m2}, \code{m3}, \code{m4} och \code{m5} ovan representerar en äkta matris i matematisk mening? Vilken är dess dimensioner?

\SOLUTION

\TaskSolved \what

\SubtaskSolved   \includegraphics{../img/w09-solutions/1a} \\
Typ: \code{Vector[Vector[Int]]}\\
Värde: \code{Vector(Vector(1, 2, 3, 4, 5), Vector(3, 4, 5, 6, 7))} \\
Dimensioner: $2 \times 5$\\
Inom matematiken sker indexering enligt konvention med 1 som lägsta index. I scala är lägsta index 0, man använder s.k. 0-indexering. \footnote{Detta är inte fallet i alla programmeringsspråk, vilket du kan läsa mer om på \url{https://en.wikipedia.org/wiki/Array\_data\_type\#Index\_origin}}

\SubtaskSolved
\begin{REPL}
scala> val m = Vector((1 to 5).toVector, (3 to 7).toVector)
m: Vector[Vector[Int]] = Vector(Vector(1, 2, 3, 4, 5), Vector(3, 4, 5, 6, 7))

scala> m.apply(0).apply(1)
res4: Int = 2

scala> m(1)
res5: Vector[Int] = Vector(3, 4, 5, 6, 7)

scala> m(1)(4)
res6: Int = 7
\end{REPL}

\SubtaskSolved  \\
m2: \code{Vector[Vector[Int]]}\\
m3: \code{Vector[Vector[Int | Double]]}\\
m4: \code{Vector[Vector[Int | Double | String]]}\\
m5: \code{Vector[Vector[Int]]}

\SubtaskSolved  m5, $42 \times 2$

\QUESTEND





\WHAT{Skapa och iterera över matriser.}

\QUESTBEGIN

\Task  \label{matrices:task:yatzy} \what~  Du ska skapa matriser där varje rad representerar 5 kast med en tärning i spelet Yatzy.\footnote{\href{https://sv.wikipedia.org/wiki/Yatzy}{sv.wikipedia.org/wiki/Yatzy}}


\Subtask Definiera i REPL en funktion \code{def throwDie: Int = ???} som returnerar ett slumptal mellan 1 och 6.

\Subtask Skapa nedan heltalsmatris i REPL. Vilken dimension får matrisen?
\begin{REPL}
scala> val ds1 = for (i <- 1 to 1000) yield 
            for (j <- 1 to 5) yield throwDie
          
\end{REPL}

\Subtask Man kan också använda nedan varianter för att skapa en heltalsmatris. Vilken av varianterna \code{ds1} ... \code{ds6} tycker du är lättast att läsa och förstå? Prova respektive variant i REPL och ange vilken typ på \code{ds1} ... \code{ds6} som härleds av kompilatorn.
\begin{REPL}
val ds2 = (1 to 1000).map(i => (1 to 5).map(j => throwDie))
val ds3 = (1 to 1000).map(i => Vector.fill(5)(throwDie))
val ds4 = for (i <- 1 to 1000) yield Vector.fill(5)(throwDie)
val ds5 = Vector.fill(1000)(Vector.fill(5)(throwDie))
val ds6 = Vector.fill(1000, 5)(throwDie)
\end{REPL}


\Subtask Definiera en funktion \\ \code{def roll(n: Int): Vector[Int] = ???}\\ som ger en heltalsvektor med $n$ stycken slumpvisa tärningskast. Kasten ska vara sorterade i växande ordning; använd för detta ändamål samlingsmetoden \code{sorted}.


\Subtask \label{matrices:subtask:isyatzyforall} Definera i REPL en funktion \code{isYatzy(xs: Vector[Int]): Boolean = ???} som testar om alla elementen i en heltalsvektor är samma. Använd samlingsmetoden \code{forall}.


\Subtask Skapa en funktion  \\ \code{def diceMatrix(m: Int, n: Int): Vector[Vector[Int]] = ???} \\ som med hjälp av funktionen \code{roll} skapar en matris med \code{m} st vektorer med vardera \code{n} slumpvisa tärningskast.


\Subtask \label{matrices:subtask:diceMatrixToString} Skapa en funktion som returnerar en utskriftsvänlig sträng \\ \code{def diceMatrixToString(xss: Vector[Vector[Int]]): String = ???} \\med hjälp av \code{map} och \code{mkString}, som fungerar enligt nedan.
\begin{REPL}
scala> val dm2s = diceMatrixToString(diceMatrix(4, 5))
val dm2s: String = 1 4 4 6 6
1 1 2 6 6
2 4 4 5 6
1 1 5 6 6

scala> println(dm2s)
1 4 4 6 6
1 1 2 6 6
2 4 4 5 6
1 1 5 6 6
\end{REPL}



\Subtask Implementera funktionen \\ \code{def filterYatzy(xss: Vector[Vector[Int]]): Vector[Vector[Int]]} \\ som filtrerar fram alla yatzy-rader i matrisen \code{xss} enligt nedan. Använd din funktion \code{isYatzy} och samlingsmetoden \code{filter}.
\begin{REPL}
scala> println(diceMatrixToString(filterYatzy(diceMatrix(10000, 5))))
4 4 4 4 4
6 6 6 6 6
4 4 4 4 4
6 6 6 6 6
4 4 4 4 4
4 4 4 4 4
2 2 2 2 2
\end{REPL}



\Subtask Implementera funktionen \\
\code{def yatzyPips(xss: Vector[Vector[Int]]): Vector[Int] = ???}\\
som ska ge en vektor med de tärningsvärden som gav yatzy, för kasten i matrisen \code{xss} enligt nedan. Använd din funktion \code{filterYatzy}.
\begin{REPL}
scala> val dm = Vector(Vector(1,2,3,4,5),Vector(4,4,4,4,4),Vector(3,3,3,3,3))
scala> yatzyPips(dm)
val res42: Vector[Int] = Vector(4, 3)
\end{REPL}

\SOLUTION

\TaskSolved \what

\SubtaskSolved
\begin{Code}
def throwDie: Int = (math.random() * 6).toInt + 1
\end{Code}
Eller:
\begin{Code}
def throwDie: Int = scala.util.Random.nextInt(6) + 1
\end{Code}

\SubtaskSolved  Matrisdimension i matematisk notation: $1000 \times 5$, vilket motsvarar en matris med 1000 rader och 5 kolumner.

\SubtaskSolved
\begin{Code}
ds1: IndexedSeq[IndexedSeq[Int]]
ds2: IndexedSeq[IndexedSeq[Int]]
ds3: IndexedSeq[Vector[Int]]
ds4: IndexedSeq[Vector[Int]]
ds5: Vector[Vector[Int]]
ds6: Vector[Vector[Int]]
\end{Code}
\code{IndexedSeq} och \code{Vector} ovan finns i paketet \code{scala.collection.immutable}

\SubtaskSolved  \begin{Code}
def roll(n: Int) = Vector.fill(n)(throwDie).sorted
\end{Code}

\SubtaskSolved  \begin{Code}
def isYatzy(xs: Vector[Int]): Boolean = xs.forall(_ == xs(0))
\end{Code}



%2.g)
\SubtaskSolved  \begin{Code}
def diceMatrix(m: Int, n: Int): Vector[Vector[Int]] =
  Vector.fill(m)(roll(n))
\end{Code}

\SubtaskSolved  \begin{Code}
def diceMatrixToString(xss: Vector[Vector[Int]]): String =
  xss.map(_.mkString(" ")).mkString("\n")
\end{Code}


%2.j)
\SubtaskSolved
\begin{Code}
def filterYatzy(xss: Vector[Vector[Int]]): Vector[Vector[Int]] =
  xss.filter(isYatzy)
\end{Code}



%2.m)
\SubtaskSolved  \begin{Code}
def yatzyPips(xss: Vector[Vector[Int]]): Vector[Int] =
  filterYatzy(xss).map(_.head)
\end{Code}

\QUESTEND








\WHAT{En oföränderlig, generisk matris-klass till veckans laboration \hyperref[section:lab:\LabWeekEIGHT]{\texttt{\LabWeekEIGHT}}.}

\QUESTBEGIN

\Task\label{exe:matrices:labprep}  \what~Under veckans laboration ska du simulera en enkel form av ''liv'' som består av celler i ett rutnät. För detta ändamål har vi nytta av en matris-klass som du ska implementera steg för steg i denna övning.
Skapa case-klassen nedan med en editor i filen \code{Matrix.scala}. Testa din lösning med hjälp av valfri \hyperref[appendix:ide]{IDE}, t.ex. \code{scalaide} eller \code{idea}.
\begin{Code}
case class Matrix(data: Vector[Vector[String]]){
  def apply(row: Int, col: Int): String = data(row)(col)
}
object Matrix {
  def fill(dim: (Int, Int))(value: String): Matrix =
    Matrix(Vector.fill(dim._1, dim._2)(value))
}
\end{Code}

\begin{REPLnonum}
scala> val m = Matrix.fill(3,4)("hej")
scala> val e = m(2, 2)
\end{REPLnonum}

\Subtask Vad får \code{m} ovan för typ?

\Subtask Vad får \code{e} ovan för typ?

\Subtask På hur många ställen måste du ändra i \code{Matrix} ovan för att den i stället ska representera en matris av heltal?

\Subtask Du ska nu med hjälp av en \textbf{typparameter} göra \code{Matrix} \textbf{generisk} \Eng{generic}, så att den blir en mer användbar matrisklass som kan innehålla element av vilken typ som helst. Genomför följande ändringar i \code{Matrix.scala}:

\begin{itemize}[noitemsep, nolistsep]
  \item Lägg till en typparameter \code{T} inom klammerparenteser efter namnet \code{Matrix} på alla ställen där det förekommer \emph{utom} efter namnet på kompanjonsobjektet\footnote{Singelobjekt kan inte ha typparametrar, men deras medlemmar kan.}.
  \item Byt ut \code{String} mot \code{T} på alla ställen där \code{String} förekommer.
  \item Lägg till en typparameter \code{T} inom klammerparenteser efter \code{def fill}.
\end{itemize}
Testa din generiska klass i REPL genom att skapa en boolesk matris:
\begin{REPLnonum}
scala> val bm = Matrix.fill(3,4)(false)
scala> val be = bm(0, 0)
\end{REPLnonum}

\Subtask Vad får \code{bm} ovan för typ?

\Subtask Vad får \code{be} ovan för typ?

\Subtask Lägg en kodrad i början av klasskroppen som med hjälp av \code{require} garanterar att alla rader i matrisen är lika långa.

\Subtask Lägg till en medlem \code{val dim: (Int, Int)} i klasskroppen efter \code{require}-satsen som ger ett par (alltså en 2-tupel) med antalet rader resp. kolumner i matrisen.

\Subtask Lägg till en metod \code{def updated(row: Int, col: Int)(value: T): Matrix[T]} som ger en ny matris där element på platsen \code{(row, col)} har uppdaterats till \code{value}.

\Subtask Lägg till en metod \code{def foreachIndex(f: (Int, Int) => Unit): Unit} som för varje index i \code{data} applicerar funktionen \code{f}.

\Subtask Lägg till en metod \code{override def toString} som så att en instans av \code{Matrix} visas enligt följande:
\begin{REPLnonum}
scala> val dm = Matrix.fill(3,4)(42.0)
val dm: Matrix[Double] =
Matrix of dim (3,4):
42.0 42.0 42.0 42.0
42.0 42.0 42.0 42.0
42.0 42.0 42.0 42.0
\end{REPLnonum}


\SOLUTION


\TaskSolved \what

\SubtaskSolved Typen på \code{m} blir \code{Matrix}.

\SubtaskSolved Typen på \code{e} blir \code{String}.

\SubtaskSolved Man behöver ändra på 3 ställen från \code{String} till \code{Int}.

\SubtaskSolved Generisk matris \code{Matrix[T]} för element av godtycklig typ \code{T}:

\begin{CodeSmall}
case class Matrix[T](data: Vector[Vector[T]]):
  def apply(row: Int, col: Int): T = data(row)(col)

object Matrix:
  def fill[T](dim: (Int, Int))(value: T): Matrix[T] =
    Matrix[T](Vector.fill(dim._1, dim._2)(value))
\end{CodeSmall}

\SubtaskSolved Tack vare kompilatorns typinferens så får \code{bm} typen \code{Matrix[Boolean]}.

\SubtaskSolved Typen på \code{be} blir \code{Boolean}.

\noindent \SubtaskSolved \SubtaskSolved \SubtaskSolved \SubtaskSolved \SubtaskSolved är alla implementerade i koden nedan: \vspace{-0.5em}
\begin{CodeSmall}
case class Matrix[T](data: Vector[Vector[T]]):
  require(data.forall(row => row.length == data(0).length))

  val dim: (Int, Int) = (data.length, data(0).length)

  def apply(row: Int, col: Int): T = data(row)(col)

  def updated(row: Int, col: Int)(value: T): Matrix[T] =
    Matrix(data.updated(row, data(row).updated(col, value)))

  def foreachIndex(f: (Int, Int) => Unit): Unit =
    for r <- data.indices; c <- data(r).indices do f(r, c)

  override def toString =
    s"""Matrix of dim $dim:\n${ data.map(_.mkString(" ")).mkString("\n") }"""

object Matrix:
  def fill[T](dim: (Int, Int))(value: T): Matrix[T] =
    Matrix[T](Vector.fill(dim._1, dim._2)(value))

\end{CodeSmall}

\QUESTEND


\clearpage

\ExtraTasks %%%%%%%%%%%%%%%%%%%%%%%%%%%%%%%%%%%%%%%%%%%%%%%%%


\WHAT{Imperativa matrisalgoritmer.}

\QUESTBEGIN

\Task  \what~Imperativa angreppssätt är nödvändiga att kunna när du stöter på samlingar och/eller språk som saknar funktionella metoder och/eller funktionsprogrammeringsmöjligheter. Genom att studera imperativa lösningar till de ofta mer koncisa funktionella lösningarna, får du träning i att skapa algoritmer som använder förändring genom tilldelning vid iterering.

\Subtask Implementera \code{isYatzy} från uppgift \ref{matrices:task:yatzy}\ref{matrices:subtask:isyatzyforall} igen, men nu med ett imperativt angreppssätt som använder en \code{while}-sats i stället för funktionella \code{forall}. Ta hjälp av en variabel \code{i} som håller reda på index och en variabel \code{foundDiff} som håller reda på om ett avvikande värde upptäcks. Funktionen kräver ca 9 rader, så det kan vara lämpligt att öppna en editor att skriva i medan du klurar ut lösningen. Börja med att skriva pseudokod, gärna med penna på papper. Prova genom att klistra in i REPL.

\Subtask En imperativ implementation av \code{diceMatrixToString} från uppgift \ref{matrices:task:yatzy}\ref{matrices:subtask:diceMatrixToString} med hjälp av förändringsbara  \code{StringBuilder}\footnote{\url{https://www.scala-lang.org/api/2.12.9/scala/collection/mutable/StringBuilder.html}} visas nedan. Förklara hur nedan kod fungerar. Vad händer om \code{xss} är tom? Vad händer om \code{xss} bara innehåller tomma vektorer? Nämn en fördel och en nackdel med att använda \code{val sb: StringBuilder} och \code{append}, jämfört med en vanlig, oföränderlig \code{var s: String} och \code{+} för tillägg i slutet.
\begin{Code}
def diceMatrixToString(xss: Vector[Vector[Int]]): String = 
  val sb = new StringBuilder()
  for(m <- xss.indices) do
    for(n <- xss(m).indices) do
      sb.append(xss(m)(n).toString)
      if n < xss(m).size - 1 then sb.append(" ")
      else if m < xss.size - 1 then sb.append("\n")
    end for
  end for
  sb.toString
\end{Code}

\Subtask Gör som träning en imperativ implementation av \code{filterYatzy} med en \code{for}-\code{do}-sats (alltså utan att använda \code{filter}, och utan att använda \code{yield}).


\Subtask Förklara hur nedan funktionella implementation av \code{filterYatzy} med \code{for}-\code{yield}-uttryck fungerar. Tycker du din imperativa lösning är lättare eller svårare att läsa och förstå jämfört nedan funktionella lösning?
\begin{CodeSmall}
def filterYatzy(xss: Vector[Vector[Int]]): Vector[Vector[Int]] = 
  (for i <- xss.indices if isYatzy(xss(i)) yield xss(i)).toVector
\end{CodeSmall}


\SOLUTION

\TaskSolved \what

\SubtaskSolved  \begin{Code}
def isYatzy(xs: Vector[Int]): Boolean = 
  var foundDiff = false
  var i = 0
  while (i < xs.size && !foundDiff) do
    foundDiff = xs(i) != xs(0)
    i += 1
  end while
  !foundDiff
\end{Code}


\SubtaskSolved  Funktionen går igenom varje matrisrad, där den i sin tur går igenom
varje element på raden och lägger till i \code{StringBuilder}-objektet. Om det inte är
det sista elementet på raden läggs även ett blanktecken till, annars läggs ett
nyradstecken till. Undantaget är sista raden, där inget nyradstecken läggs till.
Slutligen konverteras \code{StringBuilder}-objektet till en \code{String} som
returneras.


Är \code{xss} tom blir \code{xss.indices} en tom \code{Range} och den yttre \code{for}-loopen hoppas över och en tom sträng returneras.
Är alla rader tomma hoppas i stället de inre \code{for}-looparna över, med samma resultat.

\emph{Fördel:} \code{StringBuilder} är snabbare vid tillägg på slutet vid stora strängar (men här kommer det inte märkas eftersom strängen är så liten).

\emph{Nackdel:} StringBuilder-koden uppfattas av många som svårare att läsa.

\SubtaskSolved
\begin{Code}
def filterYatzy(xss: Vector[Vector[Int]]): Vector[Vector[Int]] = 
  var result: Vector[Vector[Int]] = Vector()
  for i <- xss.indices if isYatzy(xss(i)) do result = result :+ xss(i)
  result
\end{Code}

\SubtaskSolved  Varje looprunda ger en vektor \code{xss(i)} om filtervillkoret är uppfyllt och resultatet av \code{for}-uttrycket blir en vektor med vektorer som är yatzyslag.

\QUESTEND



\WHAT{Strängtabell med kolumnrubriker.}

\QUESTBEGIN

\Task  \what~  %Denna övning utgör en början på laboration \hyperref[section:lab:survey]{\texttt{survey}} i avsnitt \ref{section:lab:survey} på sidan \pageref{section:lab:survey}.

\Subtask Implementera case-klassen \code{Table} enligt specifikationen nedan. Du kan förutsätta att alla rader har lika många kolumner som antalet element i \code{headings}, samt att alla rubrikerna i \code{headings} är unika. Parametern \code{sep} anger det tecken som används för att separera kolumner. Detta förutsätts också gälla för indatafiler som läses in med \code{fromFile}.

\emph{Tips:}
\begin{itemize}%[nolistsep,noitemsep]
\item Värdet \code{indexOfHeading} kan skapas med hjälp av metoden \code{zipWithIndex} som fungerar på alla sekvenssamlingar, samt metoden \code{toMap} som fungerar på sekvenser av 2-tupler. Undersök först hur metoderna fungerar i REPL och sök upp deras dokumentation.
\item Skapa en indatafil som du kan använda för att testa att \code{Table} fungerar.
\end{itemize}


\begin{CodeSmall}
case class Table(
  data: Vector[Vector[String]],
  headings: Vector[String],
  sep: Char
):
  /** A 2-tuple with (number of rows, number of columns) in data */
  val dim: (Int, Int) = ???

  /** The element in row r and column c of data, counting from 0 */
  def apply(r: Int, c: Int): String = ???

  /** The row-vector r in data, counting from 0 */
  def row(r: Int): Vector[String]= ???

  /** The column-vector c in data, counting from 0 */
  def col(c: Int): Vector[String] = ???

  /** A map from heading to index counting from 0 */
  lazy val indexOfHeading: Map[String, Int] = ???

  /** The column-vector with heading h in data */
  def col(h: String): Vector[String] = ???

  /** A vector with the distinct, sorted values of col with heading h */
  def values(h: String): Vector[String] = ???

  /** Headings and data with columns separated by sep */
  override lazy val toString: String = ???

object Table:
  /** Creates a new Table from fileName with columns split by sep */
  def fromFile(fileName: String, sep: Char = ';'): Table = ???
\end{CodeSmall}

\Subtask Skapa med hjälp av \code{Table} ett program som kan köras från terminalen med \texttt{scala run infile.csv ';'} som ger en utskrift av antalet förekomster av olika värden i respektive kolumn (alltså en variant av registrering).



\SOLUTION

\TaskSolved \what

\SubtaskSolved  \begin{CodeSmall}
case class Table(
  data: Vector[Vector[String]],
  headings: Vector[String],
  sep: Char
):

  val dim: (Int, Int) = (data.size, headings.size)

  def apply(r: Int, c: Int): String = data(r)(c)

  def row(r: Int): Vector[String]= data(r)

  def col(c: Int): Vector[String] = data.map(r => r(c))

  lazy val indexOfHeading: Map[String, Int] = headings.zipWithIndex.toMap

  def col(h: String): Vector[String] = col(indexOfHeading(h))

  def values(h: String): Vector[String] = col(h).distinct.sorted

  override def toString: String =
    val s = sep.toString
    headings.mkString(s) + "\n" +data.map(_.mkString(s)).mkString("\n")

object Table:
  def fromFile(fileName: String, sep: Char = ';'): Table = 
    val lines = scala.io.Source.fromFile(fileName).getLines.toVector
    val matrix= lines.map(_.split(sep).toVector)
    new Table(matrix.tail, matrix.head, sep)
\end{CodeSmall}

\SubtaskSolved  \begin{CodeSmall}
@main 
def run(fileName: String, separator: String): Unit = 
  require(separator.length == 1, "separator ska vara exakt ett tecken")
  val t = Table.fromFile(fileName, separator.head)
  val counts: Vector[Vector[String]] =
    (0 until t.dim._2)
      .map(i => t.values(t.headings(i))
      .map(x => s"$x: ${t.col(i).count(_ == x)}"))
      .toVector
  for (i <- 0 until t.dim._2) do
    println(s"\nColumn: ${i + 1}, ${t.headings(i)}:")
    for (j <- 0 until counts(i).length) do
      println(counts(i)(j))
\end{CodeSmall}

\QUESTEND




\WHAT{Skapa ett yatzy-spel för användning i terminalen.}

\QUESTBEGIN

\Task  \what~%
% \Subtask Skapa en yatzy-matris enligt nedan specifikation. Läs om hur de olika predikaten för att kolla olika giltiga kombinationer i Yatzy ska fungera här: \href{https://en.wikipedia.org/wiki/Yahtzee}{en.wikipedia.org/wiki/Yahtzee}. Bygg ett huvudprogram som testar dina funktioner. Kompilera och testa i terminalen allteftersom du lägger till nya funktioner.
%
% \begin{CodeSmall}
% /** En skiss på en klass som kan användas till ett förenklat yatzy-spel */
% case class YatzyRows(val rows: Vector[Vector[Int]]) {
%   /** A new YatzyRows with a new row of 5 dice rolls appended to rows  */
%   def roll: YatzyRows = ???
%
%   /** A new YatzyRows with some indices of the last row re-rolled  */
%   def reroll(indices: Vector[Int]): YatzyRows = ???
% }
%
% object YatzyRows {
%   def isYatzy(xs: Vector[Int]): Boolean = ???
%   def isThreeOfAKind(xs: Vector[Int]): Boolean = ???
%   def isFourOfAKind(xs: Vector[Int]): Boolean = ???
%   def isFullHouse(xs: Vector[Int]): Boolean = ???
%   def isSmallStraight(xs: Vector[Int]): Boolean = ???
%   def isLargeStraight(xs: Vector[Int]): Boolean = ???
% }
% \end{CodeSmall}
%
%
% \Subtask Använd \code{YatzyRows} för att med hjälp av många tärningskast beräkna sannolikheter för några olika giltiga kombinationer. Använd, om du vill, möjligheten som reglerna ger att slå om tärningar i två ytterliggare kast, där de tärningar som slås om väljs slumpmässigt.
%
%\Subtask
Bygg ett förenklat yatzy-spel i terminalen där användaren kan bestämma vilka tärningar som ska slås om. Börja med något riktigt enkelt och bygg sedan vidare på ditt spel genom att införa fler och fler funktioner.

\SOLUTION


\TaskSolved \what
     %starts with: \emph{Skapa ett yatzy-spel för %%%

 --

% \SubtaskSolved   \begin{CodeSmall}
% /** En skiss på en klass som kan användas till ett förenklat yatzy-spel */
% case class YatzyRows(val rows: Vector[Vector[Int]]) {
%
%   private def throwDie: Int = (math.random() * 6).toInt + 1
%
%   /** A new YatzyRows with a new row of 5 dice rolls appended to rows */
%   def roll: YatzyRows = new YatzyRows(rows :+ Vector.fill(5)(throwDie))
%
%   /** A new YatzyRow with some indices of the last row re-rolled */
%   def reroll(indices: Vector[Int]): YatzyRows =
%     new YatzyRows(rows :+ rows(rows.length - 1).zipWithIndex.map {
%       case (x, i) => if (indices.contains(i)) throwDie else x
%     })
% }
% object YatzyRows {
%
%   def isYatzy(xs: Vector[Int]): Boolean = xs.forall(_ == xs(0))
%
%   def isThreeOfAKind(xs: Vector[Int]): Boolean =
%     xs.exists(x => xs.count(_ == x) >= 3)
%
%   def isFourOfAKind(xs: Vector[Int]): Boolean =
%     xs.exists(x => xs.count(_ == x) >= 4)
%
%   def isFullHouse(xs: Vector[Int]): Boolean =
%     xs.exists(x => xs.count(_ == x) == 3) &&
%     xs.exists(x => xs.count(_ == x) == 2)
%
%   def isSmallStraight(xs: Vector[Int]): Boolean =
%     xs.forall(x => xs.count(_ == x) == 1) && !xs.exists(_ == 6)
%
%   def isLargeStraight(xs: Vector[Int]): Boolean =
%     xs.forall(x => xs.count(_ == x) == 1) && !xs.exists(_ == 1)
% }
%
% \end{CodeSmall}
% Observera att fem stycken 2:or uppfyller kraven för Yatzy, men även för triss och fyrtal.
%
% \SubtaskSolved   Slumpen gör att utfallet inte kommer stämma exakt överens med teorin, men för ett stort antal kast bör resultaten hamna ganska nära. De teoretiska sannolikheterna (utan omkast) finns i \ref{yatzyProb}.
% \begin{table}[h]
% \centering
% \caption{Sannolikhet för olika Yatzy-resultat}
% \label{yatzyProb}
% \begin{tabular}{ll}
% Yatzy&  $0,077\%$  \\
% $\geq3$ av samma& $21\%$\\
% $\geq4$ av samma& $2,0\%$\\
% Kåk& $3,9\%$\\
% Liten stege& $1,5\%$\\
% Stor stege& $1,5\%$
% \end{tabular}
% \end{table}
%
% Kodexempel:
% \begin{CodeSmall}
% import YatzyRows._
%
% object YatzyStats extends App {
%   val n = 1000000.0
%   var yr = YatzyRows(Vector(Vector[Int]()))
%   for (i <- 1 to n.toInt) yr = yr.roll
%   println(s"Yatzy: ${yr.rows.count(isYatzy(_)) / n * 100}%")
%   println(s"Three of a kind: ${yr.rows.count(isThreeOfAKind(_)) / n * 100}%")
%   println(s"Four of a kind: ${yr.rows.count(isFourOfAKind(_)) / n * 100}%")
%   println(s"Full house: ${yr.rows.count(isFullHouse(_)) / n * 100}%")
%   println(s"Small straight: ${yr.rows.count(isSmallStraight(_)) / n * 100}%")
%   println(s"Large straight: ${yr.rows.count(isLargeStraight(_)) / n * 100}%")
% }
% \end{CodeSmall}
%
% \SubtaskSolved  --

\QUESTEND






\clearpage

\AdvancedTasks %%%%%%%%%%%%%%%%%


\WHAT{Generiska funktioner.}

\QUESTBEGIN

\Task  \what~  En generisk funktion har (minst) en typparameter inom klammerparenteser efter namnet, till exempel \code{[T]}. Denna typ förekommer sedan som typ på (någon av) parametrarna i parameterlistan. Kompilatorn härleder en konkret typ vid kompileringstid och ersätter typparametern med denna konkreta typ. På så sätt kan en funktion fungera för många olika typer.

\Subtask Förklara för varje rad nedan vad som händer.

\begin{REPL}
scala> def tnirp[T](x: T): Unit = println(x.toString.reverse)
scala> tnirp(42)
scala> tnirp("hej")
scala> case class Gurka(vikt: Int)
scala> tnirp(Gurka(42))
scala> tnirp[String](42)
scala> tnirp[Double](42)
\end{REPL}

\Subtask Man kan kombinera generiska funktioner med funktioner som tar funktioner som parametrar. Det är så \code{map} och \code{foreach} är implementerade. Förklara för varje rad nedan vad som händer.

\begin{REPL}
scala> def compose[A, B, C](f: A => B, g: B => C)(x: A): C = g(f(x))
scala> def inc(x: Int): Int = x + 1
scala> def half(x: Int): Double = x / 2.0
scala> compose(inc, half)(42)
scala> compose(half, inc)(42)
\end{REPL}

\Subtask Hur lyder felmeddelandet på sista raden ovan? Ändra \code{inc} och/eller \code{half} så att typerna passar.

\SOLUTION

\TaskSolved \what
     %starts with: \emph{Generiska funkioner.} En %%%

%4.a)
\SubtaskSolved   \begin{enumerate}
\item --
\item Strängrepresentationen av \code{42} spegelvänds
\item \code{"hej"} spegelvänds - \code{toString} av en sträng ger en likadan sträng
\item --
\item Gurk-objektets strängrepresentation spegelvänds
\item Funktionens typparameter matchar inte parameterns typ: \code{42} är ingen sträng
\item Implicit typkonvertering till \code{Double} sker för att stämma överens med typparametern, vilket ger en strängrepresentation med decimal
\end{enumerate}

%4.b)
\SubtaskSolved   \begin{enumerate}
\item En funktion definieras så att den tar emot två andra funktioner som argument, sätter ihop dem, och matar in ett tredje argument till den den sammansatta funktionen.
\item En funktion som inkrementerar ett heltal med 1 definieras.
\item En funktion som halverar ett flyttal definieras.
\item \code{42} matas in i \code{inc()} och resultatet (\code{43}) matas vidare till \code{half()}. Inuti \code{half()} sker implicit typkonvertering till \code{Double} då talet divideras med ett flyttal (\code{2.0}) och resultatet blir \code{43.0 / 2.0}, alltså \code{21.5}.
\item Resultatet från \code{half()} är av typ \code{Double}, medan \code{inc()} tar emot ett argument av typ \code{Int}. Då flyttal generellt inte kan konverteras till heltal utan informationsförlust sker ingen implicit konvertering, istället sker ett kompileringsfel.
\end{enumerate}

%4.c)
\SubtaskSolved  \begin{Code}
def inc(x: Double): Double = x + 1.0
\end{Code}
Nu ges kompileringsfel på rad 4 istället, vilket kan lösas med följande ändring:
\begin{Code}
def half(x: Double): Double = x / 2.0
\end{Code}

\QUESTEND




\WHAT{Generiska klasser.}

\QUESTBEGIN

\Task  \what~  Även klasser kan vara generiska. En generisk klass har (minst) en typparameter inom klammerparenteser efter klassens namn.

\Subtask Testa nedan generiska klass \code{Cell[T]} i REPL. Skapa instanser av klassen \code{Cell[T]} där typparametern \code{T} binds till olika konkreta typer och förklara vad som händer.

\begin{REPL}
scala> class Cell[T](var value: T):
         override def toString = "Cell(" + value + ")"
       
scala> new Cell(42)
scala> new Cell("hej")
scala> new Cell(new Cell(math.Pi))
scala> new Cell[String](42)
scala> new Cell[Double](42)
\end{REPL}

\Subtask Lägg till metoden \code{def concat[U](that: Cell[U]):Cell[String]} i klassen \code{Cell} som konkatenerar strängrepresentationerna av de båda cellvärdena.

\begin{REPL}
scala> val a = new Cell("hej")
scala> val b = new Cell(42)
scala> a concat b
\end{REPL}

\Subtask Vilken sorts celler kan du konkatenera om du tar bort typparameternamnet \code{U} i \code{concat} samtidigt som du använder \code{Cell[T]} som typ på värdeparametern \code{that}? Vad ger det för konsekvenser för celler av annan typ än \code{Cell[String]}?

\SOLUTION

\TaskSolved \what

%5.a)
\SubtaskSolved  --

%5.b)
\SubtaskSolved  \begin{Code}
class Cell[T](var value: T):
  override def toString = "Cell(" + value + ")"
  def concat[U](that: Cell[U]): Cell[String] = 
    Cell(s"$value${that.value}")
\end{Code}

%5.c)
\SubtaskSolved   Endast celler med samma typparameter kan nu konkateneras. Eftersom \code{concat()} returnerar ett objekt av typ \code{Cell[String]} kan ett ojämnt antal celler med någon annan typparameter än \code{String} alltså inte längre konkateneras. Är antalet jämnt går det att konkatenera dem parvis och sedan konkatenera de returnerade \code{Cell[String]}-objekten, men det är något omständigt.

\QUESTEND

\WHAT{Implementera fler generiska metoder i \code{Matrix[T]}.}

\QUESTBEGIN

\Task \what~ Bygg vidare på uppgift \ref{exe:matrices:labprep} och implementera nedan specifikation. Skapa egna tester som kontrollerar att alla metoder fungerar som förväntat.

\begin{ScalaSpec}{Matrix[T]}
/** En oföränderlig, generisk Matris-klass. */
case class Matrix[T](data: Vector[Vector[T]]):
  require(???)  // garantera att alla rader har lika många kolumner

  /** Ger ett par med antal rader och kolumner. */
  val dim: (Int, Int) = ???

  /** Ger elementet på plats (row, col). */
  def apply(row: Int, col: Int): T = ???

  /** Ger en ny matris där elementet på plats (row, col) har värdet value. */
  def updated(row: Int, col: Int)(value: T): Matrix[T] =  ???

  /** Applicerar f på alla element. */
  def foreach(f: T => Unit): Unit = ???

  /** Applicerar f på alla index. */
  def foreachIndex(f: (Int, Int) => Unit): Unit = ???

  /** Ger en ny matris med resultaten av elementvis applicering av f. */
  def map[U](f: T => U): Matrix[U] = ???

  /** Ger en ny matris med resultaten av applicering av f på varje index. */
  def mapIndex[U](f: (Int, Int) => U): Matrix[U] = ???

  /** Ger en utskriftsvänlig strängrepresentation av matrisen. */
  override def toString = ???

object Matrix:
  /** Ger en matris med dimension dim där alla element har värdet value. */
  def fill[T](dim: (Int, Int))(value: T): Matrix[T] = ???
\end{ScalaSpec}

\SOLUTION


\TaskSolved \what

\begin{CodeSmall}
case class Matrix[T](data: Vector[Vector[T]]):
  require(data.forall(row => row.size == data(0).size))

  val dim: (Int, Int) = (data.length, data(0).length)

  def apply(row: Int, col: Int): T = data(row)(col)

  def updated(row: Int, col: Int)(value: T): Matrix[T] =
    Matrix(data.updated(row, data(row).updated(col, value)))

  def foreach(f: T => Unit): Unit = data.foreach(_.foreach(f))

  def foreachIndex(f: (Int, Int) => Unit): Unit =
    for r <- data.indices; c <- data(r).indices do f(r, c)

  def map[U](f: T => U): Matrix[U] = Matrix(data.map(_.map(f)))

  def mapIndex[U](f: (Int, Int) => U): Matrix[U] =
    var result = Matrix.fill(dim)(f(0,0))
    for 
      r <- data.indices
      c <- data(r).indices 
    do
      result = result.updated(r, c)(f(r, c))
    end for
    result

  override def toString =
    s"""Matrix of dim $dim:\n${ data.map(_.mkString(" ")).mkString("\n") }"""

object Matrix:
  def fill[T](dim: (Int, Int))(value: T): Matrix[T] =
    Matrix[T](Vector.fill(dim._1, dim._2)(value))
\end{CodeSmall}


\QUESTEND





% \WHAT{Skapa en generisk, oföränderlig matrisklass.}
%
% \QUESTBEGIN
%
% \Task \label{task:generic-matrix} \what~   Med hjälp av en typparameter kan vi skapa en matrisklass som kan innehålla vilka element som helst. Implementera nedan specifikation. Testa din matrisklass i REPL för olika typer av element.
%
% \begin{ScalaSpec}{Matrix[T]}
% case class Matrix[T](data: Vector[Vector[T]]){
%
%   def foreachRowCol(f: (Int, Int, T) => Unit): Unit =
%     for (r <- 0 until data.size) {
%       for (c <- 0 until data(r).size) {
%         f(r, c, data(r)(c))
%       }
%     }
%
%   def map[U](f: T => U): Matrix[U] = Matrix(data.map(_.map(f)))
%
%   /** The element at row r and column c */
%   def apply(r: Int, c: Int): T = ???
%
%   /** Gives Some[T](element) at row r and column c
%    *  if r and c are within index bounds, else None */
%   def get(r: Int, c: Int): Option[T] = ???
%
%   /** The row vector of row r */
%   def row(r: Int): Vector[T] = ???
%
%   /** The column vector of column c */
%   def col(c: Int): Vector[T] = ???
%
%   /** A new Matrix with element at row r and col c updated */
%   def updated(r: Int, c: Int, value: T): Matrix[T] = ???
% }
% object Matrix {
%   def fill[T](rowSize: Int, colSize: Int)(init: T): Matrix[T] =
%     new Matrix(Vector.fill(rowSize)(Vector.fill(colSize)(init)))
% }
% \end{ScalaSpec}
%
% \SOLUTION
%
%
% \TaskSolved \what
%      %%%TODO number  8 %%%starts with: \label{task:generic-matrix} \em%%%
%
% \SubtaskSolved  -- %%%TODO in task 8 %%%
%
%
%
% \QUESTEND
%

% \clearpage
%
% \WHAT{Skapa en Sprite-editor.}
%
% \QUESTBEGIN
%
% \Task  \what~ Använd matrisklassen från uppgift \ref{task:generic-matrix} för att göra en SpriteEditor med JColorChoser enligt nedan skiss.
%
% \begin{Code}
% object ColorChooser {
%   import java.awt.Color
%   import javax.swing.JColorChooser
%
%   var title = "Pick Color"
%   private val chooser = new JColorChooser(Color.BLACK)
%   private val dialog = JColorChooser.
%     createDialog(null, title, true, jcs, null, null)
%
%   def getColor(initColor: Color = Color.BLACK): Color = {
%     chooser.setColor(initColor)
%     dialog.setVisible(true)
%     chooser.getColor
%   }
% }
%
% class Sprite(// en bild med många lager av pixlar i olika färger
%   val id: String,
%   val size: (Int, Int),
%   val pixels: Matrix[Int],   // färg i colors, -1 betyder genomskinlig
%   var scale: Int,            // uppskalning av storlek i pixlar
%   var colors: Vector[Color], // tillgängliga färger
%   var pos: (Int, Int, Int)   // (row, col, layer)
% ){
%   def row = pos._1
%   def col = pos._2
%   def layer = pos._3
% }
%
% class SpriteEditor(
%     rows: Int = 64, cols: Int = 64,
%     scale: Int = 16, nColors: Int = 16) {
%   private val w = new SimpleWindow(???)
%   def edit: Unit = ???
% }
%
% \end{Code}
%
%
%
% \SOLUTION
%
%
% \TaskSolved \what
%      %%%TODO number  9 %%%starts with: \TODO \emph{Klasser för täta oc%%%
%
% \SubtaskSolved  -- %%%TODO in task 9 %%%
%
% \SubtaskSolved  -- %%%TODO in task 9 %%%
%
% \SubtaskSolved  -- %%%TODO in task 9 %%%
%
% \SubtaskSolved  -- %%%TODO in task 9 %%%
%
% \SubtaskSolved  -- %%%TODO in task 9 %%%
%
% \SubtaskSolved  -- %%%TODO in task 9 %%%
%
%
%
% \QUESTEND




% \WHAT{Klasser för täta och glesa matematiska matriser med flyttal.}
%
% \QUESTBEGIN
%
% \Task  \what~   Läs om matrisräkning här: \href{https://sv.wikipedia.org/wiki/Matris}{sv.wikipedia.org/wiki/Matris}
%
% \Subtask Skapa en oföränderlig klass \code{DenseMatrix} för matematiska matriser med dubbelprecisionsflyttal. \code{DenseMatrix} ska internt lagra elementen i en privat \emph{endimensionell} array av flyttal av typen \code{Array[Double]}.
%
% Klassen ska inte vara en case-klass. Det ska gå att skapa matriser med uttryck så som  \code{DenseMatrix.ofDim(3,7)(1.0,42,3.2,1.0,2.2,3)} tack vare ett kompanjonsobjekt med lämplig fabriksmetod som anropar den privata konstruktorn.  Om antalet element är för litet i förhållande till den angivna dimensionen så fyll på med nollor.
%
% \Subtask Överskugga metoderna equals och hashcode och ge \code{DenseMatrix} innehållslikhet i stället för referenslikhet.
%
% \Subtask Implementera egna innehålllikhetsmetoder med namnet \code{===} på \code{DenseMatrix} som är typsäker, d.v.s. bara tillåter innehållsjämförelse mellan täta matriser.
%
% \Subtask Läs om glesa matriser här: \href{https://sv.wikipedia.org/wiki/Gles_matris}{https://sv.wikipedia.org/wiki/Gles\_matris} och implementera \code{SparseMatrix} med ett privat attribut av typen \\ \code{mutable.Map[(Int, Int), Double]} som bara lagrar index som inte är noll.
%
% \Subtask Skapa ett \code{trait Matrix} som både \code{DenseMatrix} och \code{SparseMatrix} ärver, med lämpliga abstrakta och konkreta medlemmar. Implementera addition, subtraktion och multiplikation av täta och glesa matriser.
%
% %\Task \emph{Matriser med \jcode{ArrayList} i Java.} Om man i Java inte vet antalet element i matrisen från början kan man använda en lista av typen \jcode{ArrayList}, där varje element i sin tur innehåller en lista av typen\jcode{ArrayList}. Javas \jcode{ArrayList} är en generisk samling som motsvaras av Scalas \code{ArrayBuffer}. Generiska samlingar i Java kan endast innehålla referenstyper; vill man ha en primitiv typ, t.ex. \jcode{int}, behöver man packa in denna i en s.k. wrapper-klass, t.ex.  klassen \jcode{Integer}. Det finns en wrapper-klass för varje primitiv typ i Java. Matristypen för en heltalstyp i Java skrivs \jcode{ArrayList<ArrayList<Integer>>} där alltså \code{<T>} motsvarar Scalas hakparenteser \code{[T]} för typparametern T.
% %
% %
%
% \SOLUTION
%
% \TaskSolved \what
%      %%%TODO number  10 %%%starts with: \emph{Matriser med \jcode{Array%%%
%
% \SubtaskSolved  -- %%%TODO in task 10 %%%
% \QUESTEND

%!TEX encoding = UTF-8 Unicode
%!TEX root = ../exercises.tex

\ifPreSolution


\Exercise{\ExeWeekNINE}\label{exe:W09}

\begin{Goals}
\input{modules/w09-setmap-exercise-goals.tex}
\end{Goals}

\begin{Preparations}
\item \StudyTheory{09}
\end{Preparations}

\else

\ExerciseSolution{\ExeWeekNINE}

\fi



\BasicTasks %%%%%%%%%%%%%%%%




\WHAT{Para ihop begrepp med beskrivning.}

\QUESTBEGIN

\Task \what

\vspace{1em}\noindent Koppla varje begrepp med den (förenklade) beskrivning som passar bäst:

\begin{ConceptConnections}
\input{generated/quiz-w09-concepts-taskrows-generated.tex}
\end{ConceptConnections}

\SOLUTION

\TaskSolved \what

\begin{ConceptConnections}
\input{generated/quiz-w09-concepts-solurows-generated.tex}
\end{ConceptConnections}

\QUESTEND



\WHAT{Vad är en mängd?}
\QUESTBEGIN

\Task \what~ Förklara vad som händer nedan. Varför hamnar elementen i en ''konstig'' ordning? Varför ''försvinner'' det element?

\begin{REPL}
scala> val xs = Vector(1,2,3,1,2,3,4,5,7).toSet
xs: scala.collection.immutable.Set[Int] = Set(5, 1, 2, 7, 3, 4)
scala> xs.foreach(print)
512734
\end{REPL}

\SOLUTION

\TaskSolved \what~En mängd är en samling som snabbt kan ge svaret på frågan om ett visst element ingår i samlingen eller ej. Elementen i en mängd är unika. Tilläg av redan existerande element ignoreras. En mängd är inte en  sekvens, eftersom traversering med t.ex. \code{map} eller \code{foreach} inte (nödvändigtvis) sker i den ordning som elementen gavs när mängden konstruerades eller uppdaterades.

\QUESTEND


\WHAT{Använda mängder.}

\QUESTBEGIN

\Task \what

\vspace{1em}\noindent Para ihop varje uttryck till vänster med ett uttryck till höger som har samma värde:

\begin{ConceptConnections}
\input{generated/quiz-w09-setops-taskrows-generated.tex}
\end{ConceptConnections}

\SOLUTION

\TaskSolved \what

\begin{ConceptConnections}
\input{generated/quiz-w09-setops-solurows-generated.tex}
\end{ConceptConnections}

\QUESTEND


\WHAT{Räkna unika ord med hjälp av en mängd.}

\QUESTBEGIN

\Task \what~På veckans laboration ska vi göra automatisk språkbehandling av långa texter som vi delar upp i ord. Med metoden \code{s.split(' ').toVector} kan du dela upp en sträng \code{s} i en sekvens av ord, där \code{s} blivit uppdelad i många strängar vid varje blanktecken och alla blanktecken är borttagna.

\Subtask Använd metoderna \code{split} och \code{toSet} för skapa ett uttryck som beräknar hur många unika ord det finns i strängen \code{hej} nedan:
\begin{REPLnonum}
scala> val hej = "hej hej hemskt mycket hej"
\end{REPLnonum}

\Subtask Mängder är snabba på att kolla om ett element finns i mängden men du kan inte förvänta dig att elementen finns i någon viss ordning. Det finns en sekvenssamlingsmetod som skapar en sekvens med unika element ur en sekvens och behåller den ursprungliga ordningen. Vad heter metoden? \\\emph{Tips:} Leta i snabbreferensen eller sök på nätet. Metoden fungerar på alla samlingar som är av typen \code{Seq} och har ett namn som börjar med bokstäverna \code{di}.

\SOLUTION

\TaskSolved \what~

\SubtaskSolved
\begin{REPL}
scala> val hej = "hej hej hemskt mycket hej"
scala> val n = hej.split(' ').toSet.size
n: Int = 3
\end{REPL}

\SubtaskSolved Metoden \code{distinct} returnerar en sekvens med unika element och bibehållen ursprunglig ordning.

\QUESTEND




\WHAT{Skapa 2-tupler med metoden \code{->} som kan uttalas ''mappas till''.}

\QUESTBEGIN

\Task \what~Vi har tidigare sett hur två olika värden kan samlas i en 2-tupel, till exempel \code{(0, true)}. Par kan även skapas med hjälp av metoden \code{->} enligt nedan. Testa detta i REPL:
\begin{REPL}
scala> ("Skåne", "Lund")          // ett strängpar med vanlig 2-tupel
scala> "Skåne" -> "Lund"           // operatornotation med ->
scala> "Skåne".->("Lund")         // punktnotation med -> (inte alls vanligt)
\end{REPL}
Metoden \code{->} fungerar med alla typer och är en fabriksmetod för par. Metodnamnet liknar en högerpil och illustrerar en mappning från första till andra värdet.

\Subtask Fungerar det på par skapade med \code{->} att använda metoderna \code{_1} och \code{_2}?


\Subtask Deklarera en variabel \code{val huvudstad: Vector[(String, String)]} som innehåller mappningar mellan geografiska områden och deras huvudstäder enligt tabellen nedan.

\begin{table}[H]
  \renewcommand{\arraystretch}{1.2}
  \begin{tabular}{|l|l|}\hline
  Sverige & Stockholm \\\hline
  Danmark & Köpenhamn \\\hline
  Grönland & Nuuk \\\hline
  Skåne & Lund \\\hline
  \end{tabular}
\end{table}

\Subtask Skriv ett uttryck som plockar fram \code{"Lund"} ur \code{huvudstad}.

\SOLUTION


\TaskSolved \what

\SubtaskSolved Ja, fabriksmetoden returnerar ett helt vanligt par:
\begin{REPLnonum}
scala> val härBorJag = "Skåne" -> "Lund"
val härBorJag: (String, String) = (Skåne,Lund)

scala> härBorJag._1
val res0: String = Skåne

scala> härBorJag._2
val res1: String = Lund
\end{REPLnonum}


\SubtaskSolved

\begin{Code}
val huvudstad = Vector(
  "Sverige"  -> "Stockholm",
  "Danmark"  -> "Köpenhamn",
  "Grönland" -> "Nuuk",
  "Skåne"    -> "Lund"
)
\end{Code}

\SubtaskSolved
\begin{REPL}
scala> huvudstad(3)._2
val res2: String = Lund
\end{REPL}

\QUESTEND



\WHAT{Linjärsöka efter nyckel i sekvens av mappningar.}

\QUESTBEGIN

\Task \what~

\Subtask Implementera funktionen \code{lookupIndex} nedan med hjälp av samlingsmetoden \code{indexWhere} så att linjärsökning sker efter index för ett par i sekvensen där \code{key} finns på första platsen i paret.

\begin{Code}
def lookupIndex(xs: Vector[(String, String)])(key: String): Int = ???
\end{Code}

\Subtask Testa din funktion i REPL genom att slå upp index för Skånes huvudstad i sekvensen \code{huvudstad} från föregående uppgift.

\SOLUTION

\TaskSolved \what~

\SubtaskSolved
\begin{Code}
def lookupIndex(xs: Vector[(String, String)])(key: String): Int =
  xs.indexWhere(_._1 == key)
\end{Code}

\SubtaskSolved
\begin{REPL}
scala> val i = lookupIndex(huvudstad)("Skåne")
val i: Int = 3

scala> huvudstad(i)._2
val res2: String = Lund
\end{REPL}

\noindent Eller med funktioner som återanvändbara dellösningar:
\begin{REPL}
scala> val indexOf = lookupIndex(huvudstad) _

scala> def capital(key: String) = huvudstad(indexOf(key))._2

scala> capital("Skåne")
val res3: String = Lund

scala> capital("Sverige")
val res4: String = Stockholm
\end{REPL}

\QUESTEND



\WHAT{Nyckel-värde-tabell.}

\QUESTBEGIN

\Task \what~En nyckel-värde-tabell är en smart datastruktur som gör att du kan slå upp det värde som en nyckel mappar till \emph{utan} att linjärsökning behöver ske. Värdet plockas fram direkt på en konstant tid, d.v.s. tiden att slå upp ett värde beror \emph{inte} på antalet element i samlingen, utan sker med mycket liten fördröjning.

I Scala heter nyckelvärdetabeller \code{Map} med stort M och är praktiska att använda i många olika sammanhang. \code{Map} finns i både en oföränderlig och en förändringsbar variant. Det går med metoder på formen \code{toXXX} lätt att omvandla mellan en \code{Map} och en sekvens av par av typen \code{XXX[(Nyckeltyp, Värdetyp)]}.

\Subtask Deklarera mappen \code{telnr} nedan i REPL och använd \code{apply} för att ta reda på telefonnumret till Fröken Ur.

\Subtask Vad har \code{telnr} för typ?

\Subtask Vad har \code{telnr.toVector} för typ?

\begin{Code}
val telnr = Map(
  "Anna"     -> 46462229812L,
  "Björn"     -> 46462229009L,
  "Sandra"    -> 46462220368L,
  "Fröken Ur" -> 4690510L,
)
\end{Code}
En uppsättning \code{Map}-instanser, vid behov nästlade, kan med fördel användas för att bygga upp en i-minnet-databas där inbyggda samlingsmetoder, t.ex. \code{map}, \code{filter}, och \code{for}-\code{yield}-uttryck, ger flexibla och effektiva sökmöjligheter. På veckans laboration ska du göra detta.

Samlingen \code{Map} är en generalisering av en sekvens, där man kan ''indexera'', inte bara med ett heltal, utan med vilken typ av värde som helst, t.ex. en sträng. Datastrukturen \code{Map} kallas också \emph{associativ array}\footnote{\href{https://en.wikipedia.org/wiki/Associative_array}{https://en.wikipedia.org/wiki/Associative\_array}} och är implementerad som en s.k. \emph{hashtabell}\footnote{\href{https://en.wikipedia.org/wiki/Hash_table}{https://en.wikipedia.org/wiki/Hash\_table}}, men du får vänta till fördjupningskursen innan vi går igenom hur en sådan datastruktur implementeras.

\SOLUTION

\TaskSolved \what~

\begin{REPL}
scala> telnr("Fröken Ur")
val res0: Long = 464690510

scala> :type telnr
Map[String,Long]

scala> :type telnr.toVector
Vector[(String, Long)]
\end{REPL}

\QUESTEND



\WHAT{Använda nyckel-värdetabell.}

\QUESTBEGIN

\Task \what~

\Subtask Skapa nedan variabler i REPL.
\begin{Code}
val follow = for i <- 2 to 16 by 2 yield (i, i + 1)
val xs = follow.toMap
val ys = xs.toVector
\end{Code}
Hamnar mappningarna i \code{ys} i samma ordning som \code{follow}? Varför?

\Subtask Med \code{xs} och \code{ys} deklarerade i REPL enligt ovan, para ihop yttryck till vänster med rätt resultat till höger. Om du är osäker på de sammansatta uttrycken, prova enklare uttryck i REPL och undersök värde och typ hos delresultat.

\begin{ConceptConnections}
\input{generated/quiz-w09-mapops-taskrows-generated.tex}
\end{ConceptConnections}

\SOLUTION

\TaskSolved \what


\SubtaskSolved Nej nyckel-värde-paren lagras i någon speciell ordning som bestäms av en intern, smart lagringsprincip enligt en s.k. hashfunktion\footnote{\url{https://sv.wikipedia.org/wiki/Hashfunktion}}, för att åstadkomma snabba uppslagningar av värden från nycklar och vilket normalt inte sammanfaller med ordningen i den sekvens som de skapades ur.

\SubtaskSolved

\begin{ConceptConnections}
  \input{generated/quiz-w09-mapops-solurows-generated.tex}
\end{ConceptConnections}

%%% BELOW IS SOLVED IN SCALA 3 AND the err msg is better! :)
% \noindent \emph{Fördjupning}:  Felmeddelandet som rad 2 ovan orsakar är lurigt:

% \begin{REPL}
% scala> ys(2)
% val res22: (Int, Int) = (6,7)

% scala> ys(4)
% val res23: (Int, Int) = (12,13)

% scala> ys(2) + ys(4)
% <console>:13: error: type mismatch;
%  found   : (Int, Int)
%  required: String
%        ys(2) + ys(4)

% \end{REPL}
% Det går som förväntat inte att addera två tupler, men varför säger kompilatorn att en sträng krävs?!? Detta beror på att, i enlighet med hur det fungerar i Java, valde Scala-språkets konstruktörer att låta strängsammanfogning fungera med alla möjliga typer vilket gör att kompilatorn inte ger upp när metoden \code{+} inte finns för tupler, utan i stället gör ett misslyckat försök med strängsammanfogning.

% Det mest olyckliga med detta är inte att felmeddelanden ibland blir missvisande, utan att det i vissa situationer inte ens \emph{blir} något felmeddelande, trots att man av rent misstag råkat strängkonkatenera i stället för t.ex. lägga till ett element i en mängd eller en mappning i en tabell. Detta typosäkra beteendet av strängsammanfogning har kritiserats, men det är inte okontroversiellt att ändra detta nu när så många utvecklare skrivit så mycket Scala-kod som bygger på strängars förmåga att kunna lägga till vad som helst på slutet. Situationen i Scala är dock inte hopplös efter introduktionen av stränginterpolering i Scala 2.10, som möjliggör infogande av värden i strängar på ett typsäkert sätt.
\QUESTEND





\WHAT{Registrering i förändringsbar nyckel-värde-tabell.}

\QUESTBEGIN

\Task \what~I denna uppgift ska du implementera en hjälpklass för registrering i en frekvenstabell som du sedan ska använda på veckans laboration. Klassen ska heta  \code{FreqMapBuilder} som efter upprepade anrop av metoden \code{add(s: String): Unit} kan skapa frekvenstabeller av typen \code{Map[String, Int]}, där nyckel-värde-paren i tabellen anger antalet förekomster av en viss sträng. Du ska utgå från koden nedan.

Klassen använder en förändringsbar tabell internt. Efter att man har lagt till många strängar kan man med metoden \code{toMap} få en oföränderlig tabell för  uppslagning av frekvenser för specifika strängar. Läs i snabbreferensen om vilka extra metoder för uppdatering som erbjuds av \code{mutable.Map[K, V]}.

\begin{Code}
class FreqMapBuilder:
  private val register = collection.mutable.Map.empty[String, Int]
  def toMap: Map[String, Int] = register.toMap
  def add(s: String): Unit = ???

object FreqMapBuilder:
  def apply(xs: String*): FreqMapBuilder = ???
\end{Code}

\noindent Implementera och testa \code{FreqMapBuilder}. \emph{Tips:} Du kan t.ex. använda \code{mutable.Map}-metoderna \code{addOne} och \code{getOrElse}.

\SOLUTION

\TaskSolved \what~
\begin{Code}
class FreqMapBuilder:
  private val register = scala.collection.mutable.Map.empty[String,Int]
  def toMap: Map[String, Int] = register.toMap
  def add(s: String): Unit =
    register.addOne(s -> (register.getOrElse(s, 0) + 1))

object FreqMapBuilder:
  def apply(xs: String*): FreqMapBuilder = 
    val result = new FreqMapBuilder
    xs.foreach(result.add)
    result
\end{Code}

\QUESTEND



\WHAT{Metoden \code{sliding}.}

\QUESTBEGIN

\Task  \what~  I veckans laboration kommer du att ha nytta av metoden \code{sliding}, som ger en iterator för speciella delsekvenser av en sekvens, vilka kan liknas vid ''utsikten'' i ett ''glidande fönster''.

\Subtask Kör nedan i REPL och beskriv vad som händer.

\begin{REPL}
scala> val xs = Vector("fem", "gurkor", "är", "fler", "än", "fyra", "tomater")
scala> xs.sliding(2).toVector
scala> xs.sliding(3).toVector
scala> xs.sliding(10).toVector
\end{REPL}

\Subtask Använd \code{xs.sliding(2)} och omvandla varje element i resultatet till ett par. Gör sedan om sekvensen av par till en nyckel-värde-tabell. Vad kan tabellen användas till?

\SOLUTION

\TaskSolved \what

\SubtaskSolved
\begin{REPL}
scala> val xs = Vector("fem", "gurkor", "är", "fler", "än", "fyra", "tomater")
val xs: Vector[String] =
  Vector(fem, gurkor, är, fler, än, fyra, tomater)

scala> xs.sliding(2).toVector
val res9: Vector[Vector[String]] =
  Vector(Vector(fem, gurkor), Vector(gurkor, är), Vector(är, fler), Vector(fler, än), Vector(än, fyra), Vector(fyra, tomater))

scala> xs.sliding(3).toVector
val res10: Vector[Vector[String]] =
  Vector(Vector(fem, gurkor, är), Vector(gurkor, är, fler), Vector(är, fler, än), Vector(fler, än, fyra), Vector(än, fyra, tomater))

scala> xs.sliding(10).toVector
val res11: Vector[Vector[String]] =
  Vector(Vector(fem, gurkor, är, fler, än, fyra, tomater))

\end{REPL}
\code{xs.sliding(n).toVector} skapar en sekvens som innehåller sekvenser av längden \code{n} som bildas genom att ta varje element och dess \code{n - 1} efterföljande element.

\SubtaskSolved
\begin{REPL}
scala> xs.sliding(2).map(ys => ys(0) -> ys(1)).toMap
val res0: Map[String,String] =
  Map(är -> fler,
      än -> fyra,
      fyra -> tomater,
      gurkor -> är,
      fem -> gurkor,
      fler -> än
  )
\end{REPL}
Man kan använda tabellen till att slå upp vilket som är efterföljande ord. Det fungerar eftersom alla ord är unika. Om det funnits flera likadana ord med olika efterföljande ord så hade vi behövt skapa en tabell med nycklar som mappar till en samling som registrerar efterföljande ord. Detta ska vi göra på veckans laboration.

\QUESTEND




\WHAT{Läsa text från fil och webbservrar.}

\QUESTBEGIN

\Task \what~På laborationen ska du bygga upp tabeller från data i textformat. Då har du nytta av att kunna läsa text från filer och från webben. Testa detta i REPL:
\begin{REPL}
scala> val url = "https://fileadmin.cs.lth.se/pgk/europa.txt"
scala> val xs = io.Source.fromURL(url, "UTF-8").getLines.toVector
scala> val data = xs.map(_.split(';').toVector)
scala> data.head
scala> data.foreach(println)
\end{REPL}

\Subtask Skapa dessa tabeller ur sekvensen \code{data}:
\begin{Code}
val populationOf: Map[String, Int]    = ???  // länders invånarantal
val sizeOf:       Map[String, Int]    = ???  // länders yta i km^2
val capitalOf:    Map[String, String] = ???  // länders huvudstäder
\end{Code}
Testa tabellerna i REPL.

\Subtask Spara ner data i en textfil \code{europa.txt}. Läsa in data från filen med metoden\\ \code{Source.fromFile(filnamn, teckenkodning)} på liknande sätt som med  \code{fromURL} ovan. Om du kör i en Linux-terminal kan du enkelt ladda ner en fil så här (notera att det är stora bokstaven \code{O} och inte en nolla i optionen \code{-sLO}):
\begin{REPLnonum}
> curl -sLO https://fileadmin.cs.lth.se/pgk/europa.txt
\end{REPLnonum}
Skriv ut alla raderna i \code{europa.txt} med hjälp av \code{Source.fromFile} i REPL.

\SOLUTION

\TaskSolved \what~

\SubtaskSolved
\begin{CodeSmall}
val populationOf = data.tail.map(v => v(0) -> v(1).toInt).toMap
val sizeOf       = data.tail.map(v => v(0) -> v(2).toInt).toMap
val capitalOf    = data.tail.map(v => v(0) -> v(3)).toMap
\end{CodeSmall}

\begin{REPL}
scala> capitalOf("Sverige")
res2: String = Stockholm

scala> populationOf("Sverige")
res3: Int = 9223766

scala> sizeOf("Sverige")
res4: Int = 449964
\end{REPL}

\begin{REPL}
scala> val filename = "europa.txt"
scala> val xs = io.Source.fromFile(filename, "UTF-8").getLines.toVector
scala> val data = xs.map(_.split(';').toVector)
scala> data.map(_.map(_.take(15).padTo(15,' ')).mkString(" ")).foreach(println)
\end{REPL}
\QUESTEND





\ExtraTasks %%%%%%%%%%%%%%%%%%%%%%%%%%%%%%%%%%%%%%%%%%%%%%%%%%%%%%%%%%%%%%%%%%%%

\WHAT{Skapa ett textspel med hjälp av tabeller.}

\QUESTBEGIN

\Task \what~Gör ett enkelt spel för att träna på olika fakta om Europas länder och huvudstäder genom att läsa data från URL:en:\\ \url{https://fileadmin.cs.lth.se/pgk/europa.txt}
\\Där finns text kodad i UTF-8 med följande innehåll (endast de första raderna visas):
\begin{Code}
Land;Invånarantal;Storlek(km^2);Huvudstad
Albanien;3581655;28748;Tirana
Andorra;71201;468;Andorra la Vella
Belgien;10584534;30528;Bryssel
Bosnien-Hercegovina;4590310;51129;Sarajevo
Bulgarien;7385367;110910;Sofia
Cypern;854000;9250;Nicosia
Danmark;5475791;43094;Köpenhamn
Estland;1324333;45226;Tallinn
Finland;5315280;338145;Helsingfors
Frankrike;61538322;551695;Paris
Färöarna;48344;139574;Torshamn
Grekland;10964021;131940;Aten
// ... etcetera för alla Europas länder.
\end{Code}
Låt till exempel användaren svara på slumpvisa frågor av typen:
\begin{itemize}[noitemsep]
  \item Har Andorra fler invånare än Cypern?
  \item Vad heter huvudstaden i Bulgarien?
  \item Har Danmark större yta än Finland?
\end{itemize}
Använd oföränderliga tabeller med lämpliga nycklar och värden. Du kan använda en mängd med länder/huvudstäder som användaren hittills svarat rätt på för att kunna förhindra att dessa återkommer igen.
\SOLUTION

\TaskSolved --

\QUESTEND



\AdvancedTasks %%%%%%%%%%%%%%%%%%%%%%%%%%%%%%%%%%%%%%%%%%%%%%%%%%%%%%%%%%%%%%%%%


\WHAT{Registrering med \code{groupBy}.}

\QUESTBEGIN

\Task \what~Vi ska nu utnyttja ett riktigt listigt trick för att via en enda kodrad implementera registrering med hjälp av samlingsmetoderna \code{groupBy} och \code{map}.

\Subtask Läs om metoden \code{groupBy} i snabbreferensen. Du hittar den under rubriken \emph{''Methods in trait \code{Iterable[A]}''} eftersom \code{groupBy} fungerar på alla samlingar. Testa \code{groupBy} enligt nedan och beskriv vad som händer.

\begin{REPL}
scala> val xs = Vector(1, 1, 2, 2, 4, 4, 4).groupBy(x => x > 2)
scala> val ys = Vector(1, 1, 2, 2, 4, 4, 4).groupBy(x => x)
\end{REPL}

\Subtask Skapa en funktion \code{freq} med nedan funktionshuvud som returnerar en tabell med antalet förekomster av olika heltal i \code{xs}. Testa \code{freq} på en sekvens av 1000 slumpvisa tärningskast och förklara hur funktionen \code{freq} fungerar. \emph{Tips:} Gör först \code{groupBy(???)} och sedan \code{map(???)}.

\begin{Code}
def freq(xs: Vector[Int]): Map[Int, Int] = ???

def kasta(n: Int): Vector[Int] =
  Vector.fill(n)(scala.util.Random.nextInt(6) + 1)
\end{Code}

\SOLUTION

\TaskSolved \what~

\SubtaskSolved Metoden \code{groupBy} skapar en nyckel-värde-tabell där värdena i tabellen är en sekvens med elementen grupperade på ett speciellt sett.
Mer precist:

Resultatet av \code{xs.groupBy(f: K => V)} för en sekvens \code{xs} av typen \code{Vector[K]} blir en tabell av typen \code{Map[V,Vector[K]]} där varje element \code{e} i \code{xs} är grupperade i samma tabellvärde om de lika är enligt \code{f(e)}. Varje grupp får tabellnyckeln \code{f(e)}.

\emph{Listigt trick:} Om man låter funktionen \code{f} vara enhetsfunktionen som avbildar varje element på sig själv, alltså \code{x => x}, så grupperas värdena i samma sekvens om de är lika.

\begin{REPL}
scala> val xs = Vector(1, 1, 2, 2, 4, 4, 4).groupBy(x => x > 2)
val xs: Map[Boolean,Vector[Int]] =
  Map(false -> Vector(1, 1, 2, 2), true -> Vector(4, 4, 4))

scala> val ys = Vector(1, 1, 2, 2, 4, 4, 4).groupBy(x => x)
val ys: Map[Int,Vector[Int]] =
  Map(2 -> Vector(2, 2), 4 -> Vector(4, 4, 4), 1 -> Vector(1, 1))
\end{REPL}


\SubtaskSolved

\begin{Code}
def freq(xs: Vector[Int]): Map[Int, Int] =
  xs.groupBy(x => x).map(p => p._1 -> p._2.size)
\end{Code}
Förklaring: metoden \code{groupBy} skapar en tabell med par \code{k, v} där \code{v} är en sekvens med så många \code{k} som antalet gånger \code{k} förekommer i \code{xs}. Genom att omvandla alla värden \code{p._2} till storleken \code{p._2.size} får vi en frekvenstabell.

\begin{REPL}
scala> freq(kasta(1000))
val res0: Map[Int,Int] = 
  Map(5 -> 163, 1 -> 174, 6 -> 161, 2 -> 169, 3 -> 167, 4 -> 166)

scala> freq(kasta(1000)).toVector.sortBy(_._1).foreach(println)
(1,183)
(2,167)
(3,169)
(4,179)
(5,154)
(6,148)
\end{REPL}

\QUESTEND





\WHAT{Skriva till fil.}

\QUESTBEGIN

\Task \what~Som hjälp när du skapar egna intressanta applikationer eller bygger vidare på kursens laborationer och övningar med frivilliga extrauppgifter, kan du använda funktionerna i singelobjektet \code{IO} nedan, som finns i kursens scala-bibliotek \href{https://fileadmin.cs.lth.se/pgk/api}{introprog}.\footnote{Källkoden finns här och även på sidan \pageref{disk-access-code}:\\ \href{https://github.com/lunduniversity/introprog/blob/master/compendium/workspace/introprog/src/main/scala/introprog/IO.scala}{https://github.com/lunduniversity/introprog-scalalib/blob/master/src/main/scala/introprog/IO.scala}}

IO-modulen använder \code{scala.io.Source} för att serialisera och de-serialisera strängar till och från vanliga textfiler. IO-modulen använder även paketet \code{java.io} för att erbjuda funktioner som gör det enkelt att serialisera/de-serialisera godtyckliga objekt skapade med hjälp av serialserbara klasser till/från binärfiler. Case-klasser i Scala blir automatiskt serialiserbara.

I implementationen av \code{IO} används \code{try ... finally} för att säkerställa att filer inte lämnas öppnade även om något går fel under den läs/skriv-process som sköts av det underliggande operativsystemet.

\Subtask
Kompilera och resta nedan med \code{introprog} på classpath, t.ex. med hjälp av \code{sbt}.
\begin{Code}
import introprog.IO

case class Player(name: String)

@main def run(): Unit = 
  println("Test of output/input objects to/from disk:")
  val highscores = Map(Player("Sandra") -> 42, Player("Björn") -> 5)
  IO.saveObject(highscores,"highscores.ser")
  val highscores2 = IO.loadObject[Map[Player, Int]]("highscores.ser")
  val isSameContents = highscores2 == highscores
  val testResult = if (isSameContents) "SUCCESS :)" else "FAILURE :("
  println(testResult)
\end{Code}

\Subtask
Använd \code{IO}-modulen för att spara användarens poängresultat i ditt spel om Europas länder och städer, i extrauppgiften ovan. Implementationen av \code{introprog.IO} finns här: \url{https://github.com/lunduniversity/introprog-scalalib/blob/master/src/main/scala/introprog/IO.scala} 

% \begin{figure}
% %  \scalainputlisting[basicstyle=\ttfamily\fontsize{9.2}{11}\selectfont]{examples/IO.scala}
%   \scalainputlisting[basicstyle=\ttfamily\fontsize{9.2}{11}\selectfont]{../workspace/introprog/src/main/scala/introprog/IO.scala}
%   \label{disk-access-code}
% \end{figure}
\SOLUTION

\TaskSolved --

\QUESTEND



%
%
% \subsection{\TODO Värdera nedan gamla uppgifter}
%
%
%
% \WHAT{Objekt med attribut (fält).}
%
% \QUESTBEGIN
%
% \Task  \what~  Ett objekt kan samla data som hör ihop och på så sätt skapa en datastruktur. Data i ett objekt kallas \emph{attribut} eller \emph{fält}, \Eng{field}. Objekt som samlar enbart data kallas även \emph{post} \Eng{record}.
% \begin{REPLnonum}
% scala> object mittKonto { var saldo = 0; val nummer = 12345L }
% \end{REPLnonum}
% \Subtask Skriv en sats som sätter in ett slumpmässigt belopp mellan 0 och en miljon på \code{mittKonto} ovan med hjälp av punktnotation och tilldelning.
%
% \Subtask Vad händer om du försöker ändra attributet \code{nummer}?
%
% \SOLUTION
%
%
% \TaskSolved \what
%
%
% \SubtaskSolved   \code{mittKonto.saldo = (math.random() * 1000000).toInt}
%
% \SubtaskSolved   Går ej eftersom val är oföränderlig, man får alltså ett Error.
%
%
% \QUESTEND
%
%
%
%
% %%<AUTOEXTRACTED by mergesolu>%%      %Uppgift 2
%
%
%
%
% \WHAT{Klass med attribut.}
%
% \QUESTBEGIN
%
% \Task  \what~  Om du vill ha många objekt av samma typ, kan du använda en \textbf{klass}. På så sätt kan man skapa många datastrukturer av samma typ men med olika innehåll. Man skapar nya objekt med nyckelordet \code{new} följt av klassens namn. Klassen utgör en ''mall'' för objektet som skapas. Ett objekt som skapas med \code{new Klassnamn} kallas även en \textbf{instans} av klassen \code{Klassnamn}. Nedan skapas en datastruktur \code{Konto} som samlar data om ett bankonto. Instanser av typen \code{Konto} håller reda på hur mycket pengar det finns på kontot och vilket kontonumret är. Datavärden som sparas i varje objektinstans, så som \code{saldo} och \code{nummer}, kallas \textbf{attribut} \Eng{attribute} eller \textbf{fält} \Eng{field}.
%
% \begin{REPL}
% scala> class Konto {
%          var saldo = 0
%          var nummer = 0L
%        }
% scala> val k1 = new Konto
% scala> val k2 = new Konto
% scala> k1.saldo = 1000
% scala> k1.nummer = 12345L
% scala> k2.saldo = 2000
% scala> k2.nummer = 67890L
% scala> println("Konto: " + k1.nummer + " Saldo:" + k1.saldo)
% scala> println("Konto: " + k2.nummer + " Saldo:" + k2.saldo)
% \end{REPL}
%
% \Subtask\Pen Rita hur minnessituationen ser ut efter att ovan rader har exekverats.
%
% \Subtask\Pen Vad hade det fått för konsekvenser om attributet \code{nummer} vore oföränderligt i klassen ovan? (Jämför med objektet \code{mittKonto}.)
%
%
% \SOLUTION
%
%
% \TaskSolved \what
%
%
% \SubtaskSolved   \includegraphics[scale=0.5]{../img/w04-solutions/uppgift-3a}
%
% \SubtaskSolved
% Tilldelningen på rad 8 \code{k1.nummer = 12345L} ger felmeddelande eftersom variablen är oföränderlig.
%
%
% \QUESTEND
%
%
%
%
% %%<AUTOEXTRACTED by mergesolu>%%      %Uppgift 3
%
%
%
%
% \WHAT{Klass med attribut som parametrar.}
%
% \QUESTBEGIN
%
% \Task  \what~  Om man vill ge attributen initialvärden när objektet skapas med \code{new}, kan man placera attributen i en parameterlista till klassen. Koden som körs när objektet skapas och attributen tilldelas sina initialvärden, kallas \textbf{konstruktor} \Eng{constructor}.
%
% \begin{REPL}
% scala> class Konto(var saldo: Int, val nummer: Long)
% scala> val k = new Konto(0, 12345L)
% scala> println("Konto: " + k.nummer + " Saldo:" + k.saldo)
% scala> println(k)
% scala> k.toString
% \end{REPL}
%
% \Subtask Den två sista raderna ovan skriver ut den identifierare som JVM använder för att hålla reda på objektet i sina interna datastrukturer. Vad skrivs ut?
%
% \Subtask Skapa ännu en instans av klassen Konto  med samma saldo och nummer som \code{k} ovan och spara den i \code{val k2} och undersök dess objektidentifierare. Får objekten \code{k} och \code{k2} olika objektidentifierare?
%
% \Subtask Sätt in olika belopp på respektive konto.
%
% \Subtask Vad händer om du försöker ändra attributet \code{nummer}?
%
% \Subtask\Pen Ibland räcker det fint med en tupel, men ofta vill man ha en klass istället. Beskriv några fördelar med en Konto-klassen ovan jämfört med en tupel av typen \code{(Int, Long)}.
%
% \begin{REPLnonum}
% scala> var k3 = (0, 12345L)
% scala> k3 = (k3._1 + 100, k3._2)
% \end{REPLnonum}
%
% \SOLUTION
%
%
% \TaskSolved \what
%
%
% \SubtaskSolved   \code{String = Konto@cd576}, där \code{Konto@cd576} är ett unikt namn som identifierar instansen.
%
% \SubtaskSolved   Ja.
%
% \SubtaskSolved
% \begin{REPLnonum}
% scala> k.saldo = 42
% scala> k2.saldo = 67
% \end{REPLnonum}
%
% \SubtaskSolved   Eftersom variablen är oföränderlig ges ett felmeddelande.
%
% \SubtaskSolved   En fördel med klass är att man kan specificera att variablen ska kunna vara föränderlig. En till är att man kan inkludera metoder i klassen som man vill kunna använda på värdena.
%
%
% \QUESTEND
%
%
%
%
% %%<AUTOEXTRACTED by mergesolu>%%      %Uppgift 4
%
%
%
%
% \WHAT{Publikt eller privat attribut?}
%
% \QUESTBEGIN
%
% \Task  \what~  Man kan förhindra att ett attribut syns utanför klassen med hjälp av nyckelordet \code{private}.
%
% \begin{REPL}
% scala> class Konto1(val nummer: Long){ var saldo = 0 }
% scala> val k1 = new Konto1(12345678901L)
% scala> k1.nummer
% scala> k1.saldo += 1000
% scala> class Konto2(val nummer: Long){ private var saldo = 0 }
% scala> val k2 = new Konto2(12345678901L)
% scala> k2.nummer
% scala> k2.saldo += 1000
% \end{REPL}
%
% \Subtask Vad händer ovan?
%
% \Subtask Gör en ny version av klassen \code{Konto} enligt nedan:
%
% \begin{Code}
% class Konto(val nummer: Long){
%   private var saldo = 0
%   def in(belopp: Int): Unit = {saldo += belopp}
%   def ut(belopp: Int): Unit = {saldo -= belopp}
%   def show: Unit =
%     println("Konto Nr: " + nummer + " saldo: " + saldo)
% }
%
% object Main {
%   def main(args: Array[String]): Unit = {
%     val k = new Konto(1234L)
%     k.show
%     k.in(1000)
%     println("Uttag: " + k.ut(500))
%     println("Uttag: " + k.ut(1000))
%     k.show
%   }
% }
% \end{Code}
%
% \Subtask Spara koden i en fil, kompilera med \code{scalac} och kör. Testa även vad som händer om du försöker komma åt attributet \code{saldo} i main-metoden med t.ex. \code{println(k.saldo)} eller \code{k.saldo += 1000}.
%
% \Subtask Vi ska nu förhindra överuttag. Ändra i metoden \code{ut} så att den får signaturen \code{ut(belopp: Int): (Int, Int) = ???} och implementera \code{ut} så att den returnerar både beloppet man verkligen kan ta ut och kvarvarande saldo. Om man försöker ta ut mer än det finns på kontot så ska saldot bli 0 och man får bara ut det som finns kvar. Spara, kompilera, kör.
%
% \Subtask Förbättra metoderna \code{in} och \code{ut} så att man inte kan sätta in eller ta ut negativa belopp.
%
% \Subtask Vad är fördelen med att göra föränderliga attribut privata och bara påverka deras värden indirekt via metoder?
%
% \SOLUTION
%
%
% \TaskSolved \what
%
%
% \SubtaskSolved
% Det går bra att ändra på variablen saldo i instansen av Konto1 men inte av Konto2 där man får ett error på raden ''k2.saldo += 1000''
%
% \SubtaskSolved  -
%
% \SubtaskSolved
% ''println(k.saldo)'' och ''k.saldo += 1000'' ger båda error, pga privat attribut.
%
% \SubtaskSolved
% \begin{Code}
% def ut(belopp: Int): (Int, Int) = {
% 	if(saldo >= belopp) {
% 		saldo -= belopp
% 		(belopp, saldo)
% 	} else {
% 		val temp = saldo
% 		saldo = 0
% 		(temp, 0)
% 	}
% }
% \end{Code}
%
% \SubtaskSolved
% Lägg till en if-sats i båda funktionerna som omsluter den gamla koden.
% \begin{Code}
% def ut(belopp: Int): (Int, Int) = {
%   if(belopp >= 0) {
%     if(saldo >= belopp) {
%       saldo -= belopp
%       (belopp, saldo)
%     } else {
%       val temp = saldo
%       saldo = 0
%       (temp, 0)
%     }
%   }
% }
%
% def in(belopp: Int): Unit = {
%   if(belopp >= 0) {
%     saldo += belopp
%   }
% }
% \end{Code}
%
% \SubtaskSolved
% Genom att göra attributet privat och gör egna metoder kan man se till att attriuten endast ändras på säkra sätt. Så inte fel uppstår.
%
%
% \QUESTEND
%
%
%
%
% %%<AUTOEXTRACTED by mergesolu>%%      %Uppgift 5
%
%
%
%
% \WHAT{Vilken typ har ett objekt?}
%
% \QUESTBEGIN
%
% \Task  \what~  Objektets typ bestäms av klassen. Vid tilldelning måste typerna passa ihop.
%
% \Subtask Vilka rader nedan ger felmeddelande? Hur lyder felmeddelandet?
% \begin{REPL}
% scala> class Punkt(val x: Double, val y: Double)
% scala> val pt: Punkt = new Punkt(10.0, 10.0)
% scala> val i: Int = pt.x
% scala> val (x: Double, y: Double) = (pt.x, pt.y)
% scala> val p: Double = new Punkt(5.0, 5.0)
% scala> val p = new Punkt(5.0, 5.0): Double
% scala> val p = new Punkt(5.0, 5.0): Punkt
% scala> pt: Punkt
% \end{REPL}
%
%
% \Subtask Man kan undersöka om ett objekt är av en viss typ med metoden \\ \code{isInstanceOf[Typnamn]}. Vad ger nedan anrop av metoden \code{isInstanceOf} för värde?
% \begin{REPL}
% scala> class Punkt(val x: Double, val y: Double)
% scala> val pt: Punkt = new Punkt(1.0, 2.0)
% scala> pt.isInstanceOf[Punkt]
% scala> pt.isInstanceOf[Double]
% scala> pt.x.isInstanceOf[Punkt]
% scala> pt.x.isInstanceOf[Double]
% scala> pt.x.isInstanceOf[Int]
% \end{REPL}
%
% \SOLUTION
%
%
% \TaskSolved \what
%
%
% \SubtaskSolved
% ''val i: Int = pt.x'' error: type mismatch;
% Eftersom typen Int ej är kompatibel med ett värde av typen Double.
%
% ''val p: Double = new Punkt(5.0, 5.0)'' error: type mismatch;
% Eftersom typen Double ej är kompatibel med ett värde av typen Punkt.
%
% ''val p = new Punkt(5.0, 5.0): Double'' error: type mismatch;
% Eftersom typen Double ej är kompatibel med ett värde av typen Punkt.
%
% \SubtaskSolved
% Rad 3 till 7 i respektive ordning: true, false, false, true och false.
%
%
% \QUESTEND
%
%
%
%
% %%<AUTOEXTRACTED by mergesolu>%%      %Uppgift 6
%
%
%
%
% \WHAT{Topptypen \code{Any}.}
%
% \QUESTBEGIN
%
% \Task  \what~ Alla klasser är också av typen \code{Any}. Alla klasser får därmed med sig några gemensamma metoder som finns i den fördefinierade klassen \code{Any}, däribland metoderna  \code{isInstanceOf} och \code{toString}.  Vad blir resultatet av respektive rad nedan? Vilken rad ger ett felmeddelande?
%
%
% \begin{REPL}
% scala> class Punkt(val x: Double, val y: Double)
% scala> val pt: Punkt = new Punkt(1.0, 2.0)
% scala> pt.isInstanceOf[Punkt]
% scala> pt.isInstanceOf[Any]
% scala> pt.x.toString
% scala> println(pt.x)
% scala> val a: Any = pt
% scala> println(a.x)
% scala> a.toString
% scala> pt.y.toString
% scala> a.y.toString
% \end{REPL}
%
% \SOLUTION
%
%
% \TaskSolved \what
%
% \begin{enumerate}
% \item Definierar klassen Punkt.
% \item En variabel pt: Punkt skapas.
% \item true
% \item true
% \item String = 1.0
% \item skriver ut: 1.0
% \item En variabel med namnet a skapas med typen Any.
% \item error: value x is not a member of Any
% \item a ges nu typen String
% \item String = 2.0
% \item error: value y is not a member of Any
% \end{enumerate}
%
%
% \QUESTEND
%
%
%
%
% %%<AUTOEXTRACTED by mergesolu>%%      %Uppgift 7
%
%
%
%
% \WHAT{Byta ut metoden \code{toString}}.
%
% \QUESTBEGIN
%
% \Task  \what~ I klassen \code{Any} finns metoden \code{toString} som skapar en strängrepresentation av objektet. Du kan byta ut metoden \code{toString} i klassen \code{Any} mot din egen implementation. Man använder nyckelordet \code{override} när man vill byta ut en metodimplementation.
%
% \begin{REPL}
% scala> class Punkt(val x: Double, val y: Double) {
%          override def toString: String = "[x=" + x + ",y=" + y + "]"
%        }
% scala> val pt = new Punkt(1.0, 42.0)
% scala> pt.toString
% scala> println(pt)
% \end{REPL}
%
% \Subtask Vad händer egentligen på sista raden ovan?
%
% \Subtask Omdefiniera toString så att den ger en sträng på formen \code{Punkt(1.0, 42.0)}.
%
% \Subtask Vad händer om du utelämnar nyckelordet \code{override} vid omdefiniering?
%
% \SOLUTION
%
%
% \TaskSolved \what
%
%
% \SubtaskSolved
% ''println(pt)'' kallar på pt.toString, och eftersom metoden är överskriven kallas den nya version.
%
% \SubtaskSolved   \code{override def toString: String = ''Punkt('' + x + '', '' + y + '').''}
%
% \SubtaskSolved
% error: overriding method toString in class Object of type ()String;
%
%
% \QUESTEND
%
%
%
%
% %%<AUTOEXTRACTED by mergesolu>%%      %Uppgift 8
%
%
%
%
% \WHAT{Objektfabrik med \code{apply}-metod.}
%
% \QUESTBEGIN
%
% \Task  \what~  Man kan ordna så att man slipper skriva \code{new} med ett s.k. \emph{fabriksobjekt} \Eng{factory object}.
% \begin{Code}
% class Pt(val x: Double, y: Double) {
%   override def toString: String = "Pt(x=" + x + ",y=" + y + ")"
% }
% object Pt {
%   def apply(x: Double, y: Double): Pt = new Pt(x, y)
% }
% \end{Code}
%
% \Subtask Skriv satser som använder metoden \code{apply} i fabriksobjektet \code{object Pt} för att skapa flera olika punkter.
%
% \Subtask Ge applymetoden default-argument 0.0 för både x och y så att \code{Pt()} skapar en punkt i origo.
%
% \Subtask Skapa en klass \code{Rational} som representerar rationellt tal som en kvot mellan två heltal. Ge klassen två oföränderliga, publika klassparameterattribut med namnen \code{nom} för täljaren och \code{denom} för nämnaren.
%
% \Subtask Skapa ett fabriksobjekt med en \code{apply}-metod som tar två heltalsparametrar och skapar en instans av klassen \code{Rational}.
%
% \Subtask Skapa olika instanser av din klass \code{Rational} ovan med hjälp av fabriksobjektet.
%
%
% \SOLUTION
%
%
% \TaskSolved \what
%
%
% \SubtaskSolved
% \begin{REPL}
% scala> val pt = Pt(1.0, 2.0)
% pt: Pt = Pt(x=1.0,y=2.0)
%
% scala> Pt(4.0, 2.0)
% res0: Pt = Pt(x=4.0,y=2.0)
%
% scala> Pt(6.0, 3.0)
% res1: Pt = Pt(x=6.0,y=3.0)
%
% scala> Pt(666.0, 1337.0)
% res2: Pt = Pt(x=666.0,y=1337.0)
% \end{REPL}
%
% \SubtaskSolved  \code{def apply(): Pt = new Pt(0, 0)}
%
% \SubtaskSolved  \code{class Rational(val nom: Int, val denom: Int)}
%
% \SubtaskSolved
% \begin{REPLnonum}
% object Rational {
% def apply(nom: Int, denom: Int): Rational = new Rational(nom, denom)
% }
% \end{REPLnonum}
%
% \SubtaskSolved
% \begin{REPL}
% scala> Rational(2, 5)
% scala> Rational(2, 7)
% scala> Rational(7, 4)
% scala> Rational(666, 1337)
% \end{REPL}
%
%
% \QUESTEND
%
%
%
%
% %%<AUTOEXTRACTED by mergesolu>%%      %Uppgift 9
%
%
%
%
% \WHAT{Skapa en case-klass.}
%
% \QUESTBEGIN
%
% \Task  \what~  Med en case-klass får man \code{toString} och fabriksobjekt på köpet. Man behöver inte skriva \code{val} framför klassparametrar i case-klasser; klassparametrar blir publika, oföränderliga attribut automatiskt när man deklarerar en case-klass.
%
% \begin{REPL}
% scala> case class Pt(x: Double, y: Double)
% scala> val p = Pt(1.0, 42.0)
% scala> p.toString
% scala> println(p)
% scala> println(Pt(5,6))
% \end{REPL}
%
% \Subtask Implementera din klass \code{Rational} från föregående uppgift, men nu som en case-klass.
%
% \SOLUTION
%
%
% \TaskSolved \what
%
% \SubtaskSolved  \code{case class Rational(nom: Int, denom: Int)}
%
%
% \QUESTEND
%
%
%
%
% %%<AUTOEXTRACTED by mergesolu>%%      %Uppgift 10
%
%
%
%
% \WHAT{Metoder på datastrukturer.}
%
% \QUESTBEGIN
%
% \Task \label{task:point} \what~   En datastruktur blir mer användbar om det finns metoder som kan användas på datastrukturen. Metoder i Scala kan även ha (vissa) specialtecken som namn, t.ex. \code{+} enligt nedan.
% \begin{REPL}
% scala> case class Point(x: Double, y: Double) {
%          def distToOrigin: Double = math.hypot(x, y)
%          def add(p: Point): Point = Point(x + p.x, y + p.y)
%          def +(p: Point): Point = add(p)
%        }
% \end{REPL}
%
% \Subtask Använd metoden \code{distToOrigin} för att ta reda på vad punkten med koordinaterna (3, 4) har för avstånd till origo?
%
% \Subtask Skriv satser som skapar två punkter (3,4) och (5, 6) och låt variablerna p1 och p2 referera till respektive punkt. Låt variabeln p3 bli summan av p1 och p2 med hjälp av metoden \code{add}. Vad får uttrycken \code{p3.x} resp. \code{p3.y} för värden?
%
%
%
% \SOLUTION
%
%
% \TaskSolved \what
%
%
% \SubtaskSolved
% \begin{REPLnonum}
% scala> Point(3, 4).distToOrigin
% res0: Double = 5.0
% \end{REPLnonum}
%
% \SubtaskSolved
% p3.x = 8
% p3.y = 10
%
%
% \QUESTEND
%
%
%
%
% %%<AUTOEXTRACTED by mergesolu>%%      %Uppgift 11
%
%
%
%
% \WHAT{Operatornotation.}
%
% \QUESTBEGIN
%
% \Task  \what~  Vid punktnotation på formen: \\ \code{objekt.metod(argument)} \\ kan man skippa punkten och parenteserna och skriva:\\ \code{objekt metod argument}  \\
% Detta förenklade skrivsätt kallas \textbf{operatornotation}.
%
% \Subtask Använd klassen \code{Point} från uppgift \ref{task:point} och prova nedan satser. Vilka rader använder operatortnotation och vilka rader använder punktnotation? Vilka rader ger felmeddelande?
% \begin{REPL}
% scala> val p1 = Point(3,4)
% scala> val p2 = Point(3,4)
% scala> p1.add(p2)
% scala> p1 add p2
% scala> p1.+(p2)
% scala> p1 + p2
% scala> 42 + 1
% scala> 42.+(1)
% scala> 42.+ 1
% scala> 42 +(1)
% scala> 1.to(42)
% scala> 1 to 42
% scala> 1.to(42)
% \end{REPL}
%
% \Subtask Implementera metoderna \code{sub} och \code{-} i klassen \code{Point} och skriv uttryck som kombinerar add och sub, samt + och - i både punktnotation och operatornotation.
%
% \Subtask Operatornotation fungerar även med flera argument. Man använder då parenteser om listan med argumenten:
% \code{ objekt metod (arg1, arg2)}  \\
% Definiera en metod \\
% \code{def scale(a: Double, b: Double) = Point(x * a, y * b)} \\
% i klassen \code{Point} och skriv satser som använder metoden med punktnotation och operatornotation.
%
%
%
%
%
% \SOLUTION
%
%
% \TaskSolved \what
%
%
% \SubtaskSolved
% \\Operatornotation:	4, 6, 10, 12
% \\Punktnotation:		3, 5, 8, 9, 11, 13
% \\Felmeddelande:		9
%
% \SubtaskSolved
% \begin{Code}
% case class Point(x: Double, y: Double) {
%   def distToOrigin: Double = math.hypot(x, y)
%   def add(p: Point): Point = Point(x + p.x, y + p.y)
%   def +(p: Point): Point = add(p)
%   def sub(p: Point): Point = Point(x - p.x, y - p.y)
%   def -(p: Point): Point = sub(p)
% }
% \end{Code}
% \begin{REPL}
% scala> val p1: Point = Point(1, 9)
% scala> val p2: Point = Point(9, 6)
% scala> p1.sub(p2)
% scala> p1.-(p2)
% scala> p2 sub p1
% scala> p2 - p2
% scala> p1.add(p2.sub(p1))
% scala> p1 + (p2 - p1)
% \end{REPL}
%
% \SubtaskSolved
% \begin{Code}
% case class Point(x: Double, y: Double) {
%   def distToOrigin: Double = math.hypot(x, y)
%   def add(p: Point): Point = Point(x + p.x, y + p.y)
%   def +(p: Point): Point = add(p)
%   def sub(p: Point): Point = Point(x - p.x, y - p.y)
%   def -(p: Point): Point = sub(p)
%   def scale(a: Double, b: Double) = Point(x * a, y * b)
% }
% \end{Code}
% \begin{REPL}
% scala> val p: Point(13,  37)
% scala> p.scale(4, 2)
% scala> p scale (3, 7)
% \end{REPL}
%
%
% \QUESTEND
%
%
%
%
% %%<AUTOEXTRACTED by mergesolu>%%      %Uppgift 12
%
%
%
%
% \WHAT{Föränderlighet och oföränderlighet.}
%
% \QUESTBEGIN
%
% \Task  \what~  Oföränderliga och föränderliga objekt beter sig olika vid tilldelning.
%
% \Subtask\Pen Innan du kör nedan kod: Försök lista ut vad som kommer att skrivas ut. Rita minnessituationen efter varje tilldelning.
%
% \begin{Code}
% println("\n--- Example 1: mutable value assigmnent")
% var x1 = 42
% var y1 = x1
% x1 = x1 + 42
% println(x1)
% println(y1)
% \end{Code}
%
% \Subtask\Pen Innan du kör nedan kod: Försök lista ut vad som kommer att skrivas ut. Rita minnessituationen efter varje tilldelning.
%
% \begin{Code}
% println("\n--- Example 2: mutable object reference assignment")
% class MutableInt(private var i: Int) {
%   def +(a: Int): MutableInt = { i = i + a; this }
%   override def toString: String = i.toString
% }
% var x2 = new MutableInt(42)
% var y2 = x2
% x2 = x2 + 42
% println(x2)
% println(y2)
% \end{Code}
%
% \Subtask\Pen Innan du kör nedan kod: Försök lista ut vad som kommer att skrivas ut. Rita minnessituationen efter varje tilldelning.
%
% \begin{Code}
% println("\n--- Example 3: immutable object reference assignment")
% class ImmutableInt(val i: Int) {
%   def +(a: Int): ImmutableInt = new ImmutableInt(i + a)
%   override def toString: String = i.toString
% }
% var x3 = new ImmutableInt(42)
% var y3 = x3
% x3 = x3 + 42
% println(x3)
% println(y3)
% \end{Code}
%
% \Subtask\Pen Vad finns det för fördelar med oföränderliga datastrukturer?
%
%
% \SOLUTION
%
%
% \TaskSolved \what
%
%
% \SubtaskSolved   \includegraphics[scale=0.5]{../img/w04-solutions/uppgift-13a}
%
% \SubtaskSolved
% \begin{enumerate}
% \item \includegraphics[scale=0.5]{../img/w04-solutions/uppgift-13b-1}
% \item \includegraphics[scale=0.5]{../img/w04-solutions/uppgift-13b-2}
% \item \includegraphics[scale=0.5]{../img/w04-solutions/uppgift-13b-3}
% \end{enumerate}
%
% \SubtaskSolved
% \begin{enumerate}
% \item \includegraphics[scale=0.5]{../img/w04-solutions/uppgift-13c-1}
% \item \includegraphics[scale=0.5]{../img/w04-solutions/uppgift-13c-2}
% \item \includegraphics[scale=0.5]{../img/w04-solutions/uppgift-13c-3}
% \end{enumerate}
%
% \SubtaskSolved   En stor fördel är att vi till exempel kan skicka med en immutable som argument till en metod och vara säkra på att metoden inte ändrar på värdet.
%
%
% \QUESTEND
%
%
%
%
% %%<AUTOEXTRACTED by mergesolu>%%      %Uppgift 13
%
%
%
%
% \WHAT{Några användbara samlingar.}
%
% \QUESTBEGIN
%
% \Task  \what~  En \textbf{samling} \Eng{collection} är en datastruktur som samlar många objekt av samma typ. I \code{scala.collection} och \code{java.util} finns många olika samlingar med en uppsjö användbara metoder. De olika samlingarna i \code{scala.collection} är ordnade i en gemensam hierarki med många gemensamma metoder; därför har man nytta av det man lär sig om metoderna i en Scala-samling när man använder en annan samling. Vi har redan tidigare sett samlingen \code{Vector}:
%
% \begin{REPL}
% scala> val tärningskast = Vector.fill(10000)((math.random() * 6 + 1).toInt)
% scala> tä   // tryck TAB
% scala> tärningskast.  // tryck TAB
% \end{REPL}
%
% \Subtask Ungefär hur många metoder finns det som man kan göra på objekt av typen \code{Vector}? Det är svårt att lära sig alla dessa på en gång, så vi väljer ut några få i kommande uppgifter.
%
% \Subtask Jämför överlappet mellan metoderna i \code{Vector} och \code{List} och uppskatta hur stor andel av metoderna som är gemensamma:
% \begin{REPL}
% scala> val myntkast =
%          List.fill(10000)(if (math.random() < 0.5) "krona" else "klave")
% scala> my   // tryck TAB
% scala> myntkast.  // tryck TAB
% \end{REPL}
%
% \SOLUTION
%
%
% \TaskSolved \what
%
%
% \SubtaskSolved   Ungefär 150 metoder.
%
% \SubtaskSolved   Ungefär lika många.
%
%
% \QUESTEND
%
%
%
%
% %%<AUTOEXTRACTED by mergesolu>%%      %Uppgift 14
%
%
%
%
% \WHAT{Typparameter.}
%
% \QUESTBEGIN
%
% \Task  \what~  Vissa funktioner är generella för många typer och tar en så kallad \textbf{typparameter} inom hakparenteser. Ofta slipper man skriva typparametrar, då kompilatorn kan härleda typen utifrån argumenten. Om man anger typparametrar explicit så hjälper kompilatorn dig med att kolla att det verkligen är rätt typ i samlingen.
%
% \Subtask Vad händer nedan?
% \begin{REPL}
% scala> var xs = Vector.empty[Int]
% scala> xs = xs :+ "42"
% scala> xs = xs :+ 43 :+ 64 :+ 46
% scala> xs
% scala> xs :+= "42".toInt
% scala> var ys = Vector[Int]("ett", "två", "tre")
% scala> var ingenting = Vector.empty
% scala> ingenting = Vector(1,2,3)
% \end{REPL}
%
% \Subtask Samlingar är mer användbara om de är \emph{generiska}, vilket innebär att elementens typ avgörs av en typparameter och därför kan vara av vilken typ som helst. Man kan definiera egna funktioner som tar generiska samlingar som parametrar. Förklara vad som händer här:
% \begin{REPL}
% scala> val vego = Vector("gurka", "tomat", "apelsin", "banan")
% scala> val prim = Vector(2, 3, 5, 7, 11, 13)
% scala> def först[T](xs: Vector[T]): T = xs.head
% scala> def sist[T](xs: Vector[T]) = xs.last
% scala> def förstOchSist[T](xs: Vector[T]): (T, T) = (xs.head, xs.last)
% scala> först(vego)
% scala> sist(prim)
% scala> förstOchSist(vego)
% scala> förstOchSist(prim)
% scala> def wrap[T](pair: (T, T))(xs: Vector[T]) = pair._1 +: xs :+ pair._2
% scala> wrap("Odla", "och ät!")(vego)
% scala> wrap("Odla", "och ät!")(vego).mkString(" ")
% \end{REPL}
%
%
%
%
%
% \SOLUTION
%
%
% \TaskSolved \what
%
%
% \SubtaskSolved
% \\1. Instansierar en tom vektor med element av typen int och tilldelar värdet till en variabel xs.
% \\2. Error eftersom \code{xs :+ ''42''} ger en Vector[Any] när Vector[Int] krävs.
% \\3. xs tilldelas ett nytt värde av Vector(43, 64, 46)
% \\4. xs skrivs ut.
% \\5. Lägger till talet 42 i xs.
% \\6. Error: type mismatch
% \\7. Skapar en tom Vector i variablen ingenting
% \\8. error: type mismatch; found: Int(3), required: Nothing
%
% \SubtaskSolved
% Tre metoder skapas: den första för att få första elementet i en lista, och eftersom den definieras med specialtypen T går den att använda med alla vektorer oavsett typen av variabeln i vektorn. Den andra får fram sista elementet och den sista hämtar båda två.
%
% En till function definieras längre ner med  namnet ''wrap'', som tar en lista och lägger till ett element längst fram och ett längst bak.
%
%
% \QUESTEND
%
%
%
%
% %%<AUTOEXTRACTED by mergesolu>%%      %Uppgift 15
%
%
%
%
% \WHAT{Några viktiga samlingsmetoder.}
%
% \QUESTBEGIN
%
% \Task  \what~  Deklarera följande vektorer i REPL.
% \begin{REPL}
% scala> val xs = (1 to 10).toVector
% scala> val a = Vector("abra", "ka", "dabra")
% scala> val b = Vector( "sim", "sala", "bim", "sala", "bim")
% scala> val stor = Vector.fill(100000)(math.random())
% \end{REPL}
% Undersök i REPL vad som händer nedan. Alla dessa metoder fungerar på alla samlingar som är indexerbara sekvenser. Givet deklarationerna ovan: vad har uttrycken nedan för värde och typ? Förklara vad som händer hälp av denna  översikt: \href{http://docs.scala-lang.org/overviews/collections/seqs}{docs.scala-lang.org/overviews/collections/seqs}
%
% \Subtask \code{a(1) + xs(1)}
%
% \Subtask \code{a apply 0}
%
% \Subtask \code{a.isDefinedAt(3)}
%
% \Subtask \code{a.isDefinedAt(100)}
%
% \Subtask \code{stor.length}
%
% \Subtask \code{stor.size}
%
% \Subtask \code{stor.min}
%
% \Subtask \code{stor.max}
%
% \Subtask \code{a indexOf "ka"}
%
% \Subtask \code{b.lastIndexOf("sala")}
%
% \Subtask \code{"först" +: b   //minnesregel: colon on the collection side}
%
% \Subtask \code{a :+ "sist"    //minnesregel: colon on the collection side}
%
% \Subtask \code{xs.updated(2,42)}
%
% \Subtask \code{a.padTo(10, "!")}
%
% \Subtask \code{b.sorted}
%
% \Subtask \code{b.reverse}
%
% \Subtask \code{a.startsWith(Vector("abra", "ka"))}
%
% \Subtask \code{"hejsan".endsWith("san")}
%
% \Subtask \code{b.distinct}
%
%
%
% \SOLUTION
%
%
% \TaskSolved \what
%
%
% \SubtaskSolved   String = ''ka2''
%
% \SubtaskSolved   String = ''abra''
%
% \SubtaskSolved   false
%
% \SubtaskSolved   false
%
% \SubtaskSolved   100000
%
% \SubtaskSolved   100000
%
% \SubtaskSolved   minsta talet i listan
%
% \SubtaskSolved   största talet i listan
%
% \SubtaskSolved   1
%
% \SubtaskSolved   3
%
% \SubtaskSolved   Vektor b fast med ''först'' som första element
%
% \SubtaskSolved   Vektor a fast med ''sist'' som sista element.
%
% \SubtaskSolved   plats 3 i vektorn xs får värdet 42
%
% \SubtaskSolved   En ny vektor fylld med ''!'' från och med plats 4 till 10. Men de andra värdena samma som i a.
%
% \SubtaskSolved   b sorterad i bokstavsordning
%
% \SubtaskSolved   b baklänges
%
% \SubtaskSolved   true
%
% \SubtaskSolved   true
%
% \SubtaskSolved   en vektor med alla unika element i b.
%
%
% \QUESTEND
%
%
%
%
% %%<AUTOEXTRACTED by mergesolu>%%      %Uppgift 16
%
%
%
%
% \WHAT{Några generella samlingsmetoder.}
%
% \QUESTBEGIN
%
% \Task  \what~  Det finns metoder som går att köra på \emph{alla} samlingar även om de inte är indexerbara. Givet deklarationerna i föregående uppgift: vad har uttrycken nedan för värde och typ? Förklara vad som händer med hjälp av dessa översikter: \\ \href{http://docs.scala-lang.org/overviews/collections/trait-traversable}{docs.scala-lang.org/overviews/collections/trait-traversable} \\ \href{http://docs.scala-lang.org/overviews/collections/trait-iterable}{docs.scala-lang.org/overviews/collections/trait-iterable}
%
% \Subtask \code{a ++ b}
%
% \Subtask \code{a ++ stor}
%
% \Subtask \code{val ys = xs.map(_ * 5)}
%
% \Subtask \code{b.toSet     // En mängd har inga dubletter}
%
% \Subtask \code{a.head + b.last}
%
% \Subtask \code{a.tail}
%
% \Subtask \code{a.head +: a.tail == a}
%
% \Subtask \code{Vector(a.head) ++ Vector(b.last)}
%
% \Subtask \code{a.take(1) ++ b.takeRight(1)}
%
% \Subtask \code{a.drop(2) ++ b.drop(1).dropRight(2)}
%
% \Subtask \code{a.drop(100)}
%
% \Subtask \code{val e = Vector.empty[String]; e.take(100)}
%
% \Subtask \code{Vector(e.isEmpty, e.nonEmpty)}
%
% \Subtask \code{a.contains("ka")}
%
% \Subtask \code{"ka" contains "a"}
%
% \Subtask \code{a.filter(s => s.contains("k")) }
%
% \Subtask \code{a.filter(_.contains("k")) }
%
% \Subtask \code{a.map(_.toUpperCase).filterNot(_.contains("K")) }
%
% \Subtask \code{xs.filter(x => x % 2 == 0)}
%
% \Subtask \code{xs.filter(_ % 2 == 0)}
%
%
% \SOLUTION
%
%
% \TaskSolved \what
%
%
% \SubtaskSolved
% Metoden ger tillbaka en ny Vector[String] som nu består av alla element i a plus alla element i b. I samma ordning med elementen i a först.
%
% \SubtaskSolved
% Samma som i uppgift a fast vektorn som returnas är av typen Vector[Any]. Det är eftersom Any är den närmsta typen som String och Double delar. Elementen från vektor a är fortfarande först och uppföljt av elementen i stor.
%
% \SubtaskSolved
% Variablen ys får värdet av en Vector[Int] som innehåller alla talen från xs fast multiplicerade med 5. Alltså ys = 5, 10, 15..., osv.
%
% \SubtaskSolved
% Functionen tar alla värden från en Vektor och sätter in i ett Set (mängd). Eftersom en mängd ej har dubletter så försvinner ett ''sala'' och ett ''bim'', Vector[String] som returneras blir därför (''sim'', ''sala'', ''bim'').
%
% \SubtaskSolved
% Metoden head ger första elementet i en samling, och last sista. Därför blir kombinationen av a.head och b.last en ny Vector[String] som består av a:s första element, och b:s första element.
%
% \SubtaskSolved
% Ger en Vector[String] som innehåller alla element efter det första. Alltså i detta fallet ''ka'' och ''dabra''.
%
% \SubtaskSolved
% True, eftersom head ger första elementet och tail ger resten, sedan sätter metoden +: ihop dem till en vektor med samma värden som a.
%
% \SubtaskSolved
% Eftersom ++ sätter ihop alla värden från två vektorer måste vi först omvandla från en sträng till vektor. Resultatet blir en ny vektor av samma typ som innan med a:s första element och b:S sista.
%
% \SubtaskSolved
% Samma resultat som i h, metoden take börjar från vänster och tar så många element som man skickar med som parameter och gör till en ny lista. Med 1 som parameter motsvarar det att göra Vector(a.head). Metoden takeRight gör samma sak fast från höger.
%
% \SubtaskSolved
% Metoden drop är motsvarigheten till take fast exkluderar de specifierade elementen istället för att inkludera dem i vektorn.
%
% \SubtaskSolved
% Eftersom a endast innehåller 3 element returnerar drop(100) en tom vektor.
%
% \SubtaskSolved
% Returnerar en tom vektor med element typen String
%
% \SubtaskSolved
% returnerar Vector(true, false)
%
% \SubtaskSolved
% True, metoden contains kollar om en samling innehåller ett specifikt element.
%
% \SubtaskSolved
% True. Eftersom en sträng även kan ses som Vector[Char].
%
% \SubtaskSolved
% Filtrerar vektorn a till att endast innehålla strängar som innehåller k.
%
% \SubtaskSolved
% Exakt samma som i p
%
% \SubtaskSolved
% map(\_.toUpperCase) omvandlar alla strängar i a till stora bokstäver
% filterNot(\_.contains(''K'')) tar resultatet vi precis fick och tar bort alla strängar som innehåller stora K.
%
% \SubtaskSolved
% filtrerar så att endast jämna tal finns kvar.
%
% \SubtaskSolved
% Exakt samma som i s
%
%
%
%
% \QUESTEND
%
%
%
%
% %%<AUTOEXTRACTED by mergesolu>%%      %Uppgift 17
%
%
%
%
% \WHAT{NEEDS A TOPIC DESCRIPTION}
%
% \QUESTBEGIN
%
% \Task  \what~ De olika samlingarna i \code{scala.collection} används flitigt i andra paket, exempelvis \code{scala.util} och \code{scala.io}.
%
% \Subtask Vad händer här? (Metoden \code{shuffle} skapar en ny samling med elementen i slumpvis ordning.)
% \begin{REPL}
% val xs = Vector(1,2,3)
% def blandat = scala.util.Random.shuffle(xs)
% def test = if (xs == blandat) "lika" else "olika"
% (for(i <- 1 to 100) yield test).count(_ == "lika")
% \end{REPL}
%
%
% \Subtask Skapa en textfil med namnet \code{fil.txt} som innehåller lite text och läs in den med: \\\code{scala.io.Source.fromFile("fil.txt", "UTF-8").getLines.toVector}
% \begin{REPL}
% > cat > fil.txt
% hejsan
% svejsan
% > scala
% scala> val xs = scala.io.Source.fromFile("fil.txt", "UTF-8").getLines.toVector
% scala> xs.foreach(println)
% \end{REPL}
%
%
% \Subtask Vad händer här? (Metoden \code{trim} på värden av typen \code{String} ger en ny sträng med blanktecken i början och slutet borttagna.)
% \begin{REPL}
% scala> val pgk =
%   scala.io.Source.fromURL("https://lunduniversity.github.io/pgk","UTF-8").getLines.toVector
% scala> pgk.foreach(println)
% scala> pgk.map(_.trim).
%          filterNot(_.startsWith("<")).
%          filterNot(_.isEmpty).
%          foreach(println)
% \end{REPL}
%
%
%
% \SOLUTION
%
%
% \TaskSolved \what
%
%
% \SubtaskSolved
% Vi instansierar en vektor xs med talen 1, 2 och 3.
% sedan definierar vi en metod blandat som ger oss en randomiserad version av xs.
% sedan definierar vi en till metod som testar om xs är lika med resultatet från blandat. Om det är så returnerar den strängen ''lika'' annars ''olika''.
% Sist kör vi en for-loop där vi 100 gånger kör testet, samtidigt räknas hur många gånger ''lika'' returneras.
%
% Vårt resultat är en siffra på hur många gånger xs var samma som en blandad version av sig själv, eftersom det finns 6 permutationer med 3 variabler så borde det vara ungefär 1/6 chans.
%
% \SubtaskSolved  -
%
% \SubtaskSolved
% \\ \code{map(\_.trim)} tar bort alla onödiga mellanrum i början och slutet på varje rad
% \\ \code{filterNot(\_.startsWith(''<''))} filtrerar bort alla rader som börjar med strängen ''<''
% \\ \code{filterNot(\_.isEmpty)} filtrerar bort alla tomma rader.
% \\ \code{foreach(println)} skriver ut alla rader.
%
%
% \QUESTEND
%
%
%
%
% %%<AUTOEXTRACTED by mergesolu>%%      %Uppgift 18
%
%
%
%
% \WHAT{Jämföra List och Vector.}
%
% \QUESTBEGIN
%
% \Task  \what~  En indexerbar sekvens av värden kallas vektor eller lista. I Scala finns flera klasser som kan kan indexeras, däribland klasserna \code{Vector} och \code{List}.
%
% \Subtask \emph{Likheter mellan \code{Vector} och \code{List}.} Kör nedan rader i REPL. Prova indexera i båda och studera hur stor andel av metoderna som är gemensamma.
% \begin{REPL}
% scala> val sv = Vector("en", "två", "tre", "fyra")
% scala> val en = List("one", "two", "three", "four")
% scala> sv(0) + sv(3)
% scala> en(0) + en(3)
% scala> sv. //tryck TAB
% scala> en. //tryck TAB
% \end{REPL}
%
% \Subtask \emph{Skillnader mellan \code{Vector} och \code{List}.} Klassen \code{Vector} i Scala har ''under huven'' en avancerad datastruktur i form av ett s.k. självbalanserande träd, vilket gör att \code{Vector} är snabbare än \code{List} på nästan allt, \emph{utom} att bearbeta elementen i \emph{början} av sekvensen; vill man lägga till och ta bort i början av en \code{List} så kan det ibland gå ungefär dubbelt så fort jämfört med \code{Vector}, medan alla andra operationer är lika snabba eller snabbare med \code{Vector}. Det finns ett fåtal speciella metoder, som bara finns i \code{List}, för att skapa en lista och lägga till i början av en lista. Vad händer nedan?
%
% \begin{REPL}
% scala> var xs = "one" :: "two" :: "three" :: "four" :: Nil
% scala> xs = "zero" :: xs
% scala> val ys = xs.reverse ::: xs
% \end{REPL}
%
%
% \SOLUTION
%
%
% \TaskSolved \what
%
%
% \SubtaskSolved
% I princip alla metoder delas, en lista har några fler t. ex. ''::'', '':::'', ''mapConserve'' osv.
%
% \SubtaskSolved
% Först skapas en lista med 4 sträng värden och instansierar variablen xs med det värdet.
% sedan skapar vi en ny lista, som består av ''zero'' + den gamla listan och ger värdet till xs.
% Sist instansierar vi en ny variabel ys, som får värdet av xs omvänd plus xs.
%
%
% \QUESTEND
%
%
%
%
% %%<AUTOEXTRACTED by mergesolu>%%      %Uppgift 19
%
%
%
%
% \WHAT{Mängd.}
%
% \QUESTBEGIN
%
% \Task  \what~  En mängd är en samling som garanterar att det inte finns några dubbletter. Det går dessutom väldigt snabbt, även i stora mängder, att kolla om ett element finns eller inte i mängden. Elementen i samlingen \code{Set} hamnar ibland, av effektivitetsskäl, i en förvånande ordning.
% \begin{REPL}
% scala> val s = Set("Malmö", "Stockholm", "Göteborg", "Köpenhamn", "Oslo")
% s: scala.collection.immutable.Set[String] =
%      Set(Oslo, Malmö, Köpenhamn, Stockholm, Göteborg)
%
% scala> val t = Set("Sverige", "Sverige", "Sverige", "Danmark", "Norge")
% t: scala.collection.immutable.Set[String] = Set(Sverige, Danmark, Norge)
% \end{REPL}
% Givet ovan deklarationer: vad blir värde och typ av nedan uttryck?
%
% \Subtask \code{s + "Malmö" == s}
%
% \Subtask \code{s ++ t}
%
% \Subtask \code{Set("Malmö", "Oslo").subsetOf(s)}
%
% \Subtask \code{s subsetOf Set("Malmö", "Oslo")}
%
% \Subtask \code{s contains "Lund"}
%
% \Subtask \code{s apply "Lund"}
%
% \Subtask \code{s("Malmö")}
%
% \Subtask \code{s - "Stockholm"}
%
% \Subtask \code{t - ("Norge", "Danmark", "Tyskland")}
%
% \Subtask \code{s -- t}
%
% \Subtask \code{s -- Set("Malmö", "Oslo")}
%
% \Subtask \code{Set(1,2,3) intersect Set(2,3,4)}
%
% \Subtask \code{Set(1,2,3) & Set(2,3,4)}
%
% \Subtask \code{Set(1,2,3) union Set(2,3,4)}
%
% \Subtask \code{Set(1,2,3) | Set(2,3,4)}
%
%
% \SOLUTION
%
%
% \TaskSolved \what
%
%
% \SubtaskSolved
% true, Boolean
%
% \SubtaskSolved
% En samling av alla värden i s och t, Set[String]
%
% \SubtaskSolved
% true, Boolean
%
% \SubtaskSolved
% false, Boolean
%
% \SubtaskSolved
% false, Boolean
%
% \SubtaskSolved
% false, Boolean
%
% \SubtaskSolved
% true, Boolean
%
% \SubtaskSolved
% Samlingen s utan elementet ''Stockholm'', Set[String]
%
% \SubtaskSolved
% Samlingen t utan elementen ''Norge'' och ''Danmark'', Set[String]
%
% \SubtaskSolved
% returnerar s, Set[String]
%
% \SubtaskSolved
% Samlingen s utan ''Malmö'' och ''Oslo'', Set[String]
%
% \SubtaskSolved
% Set(2, 3), Set[Int]
%
% \SubtaskSolved
% se deluppgift l
%
% \SubtaskSolved
% Set(1, 2, 3 ,4), Set[Int]
%
% \SubtaskSolved
% se deluppgift n
%
%
% \QUESTEND
%
%
%
%
% %%<AUTOEXTRACTED by mergesolu>%%      %Uppgift 20
%
%
%
%
% \WHAT{Slå upp värden från nycklar med \code{Map}.}
%
% \QUESTBEGIN
%
% \Task  \what~  Samlingen \code{Map} är mycket användbar. Med den kan man snabbt leta upp ett värde om man har en nyckel. Samlingen \code{Map} är en generalisering av en vektor, där man kan ''indexera'', inte bara med ett heltal, utan med vilken typ av värde som helst, t.ex. en sträng. Datastrukturen \code{Map} är en s.k. \emph{associativ array}\footnote{\href{https://en.wikipedia.org/wiki/Associative_array}{https://en.wikipedia.org/wiki/Associative\_array}}, implementerad som en s.k. \emph{hashtabell}\footnote{\href{https://en.wikipedia.org/wiki/Hash_table}{https://en.wikipedia.org/wiki/Hash\_table}}.
% \begin{REPL}
% scala> var huvudstad =
%   Map("Sverige" -> "Stockholm", "Norge" -> "Oslo", "Skåne" -> "Malmö")
% \end{REPL}
% Givet ovan variabel \code{huvudstad}, förklara vad som händer nedan?
%
% \Subtask \code{huvudstad apply "Skåne"}
%
% \Subtask \code{huvudstad("Sverige")}
%
% \Subtask \code{huvudstad.contains("Skåne")}
%
% \Subtask \code{huvudstad.contains("Malmö")}
%
% \Subtask \code{huvudstad += "Danmark" -> "Köpenhamn"}
%
% \Subtask \code{huvudstad.foreach(println)}
%
% \Subtask \code{huvudstad getOrElse ("Norge", "???") }
%
% \Subtask \code{huvudstad getOrElse ("Finland", "???") }
%
% \Subtask \code{huvudstad.keys.toVector.sorted}
%
% \Subtask \code{huvudstad.values.toVector.sorted}
%
% \Subtask \code{huvudstad - "Skåne"}
%
% \Subtask \code{huvudstad - "Jylland"}
%
% \Subtask \code{huvudstad = huvudstad.updated("Skåne","Lund") }
%
%
%
% \SOLUTION
%
%
% \TaskSolved \what
%
%
% \SubtaskSolved
% Returnerar strängen ''Malmö'' eftersom det värdet är indexerat på platsen ''Skåne''.
%
% \SubtaskSolved
% Returnerar strängen ''Stockholm'' eftersom det värdet är indexerat på platsen ''Sverige''.
%
% \SubtaskSolved
% true, eftersom huvudstad innehåller indexet ''Skåne''
%
% \SubtaskSolved
% false, eftersom huvudstad ej innehåller indexet ''Malmö''. Notera att det är index och inte värden vi
% kollar om det finns.
%
% \SubtaskSolved
% Lägger till indexet ''Danmark'' med värdet ''Köpenhamn'' i samlingen.
%
% \SubtaskSolved
% Skriver ut alla 2-tupler.
%
% \SubtaskSolved
% Returnerar ''Oslo'', Note: Om indexet ''Norge'' inte hade funnits hade ''???'' returnerats istället.
%
% \SubtaskSolved
% Returnerar ''???''
%
% \SubtaskSolved
% Returnerar en sorterar vektor med alla index.
%
% \SubtaskSolved
% Returnerar en sorterar vektor med alla värden.
%
% \SubtaskSolved
% Returnerar en ny mängd men med ''Skåne'' -> ''Malmö'' borttaget.
%
% \SubtaskSolved
% Returnerar huvudstad mängden eftersom det inte finns ett ''Jylland'' index att ta bort.
%
% \SubtaskSolved
% Uppdaterar indexet ''Skåne'' till att istället leda till värdet ''Lund''
%
%
% \QUESTEND
%
%
%
%
% %%<AUTOEXTRACTED by mergesolu>%%      %Uppgift 21
%
%
%
%
% \WHAT{Skapa Map från en samling.}
%
% \QUESTBEGIN
%
% \Task  \what~
%
% \Subtask Definiera denna vektor och undersök dess typ:
% \begin{Code}
% val pairs = Vector(
%   ("Björn", 46462229009L),
%   ("Maj", 46462221667L),
%   ("Gustav", 46462224906L))
% \end{Code}
%
% \Subtask Vad har variablen \code{telnr} nedan för typ: \\ \code{var telnr = pairs.toMap}
%
% \Subtask Använd \code{telnr} för att slå upp telefonnummer för Maj och Kim med hjälp av metoderna \code{apply} och \code{get}.
%
% \Subtask Använd metoden \code{getOrElse} vid upplagningar av \code{telnr} och ge \code{-1L} som telefonnummer i händelse av att ett nummer inte finns.
%
% \Subtask Lägg till \code{("Fröken Ur", 464690510L)} i \code{telnr}-mappen.
%
% \Subtask Skapa en \code{Vector[(String, String)]} enligt nedan, så att telefonnumret blir en sträng utan inledande landsnummer men med en nolla i riktnumret. Byt ut \code{???} mot lämpligt uttryck.
% \begin{REPL}
% scala> telnr.toVector.map(p => ???)
% res85: Vector[(String, String)] = Vector(("Björn", "0462229009"), ("Maj",
% "0462221667"), ("Gustav", "0462224906"), ("Fröken Ur", 04690510"))
%
% \end{REPL}
%
% \Subtask Använd vektorn i resultatet ovan för att skapa en ny \code{Map[String, String]} med nationella telefonnumer. Slå upp numret till Fröken Ur.
%
% \SOLUTION
%
%
% \TaskSolved \what
%
%
% \SubtaskSolved
% \begin{REPLnonum}
% pairs: scala.collection.immutable.Vector[(String, Long)] =
% 					Vector((Björn,444), (Maj,441), (Lucy,666))
% \end{REPLnonum}
%
% \SubtaskSolved
% Map[String, Long]
%
% \SubtaskSolved
% \begin{REPLnonum}
% scala> telnr(''Maj'')
% res0: Long = 441
%
% scala> telnr.get(''Maj'')
% res1: Option[Long] = Some(441)
%
% scala> telnr(''Kim'')
% java.util.NoSuchElementException: key not found: 'Kim
%   at scala.collection.MapLike$class.default(MapLike.scala:228)
%   at scala.collection.AbstractMap.default(Map.scala:59)
%   at scala.collection.MapLike$class.apply(MapLike.scala:141)
%   at scala.collection.AbstractMap.apply(Map.scala:59)
%   ... 32 elided
%
% scala> telnr.get(''Kim'')
% res2: Option[Long] = None
% \end{REPLnonum}
%
% \SubtaskSolved
% \begin{REPLnonum}
% scala> telnr.getOrElse(''Maj'', -1L)
% res0: Long = 441
%
% scala> telnr.getOrElse(''Kim'', -1L)
% res1: Long = -1
% \end{REPLnonum}
%
% \SubtaskSolved
% telnr += ''Fröken Ur'' -> 464690510L
%
% \SubtaskSolved
% telnr.toVector.map(p => p.\_1 -> (''0'' + p.\_2.toString.substring(2)))
%
% \SubtaskSolved
% Använd metoden toMap och apply.
%
%
%
%
% \QUESTEND
%
%
%
%
% %%<AUTOEXTRACTED by mergesolu>%%      %Uppgift 22
%
%
%
%
% \WHAT{Samlingsmetoden \code{maxBy}.}
%
% \QUESTBEGIN
%
% \Task  \what~  Med samlingsmetoden \code{maxBy} kan man själv definiera vad som ska maximeras. (Denna metod kommer du att behöva i veckans laboration.)
%
% \Subtask Förklara vad som händer nedan.
% \begin{REPL}
% scala> val xs = Vector((2,3), (1,5), (-1, 1), (7, 2))
% scala> xs.maxBy(x => x._1)
% scala> xs.maxBy(x => x._2)
% \end{REPL}
%
% \Subtask Om man bara använder en parameter i en anonym funktion, till exempel parametern \code{x} i lambdauttrycket \code{x => x + 1} \emph{en enda} gång, och kompilatorn kan gissa alla typer, kan man använda understreck som ''platshållare'' för att förkorta lambdauttrycket så här: \code{ _ + 1}
%
% Skriv uttrycken på raderna 2 och 3 i föregående deluppgift på ett kortare sätt med hjälp platshållarsyntax \Eng{place holder syntax}.
%
% \Subtask På motsvarande sätt kan man använda \code{minBy} för att välja vilken funktion som definierar minimum. Prova \code{minBy} på motsvarande sätt som i föregående deluppgifter.
%
% \SOLUTION
%
%
% \TaskSolved \what
%
%
% \SubtaskSolved   Metoden maxBy hämtar det element som är ''störst'', på rad två gör \code{x => x._1} att första värdet i tuplerna används för att bestämma vilken som är störst. Likt gör \code{x => x._2} på rad tre att istället det andra värdet används.
%
% \SubtaskSolved
% \begin{REPLnonum}
% scala> xs.maxBy(_._1)
% scala> xs.maxBy(_._2)
% \end{REPLnonum}
%
% \SubtaskSolved
% \begin{REPLnonum}
% scala> xs.minBy(_._1)
% scala> xs.minBy(_._2)
% \end{REPLnonum}
%
%
%
% \QUESTEND
%
%
%
%
%
%
%
%
% \WHAT{NEEDS A TOPIC DESCRIPTION}
%
% \QUESTBEGIN
%
% \Task  \what~ Skriv nedan program med en editor och kompilera från terminalen. Lägg till kod i huvudprogrammet som testar klassen \code{Account} och kompilera och kör. Utvidga sedan klassen \code{Account} med fler attribut och funktioner som du väljer själv.
%
% \begin{Code}
% class Account(val number: Long, val maxCredit: Int){
%   private var balance = 0
%
%   def deposit(amount: Int): Int = {
%     if (amount > 0) {balance += amount}
%     balance
%   }
%
%   def withdraw(amount: Int): (Int, Int) = if (amount > 0) {
%     val allowedWithdrawal =
%       if (amount < balance + maxCredit) amount
%       else balance + maxCredit
%     balance = balance - allowedWithdrawal
%     (allowedWithdrawal, balance)
%   } else (0, balance)
%
%   def show: Unit =
%     println("Account Nbr: " + number + " balance: " + balance)
% }
%
% object Main {
%   def main(args: Array[String]): Unit = {
%     ???
%   }
% }
% \end{Code}
%
%
%
% \SOLUTION
%
%
% \QUESTEND
%
%
%
%
%
%
% \WHAT{NEEDS A TOPIC DESCRIPTION}
%
% \QUESTBEGIN
%
% \Task \label{task:keno-set} \what~  Läs om reglerna för spelet Keno här: \\ \url{https://sv.wikipedia.org/wiki/Keno} och gör deluppgifterna nedan.
%
% \Subtask Skapa en klass \code{Keno} som kan användas för att genomföra en Kenodragning. Låt klassen ha ett privat attribut \code{balls} som är en föränderlig mängd med heltal och som från början är tom. Implementera lämpliga metoder i klassen för att användaren av klassen ska kunna dra nya slumpmässiga bollar som inte redan är dragna.
%
% \Subtask Skapa en \code{case class KenoBet(bet: Set[Int])} för att hålla reda vilka 11 bollar en viss person satsar på. Definiera en metod \\ \code{def numberOfHits(keno: Keno): Int = ???}\\ i case-klassen \code{KenoBet} som givet en kenodragning räknar ut hur många bollar som satsats rätt.
%
% \Subtask Skriv ett huvudprogram som simulerar en enkel Kenodragning. Låt två personer satsa på 11 slumpmässiga bollar, genomför en dragning av 20 bollar ur 70 möjliga och kontrollera sedan hur många bollar som personerna har prickat rätt.
%
%
%
%
%
% \SOLUTION
%
%
% \QUESTEND
%
%
%
%
%
%
% \WHAT{Dokumentationen för \code{Any}.}
%
% \QUESTBEGIN
%
% \Task  \what~  Undersök vilka metoder som finns i klassen Any här: \href{http://www.scala-lang.org/api/current/scala/Any.html}{http://www.scala-lang.org/api/current/scala/Any.html}. Prova några av metoderna i REPL.
%
% \SOLUTION
%
%
% \QUESTEND
%
%
%
%
%
%
% \WHAT{Dokumentationen för samlingar.}
%
% \QUESTBEGIN
%
% \Task  \what~  Leta upp metoden \code{tabulate} i dokumentationen för objektet \code{Vector} nästan längst ner i listan här: \\ \href{http://www.scala-lang.org/api/current/scala/collection/immutable/Vector.html}{http://www.scala-lang.org/api/current/scala/collection/immutable/Vector.html} \\Leta upp den variant av \code{tabulate} som har signaturen:\\ \code{def tabulate[A](n: Int)(f: (Int) => A): Vector[A] }\\ Klicka på den gråfyllda trekanten till vänster om signaturen som fäller ut beskrivningen
%
% \Subtask Förklara vad som händer här:
% \begin{REPLnonum}
% scala> Vector.tabulate(10)(i => i % 3)
% \end{REPLnonum}
%
% \Subtask Klicka på det blåa stora o-et överst på sidan, för att växla till klass-vyn och studera listan med alla metoder  i klassen \code{Vector}.
%
%
% \SOLUTION
%
%
% \QUESTEND
%
%
%
%
%
%
% \WHAT{Fler metoder på indexerbara sekvenser.}
%
% \QUESTBEGIN
%
% \Task  \what~  Deklarera följande vektorer i REPL.
% \begin{REPL}
% scala> val xs = (1 to 10).toVector
% scala> val a = Vector("abra", "ka", "dabra")
% scala> val b = Vector( "sim", "sala", "bim", "sala", "bim")
% \end{REPL}
% Undersök i REPL vad som händer nedan. Alla dessa metoder fungerar på alla samlingar som är indexerbara sekvenser. Vad har uttrycken för värde och typ? Förklara vad metoden gör. Studera även denna  översikt: \href{http://docs.scala-lang.org/overviews/collections/seqs}{docs.scala-lang.org/overviews/collections/seqs}
%
% \Subtask \code{b.indexWhere(s => s.startsWith("b"))}  % advanced
%
% \Subtask \code{a.indices}  % advanced
%
% \Subtask \code{xs.patch(1, Vector(42,43,44), 7)} % advanced
%
% \Subtask \code{xs.segmentLength(_ < 8, 2)} % advanced
%
% \Subtask \code{b.sortBy(_.reverse)}  % advanced
%
% \Subtask \code{b.sortWith((s1, s2) => s1.size < s2.size)} % advanced
%
% \Subtask \code{a.reverseMap(_.size)}	% advanced
%
% \Subtask \code{a intersect Vector("ka", "boom", "pow")} % advanced
%
% \Subtask \code{a diff Vector("ka")} % advanced
%
% \Subtask \code{a union Vector("ka", "boom", "pow")} % advanced
%
%
%
% \SOLUTION
%
%
% \QUESTEND
%
%
%
%
% \WHAT{NEEDS A TOPIC DESCRIPTION}
%
% \QUESTBEGIN
%
% \Task  \what~ För samlingen \code{List} finns en alternativ metod till \code{+:} som heter \code{::} och kallas ''cons'' och som i kombination med objektet \code{Nil} kan användas för att med alternativ syntax bygga listor. Läs om detta här: \\ \href{http://alvinalexander.com/scala/how-create-scala-list-range-fill-tabulate-constructors}{alvinalexander.com/scala/how-create-scala-list-range-fill-tabulate-constructors} \\ och hitta på några egna övningar för att undersöka hur cons och Nil fungerar. Metoder som slutar med kolon är högerassociativa. Läs mer om detta här: \href{http://www.artima.com/pins1ed/basic-types-and-operations.html#5.8}{http://www.artima.com/pins1ed/basic-types-and-operations.html\#5.8}\SOLUTION
%
%
% \QUESTEND


%!TEX encoding = UTF-8 Unicode
%!TEX root = ../exercises.tex

\ifPreSolution

\Exercise{\ExeWeekTEN}\label{exe:W10}

\begin{Goals}
\input{modules/w10-inheritance-exercise-goals.tex}
\end{Goals}

\begin{Preparations}
\item \StudyTheory{10}
\end{Preparations}

\BasicTasks

\else

\ExerciseSolution{\ExeWeekTEN}

\BasicTasks

\fi



\WHAT{Para ihop begrepp med beskrivning.}

\QUESTBEGIN

\Task \what

\vspace{1em}\noindent Koppla varje begrepp med den (förenklade) beskrivning som passar bäst:

\begin{ConceptConnections}
\input{generated/quiz-w10-concepts-taskrows-generated.tex}
\end{ConceptConnections}

\SOLUTION

\TaskSolved \what

\begin{ConceptConnections}
\input{generated/quiz-w10-concepts-solurows-generated.tex}
\end{ConceptConnections}

\QUESTEND





\WHAT{Gemensam bastyp.}

\QUESTBEGIN

\Task  \what~  Man vill ofta lägga in objekt av olika typ i samma samling.
\begin{REPL}
scala> class Gurka(val vikt: Int)
scala> class Tomat(val vikt: Int)
scala> val gurkor = Vector(Gurka(100), Gurka(200))
scala> val grönsaker = Vector(Gurka(300), Tomat(42))
\end{REPL}

\Subtask Om en samling innehåller objekt av flera olika typer försöker kompilatorn härleda den mest specifika typen som objekten har gemensamt. Vad blir det för typ på värdet \code{grönsaker} ovan?

\Subtask Försök ta reda på summan av vikterna enligt nedan. Vad ger andra raden för felmeddelande? Varför?

\begin{REPL}
scala> gurkor.map(_.vikt).sum     // fungerar
scala> grönsaker.map(_.vikt).sum  // fungerar inte
\end{REPL}

\Subtask Du ska nu göra så att du kan komma åt vikten på alla grönsaker genom att ge gurkor och tomater en gemensam bastyp som de olika konkreta grönsakstyperna utvidgar med nyckelordet \code{extends}. Det heter att subtyperna \code{Gurka} och \code{Tomat} \textbf{ärver} egenskaperna hos supertypen \code{Grönsak}.

Skapa en bastyp \code{Grönsak} med ett abstrakt attribut \code{vikt}. Låt sedan de konkreta grönsakerna ärva bastypen:

\begin{REPL}
scala> trait Grönsak { val vikt: Int }
scala> class Gurka(val vikt: Int) extends Grönsak
scala> class Tomat(val vikt: Int) extends Grönsak
scala> val gurkor = Vector(Gurka(100), Gurka(200))
scala> val grönsaker = Vector(Gurka(300), Tomat(42))
\end{REPL}
När sker initialisering av attributet \code{vikt}?

\Subtask Vad blir det nu för typ på variabeln \code{grönsaker} ovan?

\Subtask Går det nu att summera vikterna i \code{grönsaker} med uttrycket nedan? Varför?\\ \code{grönsaker.map(_.vikt).sum}


\Subtask En trait liknar en klass, men man kan inte instansiera den direkt. Vad blir det för felmeddelande om du försöker skapa en instans av en trait enligt nedan?
\begin{REPL}
scala> trait Grönsak { val vikt: Int }
scala> new Grönsak
\end{REPL}


\Subtask Traiten \code{Grönsak} har en abstrakt medlem \code{vikt}. Den sägs vara abstrakt eftersom den saknar implementation -- medlemmen har bara ett namn och en typ men inget värde. Du kan instansiera den abstrakta traiten \code{Grönsak} om du fyller i det som ''fattas'', nämligen ett värde på \code{vikt}. Man kan fylla på det som fattas i genom att ''hänga på'' ett block efter typens namn vid instansiering. Man får då vad som kallas en \textbf{anonym klass}, i detta fall en ganska konstig grönsak som inte är någon speciell sorts grönsak med som ändå har en vikt.

Vad får \code{anonymGrönsak} nedan för typ och strängrepresenation?
\begin{REPL}
scala> val anonymGrönsak = new Grönsak { val vikt = 42 }
\end{REPL}

\Subtask Vad blir felmeddelandet om du skapar en anonym klass \code{Grönsak} med en kropp som saknar definition av vikt?

\SOLUTION


\TaskSolved \what


\SubtaskSolved  \code{Vector[Object]}. Typen \code{Object} i JVM är motsvarar typen \code{AnyRef} som är bastyp för alla referenstyper.

\SubtaskSolved  Felmeddelande:
\begin{REPLnonum}
scala> grönsaker.map(_.vikt).sum  
-- Error:                                                                                 
1 |grönsaker.map(_.vikt).sum
  |              ^^^^^^
  |             value vikt is not a member of Object - did you mean wait?
-- Error:
1 |grönsaker.map(_.vikt).sum
  |                         ^
  |ambiguous implicit arguments: both object DoubleIsFractional in object Numeric and object ShortIsIntegral in object Numeric match type Numeric[B] of parameter num of method sum in trait IterableOnceOps
\end{REPLnonum}
Det första felmeddelandet beror på att vektorns element är av typen \code{Object} och medlemmen \code{vikt} är inte definierat för denna typ. Det andra felmeddelandet är ett följdfel som beror på att en sekvens med element av typen \code{Object} inte kan summeras eftersom kompilatorn inte kan härleda att elementtypen är numerisk.

\SubtaskSolved  Attributet \code{vikt} initialiseras vid konstruktion av \code{Gurka} resp. \code{Tomat}. Värdet ges av resp. klassparameter.

\SubtaskSolved  \code{Vector[Grönsak]}.

\SubtaskSolved  Ja. Eftersom den statiska typen för elementen i sekvensen är \code{Grönsak} (den dynamiska typen kan vara godtycklig subtyp av \code{Grönsak}) och alla instanser av denna typ garanterat har attributet \code{vikt} som är av typen \code{Int} så kan kompilatorn vid \emph{kompileringstid} dra slutsatsen att summeringen är giltig och därmed kan kompilatorn kompilera koden till körbar maskinkod.

\SubtaskSolved  
\begin{REPLnonum}
scala> new Grönsak
-- Error:
1 |new Grönsak
  |    ^^^^^^^
  |    Grönsak is a trait; it cannot be instantiated
\end{REPLnonum}

\SubtaskSolved  
\begin{REPLnonum}
scala> val anonymGrönsak = new Grönsak { val vikt = 42 }
val anonymGrönsak: Grönsak = anon$1@1edde8b6
scala> anonymGrönsak.toString                                                                                      
val res0: String = anon$1@1edde8b6
\end{REPLnonum}
Typen är \code{Grönsak} och blir här en s.k. \emph{anonym klass}, eftersom vi inte har använt en namngiven klass med \code{extends}, utan bara ''hängt på'' en klasskropp inom klammerparenteser direkt vid konstruktion. När du skapar anonyma klasser måste du använda nyckelordet \code{new}.

Kompilatorn hittar på ett unikt klassnamn, här anon\$1, för att hålla reda på den anonyma klassen under kompilering till maskinkod. Strängrepresentationen innehåller ett hexadecimalt heltal som är unikt för instansen, här \code{1edde8b6}.

\SubtaskSolved  

\begin{REPLsmall}
scala> new Grönsak { }
-- Error:
1 |new Grönsak { }
  |^
  |object creation impossible, since val vikt: Int in trait Grönsak is not defined 

\end{REPLsmall}


\QUESTEND






\WHAT{Polymorfism vid arv, s.k. subtypspolymorfism.}

\QUESTBEGIN

\Task  \what~  Polymorfism betyder ''många skepnader''. I samband med arv  innebär det att flera subtyper, till exempel \code{Ko} och \code{Gris}, kan hanteras gemensamt som om de vore instanser av samma supertyp, så som \code{Djur}. Subklasser kan implementera en metod med samma namn på olika sätt. Vilken metod som exekveras bestäms vid körtid beroende på vilken subtyp som instansieras. På så sätt kan djur komma i många skepnader.

\Subtask Implementera funktionen \code{skapaDjur} nedan så att den returnerar antingen en ny \code{Ko} eller en ny \code{Gris} med lika sannolikhet.

\begin{REPL}
scala> trait Djur { def väsnas: Unit }
scala> class Ko   extends Djur { def väsnas = println("Muuuuuuu") }
scala> class Gris extends Djur { def väsnas = println("Nöffnöff") }
scala> def skapaDjur(): Djur = ???
scala> val bondgård = Vector.fill(42)(skapaDjur())
scala> bondgård.foreach(_.väsnas)
\end{REPL}

\Subtask Lägg till ett djur av typen Häst som väsnas på lämpligt sätt och modifiera \code{skapaDjur} så att det skapas kor, grisar och hästar med lika sannolikhet.


\SOLUTION


\TaskSolved \what


\SubtaskSolved
\begin{Code}
def skapaDjur(): Djur = 
  if math.random() > 0.5 then Ko() else Gris()
\end{Code}

\SubtaskSolved
\begin{Code}
class Häst extends Djur: 
  def väsnas = println("Gnääääägg") 

def skapaDjur(): Djur = 
   math.random() match
    case r if r < 0.33 => Ko() 
    case r if r < 0.67 => Gris() 
    case _             => Häst()
\end{Code}


\QUESTEND





\WHAT{Olika typer av heltalspar till laborationen \hyperref[section:lab:\LabWeekTEN]{\texttt{\LabWeekTEN}}.}


\QUESTBEGIN


\Task\label{exe:inheritance:labprep-pair}  \what~\textbf{OBS! Gör denna uppgift \textit{innan} du kollar på given kod i labben så att du inte spojlar uppgiften.}

Under veckans laboration ska du använda olika typer av par som representerar riktning och position på en tvådimensionell spelplan, samt spelplanens storlek. I stället för att använda en vanlig 2-tupel till dessa tre olika typer av par ska du skapa egna, specifika  typer som alla ärver bastypen \code{Pair[T]}. Dessa typer ska alla ligga i filen \code{pairs.scala} i \code{package snake}.
\begin{Code}
// detta är en skiss på filen pairs.scala
package snake

trait Pair[T]:
  def x: T
  def y: T
  // uppgift a) lägg till den konkreta metoden tuple

// efterföljande deluppgifterna implementerar dessa subtyper till Pair:
//   case klass Dim beskriver en 2-dimensionell ytas storlek
//   case klass Pos beskriver en position på en yta av Dim storlek
//   enum Dir beskriver förflyttning mot North, South, East, West
\end{Code}
Skillnaden mellan \code{Pair[T]} och en vanlig 2-tupel är att medlemmarna \code{x} och \code{y} garanterat är av \emph{samma} typ, medan en 2-tupel kan innehålla element av olika typ.

I fig. \ref{snake:fig:pairs-uml} visas en bild av klasshierarkin som du steg-för-steg ska utveckla i efterföljande  uppgifter. Fördelen med att ha olika typer av par är att det är mer typsäkert \Eng{type safe}: vi får hjälp av kompilatorn att upptäcka om vi av misstag förväxlar t.ex. en position med en riktning.

\begin{figure}[H]
\begin{center}
\newcommand{\TextBox}[1]{\raisebox{0pt}[1em][0.5em]{#1}}
\tikzstyle{umlclass}=[rectangle, draw=black,  thick, anchor=north, text width=2cm, rectangle split, rectangle split parts = 3]
\begin{tikzpicture}[inner sep=0.5em,scale=1.2, every node/.style={transform shape}]

  \node [umlclass, rectangle split parts = 1, xshift=0cm, yshift=4.5cm] (BaseType1)  {
              \textit{\textbf{\centerline{\TextBox{\code{Pair[T]}}}}}
%              \nodepart[align=left]{second}\code{def x: T} \newline \code{def y: T}
          };


  \node [umlclass, rectangle split parts = 1, xshift=-3cm, yshift=2.5cm] (SubType1)  {
              \textit{\textbf{\centerline{\TextBox{\code{Dim}}}}}
%              \nodepart[align=left]{second}\code{val x: Int} \newline \code{val y: Int}
          };

\node [umlclass, rectangle split parts = 1, xshift=0cm, yshift=2.5cm] (SubType2)  {
            \textit{\textbf{\centerline{\TextBox{\code{Pos}}}}}
%            \nodepart[]{second}\TextBox{\code{val dim: Int}}
        };

\node [umlclass, rectangle split parts = 1, xshift=3cm, yshift=2.5cm] (SubType3)  {
            \textit{\textbf{\centerline{\TextBox{\code{Dir}}}}}
%            \nodepart[]{second}\TextBox{\code{val dim: Int}}
        };


\draw[umlarrow] (SubType1.north) -- ++(0,0.5) -| (BaseType1.south);
\draw[umlarrow] (SubType2.north) -- ++(0,0.5) -| (BaseType1.south);
\draw[umlarrow] (SubType3.north) -- ++(0,0.5) -| (BaseType1.south);

\end{tikzpicture}
\end{center}
\caption{Arvshierarki med \code{Pair[T]} som bastyp.}
\label{snake:fig:pairs-uml}
\end{figure}

\Subtask Öppna en editor och koda \code{trait Pair[T]} i en fil \code{pairs.scala}. Lägg dessutom till en konkret metod \code{tuple} i \code{Pair[T]} som returnerar en 2-tupel med de båda elementen i paret, så att det vid behov går att omvandla \code{Pair}-instanser till 2-tupler. Använd REPL för att testa att detta fungerar:
\begin{REPLnonum}
scala> val p = new Pair[Int] { override val x = 10; override val y = 20 }
p: Pair[Int]{val x: Int; val y: Int} = $anon$1@784223e9

scala> p.tuple
val res0: (Int, Int) = (10,20)
\end{REPLnonum}

\Subtask Fungerar koden ovan även utan nyckelordet \code{override} (testa i REPL)? Varför? När \emph{måste} \code{override} användas? Vad är fördelen resp. nackdelen med att använda \code{override} även när det inte är nödvändigt? 

\Subtask Skapa en case-klass \code{Dim} som ärver \code{Pair[Int]}. Instanser av denna klass kommer du att använda under veckans laboration för att representera en spelplans storlek genom att låta \code{x} ange antalet horisontella positioner och \code{y} antalet vertikala positioner.

Lägg även till ett kompanjonsobjekt \code{Dim} med en \code{apply}-metod som kan skapa \code{Dim}-instanser givet en 2-tupel.
Testa i REPL enligt nedan.
\begin{REPLnonum}
scala> Dim(50, 60)
val res1: Dim = Dim(50,60)

scala> Dim((60, 50))
val res2: Dim = Dim(60,50)

scala> res2.tuple
val res3: (Int, Int) = (60,50)
\end{REPLnonum}

\Subtask Lägg till en case-klass \code{Pos} som ärver \code{Pair[Int]} som representerar en position med en \code{x}-koordinat och en \code{y}-koordinat, båda klassparametrar. Kordinaterna ska hållas inom en spelplansstorlek som ges av klassparametern \code{dim} av typen \code{Dim}. Kordinatpositionerna är heltal och räknas från \code{0} till (men inte med) \code{dim.x} resp. \code{dim.y}.

Gör primärkonstruktorn i case-klassen \code{Pos} \textbf{privat}, genom att skriva nyckelordet \code{private} efter klassnamnet men före klassparameterlistan, så att det inte går att skapa instanser via primärkonstruktorn utanför klasskroppen och kompanjonsobjektet. 

Implementera metoderna \code{+} och \code{-} i case-klassen \code{Pos}. Båda metoderna ska ta en parameter \code{p} av typen \code{Pair[Int]} och returnera en ny \code{Pos}, där \code{p.x} resp. \code{p.y} är adderat resp. subtraherat från aktuell position. Observera att du inte ska skriva \code{new} när du skapar en ny instans, eftersom dessa alltid ska skapas via kompanjonsobjektets \code{apply}-metod, som är en ''smart'' fabriksmetod som garanterar håller koordinaterna inom spelplanen. 

Lägg till ett kompanjonsobjekt \code{Pos} med en \code{apply}-metod som skapar en ny \code{Pos}-instans som ser till att koordinaterna alltid är inom \code{dim}. Aritmetiken ska ske modulo storleken \code{dim}, d.v.s en position ska aldrig kunna hamna utanför spelplanen; i stället så börjar man om på andra sidan (se exempel i REPL nedan). \\ \emph{Tips:} Använd  \code|java.lang.Math.floorMod| som hanterar negativa argument så att resultatet blir positivt (till skillnad från modulo-operatorn \%).

Lägg även till fabriksmetoden \code{random} som kan skapa nya slumpmässiga positioner inom \code{dim}. \emph{Tips:} Använd \code{scala.util.Random.nextInt}.

Testa att det fungerar enligt nedan:
\begin{REPLnonum}
scala> Pos(-1,20,Dim(10,20))
val res4: Pos = Pos(9,0,Dim(10,20))

scala> new Pos(-1,20,Dim(10,20))  // förbjuds med privat primärkonstruktor
-- Error:
1 |new Pos(-1,20,Dim(10,20))
  |    ^^^
  |constructor Pos cannot be accessed as a member of Pos

scala> Pos(0,0,Dim(5,5)) + Pos(6,12, Dim(5,5))                                                                     
val res5: Pos = Pos(1,2,Dim(5,5))

scala> Pos(0,0,Dim(5,5)) - Pos(1,2, Dim(5,5))                                                                     
val res6: Pos = Pos(4,3,Dim(5,5))

scala> for (_ <- 1 to 3) yield Pos.random(Dim(10,10))
val res7: IndexedSeq[Pos] = 
  Vector(Pos(8,8,Dim(10,10)), Pos(2,6,Dim(10,10)), Pos(3,7,Dim(10,10)))
\end{REPLnonum}

\Subtask Vad händer om du glömmer skriva \code{new} när du anropar den privata konstruktorn i din \code{apply}-metod? Varför finns inte detta problem i \code{apply}-metoden för \code{Dim}?

\Subtask Lägg till en \code{enum Dir} som ärver \code{Pair[Int]} och har två \code{val}-parametrar \code{x} och \code{y}. Lägg också till fyra fall med \code{case} som alla ärver \code{Dir} och som representerar en enstegsförflyttning i de fyra väderstrecken, genom att ge parametrarna \code{x} resp. \code{y} något av värden $1$, $-1$ eller $0$. Norrut ska anges med x-koordinaten $0$ och y-koordinaten $-1$, etc. Verifiera i REPL att enumerationen fungerar.

Lägg till en \code{export} som gör så att det räcker att importera \code{snake.*} för att få alla fyra riktningar synliga direkt (annars behövs även import av \code{Dir.*} på alla ställen där riktning används i och utanför paketet \code{snake})


\SOLUTION


\TaskSolved \what

\SubtaskSolved
\begin{CodeSmall}
trait Pair[T]:
  def x: T
  def y: T
  def tuple: (T, T) = (x, y)

\end{CodeSmall}

\SubtaskSolved 
\begin{itemize}
  \item Fungerar koden ovan även utan nyckelordet \code{override}? Varför? 
  \item[] Ja den fungerar eftersom \code{override} ej måste anges när ärvda \emph{abstrakta} medlemmar implementeras i en subtyp. Abstrakta medlemmar saknar implementation och det finns inget som behöver överskuggas. 
  \item När \emph{måste} \code{override} användas? 
  \item[] Det krävs \code{override} om du vill ge en ärvd medlem en \emph{annan} implementation i subtypen, om denna medlemmen redan \emph{har} en implementation i supertypen. Din nya implementation överskuggar (ersätter) den ärvda medlemmens implementation. 
  \item Vad är fördelen resp. nackdelen med att använda \code{override} även när det inte är nödvändigt?
  \item[] \emph{Fördel:} du får hjälp av kompilatorn att kontrollera att du verkligen implementerar en ärvd medlem och inte t.ex. råkat stava medlemmens namn fel.
  \item[] \emph{Nackdel:} mer att skriva och därmed även längre att läsa.
\end{itemize}

\SubtaskSolved
\begin{CodeSmall}
case class Dim(x: Int, y: Int) extends Pair[Int]
object Dim:
  def apply(dim: (Int, Int)): Dim = Dim(dim._1, dim._2)  
\end{CodeSmall}

\SubtaskSolved
\begin{CodeSmall}
case class Pos private (x: Int, y: Int, dim: Dim) extends Pair[Int]:
  def +(p: Pair[Int]): Pos = Pos(x + p.x, y + p.y, dim)
  def -(p: Pair[Int]): Pos = Pos(x - p.x, y - p.y, dim)

object Pos:
  def apply(x: Int, y: Int, dim: Dim): Pos = 
    import java.lang.Math.floorMod as mod
    new Pos(mod(x, dim.x), mod(y, dim.y), dim) //OBS: new nödvändig här!

  def random(dim: Dim): Pos = 
    import scala.util.Random.nextInt as rni
    Pos(rni(dim.x), rni(dim.y), dim)
\end{CodeSmall}

\SubtaskSolved Om du glömmer skriva \code{new} explicit i kompanjonsobjektets \code{apply}-metod så blir det ett rekursivt anrop som resulterar i en oändlig loop vid körtid. Med \code{new} så är det garanterat den privata primärkonstruktorn för \code{Pos} som anropas. 

I \code{Dim.apply} så skiljer sig parametertyperna åt mellan fabriksmetoden och primärkonstruktorn och kompilatorn väljer då primärkonstruktorn eftersom den passar med de givna två separata heltalen och inte med en 2-tupel.

\SubtaskSolved
\begin{CodeSmall}
enum Dir(val x: Int, val y: Int) extends Pair[Int]:
  case North extends Dir( 0, -1)
  case South extends Dir( 0,  1)
  case East  extends Dir( 1,  0)
  case West  extends Dir(-1,  0)
export Dir.*  // gör så att North etc blir synliga i paketet snake
\end{CodeSmall}

\QUESTEND






\WHAT{Supertyp med parameter.}

\QUESTBEGIN

\Task  \what~  Utbildningsdepartementet vill med sitt nya datasystem hålla koll på vissa personer och skapar därför en klasshierarki enligt nedan. Skriv in koden i en editor och testa i REPL med \code{sbt}.
\begin{Code}
class Person(val namn: String)

class Akademiker(
  namn: String,
  val universitet: String) extends Person(namn)

class Student(
  namn: String,
  universitet: String,
  program: String) extends Akademiker(namn, universitet)

class Forskare(
  namn: String,
  universitet: String,
  titel: String) extends Akademiker(namn, universitet)
\end{Code}


\Subtask Deklarera fyra olika \code{val}-variabler med lämpliga namn som refererar till olika instanser av alla olika klasser ovan och ge attributen valfria initialvärden genom olika parametrar till konstruktorerna.

\Subtask Skriv två satser: en som först stoppar in instanserna i en \code{Vector} och en som sedan loopar igenom vektorn och skriv ut alla instansers \code{toString} och \code{namn}.

\Subtask Utbildningsdepartementet vill att det inte ska gå att instansiera objekt av typerna \code{Person} och \code{Akademiker}. Det kan åstadkommas genom att placera nyckelordet \code{abstract} före \code{class}. Uppdatera koden i enlighet med detta. Vilket blir felmeddelande om man försöker instansiera en \code{abstract class}? Går det lika bra med en \code{trait}?

\Subtask Utbildningsdeparetementet vill slippa implementera \code{toString}. Gör därför om typerna \code{Student} och \code{Forskare} till case-klasser. \emph{Tips:} För att undkomma ett kompileringsfel (vilket?) behöver du använda \code{override val} på lämpligt ställe.
Skapa instanser av de nya case-klasserna \code{Student} och \code{Forskare} och skriv ut deras \code{toString}. 

\Subtask 
%Eftersom \code{Person} och \code{Akademiker} nu är abstrakta, vill utbildningsdepartementet att du gör om dessa typer till traits med abstrakta attribut istället för klasser. 
Använd abstrakta attribut i stället för parametrar för typerna som är abstrakta, så att du inte behöver skriva \code{override val} i klassparametrarna till de konkreta case-klasserna.
Du ska också införa en case-klass \code{IckeAkademiker} som ska användas i olika statistiska medborgarundersökningar.
Dessutom förser man alla personer med ett personnummer representerat som en \code{Long}.
Hur ser utbildningsdepartementets kod ut nu, efter alla ändringar? Skriv ett testprogram som skapar några instanser och skriver ut deras attribut.

\SOLUTION


\TaskSolved \what


\SubtaskSolved
\begin{Code}
val person = new Person("Person1")
val akademiker = new Akademiker("Person2", "LTH")
val student = new Student("Person3", "LTH", "D")
val forskare = new Forskare("Person4", "LTH", "Doktorand")
\end{Code}

\SubtaskSolved
\begin{Code}
val vec = Vector(person, akademiker, student, forskare)
for(i <- vec){ print(i.toString + i.namn) }
\end{Code}

\SubtaskSolved  
Felmeddelande vid instansiering av \code{abstract class Akademiker}:\\
\texttt{Akademiker is abstract; it cannot be instantiated}

Det går \emph{inte} lika bra med en \code{trait} i det speciella fallet \code{Akademiker}, eftersom en trait inte får skicka vidare parametrar till en supertyp. Felmeddelande:\\
\texttt{trait Akademiker may not call constructor of trait Person}
\begin{Code}
trait Person(val namn: String)

abstract class Akademiker(
  namn: String,
  val universitet: String) extends Person(namn)

class Student(
  namn: String,
  universitet: String,
  program: String) extends Akademiker(namn, universitet)

class Forskare(
  namn: String,
  universitet: String,
  titel: String) extends Akademiker(namn, universitet)
\end{Code}



\SubtaskSolved  
\begin{REPLnonum}
scala>  
     |trait Person(val namn: String)                                                                              
     | 
     | abstract class Akademiker(
     |   namn: String,
     |   val universitet: String) extends Person(namn)
     | 
     | case class Student(
     |   namn: String,
     |   universitet: String,
     |   program: String) extends Akademiker(namn, universitet)
     | 
     | case class Forskare(
     |   namn: String,
     |   universitet: String,
     |   titel: String) extends Akademiker(namn, universitet)
-- Error:     
8 |  namn: String,
  |  ^
  |  error overriding value namn in trait Person of type String;
  |    value namn of type String needs `override` modifier
-- Error:
9 |  universitet: String,
  |  ^
  |  error overriding value universitet in class Akademiker of type String;
  |    value universitet of type String needs `override` modifier
-- Error:
13 |  namn: String,
   |  ^
   |  error overriding value namn in trait Person of type String;
   |    value namn of type String needs `override` modifier
-- Error:
14 |  universitet: String,
   |  ^
   |  error overriding value universitet in class Akademiker of type String;
   |    value universitet of type String needs `override` modifier
\end{REPLnonum}

\begin{Code}
trait Person(val namn: String)

abstract class Akademiker(
  namn: String,
  val universitet: String) extends Person(namn)

case class Student(
  override val namn: String,
  override val universitet: String,
  program: String) extends Akademiker(namn, universitet)

case class Forskare(
  override val namn: String,
  override val universitet: String,
  titel: String) extends Akademiker(namn, universitet)
\end{Code}

\begin{REPLsmall}
scala> val ps = Vector(Student("Kim", "Lund", "D"), Forskare("Herz", "Lund", "Dr"))
val ps: Vector[Akademiker] = Vector(Student(Kim,Lund,D), Forskare(Herz,Lund,Dr))
scala> ps :+ new Person("Abstrakt") {}
val res0: Vector[Person] = 
  Vector(Student(Kim,Lund,D), Forskare(Herz,Lund,Dr), anon1@1941bbf3)
\end{REPLsmall}

\SubtaskSolved
\begin{Code}
trait Person: 
  val namn: String 
  val nbr: Long

trait Akademiker extends Person:
  val universitet: String

case class Student(
  namn: String,
  nbr: Long,
  universitet: String,
  program: String) extends Akademiker

case class Forskare(
  namn: String,
  nbr: Long,
  universitet: String,
  titel: String) extends Akademiker

case class IckeAkademiker(
    namn: String,
    nbr: Long) extends Person
\end{Code}



\QUESTEND




%\clearpage




\ExtraTasks %%%%%%%%%%%%%%%%%





%\WHAT{Uppräknade värden.}

%\QUESTBEGIN

% \Task  \what~  Ett sätt att hålla reda på uppräknade värden, t.ex. färgen på olika kort i en kortlek, är att använda olika heltal som får representera de olika värdena, till exempel så här:\footnote{Om namnkonventioner för konstanter i Scala: läs under rubriken ''Constants, Values, Variable and Methods'' här \href{http://docs.scala-lang.org/style/naming-conventions.html}{docs.scala-lang.org/style/naming-conventions.html}}
% \begin{Code}
% object Färg {
%   val Spader = 1
%   val Hjärter = 2
%   val Ruter = 3
%   val Klöver = 4
% }
% \end{Code}
% Dessa kan sedan användas som parametrar till denna case-klass vid skapande av kortobjekt:
% \begin{lstlisting}[language=,keywords={case,class}]
% case class Kort(färg: Int, valör: Int)
% \end{lstlisting}
% Man kan hålla reda på färgen med en variabel av typen \code{Int} och tilldela den en viss färg med ovan konstanter. Och när du skapar ett kort kan du använda färgnamnet och du slipper därmed att behöva komma ihåg vilket heltal som representerar färgen.
% \begin{REPL}
% scala> val f = Färg.Spader
% scala> import Färg._
% scala> Kort(Ruter, 7)
% \end{REPL}
% En annan fördelen med detta är att man lätt kan iterera över alla färger:
% \begin{REPL}
% scala> val kortlek = for (f <- 1 to 4; v <- 1 to 13) yield Kort(f, v)
% \end{REPL}
% Men den stora nackdelen med detta är att kompilatorn vid kompileringstid inte kollar om variablerna av misstag råkar ges något värde utanför det giltiga intervallet, eftersom alla heltal är möjliga. Detta får vi själv hålla koll på vid körtid, till exempel med funktionen \code{require} eller \code{if}-satser, etc.

% Istället kan man använda uppräknade värden med hjälp av case-objekt enligt nedan deluppgifter och därmed få hjälp av kompilatorn att hitta eventuella fel vid kompileringstid.  Ett case-objekt är som ett vanligt singelton-objekt men det får bl.a. automatiskt en \code{toString} som är samma som namnet. Case-objekt kan dessutom användas som värden i mönstermatchningar (mer om detta i kapitel \ref{chapter:W10}).

% \Subtask Deklarera följande uppräknade värden som singelton-objekt med gemensam bastyp. Med nyckelordet \code{sealed} så ''förseglas'' klassen och inga andra direkta subtyper tillåts förutom de som finns i samma kod-fil eller block. I detta exempel  med kortfärger vet vi ju att det inte finns fler än dessa fyra färger.
% \begin{Code}
% sealed trait Färg
% case object Spader extends Färg
% case object Hjärter extends Färg
% case object Ruter extends Färg
% case object Klöver extends Färg
% \end{Code}
% Dessa kan sedan användas som parametrar till denna case-klass vid skapande av kortobjekt:
% \begin{lstlisting}[language=,keywords={case,class}]
% case class Kort(färg: Färg, valör: Int)
% \end{lstlisting}
% Skapa därefter några exempelinstanser av klassen \code{Kort}. Vad är fördelen med ovan angreppssätt jämfört med att använda heltalskonstanter?

% \Subtask Om man vill kunna iterera över alla värden är det bra om de finns i en samling med alla värden. Vi kan lägga en sådan i ett kompanjonsobjekt till bastypen enligt nedan. Skriv ut alla färgvärden med en \code{for}-sats.

% \begin{Code}
% sealed trait Färg
% object Färg {
%   val values = Vector(Spader, Hjärter, Ruter, Klöver)
% }
% case object Spader extends Färg
% case object Hjärter extends Färg
% case object Ruter extends Färg
% case object Klöver extends Färg
% \end{Code}
% Skapa en kortlek med 52 olika kort och blanda den sedan med \code{Random.shuffle} enligt nedan. Använd en dubbel \code{for}-sats och \code{yield}.
% \begin{REPL}
% scala> val kortlek: Vector[Kort] = ???
% scala> val blandad = scala.util.Random.shuffle(kortlek)
% \end{REPL}

% \Subtask Skriv en funktion \code{ def blandadKortlek: Vector[Kort] = ???} som ger en ny blandad kortlek varje gång den anropas med metoden i föregående uppgift.

% \Subtask Om man även vill ha en heltalsrepresentation med en medlem \code{toInt} för alla värden, kan man ge bastypen en parameter och i stället för en trait (som inte kan ha några parametrar) använda en abstrakt klass.

% \begin{Code}
% sealed abstract class Färg(final val toInt: Int)
% object Färg {
%   val values = Vector(Spader, Hjärter, Ruter, Klöver)
% }
% case object Spader  extends Färg(0)
% case object Hjärter extends Färg(1)
% case object Ruter   extends Färg(2)
% case object Klöver  extends Färg(3)
% \end{Code}
% Skapa en funktion \code{färgPoäng} som räknar ut summan av heltalsrepresentationen av alla färger hos en vektor med kort, och använd den sedan för att räkna ut \code{färgPoäng} för de första fem korten.
% \begin{REPL}
% scala> def blandadKortlek: Vector[Kort] = ???
% scala> def färgPoäng(xs: Vector[Kort]): Int = ???
% scala> färgPoäng(blandadKortlek.take(5))
% \end{REPL}


% \SOLUTION

% \TaskSolved \what

% \SubtaskSolved  Sättet är säkrare då man inte kan tilldela korten en färg som inte finns. Med heltalskonstanterna kan man till exempel ge ett kort färgen 5, vilken inte korresponderar till någon riktig färg.

% \SubtaskSolved  \code{for (f <- Färg.values; v <- 1 to 13) yield Kort(f,v)}

% \SubtaskSolved
% \begin{Code}
% def blandadKortlek: Vector[Kort] = {
%   val kortlek =
%     for (f <- Färg.values; v <- 1 to 13) yield Kort(f,v)
%   scala.util.Random.shuffle(kortlek)
% }
% \end{Code}

% \SubtaskSolved  \code{def färgPoäng(xs: Vector[Kort]): Int = xs.map(_.färg.toInt).sum}

% \QUESTEND







\WHAT{Bastypen \code{Shape} och subtyperna \code{Rectangle} och \code{Circle}.}

\QUESTBEGIN

\Task  \what~  Du ska i denna uppgift skapa ett litet bibliotek för geometriska former med oföränderliga objekt implementerade med hjälp av case-klasser. De geometriska formerna har en gemensam bastyp kallad \code{Shape}. Utgå från koden nedan.

\begin{CodeSmall}
case class Point(x: Double, y: Double):
  def move(dx: Double, dy: Double): Point = Point(x + dx, y + dy)

trait Shape:
  def pos: Point
  def move(dx: Double, dy: Double): Shape

case class Rectangle(pos: Point, width: Double, height: Double) extends Shape:
  def move(dx: Double, dy: Double): Rectangle = copy(pos = pos.move(dx, dy))

case class Circle(pos: Point, radius: Double) extends Shape:
  def move(dx: Double, dy: Double): Circle = copy(pos = pos.move(dx, dy))

\end{CodeSmall}

\Subtask Instansiera några cirklar och rektanglar och gör några relativa förflyttningar av dina instanser genom att anropa \code{move}.

\Subtask Lägg till en konkret metod \code{moveTo} i \code{Point} som gör en absolut förflyttning till koordinaterna \code{x} och \code{y}. Lägg till en abstrakt metod \code{moveTo} \code{Shape} som implementeras i subklasserna. Testa med REPL på några instanser av \code{Rectangle} och \code{Circle}.

\Subtask Lägg till metoden \code{distanceTo(that: Point): Double } i case-klassen \code{Point} som räknar ut avståndet till en annan punkt med hjälp av \code{math.hypot}. Klistra in i REPL och testa på några instanser av \code{Point}.

\Subtask Lägg till en konkret metod \code{distanceTo(that: Shape): Double } i traiten \code{Shape} som räknar ut avståndet till positionen för en annan Shape. Testa i REPL på några instanser av \code{Rectangle} och \code{Circle}.

\Subtask Gör så att \code{distanceTo} kan anropas med operatornotation.

\SOLUTION


\TaskSolved \what


\SubtaskSolved
\begin{CodeSmall}
val c1 = Circle(Point(1, 1), 42)
val r1 = Rectangle(Point(3, 3), 20, 30)
c1.move(2, 3)
r1.move(3, 2)
\end{CodeSmall}

\SubtaskSolved  
\begin{CodeSmall}
case class Point(x: Double, y: Double):
  def move(dx: Double, dy: Double): Point = Point(x + dx, y + dy)
  def moveTo(x: Double, y: Double): Point = Point(x, y)

trait Shape:
  def pos: Point
  def move(dx: Double, dy: Double): Shape
  def moveTo(x: Double, y: Double): Shape

case class Rectangle(pos: Point, width: Double, height: Double) extends Shape:
  def move(dx: Double, dy: Double): Shape = copy(pos = pos.move(dx, dy))
  def moveTo(x: Double, y: Double): Shape = copy(pos.moveTo(x, y))

case class Circle(pos: Point, radius: Double) extends Shape:
  def move(dx: Double, dy: Double): Shape = copy(pos = pos.move(dx, dy))
  def moveTo(x: Double, y: Double): Shape = copy(pos.moveTo(x, y))
\end{CodeSmall}


\SubtaskSolved \code{def distanceTo(that: Point): Double = math.hypot(that.x - x, that.y - y)}

\SubtaskSolved \code{def distanceTo(that: Shape): Double = pos.distanceTo(that.pos)}.

\SubtaskSolved  
\begin{CodeSmall}
case class Point(x: Double, y: Double):
  def move(dx: Double, dy: Double): Point = Point(x + dx, y + dy)
  def moveTo(x: Double, y: Double): Point = Point(x, y)
  infix def distanceTo(that: Point): Double = math.hypot(that.x - x, that.y - y)

trait Shape:
  def pos: Point
  def move(dx: Double, dy: Double): Shape
  def moveTo(x: Double, y: Double): Shape
  infix def distanceTo(that: Shape): Double = pos.distanceTo(that.pos)

case class Rectangle(pos: Point, width: Double, height: Double) extends Shape:
  def move(dx: Double, dy: Double): Shape = copy(pos = pos.move(dx, dy))
  def moveTo(x: Double, y: Double): Shape = copy(pos.moveTo(x, y))

case class Circle(pos: Point, radius: Double) extends Shape:
  def move(dx: Double, dy: Double): Shape = copy(pos = pos.move(dx, dy))
  def moveTo(x: Double, y: Double): Shape = copy(pos.moveTo(x, y))
\end{CodeSmall}

\QUESTEND






% \WHAT{Regler för \code{override}, \code{private} och \code{final}.}

% \QUESTBEGIN

% \Task  \what~

% \Subtask \label{subtask:overriderules} Undersök överskuggningning av abstrakta, konkreta, privata och finala medlemmar genom att skriva in raderna nedan en i taget i REPL. Vilka rader ger felmeddelande? Varför? Vid felmeddelande: notera hur felmeddelandet lyder och ändra deklarationen av den felande medlemmen så att koden blir kompilerbar (eller om det är enda rimliga lösningen: ta bort den felaktiga medlemmen), innan du provar efterkommande rad.

% \begin{REPL}
% trait Super1 { def a: Int; def b = 42; private def c = "hemlis" }
% class Sub2 extends Super1 { def a = 43; def b = 43; def c = 43 }
% class Sub3 extends Super1 { def a = 43; override def b = 43 }
% class Sub4 extends Super1 { def a = 43; override def c = "43" }
% trait Super5 { final def a: Int; final def b = 42 }
% class Sub6 extends Super5 { override def a = 43; def b = 43 }
% class Sub7 extends Super5 { def a = 43; override def b = 43 }
% class Sub8 extends Super5 { def a = 43; override def c = "43" }
% trait Super9 { val a: Int; val b = 42; lazy val c: String = "lazy" }
% class Sub10 extends Super9 { override def a = 43; override val b = 43 }
% class Sub11 extends Super9 { val a = 43; override lazy val b = 43 }
% class Sub12 extends Super9 { val a = 43; override var b = 43 }
% class Sub13 extends Super9 { val a = 43; override lazy val c = "still lazy" }
% class SubSub extends Sub13 { override val a = 44}
% trait Super14 { var a: Int; var b = 42; var c: String }
% class Sub15 extends Super14 { def a = 43; override var b = 43; val c = "?" }
% \end{REPL}

% \Subtask Skapa instanser av klasserna \code{Sub3}, \code{Sub13} och \code{SubSub} från ovan deluppgift och undersök alla medlemmarnas värden för respektive instans. Förklara varför de har dessa värden.

% %\Subtask Läs igenom reglerna i kapitel  \ref{slideW07:overriderules} om vad som gäller vid arv och överskuggning av medlemmar. Gör några egna undersökningar i REPL som försöker bryta mot någon regel som inte testades i deluppgift \ref{subtask:overriderules}.

% \SOLUTION


% \TaskSolved \what


% \SubtaskSolved  2. Måste ha \code{override} framför \code{b} för att kunna ändra på metoden. \\
% 4. \code{c} är \code{private}, vilket betyder att den är gömd för subklasserna. Därför kan den inte överskuggas. Genom att ta bort \code{override} fungerar klassen. \\
% 5. En \code{final}-medlem måste ha ett bestämt värde. Kan lösas genom att tilldela \code{final a} ett värde eller ta bort \code{final}. \\
% 6. En \code{final}-medlem kan inte överskuggas, varken med eller utan \code{override}. Här får konflikterna tas bort.  \\
% 7. Se 6. \\
% 8. Eftersom \code{c} inte finns i \code{Super5} kan den inte överskuggas. Genom att ta bort \code{override} fungerar klassen. \\
% 10. Överskuggningen av \code{val} måste vara oföränderlig (immutable); detta är inte nödvändigtvis \code{def}. Löses genom att byta ut \code{def a} mot \code{val a} hos \code{Sub10}.  \\
% 11. Samma problem som i 10.; \code{lazy val} kan vara föränderlig. Löses genom att ta bort \code{lazy}. \\
% 12. Samma problem igen! \code{var} är föränderlig, vilket bryter mot typsäkerheten när man försöker överskugga en \code{val}. Löses genom att ändra \code{var} till \code{val}. \\
% 15.\code{def a = 43} och \code{val c = "?"} täcker inte allt som \code{var} kräver. Det behövs en setter för att kunna uppfylla kraven för överskuggning för en \code{var}. Dessutom finns det ingen anledning för en \code{val} att överskuggas; man kan ju ändra på den lite hur man vill!

% \SubtaskSolved  Sub3: a = 43, b = 43 eftersom medlemmen är överskuggad. c hittas inte eftersom den är \code{private}.

% Sub13: a = 43, b = 42, c = "still lazy" eftersom medlemmen överskuggas.

% SubSub: a = 44 eftersom medlemmen överskuggas, b = 42, c = "still lazy".

% \SubtaskSolved  -.


% \QUESTEND





%\clearpage





\AdvancedTasks %%%%%%%%%%%%%%%%%

% \WHAT{Använda \code{trait} eller \code{class}?}

% \QUESTBEGIN

% \Task \what~ I vilka sammanhang är det nödvändigt att använda en \code{trait} respektive en \code{class}? Läs här för fördjupning:\\  \href{http://www.artima.com/pins1ed/traits.html\#12.7}{http://www.artima.com/pins1ed/traits.html\#12.7}.


% \SOLUTION


% \TaskSolved \what~Man måste använda en klass om man behöver klassparametrar. Man måste använda en trait om man vill göra in-mixning med \code{with}. \\

%  \QUESTEND



\WHAT{Inmixning.}

\QUESTBEGIN

\Task \label{task:fyle} \what~   Man kan utvidga en klass med multipla traits med en kommaseparerad lista. På så sätt kan man fördela medlemmar i olika traits och återanvända gemensamma koddelar genom så kallad \textbf{inmixning}, så som nedan exempel visar.

En alternativ fågeltaxonomi, speciellt populär i Skåne, delar in alla fåglar i två specifika kategorier: Kråga respektive Ånka. Krågor kan flyga men inte simma, medan Ånkor kan simma och oftast även flyga. Fågel i generell, kollektiv bemärkelse kallas på gammal skånska för Fyle.%
\footnote{\href{http://www.klangfix.se/ordlista.htm}{www.klangfix.se/ordlista.htm}}

\begin{CodeSmall}
trait Fyle:
  val läte: String
  def väsnas: Unit = print(läte * 2)
  val ärSimkunnig: Boolean
  val ärFlygkunnig: Boolean

trait KanSimma       { val ärSimkunnig = true }
trait KanInteSimma   { val ärSimkunnig = false }
trait KanFlyga       { val ärFlygkunnig = true }
trait KanKanskeFlyga { val ärFlygkunnig = math.random() < 0.8 }

class Kråga extends Fyle, KanFlyga, KanInteSimma:
  val läte = "krax"

class Ånka extends Fyle, KanSimma, KanKanskeFlyga: 
  val läte = "kvack"
  override def väsnas = print(läte * 4)
\end{CodeSmall}

\Subtask En flitig ornitolog hittar 42 fåglar i en perfekt skog där alla fågelsorter är lika sannolika, representerat av vektorn \code{fyle} nedan. Skriv i REPL ett uttryck som undersöker hur många av dessa som är flygkunniga Ånkor, genom att använda metoderna \code{filter}, \code{isInstanceOf}, \code{ärFlygkunnig} och \code{size}.

\begin{REPL}
scala> val fyle =
         Vector.fill(42)(if math.random() > 0.5 then new Kråga else new Ånka)
scala> fyle.foreach(_.väsnas)
scala> val antalFlygånkor: Int = ???
\end{REPL}

\Subtask \label{subtask:fyle:sound} Om alla de fåglar som ornitologen hittade skulle väsnas exakt en gång var, hur många krax och hur många kvack skulle då höras? Använd metoderna \code{filter} och \code{size}, samt predikatet \code{ärSimkunnig} för att beräkna antalet krax respektive kvack.
\begin{REPL}
scala> val antalKrax: Int = ???
scala> val antalKvack: Int = ???
\end{REPL}

\SOLUTION


\TaskSolved \what


\SubtaskSolved
Det finns många olika sätt, några exempellösningar:
\begin{Code}
val antalFlygånkor: Int = 
  fyle.count(f => f.isInstanceOf[Ånka] && f.ärFlygkunnig)
\end{Code}

\begin{Code}
val antalFlygånkor: Int = 
  fyle.filter(f => f.isInstanceOf[Ånka] && f.ärFlygkunnig).size
\end{Code}

\begin{Code}
val antalFlygånkor: Int = 
  fyle.collect{case f: Ånka if f.ärFlygkunnig}.size
\end{Code}

\begin{Code}
val antalFlygånkor: Int = fyle.map(_ match
  case f: Ånka if f.ärFlygkunnig => 1
  case _ => 0
).sum
\end{Code}

\SubtaskSolved
\begin{Code}
val antalKrax: Int = fyle.filter(f => !f.ärSimkunnig).size * 2
val antalKvack: Int = fyle.filter(f => f.ärSimkunnig).size * 4
\end{Code}


\QUESTEND











\WHAT{Finala klasser.}

\QUESTBEGIN

\Task  \what~  Om man vill förhindra att man kan göra \code{extends} på en klass kan man göra den final genom att placera nyckelordet \code{final} före nyckelordet \code{class}.

\Subtask Eftersom klassificeringen av fåglar i uppgiften ovan i antingen Ånkor eller Krågor är fullständig och det inte finns några subtyper till varken Ånkor eller Krågor är det lämpligt att göra dessa finala. Ändra detta i din kod.

\Subtask Testa att ändå försöka göra en subklass \code{Simkråga extends Kråga}. Vad ger kompilatorn för felmeddelande om man försöker utvidga en final klass?


\SOLUTION


\TaskSolved \what


\SubtaskSolved  Sätt \code{final} framför \code{class} i klasserna.

\SubtaskSolved  error: illegal inheritance from final class Kråga.


\QUESTEND






\WHAT{Accessregler vid arv och nyckelordet \code{protected}.}

\QUESTBEGIN

\Task  \what~  Om en medlem i en supertyp är privat så kan man inte komma åt den i en subklass. Ibland vill man att subklassen ska kunna komma åt en medlem även om den ska vara otillgänglig i annan kod.

\begin{Code}
trait Super:
  private val minHemlis = 42
  protected val vårHemlis = 42

class Sub extends Super:
  def avslöja = minHemlis
  def kryptisk = vårHemlis * math.Pi

\end{Code}

\Subtask Vad blir felmeddelandet när klassen \code{Sub} försöker komma åt \code{minHemlis}?

\Subtask Deklarera \code{Sub} på nytt, men nu utan den förbjudna metoden \code{avslöja}. Vad blir felmeddelandet om du försöker komma åt \code{vårHemlis} via en instans av klassen \code{Sub}? Prova till exempel med \code{(new Sub).vårHemlis}

\Subtask Fungerar det att anropa metoden \code{kryptisk} på instanser av klassen \code{Sub}?

\SOLUTION


\TaskSolved \what


\SubtaskSolved  
\begin{REPL}
2 |  def avslöja = minHemlis
  |                ^^^^^^^^^
  |                Not found: minHemlis
\end{REPL}

\SubtaskSolved  
\begin{REPL}
scala> class Sub extends Super:
         def kryptisk = vårHemlis * math.Pi
scala> (new Sub).vårHemlis
-- Error:
1 |(new Sub).vårHemlis
  |^^^^^^^^^^^^^^^^^^^
  |value vårHemlis in trait Super cannot be accessed as a member of Sub.
  | Access to protected value vårHemlis not permitted because enclosing object 
  | is not a subclass of trait Super where target is defined
\end{REPL}

\SubtaskSolved  Ja.


\QUESTEND






\WHAT{Använding av \code{protected}.}

\QUESTBEGIN

\Task  \what~  Den flitige ornitologen från uppgift \ref{task:fyle} ska ringmärka alla 42 fåglar hen hittat i skogen. När hen ändå håller på bestämmer hen att även försöka ta reda på hur mycket oväsen som skapas av respektive fågelsort. För detta ändamål apterar den flitige ornitologen en Linuxdator på allt infångat fyle. Du ska hjälpa ornitologen att skriva programmet.

\Subtask Inför en \code{protected var räknaLäte} i traiten \code{Fyle} och skriv kod på lämpliga ställen för att räkna hur många läten som respektive fågelinstans yttrar.

\Subtask Inför en metod \code{antalLäten} som returnerar antalet krax respektive kvack som en viss fågel yttrat sedan dess skapelse.

\Subtask Varför inte använda \code{private} i stället for \code{protected}?

\Subtask Varför är det bra att göra räknar-variabeln oåtkomlig från ''utsidan''?



\SOLUTION


\TaskSolved \what


\SubtaskSolved  I Fyle:
\begin{Code}
protected var räknaLäte: Int = 0
def väsnas: Unit = { print(läte * 2); räknaLäte += 2 }
\end{Code}

I Ånka: \code| override def väsnas = { print(läte * 4); räknaLäte += 4 }|

\SubtaskSolved  \code{ def antalLäten: Int = räknaLäte }

\SubtaskSolved  Om en klass som representerar en fågel som skulle ge ifrån sig fler/färre läten än en vanlig \code{Fyle}, behöver \code{väsnas} ändras. Denna metod behöver tillgång till \code{räknaLäte}, vilken inte får vara \code{private}.

\SubtaskSolved  Räknar-variabeln ska inte kunna påverkas i någon annan del av programmet.


\QUESTEND





\WHAT{Inmixning av egenskaper.}

\QUESTBEGIN

\Task  \what~ Det visar sig att vår flitige ornitolog från uppgift \ref{task:fyle} på sidan \pageref{task:fyle} sov på en av föreläsningarna i zoologi när hen var nolla på Natfak, och därför helt missat fylekategorin \code{Pjodd}. Hjälp vår stackars ornitolog så att fylehierarkin nu även omfattar Pjoddar. En Pjodd kan flyga som en Kråga men den \code{ÄrLiten} medan en Kråga \code{ÄrStor}. En Pjodd kvittrar dubbelt så många gånger som en Ånka kvackar. En Pjodd \code{KanKanskeSimma} där simkunnighetssannolikheten är $0.2$. Låt ornitologen ånyo finna 42 slumpmässiga fåglar i skogen och filtrera fram lämpliga arter. Undersök sedan hur dessa väsnas, i likhet med deluppgift \ref{task:fyle}\ref{subtask:fyle:sound}.


\SOLUTION

\TaskSolved \what


\begin{Code}
trait Fyle:
  val läte: String
  def väsnas: Unit = { print(läte * 2); räknaLäte += 2 }
  protected var räknaLäte: Int = 0
  val ärSimkunnig: Boolean
  val ärFlygkunnig: Boolean
  val ärStor : Boolean
  def antalLäten: Int = räknaLäte

trait KanSimma { val ärSimkunnig = true }
trait KanInteSimma { val ärSimkunnig = false }
trait KanFlyga { val ärFlygkunnig = true }
trait KanKanskeFlyga { val ärFlygkunnig = math.random() < 0.8 }
trait KanKanskeSimma { val ärSimkunnig = math.random() < 0.2 }
trait ÄrStor { val ärStor = true }
trait ÄrLiten { val ärStor = false }

final class Kråga extends Fyle, KanFlyga, KanInteSimma, ÄrStor:
  val läte = "krax"

final class Ånka extends Fyle, KanSimma, KanKanskeFlyga, ÄrStor:
  val läte = "kvack"
  override def väsnas = { print(läte * 4); räknaLäte += 4 }

final class Pjodd extends Fyle, KanFlyga, KanKanskeSimma, ÄrLiten:
  val läte = "kvitter"
  override def väsnas = { print(läte * 8); räknaLäte += 8 }
\end{Code}

I REPL:
\begin{REPL}
val fyle = Vector.fill(42)(
  if math.random() < 0.33 then Kråga()
  else if math.random() < 0.5 then Ånka()
  else Pjodd()
)
fyle.filter(f => f.isInstanceOf[Kråga]).size * 2
fyle.filter(f => f.isInstanceOf[Ånka]).size * 4
fyle.filter(f => f.isInstanceOf[Pjodd]).size * 8
\end{REPL}

\QUESTEND





% \WHAT{Typtest och typkonvertering.}

% \QUESTBEGIN

% \Task  \what~I Scala kan man testa körtidstyp och samtidigt konvertera till en mer specifik typ på ett typsäkert sätt med hjälp av \emph{mönstermatchning} i \code{match}-uttryck som vi ska se i kommande övning \texttt{\ExeWeekTEN}. För att underlätta interoperabilitet med Java finns  Scala-metoderna \code{isInstanceOf} och \code{asInstanceOf}, som motsvarar hur typtest och typkonvertering görs i Java.\footnote{\code{isInstanceOf} och \code{asInstanceOf} används sällan i Scala eftersom \code{match} är kraftfullare och säkrare.}

% Gör nedan deklarationer.
% \begin{REPL}
% scala> trait A; trait B extends A; class C extends B; class D extends B
% scala> val (c, d) = (new C, new D)
% scala> val a: A = c
% scala> val b: B = d
% \end{REPL}

% \Subtask Rita en bild över vilka typer som ärver vilka.

% \Subtask Vilket resultat ger dessa typtester? Varför?
% \begin{REPL}
% scala> c.isInstanceOf[C]
% scala> c.isInstanceOf[D]
% scala> d.isInstanceOf[B]
% scala> c.isInstanceOf[A]
% scala> b.isInstanceOf[A]
% scala> b.isInstanceOf[D]
% scala> a.isInstanceOf[B]
% scala> c.isInstanceOf[AnyRef]
% scala> c.isInstanceOf[Any]
% scala> c.isInstanceOf[AnyVal]
% scala> c.isInstanceOf[Object]
% scala> 42.isInstanceOf[Object]
% scala> 42.isInstanceOf[Any]
% \end{REPL}

% \Subtask Vilka av dessa typkonverteringar ger felmeddelande? Vilket och varför?
% \begin{REPL}
% scala> c.asInstanceOf[B]
% scala> c.asInstanceOf[A]
% scala> d.asInstanceOf[C]
% scala> a.asInstanceOf[B]
% scala> a.asInstanceOf[C]
% scala> a.asInstanceOf[D]
% scala> a.asInstanceOf[E]
% scala> b.asInstanceOf[A]
% \end{REPL}



% \SOLUTION


% \TaskSolved \what


% \SubtaskSolved  B ärver A. C och D ärver B.

% \SubtaskSolved  1. True eftersom c är av typen C. \\
% 2. False eftersom c inte är av typen D. \\
% 3. True eftersom d är av typen D som är en subtyp av B. \\
% 4. True eftersom c är av typen C som är en subtyp av B, som i sin tur är en subtyp av A. \\
% 5. True eftersom b är av typen D, som är en subtyp av B, som i sin tur är en subtyp av A. \\
% 6. True eftersom b är av typen D. \\
% 7. True eftersom a är av typen C som är en subtyp av B. \\
% 8. True eftersom c är av typen C som är en subtyp av AnyRef. \\
% 9. True eftersom c är av typen C som är en subtyp av Any. \\
% 10. Error eftersom \code{isInstanceOf} inte kan använda sig av \code{AnyVal}.  \\
% 11. True eftersom c är av typen C som är en subtyp av Object (Object är java-representationen av AnyRef). \\
% 12. Error eftersom \code{isInstanceOf} inte kan testa om värdetyper (i detta fallet \code{42}) är referenstyper. \\
% 13. True eftersom \code{42} är av typen \code{Int} som är en subtyp av Any. \\

% \SubtaskSolved  3. Går inte eftersom c inte är av typen D, utan typen C. \\
% 6. Går inte eftersom a inte är av typen D, utan typen C. \\
% 7. Går inte eftersom typen E inte finns. \\


% \QUESTEND













% \WHAT{Saknad referens med \texttt{null} och bottentypen \texttt{Nothing}.}

% \QUESTBEGIN

% \Task  \what~ Hitta på en egen fördjupningsuppgift inspirerat av denna artikel på Stackoverflow: \url{http://stackoverflow.com/questions/16173477/usages-of-null-nothing-unit-in-scala}

% \SOLUTION


% \QUESTEND






\WHAT{Arvshierarki med matematiska tal.}

\QUESTBEGIN

\Task  \what~ Studera den djupa arvshierarkin i paketet \code{numbers} i koden på efterföljande sidor. Paketet  \code{numbers} modellerar olika sorters tal i matematiken, med syftet att erbjuda ett s.k. DSL \footnote{\url{https://en.wikipedia.org/wiki/Domain-specific_language}}, alltså ett specialspråk för en viss applikationsdomän\footnote{\url{https://stackoverflow.com/questions/49216312/what-is-dsl-in-scala}}, här: domänen matematiska tal.

Du kan ladda ner koden för \code{numbers} här: \\
\href{https://github.com/lunduniversity/introprog/blob/master/compendium/examples/numbers.scala}{github.com/lunduniversity/introprog/blob/master/compendium/examples/numbers.scala}
\\ Notera speciellt metoden \code{reduce} som reducerar ett tal till sin enklaste form. Metoden \code{reduce} överskuggas på lämpliga ställen med relevant reduktion.

\Subtask Rita en bild över typhierarkin, t.ex. som ett upp-och-nedvänt träd med bastypen  \code{Number} som rot.

\Subtask Skriv kod som använder de olika konkreta klasserna i \code{package numbers}. 
\begin{REPL}
scala> numbers.  // Tryck Tab
AbstractComplex   AbstractNatural    AbstractReal   Frac    Nat      Polar
AbstractInteger   AbstractRational   Complex        Integ   Number   Real

scala> numbers.Integ(12)
res0: numbers.Integ = Integ(12)

scala> import numbers.Syntax._
import numbers.Syntax._

scala> 42.j
res1: numbers.Complex = Complex(Real(0),Real(42))

scala> 42.42.j
res2: numbers.Complex = Complex(Real(0),Real(42.42))

\end{REPL}

\Subtask Ändra på metoden \code{+} i \code{trait Number} så att den blir abstrakt och implementera den i alla konkreta klasser.

\Subtask Implementera fler räknesätt och bygg vidare på koden så som du finner intressant.

\Subtask Gör så att metoden \code{reduce} i klassen \code{AbstractRational} använder algoritmen Greatest Common Divisor (GCD)\footnote{\url{https://sv.wikipedia.org/wiki/St\%C3\%B6rsta\_gemensamma\_delare}} så som beskrivs här: \\ \href{http://www.artima.com/pins1ed/functional-objects.html#6.8}{www.artima.com/pins1ed/functional-objects.html\#6.8} \\ så att täljare och nämnare blir så små som möjligt.

%\clearpage

\scalainputlisting[numbers=left, basicstyle=\ttfamily\fontsize{9.1}{12.2}\selectfont]{examples/numbers.scala}\SOLUTION


\QUESTEND


%!TEX encoding = UTF-8 Unicode
%!TEX root = ../exercises.tex

\ifPreSolution

\Exercise{\ExeWeekELEVEN}\label{exe:W11}

\begin{Goals}
\item Kunna förklara vad en kontextparameter är. 
\item Kunna förklara nyttan med kontextparametrar jämfört med en globala variabler och defaultargument vid lösning av konfigurationsproblemet.
%\item Känna till hur givna sorteringsordningar används för egendefinierade typer.
\item Kunna använda enkla kontextuella abstraktioner med \code{given} och \code{using}.
%\item Känna till existensen av, funktionen hos, och relationen mellan klasserna \code{Ordering} och \code{Comparator}, samt  \code{Ordered} och \code{Comparable}.

\end{Goals}

\begin{Preparations}
\item \StudyTheory{11}
\end{Preparations}

\BasicTasks %%%%%%%%%%%%%%%%

\else

\ExerciseSolution{\ExeWeekELEVEN}

\BasicTasks %%%%%%%%%%%

\fi

\WHAT{Kontextparameter.}

\QUESTBEGIN

\Task  \what~Deklarera följande funktioner som tar ett heltal som kontextparameter. Skapa även en \code{given}-deklaration som erbjuder det givna heltalsvärdet noll:
\begin{REPLnonum}
scala> def f(using i: Int) = i + 1

scala> def g(x: Int)(using y: Int) = x + y
\end{REPLnonum}

\Subtask Anropa funktionerna \code{f} och \code{g} med ett explicit givet argument som skiljer sig från det givna heltalsvärdet med hjälp av \code{using} i anropet. Vad händer om du utelämnar \code{using}?

\Subtask Anropa funktionerna \code{f} och \code{g} utan att ange \code{using}-argument. Förklara vad som händer. 

\Subtask Går det att blanda vanliga parametrar och kontextparametrar i samma parameterlista? Om inte vad händer?

\SOLUTION

\TaskSolved \what

\SubtaskSolved
\begin{REPLnonum}
scala> given Int = 0
val res0: Int = 83

scala> f(using 41)
val res1: Int = 42

scala> g(41)(using 42)
val res2: Int = 83
\end{REPLnonum}
\noindent 
Om man glömmer \code{using} vid explicit kontextargument blir det kompileringsfel. Kompilatorn blir ''förvirrad'' och tror att du försöker ge ett ''vanligt'' argument till en (i detta fallet) icke-existerande ''vanlig'' parameterlista.  
\begin{REPLnonum}
scala> f(41)
-- [E050] Type Error: ----------------------------------------------
1 |f(41)
  |^
  |method f does not take more parameters
  |
  | longer explanation available when compiling with `-explain`
1 error found

scala> :setting -explain

scala> f(41)
-- [E050] Type Error: ----------------------------------------------
1 |f(41)
  |^
  |method f does not take more parameters
  |-----------------------------------------------------------------
    | Explanation (enabled by `-explain`)
  |- - - - - - - - - - - - - - - - - - - - - - - - - - - - - - - - -
  | You have specified more parameter lists than defined in the 
    method definition(s).
   -----------------------------------------------------------------

scala> g(41)(42)
-- [E050] Type Error: ----------------------------------------------
1 |g(41)(42)
  |^^^^^
  |method g does not take more parameters
  |-----------------------------------------------------------------
    | Explanation (enabled by `-explain`)
  |- - - - - - - - - - - - - - - - - - - - - - - - - - - - - - - - -
  | You have specified more parameter lists than defined in the 
    method definition(s).
   -----------------------------------------------------------------
\end{REPLnonum}
Det är inte vanligt att ange \code{using}-parametrar explicit; det vanligaste är att låta kompilatorn framkalla ett givet värde.

\SubtaskSolved Det givna värdet \code{0} binds till motsvarande kontextparameter, som ska vara deklarerad i en egen parameterlista som börjar med \code{using}.
\begin{REPLnonum}
scala> f
val res3: Int = 1

scala> g(42)
val res4: Int = 42
\end{REPLnonum}

\SubtaskSolved Nej, det blir kompileringsfel om man försöker blanda vanliga parametrar och kontextparametrar i en och samma parameterlista:
\begin{REPLnonum}
scala> def h(i: Int, using j: Int) = i + j
-- [E040] Syntax Error: ---------------------------------------------
1 |def h(i: Int, using j: Int) = i + j
  |                    ^
  |                    ':' expected, but identifier found
\end{REPLnonum}
Det är ett medvetet val att kräva separata parameterlistor, så att det inte ska uppstå förvirring om huruvida en vanlig parameter eller kontextparameter avses. 

\QUESTEND



\WHAT{Flera olika givna värden i lokal kontext.}

\QUESTBEGIN

\Task \what~Olika värden beroende på kontext.
\begin{Code}
case class Delta(value: Int)
object Delta:
  given default: Delta = Delta(1)

def inc(x: Int)(using dx: Delta) = x + dx.value

object Context1:
  val a = inc(1)

object Context2:
  given Delta = Delta(42)
  val a = inc(1)

\end{Code}

\Subtask Vilket värde har \code{Context1.a}? 

\Subtask Vilket värde har \code{Context2.a}? 

\Subtask Förklara vad som händer.

\SOLUTION

\TaskSolved \what

\SubtaskSolved 2

\SubtaskSolved 43

\SubtaskSolved När kompilatorn försöker framkalla ett givet värde att automatiskt använda som argument till \code{using}-parametern \code{dx}, så letar den i den kontext som är närmast anropet först. Om det finns ett givet värdet i kompanjonsobjektet för parametertypen så tar kompilatorn detta i sista hand, om inget annat givet värde hittas närmare anropet.

\QUESTEND




\WHAT{Lösning på konfigurationsproblemet med hjälp av givna värden.}

\QUESTBEGIN

\Task  \what~ Antag att vi vill kunna konfigurera beteendet hos en funktion för att göra den mer flexibel. Nedan visas tre principiellt olika sätt att göra detta på för en funktion \code{greet} som skriver ut en hälsning: 1) en globalt åtkomlig variabel, 2) defaultargument, samt 3) kontextuell abstraktion med \code{given} och \code{using}.

\begin{Code}
object GlobalVar:
  case class GreetConfig(greeting: String, receiver: String)
  object GreetConfig:
    val default = GreetConfig(greeting = "Hello", receiver = "World")
    var config = default
  
  def greetMsg = 
    s"${GreetConfig.config.greeting} ${GreetConfig.config.receiver}!"

object DefaultArgs:
  case class GreetConfig(greeting: String, receiver: String)
  object GreetConfig:
    val default = GreetConfig(greeting = "Hello", receiver = "World")
  
  def greetMsg(config: GreetConfig = GreetConfig.default) =
    s"${config.greeting} ${config.receiver}!"

object GivenVal:
  case class GreetConfig(greeting: String, receiver: String)
  object GreetConfig:
    given default: GreetConfig = GreetConfig("Hello", "World")
  
  def greetMsg(using g: GreetConfig) = s"${g.greeting} ${g.receiver}"
\end{Code}

\Subtask Skriv kod som testar de olika varianterna ovan. Visa speciellt hur du kan använda default-konfigurationen och därefter ge en konfiguration som skiljer sig från \code{default}. 

\Subtask Vad är för- och nackdelar med de olika varianterna ovan? Diskutera speciellt vilken/vilka lösningar som medger flera lokala konfigurationer utan att de påverkar varandra.

\Subtask Förklara vad som händer vid anrop av \code{summon[GivenVal.GreetConfig]}. 

\Subtask Vad händer om du försöker framkalla ett givet värde för en typ som inte har något sådant?

\Subtask Måste det givna värdet vara unikt?

\SOLUTION


\TaskSolved \what\\

\noindent Nedan visas test av de tre olika lösningarna som givits i uppg. \SubtaskSolved

\noindent Efter varje test diskuteras tillhörande för- och nackdelar, som efterfrågas i uppg. \SubtaskSolved

\begin{Code}
def testGlobalVar(useDefault: Boolean = true) = 
  import GlobalVar.*
  if useDefault then println(greetMsg) else 
    GreetConfig.config = GreetConfig("Godmorgon", "världen")
    println(greetMsg)
\end{Code}
Eftersom \code{config} här är en förändringsbar variabel, så kan en ändring på ett ställe påverka helt andra delar av programmet, vilket ibland kan vara en fördel, men ofta en nackdelen eftersom det kan vara svårt att förstå vad som händer bara genom att läsa en enskild del av programmet -- en förändring av \code{config} kan ju ske varsomhelst. Det är lätt att glömma ändra till baka till default-värdet, om det är det som förväntas.
\begin{REPL}
scala> testGlobalVar(); testGlobalVar(false); testGlobalVar()
Hello World!
Godmorgon världen!
Godmorgon världen!
\end{REPL}

\begin{Code}[numbers=left]
def testDefaultArgs(useDefault: Boolean = true) =
  import DefaultArgs.*
  if useDefault then println(greetMsg()) else 
    println(greetMsg(GreetConfig("Godmorgon","världen")))
\end{Code}
Här sker ingen tillståndsförändring och default-användning är enkel, men det går inte enkelt att göra avsteg från default som  gäller i en lokal kontext; vid \emph{varje} enskilt anrop behöver du explicit ange alla de argument som inte ska vara default, så som visas på rad 4 ovan. Ändring av default har bara lokal påverkan. Om alla argument ska följa default, så gäller det att inte glömma anropa med tomt parentespar: \code{greetMsg()}. (Vad händer annars?)
\begin{REPL}
scala> testDefaultArgs(); testDefaultArgs(false); testDefaultArgs()
Hello World!
Godmorgon världen!
Hello World!
\end{REPL}
Med kontextparametrar är flexibiliteten större; \code{using}-parametrar låter användaren själv styra vad som gäller i olika sammanhang och själva anropet blir enkelt oavsett om det är default-värdet eller andra, i den lokala kontexten, givna värden som önskas. Ändring av default har bara lokal påverkan, men den har påverkan på godtyckligt många anrop i den lokala kontexten --  argument som skiljer sig kan alltså vara givna en gång utan att behöva upprepas vid varje anrop. Vid anrop där man vill låta kompilatorn framkallar givna värden för kontextparametern ska inga parenteser användas, och anropen bli därmed korta och enkla.
\begin{Code}
def testGivenVal(using g: GivenVal.GreetConfig) = println(g.greetMsg)
\end{Code}

\begin{REPL}
scala> testGivenVal
Hello World

scala> def localContext =
         import GivenVal.*
         given GreetConfig = GreetConfig("Godmorgon","världen")
         testGivenVal

scala> localContext
Godmorgon världen

scala> testGivenVal
Hello World
\end{REPL}


\SubtaskSolved  Kompilatorn framkallar ett givet värde i den lokala kontexten:
\begin{REPL}
scala> summon[GivenVal.GreetConfig]
val res0: GivenVal.GreetConfig = GreetConfig(Hello,World)
\end{REPL}
Kompilatorn följer denna prioritetsordning i sökandet efter ett unikt givet värde:
\begin{enumerate}[nolistsep,noitemsep]
\item \textbf{Explicita} argument till kontextparametrar märkta med \code{using}
\item \code{given} och \code{import given ...} i aktuell namnrymd \Eng{current scope} 
\item \code{given}-värden i \textbf{kompanjonsobjekt} för den använda typen.
\end{enumerate}
Om flera givna värden kan framkallas för typer som ingår i en gemensam arvshierarki så väljer kompilatorn det givna värdet som är av den \emph{mest specifika} typen.\\

\SubtaskSolved Det blir kompileringsfel om kompilatorn inte hittar ett givet värde för den typ som avses.

\begin{REPL}
scala> summon[Long]
-- Error: ----------------------------------------------------------------
1 |summon[Long]
  |            ^
  |            no given instance of type Long was found for parameter x of 
               method summon in object Predef
1 error found
\end{REPL}

\SubtaskSolved Ja! Det får \emph{inte} vara tvetydigt vilket givet värde som ska framkallas:
\begin{REPL}
scala> def tvetydigt =
     |   given a: Int = 42
     |   given b: Int = 43
     |   summon[Int]
     | 
-- Error: ------------------------------------------------------------------
4 |  summon[Int]
  |             ^
  |ambiguous given instances: both given instance b and given instance a 
  |match type Int of parameter x of method summon in object Predef
1 error found

\end{REPL}
Läs mer om kontexuella abstraktioner här:\\\url{https://docs.scala-lang.org/scala3/reference/contextual/}


\QUESTEND


\ExtraTasks


\WHAT{Kontextparameter och givet värde.}

\QUESTBEGIN

\Task  \what~Prova nedan i REPL.
\begin{REPL}
scala> def add(x: Int)(using y: Int) = x + y
scala> add(1)(using 2)
scala> add(1)
scala> given ngtNamn = 42
scala> add(1)
\end{REPL}
\Subtask Vad blir felmeddelandet på rad 3 ovan? 

\Subtask Varför fungerar det på rad 5 utan fel?

\Subtask Definiera och testa en motsvarande funktion \code{sub} som kan subtrahera ett givet värde.

\SOLUTION


\TaskSolved \what

\SubtaskSolved 
\begin{REPL}
scala> add(1)
-- Error: ------------------------------------------------------------------
1 |add(1)
  |      ^
  |    no given instance of type Int was found for parameter y of method add
1 error found
\end{REPL}

\SubtaskSolved Nu finns ett givet värde som kompilatorn automatiskt kan fylla i på platsen vid anropet.

\SubtaskSolved \code{def sub(x: Int)(using y: Int) = x - y}

\QUESTEND

\AdvancedTasks %%%%%%%%%%%


\WHAT{Varians och typgränser.}

\QUESTBEGIN

\Task  \what~ Koden nedan är en modell av husdjur med följande innebörd: Husdjur kan vara friska eller sjuka och föds i normalfallet friska. Det kan finnas många katter och hundar, vilka alla är olika slags husdjur.

\begin{Code}
trait Pet(var isHealthy: Boolean = true)
class Cat extends Pet()
class Dog extends Pet()

\end{Code}

\Subtask Förändra koden nedan så att efterföljande REPL-sats \emph{inte} ger kompileringsfel?
\begin{Code}
case class Box[A](x: A)
\end{Code}
\begin{REPLnonum}
scala> val b: Box[Any] = Box[Cat](Cat())
\end{REPLnonum}

\Subtask Prova nedan i REPL och förklara vad som händer.      
\begin{REPLnonum}
scala> val v: Vector[Pet] = Vector[Cat](Cat())

scala> val s: Set[Pet] = Set[Cat](Cat())

scala> :settings -explain

scala> val s: Set[Pet] = Set[Cat](Cat())
\end{REPLnonum} 
\emph{Ledtråd:} I Scalas standardbibliotek så är ärver \code{Set[T]} funktionstypen \code{T => Boolean} som är deklarerad kontravariant i sin inparameter.

\Subtask Det ska finnas veterinärer som kan behandla husdjur och göra dem friska. Varför fungerar inte nedan kod? Är det ett kompileringsfel eller körtidsfel?

\begin{Code}
class Vet[-A]:
  def treat(x: A): Unit = x.isHealthy = true
\end{Code}

\Subtask Inför en typgräns i veterinärens typparametern som åtgärdar felet.

\Subtask Skriv valfri kod som visar 1) att kompilatorn tillåter kattveterinärer att behandla katter men 2) förhindrar att kattveterinärer får behandla godtyckliga husdjur och att 3) en veterinär som har komptens att behandla godtyckliga husdjur kan behandla både katter och hundar. Förklara varför kompilatorn tillåter/förhindrar detta.

\SOLUTION


\TaskSolved \what

\SubtaskSolved Gör lådan flexibel i sin typparameter med ett \code{+} före typparametern enligt nedan. 
\begin{Code}
case class Box[+A](x: A)
\end{Code}
Kompilatorn tillämpar reglerna för kovarians eftersom typparametern har ett plustecken framför sig: \code{Box[Cat]} är en suptyp till \code{Box[Any]} om \code{Cat} är en subtyp till \code{Any}, vilket den ju är eftersom alla typer är subtyp till \code{Any}. 

\SubtaskSolved Förklaringen till beteendet har med olika varians att göra:
\begin{itemize}
  \item Samlingen \code{Vector} är kovariant och därmed flexibel i sin typparameter (liksom andra oföränderliga sekvenser i Scalas standardbibliotek). Kompilatorn betraktar därmed \code{Vector[Cat]} som en subtyp till \code{Vector[Pet]} eftersom \code{Cat} är en subtyp till \code{Pet}. På platser i koden där en \code{Vector[Pet]} krävs så anses \code{Vector[Cat]} överensstämma med \Eng{conforms to} \code{Vector[Pet]} och får därmed duga på dessa platser.
  \item En mängd har en apply-metod från elemttypen till \code{Boolean} som ger innehållstest. Av det skälet har man låtit \code{Set[T]} ärva \code{Function1[T, Boolean]} som är deklarerad kontravariant i \code{T}, så att en mängd kan användas där en \code{T => Boolean} förväntas. Även om det skulle vara praktiskt om Set[T] vore kovariant i \code{T}, i likhet med \code{Vector}, \code{List}, \code{Seq} etc, så kan inte \code{T} vara både kovariant och kontravariant på en och samma gång. Man har därför valt att göra \code{Set} invariant och därmed är mängder ej flexibla i sin typparameter. \code{Set[Cat]} är alltså \emph{inte} en subtyp till \code{Set[Pet]} \emph{även} om \code{Cat} är en subtyp till \code{Pet}, vilket ger kompileringsfel i uppgiftens exempel. 
  Se även \url{https://stackoverflow.com/questions/676615/why-is-scalas-immutable-set-not-covariant-in-its-type}  
  \item Med \code{:settings -explain} ger kompilatorn en längre utskrift som förklarar den bevisföring som skedde under kompileringens typkontroll.
\end{itemize}



\SubtaskSolved Det blir kompileringsfel då metoden \code{isHealthy} ej existerar för godtycklig typ.

\SubtaskSolved Lägg till en övre gräns som garanterar att metoden \code{isHealthy} finns för alla typer som kan bindas till typparametern \code{A}:
\begin{Code}
class Vet[-A <: Pet]:
  def treat(x: A): Unit = x.isHealthy = true
\end{Code}
Kompilatorn garanterar alltså att typparametern \code{A} är ''mindre än eller lika med'' \code{Pet}.

\SubtaskSolved Veterinären \code{Vet} är flexibel i sin typparameter och minustecknet anger kontravarians och därmed att \code{Vet[Pet]} är en subtyp till \code{Vet[Cat]} då \code{Cat} är en subtyp till \code{Pet}. Detta kan demonstreras med nedan exempel:
\begin{REPL}
scala> val pinkPanther = Cat()
val pinkPanther: Cat = Cat@33e7ece5

scala> val somePet: Pet = Cat()
val somePet: Pet = Cat@57f1cb96

scala> val catVet = Vet[Cat]()
val catVet: Vet[Cat] = Vet@1060e784

scala> pinkPanther.isHealthy = false

scala> catVet.treat(pinkPanther)

scala> pinkPanther.isHealthy
val res2: Boolean = true

scala> somePet.isHealthy = false

scala> catVet.treat(somePet)
-- [E007] Type Mismatch Error: --------------
1 |catVet.treat(somePet)
  |             ^^^^^^^
  |             Found:    (somePet : Pet)
  |             Required: Cat

scala> val powerVet = Vet[Pet]()
val powerVet: Vet[Pet] = Vet@2eb90ae9

scala> pinkPanther.isHealthy = false

scala> powerVet.treat(pinkPanther)

scala> pinkPanther.isHealthy
val res3: Boolean = true

scala> val pluto = Dog()
val pluto: Dog = Dog@6f27db5d

scala> pluto.isHealthy = false

scala> powerVet.treat(pluto)

scala> pluto.isHealthy
val res4: Boolean = true

\end{REPL}


\QUESTEND




\WHAT{Typklasser och kontextparametrar.}

\QUESTBEGIN

\Task  \what~  I Scala finns möjligheter till avancerad funktionsprogrammering med s.k. \textbf{typklasser} (ä.k. \emph{ad hoc polymorfism}). En typklass definierar generella beteenden som fungerar för godtyckliga befintliga typer utan att implementationen av dessa behöver ändras. Vi nosar i denna uppgift på hur kontextuella abstraktioner kan användas för att skapa typklasser i Scala, illustrerat med hjälp av givna ordningarna vid sortering.

Genom att kombinera koncepten givna värden, generiska klasser och kontextparametrar får man möjligheten till ad hoc polymorfism, exemplifierat med typklassen \code{CanCompare} nedan, som vi kan få att fungera för befintliga typer \emph{utan} att de behöver ändras. Speciellt så har vi ju inte möjligheten att lägga till metoder på befintliga typer i standardbiblioteket, eftersom det inte är våran egen kod.


\Subtask 
Vad händer nedan? Vilka rader ger felmeddelande? Varför?

\begin{REPL}
scala> trait CanCompare[T]:
         def compare(a: T, b: T): Int

scala> def sort[T](a: T, b: T)(using cc: CanCompare[T]): (T, T) =
         if cc.compare(a, b) > 0 then (b, a) else (a, b)

scala> sort(42, 41)

scala> given intComparator: CanCompare[Int] with
         override def compare(a: Int, b: Int): Int = a - b

scala> sort(42, 41)

scala> sort(42.0, 41.0)
\end{REPL}

\Subtask Definiera ett givet värde som gör så att \code{sort} fungerar för värden av typen \code{Double}.

\Subtask Definiera ett givet värde som gör så att \code{sort} fungerar för värden av typen \code{String}. \emph{Tips:} Du har nytta av de befintliga jämförelseoperatorerna på strängar, men tänk på att \code{compare} fortfarande måste returnera ett heltal även vid jämförelse av strängar.


\SOLUTION


\TaskSolved \what

\SubtaskSolved 

\begin{itemize}
  \item Först deklarerar vi en \code{trait}, \code{CanCompare}, med en generisk typparameter \code{T}. Den innehåller en abstrakt metod \code{compare} som tar två parametrar av typen \code{T} och returnerar en \code{Int}.
  \item Sedan definieras en metod \code{sort} som också tar en generisk typparameter \code{T}. Metoden tar två parametrar, a och b av typen T, samt en \code{using} parameter cc som måste vara en instans av \code{CanCompare[T]}. Inuti metoden används compare-metoden från CanCompare för att bestämma om a och b ska byta plats eller inte. 
  \item När vi försöker köra \code{sort(42, 41)} så får vi felmeddelande av kompilatorn. Anledning till detta är att det inte finns en given instans av CanCompare[Int].
  \item Vi löser detta på nästa rad med \code{given intComparator} som är av typen \linebreak CanCompare[Int]. Vi definierar även vår abstrakta metod \code{compare} från CanCompare med \code{override def compare}... När vi kör \code{sort(42,41)} på nästa rad fungerar det nu som det ska och vi får tillbaka \code{(Int, Int) = (41, 42)}
  \item När vi försöker köra sort med argument av typen \code{Double} får vi ett liknande felmeddelande som vi fick tidigare, och av samma anledning att det inte finns en CanCompare för typen Double.
\end{itemize}

\SubtaskSolved 
\begin{REPL}
scala> given doubleComparator: CanCompare[Double] with
         override def compare(a: Double, b: Double): Int = (a - b).toInt
\end{REPL}

\SubtaskSolved 
\begin{REPL}
scala> given stringComparator: CanCompare[String] with
         override def compare(a: String, b: String): Int = a.compareTo(b)
\end{REPL}

\QUESTEND





\WHAT{Användning av given ordning.}

\QUESTBEGIN

\Task \label{task:implicit-ordering} \what~  Vi ska nu skapa en funktion \code{isSorted} som är generellt användbar genom att göra givna ordningsfunktioner tillgängliga för olika typer. Funktionen  \code{def isSorted(xs: Vector[Int]): Boolean = ???} fungerar bara för samlingar av typen \code{Vector[Int]}.

Om vi i stället använder
\code{def isSorted(xs: Seq[Int]): Boolean = ???} fungerar den för olika samlingar med heltal, även \code{Vector} och \code{List}. 

\Subtask  Testa nedan funktion i REPL med heltalssekvenser av olika typ.
\begin{Code}
def isSorted(xs: Seq[Int]): Boolean = xs == xs.sorted
\end{Code}

\Subtask Det blir problem med nedan försök att göra \code{isSorted} generisk. Hur lyder felmeddelandet? Vad saknas enligt felmeddelandet?
\begin{REPLnonum}
scala> def isSorted[T](xs: Seq[T]): Boolean = xs == xs.sorted
\end{REPLnonum}

\Subtask Vi vill gärna att \code{isSorted} ska fungera för godtyckliga typer T som har en ordningsdefinition. Detta kan göras med nedan funktion där den speciella typparametern \code{[T:Ordering]} betyder att \code{isSorted} är definierad för alla samlingar där typen \code{T} har en given ordning \code{Ordering[T]}. Speciellt gäller detta för alla grundtyperna \code{Int}, \code{Double}, \code{String}, etc., som alla har specifika implementationer av typklassen \code{Ordering}.
\begin{Code}
def isSorted[T:Ordering](xs: Seq[T]): Boolean = xs == xs.sorted
\end{Code}
Testa metoden ovan i REPL enligt nedan.
\begin{REPL}
scala> isSorted(Vector(1,2,3))
scala> isSorted(List(1,2,3,1))
scala> isSorted(Vector("A","B","C"))
scala> isSorted(List("A","B","C","A"))
scala> case class Person(firstName: String, familyName: String)
scala> val persons = Vector(Person("Kim", "Finkodare"), Person("Voldemort","Fulhackare"))
scala> isSorted(persons)
\end{REPL}
Vad ger sista raden för felmeddelande? Varför?


\Subtask \emph{Implicita ordningar.} En typparameter på formen \code{[T:Ordering]} kallas kontextgräns \Eng{context bound} och föranleder kompilatorn att automatiskt expandera funktionshuvudet för \code{isSorted} med en kontextparameter. I stället för att använda \code{[T:Ordering]} kan vi själva lägga till en kontextparameter som motsvarar kontextgränsen. Då får vi också tillgång till ett namn, här nedan \code{ord}, på den implicita ordningen och kan använda det namnet i funktionskroppen och anropa metoder som är medlemmar av typklassen \code{Ordering}. (Namnet på kontextparametern kan också utelämnas, men då får vi istället gå omvägen via inbyggda funktionen \code{summon[T]} för att be kompilatorn leta upp den givna instansen för den typparameter som ges vid anropet.)

\begin{CodeSmall}
def isSorted[T](xs: Seq[T])(using ord: Ordering[T]): Boolean =
  xs.zip(xs.tail).forall(x => ord.lteq(x._1, x._2))
\end{CodeSmall}

Objekt av typen \code{Ordering} har jämförelsemetoder som t.ex. \code{lteq} (förk. \emph{less than or equal}) och \code{gt} (förk. \emph{greater than}).

Det finns givna ordningar för alla grundtyper i standardbiblioteket, alltså t.ex. \code{Ordering[Int]}, \code{Ordering[String]}, etc.
Testa så att kompilatorn hittar ordningen för samlingar med värden av några grundtyper. Kontrollera även enligt nedan att det fortfarande blir problem för egendefinierade klasser, t.ex. \code{Person}  (detta ska vi råda bot på i uppgift \ref{task:custom-ordering}).
\begin{REPL}
scala> isSorted(Vector(1,2,3))
scala> isSorted(Array(1,2,3,1))
scala> isSorted(Vector("A","B","C"))
scala> isSorted(List("A","B","C","A"))
scala> class Person(firstName: String, familyName: String)
scala> val persons = Vector(Person("Kim", "Finkodare"), Person("Robin","Fulhack"))
scala> isSorted(persons)
\end{REPL}

\Subtask \emph{Importera implicita ordningsoperatorer från en \code{Ordering}.} Om man gör import på ett \code{Ordering}-objekt får man tillgång till implicita konverteringar som gör att jämförelseoperatorerna fungerar. Testa nedan variant av \code{isSorted} på olika sekvenstyper och verifiera att \code{<=}, \code{>}, etc., nu fungerar enligt nedan.
\begin{CodeSmall}
def isSorted[T](xs: Seq[T])(given ord: Ordering[T]): Boolean = {
  import ord._
  xs.zip(xs.tail).forall(x => x._1 <= x._2)
}
\end{CodeSmall}


\SOLUTION

\TaskSolved \what

\SubtaskSolved 

\SubtaskSolved 
Exempel på tester:
\begin{REPL}
scala> isSorted(Vector(1,2,3))
val res0: Boolean = true

scala> isSorted(Vector(1,2,4,3))
val res1: Boolean = false

scala> isSorted(List(1,2,3))
val res2: Boolean = true

scala> isSorted(List(1,2,4,3))
val res3: Boolean = false
\end{REPL}

\SubtaskSolved 
\begin{REPL}
scala> given stringComparator: CanCompare[String] with
         override def compare(a: String, b: String): Int = a.compareTo(b)
\end{REPL}

\QUESTEND






\WHAT{Skapa egen implicit ordning med \code{Ordering}.}

\QUESTBEGIN

\Task \label{task:custom-ordering} \what~

\Subtask Ett sätt att skapa en egen, specialanpassad ordning för dina egna klasser är att mappa dina objekt till typer som redan har en implicit ordning. Med hjälp av metoden \code{by} i objektet \code{scala.math.Ordering} kan man skapa ordningar genom att bifoga en funktion \code{T => S} där \code{T} är typen för de objekt du vill ordna och \code{S} är någon annan typ, t.ex. \code{String} eller \code{Int}, där det redan finns en given ordning.
\begin{REPL}
scala> case class Team(name: String, rank: Int)
scala> val xs =
         Vector(Team("fnatic", 1499), Team("nip", 1473), Team("lumi", 1601))
scala> xs.sorted  // Hur lyder felmeddelandet? Varför blir det fel?
scala> val teamNameOrdering: Ordering[Team] = Ordering.by(t => t.name)
scala> xs.sorted(using teamNameOrdering)   //explicit ordning
scala> given Ordering[Team] = Ordering.by(t => t.rank)
scala> xs.sorted   // Varför funkar det nu?
\end{REPL}

\Subtask Vill man sortera i omvänd ordning kan man använda
\code{Ordering.fromLessThan} som tar en funktion \code{(T, T) => Boolean} vilken ska ge \code{true} om första parametern ska komma före, annars \code{false}. Om vi vill sortera efter \code{rank} i omvänd ordning kan vi göra så här:
\begin{REPL}
scala> val highestRankFirst: Ordering[Team] =
         Ordering.fromLessThan((t1, t2) => t1.rank > t2.rank)
scala> xs.sorted(using highestRankFirst)
\end{REPL}

\Subtask Om du har en case-klass med flera fält och vill ha en fördefinierad implicit sorteringsordning samt \emph{även} erbjuda en alternativ sorteringsordning, så kan du placera en default ordningsdefinition i ett kompanjonsobjekt; detta är nämligen ett av de ställen där kompilatorn söker sist efter eventuella implicita värden innan den ger upp att leta.
\begin{Code}
case class Team(name: String, rank: Int)
object Team:
  given highestRankFirst: Ordering[Team] = 
    Ordering.fromLessThan((t1, t2) => t1.rank > t2.rank)
  val nameOrdering: Ordering[Team] = Ordering.by(t => t.name)
\end{Code}
\begin{REPL}
scala> val xs =
         Vector(Team("fnatic", 1499), Team("nip", 1473), Team("lumi", 1601))
scala> xs.sorted
scala> xs.sorted(Team.nameOrdering)
\end{REPL}



\Subtask Det går också med kompanjonsobjektet ovan att få jämförelseoperatorer att fungera med din case-klass, genom att importera medlemmarna i lämpligt ordningsobjekt. Verifiera att så är fallet enligt nedan:
\begin{REPL}
scala> Team("fnatic",1499) < Team("gurka", 2)  // Vilket fel? Varför?
scala> import Team.highestRankFirst.given
scala> Team("fnatic",1499) < Team("gurka", 2)  // Inget fel? Varför?
\end{REPL}


\SOLUTION


\TaskSolved \what 

\SubtaskSolved \TODO

\SubtaskSolved \TODO

\SubtaskSolved \TODO

\SubtaskSolved \TODO


\QUESTEND






\WHAT{Specialanpassad ordning genom att ärva från \code{Ordered}}

\QUESTBEGIN

\Task  \what~  Om det finns \emph{en} väldefinierad, specifik ordning som man vill ska gälla för sina case-klass-instanser kan man göra den ordnad genom att låta case-klassen mixa in traiten \code{Ordered} och implementera den abstrakta metoden \code{compare}. (Detta illustrerar användning av subtypspolymorfism (d.v.s arv) i stället för ad hoc polymorfism med typklasser.)

\begin{Background}
En trait som används på detta sätt kallas \textbf{gränssnitt} \Eng{interface}, och om man \emph{implementerar} ett gränssnitt så uppfyller man ett ''kontrakt'', som i detta fall innebär att man implementerar det som krävs av ordnade objekt, nämligen att de har en konkret \code{compare}-metod. Du lär dig mer om gränssnitt i kommande kurser.
\end{Background}

\Subtask Implementera case-klassen \code{Team} så att den är en subtyp till \code{Ordered} enligt nedan skiss. Högre rankade lag ska komma före lägre rankade lag. Metoden \code{compare} ska ge ett heltal som är negativt om \code{this} kommer före \code{that}, noll om de ordnas lika, annars positivt.

\begin{Code}
case class Team(name: String, rank: Int) extends Ordered[Team]:
  override def compare(that: Team): Int = ???
\end{Code}
\emph{Tips:} Du kan anropa metoden \code{compare} på alla grundtyper, t.ex. \code{Int}, eftersom de implementerar gränssnittet \code{Oredered}. Genom att negera uttrycket blir ordningen den omvända. 

\begin{REPLnonum}
scala> -(2.compare(1))
\end{REPLnonum}

\Subtask Testa att  din case-klass nu uppfyller det som krävs för att vara ordnad.
\begin{REPLnonum}
scala> Team("fnatic",1499) < Team("gurka", 2)
\end{REPLnonum}


\Subtask Diskutera med handledare eller kursare skillnader och likheter mellan gränssnitt och typklasser, med ledning av denna och föregående uppgifter.
\SOLUTION


\TaskSolved \what

\SubtaskSolved

\begin{Code}
case class  Team(name: String, rank: Int) extends Ordered[Team]:
  override def compare(that: Team): Int = -rank.compare(that.rank)
\end{Code}

\SubtaskSolved

\begin{REPLnonum}
scala> Team("fnatic",1499) < Team("gurka", 2)
val res1: Boolean = true
\end{REPLnonum}

\SubtaskSolved Ad hoc polymorfism är mer flexibel. \TODO{mer diskussion om likheter och skillnader här...}

\QUESTEND



% \WHAT{Sortering med inbyggda sorteringsmetoder.}

% \QUESTBEGIN

% \Task  \what~  För grundtyperna (\code{Int}, \code{Double}, \code{String}, etc.) finns en fördefinierad ordning som gör så att färdiga sorteringsmetoder fungerar på alla samlingar i \code{scala.collection}. Även jämförelseoperatorerna i uppgift \ref{task:string-order-operators} fungerar enligt den fördefinierade ordningsdefinitionen för alla grundtyper. Denna ordningsdefinition är \textit{implicit tillgänglig} vilket betyder att kompilatorn hittar ordningsdefinitionen utan att vi explicit måste ange den. Detta fungerar i Scala även med primitiva \code{Array}.

% \Subtask Testa metoden \code{sorted} på några olika samlingar. Förklara vad som händer. Hur lyder felmeddelandet på sista raden? Varför blir det fel?

% \begin{REPL}
% scala> Vector(1.1, 4.2, 2.4, 42.0, 9.9).sorted
% scala> val xs = (100000 to 1 by -1).toArray
% scala> xs.sorted
% scala> xs.map(_.toString).sorted
% scala> xs.map(_.toByte).sorted.distinct
% scala> case class Person(firstName: String, familyName: String)
% scala> val ps = Vector(Person("Robin", "Finkodare"), Person("Kim","Fulhack"))
% scala> ps.sorted
% \end{REPL}

% \Subtask Om man har en samling med egendefinierade klasser eller man vill ha en annan sorteringsordning får man definiera ordningen själv. Ett helt generellt sätt att göra detta på  illustreras i uppgift \ref{task:custom-ordering}, men de båda hjälpmetoderna \code{sortWith} och \code{sortBy} räcker i de flesta fall. Hur de används illustreras nedan. Metoden \code{sortBy} kan användas om man tar fram ett värde av grundtyp och är nöjd med den inbyggda sorteringsordningen.

% Metoden \code{sortWith} används om man vill skicka med ett eget jämförelsepredikat som ordnar två element; funktionen ska returnera \code{true} om det första elementet ska vara först, annars \code{false}.

% \begin{REPL}
% scala> case class Person(firstName: String, familyName: String)
% scala> val ps = Vector(Person("Robin", "Finkodare"), Person("Kim","Fulhack"))
% scala> ps.sortBy(_.firstName)
% scala> ps.sortBy(_.familyName)
% scala> ps.sortBy  // tryck TAB två gånger för att se signaturen
% scala> ps.sortWith((p1, p2) => p1.firstName > p2.firstName)
% scala> ps.sortWith  // tryck TAB två gånger för att se signaturen
% scala> Vector(9,5,2,6,9).sortWith((x1, x2) => x1 % 2 > x2 % 2)
% \end{REPL}
% Vad har metoderna \code{sortWith} och \code{sortBy} för signaturer?

% \Subtask Lägg till attributet \code{age: Int} i case-klassen \code{Person} ovan och lägg till fler personer med olika namn och ålder i en vektor och sortera den med \code{sortBy} och \code{sortWith} för olika attribut. Välj själv några olika sätt att sortera på.



% \SOLUTION


% \TaskSolved \what


% \SubtaskSolved
% \begin{enumerate}
% \item Returnerar en sorterad \code{Vector} av \code{double}-värden
% \item Skapar en variabel xs och sparar en \code{Array} med \code{Int}-värden mellan 100000 till 1.
% \item Sorterar \code{xs = 1,2,3...}
% \item Konverterar xs till en \code{Array} av \code{String}-värden och sorterar dem lexikografiskt: \code{xs = "1", "10", "100"} etc.
% \item Konverterar xs till en \code{Array} av \code{Byte}-värden (max 127, min -128) och sorterar dem, samt tar bort dubletter: \code{xs = -128, -127, -1...}
% \item Skapar en ny klass \code{Person} som tar 2 \code{String}-argument i konstruktorn
% \item Sparar en Vector med två \code{Person}-objekt i en variabel ps
% \item Försöker kalla på \code{sorted}-metoden för klassen \code{Person}. Eftersom vi skrivit denna klass själva och inte berättat för Scala hur \code{Person}-objekt ska sorteras, resulterar detta i ett felmeddelande.
% \end{enumerate}

% \SubtaskSolved

% \begin{enumerate}
% \item ---
% \item ---
% \item Sorterar \code{Person}-objekten i ps med avseende på värdet i \code{firstName}
% \item Sorterar \code{Person}-objekten i ps med avseende på värdet i \code{familyName}
% \item \code{sortBy} tar en funktion f som argument. f ska ta ett argument av typen \code{Person} och returnera en generisk typ B.
% \item Sortera \code{Person}-objekten i ps med avseende på \code{firstName} i sjunkande ordning (omvänt från tidigare alltså)
% \item \code{sortWith} tar en funktion lt som argument. lt ska i sin tur ta två argument av typen \code{Person} och returnera ett booleskt värde.
% \item Sorterar en vektor så att värdena som är minst delbara med 2 hamnar först, och de mest delbara med 2 hamnar sist. Detta delar alltså upp udda och jämna tal.
% \end{enumerate}

% \SubtaskSolved
% Klassens signatur blir då:
% \begin{REPLnonum}
% case class Person(firstName: String, lastName: String, age: Int)
% \end{REPLnonum}

% Lägg in dem i en vektor:
% \begin{REPLnonum}
% val ps2 = Vector(Person("a", "asson", 34), Person("asson", "assonson", 1234),
% Person("anna", "Book", 2))
% \end{REPLnonum}

% Sortera dem på olika sätt:
% \begin{enumerate}
% \item
% Vektorn blir sorterad med avseende på personernas ålder i stigande ordning
% \begin{REPLnonum}
% scala> ps2.sortWith((p1, p2) => p1.age < p2.age)
% res40: scala.collection.immutable.Vector[Person] = Vector(Person(anna,Book,2),
% Person(a,asson,34), Person(asson,assonson,1234))
% \end{REPLnonum}

% \item
% Sorterar vektorn med avseende på namn, men också med avseende på ålder (i sjunkande ordning). För att komma före någon i ordningen måste alltså både namnet komma före, och åldern vara högre.
% \begin{REPLnonum}
% scala> ps2.sortWith((p1, p2) => (p1.firstName < p2.firstName) &&
% (p1.age > p2.age))
% res42: scala.collection.immutable.Vector[Person] = Vector(Person(a,asson,34),
% Person(asson,assonson,1234), Person(anna,Book,2))
% \end{REPLnonum}
% \end{enumerate}



% \QUESTEND





%!TEX encoding = UTF-8 Unicode
%!TEX root = ../exercises.tex

\ifPreSolution

\Exercise{\ExeWeekTWELVE}\label{exe:W12}

\begin{Goals}
\item Denna veckas övning innehåller valfri fördjupning.
\item Sökning och sortering: 
\begin{itemize}
\item Kunna använda inbyggda sökmetoder.
\item Förstå när binärsökning är lämplig och möjlig.
\item Kunna implementera binärsökning.
\item Kunna implementera urvalssortering, både till ny samling och på plats.
\end{itemize}
\item Trådar och jämlöpande exekvering:
\begin{itemize}
\item Känna till vad en tråd är och kunna förklara begreppet jämlöpande exekvering.
\item Känna till vad metoderna \code{run} och \code{start} gör i klassen \code{Thread}.
\item Kunna skapa och starta en tråd med överskuggad \code{run}-metod.
\item Kunna skapa ett enkelt program som från två trådar tävlar om att uppdatera en variabel och förklara varför beteendet kan bli oförutsägbart.
\item Kunna använda en \code{Future} för att köra igång flera parallella beräkningar.
\item Kunna registrera en callback på en \code{Future} med metoden \code{onComplete}.
%\item Känna till att webbsidor beskrivs av HTML-kod och kunna skapa en minimal webbsida.
%\item Kunna ladda ner en webbsida med \code{scala.io.Source.fromURL}.
\end{itemize}
\end{Goals}

% \begin{Preparations}
% \item \StudyTheory{12}
% \end{Preparations}

%\BasicTasksNoLab %%%%%%%%%%%%%%%%

\subsection{Uppgifter om sökning och sortering}

\else

\ExerciseSolution{\ExeWeekTWELVE}

\subsection{Uppgifter om sökning och sortering}
%\BasicTasksNoLab

\fi




\WHAT{Tidmätning.}

\QUESTBEGIN

\Task \label{task:timed} \what~  I kommande uppgifter kommer du att ha nytta av funktionen \code{timed} enligt nedan.
\begin{Code}
def timed[T](code: => T): (T, Long) = 
  val now = System.nanoTime
  val result = code
  val elapsed = System.nanoTime - now
  println("\ntime: " + (elapsed / 1e6) + " ms")
  (result, elapsed)
\end{Code}
\Subtask Klistra in \code{timed} i REPL och testa så att den fungerar, genom att mäta hur lång tid nedan uttryck tar att exekvera.
\begin{REPL}
scala> val (v, t1) = timed{ (1 to 1000000).toVector.reverse }
scala> val (s, t2) = timed{ v.toSet }
scala> timed{ v.find(_ == 1) }
scala> timed{ s.find(_ == 1) }
scala> timed{ s.contains(1) }
\end{REPL}
\Subtask\Pen Försök förklara skillnaderna i exekveringstid mellan de olika sätten att söka reda på  talet $1$ i samlingen. Ungefär hur många gånger behöver man använda \code{contains} på heltalsmängden \code{s} för att det ska löna sig att skapa \code{s} i stället för att linjärsöka i \code{v} med \code{find} i ovan exempel?


\SOLUTION


\TaskSolved \what


\SubtaskSolved
Exekvera koden och du bör finna att det tar längre tid att hitta värdet 1 i vårt Set s än i vektorn v.

\SubtaskSolved

En vektor har en sekventiell ordning som find kan använda, medan \code{Set} är internt ordnad  på ett annat sätt för att innehållskontroll ska gå extra snabbt. Anledningen att det tar tid för \code{find} på \code{Set} är att det först måste skapas en iterator innan vår mängd kan gås igenom från början till slut. Metoden \code{contains} på \code{Set} däremot är rasande snabb beroende på den interna strukturen hos objekt av typen \code{Set} (som är smart designad med s.k. hash-koder, där det går lika snabbt att hitta ett element oavsett vart det befinner sig).



\QUESTEND




\WHAT{Sökning med inbyggda sökmetoder.}

\QUESTBEGIN

\Task  \what~

\Subtask \emph{Linjärsökning framifrån med \code{indexOfSlice}}. Studera dokumentationen för Scalas samlingsmetod \code{indexOfSlice}\footnote{\href{http://docs.scala-lang.org/overviews/collections/seqs.html}{docs.scala-lang.org/overviews/collections/seqs.html}} och skriv 8 olika uttryck i REPL som, både med en sträng och med en vektor med heltal, provar 4 olika fall: (1) finns i början, (2) finns någonstans i mitten, (3) finns i slutet, samt (4) finns ej.

\Subtask \emph{Linjärsökning bakifrån med \code{lastIndexOfSlice}}. Studera dokumentationen för Scalas samlingsmetod \code{lastIndexOfSlice}\footnote{\href{http://docs.scala-lang.org/overviews/collections/seqs.html}{docs.scala-lang.org/overviews/collections/seqs.html}} och skriv 8 olika uttryck i REPL som, både med en sträng och med en vektor med heltal, provar 4 olika fall: (1) finns i början, (2) finns någonstans i mitten, (3) finns i slutet, samt (4) finns ej.

\Subtask \emph{Sökning med inbyggd binärsökning.} Om en samling är sorterad kan man utnyttja detta för att göra snabbare sökning. Vid \textbf{binärsökning} \Eng{binary search}\footnote{\label{footnote:binarysearch}\href{https://en.wikipedia.org/wiki/Binary_search_algorithm}{en.wikipedia.org/wiki/Binary\_search\_algorithm}} börjar man på mitten och kollar vilken halva att  söka vidare i; sedan delar man upp denna halva på mitten och kollar vilken fjärdedel att söka vidare i, etc.

I objektet \code{scala.collection.Searching}\footnote{\href{http://www.scala-lang.org/api/current/scala/collection/Searching\$.html}{http://www.scala-lang.org/api/current/scala/collection/Searching\$.html}} finns en metod \code{search} som, om den importeras, erbjuder binärsökning för alla sorterade sekvenssamlingar. Om samlingen är sorterad ger den ett objekt av case-klassen \code{Found} som innehåller indexet för platsen där elementet först hittats; alternativt om det som eftersöks ej finns, ges ett objekt av case-klassen \code{InsertionPoint} som innehåller indexet där elementet borde ha varit placerad om det funnits i samlingen. Observera att om samlingen inte är sorterad är resultatet ''odefinierat'', d.v.s. något returneras men det är \emph{inte} att lita på; man måste alltså först sortera samlingen eller vara helt säker på att den är sorterad.

Undersök hur \code{search} fungerar genom att förklara vad som händer nedan. Vilken är snabbast av \code{lin} och \code{bin} nedan? Använd \code{timed} från uppgift \ref{task:timed}.

\begin{REPL}
scala> val udda = (1 to 1000000 by 2).toVector
scala> import scala.collection.Searching._
scala> udda.search(udda.last)
scala> udda.search(udda.last + 1)
scala> udda.reverse.search(udda(0))
scala> def lin(x: Int, xs: Seq[Int]) = xs.indexOf(x)
scala> def bin(x: Int, xs: Seq[Int]) = xs.search(x) match 
         case Found(i) => i
         case InsertionPoint(i) => -i
scala> timed{ lin(udda.last, udda) }
scala> timed{ bin(udda.last, udda) }
\end{REPL}

\Subtask Om en samling innehåller $n$ element, hur många jämförelser behövs då vid binärsökning i värsta fall? \emph{Tips:} Läs om algoritmen på Wikipedia\textsuperscript{\ref{footnote:binarysearch}}.


\SOLUTION


\TaskSolved \what


\SubtaskSolved
Förslag på test av \code{indexOfSlice}:
\begin{REPLnonum}
scala> List(1,2,3,35,1,23).indexOfSlice(List(35,1,23))
res73: Int = 3
scala> List(1,2,3,35,1,23).indexOfSlice(List(35,1,3))
res74: Int = -1
\end{REPLnonum}

\SubtaskSolved
Förslag på test av \code{lastIndexOfSlice}:
\begin{REPLnonum}
Vector(1,2,3,4,1,2).lastIndexOfSlice(Vector(1,2))
res2: Int = 4
Vector("apa", "banan", "majs", "banan").lastIndexOfSlice(Vector("banan"))
res3: Int = 3
Vector("apa", "banan", "majs", "banan").lastIndexOfSlice(Vector("banand"))
res4: Int = -1
\end{REPLnonum}

\SubtaskSolved
Observera att metoden \code{search} antar att samlingen är sorterad i stigande ordning. När vi inverterar ordningen kan \code{search} oftast inte hitta det vi letar efter, eftersom den kommer leta i fel halva av samlingen.

\begin{REPLnonum}
scala> val udda = (1 to 1000000 by 2).toVector
scala> import scala.collection.Searching._
scala> udda.search(udda.last)
res18: collection.Searching.SearchResult = Found(499999)
//Search hittar det sista elementet på plats 499999 i samlingen.

scala> udda.search(udda.last + 1)
res19: collection.Searching.SearchResult = InsertionPoint(500000)
//Search kan inte hitta udda.last + 1 eftersom det inte existerar i samlingen
//och returnerar således ett objekt av typen InsertionPoint med värdet 500000.
//Vårt element udda.last + 1 hade alltså legat på plats 500000 om det funnits.

scala> udda.reverse.search(udda(0))
res20: collection.Searching.SearchResult = InsertionPoint(0)
//Som förklarat innan så förutsätter search att listan är sorterad i stigande
//ordning, så den kan inte hitta elementet udda(0) = 1 när listan är inverterad.

scala> def lin(x: Int, xs: Seq[Int]) = xs.indexOf(x)
scala> def bin(x: Int, xs: Seq[Int]) = xs.search(x) match 
	case Found(i) => i
	case InsertionPoint(i) => -i

//Definierar en metod bin som använder sig av metoden search på en sekvens.
//Den ser sedan till med hjälp av "pattern matching" att bara returnera positionen
//i, och inte ett objekt av typen Found eller InsertionPoint.

scala> timed{ lin(udda.last, udda) }
time: 42.294821 ms
res22: (Int, Long) = (499999,42294821)
//För att hitta udda.last = 499999 med linjärsökning tog det ca 42ms.

scala> timed{ bin(udda.last, udda) }
time: 0.147314 ms
res23: (Int, Long) = (499999,147314)
//Binärsökning för att hitta värdet 499999 tog extremt mycket kortare tid.
//Detta för att vid varje steg i binärsökningen halveras mängden tal som
//sökningen måste kolla i. Detta är dock ett extremfall eftersom vi söker
//talet längst bak i listan. Om vi istället gjort en linjärsökning efter
//det första talet 1, hade detta gått minst lika snabbt som binärsökning.
\end{REPLnonum}

\SubtaskSolved
Det behövs $log_2(n)$ jämförelser. Detta eftersom att vi hela tiden halverar antalet element i listan vi behöver söka igenom. Så efter första jämförelsen har vi $\frac{n}{2}$ element kvar. Efter andra jämförelsen har vi $\frac{n}{2*2}$ element kvar etc. När vi bara har ett element kvar har vi hittat det vi söker efter, och har då gjort $b$ antal jämförelser. Ekvationen ser då ut på följande vis:
\begin{equation*}
\frac{n}{2^b} = 1
\end{equation*}
Enligt lagarna för logaritmer kan vi nu komma fram till vad b är:
\begin{equation*}
log_2(n) = b
\end{equation*}

\QUESTEND




\WHAT{Sök bland LTH:s kurser med linjärsökning.}

\QUESTBEGIN

\Task \label{task:linsearch-lth}\what~

\Subtask Via denna URL kan du ladda ner en tab-separerad lista med alla kurser som ges på LTH under innevarande läsår: \url{http://cs.lth.se/pgk/kurser} \\Vilken data finns i filen? Du kan undersöka detta t.ex. med:
\begin{REPLnonum}
scala> import scala.io.Source.fromURL
scala> val url = "https://fileadmin.cs.lth.se/pgk/lthkurser201819.txt"
scala> val data = fromURL(url,"UTF-8").getLines.mkString("\n")
\end{REPLnonum}

\Subtask \label{subtask:download-lthcourses} Klistra in objektet \code{courses} på sidan \pageref{lth-courses} i REPL.\footnote{Du kan ladda ner koden från: \\ \href{https://raw.githubusercontent.com/lunduniversity/introprog/master/compendium/examples/lth-courses/courses.scala}{github.com/lunduniversity/introprog/tree/master/compendium/examples/lth-courses/courses.scala}} Vad gör koden? Hur många kurser innehåller \code{courses.lth}?

\begin{figure}[h]
  \scalainputlisting[basicstyle=\ttfamily\fontsize{10.9}{14}\selectfont]{examples/lth-courses/courses.scala}
  \caption{Kod för att ladda ner data om alla kurser på LTH.}
  \label{lth-courses}
\end{figure}


\Subtask \emph{Linjärsökning med find.} Teknologen Oddput Clementina vill gå första bästa datavetenskapskurs som är på G2-nivå. Hjälp Oddput med att söka upp första förekommande kurs genom linjärsökning med samlingsmetoden \code{find}. Kurskoder vid datavetenskap börjar på EDA eller ETS\footnote{Detta är en förenklad bild av LTH:s kurskodnamnsystem. Några kurser från EIT-institutionen  kommer att slinka med, men det bortser vi ifrån i denna uppgift.}. \emph{Tips:} Du har nytta av att definiera predikatet \code{def isCS(s: String): Boolean} som i sin tur lämpligen nyttjar strängmetoden \code{startsWith}.

\Subtask \emph{Implementera linjärsökning.} Som träning ska du nu implementera en egen linjärsökningsfunktion med signaturen: \\ \code{def linearSearch[T](xs: Seq[T])(p: T => Boolean): Int = ???}
\\ Funktionen ska ta en sekvenssamling \code{xs} och ett predikat \code{p} som är en funktion som tar ett element och returnerar ett booleskt värde. Typen \code{Seq} är supertyp till alla sekvenssamlingar, så om vi använder den som parametertyp för parametern \code{xs} så fungerar funktionen för \code{Vector}, \code{Array}, \code{List}, etc. Genom typparametern \code{T} blir funktionen generisk och fungerar för godtycklig typ.
Funktionen \code{p} ska ge \code{true} om parametern är ett eftersökt element. Funktionen \code{linearSearch} ska returnera index för första hittade elementet i \code{xs} där \code{p} gäller. Om det inte finns något element som uppfyller predikatet ska -1 returneras. Skriv först pseudokod för funktionen med penna och papper. Du ska använda \code{while}.



\Subtask \label{subtask:linsearch-rndCode} Implementera en funktion \code{def rndCode: String} som genererar slumpmässiga kurskoder som består av 4 bokstäver mellan A och Z följt av 2 siffror mellan 0 och 9. \emph{Tips:} Använd REPL i kombination med en editor för att stegvis skapa och testa hjälpfunktioner som löser lämpliga delproblem.


\Subtask Använd \code{rndCode} från föregående deluppgift för att fylla en vektor kallad \code{xs} med en halv miljon slumpmässiga kurskoder. För varje slumpkod i \code{xs} sök med din funktion \code{linearSearch} efter index i vektorn \code{courses.lth} från deluppgift \ref{subtask:download-lthcourses}. Mät totala tiden för de $500000$ linjärsökningarna med hjälp av funktionen \code{timed} från uppgift \ref{task:timed}. Hur många av de slumpmässiga kurskoderna hittades bland de verkliga kurskoderna på LTH?



\Subtask Hur kan du implementera \code{linearSearch} med den inbyggda samlingsmetoden \code{indexWhere}?



\SOLUTION


\TaskSolved \what


\SubtaskSolved
Första raden innehåller kolumnnamnen \code{Kurskod KursSve KursEng Hskpoang Niva}. Därefter kommer en rad för varje kurs med kursdata enligt kolumnnamnen.

\SubtaskSolved
Koden laddar ner data och skapar en vektor med instanser av case-klassen \code{Course} med hjälp av metoden \code{fromLine}. Eftersom variabeln \code{lth} är deklarerad som \code{lazy} kommer inte \code{download()} bli anropad förrän första gången som variablen \code{lth} refereras. Antalet kurser ges av:
\begin{REPLnonum}
scala> val n = courses.lth.length
n: Int = 1104
\end{REPLnonum}

\SubtaskSolved
\begin{REPL}
scala> def isCS(s: String) = s.startsWith("EDA") || s.startsWith("ETS")
scala> val x = courses.lth.find(c => isCS(c.code) && c.level == "G2")
x: Option[courses.Course] = Some(Course(EDAF05,Algoritmer, datastrukturer och
   komplexitet,Algorithms, Data Structures and Complexity,5.0,G2))
\end{REPL}

\SubtaskSolved
\begin{Code}
def linearSearch[T](xs: Seq[T])(p: T => Boolean): Int = 
   var i = 0
   while(i < xs.length && !p(xs(i))) i += 1
   if (i < xs.length) i else -1
\end{Code}

\SubtaskSolved

\begin{Code}
def rndCode: String = 
   //randomizes from 0 to n (inclusive)
   def rnd(n: Int) = (math.random() * (n + 1)).toInt

   def letter = (rnd('Z' - 'A') + 'A').toChar
   def dig = ('0' + rnd(9)).toChar
   Seq(letter, letter, letter, letter, dig, dig).mkString
\end{Code}

\SubtaskSolved

\begin{Code}
val xs = Vector.fill(500000)(rndCode)
val(ixs, elapsedLin) =
  timed { xs.map(x => linearSearch(courses.lth)(_.code == x)) }
val found = ixs.filterNot(_== -1).size
\end{Code}

\SubtaskSolved

\begin{Code}
def linearSearch[T](xs: Seq[T])(p: T => Boolean): Int = xs.indexWhere(p)
\end{Code}



\QUESTEND

%%%%%% GAMLA VARIANTEN AV OVAN UPPGIFT
%%%%%% -- Funkar ej längre URL-api till LTH:S databas
% \WHAT{Sök bland LTH:s kurser med linjärsökning.}
%
% \QUESTBEGIN
%
% \Task \label{task:linsearch-lth} \what~ OBS! Använd \code{https} och \emph{inte} \code{http} i webbadresserna i denna och nästa uppgift, för att det ska fungera.
%
% \Subtask Surfa till denna URL:\\{%\nolinebreak[4]
% \footnotesize\url{https://kurser.lth.se/lot/?lasar=17_18&soek_text=&sort=kod&val=kurs&soek=t}}
% \\
% och inspektera HTML-koden i din webbläsare genom att trycka \emph{Ctrl+U} (fungerar i Firefox och Chrome). Rulla ner till rad 171 och framåt. Var finns antalet poäng för respektive kurs i HTML-koden?
%
% \Subtask \label{subtask:download-lthcourses} Klistra in objektet \code{courses} på sidan \pageref{lth-courses} med kommandot \code{:paste} i REPL.\footnote{Du kan ladda ner koden från: \\ \href{https://raw.githubusercontent.com/lunduniversity/introprog/master/compendium/examples/lth-courses/courses.scala}{github.com/lunduniversity/introprog/tree/master/compendium/examples/lth-courses/courses.scala}} Vad gör koden? Hur många kurser innehåller \code{lth2017}?
%
% \begin{figure}
%   \scalainputlisting[basicstyle=\ttfamily\fontsize{10.9}{14}\selectfont]{examples/lth-courses/courses.scala}
%   \caption{Kod för att söka bland kurser från LTH:s webbsida.}
%   \label{lth-courses}
% \end{figure}
%
%
% \Subtask \emph{Linjärsökning med find.} Teknologen Oddput Clementina vill gå första bästa datavetenskapskurs som är på G2-nivå. Hjälp Oddput med att söka upp första bästa kurs genom linjärsökning med samlingsmetoden \code{find}. Kurskoder vid datavetenskap börjar på EDA eller ETS\footnote{Detta är en förenklad bild av LTH:s kurskodnamnsystem. Några kurser från EIT-institutionen  kommer att slinka med, men det bortser vi ifrån i denna uppgift.}. \emph{Tips:} Du har nytta av att definiera predikatet \code{def isCS(s: String): Boolean} som i sin tur lämpligen nyttjar strängmetoden \code{startsWith}.
%
% \Subtask \emph{Implementera linjärsökning.} Som träning ska du nu implementera en egen linjärsökningsfunktion med signaturen: \\ \code{def linearSearch[T](xs: Seq[T])(p: T => Boolean): Int = ???}
% \\ Funktionen ska ta en sekvenssamling \code{xs} och ett predikat \code{p} som är en funktion som tar ett element och returnerar ett booleskt värde. Funktionen \code{p} ska ge \code{true} om parametern är ett eftersökt element. Funktionen \code{linearSearch} ska returnera index för första hittade elementet i \code{xs} där \code{p} gäller. Om det inte finns något element som uppfyller predikatet ska -1 returneras. Skriv först pseudokod för funktionen med penna och papper. Använd \code{while}.
%
% Typen \code{Seq} är supertyp till alla sekvenssamlingar, så om vi använder den som parametertyp för parametern \code{xs} så fungerar funktionen för \code{Vector}, \code{Array}, \code{List}, etc. Genom typparametern \code{T} blir funktionen generisk och fungerar för godtycklig typ.
%
%
%
% \Subtask \label{subtask:linsearch-rndCode} Implementera en funktion \code{def rndCode: String} som genererar slumpmässiga kurskoder som består av 4 bokstäver mellan A och Z följt av 2 siffror mellan 0 och 9. \emph{Tips:} Använd REPL  för att stegvis bygga upp hjälpfunktioner som du, när de fungerar som de ska, klistrar in i ett editorfönster som lokala funktioner där du utvecklar den slutliga koden för en lättläst, koncis och fungerande \code{rndCode}.
%
%
% \Subtask Använd \code{rndCode} från föregående deluppgift för att fylla en vektor kallad \code{xs} med en halv miljon slumpmässiga kurskoder. För varje slumpkod i \code{xs} sök med din funktion \code{linearSearch} efter index i vektorn \code{courses.lth2017} från deluppgift \ref{subtask:download-lthcourses}. Mät totala tiden för de $500000$ linjärsökningarna med hjälp av funktionen \code{timed} från uppgift \ref{task:timed}. Hur många av de slumpmässiga kurskoderna hittades bland de verkliga kurskoderna på LTH?
%
%
%
% \Subtask\Pen Hur kan du implementera \code{linearSearch} med den inbyggda samlingsmetoden \code{indexWhere}?
%
%
%
% \SOLUTION
%
%
% \TaskSolved \what
%
%
% \SubtaskSolved
% Den finns som värde för en \emph{td} tagg, på följande vis: \code{<td class="mitt">2</td>}.
%
% \SubtaskSolved
% Koden laddar ner html-koden för sidan \\ \mbox{\small\url{https://kurser.lth.se/lot/?lasar=17_18&soek_text=&sort=kod&val=kurs&soek=t}} och sparar den i en vektor. Sedan filtreras ut endast de rader som innehåller strängen ”kurskod” så att all onödig HTML-kod försvinner. Sedan konverteras detta, för varje rad, till \code{Course}-objekt med hjälp av metoden \code{fromHtml}. Eftersom variabeln \code{lth2017} är deklarerad som \code{lazy} kommer inte \code{download()} bli anropad förrän vi vill komma åt variabeln. Vi startar alltså processen genom att referera variabeln \code{lth2017} i objektet \code{courses}:
%
% \begin{REPLnonum}
% courses.lth2017
% \end{REPLnonum}
% Detta generarar en lång lista med \code{Course}-objekt. Antalet kurser är således lika med storleken på vektorn \code{lth2017}.
%
% \begin{REPLnonum}
% courses.lth2017.size
% res38: Int = 1101
% \end{REPLnonum}
%
% \SubtaskSolved
% \begin{REPL}
% scala> def isCS(s: String) = s.startsWith("EDA") || s.startsWith("ETS")
% scala> val x = courses.lth2017.find(c => isCS(c.code) && c.level == "G2").get
% x: courses.Course = Course(EDAF05,Algoritmer, datastrukturer och komplexitet,Algorithms, Data Structures and Complexity,5.0,G2)
% \end{REPL}
% Obs: metoden \code{find} returnerar ett objekt av typen \code{Option}. För att få värdet som är lagrat i detta objekt krävs det att man kallar på \code{get}.
%
% \SubtaskSolved
% \begin{Code}
% def linearSearch[T](xs: Seq[T])(p: T => Boolean): Int = {
%    var i = 0
%    while(i < xs.size && !p(xs(i))) i += 1
%    if (i < xs.size) i else -1
% }
% \end{Code}
%
% \SubtaskSolved
%
% \begin{Code}[language=Scala]
% def rndCode: String = {
%    //randomizes from 0 to n (inclusive)
%    def rnd(n: Int) = (math.random() * (n + 1)).toInt
%
%    def letter = (rnd('Z' - 'A') + 'A').toChar
%    def dig = ('0' + rnd(9)).toChar
%    Seq(letter, letter, letter, letter, dig, dig).mkString
% }
% \end{Code}
%
% \SubtaskSolved
%
% \begin{Code}
% val lthCourses = courses.lth2017 //avoid including download time
% val xs = Vector.fill(500000)(rndCode)
% val(ixs, elapsedLin) = timed{
% xs.map(x => linearSearch(lthCourses)(_.code == x))}
% val found = ixs.filterNot(_== -1).size
% \end{Code}
%
% \SubtaskSolved
%
% \begin{Code}
% def linearSearch[T](xs: Seq[T])(p: T => Boolean): Int =
%   xs.indexWhere(p)
% \end{Code}
%
%
%
% \QUESTEND






\WHAT{Sök bland LTH:s kurser med binärsökning.}

\QUESTBEGIN

\Task  \what~Sökalgoritmen BINSEARCH kan formuleras med nedan pseudokod:

\begin{algorithm}[H]
 \SetKwInOut{Input}{Indata}\SetKwInOut{Output}{Utdata}

 \Input{En växande sorterad sekvens $xs$ med $n$ heltal och \\ ett eftersökt heltal $key$}
 \Output{Ett heltal $i \geq 0$ som anger platsen där $x$ finns, eller ett negativt tal $i$ där $-i$ motsvarar platsen där $x$ ska sättas in i sorterad ordning om $x$ ej finns i samlingen.}
 sätt intervallet ($low$, $high$) till ($0$, $n - 1$) \\
 $found \leftarrow \bf{false}$ \\
 $mid \leftarrow -1$\\
 \While{$low \leq high$~$\bf{and}~\bf{not}$ $found$}{
   $mid \leftarrow $ platsen mitt emellan $low$ och $high$\\
   \eIf{$xs(mid)$ == $key$}{$found \leftarrow \bf{true}$}{
     \eIf{$xs(mid) < key$}{$low \leftarrow mid + 1$}{$high \leftarrow mid - 1$}
    }
 }
 \eIf{$found$}{$mid$}{$-(low + 1)$}
\end{algorithm}

\Subtask Prova algoritmen ovan med penna och papper på en sorterad sekvens med mindre än 10 heltal. Prova om algoritmen fungerar med ett jämnt antal tal, ett udda antal tal, en sekvens med ett heltal och en tom sekvens. Prova både om talet du letar efter finns och om det inte finns.

\Subtask Implementera binärsökning i en funktion med signaturen\\
\code{def binarySearch(xs: Seq[String], key: String): Int = ??? }\\
och testa i REPL för olika fall. Vad händer om sekvensen inte är sorterad?

\Subtask Använd \code{binarySearch} för att leta efter LTH-kurser enligt nedan. Använd \code{rndCode}, \code{timed} och \code{courses} från tidigare uppgifter.
\begin{Code}
def binarySearch(xs: Seq[String], key: String): Int = ???

val lthCodesSorted = courses.lth.map(_.code).sorted
val xs = Vector.fill(500000)(rndCode)
val (_, elapsedBin) =
  timed{xs.map(x => binarySearch(lthCodesSorted, x))}
val (_, elapsedLin) =
  timed{xs.map(x => linearSearch(lthCodesSorted)(_ == x))}
println(elapsedLin / elapsedBin)
\end{Code}


\Subtask Hur mycket snabbare blev binärsökningen jämfört med linjärsökningen?\footnote{Vid en körning på en i7-4970K med 4.0GHz tog \code{elapsedLin} cirka $3000~ms$ och \code{elapsedBin} cirka $60~ms$. Binärsökning var alltså i detta fall ungefär $50$ gånger snabbare än linjärsökning.}


\SOLUTION


\TaskSolved \what


\SubtaskSolved ---

\SubtaskSolved
\begin{Code}
def binarySearch(xs: Seq[String], key: String): Int = 
  var (low, high) = (0, xs.size - 1)
  var found = false
  var mid = -1

  while (low <= high && !found) do
    mid = (low + high) / 2
    if xs(mid) == key then found = true
    else if xs(mid) < key then low = mid + 1
    else high = mid - 1
  end while
  if found then  mid else -(low + 1)
\end{Code}

\SubtaskSolved
Med en i7-3770K @ 3.50Hz tog sökningarna följande tid:

\begin{itemize}
\item Binärsökning: \code{time: 142.6 ms}
\item Linjärsökning: \code{time: 3316.5 ms}
\end{itemize}

Med en i7-8700T @ 2.40GHz tog sökningarna följande tid:
\begin{itemize}
\item Binärsökning: \code{time: 81.5 ms}
\item Linjärsökning: \code{time: 5138.6 ms}
\end{itemize}




\SubtaskSolved
Binärsökningen var ca 23 gånger snabbare på en i7-3770K @ 3.50Hz och ca 63 gånger snabbare på en i7-8700T CPU @ 2.40GHz.



\QUESTEND





\WHAT{Insättningssortering.}

\QUESTBEGIN

\Task  \what~ Implementera sortering av en heltalssekvens till en  sekvens med \textbf{insättningssortering} \Eng{insertion sort} i en funktion med följande signatur:
\begin{Code}
def insertionSort(xs: Seq[Int]): Seq[Int] = ???
\end{Code}

\emph{Lösningsidé:} Skapa en ny, tom sekvens som ska bli vårt sorterade resultat. För varje element i den osorterade sekvensen: Sätt in det på rätt plats i den nya sorterade sekvensen.

\Subtask \emph{Pseudokod:} Kör nedan pseudokod med papper och penna t.ex. på sekvensen 5 1 4 3 2 1. Rita minnessituationen efter varje runda i loopen. Här använder vi internt i funktionen föränderliga \code{ArrayBuffer} som är snabb på insättning och avslutar med \code{toVector} så att vi lämnar ifrån oss en oföränderlig sekvens.

\begin{algorithm}[H]
    $result \leftarrow$ en ny, tom ArrayBuffer \\
    \ForEach{element $e$ \bf{in} $xs$}{
      $pos \leftarrow$  leta upp rätt position i $result$ \\
      stoppa in $e$ på plats $pos$ i $result$
    }
    $result$.toVector
\end{algorithm}


\Subtask Implementera \code{insertionSort}. Använd en \code{while}-loop för att implementera rad 3 i pseudokoden. Sök upp dokumentationen för metoden \code{insert} på \code{ArrayBuffer}. Testa  \code{insert} på \code{ArrayBuffer} i REPL och verifiera att den kan användas för att stoppa in på slutet på den ''oanvända'' positionen som är precis efter sista positionen. Vad händer om man gör \code{insert} på positionen \code{size + 2}?

Klistra in din implementation av \code{insertionSort} i REPL och testa så att allt fungerar:
\begin{REPL}
scala> insertionSort(Vector())
res0: Seq[Int] = Vector()

scala> insertionSort(Vector(42))
res1: Seq[Int] = Vector(42)

scala> insertionSort(Vector(1,2,3))
res2: Seq[Int] = Vector(1, 2, 3)

scala> insertionSort(Vector(5,1,4,3,2,1))
res3: Seq[Int] = Vector(1, 1, 2, 3, 4, 5)
\end{REPL}


\SOLUTION

\TaskSolved \what


\SubtaskSolved ---

\SubtaskSolved

\begin{Code}
def insertionSort(xs: Seq[Int]): Seq[Int] = 
  val result = scala.collection.mutable.ArrayBuffer.empty[Int]
  for e <- xs do
    var pos = 0
    while pos < result.size && result(pos) < e do pos += 1
    result.insert(pos,e)
  end for
  result.toVector
\end{Code}

\QUESTEND





\WHAT{Sortering på plats.}

\QUESTBEGIN

\Task  \what~ Implementera sortering på plats \Eng{in-place} i en \code{Array[String]} med urvalssortering \Eng{selection sort}

\emph{Lösningsidé:} För alla index $i$: sök $minIndex$ för ''minsta'' strängen från plats $i$ till sista plats och byt plats mellan strängarna på plats $i$ och plats $minIndex$. Se även animering här: \href{https://sv.wikipedia.org/wiki/Urvalssortering}{sv.wikipedia.org/wiki/Urvalssortering}

Implementera enligt nedan skiss.  \emph{Tips:} Du har nytta av en modifierad variant av lösningen till uppgift \ref{task:minindex} i kapitel \ref{chapter:W02}.
\begin{Code}
def selectionSortInPlace(xs: Array[String]): Unit = 
  def indexOfMin(startFrom: Int): Int = ???
  def swapIndex(i1: Int, i2: Int): Unit = ???
  for i <- 0 to xs.size - 1 do swapIndex(i, indexOfMin(i))
\end{Code}




\SOLUTION


\TaskSolved \what


\begin{Code}
def selectionSortInPlace(xs: Array[String]): Unit = 
  def indexOfMin(startFrom: Int): Int = 
    var minPos = startFrom
    var i = startFrom + 1
    while (i < xs.size) do
      if (xs(i) < xs(minPos)) minPos = i
      i += 1
    end while
    minPos
  end indexOfMin

  def swapIndex(i1: Int, i2: Int): Unit = 
    val temp = xs(i1)
    xs(i1) = xs(i2)
    xs(i2) = temp
  end swapIndex  

  for i <- 0 to xs.size - 1 do swapIndex(i, indexOfMin(i))
end selectionSortInPlace
\end{Code}


\QUESTEND


\clearpage

%\ExtraTasks %%%%%%%%%%%%%%%%%%%




\WHAT{Undersök om en sekvens är sorterad.}

\QUESTBEGIN

\Task \label{task:isSorted} \what~   Ett enkelt och lättläst sätt att undersöka om en sekvens är sorterad visas nedan.
\begin{REPL}
scala> def isSorted(xs: Vector[Int]): Boolean = xs == xs.sorted
\end{REPL}


\Subtask\Pen  Om \code{xs} har $10^6$ element, hur många jämförelser kommer i värsta fall att ske med \code{isSorted} enligt ovan? Metoden \code{sorted} använder algoritmen Timsort\footnote{\href{http://stackoverflow.com/questions/14146990/what-algorithm-is-used-by-the-scala-library-method-vector-sorted}{stackoverflow.com/questions/14146990/what-algorithm-is-used-by-the-scala-library-method-vector-sorted}}. Sök upp antalet jämförelser i värstafallet på Wikipedia.

Denna lösning är dock relativt långsam för stora samlingar. Man behöver ju inte först sortera  för att avgöra om det är sorterat (om man inte ändå hade tänkt sortera av andra skäl), det räcker att kolla att elementen är i växande ordning.

\Subtask\label{subtask:issorted} Implementera en effektivare variant av \code{isSorted} som använder en \code{while}-sats och kollar att elementen är i växande ordning. Din algoritm ska sluta söka så fort osorterade element hittats.

\Subtask\Pen Vad blir antalet jämförelser i värstafallet med metoden i deluppgift \ref{subtask:issorted} om du har $n$ element?


\Subtask \label{subtask:isSorted-zip} Man kan kolla om en sekvens är sorterad med det listiga tricket att först zippa sekvensen med sin egen svans och sedan kolla om alla element-par uppfyller sorteringskriteriet, alltså \code{xs.zip(xs.tail).forall(???)} där \code{???} byts ut mot lämpligt predikat. Vilken typ har 2-tupeln \code{xs.zip(xs.tail))} om \code{xs} är av typen \code{Vector[Int]}? Implementera \code{isSorted} med detta listiga trick. 

\SOLUTION


\TaskSolved \what



\SubtaskSolved Det tar i värsta fall $O(n*log(n))$ för timsort att sortera listan med $n$ element. Sedan krävs $n$ stycken jämförelser mellan den sorterade och osorterade listan. Det totala antalet jämförelser i värstafallet uppgår därför till max $n + n*log(n)$. För $10^6$ element blir det ca $10^7$ jämförelser.
\begin{REPLnonum}
scala> val n = 1E6
val n: Double = 1000000.0

scala> def worstCase(n: Double) = n + n * math.log(n)
def worstCase(n: Double): Double

scala> println(s"i värsta fall med n=$n så blir det ${worstCase(n)} jämförelser")
i värsta fall med n=1000000.0 så blir det 1.4815510557964273E7 jämförelser
\end{REPLnonum}

\SubtaskSolved En mer effektiv version av \code{isSorted} som avbryter sökningen när ett osorterat element upptäcks:
\begin{Code}
def isSorted(xs: Vector[Int]): Boolean = 
  if xs.length > 1 then
    var i = 0
    var result = true
    while i < xs.length-1 && result do 
      if xs(i) > xs(i+1) then result = false
      i += 1
    end while
    result
  else true
end isSorted
\end{Code}

\SubtaskSolved I värsta fall behöver man göra $n - 1$ parvisa jämförelser, om alla ligger i sorterad ordning utom den sista.


\SubtaskSolved 2-tupeln är av typen \code{(Int, Int)}.

\begin{Code}
def isSorted(xs: Vector[Int]): Boolean =
  xs.zip(xs.tail).forall(x => x._1 <= x._2)
\end{Code}



\QUESTEND






\WHAT{Insättningssortering på plats.}

\QUESTBEGIN

\Task  \what~ Implementera och testa sortering på plats i en array med heltal med \footnote{\href{https://en.wikipedia.org/wiki/Insertion_sort}{en.wikipedia.org/wiki/Insertion\_sort}}.

Implementera och testa funktionen nedan i Scala med följande signatur:
\begin{Code}
  def insertionSort(xs: Array[Int]): Unit
\end{Code}
Placera metoden i ett objekt med lämpligt namn, samt skapa ett huvudprogram med testkod. Kompilera och kör från terminalen. Börja med att skriva sorteringsalgoritmen i pseudokod.

% \Subtask Implementera och testa metoden nedan i Java med följande signatur:
% \begin{Code}[language=Java]
%   public static void insertionSort(int[] xs)
% \end{Code}
% Placera metoden i en klass med lämpligt namn, samt skapa ett huvudprogram med testkod. Börja med att skriva sorteringsalgoritmen i pseudokod.

\SOLUTION


\TaskSolved \what


\begin{Code}
def insertionSort(xs: Array[Int]): Unit = 
  for elem <- 1 until xs.length if xs.length > 0 do
    var pos = elem
    while pos > 0 && xs(pos) < xs(pos - 1) do
      val temp = xs(pos -1)
      xs(pos -1) = xs(pos)
      xs(pos) = temp
      pos -= 1
    end while
  end for
end insertionSort
\end{Code}

% \SubtaskSolved

% \begin{Code}[language=Java]
% public static void insertionSort(int[] xs) {

%     if (xs.length < 1)
%         return;

%     for (int i = 1; i < xs.length; i++) {
%         int pos = i;

%         for (; pos > 0 && xs[pos] < xs[pos - 1]; pos--) {
%             int temp = xs[pos - 1];
%             xs[pos - 1] = xs[pos];
%             xs[pos] = temp;
%         }
%     }
% }
% \end{Code}



\QUESTEND



\clearpage

%\AdvancedTasks


\WHAT{Sortering till ny sekvens med urvalssortering.}

\QUESTBEGIN

\Task  \what~ Implementera och testa sortering till ny sekvens med urvalssortering\footnote{\href{https://en.wikipedia.org/wiki/Selection_sort}{en.wikipedia.org/wiki/Selection\_sort}} i Scala, enligt nedan skiss.  Du har nytta av lösningen till uppgift \ref{task:minindex} i kapitel \ref{chapter:W02}.
\begin{Code}
def selectionSort(xs: Seq[String]): Seq[String] = 
  def indexOfMin(xs: Seq[String]): Int = ???
  val unsorted = xs.toBuffer
  val result = scala.collection.mutable.ArrayBuffer.empty[String]
  /*
  så länge unsorted inte är tom 
    minPos = indexOfMin(unsorted)
    elem   = unsorted.remove(minPos)
    result.append(elem)
  */
  result.toVector
end selectionSort
\end{Code}



\SOLUTION


\TaskSolved \what


\begin{Code}
def selectionSort(xs: Seq[String]): Seq[String] = 
  def indexOfMin(xs: Seq[String]): Int = xs.indexOf(xs.min)
  val unsorted = xs.toBuffer
  val result = scala.collection.mutable.ArrayBuffer.empty[String]
  while !unsorted.isEmpty do
    val minPos = indexOfMin(unsorted)
    val elem = unsorted.remove(minPos)
    result.append(elem)
  end while
  result.toVector
end selectionSort
\end{Code}


\QUESTEND














%
%
% \WHAT{NEEDS A TOPIC DESCRIPTION}
%
% \QUESTBEGIN
%
% \Task  \what~ Fördjupa dig inom webbteknologi.
%
% \Subtask Lär dig om HTML här: \url{http://www.w3schools.com/html/}
%
% \Subtask Lär dig om Javascript här: \url{http://www.w3schools.com/js/}
%
% \Subtask Lär dig om CSS här: \url{http://www.w3schools.com/css/}
%
% \Subtask Lär dig om Scala.JS här: \url{http://www.scala-js.org/}\SOLUTION
%
%
% \TaskSolved \what
%
% \QUESTEND


\subsection{Uppgifter om trådar och jämlöpande exekvering}

\WHAT{Trådar.}

\QUESTBEGIN

\Task  \what~   Klassen \code{java.lang.Thread} används för att skapa  \textbf{trådar} med jämlöpande exekvering \Eng{concurrent execution}. På så sätt kan man få olika koddelar att köra samtidigt.

Klassen \code{Thread} definierar en tom \code{run}-metod. Vill man att tråden ska göra något vettigt får man överskugga \code{run} med det man vill ska göras.

En tråd körs igång med metoden \code{start} och då anropas automatiskt \code{run}-metoden och tråden exekverar koden i \code{run} jämlöpande med övriga trådar. Om man anropar \code{run} direkt blir det \emph{inte} jämlöpande exekvering.

\Subtask Skapa en tråd som gör något som tar lite tid och kör med \code{run} resp. \code{start}.
\begin{REPL}
def zzz = { print("zzzzzz"); Thread.sleep(5000); println(" VAKEN!")}
zzz
val t2 = new Thread{ override def run = zzz }
t2.run; println("Gomorron!")
t2.start; println("Gomorron!")
t2.start
\end{REPL}

\Subtask Vad händer om man anropar \code{start} mer än en gång på samma tråd?

\Subtask Skapa två trådar med överskuggade \code{run}-metoder och kör igång dem samtidigt enligt nedan. Vilken ordning skrivs hälsningarna ut efter rad 3 resp. rad 4 nedan? Förklara vad som händer.
\begin{REPL}
val g = new Thread{ override def run = for i <- 1 to 100 do print("Gurka ") }
val t = new Thread{ override def run = for i <- 1 to 100 do print("Tomat ") }
g.run; t.run
g.start; t.start
\end{REPL}

\Subtask Använd \code{Thread.sleep} enligt nedan. Är beteendet helt förutsägbart (deterministiskt)? Förklara vad som händer. Du kan avbryta REPL med CTRL+C eller SHIFT+CTRL+C, beroende på din terminalinställningar.%
\footnote{\href{http://stackoverflow.com/questions/6248884/can-i-stop-the-execution-of-an-infinite-loop-in-scala-repl}{stackoverflow.com/questions/6248884/can-i-stop-the-execution-of-an-infinite-loop-in-scala-repl}}.
\begin{REPL}
def ibland(block: => Unit) = new Thread {
  override def run = while(true) { block; Thread.sleep(600) }
}.start
ibland(print("zzz ")); ibland(print("snark ")); ibland(println("hej!"))
\end{REPL}


\SOLUTION


\TaskSolved \what
     %%%TODO number  1 %%%starts with: \emph{Trådar.}  %%%

\SubtaskSolved   -

\SubtaskSolved  \code {java.lang.IllegalThreadStateException}. Det går inte att starta en tråd mer än en gång. Tråden kan därför inte startas om när den redan har exekverats.

\SubtaskSolved   När \code {start} anropas exekveras koden i \code{run} parallellt. Därför skrivs \code{Gurka} och \code{Tomat} ut omlöpande. Om istället \code{run} anropas direkt blir det inte jämnlöpande exekvering och \code{Gurka} skrivs ut 100 gånger, sedan skrivs \code{Tomat} ut 100 gånger.

\SubtaskSolved   \code{Thread.sleep} pausar inte tråden i exakt den tiden som angets. Alltså kommer det skrivas ut \code{zzz snark hej!} i de flesta fall, men det är inte garanterat.



\QUESTEND






\WHAT{Jämlöpande variabeluppdatering.}

\QUESTBEGIN

\Task \label{task:racecondition} \what~   Skriv klasserna \code{Bank} och \code{Kund} i en editor och klistra sedan in koden i REPL.

\begin{Code}
class Bank:
  private var _saldo = 0;
  def saldo: Int = _saldo
  def sättIn(): Unit = _saldo += 1 
  def taUt(): Unit   = _saldo -= 1 
end Bank

class Kund(bank: Bank):
  def slösaSpara(): Unit = 
    bank.taUt()
    Thread.sleep(1)
    bank.sättIn()
  end slösaSpara
end Kund
\end{Code}

\Subtask Använd funktionen \code{ibland} från föregående uppgift och kör nedan rader i REPL. Resultatet av jämlöpande variabeluppdatering blir här heltokigt och leder till mycket upprörda bankkunder och -ägare. Förklara vad som händer.

\begin{REPL}
val bank = new Bank
println(bank.saldo)
bank.sättIn()
println(bank.saldo)
bank.taUt()
println(bank.saldo)

val bamse = new Kund(bank)
val skutt = new Kund(bank)

bamse.slösaSpara()
skutt.slösaSpara()
println(bank.saldo)

def ofta(block: => Unit) = new Thread { // varje millisekund
  override def run = while true do { block;  Thread.sleep(1)} 
}.start

ofta(bamse.slösaSpara()); ofta(skutt.slösaSpara())

def ibland(block: => Unit) = new Thread {  // varje 600 ms
  override def run = while(true) do { block; Thread.sleep(600) }
}.start

ibland(println(bank.saldo))
\end{REPL}


\SOLUTION


\TaskSolved \what
     %%%TODO number  2 %%%starts with: \emph{Jämlöpande variabeluppdat%%%

\SubtaskSolved  I \code{slösaSpara} hämtas saldot, ändras och placeras tillbaka i minnet -  fördröjs -  upprepas. Om \code{bamse} blir klar med att ladda, ändra och lagra innan skutt gör detsamma blir det problem, då de tävlar om vem som får uppdatera \Eng{race contiion}. Problemet innan en tråd kan lagra det förändrade värdet laddar den andra tråden det gamla värdet. Bara en av dessa trådar vinner racet och får lagra sitt ändrade tal och den andra ändringen går förlorad. \code{skutt} och \code{bamse} blir alltså upprörda för att inte alla dess uttag och insättningar registreras.


\QUESTEND






\WHAT{Trådsäkra \code{AtomicInteger}.}

\QUESTBEGIN

\Task  \what~  Det finns stöd i JVM för att åstadkomma uppdateringar som inte kan avbrytas av andra trådar under pågånde minnesskrivning. En operation som inte kan avbrytas kallas \textbf{atomär} \Eng{atomic}. Studera dokumentationen för \code{AtomicInteger}\footnote{\href{https://docs.oracle.com/javase/8/docs/api/java/util/concurrent/atomic/AtomicInteger.html}{docs.oracle.com/javase/8/docs/api/java/util/concurrent/atomic/AtomicInteger.html}} och prova nedan kod. Förklara vad som händer.

Använd funktionerna \code{ofta} och \code{ibland} från tidigare uppgifter.
\begin{Code}
class SäkerBank:
  import java.util.concurrent.atomic.AtomicInteger
  private var _saldo = new AtomicInteger
  def saldo: Int = _saldo.get
  def sättIn(): Unit = _saldo.incrementAndGet()
  def taUt(): Unit   = _saldo.decrementAndGet()
end SäkerBank

class SäkerKund(bank: SäkerBank):
  def slösaSpara = 
    bank.taUt()
    Thread.sleep(1)
    bank.sättIn()
  end slösaSpara
end SäkerKund

\end{Code}
\begin{REPL}
val sb = new SäkerBank
val farmor = new SäkerKund(sb)
val vargen = new SäkerKund(sb)

ofta(farmor.slösaSpara); ofta(vargen.slösaSpara)

ibland(println(sb.saldo))
\end{REPL}


\SOLUTION


\TaskSolved \what
     %%%TODO number  3 %%%starts with: \emph{Jämlöpande exekvering med%%%

Nu är \code{farmor}-tråden garanterad att kunna ladda saldot, ta ut pengar/ändra och lagra innan \code{vargen}-tråden kan skriva över resultatet. I \code{slösaSpara} pausas tråden i en millisekund så \code{vargen}-tråden kan hinna ta ut pengar innan \code{farmor}-sätter hinner sätta in pengar igen och saldot blir negativt. Dock kommer alla uttag och insättningar registreras eftersom operationerna är atomära och saldot kommer återställas till noll, utan att insättningar går förlorade.


\QUESTEND






\WHAT{Jämlöpande exekvering med \code{scala.concurrent.Future}.}

\QUESTBEGIN

\Task \label{task:future} \what~   Att skapa och hålla reda på trådar kan bli ganska omständligt och knepigt att få rätt på.
Med hjälp av \code{scala.concurrent.Future} kan man på ett enklare sätta skapa jämlöpande exekvering.

\begin{Background}
Med en \code{Future} skapas jämlöpande exekvering som ''under huven'' använder ett ramverk som heter Akka\footnote{\url{http://akka.io/}}, skrivet i Scala och Java. Akka erbjuder automatisk  multitrådning med s.k. trådpooler och möjliggör avancerad parallellprogrammering på en hög  abstraktionsnivå, där man själv slipper skapa instanser av klassen \code{Thread}. I stället kan man helt enkelt placera sin kod inramad med \code|Future{ "körs parallellt" }| efter att man importerat det som behövs.
\end{Background}

\Subtask För att skapa jämlöpande exekvering med \code{Future} behöver man först göra import enligt nedan; då skapas ett exekveringssammanhang med trådpooler redo för användning. Starta om REPL och studera felmeddelandet efter rad 1 nedan. Importera därefter enligt nedan. Vad har \code{f} för typ?
\begin{REPL}
scala> concurrent.Future { Thread.sleep(1000); println("En sekund senare!") }
scala> import scala.concurrent._
scala> import ExecutionContext.Implicits.global
scala> val f = Future { Thread.sleep(1000); println("En sekund senare!") }
\end{REPL}

\Subtask Skapa en procedur \code{printLater} enligt nedan som skriver ut argumentet efter slumpmässig tid. Förklara vad som händer nedan.
\begin{REPL}
scala> def printLater(a: Any): Unit =
         Future { Thread.sleep((math.random() * 10000).toInt); print(a + " ") }
scala> (1 to 42).foreach(i => printLater(i)); println("alla är igång!")
\end{REPL}

\Subtask Skapa enligt nedan en \code{Future} som räknar ut hur många siffror det är i ett väldigt stort tal. Med \code{onComplete} kan man ange vad som ska göras när den tunga beräkningen är färdig; detta kallas att ''registrera en callback''. Vilken returtyp har \code{big}? Hur många siffror har det stora talet? Vad har \code{r} för typ? Justera argumentet till \code{big} om du inte orkar vänta på resultatet...

\begin{REPL}
scala> BigInt(10).pow(100)
scala> BigInt(10).pow(100).toString.size
scala> def big(n: Int) = Future { BigInt(n).pow(n).toString.size }
scala> big(1234567).onComplete{r => println(r + " siffror") }
\end{REPL}

\Subtask Den stora vinsten med \code{Future} är att man kan köra vidare under tiden, varför anropet av \code{Future} kallas \textbf{icke-blockerande} \Eng{non-blocking}. Det händer ibland att man ändå vill blockera exekveringen i väntan på ett resultat. Man kan då använda objektet \code{scala.concurrent.Await} och dess metod \code{result} enligt nedan. Använd \code{big} från föregående uppgift och gör en blockerande väntan på resultatet enligt nedan. Vad händer? Vad händer om du väntar för kort tid?

\begin{REPL}
scala> import scala.concurrent.duration._
scala> Await.result(big(1234567), 20.seconds)
\end{REPL}



\SOLUTION


\TaskSolved \what
     %%%TODO number  4 %%%starts with: TODO  %%%%%%%%%%%%%%%%%%%\Advan%%%

\SubtaskSolved  error: Cannot find an implicit ExecutionContext. Future behöver en ExecutionContext för att kunna köras. \code{f} är av typen Future[Unit].

\SubtaskSolved  Funktionen \code{printLater} har en Future, vilket innebär att när både \code{printLater} och \code{println} anropas i foreach-loopen exekveras de jämnlöpande. Eftersom det tar längre tid att starta upp en Future för datorn är \code{println} snabbare och skriver ut att alla är igång först. Sedan skrivs siffrorna från 1 - 42 ut med oregelbundna mellanrum eftersom tråden pausas olika länge.

\SubtaskSolved  \code{big} är en Future[Int]. Det stora talet har 7 520 383 siffror. \code{r} är av typen Try[Int] (se dokumentationen för Future om du är osäker)

\SubtaskSolved  Eftersom exekveringen blockas tills den har fått ett resultat går det inte att fortsätta skriva i REPL medan uträkningen pågår. Väntar man för kort tid får man ett TimeOutException och uträkningen avbryts.


\QUESTEND






\WHAT{Använda \code{Future} för att göra flera saker samtidigt.}

\QUESTBEGIN

\Task  \what~
I denna uppgift ska du ladda ner webbsidor parallellt med hjälp av \code{Future}, så att en nedladdning kan avslutas under tiden en annan dröjer.

\Subtask Koden för en minimal webbsida ser ut som nedan. Du kan beskåda sidan här: \url{http://fileadmin.cs.lth.se/pgk/mini.html} eller skriva in nedan kod i en fil som heter något som slutar på \texttt{.html} och öppna filen i din webbläsare.

\begin{verbatim}
<!DOCTYPE html>
<html>
<body>
HELLO WORLD!
</body>
</html>
\end{verbatim}

\Subtask För att simulera slöa webbservrar kan man ladda ner en sida via sajten \\\texttt{http://deelay.me/} \\ Ladda ner ovan sida med 2 sekunders fördröjning:\\
\url{http://deelay.me/2000/http://fileadmin.cs.lth.se/pgk/mini.html}

\Subtask Man kan ladda ner webbsidor med \code{scala.io.Source}. Vad händer nedan? Försök, med ledning av hur \code{delay} beräknas, uppskatta hur lång tid du måste vänta i medeltal, i bästa fall, respektive värsta fall, innan du kan se första webbsidan i vektorn \code{laddningar} nedan?

\begin{REPL}
scala> def ladda(url: String) = scala.io.Source.fromURL(url).getLines.toVector
scala> def slöladda(url: String) = 
         val delay = (math.random() * 1000 + 2000).toInt
         val delaySite = s"http://deelay.me/$delay/"
         ladda(delaySite+url)
       end slöladda
scala> ladda("http://fileadmin.cs.lth.se/pgk/mini.html")
scala> def seg = slöladda("http://fileadmin.cs.lth.se/pgk/mini.html")
scala> val laddningar = Vector.fill(10)(seg)
scala> laddningar(0)
\end{REPL}

\Subtask Innan vi kan köra igång en \code{Future} så måste vi, som visats i uppgift \ref{task:future} importera den underliggande exekveringsmiljön som är redo att parallelisera ditt program i trådar utan att du själv måste skapa dem. Vad händer nedan?
\begin{REPL}
scala> import scala.concurrent._
scala> import ExecutionContext.Implicits.global
scala> val f = Future(seg)
scala> f   // kolla om den är klar annars prova igen senare
scala> f
\end{REPL}

\Subtask Ladda indata utan att blockera \Eng{non-blocking input}. Förklara vad som händer nedan.
\begin{REPL}
scala> val nonBlocking = Future(Vector.fill(10)(seg))
scala> nonBlocking   // kolla igen senare om ej klar
scala> nonBlocking
\end{REPL}

\Subtask Ladda indata separat i olika parallella trådar. Förklara vad som händer nedan. Kör uttrycket på rad 3 nedan upprepade gånger i snabb följd efter varandra med pil-upp+Enter i REPL.
\begin{REPL}
scala> val para = Vector.fill(10)(Future(seg))
scala> para
scala> para.map(_.isCompleted)
scala> para.map(_.isCompleted) // studera hur de blir färdiga en efter en
scala> para(0)
\end{REPL}

\Subtask Registrera en callback med metoden \code{onComplete}. Förklara vad som händer nedan.

\begin{REPL}
scala> val action = Vector.fill(10)(Future(seg))
scala> action(0).onComplete(xs => println(s"ready:$xs"))
scala> // vänta tills laddning på plats 0 är klar
\end{REPL}

\Subtask Registrera en callback för felhantering i händelse av undantag med metoden \code{onFailure}. Förklara vad som händer nedan.
\begin{REPL}
scala> def lycka  = { Thread.sleep(3000); println(":)") }
scala> def olycka = { Thread.sleep(3000); 42 / 0; lycka }
scala> Future(lycka ).onFailure{ case e => println(s":( $e") }
scala> Future(olycka).onFailure{ case e => println(s":( $e") }
\end{REPL}



\SOLUTION


\TaskSolved \what
     %%%TODO number  5 %%%starts with: Sök upp och studera dokumentati%%%

\SubtaskSolved  -

\SubtaskSolved  -

\SubtaskSolved  Varje sida fördröjs med mellan 2 upp till 3 sekunder (2000-3000 millisekunder). Så i medeltal tar det 2.5 sekunder för varje sida att laddas. Vektorn måste fyllas innan exekveringen kan fortsätta. Därför laddas alla 10 stycken sidor in innan man kan se första websidan. Det tar därför i medeltal 2.5 x 10 = 25 sekunder.

\SubtaskSolved  \code{f} ger en Vektor fylld med strängar där varje element ges av en rad på hemsidan. Då \code{f} körs i bakgrunden kan programmet fortlöpa medan innehållet räknas ut. Du kan därför skriva \code{f} i REPL:n men det är inte säkert att processen är klar och det slutgiltiga resultatet visas.

\SubtaskSolved  Samma som ovan, förutom att det blir en vektor där varje element är i sig en vektor med strängar.

\SubtaskSolved  Ladda data parallellt så att nedladdningen sker samtidigt, men det går olika snabbt pga metoden seg.

\SubtaskSolved  Eftersom datan laddas i parallella trådar utan blockering blir de inte klara i ordning, utan i den ordningen tråden körs klart. Till slut blir alla klara och resultatet visar en vektor med \code{true} värden.

\SubtaskSolved  Metoden \code{lycka} är väldefinerad och kastar därför inga undantag. Den skriver alltid ut \code{:)}. Metoden \code{olycka} är inte väldefinierad då division med 0 ger \\\code{java.lang.ArithmeticException}. Detta fångas upp vid callbacken och det skrivs ut \code{:(} samt det specificerade undantaget.

\ExtraTasks %%%%%%%%%%%%


\QUESTEND






\WHAT{}

\QUESTBEGIN

\Task  \what~ Räkna ut stora primtal parallellt genom att använda nedan funktioner. Implementera \code{isPrime} enligt pseudokod från den engelska wikipediasidan om primtalstest\footnote{\href{https://en.wikipedia.org/wiki/Primality_test}{en.wikipedia.org/wiki/Primality\_test}} med den s.k. ''naiva algoritmen''.  Räkna ut 10 st slumpvisa primtal med 16 siffror vardera. Gör beräkningarna parallellt med hjälp av \code{Future}.

\begin{Code}
def isPrime(n: BigInt): Boolean = ???

def nextPrime(start: BigInt): BigInt = 
  var i = start
  while !isPrime(i) do i += 1 
  i
end nextPrime

def randomBigInt(nDigits: Int): BigInt = 
   def rndChar = ('0' + (math.random() * 10).toInt).toChar
   val str = Array.fill(nDigits)(rndChar).mkString
   BigInt(str)
randomBigInt
\end{Code}

\SOLUTION


\TaskSolved \what
  %%%TODO number  6 %%%

\begin{Code}
def isPrime(n: BigInt): Boolean = n match 
  case _ if (n <= 1) => false
  case _ if (n <= 3) => true
  case _ if n % 2 == 0 || n % 3 == 0 => false
  case _ =>
    var i = BigInt(5)
    while i * i < n do
      if (n % i == 0 || n % (i + 2) == 0) false
      i += 6
    end while
    true
end isPrime

import scala.concurrent.*
import ExecutionContext.Implicits.global

val primes = Vector.fill(10)(Future{nextPrime(randomBigInt(16))})
primes.foreach(_.onSuccess{case i => println(i)})
\end{Code}


\QUESTEND






\WHAT{Svara på teorifrågor.}

\QUESTBEGIN

\Task  \what~\Pen

\Subtask Vad är en tråd?

\Subtask Hur skapar man en tråd med klassen \code{Thread}?

\Subtask Hur startar man en tråd?

\Subtask Vilka problem kan man råka ut för om man uppdaterar samma resurs i flera olika trådar?

\Subtask Vad innbär det att kod är \emph{trådsäker}?

\Subtask Nämn några fördelar med att använda Future jämfört med att använda trådar direkt.


\SOLUTION


\TaskSolved \what
 %%%TODO number  7 %%%

\SubtaskSolved  Stackoverflow ger följande förklaring:

A thread is an independent set of values for the processor registers (for a single core). Since this includes the Instruction Pointer (aka Program Counter), it controls what executes in what order. It also includes the Stack Pointer, which had better point to a unique area of memory for each thread or else they will interfere with each other.

\SubtaskSolved

\begin{Code}
val thread = new Thread(new Runnable{
	def run(){println(''Det här är en tråd'')}
})
\end{Code}

\SubtaskSolved  \code{thread.start}

\SubtaskSolved  Det kan bli kapplöpning(race conditions) om vilken tråds resurser blir sparade. Vilket leder till att de andra trådarnas ändringar blir ignorerade.

\SubtaskSolved  Trådsäkerhet innebär att flera trådar kan köras parallellt utan felaktigheter i resultatet. Exempelvis får man vara väldigt försiktig om man vill ha en muterbar variabel som alla trådar ska ändra samtidigt.

\SubtaskSolved  Till exempel slipper man skapa instanser av klassen Thread eftersom man kan placera koden i en Future istället. Den löser även mycket under huven för kodaren.


\QUESTEND






\WHAT{Klasser med atomär uppdatering.}

\QUESTBEGIN

\Task  \what~ Läs om och testa klasserna AtomicBoolean, AtomicDouble och AtomicReference för atomär uppdatering i paketet \\ \code{java.util.concurrent.atomic}.

Använd några av dessa tillsammans med \code{scala.concurrent.Future}.


\SOLUTION

\TaskSolved --

\QUESTEND





\WHAT{Skapa din egen multitrådade webbserver.}

\QUESTBEGIN

\Task  \what~

\Subtask Skriv in\footnote{Eller ladda ner här: \href{https://github.com/lunduniversity/introprog/blob/master/compendium/examples/simple-web-server/webserver.scala}{github.com/lunduniversity/introprog/blob/master/compendium/examples/simple-web-server/webserver.scala}} nedan kod i en editor och spara i en fil med namn \texttt{webserver.scala} och kompilera och kör med \texttt{scala-cli run webserver.scala} och beskriv vad som händer när du med din webbläsare surfar till adressen: \\ \url{http://localhost:8089/abbasillen}

\scalainputlisting[numbers=left,basicstyle=\ttfamily\fontsize{11}{12}\selectfont]{examples/simple-web-server/webserver.scala}

\Subtask Du ska nu skapa en webbserver som gör något lite mer intressant. Den ska svara med det 13:e Fibonacci-talet\footnote{\href{https://sv.wikipedia.org/wiki/Fibonaccital}{https://sv.wikipedia.org/wiki/Fibonaccital}} om du surfar till \url{http://localhost:8089/fib/13}.
Spara din webbserver från föregående deluppgift under det nya namnet \texttt{fibserver.scala} och använd koden nedan och lägg till och ändra så att din server kan svara med Fibonaccital. Vi börjar med att räkna ut Fibonaccital i funktionen \code{compute.fib} nedan på ett onödigt processorkrävande sätt med exponentiell tidskomplexitet så att webbservern verkligen får jobba, för att i senare deluppgifter implementera \code{compute.fib} med linjär tidskomplexitet och därmed undvika onödig planetuppvärmning.
\begin{CodeSmall}
// lägg till nedan i webserver.scala från 
//    https://github.com/lunduniversity/introprog/blob/master/compendium/examples/simple-web-server/webserver.scala

object compute:
  def fib(n: BigInt): BigInt = 
    if n < 0 then 0 else
    if n == 1 || n == 2 then 1
    else fib(n - 1) + fib(n -2)
  end fib
end compute

def fibResponse(num: String) = 
  num.toIntOption match 
    case Some(n) => html.page(s"fib($n) == " + compute.fib(n))
    case None    => html.page(s"FEL: skriv ett heltal, inte $num")

def errorResponse(uri:String) = html.page(s"Error: $uri </br> use /fib/heltal")


// ändra handleRequest i start i webserver.scala till
  def handleRequest(cmd: String, uri: String, socket: Socket): Unit = 
    val os = socket.getOutputStream
    val afterSlash = uri.toString.drop(1) // skip initial slash
    println(s"afterSlash:$afterSlash")
    val response: String = 
      if afterSlash.startsWith("fib/") then fibResponse(afterSlash.stripPrefix("fib/"))
      else errorResponse(uri)
    os.write(html.header(response.size).getBytes("UTF-8"))
    os.write(response.getBytes("UTF-8"))
    os.close
    socket.close
  end handleRequest
\end{CodeSmall}

Kör i terminalen med \texttt{scala-cli run webserver.scala} och beskriv vad som händer i din webbläsare när du surfar till servern.


%%%\textbf{KOD TILL FACIT:}
%%%\scalainputlisting[numbers=left,basicstyle=\ttfamily\fontsize{11}{12}\selectfont]{examples/simple-web-server/fibserver.scala}


\Subtask Surfa efter flera stora Fibonacci-tal samtidigt i olika flikar i din browser. Hur märks det att servern bara kör i en enda tråd?

\Subtask Gör din server multitrådad med hjälp av den nya server-loopen nedan.

\begin{CodeSmall}
import scala.concurrent._
import ExecutionContext.Implicits.global

  def serverLoop(server: ServerSocket): Unit = {
    println(s"http://localhost:${server.getLocalPort}/hej")
		while (true) {
  		Try {
  		  var socket = server.accept  // blocks thread until connect
	  	  val scan = new Scanner(socket.getInputStream, "UTF-8")
		    val (cmd, uri) = (scan.next, scan.next)
			  println(s"Request: $cmd $uri")
		    Future { handleRequest(cmd, uri, socket) }.onFailure {
		      case e => println(s"Reqest failed: $e")
		    }
		  }.recover{ case e: Throwable => s"Connection failed: $e" }
		}
  }
\end{CodeSmall}

\Subtask Surfa efter flera stora Fibonacci-tal samtidigt i olika flikar i din browser. Hur märks det att servern är multitrådad?


\Subtask Det är onödigt att räkna ut samma Fibonacci-tal flera gånger. Med hjälp av en cache i form av en föränderlig \code{Map} kan du spara undan redan uträknade värden. Det funkar dock inte med en vanlig \code{scala.collection.mutable.Map} i vår multitrådade webbserver, eftersom den inte är \textbf{trådsäker} \Eng{thread-safe}. Med trådosäkra föränderliga datastrukturer blir det samma besvärliga beteende som i uppgift \ref{task:racecondition}.

Du ska i stället använda \code{java.util.concurrent.ConcurrentHashMap}. Sök upp  dokumentationen för \code{ConcurrentHashMap} och försök förstå koden nedan. Hur fungerar metoderna \code{containsKey}, \code{put} och \code{get}?
\begin{Code}
object compute {
  import java.util.concurrent.ConcurrentHashMap
  val memcache = new ConcurrentHashMap[BigInt, BigInt]

  def fib(n: BigInt): BigInt =
    if (memcache.containsKey(n)) {
      println("CACHE HIT!!! no need to compute: " + n)
      memcache.get(n)
    } else {
      println("cache miss :( must compute fib:  " + n)
      val f = fastFib(n)
      memcache.put(n, f)
      f
    }

  private def fastFib(n: BigInt): BigInt = {
    if (n < 0) 0 else
    if (n == 1 || n == 2) 1
    else fib(n - 1) + fib(n -2)
  }
}
\end{Code}

\Subtask Använd ovan \code{fib}-objekt i en ny version av din webserver. Spara den i en ny kodfil med namnet \texttt{fibserver-memcached.scala}. Undersök hur snabbt det går med stora Fibonaccital med den nya varianten. Hur stora tal kan du räkna ut? Kan servern fortsätta efter överflödad stack? Förklara varför.

\Subtask Nu när vi kan få väldigt stora Fibonacci-tal kan det vara användbart att stoppa in radbrytningar på webbsidan. Html-taggen \texttt{</br>} ger en radbrytning.
\begin{Code}
  def insertBreak(s: String, n: Int = 80): String = {
    if (s.size < n) s
    else s.take(n) + "</br>" + insertBreak(s.drop(n),n)
  }
\end{Code}
Använd den rekursiva funktionen ovan för att pilla in radbrytningstaggar på var $n$:te position i långa strängar. Testa hur det ser ut på webbsidan med ovan funktion när din server svarar med väldigt stora tal.

\Subtask Vi ska nu använda det större heap-minnet i stället för stack-minnet och därmed inte begränsas av stackens max-storlek. Skriv om \code{fastFib} så att den använder en \code{while}-sats i stället för ett rekursivt anrop. Denna uppgift är ganska klurig, men om du kör fast kan du snegla i lösningarna i Appendix för inspiration.

Hur stora tal klarar din server nu? Vad händer med servern när minnet tar slut? Hur kan du skydda servern så att den inte kan hänga sig?

\SOLUTION


\TaskSolved \what
 %%%TODO number  9 %%%

\SubtaskSolved  \code{abbasillen} skrivs ut baklänges till \code{nellisabba}.

\SubtaskSolved

\SubtaskSolved

\SubtaskSolved

\SubtaskSolved

\SubtaskSolved

\SubtaskSolved

\SubtaskSolved

\SubtaskSolved

Lösningsförslag:
\scalainputlisting[numbers=left,basicstyle=\ttfamily\fontsize{11}{12}\selectfont]{examples/simple-web-server/fibserver-threaded-memcached-while.scala}


\QUESTEND






% \WHAT{}

% \QUESTBEGIN

% \Task  \what~ Utöka din server med fler beräkningsintensiva funktioner. Exempelvis primtalsberäkningar eller beräkningar av valfritt antal decimaler av $\pi$ eller $e$. Utnyttja gärna det du lärt dig i  matematiken om summor och serieutvecklingar.

% \SOLUTION


% \TaskSolved \what
%  %%%TODO number  10 %%%

% ---


% \QUESTEND






% \WHAT{}

% \QUESTBEGIN

% \Task  \what~ Läs mer om \code{Future} och jämlöpande exekvering i Scala här:\\
% \href{http://alvinalexander.com/scala/future-example-scala-cookbook-oncomplete-callback}{alvinalexander.com/scala/future-example-scala-cookbook-oncomplete-callback}

% \SOLUTION


% \TaskSolved \what
%  %%%TODO number  11 %%%

% ---


% \QUESTEND






% \WHAT{}

% \QUESTBEGIN

% \Task  \what~ Läs mer om jämlöpande exekvering och multitrådade program i Java här: \href{http://www.tutorialspoint.com/java/java_multithreading.htm}{www.tutorialspoint.com/java/java\_multithreading.htm}  \\
% \noindent När man skriver program med jämlöpande exekvering finns det många fallgropar; det kan bli kapplöpning \Eng{race conditions} om gemensamma resurser och dödläge \Eng{deadlock} där inget händer för att trådar väntar på varandra. Mer om detta i senare kurser.


% \SOLUTION


% \TaskSolved \what
%  %%%TODO number  12 %%%

% ---


% \QUESTEND






% \WHAT{Studera dokumentationen i \code{scala.concurrent}.}

% \QUESTBEGIN

% \Task  \what~\Pen

% \Subtask Studera dokumentationen för \code{scala.concurrent.Future}\footnote{\href{http://www.scala-lang.org/api/current/scala/concurrent/Future.html}{http://www.scala-lang.org/api/current/scala/concurrent/Future.html}}. Hur samverkar \code{Future} med \code{Try} och \code{Option}? Vilka vanliga samlingsmetoder känner du igen?

% \Subtask Studera dokumentationen för \code{scala.concurrent.duration.Duration}\footnote{\href{http://www.scala-lang.org/api/current/scala/concurrent/duration/Duration.html}{www.scala-lang.org/api/current/scala/concurrent/duration/Duration.html}}. Vilka tidsenheter kan användas?

% \Subtask Vid import av \code{scala.concurrent.duration.* } dekoreras de numeriska klasserna med metoder för att skapa instanser av klassen \code{Duration}. Detta möjligörs med hjälp av klassen \code{scala.concurrent.duration.DurationConversions}. Studera dess dokumentation och testa att i REPL skapa några tidsperioder med metoderna på \code{Int}.



% \SOLUTION


% \TaskSolved \what
%  %%%TODO number  13 %%%

% \SubtaskSolved

% \SubtaskSolved

% \SubtaskSolved


% \QUESTEND






% \WHAT{}

% \QUESTBEGIN

% \Task  \what~ Fördjupa dig inom webbteknologi.

% \Subtask Lär dig om HTML, CSS och JavaScript här: \url{https://developer.mozilla.org/en-US/docs/Learn}

% \Subtask Lär dig om Scala.JS här: \url{http://www.scala-js.org/}\SOLUTION


% \TaskSolved \what
%  %%%TODO number  14 %%%

% \SubtaskSolved  ---

% \SubtaskSolved  ---

% \SubtaskSolved  ---

% \SubtaskSolved  ---
% \QUESTEND

\input{modules/w13-examprep-exercise.tex}
\input{modules/w14-extra-exercise.tex}

%!TEX encoding = UTF-8 Unicode
%!TEX root = ../compendium.tex

\ifPreSolution

\Exercise{java}\label{exe:java}

\begin{Goals}
\item Kunna förklara och beskriva viktiga skillnader mellan Scala och Java.
\item Kunna översätta enkla algoritmer, klasser och singeltonobjekt från Scala till Java och vice versa.
\item Känna till vad en case-klass innehåller i termer av en Javaklass.
%\item Förstå hur autoboxing fungerar.
\item Kunna använda Javatyperna \code{List}, \code{ArrayList}, \code{Set}, \code{HashSet} och översätta till deras Scalamotsvarigheter med \code{CollectionConverters}.
\item Kunna förklara hur autoboxning fungerar i Java, samt beskriva fördelar och fallgropar.
\end{Goals}

\begin{Preparations}
\item Studera teori i början av detta Appendix.
\end{Preparations}

\BasicTasks %%%%%%%%%%%%%%%%

\else

\ExerciseSolution{java}

\BasicTasks %%%%%%%%%%%

\fi





\WHAT{Översätta metoder från Java till Scala.}

\QUESTBEGIN

\Task  \what~  I denna uppgift ska du översätta en Java-klass som används som en modul\footnote{\href{https://en.wikipedia.org/wiki/Modular_programming}{en.wikipedia.org/wiki/Modular\_programming}} och bara innehåller statiska metoder och inget förändringsbart tillstånd som kan ändras utifrån. (I nästa uppgift ska du sedan översätta klasser med förändringsbara  tillstånd.)

Vi börjar med att göra översättningen från Java till Scala rad för rad och du ska behålla så mycket som möjligt av syntax och semantik så att Scala-koden blir så Java-lik som möjligt. I efterföljande deluppgift ska du sedan omforma översättningen så att Scala-koden blir mer idiomatisk\footnote{\href{https://sv.wikipedia.org/wiki/Idiom_\%28programmering\%29}{sv.wikipedia.org/wiki/Idiom\_\%28programmering\%29}}.

\Subtask Studera klassen \code{Hangman} nedan. Du ska översätta den från Java till Scala enlig de riktlinjer och tips som följer efter koden. Läs igenom alla riktlinjer och tips innan du börjar.

\javainputlisting[numbers=left]{examples/scalajava/Hangman.java}

\noindent\emph{Riktlinjer och tips för översättningen:}

\begin{enumerate}[noitemsep]

\item Skriv Scala-koden med en texteditor i en fil som heter \texttt{hangman1.scala} och kompilera.

\item Översätt i denna första deluppgift rad för rad så likt den ursprungliga Java-kodens utseende (syntax)  som möjligt, med så få ändringar som möjligt. Du ska alltså ha kvar dessa Scalaovanligheter, även om det inte alls blir som man brukar skriva i Scala:
\begin{enumerate}[nolistsep, noitemsep]
\item långa indrag, \item onödiga semikolon, \item onödiga \code{()}, \item onödiga \code|{}|, \item onödiga \code{System.out}, och \item onödiga \code{return}.
\end{enumerate}

\item Försök också i denna deluppgift göra så att betydelsen (semantiken) så långt som möjligt motsvarar den i Java, t.ex. genom att använda \code{var} överallt, även där man i Scala normalt använder \code{val}.

\item En Javaklass med bara statiska medlemmar motsvarar ett singeltonobjekt i Scala, alltså en \code{object}-deklaration innehållande ''vanliga'' medlemmar.

\item För att tydliggöra att du använder Javas \code{Set} och \code{HashSet} i din Scala-kod, använd följande import-satser i \code{hangman1.scala}, som därmed döper om dina importerade namn och gör så att de inte krockar med Scalas inbyggda \code{Set}. Denna form av import går inte att göra i Java.
\begin{Code}
import java.util.{Set => JSet};
import java.util.{HashSet => JHashSet};
\end{Code}

\item Javas \code{i++} fungerar inte i Scala; man får istället skriva \code{i += 1} eller mindre vanliga \code{i = i + 1}.

\item Typparametrar i Java skrivs inom \code{<>} medan Scalas syntax för typparametrar använder \code{[]}.

\item Till skillnad från Java så har Scalas metoddeklarationer ett tilldelningstecken \code{=} efter returtypen, före kroppen.

\item Du kan ladda ner Java-koden till \code{Hangman}-klassen nedan från kursens repo%
\footnote{\href{https://github.com/lunduniversity/introprog/blob/master/compendium/examples/scalajava/Hangman.java}{github.com/lunduniversity/introprog/blob/master/compendium/examples/scalajava/Hangman.java}}. I samma bibliotek ligger även lösningarna till översättningen i Scala, men kolla \emph{inte} på dessa förrän du gjort klart översättningarna och fått dem att kompilera och köra felfritt! Tanken är att du ska träna på att läsa felmeddelande från kompilatorn och åtgärda dem i en upprepad kompilera-testa-rätta-cykel.

\end{enumerate}







\Subtask Skapa en ny fil \code{hangman2.scala} som till att börja med innehåller en kopia av din direkt-översatta Java-kod från föregående deluppgift. Omforma koden så att den blir mer som man brukar skriva i Scala, alltså mer Scala-idiomatisk. Försök förenkla och förkorta så mycket du kan utan att göra avkall på läsbarheten.

\emph{Tips och riktlinjer:}

\begin{enumerate}[nolistsep, noitemsep]

\item Kalla Scala-objektet för \code{hangman}. När man använder ett Scalaobjekt som en modul (alltså en samling funktioner i en gemensam, avgränsad namnrymd) har man gärna liten begynnelsebokstav, i likhet med konventionen för paketnamn. Ett paket är ju också en slags modul och med en namngivningskonvention som är gemensam kan man senare, utan att behöva ändra koden som använder modulen, ändra från ett singelobjekt till ett paket och vice versa om man så önskar.

\item Gör alla metoder publikt tillgängliga och låt även strängvektorn \code{hangman} vara publikt tillgänglig. Deklarera \code{hangman} som en \code{val} och konstruera den med \code{Vector}. Eftersom \code{Vector} är oföränderlig och man inte kan ärva från singelobjekt och \code{hangman} är deklarerad med \code{val} finns inga speciella risker med att göra den konstanta vektorn publik om  vi inte har något emot att annan kod kan läsa (och eventuellt göra sig beroende av) vår hänggubbetext.

\item I metoden \code{renderHangman}, använd \code{take} och \code{mkString}.

\item I metoden \code{hideSecret}, använd \code{map} i stället för en \code{for}-sats.

\item Det går att ersätta metoden \code{foundAll} med det kärnfulla uttrycket \\ \code{(secret forall found)} där \code{secret} är en sträng och \code{found} är en mängd av tecken (undersök gärna i REPL hur detta fungerar). Skippa därför den metoden helt och använd det kortare uttrycket direkt.

\item I metoden \code{makeGuess}, i stället för \code{Scanner}, använd \code{scala.io.StdIn.readLine}.

\item Om du vill träna på att använda rekursion i stället för imperativa loopar: Gör metoden \code{makeGuess} rekursiv i stället för att använda \code{do}-\code{while}.

\item I metoden \code{download}, i stället för \code{java.net.URL} och \code{java.util.ArrayList}, använd \code{scala.io.Source.fromURL(address, coding).getLines.toVector} och gör en lokal import av \code{scala.io.Source.fromURL} överst i det block där den används. Det går inte att ha lokala \code{import}-satser i Java.

\item Låt metoden \code{download} returnera en \code{Option[String]} som i fallet att nedladdningen misslyckas returnerar \code{None}.

\item I metoden \code{download}, i stället för \code{try}-\code{catch} använd \code{scala.util.Try} och dess smidiga metod \code{toOption}.

\item Om du vill träna på att använda rekursion i stället för imperativa loopar: Använd, i stället för \code{while}-satsen i metoden \code{play}, en lokal rekursiv funktion med denna signatur:
\begin{Code}
  def loop(found: Set[Char], bad: Int): (Int, Boolean)
\end{Code}
Funktionen \code{loop} returnerar en 2-tupel med antalet felgissningar och \code{true} om man hittat alla bokstäver eller \code{false} om man blev hängd.

\end{enumerate}





\SOLUTION


\TaskSolved \what
     %%%TODO number  1 %%%starts with: \emph{Översätta algoritmer och %%%

\SubtaskSolved  \scalainputlisting[numbers=left,basicstyle=\ttfamily\fontsize{10.3}{12}\selectfont]{examples/scalajava/hangman1.scala}

\SubtaskSolved  \scalainputlisting[numbers=left,basicstyle=\ttfamily\fontsize{11.2}{13}\selectfont]{examples/scalajava/hangman2.scala}



\QUESTEND






\WHAT{Översätta mellan klasser i Scala och klasser i Java.}

\QUESTBEGIN

\Task  \what~
Klassen \code{Point} nedan är en modell av en punkt som kan sparas på begäran i en lista. Listan är privat för kompanjonsobjektet och kan skrivas ut med en metod \code{showSaved}. I koden används en \code{ArrayBuffer}, men i framtiden vill man, vid behov, kunna ändra från \code{ArrayBuffer} till en annan sekvenssamlingsimplementation, t.ex. \code{ListBuffer}, som uppfyller egenskaperna hos supertypen \code{Buffer}, men har andra prestandaegenskaper för olika operationer. Därför är attributet \code{saved} i kompanjonsobjektet deklarerat med den mer generella typen.

\scalainputlisting[numbers=left]{examples/scalajava/Point.scala}

\Subtask Översätt klassen \code{Point} ovan från Scala till Java. Vi ska i nästa deluppgift kompilera både Scala-programmet ovan och ditt motsvarande Java-program i terminalen och testa i REPL att klasserna har motsvarande funktionalitet.

\emph{Tips och riktlinjer:}
\begin{enumerate}[nolistsep, noitemsep]
\item För att namnen inte ska krocka i våra kommande tester, kalla Javatypen för \code{JPoint}.
\item  I stället för Scalas \code{ArrayBuffer} och \code{Buffer}, använd Javas \code{ArrayList} och \code{List} som båda ligger i paketet \code{java.util}.
\item Undersök dokumentationen för \code{java.util.List} för att hitta en motsvarighet till \code{prepend} för att lägga till i början av listan.
\item I stället för default-argumentet i Scalas primärkonstruktor, använd en extra Java-konstruktor.
\item Det finns inga singelobjekt och inga kompanjonsobjekt i Java; istället kan man använda statiska klassmedlemmar. Placera kompanjonsobjektets medlemmars motsvarigheter \emph{inuti} Java-klassen och gör dem till \jcode{static}-medlemmar.
\item Kod i klasskroppen i Scalaklassen, så som if-satsen på rad 4, placeras i lämplig konstruktor i Javaklassen.
\item Utskrifter med \code{print} och \code{println} behöver i Java föregås av \code{System.out}.
\item Det finns inget nyckelord \code{override} i Java, men en s.k. annotering som ger samma kompilatorhjälp. Den skrivs med ett snabel-a och stor begynnelsebokstav, så här: \jcode{ @Override }  före metoddeklarationen.
\item I Java används konventionen att börja getter-metoder med ordet \code{get}, t.ex. \code{getX()}.
\item Det finns ingen motsvarighet till \code{mkString} för \code{List} så du behöver själv gå igenom listan och hämta elementreferenser för utskrift med en \jcode{for}-loop. Notera att efter sista elementet ska radbrytning göras i utskriften och att inget komma ska skrivas ut efter sista elementet.
\item I Java behövs en ny \jcode{import}-deklaration om man vill importera ännu en typ från samma paket. Man kan även i Java använda asterisk \code{*}, (motsvarande \code{_} i Scala), för att importera allt i ett paket, men då får man med alla möjliga namn och det vill man kanske inte.
\item Metoder i Java slutar med \code{()} om de saknar parametrar.
\item Alla satser i Java slutar med lättglömda semikolon. (Efter att man i skrivit mycket Javakod och växlar till Scalakod är det svårt att vänja sig av med att skriva semikolon...)
\end{enumerate}


\Subtask Starta REPL i samma bibliotek som du kompilerat kodfilerna. Testa så att klasserna \code{Point} och \code{JPoint} beter sig på samma vis enligt nedan. Skriv även testkod i REPL för att avläsa de attributvärden som har getters och undersök att allt funkar som det ska.
\begin{REPLnonum}
> scalac Point.scala
> javac JPoint.java
> scala
scala> val (p, jp) = (new Point, new JPoint)
scala> p.distanceTo(new Point(3, 4))
scala> Point.showSaved
scala> jp.distanceTo(new JPoint(3, 4))
scala> JPoint.showSaved
scala> for (i <- 1 to 10) { new Point(i, i, true) }
scala> Point.showSaved
scala> for (i <- 1 to 10) { new JPoint(i, i, true) }
scala> JPoint.showSaved
\end{REPLnonum}


\Subtask Översätt nedan Javaklass \code{JPerson} till en \code{case class Person} i Scala med  motsvarande funktionalitet.


\javainputlisting[numbers=left]{examples/scalajava/JPerson.java}


\Subtask\Pen Undersök i REPL vilken funktionalitet i Scala-case-klassen \code{Person} som \emph{inte} är implementerad i Java-klassen \code{JPerson} ovan. Skriv upp namnen på några av case-klassens extra metoder samt deras signatur genom att för en \code{Person}-instans, och för kompanjonsobjektet \code{Person}, trycka på TAB-tangenten. Prova några av de extra metoderna i REPL och förklara vad de gör.

\begin{REPL}
scala> val p = Person("Björn", 49)
scala> p.      // tryck TAB en gång
scala> Person. // tryck TAB en gång
scala> p.copy  // tryck TAB en gång
scala> p.copy()
scala> p.copy(age = p.age + 1)
scala> Person.unapply(p)
\end{REPL}


\SOLUTION


\TaskSolved \what
     %%%TODO number  2 %%%starts with: \emph{Översätta mellan klasser %%%

\SubtaskSolved   \javainputlisting[numbers=left]{examples/scalajava/JPoint.java}

\SubtaskSolved   -

\SubtaskSolved   \begin{Code}
case class Person(name: String, age: Int = 0)
\end{Code}

\SubtaskSolved  p.*TAB* - copy, producArity, ProductIterator, productElement, productPrefix

Person.*TAB* - apply, curried, tupled, unapply

\begin{REPLnonum}
scala> p.copy
   def copy(name: String,age: Int): Person

scala> p.copy()
res0: Person = Person(Björn,49)

scala> p.copy(age = p.age + 1)
res1: Person = Person(Björn,50)

scala> Person.unapply(p)
res2: Option[(String, Int)] = Some((Björn,49))
\end{REPLnonum}



\QUESTEND



\WHAT{Oföränderlig Java-klass.}

\QUESTBEGIN

\Task \what~Översätt nedan Scala-klass till Java-klassen \code{JPoint3D}. Alla attribut ska vara privata (varför?). Översätt defaultargumentet till en alternativ konstruktor. Kalla getters för t.ex. \jcode{getX()}. Kör \code{javac} och testa i REPL.

\begin{Code}
class Point3D(val x: Int, val y: Int, val z: Int = 0)
\end{Code}

\SOLUTION

\TaskSolved \what~

\javainputlisting[numbers=left]{examples/JPoint3D.java}

\begin{REPL}
> code JPoint3D.java
> javac JPoint3D.java
> ls
JPoint3D.class  JPoint3D.java
> scala

scala> val p = new JPoint3D(1,2)
val p: JPoint3D = JPoint3D@53b1a3f8

scala> p.x
1 |p.x
  |^^^
  |value x is not a member of JPoint3D

scala> p.getX
val res0: Int = 1
\end{REPL}

\QUESTEND



\WHAT{Förändringsbar Java-klass.}

\QUESTBEGIN

\Task \what~\\Översätt nedan Scala-klass till Java-klassen \code{JMutablePoint3D}. Alla attribut ska vara privata (varför?). Översätt defaultargumentet till en alternativ konstruktor. Kalla setters för t.ex. \jcode{setX}. Kör \code{javac} och testa i REPL.

\begin{Code}
class MutablePoint3D(var x: Int, var y: Int, var z: Int = 0)
\end{Code}

\SOLUTION

\TaskSolved \what~

\javainputlisting[numbers=left]{examples/JMutablePoint3D.java}

\begin{REPL}
> code JMutablePoint3D.java
> javac JMutablePoint3D.java
> ls
JMutablePoint3D.class  JMutablePoint3D.java
> scala

scala> val p = new JMutablePoint3D(1,2)
val p: JMutablePoint3D = JMutablePoint3D@625b215b

scala> p.x
1 |p.x
  |^^^
  |value x is not a member of JMutablePoint3D

scala> p.getZ
val res0: Int = 0

scala> p.setZ(3)

scala> p.getZ
val res1: Int = 3
\end{REPL}

\QUESTEND






\WHAT{Jämföra strängar i Java.}

\QUESTBEGIN

\Task  \what~  I Java kan man \textbf{inte} jämföra strängar med operatorerna \code{<}, \code{<=}, \code{>}, och \code{>=}. Dessutom ger operatorerna \code{==} och \code{!=} inte innehålls(o)likhet utan referens(o)likhet. Istället får man använda metoderna \code{equals} och \code{compareTo}, vilka också fungerar i Scala eftersom strängar i Scala och Java är av samma typ, nämligen \code{java.lang.String}.


\Subtask Vad ger följande uttryck för värde?

\begin{REPL}
scala> "hej".getClass.getTypeName
scala> "hej".equals("hej")
scala> "hej".compareTo("hej")
\end{REPL}


\Subtask Studera dokumentationen för metoden \code{compareTo} i \code{java.lang.String}\footnote{\href{https://docs.oracle.com/javase/8/docs/api/java/lang/String.html\#compareTo-java.lang.String-}{docs.oracle.com/javase/8/docs/api/java/lang/String.html\#compareTo-java.lang.String-}} och skriv minst 3 olika uttryck i Scala REPL som testar hur metoden fungerar i olika fall.

\Subtask Studera dokumentationen \code{compareToIgnoreCase} \footnote{\href{https://docs.oracle.com/javase/8/docs/api/java/lang/String.html\#compareToIgnoreCase-java.lang.String-}{docs.oracle.com/javase/8/docs/api/java/lang/String.html\#compareToIgnoreCase-java.lang.String-}} och skriv minst 3 olika stränguttryck i Scala REPL som testar hur metoden fungerar i olika fall.

\Subtask Vad skriver följande Java-program ut?
\javainputlisting{examples/StringEqTest.java}


\SOLUTION


\TaskSolved \what

\SubtaskSolved
\begin{REPL}
String = java.lang.String
Boolean = true
Int = 0
\end{REPL}

\SubtaskSolved
Exempel på 3 olika uttryck för att testa \code{compareTo}:

\begin{enumerate}
\item
Hej kommer först då \code{H < h}.
\begin{REPLnonum}
	"hej".compareTo("Hej")
	res: Int = 32
\end{REPLnonum}

\item
Dessa är ekvivalenta, så \code{compareTo} returnerar 0.
\begin{REPLnonum}
	"hej".compareTo("hej")
	res: Int = 0
\end{REPLnonum}

\item
\emph{h} kommer före \emph{ö}.
\begin{REPLnonum}
	"hej".compareTo("ö")
	res: Int = -142
\end{REPLnonum}
\end{enumerate}

\SubtaskSolved
Exempel på 3 olika uttryck för att testa \code{compareToIgnoreCase}:

\begin{enumerate}

\item
\begin{REPLnonum}
	"hej".compareToIgnoreCase("HEj")
	res: Int = 0
\end{REPLnonum}

\item
\begin{REPLnonum}
	"hej".compareToIgnoreCase("Ö")
	res: Int = -142
\end{REPLnonum}

\item
Samma som ovan, då Ö omvandlas till ö innan jämförelse.
 \begin{REPLnonum}
	"hej".compareToIgnoreCase("ö") \\ res: Int = -142
\end{REPLnonum}
\end{enumerate}

\SubtaskSolved
\begin{REPL}
false
true
0
\end{REPL}



\QUESTEND





\WHAT{Linjärsökning i Java.}

\QUESTBEGIN

\Task  \what~  Denna uppgift bygger vidare på uppgift \ref{task:arraymatrix-java} i kapitel \ref{chapter:W08}. Du ska göra en variant på linjärsökning som innebär att leta upp första yatzy-raden i en matris där varje rad innehåller utfallet av 5 tärningskast.

\Subtask Du ska lägga till metoderna \code{isYatzy} och \code{findFirstYatzyRow} i klassen \code{ArrayMatrix} i uppgift \ref{task:arraymatrix-java} i kapitel \ref{chapter:W08} enligt nedan skiss. Vi börjar med metoden  \code{isYatzy} i denna deluppgift (nästa deluppgift handlar om \code{findFirstYatzyRow}). OBS! Det finns en bugg i \code{isYatzy} -- rätta buggen och testa så att den fungerar.

\begin{Code}[language=Java]
    public static boolean isYatzy(int[] dice){ /* has one bug! */
        int col = 1;
        boolean allSimilar = true;
        while (col < dice.length && allSimilar) {
          allSimilar = dice[0] == dice[col];
        }
        return allSimilar;
    }

    /** Finds first yatzy row in m; returns -1 if not found */
    public static int findFirstYatzyRow(int[][] m){
        int row = 0;
        int result = -1;
        while (???) {
             /* linear search  */
        }
        return result;
    }
\end{Code}


\Subtask Implementera \code{findFirstYatzyRow}. Skapa först pseudo-kod för linjärsökningsalgoritmen innan du skriver implementationen i Java.
Testa ditt program genom att lägga till följande rader i huvudprogrammet.
Metoden \code{fillRnd} ingår i uppgift \ref{task:arraymatrix-java} i kapitel \ref{chapter:W08}.
\begin{Code}[language=Java]
        int[][] yss = new int[2500][5];
        fillRnd(yss, 6);
        int i = findFirstYatzyRow(yss);
        System.out.println("First Yatzy Index: " + i);
\end{Code}




\SOLUTION


\TaskSolved \what


\SubtaskSolved
\begin{Code}[language=Java]
public static boolean isYatzy(int[] dice){
    int col = 1;
    boolean allSimilar = true;
    while(col < dice.length && allSimilar){
        allSimilar = (dice[0] == dice[col]);
        col++; //denna raden saknades
    }
    return allSimilar;
}
\end{Code}

\SubtaskSolved

\begin{Code}[language=Java]
public static int findFirstYatzyRow(int[][] m){
    int row = 0;
    int result = -1;
    while(row < m.length){
        if(isYatzy(m[row])){
           result = row;
           break;
        }
        row++;
    }
    return result;
}
\end{Code}



\QUESTEND


\WHAT{Jämförelsestöd i Java.}

\QUESTBEGIN

\Task  \what~
Java har motsvarigheter till Scalas \code{Ordering} och \code{Ordered}, som heter \code{java.util.Comparator} och \code{java.lang.Comparable}. I själva verket så är Scalas \code{Ordering} en subtyp till Javas \code{Comparator}, medan Scalas \code{Ordered} är en subtyp till Javas \code{Comparable}.
\begin{itemize}[nolistsep, noitemsep]
\item Javas \code{Comparator} och Scalas \code{Ordering} används för att skapa fristående ordningar som kan jämföra \emph{två olika} objekt. I Scala kan dessa göras implicit tillgängliga. I Javas samlingsbibliotek skickas instanser av \code{Comparator} med som explicita argument.
\item Javas \code{Comparable} och Scalas \code{Ordered} används som supertyp för klasser som vill kunna jämföra ''sig själv'' med andra objekt och har \emph{en} naturlig ordningsdefinition.
\end{itemize}

\Subtask\Pen Sök upp dokumentationen för \code{java.util.Comparator}. Vilken abstrakt metod måste implementeras och vad gör den?

\Subtask  I paketet \code{java.util.Arrays} finns en metod \code{sort} som tar en \code{Array[T]} och en \code{Comparable[T]}. Testa att använda dessa i REPL enligt nedan skiss. Starta om REPL så att ev. tidigare implicita ordningar för \code{Team} inte finns kvar.
\begin{REPL}
scala> import java.util.Comparator
scala> val teamComparator = new Comparator[Team]{
         def compare(o1: Team, o2: Team) = ???
       }
scala> val xs =
         Array(Team("fnatic", 1499), Team("nip", 1473), Team("lumi", 1601))
scala> java.util.Arrays.sort(xs.toArray, teamComparator)
scala> xs
\end{REPL}
%\begin{Code}
%// kod till facit
%val teamComparator = new Comparator[Team]{
%  def compare(o1: Team, o2: Team) = o2.rank - o1.rank
%}
%\end{Code}

\Subtask I Scala finns en behändig metod \code{Ordering.comparatorToOrdering} som skapar en implicit tillgänglig ordning om man har en \code{java.util.Comparator}. Testa detta enligt nedan i REPL, med deklarationerna från föregående deluppgift.
\begin{REPL}
scala> implicit val teamOrd = Ordering.comparatorToOrdering(teamComparator)
scala> xs.sorted
\end{REPL}



\Subtask\Pen Sök upp dokumentationen för \code{java.lang.Comparable}. Vilken abstrakt metod måste implementeras och vad gör den?

\Subtask Gör så att klassen \code{Point} är \code{Comparable} och att punkter närmare origo sorteras före punkter som är längre ifrån origo enligt nedan skiss. I Scala är typer som är \code{Comparable} implicit även \code{Ordered}, varför sorteringen nedan funkar. Verfiera detta i REPL när du klurat ut hur implementera \code{compareTo}.

\begin{Code}
case class Point(x: Int, y: Int) extends Comparable[Point] {
  def distanceFromOrigin: Double = ???
  def compareTo(that: Point): Int = ???
}
\end{Code}
\begin{REPL}
scala> val xs = Seq(Point(10,10), Point(2,1), Point(5,3), Point(0,0))
scala> xs.sorted
\end{REPL}
%\begin{Code}
%// kod till facit
%case class Point(x: Int, y: Int) extends Comparable[Point] {
%  def distanceFromOrigin: Double = math.hypot(x, y)
%  def compareTo(that: Point): Int =
%    (distanceFromOrigin - that.distanceFromOrigin).round.toInt
%}
%\end{Code}


\SOLUTION


\TaskSolved \what


\SubtaskSolved

\SubtaskSolved  %% b
\begin{Code}
val teamComparator = new Comparator[Team]{
  def compare(o1: Team, o2: Team) = o2.rank - o1.rank
}
\end{Code}


\SubtaskSolved

\SubtaskSolved

\SubtaskSolved

\begin{Code}
case class Point(x: Int, y: Int) extends Comparable[Point] {
  def distanceFromOrigin: Double = math.hypot(x, y)
  def compareTo(that: Point): Int =
    (distanceFromOrigin - that.distanceFromOrigin).round.toInt
}
\end{Code}


\QUESTEND




\WHAT{\texttt{java.util.Arrays.binarySearch}}

\QUESTBEGIN

\Task  \what~ I klassen \code{java.util.Arrays}\footnote{\href{https://docs.oracle.com/javase/8/docs/api/java/util/Arrays.html}{docs.oracle.com/javase/8/docs/api/java/util/Arrays.html}} finns en statisk metod \code{binarySearch} som kan användas enligt nedan.
\begin{REPL}
scala> val xs = Array(5,1,3,42,-1)
scala> java.util.Arrays.sort(xs)
scala> xs
scala> java.util.Arrays.binarySearch(xs, 42)
scala> java.util.Arrays.binarySearch(xs, 43)
\end{REPL}
Skriv ett valfritt Javaprogram som testar \code{java.util.Arrays.binarySearch}. Använd en array av typen \code{int[]} med några heltal som först sorteras med \code{java.util.Arrays.sort}.  Skriv ut det som returneras från  \code{java.util.Arrays.binarySearch}  i olika fall genom att söka efter tal som finns först, mitt i, sist och tal som saknas.
\emph{Tips:} Man kan deklarera en array, allokera den och fylla den med värden så här i Java: \\
\jcode|int[] xs = new int[]{5, 1, 3, 42, -1};|


\SOLUTION

\TaskSolved \what

\QUESTEND


\WHAT{Auto(un)boxing.}

\QUESTBEGIN

\Task  \what~  I JVM måste typparametern för generiska klasser vara av referenstyp. I Scala löser kompilatorn detta åt oss så att vi ändå kan ha t.ex. \code{Int} som argument till en typparameter i Scala, medan man i Java \emph{inte} direkt kan ha den primitiva typen \jcode{int} som typparameter till t.ex. \code{ArrayList}.

I Java och i den underliggande plattformen JVM används s.k. wrapper-klasser för att lösa detta, t.ex. genom wrapper-klassen \code{Integer} som boxar den primitiva typen \jcode{int}. Java-kompilatorn har stöd för att automatiskt packa in värden av primitiv typ i sådana wrapper-klasser för att skapa referenstyper och kan även automatiskt packa upp dem.

\Subtask Studera hur Scala-kompilatorn låter oss arbeta med en \code{Cell[Int]} även om det underliggande JVM:ens körtidstyp \Eng{runtime type} är en wrapper-klass. Man kan se JVM-körtidstypen med metoderna \code{getClass} och \code{getTypeName} enligt nedan.
\begin{REPL}
scala> class Cell[T](var value: T){
         val typeName: String = value.getClass.getTypeName
         override def toString = "Cell[" + typeName + "](" + value + ")"
       }
scala> val c = new Cell[Int](42)
scala> c.value.getClass.getTypeName
\end{REPL}


\Subtask Vad är körtidstypen för \code{c.value} ovan? Förklara hur det kan komma sig trots att vi deklarerade med typargumentet \code{Int}?

\Subtask Studera dokumentationen för \code{java.lang.Integer}\footnote{\href{https://docs.oracle.com/javase/8/docs/api/java/lang/Integer.html}{docs.oracle.com/javase/8/docs/api/java/lang/Integer.html}} och testa i REPL några av \emph{klassmetoderna} (de som är \jcode{static} och därmed kan anropas med punktnotation direkt på klassens namn utan \code{new}) och några av \emph{instansmetoderna} (de som inte är \jcode{static}).
\begin{REPL}
scala> Integer.  //tryck TAB
scala> Integer.
scala> Integer.toBinaryString(42)
scala> Integer.valueOf(42)
scala> val i = new Integer(42)
scala> i.  // tryck TAB
scala> i.toString
scala> i.compareTo  // tryck TAB 2 gånger
scala> i.compareTo(Integer.valueOf(42))
scala> i.compareTo(42)  // varför fungerar detta?
\end{REPL}

\Subtask\Pen Enligt dokumentationen\footnote{\href{https://docs.oracle.com/javase/8/docs/api/java/lang/Integer.html\#compareTo-java.lang.Integer-}{docs.oracle.com/javase/8/docs/api/java/lang/Integer.html\#compareTo-java.lang.Integer-}} tar instansmetoden \code{compareTo} i klassen \code{Integer} en \code{Integer} som parameter. Hur kan det då komma sig att sista raden ovan fungerar med en \code{Int}?

\Subtask Studera nedan Java-program och beskriv vad som kommer att skrivas ut \emph{innan} du kompilerar och testkör.

\javainputlisting[numbers=left]{examples/scalajava/Autoboxing.java}

\Subtask Ändra i programmet ovan så att autoboxing och autounboxing utnyttjas på alla ställen där så är möjligt. Utnyttja även att \code{toString}-metoden på \code{Integer} ger samma strängrepresentation som \jcode{int} vid utskrift. Fixa också så att du undviker \emph{fallgropen} att i Java jämföra med referenslikhet i stället för att använda \code{equals}. Testa så att allt fungerar som det borde efter dina ändringar.


\Subtask\Pen Antag att du råkar skriva \jcode{xs.add(0, pos)} på rad 14 i ditt program från föregående uppgift. Förklara hur autoboxingen stjälper dig i en \emph{fallgrop} då.

\Subtask\Pen Med ledning av de båda tidigare deluppgifterna: sammanfatta de två nämnda fallgropar med autoboxing i Java i två generella punkter, så att du har nytta av att memorera dem inför din framtida Javakodning.


\SOLUTION


\TaskSolved \what
     %%%TODO number  3 %%%starts with: \emph{Auto(un)boxing.} I JVM må%%%

\SubtaskSolved   -

\SubtaskSolved   Cell har typen java.lang.Integer. När man hämtar ut värdet med \code{c.value} hämtas den primitiva typ \code{int} ut.

\SubtaskSolved   Med hjälp av autoboxing förvandlas 42 till typen \code{Integer} och kan därför jämföras med en annan \code{Integer}.

\SubtaskSolved   i.compareTo(42) fungerar på grund av autoboxing. Då JVM packar in den primitiva typ int i en Integer-objekt automatiskt.

\SubtaskSolved
\begin{REPLnonum}
0 10 20 30 40 50 60 ... 390 400 410

[0]: 0
[42]: 0
NOT EQUAL
\end{REPLnonum}

\SubtaskSolved   \javainputlisting[numbers=left]{examples/scalajava/Autoboxing2.java}

\SubtaskSolved   42 kommer läggas längst fram i listan istället för längst bak, då autounboxing kommer göra Integer(0) till 0 och tvärtom med variablen \code{pos}.

\SubtaskSolved   Om man ska undersöka om två int-variabler är lika ska man använda ==, men om variablerna är av typen Integer måste man använda \code{equals}.

JVM kommer inte varna om man vänder på \code{Integer} och \code{int}, som i \code{xs.add(0, pos)}.



\QUESTEND






\WHAT{CollectionConverters.}

\QUESTBEGIN

\Task  \what~  Med \code{import scala.jdk.CollectionConverters._} får man i sina Scalaprogram tillgång till de smidiga metoderna \code{asJava} och \code{asScala} som översätter mellan motsvarande samlingar i resp språks standardbibliotek. Kör nedan i REPL och gör efterföljande deluppgifter.

\begin{REPL}
scala> val sv = Vector(1,2,3)
scala> val ss = Set('a','b','c')
scala> val sm = Map("gurka" -> 42, "tomat" -> 0)
scala> val ja = new java.util.ArrayList[Int]
scala> ja.add(42)
scala> val js = new java.util.HashSet[Char]
scala> js.add('a')
scala> import scala.jdk.CollectionConverters._
\end{REPL}

\Subtask Till vilka typer konverteras Scalasamlingarna
\code{Vector[Int]}, \code{Set[Char]} och \\ \code{Map[String, Int]} om du anropar metoden \code{asJava} på dessa?

\Subtask Till vilka typer konverteras Javasamlingarna \code{ArrayList[Int]} och \code{HashSet[Char]}  om du anropar metoden \code{asScala} på dessa? Blir det föränderliga eller oföränderliga motsvarigheter?

\Subtask Vad får resultatet för typ om du kör \code{toSet} på en samling av typen \code{mutable.Set}?

\Subtask Undersök hur du kan efter att du gjort \code{sm.asJava.asScala} anropa ytterligare en metod för att få tillbaka en oföränderlig \code{immutable.Map}.

\Subtask Läs mer i dokumentationen om CollectionConverters\footnote{\href{https://docs.scala-lang.org/overviews/collections-2.13/conversions-between-java-and-scala-collections.html}{docs.scala-lang.org/overviews/collections-2.13/conversions-between-java-and-scala-collections.html}}
och prova några fler konverteringar.



\SOLUTION


\TaskSolved \what
     %%%TODO number  4 %%%starts with: \emph{CollectionConverters.} Med \cod%%%

\SubtaskSolved

Vector[Int] -> java.util.List[Int]

Set[Char] -> java.util.Set[Char]

Map[String, Int] -> java.util.Map[String, Int]

\SubtaskSolved

ArrayList[Int] -> scala.collection.mutable.Buffer[Int]

HashSet[Char] -> scala.collection.mutable.Set[Char]

Båda blir föränderliga motsvarigheter. Det visas genom att de till hör \code{scala.collection.mutable} och både \code{ArrayList} och \code{HashSet} är föränderliga i Java.

\SubtaskSolved   \code{scala.collection.immutable.Set}

\SubtaskSolved   \code{sm.asJava.asScala} ger typen \code{scala.collection.mutable.Map[String,Int]}

\code{sm.asJava.asScala.toMap} ger typen \code{scala.collection.immutable.Map[String,Int]}

\SubtaskSolved   -

\QUESTEND


\WHAT{Hur fungerar en \jcode{switch}-sats i Java (och flera andra språk)?}

\QUESTBEGIN

\Task \label{task:switch} \what~   Det händer ofta att man vill testa om ett värde är ett av många olika alternativ. Då kan man använda en sekvens av många \code{if}-\code{else}, ett för varje alternativ. Men det finns ett annat sätt i Java och många andra språk: man kan använda \jcode{switch} som kollar flera alternativ i en och samma sats, se t.ex. \href{https://en.wikipedia.org/wiki/Switch_statement}{en.wikipedia.org/wiki/Switch\_statement}.

\Subtask Skriv in nedan kod i en kodeditor. Spara med namnet \texttt{Switch.java} och kompilera filen med kommandot \texttt{javac Switch.java}. Kör den med \texttt{java Switch} och ange din favoritgrönsak som argument till programmet. Vad händer? Förklara hur \jcode{switch}-satsen fungerar.

\javainputlisting[numbers=left,basicstyle=\ttfamily\fontsize{9}{11}\selectfont]{examples/Switch.java}

\Subtask \label{subtask:break} Vad händer om du tar bort \jcode{break}-satsen på rad 16?




\SOLUTION


\TaskSolved \what


\SubtaskSolved  Beroende på första bokstaven i din favoritgrönsak får du olika svar såsom \textit{gurka är gott!} vid första bokstaven $g$.\\
Javas \jcode{switch}-sats testar den första bokstaven på favoritgrönsaken genom att stegvis jämföra den med \jcode{case}-uttrycken. Om första bokstaven \jcode{firstChar} matchar bokstaven efter ett \jcode{case} körs koden efter kolonet till \jcode{switch}-satsens slut eller tills ett \jcode{break} avbryter \jcode{switch}-satsen.\\
Matchar inte \jcode{firstChar} något \jcode{case} så finns även \jcode{default}, som körs oavsett vilken första bokstaven är, ett generellt fall.

\SubtaskSolved  Om \jcode{case 't'} körs kommer både  \textit{tomat är gott!} och \textit{broccoli är gott!} skrivas ut, man säger att koden $"$faller igenom$"$. Utan \jcode{break}-satsen i Java körs koden i efterkommande \jcode{case} tills ett \jcode{break} avbryter exekveringen eller \jcode{switch}-satsen tar slut.



\QUESTEND




\WHAT{Fånga undantag i Java med en \jcode{try}-\jcode{catch}-sats.}

\QUESTBEGIN

\Task \label{task:javatry} \what~   Det finns som vi såg i förra uppgiften inbyggt stöd i JVM för att hantera när program avbryts på oväntade sätt, t.ex. på grund av division med noll eller ej förväntade indata från användaren. Spara koden nedan\footnote{\url{https://github.com/lunduniversity/introprog/blob/master/compendium/examples/TryCatch.java}} i en fil med namnet \texttt{TryCatch.java} och kompilera med \texttt{javac TryCatch.java} i terminalen.

\javainputlisting[numbers=left,basicstyle=\ttfamily\fontsize{11}{12}\selectfont]{examples/TryCatch.java}

\Subtask Förklara vad som händer när du kör programmet med olika indata:
\begin{REPL}
> java TryCatch 42
> java TryCatch 0
> java TryCatch safe 42
> java TryCatch safe 0
> java TryCatch
\end{REPL}

\Subtask Vad händer om du ''glömmer bort'' raden 15 och därmed missar att initialisera input? Hur lyder felmeddelandet? Är det ett körtidsfel eller kompileringsfel?

%\Subtask Beskriv några skillnader och likheter i syntax och semantik mellan \code{try}-\code{catch} i Java respektive Scala.



\SOLUTION


\TaskSolved \what


\SubtaskSolved  \begin{enumerate}
\item Eftersom första argumentet inte är strängen \textit{safe} görs en oskyddad division av 42 med 42 där slutsvaret 1 visas.
\item Eftersom första argumentet inte är strängen \textit{safe} görs en oskyddad division av 42 med 0 som ger \code{ArithmeticException} eftersom ett tal inte kan delas med noll.
\item Eftersom första argumentet är strängen \textit{safe} görs en skyddad division av 42 med 42 där slutsvaret 1 visas.
\item Eftersom första argumentet är strängen \textit{safe} görs en skyddad division av 42 med 0. Denna gång fångas \code{ArithmeticException} av \code{try-catch}-satsen vilket ersätter den gamla division med en säker division med 1 där slutsvaret 42 visas.
\item Eftersom inga argument givits kastas ett \code{ArrayIndexOutOfBoundsException} när programmet försöker anropa \code{equals} metoden hos en sträng som inte finns. Detta kunde också kontrollerats av en \code{try-catch}-sats.
\end{enumerate}

\SubtaskSolved  \begin{REPL}
TryCatch.java:16: error: variable input might not have been initialized
\end{REPL}
Ett kompileringsfel uppstår på grund av risken att \code{input} inte blivit definierad vid division.

% \SubtaskSolved  Den mest markanta skillnaden mellan språken är att Scala varken kräver att ett undantag fångas av en \code{catch} eller att ett undantag behöver deklareras innan det kastas med en \code{@throws}. Dessutom saknar \code{catch}-metoden hos Java de \code{match}-egenskaper Scala har. Inte heller returnerar \code{catch} hos Java något värde vilket gör det nödvändigt att definiera variabler för detta innan. I övrigt är semantiken och syntaxen väldigt lika mellan båda språken. De använder samma struktur och samma ord, dessutom har de en hel del \code{Exception} gemensamt.



\QUESTEND




\WHAT{Matriser med array i Java.}

\QUESTBEGIN

\Task \label{task:arraymatrix-java} \what~   Om man redan vid allokering vet hur många element en matris ska ha, använder man i Java gärna en array av arrayer. En heltalsmatris (en array av array av heltal) skrivs i Java med dubbla hakparentespar \jcode{int[][]} direkt efter typen. Vid allokering använder man nyckelordet \code{new} och antalet element i respektive dimension anges inom hakparenteserna; t.ex. så ger \jcode{new int[42][21]} en matris med 42 rader och 21 kolumner, vilket motsvarar att man i Scala skriver \code{Array.ofDim[Int](42,21)}%
\footnote{
Ett annat sätt att skriva detta i Scala där initialvärdet framgår explicit: \code{Array.fill(42,21)(0)}
}. Alla element får defaultvärdet för typen, här \code{0} för heltal.

\Subtask Skriv nedan program i en editor och spara koden i filen \texttt{JavaArrayTest.java} och kompilera med \texttt{javac JavaArrayTest.java} och kör i terminalen med \texttt{java JavaArrayTest} och undersök utskriften. Förklara vad som händer. Notera några skillnader i hur matriser används i Scala och Java.


\begin{Code}[language=Java]
public class JavaArrayTest {

    public static void showMatrix(int[][] m){
        System.out.println("\n--- showMatrix ---");
        for (int row = 0; row < m.length; row++){
            for (int col = 0; col < m[row].length; col++) {
                System.out.print("[" + row + "]");
                System.out.print("[" + col + "] = ");
                System.out.print(m[row][col] + "; ");
            }
            System.out.println();
        }
    }

    public static void main(String[] args) {
        System.out.println("Hello JavaArrayTest!");
        int[][] xss = new int[10][5];
        showMatrix(xss);
    }
}
\end{Code}

\Subtask Implementera nedan metod \code{fillRnd} inuti klassen \code{JavaArrayTest}. Skriv kod som fyller matrisen \code{m} med slumptal mellan \code{1} och \code{n}.
\begin{Code}[language=Java]
    public static void fillRnd(int[][] m, int n){
        /* ??? */
    }
\end{Code}
\noindent \emph{Tips:} med detta uttryck skapas ett slumptal mellan 1 och 42 i Java:\\
\jcode{(int) (Math.random() * 42 + 1);} \\
där typkonverteringen \jcode{(int)} ger samma effekt som ett anrop av metoden \code{toInt} i Scala; alltså att dubbelprecisionsflyttal omvandlas till heltal genom avkortning av alla decimaler.


Ändra huvudprogrammet så det anropar \jcode{fillRnd(xss, 6)}. %
% \begin{Code}[language=Java]
%     public static void main(String[] args) {
%         System.out.println("Hello JavaArrayTest!");
%         int[][] xss = new int[10][5];
%         fillRnd(xss, 6);
%         showMatrix(xss);
%     }
% \end{Code}
Programmet ska ge en utskrift som liknar följande:
\begin{REPL}
Hello JavaArrayTest!

--- showMatrix ---
[0][0] = 6; [0][1] = 2; [0][2] = 6; [0][3] = 3; [0][4] = 5;
[1][0] = 2; [1][1] = 4; [1][2] = 6; [1][3] = 1; [1][4] = 1;
[2][0] = 5; [2][1] = 4; [2][2] = 4; [2][3] = 1; [2][4] = 5;
[3][0] = 4; [3][1] = 6; [3][2] = 6; [3][3] = 1; [3][4] = 3;
[4][0] = 4; [4][1] = 6; [4][2] = 2; [4][3] = 3; [4][4] = 2;
[5][0] = 2; [5][1] = 4; [5][2] = 5; [5][3] = 5; [5][4] = 3;
[6][0] = 6; [6][1] = 5; [6][2] = 2; [6][3] = 4; [6][4] = 3;
[7][0] = 1; [7][1] = 6; [7][2] = 1; [7][3] = 6; [7][4] = 2;
[8][0] = 1; [8][1] = 1; [8][2] = 5; [8][3] = 3; [8][4] = 2;
[9][0] = 1; [9][1] = 1; [9][2] = 1; [9][3] = 5; [9][4] = 4;

\end{REPL}

\SOLUTION

\TaskSolved \what
     %starts with: \label{task:arraymatrix-java} \%%%

%6.a)
\SubtaskSolved  Vid initialisering fylls alla element i \code{xss} med standardvärdet för typen, \code{0} i fallet med \code{int}. Den yttre \code{for}-loopen i \code{showMatrix()} itererar över raderna i \code{xss}. Den inre \code{for}-loopen itererar i sin tur längs med elementen på den aktuella raden och skriver ut rad, kolumn och innehåll. Efter varje rad sker en radbrytning, så att en rad i utskriften även motsvarar en rad i matrisen.\\
Exempel på skillnader mellan användning av matriser i scala och java:
\begin{itemize}
\item åtkomst: \code{minArray(rad)(kolumn)} respektive \code{minArray[rad][kolumn]}
\item typnamn: \code{Array[Array[elementTyp]]} respektive  \code{elementTyp[][]}
\item allokering: \code{Array.ofDim[typ](xDim,yDim)} respektive \code{new typ[xDim][yDim]}
\end{itemize}

%6.b)
\SubtaskSolved  \begin{Code}[language=Java]
public class JavaArrayTest {

	public static void showMatrix(int[][] m){
		System.out.println("\n--- showMatrix ---");
		for (int row = 0; row < m.length; row++){
			for (int col = 0; col < m[row].length; col++) {
				System.out.print("[" + row + "]");
				System.out.print("[" + col + "] = ");
				System.out.print(m[row][col] + ";");
			} System.out.println();
		}
	}

	public static void fillRnd(int[][] m, int n){
		for (int row = 0; row < m.length; row++){
			for (int col = 0; col < m[row].length; col++) {
				m[row][col] = (int) (Math.random() * n + 1);
			}
		}
	}

	public static void main(String[] args) {
    System.out.println("Hello JavaArrayTest!");
		int[][] xss = new int[10][5];
		fillRnd(xss, 6);
		showMatrix(xss);
	}
}
\end{Code}

\QUESTEND


%\ExtraTasks %%%%%%%%%%%%%%%%%%%


\WHAT{Översätta från Java till Scala.}

\QUESTBEGIN

\Task  \what~ Översätt nedan kod från Java till Scala. Skriv koden i en fil som heter \texttt{showInt.scala} och kalla Scala-objektet med \code{main}-metoden för \code{showInt}. Läs tipsen som följer efter koden innan du börjar.

\javainputlisting[numbers=left]{examples/scalajava/JShowInt.java}

\emph{Tips:}
\begin{itemize}[nolistsep, noitemsep]
\item En Javaklass med bara statiska medlemmar motsvaras av ett singeltonobjekt i Scala, alltså en \code{object}-deklaration. Scala har därför inte nyckelordet \jcode{static}.
\item Typen \jcode{Object} i Java motsvaras av Scalas \code{Any}.
\item Du kan använda Scalas möjlighet med default-argument (som saknas i Java) för att bara definiera en enda \code{show}-metod med en tom sträng som default \code{msg}-argument.
\item I Scala har objekt av typen \code{Char} en metod \code{def *(n: Int): String} som skapar en sträng med tecknet repeterat \code{n} gånger. Men du kan ju välja att ändå implementera metoden \code{repeatChar} med \code{StringBuilder} som nedan om du vill träna på att översätta en \code{for}-loop från Java till Scala.
\item I stället för \code{Scanner.nextLine} kan du använda \code{scala.io.StdIn.readLine} som tar en prompt som parameter, men du kan också använda \code{Scanner} i Scala om du vill träna på det.
\item I Java \emph{måste} man använda nyckelordet \jcode{return} om metoden inte är en \jcode{void}-metod, medan man i Scala faktiskt \emph{får} använda \code{return} även om man brukar undvika det och i stället utnyttja att satser i Scala också är uttryck.
\end{itemize}
Kompilera din Scala-kod och kör i terminalen och testa så att allt funkar. Vill du även kompilera Java-koden så finns den i kursens repo i filen\\ \texttt{compendium/examples/scalajava/JShowInt.java}


\SOLUTION


\TaskSolved \what


\begin{Code}[numbers=left]
object showInt {
  def show(obj: Any, msg: String = ""): Unit = println(msg + obj)

  def repeatChar(ch: Char, n: Int): String = ch.toString * n

  def showInt(i: Int): Unit = {
    val leading = Integer.numberOfLeadingZeros(i)
    val binaryString = repeatChar('0', leading) + i.toBinaryString
    show(i,               "Heltal : ")
    show(i.asInstanceOf[Char],         "Tecken : ")
    show(binaryString,    "Binärt : ")
    show(i.toHexString,   "Hex    : ")
    show(i.toOctalString, "Oktal  : ")
  }


  import scala.io.StdIn.readLine
  import scala.util.{Try,Success,Failure}

  def loop: Unit =
    Try { readLine("Heltal annars pang: ").toInt } match {
      case Failure(e) => show(e); show("PANG!")
      case Success(i) => showInt(i); loop
    }

  def main(args: Array[String]): Unit =
    if(args.length > 0) args.foreach(i => showInt(i.toInt))
    else loop
}
\end{Code}



\QUESTEND






\WHAT{Innehållslikhet och referenslikhet i Java.}

\QUESTBEGIN

\Task  \what~ Studera och prova denna fallgrop med innehållslikhet: \href{https://github.com/bjornregnell/lth-eda016-2015/blob/master/lectures/examples/eclipse-ws/lecture-examples/src/week10/generics/TestPitfall3.java}{TestPitfall3.java}







\SOLUTION


\TaskSolved \what
     %%%TODO number  6 %%%starts with: \TODO Fallgrop med Point som in%%%



\QUESTEND




%\AdvancedTasks %%%%%%%%%%%%%%%%%


\WHAT{Implementera innehållslikhet i Java.}

\QUESTBEGIN

\Task  \what~\Pen Studera fallgropar för hur man skriver en \code{equals}-metod i Java här:
\href{http://www.artima.com/lejava/articles/equality.html}{www.artima.com/lejava/articles/equality.html} och jämför med  det fullständiga receptet för hur man skriver en välfungerande \code{equals} och \code{hashcode} i Scala här: \href{http://www.artima.com/pins1ed/object-equality.html}{www.artima.com/pins1ed/object-equality.html}

\Subtask Vilka skillnader och likheter finns vid överskuggning av equals i Java respektive Scala, som ska ge en fungerande innehållstest för en hierarki med bastyper och subtyper?

\Subtask Vilka fallgropar är gemensamma för Java och Scala?\SOLUTION


\TaskSolved \what
     %%%TODO number  7 %%%starts with: \TODO \emph{Gränssnitt i Scala %%%



\QUESTEND


\WHAT{Array och \code{for}-sats i Java.}

\QUESTBEGIN

\Task  \what~Ladda ner programet nedan från kursens GitHub-repo: \href{https://raw.githubusercontent.com/lunduniversity/introprog/master/compendium/examples/DiceReg.java}{\texttt{compendium/examples/DiceReg.java}}


\Subtask
Kompilera med \code{javac DiceReg.java} och kör med \code{java DiceReg 10000 42} och förklara vad som händer.

\javainputlisting{examples/DiceReg.java}

\Subtask Beskriv skillnaderna mellan Scala och Java, vad gäller syntaxen för array och \code{for}-sats. Beskriv några andra skillnader mellan språken som syns i programmet ovan.

\Subtask Ändra i programmet ovan så att loop-variabeln \code{i} skrivs ut i varje runda i varje \code{for}-sats. Kompilera om och kör.

\Subtask Skriv om programmet ovan genom att abstrahera huvudprogrammets delar till de statiska metoderna \code{parseArguments}, \code{registerPips} och \code{printReg} enligt nedan skelett. Spara programmet i filen \code{DiceReg2.java} och kompilera med \texttt{javac DiceReg2.java} i terminalen.

\begin{Code}[language=Java]
// DiceReg2.java
import java.util.Random;

public class DiceReg2 {
    public static int[] diceReg = new int[6];
    private static Random rnd = new Random();

    public static int parseArguments(String[] args) {
        // ???
        return n;
    }

    public static void registerPips(int n){
        // ???
    }

    public static void printReg() {
        // ???
    }

    public static void main(String[] args) {
        int n = parseArguments(args);
        registerPips(n);
        printReg();
    }
}
\end{Code}

\Subtask Starta Scala REPL i samma katalog som filen \texttt{DiceReg2.class} ligger i och kör nedan 7 rader i REPL och förklara vad som händer:
\begin{REPL}
scala> DiceReg2.main(Array("1000","42"))
scala> DiceReg2.diceReg
scala> DiceReg2.registerPips(1000)
scala> DiceReg2.printReg
scala> DiceReg2.registerPips(1000)
scala> DiceReg2.printReg
scala> DiceReg2.rnd
\end{REPL}

\SOLUTION

\TaskSolved \what

\SubtaskSolved Programmet simulerar 10000 tärningskast (med slumptalsfrö 42) och skriver ut förekomsten av respektive tärningskast.

\begin{REPL}
Rolling the dice 10000 times with seed 42
Number of 1's: 1654
Number of 2's: 1715
Number of 3's: 1677
Number of 4's: 1629
Number of 5's: 1643
Number of 6's: 1682
\end{REPL}

\SubtaskSolved  I Java används hakparenteser medan Scala har ''vanliga'' parenteser. En array i scala deklareras så här: \\
 \code{val scalaArray = Array.ofDim[Int](6)} \\
 vilket i java motsvarar: \code{int[] javaArray = new int[6];}

\code{for}-sats i scala skrivs: \code|for(i <- 0 to n) {...}| medan i Java skrivs: \\ \code|for (int i = 0; i < n; i++) { ... }|.

I Java måste semikolon skrivas efter varje sats och typen måste anges explicit vid varje variabeldeklaration.

I scala behövs inte semikolon (förutom för att separera satser på samma rad) och typer kan ofta härledas i Scala av kompilatorn och behöver inte alltid skrivas explicit.

\SubtaskSolved  Lägg till \code{System.out.println(i);} i for-looparna

\SubtaskSolved  \begin{Code}[language=Java]
// DiceReg2.java
import java.util.Random;
public class DiceReg2{
	public static int[] diceReg = new int[6];
	private static Random rnd = new Random();

	public static int parseArguments(String[] args){
		int n = 100;
		if(args.length > 0) {
			n = Integer.parseInt(args[0]);
		}
		if(args.length > 1) {
			int seed = Integer.parseInt(args[1]);
			rnd.setSeed(seed);
		}
		return n;
	}

	public static void registerPips(int n) {
		for(int i = 0; i<n; i++) {
			int pips = rnd.nextInt(6);
			diceReg[pips]++;
		}
	}

	public static void main(String[] args) {
		int n = parseArguments(args);
		registerPips(n);
		printReg();
	}
}
\end{Code}

\SubtaskSolved

\begin{REPL}
  // Skriver ut förekomsten av 1000 tärningskast med slumptalsfrö 42.
Number of 1's: 165
Number of 2's: 163
Number of 3's: 178
Number of 4's: 183
Number of 5's: 156
Number of 6's: 155

  // Skriver ut diceReg-attributet
res1: Array[Int] = Array(165, 163, 178, 183, 156, 155)

  // Skriver ut diceReg-attributet efter 1000 till kast.
res2: Array[Int] = Array(329, 325, 349, 360, 324, 313)

  // Skriver ut diceReg-attributet efter 1000 till kast.
res3: Array[Int] = Array(498, 484, 531, 513, 485, 489)

  // Det blir kompileringsfel då attributet rnd är privat
<console>:11: error: value rnd is not a member of object DiceReg2
	DiceReg2.rnd
				    ^
\end{REPL}

\QUESTEND





\WHAT{Läsa in sekvens av tal med \code{Scanner} i Java.}

\QUESTBEGIN

\Task  \what~  Läs i Java-delen av snabbreferensen om \code{java.util.Scanner}. Med \jcode{new Scanner(System.in)} skapas ett objekt som kan läsa in tal från teckensträngar som användaren skriver i terminalfönstret, så som visas i Java-programmet nedan:

\javainputlisting{examples/DiceScanBuggy.java}
Ladda ner programmet   \href{https://raw.githubusercontent.com/lunduniversity/introprog/master/compendium/examples/DiceReg.java}{\texttt{compendium/examples/DiceScanBuggy.java}}
och kompilera och kör med indatasekvensen \texttt{1 2 3 4 -1} och notera hur registreringen sker.

\Subtask Sök upp och läs JDK8-dokumentationen av \code{java.util.Scanner}. Vad gör \jcode{hasNextInt()} och \jcode{nextInt()}?


\Subtask Programmet fungerar inte som det ska. Du behöver korrigera 3 saker för att programmet ska göra rätt. Rätta buggarna och spara det rättade programmet som \texttt{DiceScan.java}. Kompilera och testa det rättade programmet.

\SOLUTION

\TaskSolved \what

\SubtaskSolved

\code{hasNextInt()} kollar om det finns ett till tal och returnerar \code{true} eller \code{false}. \code{nextInt()} läser nästa tal.

Se \url{https://docs.oracle.com/javase/8/docs/api/java/util/Scanner.html#hasNextInt%28%29} och \\ \url{https://docs.oracle.com/javase/8/docs/api/java/util/Scanner.html#nextInt%28%29 }.

\SubtaskSolved

\begin{Code}[language=Java,numbers=left]
import java.util.Random;
import java.util.Scanner;

public class DiceScanBuggy {
	public static int[] diceReg = new int[6];
	public static Scanner scan = new Scanner(System.in);

	public static void registerPips() {
		System.out.println("Enter pips separated by blanks: ");
		System.out.println("End with -1 and <Enter>.");
		boolean isPips = true;
		while(isPips && scan.hasNextInt()){
			int pips = scan.nextInt();
			if(pips >= 1 && pips <= 6) {
				diceReg[pips-1]++;
			} else {
				isPips = false;
			}
		}
	}

	public static void printReg(){
		for(int i = 1; i<7; i++) {
		System.out.println("Number of " + i + "'s: " + diceReg[i-1]);
		}
	}

	public static void main(String[] args) {
		registerPips();
		printReg();
	}
}
\end{Code}

\QUESTEND



%\chapter{Snabbreferens}\label{chapter:quickref}
%
%Detta appendix innehåller en snabbreferens för Scala och Java. Snabbreferensen är enda tillåtna hjälpmedel under kursens skriftliga tentamen.
%
%Lär dig vad som finns i snabbreferensen så att du snabbt hittar det du behöver och träna på hur du  effektivt kan dra nytta av den när du skriver program med papper och penna utan datorhjälpmedel.
%
%\clearpage
%~
%\clearpage
%
%\includepdf[pages={1-12}, scale=0.77, frame]{../quickref/quickref.pdf}


\end{document}
