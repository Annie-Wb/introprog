%!TEX encoding = UTF-8 Unicode
%!TEX root = ../compendium2.tex

\section{JetBrains IntelliJ IDEA med Scala-plugin}\label{appendix:ide:intellij}

IntelliJ IDEA%
\footnote{\href{https://en.wikipedia.org/wiki/IntelliJ_IDEA}{en.wikipedia.org/wiki/IntelliJ\_IDEA}}
 är en professionell IDE som stödjer många olika programmeringsspråk. IntelliJ är skriven i Java och utvecklas av det tjeckiska företaget JetBrains.

IntelliJ IDEA finns i två varianter: en gratis gemenskapsvariant med öppenkällkodslicens \Eng{Community edition}, samt en betalvariant med sluten källkod och support-tjänster.

\begin{figure}
\centering
\includegraphics[width=1.0\textwidth]{../img/intellij/idea-hello}
\caption{Den integrerade utvecklingsmiljön Intellij IDEA.\label{appendix-ide:intellij-hello}}
\end{figure}

IntelliJ IDEA är en omfattande och avancerad programmeringsmiljö med många funktioner och inställningar. Det finns även en omfattande uppsättning insticksmoduler och tilläggsprogram som underlättar utveckling av t.ex. mobilappar, webbprogram, databaser och mycket annat.

Till IntelliJ IDEA finns en insticksmodul \Eng{plug-in} som stöd för Scala med tillhörande standardbibliotek och byggverktyget \code{sbt}, med mera. Scala-insticksmodulen kan inkluderas genom att välja Scala i en av de dialoger som visas vid första körningen, enligt instruktioner nedan.

I detta avsnitt ges länkar till installation samt tips om hur du kommer igång med att använda IntelliJ IDEA med Scala. Det går ganska snabbt att lära sig grunderna, men det kräven en viss ansträngning att lära sig de mer avancerade funktionerna. Det finns omfattande resurser på nätet som hjälper dig vidare.

Google tillkännagav 2013 att företaget övergår från Eclipse till IntelliJ som den officiellt understödda utvecklingsmiljön för Android och 2014 lanserades utvecklingsmiljön Android Studio%
\footnote {\href{https://en.wikipedia.org/wiki/Android_Studio}{en.wikipedia.org/wiki/Android\_Studio}}
 som bygger vidare på IntelliJ.

\subsection{Installera IntelliJ IDEA}\label{appendix:ide:intellij:install}

IntelliJ med Scala-plug-in är förinstallerat på LTH:s datorer och startas med kommandot \texttt{idea} i ett terminalfönster.

Du kan installera IntelliJ på din egen dator genom att följa instruktionerna för ditt operativsystem (Windows/macOS/Linux) här: \\
\url{https://www.jetbrains.com/help/idea/run-for-the-first-time.html}


Du behöver Scala-plugin som du kan välja under installationen av IntelliJ, men det går också att installera plugin för Scala i efterhand, se vidare här:\\
\url{https://www.jetbrains.com/help/idea/discover-intellij-idea-for-scala.html} 

