%!TEX encoding = UTF-8 Unicode
%!TEX root = ../compendium2.tex

\newcommand{\sbt}{\texttt{sbt}}

\chapter{Byggverktyg}\label{appendix:build}

\section{Vad gör ett byggverktyg?}

Ett \textbf{byggverktyg} \Eng{build tool} används för att
\begin{itemize}
\item ladda ner,
\item kompilera,
\item testköra,
\item paketera och
\item distribuera
\end{itemize}
programvara. Ett stort utvecklingsprojekt kan innehålla många hundra kodfiler och under utvecklingens gång vill man kontinuerligt testköra systemet för att kontrollera att allt fortfarande fungerar; även den kod som inte ändrats, men som kanske ändå påverkas av ändringen. Ett byggverktyg används för att \textit{automatisera} denna process.

Ett viktigt begrepp i byggsammanhang är \textbf{beroende} \Eng{dependency}. Om koden X behöver annan kod Y för att fungera, sägs kod X ha ett beroende till kod Y.

I konfigurationsfiler, som är skrivna i ett format som byggverktyget kan läsa, specificeras de beroenden som finns mellan olika koddelar. Byggverktyget analyserar dessa beroenden och, baserat på ändringstidsmarkeringar för kodfilerna, avgör byggverktyget vilken delmängd av kodfilerna som behöver \textbf{omkompileras} efter en ändring. Detta snabbar upp kompileringen avsevärt jämfört med en total omkompilering från grunden, som för ett stort projekt kan ta många minuter eller till och med timmar. Efter omkompilering av det som ändrats, kan byggverktyget instrueras att köra igenom testprogram och rapportera om testernas utfall, men även ladda upp körbara programpaket till t.ex. en webbserver.


En vanlig typ av beroende är färdiga programbibliotek som utnyttjas av systemet under utveckling, vilket i praktiken ofta innebär att en sökväg till den kompilerade koden för programbiblioteket behöver göras tillgänglig. I JVM-sammanhang innebär detta att sökvägen till alla nödvändiga jar-filer behöver finnas på sökvägslistan kallad \textbf{classpath}.

Många byggverktyg kan utföra så kallad \textbf{beroendeupplösning} \Eng{dependency resolution}, vilket innebär att nätverket av beroenden analyseras och rätt uppsättning programpaket görs tillgänglig under bygget. Detta kan även innebära att programpaket som är tillgängliga via nätet automatiskt laddas ned inför bygget, t.ex. via lagringsplatser för öppen källkod.

Även om man bara har ett litet kodprojekt med några få kodfiler, är det ändå smidigt att använda ett byggverktyg. Man kan nämligen göra så att byggverktyget är aktivt i bakgrunden och, så fort man sparar en ändring av koden, gör omkompilering och rapporterar eventuella kompileringsfel.

Det är klokt att kompilera om ofta, helst vid varje liten ändring, och rätta eventuella fel \textit{innan} nya ändringar görs, eftersom det är mycket lättare att klura ut ett enskilt problem efter en mindre ändring, än att åtgärda en massa svåra följdfel, som beror på en sekvens av omfattande ändringar, där misstaget begicks någon gång långt tidigare.

En integrerad utvecklingsmiljö, så som VS Code eller IntelliJ IDEA, bygger om koden kontinuerligt och kan ofta konfigureras att kommunicera med flera olika byggverktyg. Exempelvis kan du med VS Code välja om du vill att Scala CLI eller \sbt~ ska användas för att bygga ditt projekt.

Det finns många olika byggverktyg. Några allmänt kända byggverktyg listas nedan så att du ska känna igen vilket byggverktyg som används i öppen-källkods-projekt som du stöter på, t.ex. på GitHub.

\begin{itemize}

\item \texttt{Scala CLI}. Verktyget Scala CLI (Command Line Interface) är öppen källkod utvecklad av VirtusLab\footnote{\url{https://scala-cli.virtuslab.org/}} för att kompilera och köra Scala- och Java-program och innehåller också grundläggande byggverktygsfunktioner, så som att köra testfall, paketera jar-filer och skapa dokumentation. Kommandot \code{scala-cli} övertog år 2023 rollen som det officiella \code{scala}-kommandot. Detta är det enklaste och rekommenderade sättet att bygga system med Scala-kod. Grundläggande användning av Scala CLI beskrivs i Appendix \ref{appendix:compile:scala-cli}, medan en mer utförlig beskrivning återfinns nedan i avsnitt \ref{appendix:build:scala-cli}.

\item \sbt. Även kallad \textit{Scala Build Tool}. Används för att bygga Java- och Scala-program i samexistens, men även för att automatisera en mängd andra saker. \sbt~är utvecklat i Scala och konfigurationsfilerna, som heter \texttt{build.sbt}, innehåller Scala-kod som styr byggprocessen. \sbt~är avancerat och klarar bygga system som består av många projekt \Eng{multi-project build}. \sbt~är det i särklass vanligaste byggverktyget för Scala och många öppen-källkodsprojekt använder \sbt. Läs mer om \sbt~i avsnitt \ref{appendix:build:sbt} nedan.

Efter kritik om att \sbt~är komplicerat så har flera alternativa byggverktyg för Scala utvecklats, däribland \code{bleep}\footnote{\url{https://bleep.build/docs/}} och \code{mill}\footnote{\url{https://mill-build.com/mill/Intro_to_Mill.html}}. 


\item Apache Maven, \texttt{mvn} är också skriven i Java och är en efterföljare till \texttt{ant}. Maven används av många Java-utvecklare. Konfigurationsfilerna heter \texttt{pom.xml} och innehåller en s.k. projektobjektmodell specificerad i XML enligt  speciella regler.

\item \texttt{gradle} bygger vidare på idéerna från \texttt{ant} och \texttt{maven} och är skrivet i Java och Groovy.  Konfigurationsfilerna skrivs i Groovy och heter \texttt{build.gradle}.

\item Apache \texttt{ant}. Detta byggverktyg är utvecklat i Java som ett alternativ till \texttt{make} och används fortfarande i många Java-projekt, även om Maven och Gradle är vanligare numera. Konfigurationsfilerna heter \texttt{build.xml} och skrivs i det standardiserade språket XML enligt  speciella regler.

\item \texttt{make}. Detta anrika byggverktyg har varit med ända sedan 1970-talet och används fortfarande för att bygga många system under Linux, och är populärt vid utveckling med programspråken C och C++. En konfigurationsfil för \texttt{make} heter \texttt{Makefile} och har en egen, speciell syntax.
\end{itemize}

\section{Scala Command Line Interface \texttt{scala-cli}}\label{appendix:build:scala-cli}

Utvecklingen av Scala CLI\footnote{CLI är en förkortning för \textit{command line interface}. Läs mer om Scala CLI här: \url{https://scala-cli.virtuslab.org/docs/overview}} påbörjades 2022 och planeras under 2024 bli det officiella bygg- och körverktyget. Det kommer då medfölja den officiella installationen av Scala via \url{https://www.scala-lang.org}. Scala CLI kan även installeras separat från \url{https://scala-cli.virtuslab.org} och köras med kommandot \code{scala-cli}, men någon gång under 2024 när Scala 3.5 släpps så blir kommandot \code{scala} liktydigt med \code{scala-cli}.\footnote{I skrivande stund har gamla \code{scala}-kommandot ännu inte uppgraderats, men det förväntas ske i början av hösten. När så väl sker kan \code{scala-cli} ersättas med \code{scala} om du uppdaterar till Scala 3.5.0 eller senare.}

Efter nyinstallation av Scala CLI kan du ange följande kommando för att, en gång för alla, få tillgång till kompletteringar av optioner med Tab-tangenten i terminalen:
\begin{REPLsmall}
scala-cli install-completions 
\end{REPLsmall}

Innan du börjar skriva källkod i en ny katalog i VS Code kan du göra VS Code redo för att använda Scala CLI som byggverktyg i aktuell katalog med följande kommando (vänta med att öppna VS Code till efter att du kört kommandot): 
\begin{REPLsmall}
mkdir minNyaKatalog
cd minNyaKatalog
scala-cli setup-ide . 
code .
\end{REPLsmall}
I senare versionerna av VS Code med Metals och Scala 3.4 behövs ej \code{setup-ide} då Metals kör scala-cli som default.

\subsection{Exempel på användning av Scala CLI}\label{appendix:build-scala-cli-watch-mode}

Nedan beskrivs de viktigaste Scala-CLI-kommandona för att stegvis bygga upp din kod med många små steg och experimenterande med dellösningar. 

\begin{itemize}
  \item Med kommandot \code{scala-cli compile . -w} i ett eget terminalfönster bredvid din editor startar du Scala CLI i så kallad \textit{watch mode}. Då bevakas alla filändringar och omkompilering sker direkt när någon ny ändring sparats och du kan se eventuella kompileringsfel direkt. Åtgärda helst ett kompileringsfel innan du bygger vidare på din kod, då följdfel kan vara svåra att lösa speciellt om de är många. Dela upp stora kodändringar i små steg och försök att så fort som möjligt få den senaste ändringen att kompilera felfritt. 
  \item
    Med kommandot \code{scala-cli repl .}~ i ett annat terminalfönster startar du REPL med dina klasser i aktuella katalogen  automatiskt tillgängliga på classpath (därav punkten efter blanksteg efter \code{scala repl}) och du kan anropa dina metoder, efter ev. \code{import} när du gör experiment inför nästa steg. 
    
    På så sätt kan du skapa och testa små funktioner och få dem att att fungera innan inför dem i ditt program och sätter samman dessa med redan skapade funktioner. Det är ofta lättare att felsöka och bygga upp ett större program om du har många små funktioner som samverkar, snarare än få väldigt stora funktioner. Och det är oftast lättare att testköra nya lösningsidéer i REPL innan du skapar ''färdig'' kod i ditt program.
  \item
    Med \texttt{scala-cli run .} sker kompilering och körning av \code{main}-metoden i aktuella katalogen. Du kan ange en annan katalog genom att skicka med sökvägen som argument till kommandot.  Du kan också ange optionen \code{-w} för \textit{watch mode} vid körning. Då kommer ditt program att köras om vid varje ändring. Watch mode vid körning är användbart om programmet ger resultat utan att vänta på input, men inte så smidigt om varje körning kräver att användaren skriver indata eller om fönster måste stängas.
  \item
    Om det finns flera \texttt{main}-metoder i aktuella katalogen, går det att specificera vilken av dessa som ska exekveras med optionen \verb|--main-class|, exempelvis\\ \verb|scala-cli run .  --main-class hello| 
  \item Argument till ditt huvudprogam anges efter dubbla minustecken \verb|--| så här: \\\verb|scala-cli run . -- arg1 arg2 arg3|
\end{itemize}



\subsection{Grundläggande byggfunktioner i Scala CLI}

De grundläggande funktionerna sammanfattas nedan (se även Appendix \ref{appendix:compile:scala-cli}):

\begin{table}[H]
\begin{tabular}{l p{6.5cm}}
\texttt{scala-cli repl} & Starta Scala REPL.  Det går även bra med enbart \texttt{scala-cli}\\
\texttt{scala-cli repl hello.scala} & Starta Scala REPL med kompilerade koden i \texttt{hello.scala} på classpath.  \\
\texttt{scala-cli repl .} & Starta repl med kodfiler i aktuell katalog tillgängliga på classpath. \\
\texttt{scala-cli compile hello.scala} & Kompilera koden i filen \texttt{hello.scala}  \\
\texttt{scala-cli compile .} & Kompilera alla kodfiler i aktuell katalog. \\
\texttt{scala-cli run hello.scala} & Kompilera koden i filen \texttt{hello.scala} och kör igång eventuellt huvudprogram om kompileringen gick bra. \\
\texttt{scala-cli run .} & Kompilera och kör alla kodfiler i aktuell katalog. \\
\texttt{scala-cli run . -{}-list-main-class} & Lista alla huvudprogram. \\
\texttt{scala-cli run . -M mypkg.myMain} & Kör ett specifikt huvudprogram. Förk. \texttt{-M} kan även skrivas \texttt{-{}-main-class}\\
\texttt{scala-cli package .} & Paketera all kompilerade kodfiler i en körbar fil. \\
\\
\end{tabular}
\end{table}

\subsection{Använda optioner för att styra Scala CLI}

\noindent Det finns en mängd olika optioner som du kan lägga till för att styra vad Scala CLI ska göra. Se exempel nedan och förklaringar i efterföljande tabell:
\begin{REPLsmall}
scala-cli run . -S 3.3 -O -unchecked --dep se.lth.cs::introprog::1.3.1 -w
\end{REPLsmall}

\noindent Här förklaras några vanliga optioner som kan användas vid både kompilering och exekvering: 
\begin{table}[H]
\begin{tabular}{l p{6.5cm}}
\texttt{-{}-scala 3.3} & Använd version 3.3 av Scala. Optionen \texttt{-{}-scala} kan förkortas med \texttt{-S} \\
\texttt{-{}-watch} & Upprepa kommando vid sparad ändring. Optionen \texttt{-{}-watch} förkortas med \texttt{-w} \\
\texttt{-{}-jar introprog.jar} & Lägg till en jar-fil på classpath. \\
\texttt{-{}-dep se.lth.cs::introprog::1.3.1} & Lägg till ett beroende på classpath. \\
\texttt{-{}-scalac-option -unchecked} & Lägg till en kompilator-option som ger extra varningar vid osäker kod.  Optionen \texttt{-{}-scalac-option} kan förkortas med \texttt{-O}\\
\end{tabular}
\end{table}

\noindent Fördelen med att explicit ange en viss Scala-version är att byggprocessen blir \emph{upprepningsbar} även på en annan dator som kanske råkar har en annan Scala-version installerad. Det går att ''spika fast'' Scala-versionen till en ännu mer precis version, t.ex. \code{3.2.2}. Om inte den versionen av kompilatorn finns installerad på datorn så kommer Scala CLI att automatiskt ladda ner och använda den explicit efterfrågade versionen under byggandet.

Det går också att be om den absolut mest rykande färska kompilatorversionen om man vill använda det allra senaste i Scala-språkets utveckling med \code{3.nigthly}. Speciellt kräver s.k. experimentella funktioner att du använder \code{nightly}-versionen\footnote{\url{https://stackoverflow.com/questions/40622878/how-do-i-use-a-nightly-build-of-scala}}. 

\subsection{Generera dokumentation med Scala CLI}

Scala CLI kan också skapa dokumentation baserat på dokumentationskommentarer (se vidare Appendix \ref{appendix:doc}), enligt nedan. Med optionen \code{--ouput} kan du ange destinationskatalog och med \code{--force} så skrivs ev. gammal dokumentation över.
\begin{REPLsmall}
scala-cli doc . --output apidoc --force
\end{REPLsmall}

\subsection{Paketering av exekverbar fil med Scala CLI}

\noindent Scala CLI kan paketera din kod i en exekverbar fil så här:
\begin{REPLsmall}
scala-cli package . --force --standalone --output myapp
\end{REPLsmall}

\noindent Här förklaras några vanliga optioner som kan användas vid paketering: 
\begin{table}[H]
\begin{tabular}{l p{8.5cm}}
\texttt{-{}-output} & Ange namn på utfilen med paketerad kod. Optionen \texttt{-{}-output} kan förkortas med \texttt{-o}\\
\texttt{-{}-force} & Skriv över utfilen om den redan finns. Optionen \texttt{-{}-force} kan förkortas med \texttt{-f} \\
\texttt{-{}-standalone} & Skapa en självständig, exekverbar jar-fil med din kod och dess beroenden.\\
\texttt{-{}-library} & Skapa en jar-fil med din kod för användning av andra program.\\
\texttt{-{}-assembly} & Skapa en fet jar-fil med din kod och alla dess beroenden för användning av andra program.\\
\end{tabular}
\end{table}

\subsection{Optioner som användningsdirektiv i ''magiska'' kommentarer}

\noindent I stället för att använda optioner i terminalen så kan du ge dessa som s.k. användningsdirektiv \Eng{using directives} i ''magiska'' kommentarer som börjar med \code{//> using} i början av valfri kod-fil. 

Om du har flera kodfiler i samma katalog brukar man skapa en speciell fil som vanligtvis kallas \code{project.scala} och i den samla alla användningsdirektiv som styr byggandet. Här visas ett exempel hur det kan se ut:
\begin{Code}
//> using scala 3.3
//> using option -unchecked -deprecation 
//> using option -Wunused:all -Wvalue-discard -Wsafe-init
//> using dep se.lth.cs::introprog::1.3.1
\end{Code}

\noindent De kompilatoroptioner som föreslås för att få extra varningar ovan har följande betydelser:
\begin{table}[H]
\begin{tabular}{l p{8.5cm}}
\texttt{-unchecked} & Extra varningar vid flera fall av osäker kod. \\
\texttt{-deprecation} & Förklaring vid användning av utgående funktioner. \\
\texttt{-Wunused:all} & Varning om deklarationer ej används. \\
\texttt{-Wvalue-discard} & Varning vid förlorat värde. \\
\texttt{-Wsafe-init} & Varna vid risk för ej initialiserade värden. \\
\end{tabular}
\end{table}

\noindent Du hittar mer information om Scala CLI här: \url{https://scala-cli.virtuslab.org/}



\section{Scala Build Tool \texttt{sbt}}\label{appendix:build:sbt}

Byggverktyget \sbt\ är skrivet i Scala och är det mest använda byggverktyget bland Scala-utvecklare. Med \sbt\ kan du skriva byggkonfigurationsfiler i Scala och även styra byggprocessen via ett interaktivt kommandoskal i terminalfönstret. Med inkrementell (stegvis) kompilering och parallellkörning av byggprocessens olika delar, kan den snabbas upp avsevärt.


\subsection{Installera sbt}

\sbt\ finns förinstallerat på LTH:s datorer och körs igång med kommandot \sbt\ i terminalen.

Om du vill installera \sbt\ på din egen dator,
säkerställ först att du har \code{java} på din dator med terminalkommandot \code{java -version}. Om \code{java} saknas, följ instruktionerna i avsnitt \ref{appendix:compile:install-jdk} på sidan \pageref{appendix:compile:install-jdk}.
Följ sedan instruktionerna här för att installera \sbt: \url{http://www.scala-sbt.org/download.html}

\begin{itemize}

\item \textbf{Linux}. Om du surfar till ovan sida från en Linux-dator syns några terminalkommando som du använder för att installera \sbt\ i terminalen.

\item \textbf{Windows}. Om du surfar till ovan sida från en Windows-dator visas en länk till en \code{.msi}-fil. Ladda ner och dubbelklicka på den. Innan du kör igång med sbt i en Windowsterminal är det bra att skriva \code{chcp 65001} för att särskilda tecken (t.ex. ÅÄÖ) ska fungera som de ska.

\item \textbf{macOS}. Följ instruktionerna under rubriken \textit{Manual Installation}.

\end{itemize}

\noindent När du kör sbt första gången kommer ytterligare filer att laddas ner och installeras och delar av denna process kan ta lång tid. Ha tålamod och avbryt inte körningen, även om inget speciellt ser ut att hända på ett bra tag.

%% Below is problematic for some libs noty compiled for 2.11.x as it causes dependency problems...
%\subsection{Anpassa sbt}
%För att följa de versioner av \sbt\ och Scala som vi använder i kursen, skapa med hjälp av editor en textfil med namnet \code{global.sbt} i katalogen \code{.sbt} som ligger i din hemkatalog efter att du installerat klart \sbt. Fråga vid behov någon om hjälp om hur man hittar dolda filer i ditt operativsystem, då filer som börjar med punkt ibland inte syns i filbläddraren. Filen ska ha följande innehåll:
%\begin{Code}
%scalaVersion := "2.11.8"
%
%sbtVersion := "0.13.12"
%\end{Code}
%
%\noindent När du kör igång \sbt\ igen kommer ovan inställningar eventuellt medföra vissa nedladdningar, men när det är gjort har du rätt versioner tillgängliga och \sbt\ kommer att starta snabbt nästa gång.


\subsection{Använda sbt}
\sbt\ är konstruerat för att klara mycket stora projekt, men det är enkelt att använda \sbt\ även om du bara har ett litet projekt med någon enstaka kodfil. Med \sbt\ installerat, är det bara att köra igång \sbt och skriva \texttt{run} enligt nedan
\begin{REPLnonum}
> sbt
sbt> run
\end{REPLnonum}
i terminalen i den katalog där dina kodfiler ligger. \sbt\ letar då upp och kompilerar alla de \code{.scala}-filer som ligger i katalogen och, om det bara finns ett objekt med main-metod, kör \sbt\ igång denna main-metod direkt, förutsatt att kompileringen kan avlutas utan fel. Även \code{.java}-filer kompileras automatiskt om de ligger i samma katalog.

Om du enbart skriver \sbt\ körs det interaktiva kommandoskalet igång, där du kan köra kommando så som \code{compile} och \code{run}. Om du skriver ett \code{~} före kommandot \code{run}, enligt nedan kommer \sbt\ vara aktivt i bakgrunden medan du editerar och så fort du sparar en ändring kommer omkompilering av ändrade kodfiler ske, varefter main-metoden exekveras om kompileringen lyckades.

\begin{REPLnonum}
> sbt
[info] Set current project to hello (in build file:/home/bjornr/hello/)
> ~run
[info] Running hello
Hello, World!
[success] Total time: 0 s, completed Aug 9, 2016 9:50:16 PM
1. Waiting for source changes... (press enter to interrupt)
[info] Compiling 1 Scala source to /home/bjornr/hello/target/scala-2.10/classes...
[info] Running hello
Hello again, World!
[success] Total time: 1 s, completed Aug 9, 2016 9:50:45 PM
2. Waiting for source changes... (press enter to interrupt)
\end{REPLnonum}

\noindent I ovan körning gör \sbt\ en omkompilering, efter att en ändring av utskriftssträngen sparats.
\begin{figure}[H]
\begin{Code}
// in file hello.scala

@main def run = println("Hello again, World!") // add 'again'; Ctrl+S
\end{Code}
\end{figure}

\subsubsection{Katalogstruktur}

Om man har kod i underkataloger förutsätter \sbt\ att du följer en viss, specifik katalogstruktur. Denna katalogstruktur används även av andra byggverktyg, så som Maven, och fungerar även i många utvecklingsmiljöer så som Eclipse och IntelliJ.

Det blir också mindre rörigt och lättare för alla att hitta i projektets kataloger om dina kodfiler placeras i en given struktur som är allmänt accepterad.
Placera därför gärna dina kodfiler i underkataloger enligt strukturen som visas i figur \ref{fig:sbt:dir-structure}.

\begin{figure}[H]
\centering

\begin{lstlisting}[frame=none, backgroundcolor=]
					src/
					  main/
					    resources/
					       <files to include in main jar here>
					    scala/
					       <main Scala sources>
					    java/
					       <main Java sources>
					  test/
					    resources
					       <files to include in test jar here>
					    scala/
					       <test Scala sources>
					    java/
					       <test Java sources>
\end{lstlisting}

\caption{Katalogstrukturen i ett \sbt-projekt. Bara de kataloger som har något innehåll behöver finnas.}
\label{fig:sbt:dir-structure}
\end{figure}

\noindent Lägg enligt denna struktur dina \code{.scala}-filer i underkatalogen \code{src/main/scala/} och dina \code{.java}-filer i underkatalogen \code{src/main/java/}. Om du lägger kod i någon av katalogerna \code{src/test/scala/} respektive \code{src/test/java/} kommer denna kod köras när du skriver \sbt-kommandot \code{test}. Om du lägger filer i underkatalogen \code{src/main/resources/} kommer dessa att paketeras med i jar-filen som skapas när du kör \sbt-kommandot \texttt{package}.

Om du använder t.ex. \code{package x.y.z;} i din Java-kod, måste även strukturer på underkataloger matcha och kodfilen alltså ligga i  \code{src/main/java/x/y/z/}.

I Scala är det egentligen inte nödvändigt att koden ligger i samma katalog som de kompilerade \texttt{.class}-filerna, men det kan vara bra att följa paketstrukturen även för Scala-källkoden; speciellt om du senare vill kunna köra din kod med Eclipse, som kräver denna överensstämmelse mellan paket och källkodskataloger, inte bara för Java, utan även för Scala.

\subsubsection{Konfigurera dina byggen i filen \code{build.sbt}}

Om du vill göra inställningar och även hjälpa andra att kunna återskapa dina byggen, så skapa en konfigurationsfil med namnet \code{build.sbt} och placera den i projektets baskatalog. Figur \ref{fig:sbt:build-file} visar en byggkonfigurationsfil som specificerar vilken version av Scala-kompilatorn du använder, så att andra ska kunna bygga din kod under samma förutsättningar som du.

\begin{figure}[H]
\centering
\begin{Code}
scalaVersion := "3.2.2"
\end{Code}
\caption{Exempel på konfigurationsfil för \sbt. Filen ska ha namnet \code{build.sbt} och vara placerad i projektets baskatalog.}
\label{fig:sbt:build-file}
\end{figure}

\noindent Här är ett exempel på en mer omfattande \code{build.sbt}:
\begin{CodeSmall}
scalaVersion   := "3.2.2"
scalacOptions  := Seq("-unchecked", "-deprecation") //mer info vid kompilering

fork           := true   // kör i en egen JVM, bra om ljud och grafik används 
connectInput   := true   // koppla indata till rätt JVM vid fork
outputStrategy := Some(StdoutOutput)  // koppla utdata till rätt JVM vid fork

ThisBuild / useSuperShell := false // stänger av rör(l)ig progressinformation
\end{CodeSmall}

\noindent Du kan läsa mer om alla möjligheter med \sbt\ och hur man skapar mer avancerade byggkonfigurationsfiler här: \url{http://www.scala-sbt.org}

% Du hittar ett exempel på en avancerad byggdefinition i kursens repo, som har många aggregerade underprojekt, bl.a. för att bygga detta kompedium med \code{pdflatex}. I byggdefinitionen instrueras även \sbt\ att bygga kursens workspace, samt att generera de speciella projektfiler som Eclipse+ScalaIDE kräver med en \sbt-plugin. Filen finns här: \\
% \url{https://github.com/lunduniversity/introprog/blob/master/build.sbt}

\subsubsection{Fixera versionen för sbt i \code{project/build.properties}}
Om du skapar en katalog \code{project} (om den inte redan finns) kan du i en fil med namnet \code{build.properties} fixera versionen av sbt genom att låta filen ha detta innehåll (notera punkten och avsaknaden av citationstecken):
\begin{Code}
sbt.version=1.8.2
\end{Code}
På så sätt riskerar du inga inkonsekvenser mellan en gammal \code{build.sbt} vid framtida uppdatering av sbt, ovan inställning garanterar att ditt bygge alltid kommer att byggas med denna version av sbt, och andra kan bygga din kod under samma förutsättningar som du.

\subsubsection{Lägga till kursbiblioteket \texttt{introprog} som ett beroende}

Med följande text i \code{build.sbt} får du automatisk nedladdning och tillgång till kursens Scala-bibliotek \texttt{introprog} med bl.a. klassen \code{PixelWindow} för grafiska fönster:

\begin{Code}
scalaVersion := "3.2.2"
libraryDependencies += "se.lth.cs" %% "introprog" % "1.3.1"
\end{Code}
Ändra ev. versionsnummer till senaste versionen. Notera de dubbla procent-tecknen före biblioteksnamnet, som används för Scala-bibliotek som kors-publicerats för olika versioner av Scala, t.ex. 3, 2.12 och 2.13, vilket gör att rätt biblioteksversion för rätt kompilatorversion laddas ned.

Du kan läsa mer om \code{introprog} här: 
\begin{itemize}
  \item Kod: \url{https://github.com/lunduniversity/introprog-scalalib}
	\item Dokumentation: \url{http://cs.lth.se/pgk/api}
\end{itemize}


\subsubsection{Lägga till andra beroenden}

I filen \texttt{build.sbt} kan man lägga till många beroenden till flera olika kodbibliotek. Det finns på nätplatsen \textit{Maven Central} en mycket omfattande koddatabas, som är sökbar här \url{http://search.maven.org}, med en massa användbara öppenkällkodsprojekt. Du kan be \sbt\ att ladda ner den färdigkompilerade koden till vilket som helst av projekten på \textit{Maven Central} och automatiskt lägga till jar-filen till \code{classpath} så att koden blir tillgänglig direkt i ditt program.

Till exempel kan du lägga till Java-biblioteket \code{jline} som gör det möjligt att göra terminalinläsning från tangentbordet med många bra finesser, t.ex. kommandohistorik med pil-upp, bara genom att lägga till denna rad i din \code{build.sbt} och den specifika version du önskar (notera enkla procent-tecken för Java-bibliotek):

\begin{Code}
libraryDependencies += "org.jline" % "jline" % "3.20.0"
\end{Code}
Du kan läsa mer om \code{jline} här: \url{https://jline.github.io/}

