%!TEX encoding = UTF-8 Unicode
%!TEX root = ../compendium.tex

\chapter{Dokumentation}\label{appendix:doc}

Dokumentation hjälper andra att använda din kod, men underlättar även för dig själv när du vid ett senare tillfälle ska erinra dig hur den fungerar och hur du ska använda och bygga vidare på din kod. Modern systemutveckling baseras ofta på öppen källkod och färdiga api \Eng{application programming interface}, där kvaliteten på dokumentationen är avgörande för hur lätt det är att komma igång med att använda koden.

Nedan listas exempel på olika typer av  dokumentation\footnote{\href{https://en.wikipedia.org/wiki/Software_documentation}{en.wikipedia.org/wiki/Software\_documentation}}:

\begin{itemize}
\item \textbf{Kravdokumentation} beskriver det övergripande målet med mjukvaran, samt funktionella krav och kvalitetskrav som uppfylls av systemet.
\item \textbf{Designdokumentation} beskriver arkitekturen, hur koden är organiserad i moduler, och den interna systemstrukturen t.ex. i form av klasser, objekt och deras relation.
\item \textbf{Slutanvändardokumentation} kan t.ex. vara manualer för användning av systemet och installationsanvisningar.
\item \textbf{Teknisk dokumentation} kan t.ex. vara api-dokumentation som beskriver vilka funktioner som ingår i ett programbibliotek. Sådan dokumentation genereras ofta med hjälp av ett \textbf{dokumentationsverktyg} (se avsnitt \ref{appendix:buildtool}).  Andra typer av teknisk dokumentation är instruktioner om hur man bygger koden med eventuellt tillhörande byggverktygskonfigurationsfiler; ofta beskrivs byggförfarandet steg för steg i en textfil med namnet \code{README}. (Läs mer om byggverktyg i appendix \ref{appendix:build}.)
\end{itemize}

\noindent Det är en stor utmaning att hålla dokumentationen uppdaterad allteftersom koden utvecklas. Även om man får hjälp att generera en navigerbar sajt av ett dokumentationsverktyg, måste själva \textit{innehållet} i de manuellt författade dokumentationskommentarerna vara i överensstämmelse med den aktuella versionen av koden. Uppdateras koden, måste man alltså vara noga med att uppdatera dokumentationskommentarerna, annars uppstår stor förvirring.

Detta problem är så pass allvarligt att man ska tänka sig noga för hur man kan formulera  dokumentationskommentarerna på ett framtidssäkert sätt, och hur omfattande de ska vara i förhållande till den framtida arbetsinsatsen med att hålla dem uppdaterade. Desto mer omfattande kommentarer desto mer jobb att hålla dem uppdaterade.

Det är i praktiken svårt att uppnå en optimal balans mellan bra och många kommentarer som \textit{hjälper} användaren, och å andra sidan svårunderhållna och föråldrade kommentarer som \textit{stjälper} användare.


\section{Vad gör ett dokumentationsverktyg?}\label{appendix:buildtool}

Ett dokumentationsverktyg genererar teknisk dokumentation av koden baserat på speciella \textbf{dokumentationskommentarer} som skrivs i koden omedelbart före deklarationer av det som ska dokumenteras. Dessa dokumentationskommentarer skrivs enligt en speciell syntax som dokumentationsverktyget kan tolka.

Utdata från ett dokumentationsverktyg utgörs typiskt av en webbsajt med ändamålsenlig formatering och navigationslänkar, se figur \ref{fig:appendix:doctool}.

\begin{figure}[H]
\centering
\begin{tikzpicture}[node distance=1.8cm, scale=1.5]
\node (input) [startstop] {\bf\sffamily Källkod};
\node(inptext) [right of=input, text width=6cm, xshift=4.2cm]{med speciella dokumentationskommentarer före deklarationer};
\node (compile) [process, below of=input] {\bf\sffamily Dokumentationsverktyg};
%\node(explain) [right of=compile, text width=5cm, xshift=3.0cm]{Översätter från källkod till maskinkod};
\node (output) [startstop, below of=compile] {\bf\sffamily Dokumentation};
\node(outtext) [right of=output, text width=6cm, xshift=4.2cm]{t.ex. en webbsajt med dokumentation och navigationslänkar};
\draw [arrow] (input) -- (compile);
\draw [arrow] (compile) -- (output);
\end{tikzpicture}
    \caption{Ett dokumentationsverktyg läser koden och dokumentationskommentarer och genererar dokumentation, t.ex. i form av en webbsajt.}
    \label{fig:appendix:doctool}
\end{figure}



\section{scaladoc}
\newcommand{\scaladoc}{\texttt{scaladoc}}

Med Scala-installationen följer dokumentationsverktyget \scaladoc, som genererar en webbsajt med ändamålsenlig layout och specialfunktioner för att söka, filtrera och navigera i dokumentationen.

Dokumentationen av stora bibliotek kan bli omfattande och det krävs träning i att använda dokumentationssajter för att få maximal nytta av dem. I efterföljande avsnitt beskrivs först hur du använder dokumentation som är genererad med \scaladoc. Därefter visas hur du själv kan generera dokumentation för din egen kod.


\subsection{Använda dokumentation från scaladoc}

Dokumentationen av Scalas standardbiliotek är genererad med \scaladoc\ och att navigera i denna ger bra träning i hur man använder avancerad api-dokumentation. Du hittar dokumentationen för Scalas standardbibliotek här: \\
\url{http://scala-lang.org/api/current}


När du surfar dit möts du av dokumentationen för \textit{root package}, som ger en översikt av olika paket i standardbiblioteket. I sökrutan uppe till vänster kan du skriva början på namnet på klasser, traits, eller objekt som du letar efter, så som visas i figure \ref{fig:scaladoc:root-package}.

\begin{figure}[H]
\centering
\includegraphics[width=1.0\textwidth]{../img/scaladoc/scaladoc-root}

     \caption{Förstasidan med dokumentationen av standardbiblioteket i Scala.}
    \label{fig:scaladoc:root-package}
\end{figure}

Genom att skriva text i sökrutan får du en filtrerad lista på allt som har ett namn som börjar på det söker efter. I figur \ref{fig:scaladoc:vec} visas en sökning efter \code{vec}.

\begin{figure}[H]
\centering
\includegraphics[width=1.0\textwidth]{../img/scaladoc/scaladoc-vec}

     \caption{ Sökning efter innehåll som börjar på \code{vec}.}
    \label{fig:scaladoc:vec}
\end{figure}

Om du klickar på \code{Vector} får du se dokumentationen i figur  \ref{fig:scaladoc:vector}

\begin{figure}[H]
\centering
\includegraphics[width=0.9\textwidth]{../img/scaladoc/scaladoc-vector}

     \caption{En del av dokumentationen av klassen \code{Vector}}
    \label{fig:scaladoc:vector}
\end{figure}

Genom att skriva text i den nedre, mörkgrå sökrutan kan du filtrera listan med klassmedlemmar. Om klickar på länken \code{object Vector}, eller på den runda, gröna ikonen med ett stort C, får du se kompanjonsobjektets medlemmer.
Du kan få fram en filtreringsruta med fler möjligheter genom att klicka på expanderingspilen i den mörkgrå sökrutan så som visas i figur \ref{fig:scaladoc:filter}.



\begin{figure}[H]
\centering
\includegraphics[width=0.75\textwidth]{../img/scaladoc/scaladoc-filter}

     \caption{Expanderad sökruta med extra filtreringsmöjligheter.}
    \label{fig:scaladoc:filter}
\end{figure}

%Det finns fler sökmöjligheter och finesser i \scaladoc. Prova att klicka på olika länkar och symboler och upptäck vad du kan få fram för information. Träning i att använda \scaladoc~gör dig till en mer produktiv Scala-utvecklare.

\clearpage

\subsection{Skriva dokumentationskommentarer}


Verktyget \scaladoc\ läser kommentarer som börjar med \verb|/**| och slutar med \verb|*/| och associeras till efterföljande deklaration. Notera de dubbla asteriskerna. Alla rader som följer efter \verb|/**| ska, enligt konventionen för Scalas dokumentationskommentarer, börja med en asterisk \code|*| med indrag med flera blanksteg så att den hamnar under \textit{andra} asterisken i öppningskommentaren, som nedan:
\begin{Code}
/** Först kommer en sammanfattning på en enda rad.
  *
  * Sedan kommer eventuellt en mer detaljerad beskrivning,
  * som kan vara flera rader lång.
  */
\end{Code}
Dokumentationskommentaren slutar med \code|*/| rakt under asterisk-kolumnen.

\begin{itemize}
\item Med \verb|@constructor| i början på en rad ges en speciell kommentar om konstruktorer. 
\item Med \verb|@param| i början på en rad ges en speciell kommentar angående parametrar. 
\item Med \verb|@return| i början av en rad ges en speciell kommentar angående vad som returneras vid metodanrop.
\item Länkar skrivs inom dubbla hakparenteser, enligt exempel nedan.
\end{itemize}

\begin{Code}
/** A person who uses our application.
 *
 *  @constructor create a new person with a name and age.
 *  @param name the person's name
 *  @param age the person's age in years
 */
class Person(name: String, age: Int)  

/** Factory for [[mypackage.Person]] instances. */
object Person:
  /** Creates a person with a given name and age.
   *
   *  @param name their name
   *  @param age the age of the person to create
   */
  def apply(name: String, age: Int) = ???

  /** Creates a person with a given name and birthdate
   *
   *  @param name their name
   *  @param birthDate the person's birthdate
   *  @return a new Person instance with the age determined by the
   *          birthdate and current date.
   */
  def apply(name: String, birthDate: java.util.Date) = ???

\end{Code}


Läs mer här om hur du skriver dokumentationskommentarer: \\
\url{https://docs.scala-lang.org/scala3/guides/scaladoc/}

Läs mer om dokumentation här:\\
\url{https://docs.scala-lang.org/scala3/scaladoc.html}\\
\url{https://scala-cli.virtuslab.org/docs/commands/doc}



\subsection{Generera dokumentation}


Du genererar dokumentation enklast med hjälp av körverktyget \code{scala-cli}. 
I terminalen skriv \\\code{scala-cli doc . -o api} 

När \code{scala-cli} är färdig med att generera dokumentationen så meddelas vilken katalog som dokumentationen ligger i. För att länkarna inom dokumentationen ska fungera krävs antingen att du kör en lokal webbserver i katalogen eller att du använder ett program för att konvertera länkarna till lokala sådana.

\subsubsection{Använda en lokal webbserver}
Om du har Python 3 installerat på din dator har du en inkluderad webbserver. Du startar denna i terminalen med \\\code{python -m http.server} när du står i dokumentationens katalog. För att öppna dokumentationen besöker du sedan \url{http://localhost:8000} i din webbläsare.

\subsubsection{Använda ett program för att konvertera länkar}
Om du inte har Python installerat kan du köra ett Scala-program som byter ut alla länkar till lokala sådana, vilket gör att de går att öppna direkt. Detta kan laddas ner från \\
\url{https://github.com/dixine55/Scaladoc-Local-Version-Patcher/blob/main/scaladocPatch.scala}. Placera programmet i dokumentationsmappen och kör det med \\\code{scala-cli run .} när du är i dokumentationens mapp. Du öppnar sedan dokumentationen genom filen \\\texttt{index.html} i din webbläsare.

Mer att läsa om att generera dokumentation finns här: \\
\url{https://scala-cli.virtuslab.org/docs/commands/doc}


% med terminalkommandot \scaladoc\ följt av en eller flera källkodsfiler. Med optionen \code{-d} anger du i vilken katalog sajten ska sparas. Du visar sajten genom att öppna filen \code{index.html} i en webbläsare. Nedan visas hur dokumentationen genereras för källkodsfilen i figur \ref{fig:scaladoc:mio}.
% \begin{REPLnonum}
% $ scaladoc mio.scala -d apidoc
% $ firefox apidoc/index.html
% \end{REPLnonum}

% I figur \ref{fig:scaladoc:webpage} på sidan \pageref{fig:scaladoc:webpage} visas delar av en webbsida som genererats utifrån koden i figur \ref{fig:scaladoc:mio} på sidan \pageref{fig:scaladoc:mio}. För de publika metoder där ingen dokumentationskommentar finns, visas ändå metodens signatur med parametrar, parametertyper, och returtyp. %Medlemmar som deklareras \code{private} visas inte, men om man klickar på knappen \Button{All} bredvid rubriken \textbf{Visibility} visas medlemmar som är deklarerade \code {protected}.

% %Om du klickar på symbolen \Forward\ till vänster om metodsignaturen, ändras den till symbolen \MoveDown\ som indikerar att den mer detaljerade beskrivningen av parametrar etc. har vecklats ut (i den mån detaljerade kommentarer finns).

% Om du vill ha övergripande dokumentation om ett paket \code{x}, ges det speciella objektet \code{package object x} en dokumentationskommentar med sådan information. Ofta innehåller \code{package object} medlemmar som man vill ska bli synliga vid import av paketet, så som variabler, metoder och implicita medlemmar som inte har någon annan naturlig hemvist.

% \begin{figure}[b]
% \scalainputlisting[numbers=left, basicstyle=\ttfamily\fontsize{9}{11}\selectfont]{../util/mio.scala}
%     \caption{Dokumentationskommentarer som kan kan användas för att generera en dokumentations-webbsajt. Sådana kommentarer börjar  med snedstreck och dubbla asterisker, se bl.a. raderna 8--13 ovan.}
%     \label{fig:scaladoc:mio}
% \end{figure}

% \begin{figure}[t]
% \includegraphics[width=1.0\textwidth]{../img/scaladoc/scaladoc-mio}
%     \caption{Delar av en webbsida genererad med hjälp av \scaladoc. %Mer detaljerade beskrivningar kan i förekommande fall vecklas ut eller in om man växlar mellan \Forward\ och \MoveDown.
%     }
%     \label{fig:scaladoc:webpage}
% \end{figure}


%\subsection{Lära mer om scaladoc}

% Fler tips om \scaladoc:
% \begin{itemize}%[leftmargin=*]

% %\item En video med tips om hur du söker och navigerar i \scaladoc-dokumentation:
% %\\
% %\url{http://docs.scala-lang.org/overviews/scaladoc/interface.html}

% \item
% Riktlinjer för hur du skriver dokumentationskommentarer: \\
% \url{http://docs.scala-lang.org/style/scaladoc.html}

% % \item Länksida till mer detaljerade beskrivningar: \\
% % \url{https://wiki.scala-lang.org/display/SW/Writing+Documentation} \\
% % inkluderande bland annat:
% %
% % \begin{itemize}[nolistsep]
% % \item En beskrivning av syntaxen för formatering: \\
% % \url{https://wiki.scala-lang.org/display/SW/Syntax}
% %
% % \item En beskrivning av speciella annoteringar, t.ex. \code{@param}: \\
% % \url{https://wiki.scala-lang.org/display/SW/Tags+and+Annotations}
% %
% % \end{itemize}

% \item Kör kommandot \texttt{ scaladoc -help } för att se användbara optioner.

% \item Kommandot \texttt{sbt doc} är ett smidigt sätt att automatiskt generera api-dokumentation för alla dina kodfiler om du placerat dem i underbiblioteket \code{src/main/scala} resp. \code{src/main/java}.
% Läs mer om \texttt{sbt} och api-dokumentation här: \\
% \url{http://www.scala-sbt.org/1.x/docs/Howto-Scaladoc.html}


%\end{itemize}

% \clearpage

% \section{javadoc}
% \newcommand{\javadoc}{\texttt{javadoc}}


% Med Java JDK följer dokumentationsverktyget \javadoc, som utifrån dokumentationskommentarer i Java-kod genererar en webbsajt med navigationslänkar. Webbsidor genererade med \javadoc\ erbjuder inte samma funktioner för sökning och filtrering som \scaladoc, men det fungerar bra att hitta det man söker om navigationslänkarna används tillsammans med webbläsarens inbyggda sökfunktion (Ctrl+F).

% \subsection{Använda dokumentation genererad med javadoc}

% I figur \ref{fig:javadoc:overview} visas exempel på \javadoc\ för biblioteket \code{cslib}. Om du klickar på ett paket kan du navigera till en översikt av innehållet i paketet. Om du klickar på en klass får du en översikt av klassens medlemmar, så som visas i \ref{fig:javadoc:class}.  Om du t.ex. klickar på ett metodnamn får du se mer detaljerade kommentarer.

% Ramarna till vänster på webbsidorna innehåller länkar till paket och klasser. Om du klickar på länken \textit{All Classes} överst till vänster får du en lista med navigationslänkar till alla tillgängliga klasser. De gulmarkerade rubrikerna visar vilken vy som är aktiv och navigationslänkar skrivs med blå text.

% \begin{figure}
% \includegraphics[width=1.0\textwidth]{../img/javadoc/javadoc-overview}
%     \caption{Delar av en webbsida genererad med hjälp av \javadoc.}
%     \label{fig:javadoc:overview}
% \end{figure}



% \begin{figure}
% \includegraphics[width=1.0\textwidth]{../img/javadoc/javadoc-class}
%     \caption{Delar av en webbsida med klassdokumentation genererad med hjälp av \javadoc.}
%     \label{fig:javadoc:class}
% \end{figure}






% \subsection{Skriva dokumentationskommentarer för javadoc}

% Kommentarer för \javadoc\ och \scaladoc\ ser ganska lika ut, även om det finns några skillnader. Det finns t.ex. inte lika många styrtecken för layouten i \javadoc\ som i \scaladoc, och konventionen i Java är fyra blankstegs indrag och att fortsättningsrader i dokumentationskommentarer börjar asterisken under \textit{första} asterisken i öppningskommentaren.

% Nedan visas delar av \javadoc-kommentarerna för klassen \code{SimpleWindow} och dess konstruktor:
% \begin{Code}[language=Java]
% package cslib.window;

% /** A simple window to draw in */
% public class SimpleWindow {
%    /**
%     * Creates a window and makes it visible.
%     *
%     * @param width   the width of the window
%     * @param height  the height of the window
%     * @param title   the title of the window
%     */
%     public SimpleWindow(int width, int height, String title) {
%         ...
% \end{Code}
% Annoteringen \verb|@param| i början på en rad ger en speciell kommentar angående en parameter. Vid dokumentation av metoder kan annoteringen \verb|@return| användas i början av en rad för att skapa en speciell kommentar angående vad som returneras.

% Övergripande dokumentation om innehållet i ett paket läggs i en textfil i paketets katalog med namnet \texttt{package-info.java}, se till exempel här: \\ {\href{https://github.com/lunduniversity/introprog/tree/master/workspace/cslib/src/main/java/cslib/window}{\small github.com/lunduniversity/introprog/tree/master/workspace/cslib/src/main/java/cslib/window}


% Du kan läsa mer om hur man skriver \code{javadoc}-kommentarer här:\\
% \href{http://www.oracle.com/technetwork/java/javase/documentation/index-137868.html}{www.oracle.com/technetwork/java/javase/documentation/index-137868.html}

% % conversation of diffs between javadoc and scaladoc:
% %  https://groups.google.com/forum/#!msg/scala-user/q-Vw03zcIVs/CaTR5XL-BQAJ

% \subsection{Generera dokumentationskommentarer för javadoc}

% Om du står i den katalog där din källkod finns, kan du med nedan kommando i terminalen gå igenom alla paket och underpaket och generera \javadoc-webbsidor i katalogen \texttt{doc}. Du kan därefter öppna dokumentationen i en webbläsare.
% \begin{REPLnonum}[basicstyle=\color{white}\ttfamily\fontsize{9}{11}\selectfont]
% $ javadoc -d doc -encoding UTF-8 -charset UTF-8 -sourcepath . -subpackages . *
% $ firefox doc/index.html
% \end{REPLnonum}
% Ett smidigt sätt att generera både \scaladoc\ och \javadoc\ är att använda \texttt{sbt}; det är bara att skriva \code{ sbt doc } i terminalen så genereras alla dokumentation för både Scala och Java i den katalog som \texttt{sbt} meddelar i sin resultatutskrift.

% Om du lägger in nedan i \code{settings} i din \code{build.sbt} fungerar även svenska bokstäver och andra specialtecken på alla plattformar.
% \begin{Code}
%   javacOptions in (Compile, doc) ++= Seq(
%     "-encoding", "UTF-8",
%     "-charset", "UTF-8",
%     "-docencoding", "UTF-8")
% \end{Code}

% \noindent Du kan också använda din IDE för att köra \javadoc. I Eclipse, använd menyn \MenuArrow{Project}\Menu{Generate Javadoc...}, medan du i IntelliJ hittar motsvarande i menyn \MenuArrow{Tools}\Menu{Generate Javadoc...}
