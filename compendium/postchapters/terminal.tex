%!TEX encoding = UTF-8 Unicode
%!TEX root = ../compendium2.tex

\chapter{Terminalfönster}\label{appendix:terminal}

\section{Vad är ett terminalfönster?}

I ett terminalfönster kan man skriva kommandon som kör program och hanterar filer. När man programmerar använder man ofta terminalkommandon för att kompilera och exekvera sina program.  
 
\subsubsection{Terminal i Linux}

    \begin{figure}[!b]
    \centering
    \includegraphics[width=1.0\textwidth]{../img/linux-terminal.png}
    \caption{Terminalfönster i Ubuntu öppnas med Ctrl+Alt+T.}
    \label{fig:terminal:linux}
    \end{figure}

I Ubuntu trycker du lättast \textbf{Ctrl+Alt+T} eller sök efter ''terminal'' i app-menyn.  Då öppnas ett fönster med en blinkande markör som visar att det är redo att ta emot dina textkommando. Ett exempel på kommando är \texttt{ls} som skriver ut en lista med filer i den aktuella katalogen, så som visas i fig. \ref{fig:terminal:linux}.

Det som visas i ett terminalfönster sköts av ett \textbf{kommandoskal} \Eng{command shell}, som är redo att ta emot kommando efter en prompt som slutar med ett \texttt{\$}-tecken. När du skriver ett kommando och trycker Enter anropar kommandoskalet en kommandotolk som tolkar och utför dina kommandon. Om ett kommando inte kan tolkas, skrivs ett felmeddelande. 

Det finns många användbara kortkommando, varav de viktigaste visas i tabell \ref{fig:terminal:shortcuts}. Det är bra om du lär dig dessa kortkommandon utantill så att ditt arbete i terminalen går snabbt och smidigt.

\begin{table}[H]
\renewcommand{\arraystretch}{1.15}
\begin{tabular}{@{}r | l}
pil upp/ner & bläddra i kommandohistoriken \\
Tab & ''auto-complete'', fyll i resten baserat på vad du skrivit hittills \\
Tab Tab & två tryck på Tab listar flera alternativ, om så finnes \\
Ctrl+A & ''ahead'', flytta markören till början av raden \\
Ctrl+E & ''end'', flytta markören till slutet av raden \\
Ctrl+K & ''kill'', ta bort tecken från markören till radens slut\\
Ctrl+U & ''undo'', ta bort tecken från markören till början av raden \\
Ctrl+Y & ''yank'', sätt in det som senast togs bort\\
Ctrl+Z & ''zleep'', stoppa pågående process, skriv sedan \texttt{bg} för bakgrundskörning\\
Ctrl+L & rensa terminalfönstret\\
Ctrl+D & avsluta kommandoskalet \\
\end{tabular}
    \caption{Viktiga kortkommandon i Linux terminalfönster.}
    \label{fig:terminal:shortcuts}
\end{table}

\noindent Ctrl+C orsakar normalt ett avbrott av pågående process och istället är \emph{paste} kopplat till Shift+Ctrl+C, men om du vill tvärtom att Ctrl+C ska vara ''Copy'' som vanligt för att kopiera markerad text och göra avbrott med Shift+Ctrl+C , så kan du ställa om detta med terminalfönstrets meny ''Edit $\rightarrow$ Keyboard Shortcuts'', eller liknande.




 
\subsubsection{PowerShell, Cmd och Linux i Microsoft Windows}
Det finns flera olika sätt att köra terminalkommando i Windows:

\begin{itemize}
\item \textbf{Powershell}. I Microsoft Windows finns kommandotolken \textit{Powershell} med speciell kommandosyntax. Den är inte Linux-baserad men det finns alias definierade för några vanliga Linux-kommandon, inkluderat \texttt{ls}, \texttt{cd} och \texttt{pwd}. Du startar Powershell t.ex. genom att trycka på Windows-knappen och skriva \texttt{powershell}. 
Du kan också, medan du bläddrar bland filer, klicka på filnamnsraden överst i filbläddraren och skriva \texttt{powershell} och tryck Enter; då startas Powershell i aktuellt katalog. %Ändra gärna typsnitt och bakgrundsfärg med hjälp av fönstrets menyer, så att det blir lättare för dig att läsa vad som skrivs.

\item \textbf{Cmd}. Det finns även i Windows den ursprungliga, gamla kommandotolken \textit{Cmd} med helt andra kommandon. Till exempel skriver man i Cmd kommandot \texttt{dir} i stället för \texttt{ls} för att lista filer. 

\item \textbf{WSL}. I både Windows 10 och 11 kan du även köra Ubuntu-terminalen med hjälp av Windows Linux Subsystem (WSL), vilket rekommenderas, speciellt om du inte har möjlighet att göra s.k. dual boot\footnote{Läs mer om dual boot här och be gärna någon om hjälp som gjort det förr:\\ \href{https://www.linuxtechi.com/dual-boot-ubuntu-22-04-and-windows-11/}{https://www.linuxtechi.com/dual-boot-ubuntu-22-04-and-windows-11/}}. 




\begin{itemize}[nolistsep]
\item Se vidare här om hur du kan installera WSL under Windows, (WSL2 rekommenderas före WSL1 om din maskin klarar det): 

\url{https://docs.microsoft.com/en-us/windows/wsl/install}

\item Det finns även ett smidigt tillägg till VS Code som heter Remote-WSL som gör att du kan editera filer i Windows som finns i WSL, se vidare här: 

\url{https://code.visualstudio.com/docs/remote/wsl-tutorial}

\end{itemize}

\item \textbf{Windows Terminal}. Den nya Microsoft-appen \textit{Windows Terminal} rekommenderas oavsett om du använder Powershell, Cmd eller WSL. Läs mer här om hur du installerar Windows Terminal: \\
  \url{https://docs.microsoft.com/en-us/windows/terminal/}

\end{itemize}







% \url{https://ubuntu.com/wsl} 

% Läs mer här: \href{https://www.omgubuntu.co.uk/2020/03/windows-10-linux-kernel-update}{www.omgubuntu.co.uk/2020/03/windows-10-linux-kernel-update}



\subsubsection{Terminal i Apple macOS/OS X}


Apple OS X och macOS är Unix-baserade operativsystem. De flesta vanliga terminalkommandon som fungerar i Linux fungerar också under Apple OS X och macOS. Du startar ett terminalfönster i Apples operativsystem genom att klicka på förstoringsglaset uppe till höger, skriva \texttt{terminal}, och trycka Enter.

\section{Vad är en path/sökväg?}\label{terminal:path}

När du skriver ett kommando i terminalen, eller kör vilket program som helst på din dator, behöver operativsystemet identifera i vilken fil programmets maskinkod ligger innan programmet kan köras. 

Lokaliseringen av filer sker med hjälp av en \textbf{sökväg} \Eng{path}, som anger en position i filsystemet. Ofta betraktas filsystemet som ett upp-och-ned-vänt träd, och kallas därför även ''filträdet''. Den ''översta'' positionen kallas ''rot'' \Eng{root} och betecknas med ett enkelt snedstreck \texttt{/}. Kataloger som ligger i kataloger utgör förgreningar i trädet. En sökväg pekar ut vägar genom trädet som behövs för att nå ''löven'', som utgörs av själva filerna.

Du kan se var ett program ligger i Linux med hjälp av kommandot \texttt{which} enligt nedan.\footnote{Skriv \texttt{ gcm ls } i Windows Powershell för motsvarighet till \texttt{ which ls } \\ Eller skriv \texttt{ New-Alias which get-command } för tillgång till kommandot \texttt{which} i Powershell. \\ \href{http://stackoverflow.com/questions/63805/equivalent-of-nix-which-command-in-powershell}{stackoverflow.com/questions/63805/equivalent-of-nix-which-command-in-powershell}} Listan med kataloger i sökvägen avskiljs med snedstreck.
\begin{REPLnonum}
$ which java
/usr/lib/jvm/oracle_jdk8/bin/java
$ which ls
/bin/ls
\end{REPLnonum}

En sökväg kan vara \textbf{absolut} eller \textbf{relativ}. En absolut sökväg utgår från roten och visar hela vägen från rot till destination, t.ex. \texttt{/usr/bin/firefox}, medan en relativ sökväg utgår från aktuellt katalog (där du ''står'') och börjar \textit{inte} med ett snedstreck.

Alla operativsystem håller reda på en mängd olika sökvägar för att kunna hitta speciella filer i filträdet. Dessa sökvägar lagras i s.k. \textbf{miljövariabler} \Eng{environment variables}. Det finns en \textit{speciell} miljövariabel som heter kort och gott \textbf{PATH}, i vilken alla sökvägar till de program finns, som ska vara tillgängliga för din användaridentitet direkt för exekvering genom sina filnamn, \textit{utan} att man behöver ange absoluta sökvägar. 

Du kan i Linux se vad som ligger i din PATH med kommandot \code{ echo $PATH } medan man i Windows Powershell skriver \code{$env:Path} där det bara är första bokstaven som ska vara en versal. I Linux separeras katalogerna i sökvägen med kolon, medan Windows använder semikolon.

Ibland kan du behöva uppdatera din PATH för att program som du installerat och ska bli allmänt tillgängliga. Detta görs på lite olika sätt i olika operativsystem, för Linux se t.ex. här:
\href{http://stackoverflow.com/questions/14637979/how-to-permanently-set-path-on-linux}{stackoverflow.com/questions/14637979/how-to-permanently-set-path-on-linux}

När man anger sökvägar finns några tecken med speciell betydelse:

\begin{tabular}{r  p{0.8\textwidth}}
\code|~| & ''tilde'', din hemkatalog \\
\code|/| & ''slash'', snedstreck anger filträdets rot om det finns i början av sökvägen, men utgör katalogsavskiljare inuti sökvägen \\
\code|.| & en punkt anger aktuell katalog, där du ''står'' \\
\code|..| & två punkter anger ett steg ''upp'' i filträdet \\
\code|"| & omgärda en sökväg med citationstecken, först och sist, om den innehåller annat än engelska bokstäver, t.ex. blanktecken\\
\code|\ | & \textit{backslash+blanktecken} används för att beteckna mellanslag i sökvägar som \textit{inte} omgärdas av citationstecken\\
\end{tabular}

\section{Några viktiga terminalkommando}

I tabell \ref{fig:terminal:commands} finns en lista med några viktiga terminalkommando som är bra att lära sig utantill.

En introduktion till LTH:s datorer med exempel på hur du använder vanliga Linux-kommandon finns i denna skrift \url{http://www.ddg.lth.se/perf/unix/} som används i introduktionsveckan för nybörjare på datateknikprogrammet vid LTH.

På sajten \url{http://ss64.com/} finns en mer omfattande lista med användbara terminalkommando och tillhörande förklaringarför för Linux (Bash), Windows (Powershell, Cmd) och Apple OS X (Bash).  

\begin{table}[H]
\renewcommand{\arraystretch}{1.25}
   
\begin{tabular}{@{}r | l}
\texttt{ls} & lista filer i aktuell katalog (alltså där du ''står'')\\
\texttt{ls} \textit{p}  & lista filer i katalogen  \textit{p} \\
\texttt{ls -A} & lista alla filer i aktuell katalog, även gömda \\
\texttt{man ls} & manual för kommandot \texttt{ls}; testa även \texttt{man} för andra kommandon! \\
\texttt{cd} \textit{p} & ''change directory'', ändra aktuell katalog till \textit{p}\\
\texttt{pwd} & ''print working directory'', skriv ut sökväg för aktuell katalog \\
\texttt{cp} \textit{p1 p2} & ''copy'', kopiera filen med path \textit{p1} till en ny fil kallad \textit{p2} \\
\texttt{mv} \textit{p1 p2} & ''move'', byt namn på filen \textit{p1} till \textit{p2}  \\
\texttt{rm} \textit{p} & ''remove'', ta bort filen \textit{p}\\
\texttt{rm -r} \textit{p} & ''remove recursive'', ta bort katalogen \textit{p} med allt innehåll; var försiktig!\\
\texttt{mkdir} \textit{p} & ''make dir'', skapa ett en katalog \textit{p}\\
\texttt{cat} \textit{p1 p2}& ''concatenate'', skriv ut hela innehållet i en eller flera filer \textit{p1 p2 etc.}\\
\texttt{less} \textit{p}& skriv ut innehållet i filen \textit{p}, en skärm i taget\\
\texttt{wget} \textit{url}&ladda ner \textit{url}, t.ex. \texttt{ wget http://cs.lth.se/pgk/ws -o ws.zip}\\
\texttt{unzip} \textit{p}& packa upp \textit{p}, t.ex. \texttt{ unzip ws.zip}\\
\end{tabular}

    \caption{Några viktiga terminalkommando i Linux. Med \textit{p}, \textit{p1}, \textit{p2}, etc.  avses en absolut eller relativ sökväg \Eng{path}, se avsnitt \ref{terminal:path}.}
    \label{fig:terminal:commands}

\end{table}

