\documentclass[a4paper,12pt,oneside]{memoir}
\usepackage[a4paper, total={16.2cm, 26.0cm}]{geometry}
\usepackage[utf8]{inputenc}
\usepackage{graphicx}
\usepackage[T1]{fontenc}
\usepackage[swedish]{babel}

\usepackage{microtype}
\usepackage{hyperref}
\hypersetup{hidelinks}
\usepackage{longtable}
\usepackage{booktabs}

% FONTS
\usepackage{tgtermes}

\usepackage{enumitem}
\setitemize{noitemsep,topsep=0pt,parsep=0pt,partopsep=0pt, leftmargin=*}

\pagenumbering{gobble}

\usepackage{url}
\usepackage{color}
%\newcommand{\TBD}{\colorbox{yellow}{\textbf{???}}}
%\newcommand{\TENTADATUM}{\colorbox{yellow}{8:e Januari, kl 08:00--13:00, se schema}}
%\newcommand{\KSDATUM}{\colorbox{yellow}{Tisdagen 24:e Oktober, kl 14:00--19:00, se schema}}
\newcommand{\LibVersion}{1.3.1} % latest version of introlib at https://github.com/lunduniversity/introprog-scalalib
\newcommand{\LibJar}{\texttt{introprog\_3-\LibVersion.jar}}
\newcommand{\JDKApiUrl}{\url{https://docs.oracle.com/en/java/javase/17/docs/api/}}
\newcommand{\CurrentYear}{2024}
\newcommand{\VMName}{vm2020} %TODO: update vm
\newcommand{\VMPassword}{pgkBytMig2020}
\newcommand{\VirtualBoxVersion}{7.0} %https://www.virtualbox.org/wiki/Downloads
\newcommand{\UbuntuVersion}{22.04}
\newcommand{\ScalaVersion}{3.4.2} %https://www.scala-lang.org/
\newcommand{\SbtVersion}{1.10.0} %https://eed3si9n.com/category/tags/sbt
\newcommand{\JDKVersion}{17} %https://adoptium.net/temurin/releases/?version=17
\newcommand{\KojoVersion}{2.9.28} %https://www.kogics.net/kojo-download
\newcommand{\VSCodeVersion}{1.90} %https://code.visualstudio.com/updates
\newcommand{\MetalsVersion}{v1.35} %https://marketplace.visualstudio.com/items?itemName=scalameta.metals
\newcommand{\WindowsVersion}{10}
\newcommand{\ScalaIDEVersion}{4.7.0} %%DEPRECATED
\newcommand{\OmkontrollDatum}{Torsd. 14/11 kl 13:00-18:00, E:1147}%Tors W09; används i lect-w08 och lect-w09
\newcommand{\LastLectureDate}{onsdagen den 6:e december i E:A kl 10-12}




\newcommand{\YEAR}{\CurrentYear}

\begin{document}

%% COMPATIBILITY PROBLEM In latex --version 2022 \bottomrule \addlinespace \midrule \toprule etc gives error ! Misplaced \noalign.
%% Here they are replaced by \hline and \\[1.2em]  etc
%% The commands from the booktab chapter are thus cancelled here; can the booktab package be removed?

\newcommand{\Kurskod}{EDAB05}
\section*{\Kurskod~ Programmering, grundkurs  -- Kursprogram \YEAR}
\emph{Institutionen för Datavetenskap, LTH, Lunds Universitet.} Senast uppdaterad: \today\\

\begin{longtable}[l]{ll}
\hline\\[-0.75em]%\toprule

\textbf{\Kurskod} & \textit{obl. för D1, C1}. 10,5 högskolepoäng, Läsperiod 1 \& 2 \\[-0.75em] \tabularnewline
\hline%\midrule
\endhead
\emph{Kursansvarig}   & Björn Regnell, rum E:2413,
                        \href{mailto:bjorn.regnell@cs.lth.se}
                        {\nolinkurl{bjorn.regnell@cs.lth.se}}\\
\emph{Bitr. kursansv.}   & Mattias Nordahl, rum E:2164,
                        \href{mailto:mattias.nordahl@cs.lth.se}
                        {\nolinkurl{mattias.nordahl@cs.lth.se}}\tabularnewline
                        \emph{Hemsida}        
                        & \url{https://lunduniversity.github.io/}\tabularnewline
\emph{Kurslitteratur} & Kompendium. Säljes till självkostnadspris via institutionen.\tabularnewline

%\bottomrule
\hline
\end{longtable}

\subsection{Undervisning}\label{undervisning}

\begin{itemize}
\item
  \emph{Föreläsningar}. Föreläsningarna ger en översikt av
  kursinnehållet och åskådliggör teorin med praktiska
  programmeringsexempel. Föreläsningarna ger även utrymme för diskussion
  och frågor.
\item
  \emph{Resurstider}. I kursens schema finns särskilda resurstider
  där du kan få hjälp med övningar, laborationer och
  inlämningsuppgifter. Utnyttja dessa tillfällen!
\item
  \emph{Övningar}. I kursen ingår övningar som du arbetar med
  självständigt eller tillsammans med en kamrat.
  Du kan få hjälp med övningarna av handledare under resurstiderna.
  Övningarna är förberedelser inför laborationer, projekt, muntligt prov och den skriftliga tentamen.
  %Se anvisningar i kompendiet.
\item
  \emph{Laborationer}. I kursen ingår obligatoriska laborationer.
  Laborationerna redovisas för handledare.
  %Se anvisningar i kompendiet.
\item
  \emph{Projektuppgift}. Du ska självständigt arbeta med ett större
  program som redovisas för handledare. %Se anvisningar i kompendiet.
\end{itemize}
Se vidare anvisningar om olika undervisningsmoment i kompendiet och på kurshemsidan.

\subsection{Samarbetsgrupper}\label{samarbetsgrupper}

Kursdeltagarna indelas i \emph{samarbetsgrupper} där studenter med olika förkunskapsnivåer
sammanförs. Målet är att deltagarna gemensamt ska
dela med sig av och träna på förklaringar av teori, begrepp och
programmeringspraktik. En av laborationerna görs i grupp. Ni ska hjälpa varandra att
förstå, men \emph{inte} lösa uppgifterna åt varandra.

\subsection{Examination}\label{examination}

\begin{itemize}
\item
  \emph{Obligatoriska kursmoment:}

  \begin{itemize}
  \item
    \emph{Projektuppgift och teori} en större projektuppgift och ett muntligt prov. Dessa redovisas för handledare på
    schemalagd tid. Alla laborationer ska vara godkända innan du får göra det muntliga provet.
  \item
    \emph{Laborationer i programmering} bedöms av handledare på schemalagd tid.
  \item
    \emph{Datorer och datoranvändning (dod)} laborationer med programmeringsverktyg. Dessa redovisas för handledare på
    schemalagd tid.
  \end{itemize}

\item \emph{Betyg:} 
  Godkända obligatoriska moment krävs för betyg 3. Skriftlig tentamen är valfri och kan ge betyg 4 el. 5. 
  För att få tentera krävs att alla obligatoriska moment är godkända. Enda tillåtna hjälpmedel på tentamen:
  \url{https://fileadmin.cs.lth.se/pgk/quickref.pdf} \\
  % Ordinarie tentamen: \TENTADATUM
\end{itemize}

\noindent För mer detaljer och se den \href{https://kurser.lth.se/kursplaner/senaste/\Kurskod.html}{formella kursplanen} och kursens hemsida.

\clearpage

% \subsection*{Veckoöversikt}

% \resizebox{\columnwidth}{!}{%
% {\fontsize{12pt}{20pt}\selectfont
% %!TEX encoding = UTF-8 Unicode
\begin{tabular}{l|l|l|l|l|l|l}
\textit{W} & \textit{Datum} & \textit{Lp V} & \textit{Modul} & \textit{Förel} & \textit{Övn} & \textit{Lab} \\ \hline \hline
W01 & 28/8-1/9 & Lp1V1 & Introduktion & F01 F02 & expressions & kojo \\
W02 & 4/9-8/9 & Lp1V2 & Program och kontrollstrukturer & F03 F04 & programs & -- \\
W03 & 11/9-15/9 & Lp1V3 & Funktioner och abstraktion & F05 F06 & functions & irritext \\
W04 & 18/9-22/9 & Lp1V4 & Objekt och inkapsling & F07 F08 & objects & blockmole \\
W05 & 25/9-29/9 & Lp1V5 & Klasser och datamodellering & F09 F10 & classes & blockbattle0 \\
W06 & 2/10-6/10 & Lp1V6 & Mönster och felhantering & F11 F12 & patterns & blockbattle1 \\
W07 & 9/10-13/10 & Lp1V7 & Sekvenser och enumerationer & F13 F14 & sequences & shuffle \\
KS & 25/10 & TP1 & KONTROLLSKRIVN. & -- & -- & -- \\
W08 & 30/10-3/11 & Lp2V1 & Nästlade och generiska strukturer & F15 F16 & matrices & life \\
W09 & 6/11-10/11 & Lp2V2 & Mängder och tabeller & F17 F18 & lookup & words \\
W10 & 13/11-17/11 & Lp2V3 & Arv och komposition & F19 F20 & inheritance & snake0 \\
W11 & 20/11-24/11 & Lp2V4 & Varians och kontextparametrar & F21 F22 & context & snake1 \\
W12 & 27/11-1/12 & Lp2V5 & Fördjupning, Projekt & F23 F24 & extra & Projekt0 \\
W13 & 4/12-8/12 & Lp2V6 & Repetition & F25 F26 & examprep & Projekt1 \\
W14 & 11/12-15/12 & Lp2V7 & MUNTLIGT PROV & -- & Munta & Munta \\
T & 4/1 & TP2 & VALFRI TENTAMEN & -- & -- & -- \\
\end{tabular}

% }
% }

\vspace{1.1em}\noindent\hspace*{-2.0mm}%
\noindent\textit{Preliminärt innehåll per vecka}\\~\\
\noindent\resizebox{\columnwidth}{!}
{%
{
\fontsize{8.0pt}{8.5pt}\selectfont
\begin{tabular}{l|l|p{7.4cm}}
W01 & Introduktion & sekvens, alternativ, repetition, abstraktion, editera, kompilera, exekvera, datorns delar, virtuell maskin, litteral, värde, uttryck, identifierare, variabel, typ, tilldelning, namn, val, var, def, definiera och anropa funktion, funktionshuvud, funktionskropp, procedur, inbyggda grundtyper, println, typen Unit, enhetsvärdet (), stränginterpolatorn s, aritmetik, slumptal, logiska uttryck, de Morgans lagar, if, true, false, while, for, dod: operativsystem \\
W02 & Program och kontrollstrukturer & huvudprogram, program-argument, indata, scala.io.StdIn.readLine, kontrollstruktur, iterera över element i samling, for-uttryck, yield, map, foreach, samling, sekvens, indexering, Array, Vector, intervall, Range, algoritm, implementation, pseudokod, algoritmexempel: SWAP, SUM, MIN-MAX, MIN-INDEX, dod: versionshantering \\
W03 & Funktioner och abstraktion & abstraktion, funktion, parameter, argument, returtyp, default-argument, namngivna argument, parameterlista, funktionshuvud, funktionskropp, applicera funktion på alla element i en samling, uppdelad parameterlista, skapa egen kontrollstruktur, funktionsvärde, funktionstyp, äkta funktion, stegad funktion, apply, anonyma funktioner, lambda, predikat, aktiveringspost, anropsstacken, objektheapen, stack trace, värdeandrop, namnanrop, klammerparentes och kolon vid ensam parameter, rekursion, scala.util.Random, slumptalsfrö, dod: typsättning \\
W04 & Objekt och inkapsling & modul, singelobjekt, punktnotation, tillstånd, medlem, attribut, metod, paket, filstruktur, jar, classpath, dokumentation, JDK, import, selektiv import, namnbyte vid import, export, tupel, multipla returvärden, block, lokal variabel, skuggning, lokal funktion, funktioner är objekt med apply-metod, namnrymd, synlighet, privat medlem, inkapsling, getter och setter, principen om enhetlig access, överlagring av metoder, introprog.PixelWindow, initialisering, lazy val, typalias, dod: maskinkod \\
W05 & Klasser och datamodellering & applikationsdomän, datamodell, objektorientering, klass, instans, Any, isInstanceOf, toString, new, null, this, accessregler, private, private[this], klassparameter, primär konstruktor, fabriksmetod, alternativ konstruktor, förändringsbar, oföränderlig, case-klass, kompanjonsobjekt, referenslikhet, innehållslikhet, eq, == \\
W06 & Mönster och felhantering & mönstermatchning, match, Option, throw, try, catch, Try, unapply, sealed, flatten, flatMap, partiella funktioner, collect, wildcard-mönster, variabelbindning i mönster, sekvens-wildcard, bokstavliga mönster, implementera equals, hashcode \\
W07 & Sekvenser och enumerationer & översikt av Scalas samlingsbibliotek och samlingsmetoder, klasshierarkin i scala.collection, Iterable, Seq, List, ListBuffer, ArrayBuffer, WrappedArray, sekvensalgoritm, algoritm: SEQ-COPY, in-place vs copy, algoritm: SEQ-REVERSE, registrering, algoritm: SEQ-REGISTER, linjärsökning, algoritm: LINEAR-SEARCH, tidskomplexitet, minneskomplexitet, översikt strängmetoder, StringBuilder, ordning, inbyggda sökmetoder, find, indexOf, indexWhere, inbyggda sorteringsmetoder, sorted, sortWith, sortBy, repeterade parametrar \\
TP & \multicolumn{2}{l}{--} \\
W08 & Nästlade och generiska strukturer & matris, nästlad samling, nästlad for-sats, typparameter, generisk funktion, generisk klass, fri och bunden typparameter, generiska datastrukturer, generiska samlingar i Scala \\
W09 & Mängder och tabeller & innehållstest, mängd, Set, mutable.Set, nyckel-värde-tabell, Map, mutable.Map, hash code, java.util.HashMap, java.util.HashSet, persistens, serialisering, textfiler, Source.fromFile, java.nio.file \\
W10 & Arv och komposition & arv, komposition, polymorfism, trait, extends, asInstanceOf, with, inmixning supertyp, subtyp, bastyp, override, Scalas typhierarki, Any, AnyRef, Object, AnyVal, Null, Nothing, topptyp, bottentyp, referenstyper, värdetyper, accessregler vid arv, protected, final, trait, abstrakt klass \\
W11 & Varians och kontextparametrar & övre- och undre typgräns, varians, kontravarians, kovarians, typjoker, kontextgräns, typkonstruktor, egentyp, typjoker, givet värde (given), kontextparameter (using), ad hoc polymorfism, typklass, api, kodläsbarhet, granskningar \\
W12 & Fördjupning, Projekt & välj valfritt fördjupningsområde, påbörja projekt \\
W13 & Repetition & träna på extentor, redovisa projekt, träna inför muntligt prov \\
W14 & \multicolumn{2}{l}{MUNTLIGT PROV} \\
TP & \multicolumn{2}{l}{VALFRI TENTAMEN} \\
\end{tabular}
}
}


\end{document}
